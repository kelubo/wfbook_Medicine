% 医宗金鉴 Main

\documentclass[a4paper,12pt,UTF8,twoside]{ctexbook}

% 设置纸张信息。
\RequirePackage[a4paper]{geometry}
\geometry{
	%textwidth=138mm,
	%textheight=215mm,
	%left=27mm,
	%right=27mm,
	%top=25.4mm, 
	%bottom=25.4mm,
	%headheight=2.17cm,
	%headsep=4mm,
	%footskip=12mm,
	%heightrounded,
	inner=1in,
	outer=1.25in
}

% 目录 chapter 级别加点(.)。
\usepackage{titletoc}
\titlecontents{chapter}[0pt]{\vspace{3mm}\bf\addvspace{2pt}\filright}{\contentspush{\thecontentslabel\hspace{0.8em}}}{}{\titlerule*[8pt]{.}\contentspage}

% 设置 part 和 chapter 标题格式。
\ctexset{
	part/name= {第,卷},
	part/number={\chinese{part}},
	chapter/name={第,篇},
	chapter/number={\chinese{chapter}}
}

% 设置古文原文格式。
\newenvironment{yuanwen}{\noindent\bfseries\zihao{4}}

\title{\heiti\zihao{0} 医宗金鉴}
\author{吴谦}
\date{清 - 公元 1742 年}

\begin{document}
	\maketitle
	
	\tableofcontents
	
	\frontmatter
	
	\chapter{前言}
	
	《医宗金鉴》,医学丛书。清乾隆年间由政府组织编写的大型医学丛书,刊于1742年。乾隆四年由太医吴谦负责编修的一部医学教科书。
	
	《医宗金鉴》这个名字也是由乾隆皇帝钦定的。《医宗金鉴》被《四库全书》收入,在《四库全书总目提要》中对《医宗金鉴》有很高的评价。自成书以来,这部御制钦定的太医院教科书就被一再的翻刻重印,广为流传。公元1739年,乾隆皇帝诏令太医院右院判吴谦主持编纂一套大型的医学丛书。吴谦,字六吉,安徽歙县人,他是清朝雍正、乾隆年间的名医。吴谦奉旨后,下令征集全国的各种新旧医书,并挑选了精通医学兼通文理的70多位官员共同编修。历时三年的时间,终于编辑完成。《医宗金鉴》全书共分卷,是我国综合性中医医书中比较完善而又简要的一种。全书采集了上自春秋战国,下至明清时期历代医书的精华。图、说、方、论俱备,并附有歌诀,便于记诵,尤其切合临床实用。流传极为广泛。
	
	全书采辑自《内经》至清代诸家医书,“分门聚类、删其驳杂,采其精粹,发其余蕴,补其未备”。内容有:《订正仲景全书伤寒论注》、《金匮要略注》、《四诊心法要诀》、《运气要诀》、《伤寒心法要诀》、《杂病心法要诀》、《妇科心法要诀》、《幼科杂病心法要诀》、《痘疹心法要诀》、《种痘心法要旨》、《外科心法要诀》、《眼科心法要诀》、《刺灸心法要诀》、《正骨心法要旨》。全书内容丰富完备,叙述较系统扼要。其中《伤寒》、《金匮》部分除对原文订正并予注释外,还征引了清以前伤寒各家的论述。各科心法要诀,以歌诀体概括疾病诸证的辨证施治,全面而系统,准确而精辟。切于实际,易学易用。本书刊行后深受读者的欢迎,流传颇广,成为学习中医的重要读物。
	
	1956年人民卫生出版社出版影印本,1963年出排印本,对全书作了校勘,并改编目录,附加索引。现存内府稿本、初刻本等十余种清刻本及多种石印本、铅印本等。1979年再予校点出版。
			
	清初,天花流行,危及宫廷,特别顺治皇帝死于天花,宫廷十分紧张,康熙亦曾感染天花,幸得隔离治疗保全了性命,也正因为康熙曾因天花获得免疫而得继承帝位。因此,他在位时十分重视痘疹一科与种痘术之推广。乾隆即位后,发扬康雍两朝重视医学之余风,接受太医院院使等鉴于古医书“词奥难明”“传写错讹”,自晋以下“医书甚夥”,“或博而不精,或杂 而不一,间有自相抵牾”的奏折,请求发内府医书,并征天下秘籍“分门别类,删其驳杂,采其精粹,发其余蕴,补其未备”。乾隆于四年(1739)诏令供奉内廷御医,太医院右院判吴谦, 与康雍乾三朝御医、院使刘裕铎,共同领衔编纂医书,由吴谦与刘裕铎任总纂修官,其下有纂 修官  14人,副纂修官12人,校阅官10人,收掌官(书稿保官)2人,誊录官23人,以及画家等组成编纂班子。
	
	所有参与编纂的御医等,都是按照清府批文“令太医院堂官并吴谦、刘裕铎等将平日真知灼见、精通医学兼通文理之人,保举选派”而组成的。若太医院合格人员不足者,“令翰林院……查派”,“选取字,画好者以备誊录。如不敷用,照例行文国子监……秉公考试,务择字画端楷,咨送本馆以凭选取”可见对所有编纂、绘画、誊录人员都是经过严格选择或经考 试后择优录用的。
	
	《医宗金鉴》经过三年时间完成,共90卷,15个分册。即伤寒17卷、金匮8卷,名医方论8卷,四诊1卷,运气1卷,伤寒心法3卷,杂病心法5卷,妇科心法6卷,幼科心法6卷,痘疹心法4卷,种痘心法1卷,外科心法16卷,眼科心法2卷,针灸心法8卷,正骨心法4卷。
	
	该书特点:图、说、方、论俱备,歌诀助诵。细读之有着十分明显的时代性,适应18世纪中国疾病谱。例如:公元17-18世纪,康熙、乾隆由于天花危害甚大,对太医院压力更大,因此,太医院在分科设置上也有明显的反映,如将痘疹作为一科从幼科中分出来,在《医宗金鉴》中也单独成册,特别还将《种痘心法》作为一卷与幼科心法并列,可见对天花一病的专门研究与防治得到了高度重视,促成幼科被分解为三科。由于接种人痘的推广与普及,天花之危害明显降低,太医院又将痘疹一科合并回幼科。另外,还有对正骨一科整理提高也十分明显 ,通过《医宗金鉴》的编纂,使中国历代相传的正骨理论与技术更加系统,更将宫廷上驷院绰班 (正骨) 处的丰富经验融为一体,使太医院正骨科与上驷院绰班处合并一处,理论与技术均得到提高。
	
	编纂完成后,乾隆看后十分满意,赐书名为《医宗金鉴》,正式确定该书名为《御纂医宗金鉴》,于1742年,以武英殿聚珍本与尊经阁刻本印行,在全国推广,影响巨大。1749 年即被定为太医院医学教育的教科书,“使为师者必由是而教,为弟子者必由是而学”。
	
	社会影响
	
	《医宗金鉴》逐步成为全国医学教学的必读书、准绳。由于广泛之需求,政府与商家刻本印刷十分频繁,至今其版本流传已有50余家,平均4-5年即有一次版本问世。这既是由于该书内容的丰富简约,深受读者喜爱。同时,乾隆皇帝给予的高度评价与肯定,也助了有意义的一臂之力。
	
	\chapter{序一}
	\chapter{序一}
	
	\mainmatter
	\part{}
	\chapter{上古天真论}
	
	\begin{yuanwen}
		昔在黄帝,生而神灵,弱而能言,幼而徇齐,长而敦敏,成而登天。
	\end{yuanwen}

\end{document}