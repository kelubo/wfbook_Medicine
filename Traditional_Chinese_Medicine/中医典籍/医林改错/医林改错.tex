%-*- coding: UTF-8 -*-
% 医林改错

\documentclass[a4paper,12pt,UTF8,twoside]{ctexbook}

% 目录 chapter 级别加点(.)。
\usepackage{titletoc}
\titlecontents{chapter}[0pt]{\vspace{3mm}\bf\addvspace{2pt}\filright}{\contentspush{\thecontentslabel\hspace{0.8em}}}{}{\titlerule*[8pt]{.}\contentspage}

% 设置 part 和 chapter 标题格式。
\ctexset{
	part/name= {第,卷},
	part/number={\chinese{part}},
	chapter/name={第,篇},
	chapter/number={\chinese{chapter}}
}

\title{\heiti\zihao{0} 医林改错}
\author{王清任}
\date{}

\begin{document}
	
	\maketitle
	\tableofcontents
	\frontmatter
	
	\chapter{前言}
	公元 1830 年)清.王清任(勋臣)着。二卷。作者从访验尸体后所见,提出对于脏腑解剖的己见。并载自定义方剂、及附方论。
	
	作者王清任(1768-1831),字勋臣,河北省玉田县人,世居玉田县鸭鸿桥。曾做过武库生,后至北京行医,是嘉庆至道光年间的名医。
	
	王清任的主要著作为《医林改错》,这是一部几百年来令医学界争论不休的书。书中主要阐述了两个方面的观点。其一便是“改错”,王清任认为,我国古代医书中对人体脏腑的位置、大小和重量的描述并不确切,他曾在瘟疫流行的灾区观察未掩埋的儿童尸体300多例,逐一进行了解剖和观察,绘制了大量的脏腑图。他认为前世许多医书的讲法不正确,须改正,故书名便为《医林改错》;另一主要内容主要表明了他对人体气血的一个特殊的认识。他认为气与血皆为人体生命的源泉,但同时也是致病因素。不论外感内伤,对于人体的损伤,皆伤于气血而非脏腑。气有虚实:实为邪实,虚为正虚;血有亏瘀,亏为失血,瘀为阻滞。他认为瘀血是由于正气虚,推动无力造成的,故血瘀证皆属虚中夹实。故而他倡导“补气活血”和“逐瘀活血”两大法则,这就是他的著名的“瘀血说”。
	
	\chapter{张序}
	医,仁术也。乃或术而不仁,则贪医足以误世;或仁而无术,则庸医足以杀人。古云不服药为中医,盖诚虑乎医之仁术难兼也,至于稍读方书,即行市道,全无仁术,奚以医为?余来粤数年,目击此辈甚众,辄有慨乎其中。每遇救急良方,不惜捐赀购送。今放癸丑四月,适闻佛山友人有幼子患症,医以风药投之,竟至四肢抽搐,口眼歪斜,命垂旦夕,忽得一良方,一剂稍愈,三服霍然。又有人患半身不遂者十余年,得一良方,行走如故。余甚奇之,再四访求,始知二方皆出自《医林改错》一书。遍求得之,历试多验。因于公余沉潜反复,颇悟其旨。窃叹此书之作,直翻千百年旧案,正其谬误,决其瑕疵,为希世之宝也,岂非术之精而仁之至哉!余不忍秘藏,立刊布以公于世。使今人得悉脏腑经络之实,而免受庸医之误。亦不负王勋臣先生数十年济世之苦心矣。愿同志君子勿视为寻常善书,幸甚!幸甚!
	
	咸丰癸丑仲夏顺天张润坡识
	
	\chapter{刘序}
	
	
	丁未之秋,寄迹吴门。适同乡焦子浚文来,手执脏腑全图,乃勋臣王先生《医林改错》之稿也。脏腑图汉魏以来,医家所习见,何异乎尔?异乎勋臣先生所绘之图与古人殊也。脏腑人人皆同,勋臣背古以传图,得毋炫奇立异乎?曰:否,不然也。古人之图传其误,勋臣之图传其信。天下物理之是非,闻虚而见实,寡见独虚,多见为实。古人窃诸刑余之一犯,勋臣得诸亲见之百人。集数十载之精神,考正乎数千年之遗误。譬诸清夜钟鸣,当头棒喝,梦梦者皆为之唤醒焉。医书汗牛充栋,岂尽可征。然非善读书者,独具只眼,终为古人所牢笼,而潜受其欺。孟子曰:吾于武城取二三策。武城周书也,孟子周人也,当代之书,独且不可尽信,况远者乎!是书绘图文说,定方救逆,理精识卓,绝后空前。可为黄帝之功臣,即可为长沙之畏友。抑又闻之,叶氏《指南》有久病入络之说。徐氏非之,不知入络即血瘀也。今勋臣痛快言之,而《指南》入络之说益明。坊友汪子维之见而悦之,开雕梨枣,以公诸世,斯真能刊录善书者也。是为序。
	
	道光戊申中秋日上元后学小窗氏刘必荣识
	
	
	\chapter{知非子序}
	
	
	余读勋臣先生《医林改错》一书,而叹天下事,有人力为之者,有天意成之者,先生是书,功莫大于图绘脏腑诸形。其所以能绘诸形者,则由于亲见,其所以得亲见者,则由于稻地镇之一游也。此岂非天假之缘,而使数千载之误,由先生而正之哉!惟隔膜一事,留心三十年,未能查验的确。又得恒敬公确示一切,而后脏腑诸形,得以昭晰无疑,此非有无意玉成其间哉!至先生立方医疾,大抵皆以约治博。上卷着五十种血瘀之症,以三方治之。下卷论半身不遂,以一方治之。并审出未病以前四十种气虚之形症,非细心何能至此。论吐泻转筋,治分攻补两途,方由试验中来;论小儿抽风非中风,以大补元气一方治之;以不能言之儿,查出二十种气虚之形症,平素细心,不同可知;论痘非胎毒、痘浆非血化,以六方治古人不治之六十种逆痘,颇有效者。先生之书,大抵补前人之未及,而在气虚血瘀之症为多,今特揭诸篇首。
	
	知非子书
	
	\chapter{自序}
	
	
	余着《医林改错》一书,非治病全书,乃记脏腑之书也。其中当尚有不实不尽之处,后人倘遇机会,亲见脏腑,精察增补,抑又幸矣!记脏腑后,兼记数症,不过示人以规矩,令人知外感内伤,伤人何物;有余不足,是何形状。至篇中文义多粗浅者,因业医者学问有浅深也;前后语句多覆重者,恐心粗者前后不互证也。如半身不遂内有四十种气亏之症,小儿抽风门有二十种气亏之症,如遇杂症,必于六十种内互考参观,庶免谬误。望阅是书者,须详审焉。
	
	玉田王清任书。
	
	
	\mainmatter
	\part{上卷}
	\chapter{医林改错脏腑记叙}
	
	
	古人曰:既不能为良相,愿为良医。以良医易而良相难。余曰:不然。治国良相,世代皆有;着书良医,无一全人。其所以无全人者,因前人创着医书,脏腑错误;后人遵行立论,病本先失,病本既失,纵有绣虎雕龙之笔,裁云补月之能,病情与脏腑,绝不相符,此医道无全人之由来也。夫业医诊病,当先明脏腑。尝阅古人脏腑论,及所绘之图,立言处处自相矛盾。如古人论脾胃,脾属土,土主静而不宜动;脾动则不安。既云脾动不安,何得下文又言脾闻声则动,动则磨胃化食,脾不动则食不化?论脾之动静。其错误如是,其论肺,虚如蜂窠,下无透窍,吸之则满,.呼之则虚。既云下无透窍,何得又云肺中有二十四孔,行列分布,以行诸脏之气?论肺之孔窍,其错误又如是。其论肾,有两枚,即腰子。两肾为肾,中间动气为命门。既云中间动气为命门.何得又云左肾为肾,右肾为命门?两肾一体,如何两立其名,有何凭据?若以中间动气为命门,藏动气者,又何物也?其论肾错误又如是。其论肝,左右有两经,即血管,从两胁肋起,上贯头目,下由少腹环绕阴器,至足大指而止,既云肝左右有两经,何得又云肝居于左,左胁属肝?论肝分左右,其错误又如是。其论心,为君主之官,神明出焉。意藏于心,意是心之机,怠之所专曰志,志之动变曰思,以思谋远曰虑,用虑处物曰智,五者皆藏于心。既藏于心,何得又云脾藏意智,肾主伎巧,肝主谋虑,胆主决断?据所论处处皆有灵机,究竟未说明生灵机者何物?藏灵机者何所?若用灵机,外有何神情?其论心如此含混。其论胃,主腐熟水谷。又云脾动磨胃化食,胃之上口名曰责门.饮食人胃,精气从贲门上输于脾肺,宣播于诸脉。此段议论,无情无理。胃下口名曰幽门,即小肠上口,其论小肠,为受盛之官,化物出焉。言饮食入小肠,化粪下至阑门,即小肠下口,分别清浊,粪归大肠,自肛门出,水归膀胱为尿。如此论尿从粪中渗出,其气当臭。尝用童子小便,并问及自饮小便之人,只言味咸,其气不臭。再者食与水合化为粪,粪必稀溏作泻,在鸡鸭无小便则可,在牛马有小便则不可,何况乎人?看小肠化食,水自阑门出一节,真是千古笑谈。其论心包络,细筋如丝,与心肺相连者,心包络也。又云心外黄脂是心包络。又云心下横膜之上,竖膜之下,黄脂是心包络。又云膻中有名无形者,乃心包络也。既云有名无形,何得又云手中指之经,乃是手厥阴心包络之经也?论心包络竟有如许之多,究竟心包络是何物?何能有如许之多那!其论三焦,更为可笑。《灵枢》曰:手少阴三焦主乎上,足太阳三焦主乎下,已是两三焦也。《难经》三十一难论三焦,上焦在胃之上,主内而下出:中焦在胃中脘,主腐熟水谷;下焦在脐下,主分别清浊,又云三焦者,水谷之道路。此论三焦是有形之物。又云两肾中间动气,是三焦之本。此论三焦是无形之气。在《难经》一有形,一无形,又是两三焦。王叔和所谓有名无状之三焦者,盖由此也。至陈无择以脐下脂膜为三焦,袁淳甫以人身著内一层、形色最赤者为三焦,虞天民指空腔子为三焦,金一龙有前三焦、后三焦之论。论三焦者,不可以指屈。有形无形,诸公尚无定准,何得云手无名指之经,是手少阳三焦之经也?其中有自相矛盾者,有后人议驳而未当者。总之,本源一锗,万虑皆失。
	
	余尝有更正之心,而无赃腑可见。自恨著书不明赃腑,岂不是痴人说梦;治病不明赃腑,何异于盲子夜行!虽竭思区画,无如之何。十年之久,念不少忘。至嘉庆二年丁已,余年三十,四月初旬,游于滦州之稻地镇。其时彼处小儿,正染瘟疹痢症,十死八九。无力之家,多半用代席裹埋,代席者,代棺之席也。彼处乡风,更不深埋,意在犬食,利于下胎不死。故各义冢申,破腹露脏之儿.日有百余。余每日压马过其地,初未尝不掩鼻,后因念及古人所以错论脏腑,皆由未尝亲见,遂不避污秽,每日清晨赴其义冢,就群儿之露脏者细视之。犬食之余,大约有肠胃者多,有心肝者少。互相参看,十人之内,看全不过三人。连视十日,大约看全不下三十余人,始知医书中所绘脏腑形图,与人之脏腑全不相合,即件数多寡,亦不相符。惟胸中隔膜一片,其薄如纸,最关紧要。及余看时,皆以破坏,未能验明在心下心上,是斜是正,最为遗憾,至嘉庆四年六月,余在奉天府,有辽阳州一妇年二十六岁,因疯疾打死其夫与翁,解省拟剐,跟至西关,忽然醒悟,以彼非男子,不忍近前,片刻行刑者提其心与肝肺,从面前过,细看与前次所看相同。后余在京时,嘉庆庚辰年,有打死其母之剐犯,行刑放崇文门外吊桥之南,却得近前,及至其处,虽见脏腑,膈膜已破,仍未得见,道光八年五月十四日,剐逆犯张格尔,及至其处,不能近前,自思一箦未成,不能终止。不意道光九年十二月十三日夜间,有安定门大街板厂胡同恒宅,请余看症,因谈及膈膜一事,留心四十年,未能审验明确。内有江宁布政司恒敬公言,伊曾镇守哈密,领兵于喀什噶尔,所见诛戮逆尸最多,于膈膜一事,知之最悉,余闻言喜出望外,即拜叩而问之。恒公鉴余苦衷,细细说明形状。余于脏腑一事,访验四十二年,方得的确,绘成全图。意欲刊行于世,惟恐后人未见脏腑,议余故叛经文;欲不刊行,复虑后世业医受祸,相沿又不知几千百年。细思黄帝虑生民疾苦,平素以灵枢之言下间歧伯、鬼臾区,故名《素问》。二公如知之的确,可对君言,知之不确,须待参考,何得不知妄对,遗祸后世?继而秦越人著《难经》,张世贤割裂《河图洛书》为之图注,谓心肝肺以分两计之,每件重几许;大小肠以尺丈计之,每件长若干;胃大几许,容谷几斗几升。其言彷佛似真,其实脏腑未见,以无凭之谈,作欺人之事,利己不过虚名,损人却属实祸。窃财犹谓之盗,偷名岂不为贼!千百年后岂无知者!今余刻此图,并非独出己见,评论古人之短长;非欲后人知我,亦不避后人罪我。惟愿医林中人一见此图,胸中雪亮,眼底光明,临症有所遵循,不致南辕北辙,出言含混,病或少失,是吾之厚望。幸仁人君子鉴而谅之。
	
	时道光庚寅孟冬直隶玉田县王清任书干京邸知一堂。
	
	\chapter{亲见改正脏腑图}
	左气门、右气门两管归中一管入心,由心左转出横行后接卫总管。心长在气管之下,非在肺管之下,心与肺叶上棱齐。
	
	肺管至肺分两权,入肺两叶,直贯到肺底皆有节。管内所存皆轻浮白沫,如豆腐沫有形无体。两大叶大面向背,小面向胸,上有四尖向胸,下一小片亦向胸。肺外皮实无透窍,亦无行气之二十四孔。
	
	膈膜以上仅止肺、心、左右气门,余无他物。其余皆膈膜以下物,人身膈膜是上下界物。
	
	肝四叶,胆附于肝右边第二叶,总提长于胃上,肝又长于总提之上,大面向上,后连于脊,肝体坚实,非肠、胃、膀胱可比,绝不能藏血。
	
	胃府之体质,上口贲门在胃上正中,下口幽门亦在胃上偏右,幽门之左寸许名津门,胃内津门之左有疙瘩如枣名遮食,胃外津门左名总提,肝连于其上。胃在腹是平铺卧长,上口向脊,下口向右,底向腹,连出水道。
	
	脾中有一管,体像玲珑,易于出水,故名珑管。脾之长短与胃相等,脾中间一管,即是珑管,另画珑管者,谓有出水道,今人易辨也。
	
	气府俗名鸡冠油,下棱抱小肠,气府内、小肠外乃存元气之所,元气化食,人身生命之源全在于此。此系小肠,外有气府包裹之。
	
	中是珑管,水由珑管分流两边出水道,由出水道渗出,泌入膀恍为尿。出水道中有四血管,其余皆系水管。
	
	大肠上口即小肠下口,名曰阑门,大肠下口即肛门。
	
	膀胱有下口,无上口,下口归玉茎。精道下孔亦归玉茎,精道在妇女名子宫。
	
	两肾凹处有气管两根,通卫总管,两傍肾体坚实,内无孔窍,绝不能藏精。
	
	舌后白片,名曰会厌,乃遮盖左右气门。喉门之物。
	
	古人言经络是血管,由每脏腑向外长两根。惟膀胖长四根。余亲见厅余脏腑,并无向外长血管之形,故书于图后以记之。
	
	\chapter{会厌、左气门、右气门、卫总管、荣总管、气府、血府记}
	欲知脏腑体质,先明出气、入气、与进饮食之道路。古人谓舌根后名曰喉,喉者候也,候气之出入,即肺管上口是也。喉之后名曰咽,咽音咽也,咽饮食人胃,即胃管上口是也。谓咽以纳食,喉以纳气,为千古不易之定论,自灵素至今四千年来,无人知其错而改正音,如咽咽饮食入胃,人所共知。惟喉候气之出入一节、殊欠明白。不知肺两叶大面向背,上有四尖向胸,下有一小片亦向胸,肺管下分为两权,入肺两叶,每权分九中权,每中权分九小权,每小权长数小枝,枝之尽头处,并无孔窍。其形彷佛麒麟菜,肺外皮亦无孔窍,其内所存,皆轻浮白沫,肺下实无透窍,亦无行气之二十四孔。先贤论吸气则肺满,呼气则肺虚。此等错误,不必细辩,人气向里吸,则肚腹满大,非肺满大;气向外呼,则肚腹虚小,非肺虚小。出气、入气、吐痰、吐饮、唾津、流涎,与肺毫无干涉。肺管之后,胃管之前,左右两边凹处,有气管两根,其粗如箸,上口在会厌之下,左曰左气门,右曰右气门,痰饮津涎,由此气管而出。古人误以咳嗽、喘急、哮吼等症,为肺病者,因见其症自胸中来。再者,临症查有外感,用发散而愈;有燥痰,用清凉而愈;有积热,用攻下而愈;有气虚,用补中而愈;有阴亏,用滋阴而愈;有瘀血,用逐瘀而愈。扬扬得意,立言著书,以为肺病无疑。不知左气门、右气门两管,由肺管两傍,下行至肺管前面半截处,归并一根,如树两权归一本,形粗如箸,下行人心,由心左转出,粗如笔管,从心左后行,由肺管左边过肺入脊前,下行至尾骨,名曰卫总管,俗名腰管。自腰以下,向腹长两管。粗如箸,上一管通气府,俗名鸡冠油,如倒握鸡冠花之状。气府乃抱小肠之物,小肠在气府是横长,小肠外、气府内,乃存元气之所。元气即火,火即元气,此火乃人生命之源。食由胃入小肠,全仗元气蒸化,元气足则食易化,元气虚则食难化。此记向腹之上一管。下一管,大约是通男子之精道、女子之子宫。独此一管,细心查看,未能查验的确,所以疑似。以俟后之业医者,倘遇机会,细心查看再补。卫总管,对背心两边有两管,粗如箸,向两肩长,对腰有两管,通连两肾,腰下有两管,通两胯。腰上对脊正中,有十一支管连脊。此管皆行气,行津液。气足火旺,将津液煎稠,稠者名曰痰;气虚火衰,不能煎熬津液,津液必稀,稀者名曰饮。痰饮在管,总以管中之气上攻,上行过心。由肺管前气管中,出左右气门。痰饮津涎,本气管中物,古人何以误为肺中物?因不知肺管前有气管相连而长,止知痰饮津涎自胸中来,便疑为肺中物,总是未亲见脏腑之故。手握足步。头转身摇,用行舍藏,全凭此气。人气向里吸。则气府满,气府满,则肚腹大;气向外呼,则气府虚,气府虚,则肚腹小。卫总管,行气之府,其中无血。若血归气府,血必随气而出,上行则吐血、岖血,下行则溺血,便血。卫总管之前,相连而长,粗如箸,名曰荣总管,即血管,盛血,与卫总管长短相等,其内之血由血府灌溉。血府即人胸下膈膜一片,其薄如纸,最为坚实,前长与心口凹处齐,从两胁至腰上,顺长加坡,前高后低,低处如池,池中存血,即精汁所化,名曰血府,精汁详胃津门条下。前所言会厌,即舌后之白片,乃遮盖左右气门、喉门之物也。
	
	\chapter{津门、津管、遮食、总提、珑管、出水道记}
	咽下胃之一物,在禽名曰嗉,在兽名曰肚。在人名曰胃。古人画胃图,上口在胃上,名曰贯门;下口在胃下,名曰幽门,言胃上下两门,不知胃是三门。画胃竖长,不知胃是横长,不但横长,在腹是平铺卧长,上口贲门向脊,下底向腹;下口幽门亦在胃上,偏右胁向脊;幽门之左寸许,另有一门,名曰津门。津门上有一管,名曰津管,是由胃出精汁水液之道路。津管一物,最难查看,因上有总提遮盖。总提俗名胰子,其体长于贲门之右、幽门之左,正盖津门。总提下前连气府,接小肠,后接提大肠,在胃上后连肝,肝连脊。此是膈膜以下,总提连贯胃、肝、大小肠之体质。饮食入胃,食留于胃,精汁水液,先由津门流出,入津管,津管寸许,外分三权。精汁清者,入髓府化髓;精汁浊者,由上权卧则入血府,随血化血;其水液,由下权从肝之中间,穿过入脾。脾中间有一管,体相玲珑,名曰珑管,水液由珑管分流两边,入出水道。出水道形如鱼网,俗名网油。水液由出水道渗出,沁入膀胱,化而为尿。出水道出水一段,体查最难。自嘉庆二年看脏腑时,出水道有满水铃铛者,有无水铃铛者,于理不甚透彻,以后诊病,查看久病寿终之人,临时有多饮水者,有少饮水者,有不饮水者,故后其水仍然在腹。以此与前所看者参考,与出水道出水一节,虽然近理,仍不敢为定准。后以畜较之,遂喂遂杀之畜,网油满水铃铛;三四日不喂之畜,杀之无水铃铛,则知出水道出水无疑。前言饮食入胃,食留干胃,精汁水液,自津门流出。津门即孔如箸大,能向外流精汁水液,稀粥岂不能流出?津门虽孔如箸大,其处胃体甚厚,四围靠挤缩小,所以水能出而食不能出。况胃之内,津门之左一分远,有一疙瘩,形如枣大,名曰遮食,乃挡食放水之物,待糟汁水液流尽,食方腐熟,渐入小肠,化而为粪。小肠何以化食为粪?小肠外有气府,气府抱小肠,小肠外、气府内,乃存元气之所,元气化食。此处与前气府参看。化粪入大肠,自肛门出。此篇记精汁由胃出津门,生精生血;水液由珑管出水道,入膀胱为尿;食由胃入小肠。元气蒸化为粪之原委也。
	
	\chapter{脑髓说}
	
	
	灵机记性,不在心在脑一段,本不当说,纵然能说,必不能行,欲不说,有许多病,人不知源始,至此又不得不说。不但医书论病,言灵机发放心,即儒家谈道德言性理,亦未有不言灵机在心者,因始创之人,不知心在胸中,所辨何事?不知咽喉两傍,有气管两根,行至肺管前,归并一恨,入心,由心左转出,过肺入脊,名曰卫总管,前通气府、精道,后通脊,上通两肩,中通两肾,下通两腿,此管乃存元气与津液之所,气之出入,由心所过,心乃出入气之道路,何能生灵机、贮记性?灵机记性在脑者,因饮食生气血,长肌肉,精汁之清者,化而为髓,由脊骨上行入脑,名曰脑髓。盛脑们者,名曰髓海,其上之骨,名曰天灵盖。两耳通脑,所听之声归于脑,脑气虚,脑缩小,脑气与耳窍之气不接,故耳虚聋;耳窍通脑之道路中,若有阻滞,故耳实聋,两目即脑汁所生,两目系如线,长于脑,所见之物归于脑,瞳人白色,是脑汁下注,名曰脑汁入目。鼻通干脑,所闻香臭归于脑,脑受风热,脑汁从鼻流出,涕浊气臭,名曰脑漏。看小儿初生时,脑未全,囱门软,目不灵动,耳不知听,鼻不知闻,舌不言,至周岁,脑渐生,囱门渐长,耳稍知听,目稍有灵动,鼻微知香臭,舌能言一二字。至三四岁,脑髓渐满,囱门长全,耳能听,目有灵动,鼻知香臭,言语成句。所以小儿无记性者,脑髓未满;高年无记性者,脑髓渐空。李时珍曰:脑为元神之府。金正希曰:人之记性皆在脑中。汪讱庵曰:今人每记忆往事,必闭目上瞪而思索之。脑髓中一时无气,不但无灵机,必死一时,一刻无气,必死一刻。
	
	试看痫症,俗名羊羔风,即是元气一时不能上转入脑髓。抽时正是活人死脑袋,活人者,腹中有气,四肢抽搐;死脑袋者,脑髓无气,耳聋、眼天吊如死。有先喊一声而后抽者,因脑先无气,胸中气不知出入,暴向外出也。正抽时,胸中有漉漉之声者,因津液在气管,脑无灵机之气,使津液吐咽,津液逗留在气管,故有此声。抽后头疼昏睡者,气虽转入于脑,尚未足也。小儿久病后元气虚抽风,大人暴得气厥,皆是脑中无气,故病人毫无知识。以此参考,岂不是灵机在脑之证据乎!
	
	\chapter{气血合脉说}
	
	
	脉之形,余以实憎告后人。若违心装神仙,丧灭良评论,必遭天诛。
	
	气府存气,血府存血。卫总管由气府行周身之气,故名卫总管;荣总管由血府行周身之血,故名荣总管。卫总管体厚形粗,长在脊骨之前,与脊骨相连,散布头面四肢,近筋骨长。即周身气管;荣总管体薄形细,长在卫总管之前,与卫总管相连,散布头面口肢,近皮肉长,即周身血管。气在气府,有出有入,出入者,呼吸也。目视耳听,头转身摇,掌握足步,灵机使气之动转也;血自血府入荣总管,由荣总管灌入周身血管,渗于管外,长肌肉也。气管近筋骨生,内藏难见;血管近皮肉长,外露易见。气管行气,气行则动;血管盛血,静而不动。头面四肢按之跳动者,皆是气管,并非血管。如两眉棱骨后凹处,俗名两太阳,是处肉少皮连骨,按之跳动,是通头面之气管;两足大指次指之端,是处肉少皮连骨,按之跳动,是通两足之气管;两手腕横纹高骨之上,是处肉少皮连骨,按之跳动,是通两手之气管。其管有粗有细,有直有曲,各人体质不同。胳膊肘下,近手腕肉厚,气管外露者短;胳膊肘下,近手腕肉薄,气管外露者长。如外感中人,风入气管。其管必粗,按之出肤:寒入气管,管中津液必凝,凝则阻塞其气,按之跳动必慢;火入气管,火气上炙,按之跳动必急。人壮邪气胜,管中气多,按之必实大有力,人弱正气衰,管中气少,按之必虚小无力。久病无生机之人,元气少,仅止上行头面两手,无气下行,故足面按之不动。若两手腕气管上,按之似有似无,或细小如丝,或指下微微乱动,或按之不动,忽然一跳,皆是气将绝之时。此段言人之气管,生平有粗细、曲直之下同,管有短长者,因手腕之肉有薄厚也;按之大小者,虚实也;跳动之急慢者,寒火之分也。前所言,明明是脉,不言脉者,因前人不知人有左气门、右气门、血府、气府、卫总管、荣总管、津门、津管、总提、遮食、珑管、出水道,在腹是何体质?有何用处,论脏腑、包络,未定准是何物,论经络、三焦,未定准是何物,并不能指明经络是气管、血管;论脉理,首句便言脉为血府,百骸贯通,言脉是血管,气血在内流通,周而复始:若以流通而论,此处血真能向彼处流,彼处当有空隙之地,有空隙之地,则是血虚,无空隙之地咄流归于何处?古人并不知脉是气管,竟著出许多脉快,立言虽多,论部位一人一样,并无相同者。
	
	古人论脉二十七字,余不肯深说者,非谓古人无容足之地,恐后人对症无论脉之言。诊脉断死生易,知病难。治病之要诀,在明白气血,无论外感内伤,要知初病伤人何物,不能伤脏腑,不能伤筋骨,不能伤皮肉,所伤者无非气血。气有虚实,实者邪气实,虚者正气虚。正气虚,当与半身不遂门四十种气虚之症、小儿抽风门二十种气虚之症,互相参考。血有亏瘀,血亏必有亏血之因,或因吐血、衄血,或溺血、便血,或破伤流血过多,或崩漏、产后伤血过多;若血瘀,有血瘀之症可查,后有五十种血瘀症,互相参考。惟血府之血,瘀而不活,最难分别。后半日发烧,前半夜更甚,后半夜轻,前半日不烧,此是血府血瘀。血瘀之轻者,不分四段,惟日落前后烧两时,再轻者,或烧一时,此内烧兼身热而言。若午后身凉,发烧片刻,乃气虚参耆之症,若天明身不热,发烧止一阵,乃参附之症,不可混含从事。
	
	\chapter{心无血说}
	余友薛文煌,字朗斋,通州人,素知医,道光十年二月,因赴山东,来舍辞行,闲谈言及古人论生血之源,有言心生血、脾统血者,有言脾生血,心统血者,不知宗谁?余曰:皆不可宗。血是精汁入血府所化,心乃是出入气之道路,其中无血。朗斋曰:吾兄所言不实。诸物心皆有血,何独人心无血,余曰:弟指何物心有血?曰:古方有遂心丹治癫狂,用甘遂末。以猪心血和为丸,岂不是猪心有血之凭据?余曰,此古人之错,非心内之血,因刀刺破其心,腔子内血流入于心,看不刺破之心,内并无血。余见多多。试看杀羊者,割其颈项不刺心,心内亦无血。又曰:不刺心,何死之速?余曰:满腔血从刀口流,所以先流者速,继而周身血退还腔子,所以后流者迟,血尽气散,故死之速。如入斗殴破伤,流血过多,气散血亡,渐至抽风,古人立名曰破伤风,用散风药治死受伤者,凶手拟抵,治一个,即是死两个。若明白气散气亡之义,即用黄耆半斤、党参四两,大补其气,救一人岂不是救两人?朗斋点首而别。
	
	\chapter{方叙}
	余不论三焦者,无其事也。在外分头面四肢、周身血管,在内分膈膜上下两段,膈膜以上,心肺咽喉、左右气门,其余之物,皆在膈膜以下。立通窍活血汤,治头面四肢、周身血管血瘀之症;立血府逐瘀汤,治胸中血府血瘀之症;立膈下逐瘀汤,治肚腹血瘀之症。病有千状万态,不可以余为全书。查证有王肯堂《证治准绳》,查方有周定王朱绣《普济方》,查药有李时珍《本草纲目》。三书可谓医学之渊源。可读可记,有国朝之《医宗金鉴》;理足方效,有吴又可《瘟疫论》,其余名家,虽未见脏腑,而攻发补泻之方,效者不少。余何敢云着书,不过因着《医林改锗》脏腑图记后,将平素所治气虚、血瘀之症,记数条示人以规矩,并非全书。不善读者,以余之书为全书,非余误人,是误余也。
	\chapter{通窍活血汤所治症目}
	通窍活血汤所治之病,开列于后:
	\section{头发脱落}
	
	
	伤寒、瘟病后头发脱落,各医书皆言伤血,不知皮里肉外血瘀,阻塞血路,新血不能养发,故发脱落。无病脱发,亦是血瘀。用药三付,发不脱,十付必长新发。
	
	眼疼白珠红
	
	眼疼白珠红,俗名暴发火眼。血为火烧,凝于目珠,故白珠红色。无论有云翳、无云翳,先将此药吃一付,后吃加味止痛没药散,一日二付,三两日必全愈。
	
	\section{糟鼻子}
	色红是瘀血,无论三、二十年,此方服三付可见效,二、三十付可全愈。舍此之外,并无验方。
	
	\section{耳聋年久}
	耳孔内小管通脑,管外有瘀血,靠挤管闭,故耳聋。晚报此方,早服通气散,一日两付,三、二十年耳聋可愈。
	
	\section{白癜风}
	血瘀于皮里,服三、五付可不散漫,再服三十付可痊。
	\section{紫癜风}
	血瘀于肤里,治法照白癜风,无不应手取效。
	\section{紫印脸}
	脸如打伤血印,色紫成片,或满脸皆紫,皆血瘀所致。如三、五年,十付可愈;若十余年,三、二十付必愈。
	\section{青记脸如墨}
	血瘀症,长于天庭者多,三十付可愈。白癜、紫癜、紫印、青记,自古无良方者,不知病源也。
	\section{牙疳}
	牙者骨之余,养牙者血也。伤寒、瘟疫、痘疹、痞块,皆能烧血,血瘀牙床紫,血死牙床黑,血死牙脱,人岂能活,再用凉药凝血,是促其死也。遇此症,将此药晚服一付,早服血府逐瘀汤一付,白日煎黄耆八钱,徐徐服之,一日服完。一日三付,三日可见效,十日大见效,一月可全愈。纵然牙脱五、七个,不穿腮者,皆可活。
	
	\section{出气臭}
	血府血瘀,血管血必瘀,气管与血管相连,出气安得不臭?即风从花里过来香之义。晚服此方,早服血府逐瘀汤,三、五日必效,无论何病,闻出臭气,照此法治。
	\section{妇人干劳}
	经血三、四月不见,或五、六月不见,咳嗽急喘,饮食减少,四肢无力,午后发烧,至晚尤甚。将此方吃三付,或六忖,至重者九付,未有不全愈者。
	\section{男子劳病}
	初病四肢酸软无力,渐渐肌肉消瘦,饮食减少,面色黄白,咳嗽吐沫,心烦急躁,午后潮热,天亮汗多。延医调治,始而滋阴,继而补阳,补之不效,则云虚不受补,无可如何。可笑着书者,不分别因弱致病,因病致弱,果系伤寒、瘟疫大病后,气血虚弱,因虚弱而病,自当补弱而病可痊;本不弱而生病,因病久致身弱,自当去病,病去而元气自复。查外无表症,内无里症,所见之症,皆是血瘀之症。常治此症,轻者九付可愈,重者十八付可愈。吃三付后,如果气弱,每日煎黄耆八钱,徐徐服之,一日服完,此攻补兼施之法;若气不甚弱,黄耆不必用,以待病去,元气自复。
	\section{交节病作}
	无论何病,交节病作,乃是瘀血。何以知其是瘀血?每见因血结吐血者,交节亦发,故知之。服三付不发。
	\section{小儿疳症}
	疳病初起,尿如米泔,午后潮热,日久青筋暴露,肚大坚硬,面色青黄,肌肉消瘦,皮毛憔悴,眼睛发艇。古人以此症,在大人为劳病,在小儿为疳疾。照前症再添某病,则曰某疳,如脾疳、疳泻、疳肿、疳痢、肝疳、心疳、疳渴、肺疳、肾疳、疳热、脑疳、眼疳、鼻疳、牙疳、脊疳、蛔疳、无辜疳、丁奚疳、哺露疳,分病十九条,立五十方,方内多有栀子、黄连、羚羊、石膏大寒之品。因论病源系乳食过饱,肥甘无节,停滞中脘,传化迟滞,肠胃渐伤,则生积热,热盛成疳,则消耗气血,煎灼津液,故用大寒以清积热。余初时对症用方,无一效音。后细阅其论,因饮食无节,停滞中脘,此论是停食,不宜大寒之品。以传化迟滞,肠胃渐伤,则生积热之句而论,当是虚热,又不宜用大寒之品。后遇此症,细心审查,午后潮热,至晚尤甚,乃瘀血也,青筋暴露,非筋也,现于皮肤者,血管也,血管青者,内有瘀血;渐至肚大坚硬成块,皆血瘀凝结而成。用通窍活血汤,以通血管;用血府逐瘀汤,去午后潮热;用膈下逐瘀汤,消化积块。三方轮服,未有不愈者。
	\section{通窍活血汤}
	赤芍一钱 川芎一钱 桃仁三钱研泥 红花三钱老葱三根切碎 鲜姜三钱切碎 红枣七个去核麝香五厘绢包
	
	用黄酒半斤,将前七味煎一钟,去渣,将麝香入酒内,再煎二沸,临卧服。方内黄酒,各处分两不同,宁可多二两,不可少,煎至一钟。酒亦无味,虽不能饮酒之人亦可服。方内麝香,市井易于作假,一钱真,可合一两假,人又不能辨,此方麝香最要紧,多费数文,必买好的方妥,若买当门子更佳。大人一连三晚吃三付,隔一日再吃三付。若七、八岁小儿,两晚吃一付,三、两岁小儿,三晚吃一付。麝香可煎三次,再换新的。
	方歌
	
	通窍全凭好麝香,桃红大枣老葱姜,
	
	川芎黄酒赤芍药,表里通经第一方。
	\section{加味止痛没药散}
	
	
	治初起眼疼白珠红,后起云翳。
	
	没药三钱 血竭二钱 大黄三饯 朴硝二钱石决明三钱锻为末,分四付,早晚清茶调服。眼科外症千古一方。
	
	\section{通气散}
	
	
	治耳聋不闻雷声。余三十岁立此方。
	
	柴胡一两 香附一两 川芎五钱为末,早晚开水冲服三钱。
	
	\chapter{血府逐瘀汤所治症目}
	血府逐瘀汤所治之病,开列于后。
	\section{头痛}
	头痛有外感,必有发热,恶寒之表症,发散可愈;有积热,必舌干、口渴,用承气可愈;有气虚,必似痛不痛,用参耆可愈。查患头痛者,无表症,无里症,无气虚、痰饮等症,忽犯忽好,百方下放,用此方一剂而愈。
	\section{胸疼}
	胸疼在前面,用木金散可愈;后通背亦疼,用瓜蒌薤白白酒汤可愈。在伤寒,用瓜蒌、陷胸。柴胡等,皆可愈。有忽然胸疼,前方皆不应,用此方一付,疼立止。
	\section{胸不任物}
	江西巡抚阿霖公,年七十四,夜卧露胸可睡,盖一层布压则不能睡,已经七年。召余诊之,此方五付全愈。
	\section{胸任重物}
	一女二十二岁,夜卧令仆妇坐于胸,方睡,已经二年,余亦用此方,三付而愈,设一齐问病源,何以答之?
	\section{天亮出汗}
	醒后出汗,名曰自汗;因出汗醒,名曰盗汗,盗散人之气血。此是千古不易之定论。竟有用补气固表、滋阴降火,服之下效,而反加重者,不知血瘀亦令人自汗、盗汗。用血府逐瘀汤,一、两付而汗止。
	\section{食自胸右下}
	食自胃管而下,宜从正中。食入咽,有从胸右边咽下者,胃管在肺管之后,仍由肺叶之下转入肺前,由肺下至肺前,出膈膜入腹,肺管正中,血府有瘀血,将胃管挤靠于右。轻则易治,无碍饮食也;重则难治,挤靠胃管,弯而细,有碍饮食也。此方可效,全愈难。
	\section{心里热(名曰灯笼病)}
	身外凉,心里热,故名灯笼病,内有血瘀。认为虚热,愈补愈瘀;认为实火,愈凉愈凝。三、两付,血活热退。
	\section{瞀闷}
	\section{急躁}
	\section{夜睡梦多}
	\section{呃逆(俗名打咯忒)}
	\section{饮水即呛}
	\section{不眠}
	\section{小儿夜啼}
	\section{心跳心忙}
	\section{夜不安}
	\section{俗言肝气病}
	\section{干呕}
	\section{晚发一阵热}
	\section{血府逐瘀汤}
	\chapter{隔下逐瘀汤所治症目}
	\section{积块}
	\section{小儿痞块}
	\section{痛不移处}
	\section{卧则腹坠}
	\section{肾泻}
	\section{久泻}
	\section{隔下逐瘀汤}
	\part{下卷}
	\chapter{半身不遂论叙}
	\section{半身不遂论}
	\section{半身不遂辨}
	\section{半身不遂本源}
	\section{口眼歪斜辨}
	\section{辨口角流涎非痰饮}
	\section{辨大便干燥非风火}
	\section{辨小便频数遗尿不禁}
	\section{辨语言蹇涩非痰人}
	\section{辨口噤咬牙}
	\section{记未病以前之形状}
	\section{论小儿半身不遂}
	\chapter{瘫痿论}
	\section{补阳还五汤}
	\chapter{瘟毒吐泻转筋说}
	\section{解毒活血汤}
	\section{急救回阳汤}
	\chapter{论抽风不是风}
	\section{可保立苏汤}
	\chapter{论痘非胎毒}
	\section{论痘浆不是血化}
	\section{论出痘饮水即呛}
	\section{论七、八天痘疮作痒}
	\section{通经逐瘀汤}
	\section{会厌逐瘀汤}
	\section{止泻调中汤}
	\section{保元化滞汤}
	\section{助阳止痒汤}
	\section{足卫和荣汤}
	\chapter{少腹逐瘀汤说}
	\section{少腹逐瘀汤}
	\chapter{怀胎说}
	\section{古开骨散}
	\section{古没竭散}
	\section{黄耆桃红汤}
	\section{古下瘀血汤}
	\section{抽葫芦酒}
	\section{蜜葱猪胆汤}
	\section{刺猬皮散}
	\section{小茴香酒}
	\chapter{痹症有瘀血说}
	\section{身痛逐瘀汤}
	\section{磠砂丸}
	\section{癫狂梦醒汤}
	\section{龙马自来丹}
	\section{黄耆赤风汤}
	\section{黄耆防风汤}
	\section{黄耆甘草汤}
	\section{木耳散}
	\section{玉龙膏(即胜玉膏)}
	\chapter{辨方效经错之源、论血化为汗之误}
	
\end{document}