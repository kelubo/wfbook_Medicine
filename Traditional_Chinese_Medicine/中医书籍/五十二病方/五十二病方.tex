%-*- coding: UTF-8 -*-
% 五十二病方

\documentclass{ctexbook}

\title{五十二病方}
\author{佚名}
\date{}

\begin{document}
	
	\maketitle
	\tableofcontents
	
	\chapter{前言}
	
	《 马王堆帛书五十二病方》,医方著作,约成书于战国时期,作者失考。1973年出土于湖南长沙马王堆三号汉墓之帛书,是马王堆三号汉墓出土医书中内容最丰富的一种,原无书名,因其目录列有52种病名,且在这些病名之后有“凡五十二”字样,所以整理者据此而给该书命名。是我国现存最早的医方著作。
	
	成书年代约为公元前168年以前。
	 
	《 五十二病方》 帛书现藏湖南省博物馆,马王堆汉墓帛书整理小组所编《 五十二病方》于1979 年由文物出版社出版。
	
	《 五十二病方》现存15000余字,全书分52题(实质上包括100多种疾病),每题都是治疗一类疾病的方法,少则一方、二方,多则20余方。现存医方总数283个,原数应在300个左右,有少部分已残缺了。书中提到的病名现存的有103个,所治包括内、外、妇、儿、五官各科疾病,所载尤以外科病所占比重为大,包括了外伤、动物咬伤、伤痉(破伤风)、痈疽、溃烂、肿瘤、皮肤病和肛肠病。内科疾病有癫痫、疟疾、食病、癃病、痉病、淋病及寄生虫病等;儿科疾病包括癫痫、瘈疭等;此外还涉及了产科病、眼科病等。书中对某些病症的认识,已达到相当的水平。如书中形象地描述了冥病(麻风病)的症状如螟虫啮穿植物内心,其所发无定处,或在鼻,或在口旁,或在齿龈,或在手指,使人鼻缺、指断。反映出当时对这种疾病的发病特点和症状的认识已较为深刻。又如,书中关于“伤痉,痉者,伤,风入伤,身信(伸)而不能诎(屈)”;“伤而颈(痉)者……其病甚弗能饮者,强启其口,为灌之”的记载,清楚地描述了痉病( 破伤风)的两个主要症状枣角弓反张和牙关紧闭。这些记述不仅在中国医学史上是最早的,而且都已被现代医学所证实。 
	
	《 五十二病方》 对药物学、方剂学亦有一定贡献,书中收载药物247种,有草、谷、菜、木、果等植物药,也有兽、禽、鱼、虫等动物药,还有雄黄、水银等矿物药。其中有半数为《 神农本草经》 所不载。书中很多药物的功效和适应症都与后世医药文献和临床实践相吻合。书中还记载了有关药物的采集、收藏方法等,反映了西汉以前药物学的发展。
	
	在处方用药方面,则已初步运用辨证论治原则。《五十二病方》 所载治法多种多样,除了内服汤药之外,尤以外治法最为突出。有敷贴法、药浴法、烟熏或蒸气熏法、熨法、砭法、灸法、按摩法、角法( 火罐疗法)等。治疗手段多样化,也是医药水平提高的标志之一。
	
	《 五十二病方》 所记载的方剂大多是由二味以上药物组成的复方。例如治“疽”病方中,有白敛、黄芪、芍药、桂、姜、椒、茱萸七味药。根据疽病的不同类型,调整主药的剂量,提出“骨疽倍白敛,肉疽倍黄芪,肾疽倍芍药”,体现了早期的辨证论治思想。据对书中283首医方的药物配伍、剂型、方剂用法的分析,认为该书已初具方剂学的基本内容,反映了有理论指导、有实践意义的方剂学体系在先秦已初步形成。《五十二病方》中记载的方剂虽仅明确提及丸剂,但实际上已根据疾病的情况及病人的体质,分别使用了丸、饼、曲、酒、油膏、药浆、汤、散等多种剂型,并对方剂的煎煮法、服药时间、次数、禁忌等作了一定的记载。
	
	《 五十二病方》 书中除外用内服法外,尚有灸、砭、熨、薰等多种外治法。书中有关创伤的16种疗法(止血、镇痛、清创、消毒、包扎等)以及烧灼结扎术、结扎摘除术、瘘管清除术等痔疮手术的记载,反映了当时先进的外科技术。
	
	《 五十二病方》 的出土,填补了《内经》以来我国未有临床医学著作的空白。
		
	《 五十二病方》 保存着远古时期传流下来的若干方药,是古代劳动人民长期与疾病斗争积累起来的宝贵经验。书中对一种疾病有不同的疗法,同一种药物有不同的名称,甚至一个字的写法前后不统一,又如不少的方后注明"尝试","已验","令"(即善)字样,充分证明是劳动人民群众经实践而积累成的,充分反映了西汉以前我国医药学的发展情况。
	
	使用的药物虽然比较简单,但也有药名247个,大约是《神农本草经》的三分之二,可是有一半是《神农本草经》中所未见的,有些药物用了很古老的名称,如“答”就是小豆,还有“啻牛”已不知为何物。对于方剂也是由单味药到药物的配伍使用,全书283方中,除祝由方31方,残缺不可辩认者46方,纯属灸法,熨法,不用药物者9方,其余197方中用单味药78方,两味上者119方,从这里可以看出,先秦时期以一二味药物组成方剂为多见。此时中药理论刚刚产生,正在由单味药应用向多味药配伍的过渡。
	
	与现存中医经典著作《黄帝内经》对照,可以发现《五十二病方》在医学理论和实践方面有着更为原始、古朴的特色,还看不出《内经》中已经出现的五行学说的痕迹,阴阳学说也几乎没有反映,难得提到脏腑,没有各个腧穴(即穴位)的名称,只提到过“泰阴”、“泰阳”两个脉名,书中的治疗方法有灸法、砭法而没有针法。

	\chapter{卷之一}
	\section{医学源流第一}
\end{document}	