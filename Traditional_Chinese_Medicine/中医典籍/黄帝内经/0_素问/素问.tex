%-*- coding: UTF-8 -*-
% 素问

\documentclass[UTF8]{ctexbook}

\usepackage{tocloft}
\renewcommand{\cftchapleader}{\cftdotfill{\cftdotsep}} % 给 chapters 加点

\ctexset{
	part/name= {},
	part/number={},
	chapter/name={},
	chapter/number={}
}

\title{\heiti\zihao{0} 素问}
\date{}

\begin{document}

\maketitle
\tableofcontents
\frontmatter

\chapter{序一}

启玄子王冰撰

夫释缚脱艰,全真导气,拯黎元于仁寿,济赢劣以获安者,非三圣道则不能致之矣。
孔安国序《尚书》曰:伏羲、神农、黄帝之书,谓之三坟,言大道也。
班固《汉书.艺文志》曰:《黄帝内经》 十八卷。
《素问》即其经之九卷也,兼《灵枢》九卷,乃其数焉。虽复年移代革,而授学犹存,惧非其人,而时有所隐,故第七一卷,师氏藏之,今之奉行,惟八卷尔。然而其文简,其意博,其理奥,其趣深,天地之象分,阴阳之候列,变化之由表,死生之兆彰,不谋而遐迩自同,勿约而幽明斯契,稽其言有征,验之事不忒,诚可谓至道之宗,奉生之始矣。
假若天机迅发,妙识玄通,蒇谋虽属乎生知,标格亦资于诂训,未尝有行不由径,出不由户者也。然刻意研精,探微索隐,或识契真要,则目牛无全,故动则有成,犹鬼神幽赞,而命世奇杰,时时间出焉。则周有秦公,汉有淳于公,魏有张公、华公,皆得斯妙道者也。咸日新其用,大济蒸人,华叶递荣,声实相副,盖教之著矣,亦天之假也。

冰弱龄慕道,夙好养生,幸遇真经,式为龟镜。而世本纰缪,篇目重叠,前后不伦,文义悬隔,施行不易,披会亦难,岁月既淹,袭以成弊。或一篇重出,而别立二名;或两论并吞,而都为一目;或问答未已,别树篇题;或脱简不书,而云世缺。重《经合》而冠针服,并《方宜》而为《咳篇》,隔《虚实》而为《逆从》,合经络而为论要,节《皮部》为《经络》,退《至教》以先针,诸如此流,不可胜数。且将升岱岳,非径奚为,欲诣扶桑,无舟莫适。
乃精勤博访,而并有其人,历十二年,方臻理要,询谋得失,深遂夙心。时于先生郭子斋堂,受得先师张公秘本,文字昭晢,义理环周,一以参详,群疑冰释。恐散于末学,绝彼师资,因而撰注,用传不朽。兼旧藏之卷,合八十一篇二十四卷,勒成一部。
冀乎究尾明首,寻注会经,开发童蒙,宣扬至理而已。其中简脱文断,义不相接者,搜求经论所有,迁移以补其处。篇目坠缺,指事不明者,量其意趣,加字以昭其义。篇论吞并,义不相涉,缺漏名目者,区分事类,别目以冠篇首。君臣请问,礼仪乖失者,考校尊卑,增益以光其意。错简碎文,前后重迭者,详其指趣,削去繁杂,以存其要。辞理秘密,难粗论述者,别撰《玄珠》,以陈其道。凡所加字,皆朱书其文,使今古必分,字不杂糅。庶厥昭彰圣旨,敷畅玄言,有如列宿高悬,奎张不乱,深泉净滢,鳞介咸分,君臣无夭枉之期,夷夏有延龄之望,俾工徒勿误,学者惟明,至道流行,徽音累属,千载之后,方知大圣之慈惠无穷。

时大唐宝应元年岁次壬寅序。

将仕郎守殿中丞孙兆重改误

朝奉郎守国子博士同校正医书上骑都尉赐绯鱼袋高保衡

朝奉郎守尚书屯田郎中同校正医书骑都尉赐绯鱼袋孙奇

朝散大夫守光禄卿直秘阁判登闻检院上护军林亿

\chapter{序二}

臣闻安不忘危,存不忘亡者,往圣之先务;求民之瘼,恤民之隐者,上主之深仁。在昔黄帝之御极也,以理身绪余治天下,坐于明堂之上,临观八极,考建五常。以谓人之生也,负阴而抱阳,食味而被色,外有寒暑之相荡,内有喜怒之交侵,夭昏札瘥,国家代有。将欲敛时五福,以敷锡厥庶民,乃与岐伯上穷天纪,下极地理,远取诸物,近取诸身,更相问难,垂法以福万世。于是雷公之伦,授业传之,而内经作矣。历代宝之,未有失坠。苍周之兴,秦和述六气之论,具明于左史。厥后越人得其一二,演而述难经。西汉仓公传其旧学,东汉仲景撰其遗论,晋皇甫谧刺而为甲乙,及隋杨上善纂而为太素。时则有全元起者,始为之训解,缺第七一通。迄唐宝应中,太仆王冰笃好之,得先师所藏之卷,大为次注,犹是三皇遗文,烂然可观。惜乎唐令列之医学,付之执技之流,而荐绅先生罕言之,去圣已远,其术晻昧,是以文注纷错,义理混淆。殊不知三坟之余,帝王之高致,圣贤之能事,唐尧之授四时,虞舜之齐七政,神禹修六府以兴帝功,文王推六子以叙卦气,伊尹调五味以致君,箕子陈五行以佐世,其致一也。奈何以至精至微之道,传之以至下至浅之人,其不废绝,为已幸矣。

顷在嘉祐中,仁宗念圣祖之遗事,将坠于地,乃诏通知其学者,俾之是正。臣等承乏典校,伏念旬岁。遂乃搜访中外,裒集众本,浸寻其义,正其讹舛,十得其三四,余不能具。窃谓未足以称明诏,副圣意,而又采汉唐书
录古医经之存于世者,得数十家,叙而考正焉。贯穿错综,磅礴会通,或端本以寻支,或泝流而讨源,定其可知,次以旧目,正缪误者六千余字,增注义者二千余条,一言去取,必有稽考,舛文疑义,于是详明,以之治身,可以消患于未兆,施于有政,可以广生于无穷。恭惟皇帝抚大同之运,拥无疆之休,述先志以奉成,兴微学而永正,则和气可召,灾害不生,陶一世之民,同跻于寿域矣。

国子博士臣高保衡

光禄卿直秘阁臣林亿等谨上

\mainmatter
\part{aaaa}
\chapter{上古天真论篇}

昔在黄帝,生而神灵,弱而能言,幼而徇齐,长而敦敏,成而登天。

从前的黄帝,生来十分聪明,很小的时候就善于言谈,幼年时对周围事物领会得很快,长大之后,既敦厚又勤勉,及至成年之时,登上了天子之位。

乃问于天师曰:余闻上古之人,春秋皆度百岁,而动作不衰;今时之人,年半百而动作皆衰者,时世异耶?人将失之耶?

他向歧伯问到:我听说上古时候的人,年龄都能超过百岁,动作不显衰老;现在的人,年龄刚至半百,而动作就都衰弱无力了,这是由于时代不同所造成的呢,还是因为今天的人们不会养生所造成的呢?

岐伯对曰:上古之人,其知道者,法于阴阳,和于术数,食饮有节,起居有常,不妄作劳,故能形与神俱,而尽终其天年,度百岁乃去。今时之人不然也,以酒为浆,以妄为常,醉以入房,以欲竭其精,以耗散其真,不知持满,不时御神,务快其心,逆于生乐,起居无节,故半百而衰也。

歧伯回答说:上古时代的人,那些懂得养生之道的,能够取法于天地阴阳自然变化之理而加以适应,调和养生的办法,使之达到正确的标准。饮食有所节制,作息有一定规律,既不妄事操劳,又避免过度的房事,所以能够形神俱旺,协调统一,活到天赋的自然年龄,超过百岁才离开人世;现在的人就不是这样了,把酒当水浆,滥饮无度,使反常的生活成为习惯,醉酒行房,因恣情纵欲,而使阴精竭绝,因满足嗜好而使真气耗散,不知谨慎地保持精气的充满,不善于统驭精神,而专求心志的一时之快,违逆人生乐趣,起居作息,毫无规律,所以到半百之年就衰老了。

夫上古圣人之教下也,皆谓之,虛邪贼风,避之有时;恬淡虚无,真气从之;精神内守,病安从来。是以志闲而少欲,心安而不惧,形劳而不倦,气从以顺,各从其欲,皆得所愿。故美其食,任其服,乐其俗,高下不相慕,其民故曰朴。是以嗜欲不能劳其目,淫邪不能惑其心,愚智贤不肖,不惧于物,故合于道。所以能年皆度百岁而动作不衰者,以其德全不危也。

古代深懂养生之道的人在教导普通人的时候,总要讲到对虚邪贼风等致病因素,应及时避开;心情要清净安闲,排除杂念妄想,以使真气顺畅;精神守持于内,这样疾病就无从发生。因此,人们就可以心志安闲,少有欲望,情绪安定而没有焦虑,形体劳作而不使疲倦,真气因而调顺,各人都能随其所欲而满足自己的愿望。人们无论吃什么食物都觉得甘美,随便穿什么衣服也都感到满意,大家喜爱自己的风俗习尚,愉快地生活,社会地位无论高低,都不相倾慕,所以这些人称得上朴实无华。因而任何不正当的都嗜欲都不会引起他们注目,任何淫乱邪僻的事物也都不能惑乱他们的心志。无论愚笨的,聪明的,能力大的还是能力小的,都不因外界事物的变化而动心焦虑,所以符合养生之道。他们之所以能够年龄超过百岁而动作不显得衰老,正是由于领会和掌握了修身养性的方法而身体不被内外邪气干扰危害所致。

帝曰:人年老而无子者,材力尽耶?将天数然也?  

岐伯曰:女子七岁,肾气盛,齿更发长。

二七,而天癸至,任脉通,太冲脉盛,月事以时下,故有子。

三七,肾气平均,故真牙生而长极。

四七,筋骨坚,发长极,身体盛壮。

五七,阳明脉衰,面始焦,发始堕。

六七,三阳脉衰于上,面皆焦,发始白。

七七,任脉虚,太冲脉衰少,天癸竭,地道不通,故形坏而无子也。

丈夫八岁,肾气实,发长齿更。
二八,肾气盛,天癸至,精气溢泻,阴阳和,故能有子。
三八,肾气平均,筋骨劲强,故真牙生而长极。
四八,筋骨隆盛,肌肉满壮。
五八,肾气衰,发堕齿槁。
六八,阳气衰竭于上,面焦,发鬓颁白。
七八,肝气衰,筋不能动,天癸竭,精少,肾脏衰,形体皆极。
八八,则齿发去。
肾者主水,受五脏六腑之精而藏之,故五脏盛,乃能泻。
今五脏皆衰,筋骨解堕,天癸尽矣,故发鬓白,身体重,行步不正,而无子耳。

帝曰:有其年已老而有子者,何也?
岐伯曰:此其夭寿过度,气脉常通,而肾气有余也。此虽有子,男子不过尽八八,女子不过尽七七,而天地之精气皆竭矣。
帝曰:夫道者能却老而全形,身年虽寿,能生子也。

黄帝曰:余闻上古有真人者,提擎天地,把握阴阳,呼吸精气,独立守神,肌肉若一,故能寿敝天地,无有终时,此其道生。
中古之时,有至人者,淳德全道,和于阴阳,调于四时,去世离俗,积精全神,游行天地之间,视听八达之外,此盖益其寿命而强者也,亦归于真人。
其次有圣人者,处天地之和,从八风之理,适嗜欲于世俗之间,无恚嗔之心,行不欲离于世,被服章,举不欲观于俗,外不劳形于事,内无思想之患,以恬愉为务,以自得为功,形体不敝,精神不散,亦可以百数。
其次有贤人者,法则天地,象似日月,辨列星辰,逆从阴阳,分别四时,将从上古合同于道,亦可使益寿而有极时。



黄帝说:人年纪老的时候,不能生育子女,是由于精力衰竭了呢,还是受自然规律的限定呢?岐伯说:女子到了七岁,肾气盛旺了起来,乳齿更换,头发开始茂盛。十四岁时,天癸产生,任脉通畅,太冲脉旺盛,月经按时来潮,具备了生育子女的能力。二十一岁时,肾气充满,真牙生出,牙齿就长全了。二十八岁时,筋骨强健有力,头发的生长达到最茂盛的阶段,此时身体最为强壮。三十五岁时,阳明经脉气血渐衰弱,面部开始憔悴,头发也开始脱落。四十二岁时,三阳经脉气血衰弱,面部憔悴无华,头发开始变白。四十九岁时,任脉气血虚弱,太冲脉的气血也衰少了,天葵枯竭,月经断绝,所以形体衰老,失去了生育能力。男子到了八岁,肾气充实起来,头发开始茂盛,乳齿也更换了,十六岁时,肾气旺盛,天癸产生,精气满溢而能外泻,两性交合,就能生育子女。二十四岁时,肾气充满,筋骨强健有力,真牙生长,牙齿长全。三十二岁时,筋骨丰隆盛实,肌肉亦丰满健壮。四十岁时,肾气衰退,头发开始脱落,牙齿开始枯槁。四十八岁时,上部阳气逐渐衰竭,面部憔悴无华,头发和两鬓花白。五十六岁时,肝气衰弱,筋的活动不能灵活自如。六十四岁时,天癸枯竭,精气少,肾脏衰,牙齿头发脱落,形体衰疲。肾主水,接受其他各脏腑的精气而加以贮藏,所以五脏功能旺盛,肾脏才能外泻精气。现在年老,五脏功能都已衰退,筋骨懈惰无力,天癸以竭。所以发鬓都变白,身体沉重,步伐不稳,也不能生育子女了。黄帝说:有的人年纪已老,仍能生育,是什麽道理呢?岐伯说:这是他天赋的精力超过常人,气血经脉保持畅通,肾气有余的缘故。这种人虽有生育能力,但男子一般不超过六十四岁,女子一般不超过四十九岁,精气变枯竭了。黄帝说:掌握养生之道的人,年龄都可以达到一百岁左右,还能生育吗?岐伯说:掌握养生之道的人,能防止衰老而保全形体,虽然年高,也能生育子女。
我听说上古时代有称为真人的人,掌握了天地阴阳变化的规律,能够调节呼吸,吸收精纯的清气,超然独处,令精神守持于内,锻炼身体,使筋骨肌肉与整个身体达到高度的协调,所以他的寿命同于天地而没有终了的时候,这是他修道养生的结果。中古的时候,有称为至人的人,具有淳厚的道德,能全面地掌握养生之道,和调于阴阳四时的变化,离开世俗社会生活的干扰,积蓄精气,集中精神,使其远驰于广阔的天地自然之中,让视觉和听觉的注意力守持于八方之外,这是他延长寿命和强健身体的方法,这种人也可以归属真人的行列。其次有称为圣人的人,能够安处于天地自然的正常环境之中,顺从八风的活动规律,使自己的嗜欲同世俗社会相应,没有恼怒怨恨之情,行为不离开世俗的一般准则,穿着装饰普通纹彩的衣服,举动也没有炫耀于世俗的地方,在外,他不使形体因为事物而劳累,在内,没有任何思想负担,以安静、愉快为目的,以悠然自得为满足,所以他的形体不易衰惫,精神不易耗散,寿命也可达到百岁左右。其次有称为贤人的人,能够依据天地的变化,日月的升降,星辰的位置,以顺从阴阳的消长和适应四时的变迁,追随上古真人,使生活符合养生之道,这样的人也能增益寿命,但有终结的时候。

\chapter{四气调神大论篇}

春三月,此谓发陈,天地俱生,万物以荣,夜卧早起,广步于庭,被发缓形,以使志生,生而勿杀,予而勿夺,赏而勿罚,此春气之应,养生之道也。逆之则伤肝,夏为寒变,奉长者少。  

夏三月,此谓蕃秀,天地气交,万物华实,夜卧早起,无厌于日,使志勿怒,使华英成秀,使气得泄,若所爱在外,此夏气之应,养长之道也。逆之则伤心,秋为痎疟,奉收者少,冬至重病。  

秋三月,此谓容平,天气以急,地气以明,早卧早起,与鸡俱兴,使志安宁,以缓秋刑,收敛神气,使秋气平,无外其志,使肺气清,此秋气之应,养收之道也。逆之则伤肺,冬为飧泄,奉藏者少。  

冬三月,此谓闭藏,水冰地坼,无扰乎阳,早卧晚起,必待日光,使志若伏若匿,若有私意,若已有得,去寒就温,无泄皮肤,使气极夺,此冬气之应,养藏之道也。逆之则伤肾,春为痿厥,奉生者少。  

天气,清净光明者也,藏德不止,故不下也。  
天明则日月不明,邪害空窍,阳气者闭塞,地气者冒明,云雾不精,则上应白露不下。  
交通不表,万物命故不施,不施则名木多死。  
恶气不发,风雨不节,白露不下,则菀稾不荣。  
贼风数至,暴雨数起,天地四时不相保,与道相失,则未央绝灭。  
唯圣人从之,故身无奇病,万物不失,生气不竭。  

逆春气,则少阳不生,肝气内变。  
逆夏气,则太阳不长,心气内洞。  
逆秋气,则太阴不收,肺气焦满。  
逆冬气,则少阴不藏,肾气独沉。  
夫四时阴阳者,万物之根本也。所以圣人春夏养阳,秋冬养阴,以从其根,故与万物沉浮于生长之门。逆其根,则伐其本,坏其真矣。  

故阴阳四时者,万物之始终也,死生之本也,逆之则灾害生,从之则苛疾不起,是谓得道。  
道者,圣人行之,愚者佩之。从阴阳则生,逆之则死,从之则治,逆之则乱。反顺为逆,是谓内格。  
是故圣人不治已病治未病,不治已乱治未乱,此之谓也。夫病已成而后药之,乱已成而后治之,譬犹渴而穿井,斗而铸锥,不亦晚乎!

春季的三个月,谓之发陈,是推陈出新,生命萌发的时令。天地自然,都富有生气,万物显得欣欣向荣。此时,人们应该入夜即睡眠,早些起身,披散开头发,解开衣带,使形体舒缓,放宽步子,在庭院中漫步,使精神愉快,胸怀开畅,保持万物的生机。不要滥行杀伐,多施与,少敛夺,多奖励,少惩罚,这是适应春季的时令,保养生发之气的方法。如果违逆了春生之气,便会损伤肝脏,使提供给夏长之气的条件不足,到夏季就会发生寒性病变。
夏季的三个月,谓之蕃秀,是自然界万物繁茂秀美的时令。此时,天气下降,地气上腾,天地之气相交,植物开花结实,长势旺盛,人们应该在夜晚睡眠,早早起身,不要厌恶长日,情志应保持愉快,切勿发怒,要使精神之英华适应夏气以成其秀美,使气机宣畅,通泄自如,精神外向,对外界事物有浓厚的兴趣。这是适应夏季的气候,保护长养之气的方法。如果违背了夏长之气,就会损伤心脏,使提供给秋收之起的条件不足,到秋天容易发生疟疾,冬天再次发生疾病。
秋季的三个月,谓之容平,自然界景象因万物成熟而平定收敛。此时,天高风急,地气清肃,人应早睡早起,和鸡的活动时间相仿,以保持神志的安宁,减缓秋季肃杀之气对人体的影响;收敛神气,以适应秋季容平的特征,不使神思外驰,以保持肺气的清肃功能,这就是适应秋令的特点而保养人体收敛之气的方法。若为违逆了秋收之气,就会伤及肺脏,使提供给冬藏之气的条件不足,冬天就要发生飧泄病。
冬天的三个月,谓之闭藏,是生机潜伏,万物蛰藏的时令。当此时节,水寒成冰,大地龟裂,人应该早睡晚起,待到日光照耀时起床才好,不要轻易地扰动阳气,妄事操劳,要使神志深藏于内,安静自若,好象有个人的隐秘,严守而不外泄,又象得到的渴望得到的东西,把他密藏起来一样;要躲避寒冷,求取温暖,不要使皮肤开泄而令阳气不断地损失,这是适应冬季的气候而保养人体闭藏机能的方法。违逆了冬令的闭藏之气,就要损伤肾脏,使提供给春生之气的条件不足,春天就会发生痿厥之疾。
的力量而不会下泄。如果天气阴霾晦暗,就会出现日月昏暗,阴霾邪气侵害山川,阳气闭塞不通,大地昏蒙不明,云雾弥漫,日色无光,相应的雨露不能下降。天地之气不交,万物的生命就不能绵延。生命不能绵延,自然界高大的树木也会死亡。恶劣的气候发作,风雨无时,雨露当降而不降,草木不得滋润,生机郁塞,茂盛的禾苗也会枯槁不荣。贼风频频而至,暴雨不时而作,天地四时的变化失去了秩序,违背了正常的规律,致使万物的生命未及一半就夭折了。只有圣人能适应自然变化,注重养生之道,所以身无大病,因不背离自然万物得发展规律,而生机不会竭绝。
违逆了春生之气,少阳就不会生发,以致肝气内郁而发生病变。违逆了夏长之气,太阳就不能盛长,以致心气内虚。违逆了秋收之气,太阴就不能收敛,以致肺热叶焦而胀满。违逆了冬藏之气,少阴就不能潜藏,以致肾气不蓄,出现注泻等疾病。
四时阴阳的变化,是万物生命的根本,所以圣人在春夏季节保养阳气以适应生长的需要,在秋冬季节保养阴气以适应收藏的需要,顺从了生命发展的根本规律,就能与万物一样,在生、长、收、藏的生命过程中运动发展。如果违逆了这个规律,就会牋伐生命力,破坏真元之气。因此,阴阳四时是万物的终结,是盛衰存亡的根本,违逆了它,就会产生灾害,顺从了它,就不会发生重病,这样变可谓懂得了养生之道。对于养生之道,圣人能够加以实行,愚人则时常有所违背。
顺从阴阳的消长,就能生存,违逆了就会死亡。顺从了它,就会正常,违逆了它,就会乖乱。相反,如背道而行,就会使机体与自然环境相格拒。所以圣人不等病已经发生再去治疗,而是治疗在疾病发生之前,如同不等到乱事已经发生再去治疗,而是治疗在它发生之前。如果疾病已发生,然后再去治疗,乱子已经形成,然后再去治理,那就如同临渴而掘井,战乱发生了再去制造兵器,那不是太晚了吗?

\chapter{生气通天论篇}

黄帝曰:夫自古通天者,生之本,本于阴阳。天地之间,六合之内,其气九州、九窍、五藏、十二节,皆通乎天气。其生五,其气三。数犯此者,则邪气伤人,此寿命之本也。  
苍天之气,清净则志意治,顺之则阳气固,虽有贼邪,弗能害也。此因时之序。故圣人传精神,服天气,而通神明,失之则内闭九窍,外壅肌肉,卫气散解,此谓自伤,气之削也。  
阳气者若天与日,失其所,则折寿而不彰故天运当以日光明,是故阳因而上,卫外者也。  
因于寒,欲如运枢,起居如惊,神气乃浮。因于暑,汗烦则喘喝,静则多言,体若燔炭,汗出而散。因于湿,首如裹,湿热不攘,大筋软短,小筋驰长,软短为拘,驰长为痿。因于气,为肿,四维相代,阳气乃竭。  
阳气者,烦劳则张,精绝。辟积于夏,使人煎厥。目盲不可以视,耳闭不可能听,溃溃乎若坏都,汩汩乎不可止。阳气者,大怒则形气绝;而血菀于上,使人薄厥,有伤于筋,纵,其若不容,汗出偏沮,使人偏枯。汗出见湿,乃坐痤痱。高梁之变,足生大丁,受如持虚。劳汗当风,寒薄为皴,郁乃痤。  
阳气者,精则养神,柔则养筋。开阖不得,寒气从之,乃生大偻;陷脉为瘘,留连肉腠,俞气化薄,传为善畏,及为惊骇;营气不从,逆于肉理,乃生臃肿;魄汗未尽,形弱而气烁,穴俞以闭,发为风疟。  
故风者,百病之始也。清静则肉腠闭拒,虽有大风苛毒,弗之能害,此因时之序也。  
故病久则传化,上下不并,良医弗为。故阳畜积病死,而阳气当隔,隔者当写,不亟正治,粗乃败之。故阳气者,一日而主外,平旦人气生,日中而阳气隆,日西而阳气已虚,气门乃闭。是故暮而收拒,无扰筋骨,无见雾露,反此三时,形乃困薄。  
岐伯曰:阴者,藏精而起亟也;阳者,卫外而为固也。阴不胜其阳,则脉流薄疾,并乃狂;阳不胜其阴,则五藏气争,九窍不通。是以圣人陈阴阳,筋脉和同,骨髓坚固,血气皆从;如是则内外调和,邪不能害,耳目聪明,气立如故。  
风客淫气,精乃亡,邪伤肝也。因而饱食,筋脉横解,肠癖为痔;因而大饮,则气逆;因而强力,肾气乃伤,高骨乃坏。  
凡阴阳之要,阳密乃固,两者不和,若春无秋,若冬无夏,因而和之,是谓圣度。故阳强不能密,阴气乃绝;阴平阳秘,精神乃治;阴阳离决,精气乃绝。  
于露风,乃生寒热。是以春伤于风,邪气留连,乃为洞泄;夏伤于暑,秋为痎疟;秋伤于湿,上逆而咳,发为痿厥;冬伤于寒,春必温病。四时之气,更伤五藏。  
阴之所生,本在五味,阴之五宫,伤在五味。是故味过于酸,肝气以津,脾气乃绝;味过于咸,大骨气劳,短肌,心气抑;味过于甘,心气喘满,色黑,肾气不衡;味过于苦,脾气不濡,胃气乃厚;味过于辛,筋脉沮驰,精神乃央。是故谨和五味,骨正筋柔,气血以流,腠理以密,如是则骨气以精。谨道如法,长有天命。

黄帝说:自古以来,都以通于天气为生命的根本,而这个根本不外天之阴阳。天地之间,六合之内,大如九州之域,小如人的九窍、五脏、十二节,都与天气相通。天气衍生五行,阴阳之气又依盛衰消长而各分为三。如果经常违背阴阳五行的变化规律,那麽邪气就会伤害人体。因此,适应这个规律是寿命得以延续的根本。
苍天之气清净,人的精神就相应地调畅平和,顺应天气的变化,就会阳气固密,虽有贼风邪气,也不能加害于人,这是适应时序阴阳变化的结果。所以圣人能够专心致志,顺应天气,而通达阴阳变化之理。如果违逆了适应天气的原则,就会内使九窍不通,外使肌肉壅塞,卫气涣散不固,这是由于人们不能适应自然变化所致,称为自伤,阳气会因此而受到削弱。
人身的阳气,如象天上的太阳一样重要,假若阳气失却了正常的位次而不能发挥其重要作用,人就会减损寿命或夭折,生命机能亦暗弱不足。所以天体的正常运行,是因太阳的光明普照而显现出来,而人的阳气也应在上在外,并起到保护身体,抵御外邪的作用。
由于寒,阳气应如门轴在门臼中运转一样活动于体内。若起居猝急,扰动阳气,则易使神气外越。因于暑,则汗多烦躁,喝喝而喘,安静时多言多语。若身体发高热,则象碳火烧灼一样,一经出汗,热邪就能散去。因于湿,头部象有物蒙裹一样沉重。若湿热相兼而不得排除,则伤害大小诸筋,而出现短缩或弛纵,短缩的造成拘挛,弛纵的造成痿弱。由于风,可致浮肿。以上四种邪气维系缠绵不离,相互更代伤人,就会使阳气倾竭。
在人体烦劳过度时,阳气就会亢盛而外张,使阴精逐渐耗竭。如此多次重复,阳愈盛而阴愈亏,到夏季暑热之时,便易使人发生煎厥病,发作的时候眼睛昏蒙看不见东西,耳朵闭塞听不到声音,混乱之时就象都城崩毁,急流奔泻一样不可收拾。
人的阳气,在大怒时就会上逆,血随气生而淤积于上,与身体其他部位阻隔不通,使人发生薄厥。若伤及诸筋,使筋弛纵不收,而不能随意运动。经常半身出汗,可以演变为半身不遂。出汗的时候,遇到湿邪阻遏就容易发生小的疮疮和痱子。经常吃肥肉精米厚味,足以导致发生疔疮,患病很容易,就象以空的容器接收东西一样。在劳动汗出时遇到风寒之邪,迫聚于皮腠形成粉刺,郁积化热而成疮疖。人的阳气,既能养神而使精神慧爽,又能养筋而使诸筋柔韧。汗孔的开闭调节失常,寒气就会随之侵入,损伤阳气,以致筋失所养,造成身体俯曲不伸。寒气深陷脉中,留连肉腠之间,气血不通而郁积,久而成为疮瘘。从腧穴侵入的寒气内传而迫及五脏,损伤神志,就会出现恐惧和惊骇的症象。由于寒气的稽留,营气不能顺利地运行,阻逆于肌肉之间,就会发生痈肿。汗出未止的时候,形体与阳气都受到一定的消弱,若风寒内侵,俞穴闭阻,就会发生风疟。
风是引起各种疾病的其始原因,而只要人体保持精神的安定和劳逸适度等养生的原则,那麽,肌肉腠理就会密闭而有抗拒外邪的能力,虽有大风苛毒的侵染,也不能伤害,这正是循着时序的变化规律保养生气的结果。
病久不愈,邪留体内,则会内传并进一步演变,到了上下不通、阴阳阻隔的时候,虽有良医,也无能为力了。所以阳气蓄积,郁阻不通时,也会致死。对于这种阳气蓄积,阻隔不通者,应采用通泻的方法治疗,如不迅速正确施治,而被粗疏的医生所误,就会导致死亡。人身的阳气,白天主司体表:清晨的时候,阳气开始活跃,并趋向于外,中午时,阳气达到最旺盛的阶段,太阳偏西时,体表的阳气逐渐虚少,汗孔也开始闭合。所以到了晚上,阳气收敛拒守于内,这时不要扰动筋骨,也不要接近雾露。如果违反了一天之内这三个时间的阳气活动规律,形体被邪气侵扰则困乏而衰薄。
岐伯说:阴是藏精于内不断地扶持阳气的;阳是卫护于外使体表固密的。如果阴不胜阳,阳气亢盛,就使血脉流动迫促,若再受热邪,阳气更盛就会发为狂症。如果阳不胜阴,阴气亢盛,就会使五脏之气不调,以致九窍不通。所以圣人使阴阳平衡,无所偏胜,从而达到筋脉调和,骨髓坚固,血气畅顺。这样,则会内外调和,邪气不能侵害,耳目聪明,气机正常运行。
风邪侵犯人体,伤及阳气,并逐步侵入内脏,阴精也就日渐消亡,这是由于邪气伤肝所致。若饮食过饱,阻碍升降之机,会发生筋脉弛纵、肠澼及痔疮等病症。若饮酒过量,会造成气机上逆。若过度用力,会损伤肾气,腰部脊骨也会受到损伤。
大凡阴阳的关键,以阳气的致密最为重要。阳气致密,阴气就能固守于内。阴阳二者不协调,就象一年之中,只有春天而没有秋天,只有冬天而没有夏天一样。因此,阴阳的协调配合,相互为用,是维持正常生理状态的最高标准。所以阳气亢盛,不能固密,阴气就会竭绝。阴气和平,阳气固密,人的精神才会正常。如果阴阳分离决绝,人的精气就会随之而竭绝。
由于雾露风寒之邪的侵犯,就会发生寒热。春天伤于风邪,留而不去,会发生急骤的泄泻。夏天伤于暑邪,到秋天会发生疟疾病。秋天伤于湿邪,邪气上逆,会发生咳嗽,并且可能发展为痿厥病。冬天伤于寒气,到来年的春天,就要发生温病。四时的邪气,交替伤害人的五脏。
阴精的产生,来源于饮食五味。储藏阴精的五脏,也会因五味而受伤,过食酸味,会使肝气淫溢而亢盛,从而导致脾气的衰竭;过食咸味,会使骨骼损伤,肌肉短缩,心气抑郁;过食甜味,会使心气满闷,气逆作喘,颜面发黑,肾气失于平衡;过食苦味,会使脾气过燥而不濡润,从而使胃气壅滞;过食辛味,会使筋脉败坏,发生弛纵,精神受损。因此谨慎地调和五味,会使骨骼强健,筋脉柔和,气血通畅,腠理致密,这样,骨气就精强有力。所以重视养生之道,并且依照正确的方法加以实行,就会长期保有天赋的生命力。

\chapter{金匮真言论篇}

黄帝曰:天有八风,经有五风,何谓?  
岐伯对曰:八风发邪,以为经风,触五藏,邪气发病。所谓得四时之胜者:春胜长夏,长夏胜冬,冬胜夏,夏胜秋,秋胜春,所谓四时之胜也。  
东风生于春,病在肝,俞在颈项;南风生于夏,病在心,俞在胸胁;西风生于秋,病在肺,俞在肩背;北风生于冬,病在肾,俞在腰股;中央为土,病在脾,俞在脊。  
故春气者,病在头;夏气者,病在藏;秋气者,病在肩背;冬气者,病在四支。  
故春善病鼽衄,仲夏善病胸胁,长夏善病洞泄寒中,秋善病风疟,冬善病痹厥。  
故冬不按跷,春不鼽衄,春不病颈项,仲夏不病胸胁,长夏不病洞泄寒中,秋不病风疟,冬不病痹厥,飧泄而汗出也。  
夫精者,身之本也。故藏于精者,春不病温。夏暑汗不出者,秋成风疟。此平人脉法也。  
故曰:阴中有阴,阳中有阳。平旦至日中,天之阳,阳中之阳也;日中至黄昏,天之阳,阳中之阴也;合夜至鸡鸣,天之阴,阴中之阴也;鸡鸣至平旦,天之阴,阴中之阳也。故人亦应之。  
夫言人之阴阳,则外为阳,内为阴;言人身之阴阳,则背为阳,腹为阴;言人身之藏腑中阴阳,则藏者为阴,腑者为阳,肝、心、脾、肺、肾五藏皆为阴,胆、胃、大肠、小肠、膀胱、三焦六腑皆为阳。所以欲知阴中之阴、阳中之阳者何也?为冬病在阴,夏病在阳,春病在阴,秋病在阳,皆视其所在,为施针石也。故背为阳,阳中之阳,心也;背为阳,阳中之阳,肺也;腹为阴,阴中之阴,肾也;腹为阴,阴中之阳,肝也;腹为阴,阴中之至阴,脾也。此皆阴阳表里、内外、雌雄相输应也,故以应天之阴阳也。
帝曰:五藏应四时,各有收受乎?岐伯曰:有。东方青色,入通于肝,开窍于目,藏精于肝,其病发惊骇;其味酸,其类草木,其畜鸡,其谷麦,其应四时,上为岁星,是以春气在头也,其音角,其数八,是以知病之在筋也,其臭臊。  
南方赤色,入通于心,开窍于耳,藏精于心,故病在五藏;其味苦,其类火,其畜羊,其谷黍,其应四时,上为荧惑星,是以知病之在脉也,其音徵,其数七,其臭焦。  
中央黄色,入通于脾,开窍于口,藏精于脾,故病在舌本;其味甘,其类土,其畜牛,其谷稷,其应四时上为镇星,是以知病在肉也,其音宫,其数五,其臭香。  
西方白色,入通于肺,开窍于鼻,藏精于肺,故病在背;其味辛,其类金,其畜马,其谷稻,其应四时,上为太白星,是以知病之在皮毛也,其音商,其数九,其臭腥。  
北方黑色,入通于肾,开窍于二阴,藏精于肾,故病在溪;其味咸,其类水,其畜彘,其谷豆,其应四时,上为辰星,是以知病之在骨也,其音羽,其数六,其臭腐。  
故善为脉者,谨察五藏六府,一逆一从,阴阳、表里、雌雄之纪,藏之心意,合心于精。非其人勿教,非其真勿授,是谓得道。

黄帝问道:自然界有八风,人的经脉病变又有五风的说法,这是怎麽回事呢?岐伯答说:自然界的八风是外部的致病邪气,他侵犯经脉,产生经脉的风病,风邪还会继续循经脉而侵害五脏,使五脏发生病变。一年的四个季节,有相克的关系,如春胜长夏,长夏胜冬,冬胜夏,夏胜秋,冬胜春,某个季节出现了克制它的季节气候,这就是所谓四时相胜。
东风生于春季,病多发生在肝,肝的经气输注于颈项。南风生于夏季,病多发生于心,心的经气输注于胸胁。西风生于秋季,病多发生在肺,肺的经气输注于肩背。北风生于冬季,病多发生在肾,肾的经气输注于腰股。长夏季节和中央的方位属于土,病多发生在脾,脾的经气输注于脊。所以春季邪气伤人,多病在头部:夏季邪气伤人,多病在心:秋季邪气伤人,多病在肩背:冬季邪气伤人,多病在四肢。春天多发生嬶衄,夏天多发生在胸胁方面的疾患,长夏季多发生洞泄等里寒证,秋天多发生风疟,冬天多发生痹厥。若冬天不进行按跤等扰动阳气的活动,来年春天就不会发生嬶衄和颈项部位的疾病,夏天就不会发生胸胁的疾患,长夏季节就不会发生洞泄一类的里寒病,秋天就不会发生风疟病,冬天也不会发生痺厥、飨泄、汗出过多等病症。精,是人体的根本,所以阴精内藏而不妄泄,春天就不会得温热病。夏暑阳盛,如果不能排汗散热,到秋天就会酿成风疟病。这是诊察普通人四时发病的一般规律。
所以说:阴阳之中,还各有阴阳。白昼属阳,平旦到中午,为阳中之阳。中午到黄昏,则束阳中之阴。黑夜属阴,合夜到鸡鸣,为阴中之阴。鸡鸣到平旦,则属阴中之阳。黑夜属阴,合夜到鸡鸣,为阴中之阴。鸡鸣到平旦,则属阴中之阳。人的情况也与此相应。就人体阴阳而论,外部属阳,内部属阴。就身体的部位来分阴阳,则背为阳,腹为阴。从脏腑的阴阳划分来说,则脏属阴,腑属阳,肝、心、脾、肺、肾五脏都属阴。胆、胃、大肠、小肠、膀胱三焦六腑都属阳。了解阴阳之中复有阴阳的道理是什麽呢?这是要分析四时疾病的在阴在阳,以作为治疗的依据,如冬病在阴,夏病在阳,春病在阴,秋病在阳,都要根据疾病的部位来施用针刺和砭石的疗法。此外,背为阳,阳中之阳为心,阳中之阴为肺。腹为阴,阴中之阴为肾,阴中之阳为肝,阴中的至阴为脾。以上这些都是人体阴阳表里、内外雌雄相互联系又相互对应的例证,所以人与自然界的阴阳是相应的。
黄帝说:五脏除与四时相应外,它们各自还有相类的事物可以归纳起来吗?岐伯说:有。比如东方青色,与肝相通,肝开窍于目,精气内藏于肝,发病常表现为惊骇,在五味为酸,与草木同类,在五蓄为鸡,在五谷为麦,与四时中的夏季相应,在天体为岁星,春天阳气上升,所以其气在头,在五音为角,其成数为八,因肝主筋所以它的疾病多发生在筋。此外,在嗅味为臊。南方赤色,与心相通,心开窍于耳,精气内藏与心,在五味为苦,与火同类,在五畜为羊,在五谷为黍,与四时中的夏季相应,在天体为荧惑星,他的疾病多发生在脉和五脏,在五音为徵,其成数为七。此外,在嗅味为焦。中央黄色,与脾相通,脾开窍于口,精气内藏于脾,在五味为甘,与土同类,在五畜为牛,在五谷为稷,与四时中的长夏相应,在天体为镇星,他的疾病多发生在舌根和肌肉,在五音为宫,其生数为五。此外,在嗅味为香。西方白色,与肺相通,肺开窍于鼻,精气内藏于肺,在五味为辛,与金同类,在五畜为马,在五谷为稻,与四时中的秋季相应,在天体为太白星,他的疾病多发生在背部和皮毛,在五音为商,其成数为九。此外,在嗅味为腥。北方黑色,与肾相同,肾开窍于前后二阴,精气内藏于肾,在五味为咸,与水同类,在五畜为彘,在五谷为豆,与四时中的冬季相应,在天体为辰星,他的疾病多发生在溪和骨,在五音为羽,其成数为六。此外,其嗅味为腐。所以善于诊脉的医生,能够谨慎细心地审查五脏六腑的变化,了解其顺逆的情况,把阴阳、表里、雌雄的对应和联系,纲目分明地加以归纳,并把这些精深的道理,深深地记在心中。这些理论,至为宝贵,对于那些不是真心实意地学习而又不具备一定条件的人,切勿轻易传授,这才是爱护和珍视这门学问的正确态度。

\chapter{阴阳应象大论篇 第五}
黄帝曰:阴阳者,天地之道也,万物之纲纪,变化之父母,生杀之本始,神明之府也。治病必求于本。故积阳为天,积阴为地。阴静阳躁,阳生阴长,阳杀阴藏。阳化气,阴成形。寒极生热,热极生寒;寒气生浊,热气生清;清气在下,则生飧世,浊气在上,则生真胀。此阴阳反作,病之逆从也。  
故清阳为天,浊阴为地。地气上为云,天气下为雨;雨出地气,云出天气。故清阳出上窍,浊阴出下窍;清阳发腠理,浊阴走五藏;清阳实四支,浊阴归六府。  
水为阴,火为阳。阳为气,阴为味。味归形,形归气,气归精,精归化;精食气,形食味,化生精,气生形。味伤形,气伤精,精化为气,气伤于味。  
阴味出下窍,阳气出上窍。味厚者为阴,薄为阴之阳;气厚者为阳,薄为阳之阴。味厚则泄,薄则通;气薄则发泄,厚则发热。壮火之气衰,少火之气壮,壮火食气,气食少火,壮火散气,少火生气。气味辛甘发散为阳,酸苦涌泄为阴。  
阴胜则阳病,阳性则阴病。阳胜则热,阴胜则寒。重寒则热,重热则寒。寒伤形,热伤气;气伤痛,形伤肿。故先痛而后肿者,气伤形也;先肿而后痛者,形伤气也。  
风胜则动,热胜则肿,燥胜则干,寒胜则浮,湿胜则濡泻。  
天有四时五行,以生长收藏,以生寒暑燥湿风。人有五藏化五气,以生喜怒悲忧恐。故喜怒伤气,寒暑伤形。暴怒伤阴,暴喜伤阳。厥气上行,满脉去形。喜怒不节,寒暑过度,生乃不固。故重阴必阳,重阳必阴。故曰:冬伤于寒,春必温病;春伤于风,夏生飧泄;夏伤于暑,秋必痎疟;秋伤于湿,冬生咳嗽。  
帝曰:余闻上古圣人,论理人形,列别藏府,端络经脉,会通六合,各从其经;气穴所发,各有处名;溪谷属骨,皆有所起;分部逆从,各有条理;四时阴阳,尽有经纪;外内之应,皆有表里,其信然乎?  
岐伯对曰:东方生风,风生木,木生酸,酸生肝,肝生筋,筋生心,肝主目。其在天为玄,在人为道,在地为化。化生五味,道生智,玄生神。神在天为风,在地为木,在体为筋,在藏为肝,在色为苍,在音为角,在声为呼,在变动为握,在窍为目,在味为酸,在志为怒。怒伤肝,悲胜怒;风伤筋,燥胜风;酸伤筋,辛胜酸。  
南方生热,热生火,火生苦,苦生心,心生血,血生脾,心主舌。其在天为热,在地为火,在体为脉,在藏为心,在色为赤,在音为徵,在声为笑,在变动为忧,在窍为舌,在味为苦,在志为喜,喜伤心,恐胜喜;热伤气,寒胜热,苦伤气,咸胜苦。  
中央生湿,湿生土,土生甘,甘生脾,脾生肉,肉生肺,脾主口。其在天为湿,在地为土,在体为肉,在藏为脾,在色为黄,在音为宫,在声为歌,在变动为哕,在窍为口,在味为甘,在志为思。思伤脾,怒胜思;湿伤肉,风胜湿;甘伤肉,酸胜甘。  
西方生燥,燥生金,金生辛,辛生肺,肺生皮毛,皮毛生肾,肺主鼻。其在天为燥,在地为金,在体为皮毛,在藏为肺,在色为白,在音为商,在声为哭,在变动为咳,在窍为鼻,在味为辛,在志为忧。忧伤肺,喜胜忧;热伤皮毛,寒胜热;辛伤皮毛,苦胜辛。  
北方生寒,寒生水,水生咸,咸生肾,肾生骨髓,髓生肝,肾主耳。其在天为寒,在地为水,在体为骨,在藏为肾,在色为黑,在音为羽,在声为呻,在变动为栗,在窍为耳,在味为咸,在志为恐。恐伤肾,思胜恐;寒伤血,燥胜寒;咸伤血,甘胜咸。  
故曰:天地者,万物之上下也;阴阳者,血气之男女也;左右者,阴阳之道路也;水火者,阴阳之征兆也;阴阳者,万物之能始也。故曰:阴在内,阳之守也;阳在外,阴之使也。  
帝曰:法阴阳奈何?岐伯曰:阳胜则身热,腠理闭,喘粗为之俯仰,汗不出而热,齿干以烦冤,腹满死,能冬不能夏。阴胜则身寒,汗出,身常清,数栗而寒,寒则厥,厥则腹满死,能夏不能冬。此阴阳更胜之变,病之形能也。  
帝曰:调此二者奈何?岐伯曰:能知七损八益,则二者可调,不知用此,则早衰之节也。年四十而阴气自半也,起居衰矣;年五十,体重,耳目不聪明矣;年六十,阴萎,气不衰,九窍不利,下虚上实,涕泣俱出矣。故曰:知之则强,不知则老,故同出而名异耳。智者察同,愚者察异。愚者不足,智者有余;有余则而目聪明,身体轻强,老者复壮,壮者益治。是以圣人为无为之事,乐恬淡之能,从欲快志于虚无之守,故寿命无穷,与天地终,此圣人之治身也。  
天不足西北,故西北方阴也,而人右耳目不如左明也;地不满东南,故东南方阳也,而人左手足不如右强也。帝曰:何以然?岐伯曰:东方阳也,阳者其精并于上,并于上,则明而下虚,故使耳目聪明,而手足不便也;西方阴也,阴者其精并于下,并于下,则下盛而上虚,故其耳目不聪明,而手足便也。故俱感于邪,其在上则右甚,在下则左甚,此天地阴阳所不能全也,故邪居之。  
故天有精,地有形;天有八纪,地有五里,故能为万物之父母。清阳上天,浊阴归地,是故天地之动静,神明为之纲纪,故能以生长收藏,终而复始。惟贤人上配天以养头,下象地以养足,中傍人事以养五藏。天气通于肺,地气通于嗌,风气通于肝,雷气通于心,谷气通于脾,雨气通于肾。六经为川,肠胃为海,九窍为水注之气。以天地为之阴阳,阳之汗,以天地之雨名之;阳之气,以天地之疾风名之。暴气象雷,逆气象阳。故治不法天之纪,不用地之理,则灾害至矣。  
故邪风之至,疾如风雨。故善治者治皮毛,其次治肌肤,其次治筋脉,其次治六府,其次治五藏。治五藏者,半死半生也。  
故天之邪气,感则害人五藏;水谷之寒热,感则害于六府;地之湿气,感则害皮肉筋脉。  
故善用针者,从阴引阳,从阳引阴;以右治左,以左治右;以我知彼,以表知里;以观过与不及之理,见微得过,用之不殆。  
善诊者,察色按脉,先别阴阳;审清浊,而知部分;视喘息、听音声,而知所苦;观权衡规矩,而知病所主;按尺寸,观浮沉滑涩,而知病所生。以治无过,以诊则不失矣。  
故曰:病之始起也,可刺而已;其盛,可待衰而已。故因其轻而扬之;因其重而减之;因其衰而彰之。形不足者,温之以气;精不足者,补之以味。其高者,因而越之;其下者,引而竭之;中满者,写之于内;其有邪者,渍形以为汗;其在皮者,汗而发之,其彪悍者,按而收之;其实者,散而写之。审其阴阳,以别柔刚,阳病治阴,阴病治阳;定其血气,各守其乡,血实宜决之,气虚宜掣引之。

黄帝道:阴阳是宇宙间的一般规律,是一切事物的纲纪,万物变化的起源,生长毁灭的根本,有很大道理在乎其中。凡医治疾病,必须求得病情变化的根本,而道理也不外乎阴阳二字。拿自然界变化来比喻,清阳之气聚于上,而成为天,浊阴之气积于下,而成为地。阴是比较静止的,阳是比较躁动的;阳主生成,阴主成长;阳主肃杀,阴主收藏。阳能化生力量,阴能构成形体。寒到极点会生热,热到极点会生寒;寒气能产生浊阴,热气能产生清阳;清阳之气居下而不升,就会发生泄泻之病。浊阴之气居上而不降,就会发生胀满之病。这就是阴阳的正常和反常变化,因此疾病也就有逆证和顺证的分别。
所以大自然的清阳之气上升为天,浊阴之气下降为地。地气蒸发上升为云,天气凝聚下降为雨;雨是地气上升之云转变而成的,云是由天气蒸发水气而成的。人体的变化也是这样,清阳之气出于上窍,浊阴之气出于下窍;清阳发泄于腠理,浊阴内注于五脏;清阳充实与四肢,浊阴内走于六腑。
水分为阴阳,则水属阴,火属阳。人体的功能属阳,饮食物属阴。饮食物可以滋养形体,而形体的生成又须赖气化的功能,功能是由精所产生的,就是精可以化生功能。而精又是由气化而产生的,所以形体的滋养全靠饮食物,饮食物经过生化作用而产生精,再经过气化作用滋养形体。如果饮食不节,反能损伤形体,机能活动太过,亦可以使经气耗伤,精可以产生功能,但功能也可以因为饮食不节而受损伤。
味属于阴,所以趋向下窍,气属于阳,所以趋向上窍。味厚的属纯阴,味薄的属于阴中之阳;气厚的属纯阳,气薄的属于阳中之阴。味厚的有泻下的作用,味薄的有疏通的作用;气薄的能向外发泄,气厚的能助阳生热。阳气太过,能使元气衰弱,阳气正常,能使元气旺盛,因为过度亢奋的阳气,会损害元气,而元气却依赖正常的阳气,所以过度亢盛的阳气,能耗散元气,正常的阳气,能增强元气。凡气味辛甘而有发散功用的,属于阳,气味酸苦而有通泄功用的,属于阴。
人体的阴阳是相对平衡的,如果阴气发生偏胜,则阳气受损而为病,阳气发生了偏胜,则阴气耗损而为病。阳气发生了偏生,则阴气耗损而为病。阳偏胜则表现为热性病症,阴偏胜则表现为寒性病症。寒到极点,会表现热象。寒能伤形体,热能伤气分;气分受伤,可以产生疼痛,形体受伤,可以发生肿胀。所以先痛而后肿的,是气分先伤而后及于形体;先肿而后痛的,是形体先病后及于气分。
风邪太过,则能发生痉挛动摇;热邪太过,则能发生红肿;燥气太过,则能发生干枯;寒气太过,则能发生浮肿;湿气太过,则能发生濡泻。
大自然的变化,有春、夏、秋、冬四时的交替,有木、火、土、金、水五行的变化,因此,产生了寒、暑、燥、湿、风的气候,它影响了自然界的万物,形成了生、长、化、收藏的规律。人有肝、心、脾、肺、肾五脏,五脏之气化生五志,产生了喜、怒、悲、忧、恐五种不同的情志活动。喜怒等情志变化,可以伤气,寒暑外侵,可以伤形。突然大怒,会损伤阴气,突然大喜,会损伤阳气。气逆上行,充满经脉,则神气浮越,离去形体了。所以喜怒不加以节制,寒暑不善于调适,生命就不能牢固。阴极可以转化为阳,阳极可以转化为阴。所以冬季受了寒气的伤害,春天就容易发生温病;春天受了风气的伤害夏季就容易发生飧泄;夏季受了暑气的伤害,秋天就容易发生疟疾;秋季受了湿气的伤害,冬天就容易发生咳嗽。
黄帝问道:我听说上古时代的圣人,讲求人体的形态,分辨内在的脏腑,了解经脉的分布,交会、贯通有六合,各依其经之许循行路线;气穴之处,各有名称;肌肉空隙以及关节,各有其起点;分属部位的或逆或顺,各有条理;与天之四时阴阳,都有经纬纪纲;外面的环境与人体内部相关联,都有表有里。这些说法都正确吗?
岐伯回答说:东方应春,阳生而日暖风和,草木生发,木气能生酸味,酸味能滋养肝气,肝气又能滋养于筋,筋膜柔和则又能生养于心,肝气关联于目。它在自然界是深远微妙而无穷的,在人能够知道自然界变化的道理,在地为生化万物。大地有生化,所以能产生一切生物;人能知道自然界变化的道理,就能产生一切智慧;宇宙间的深远微妙,是变化莫测的。变化在天空中为风气,在地面上为木气,在人体为筋,在五脏为肝,在五色为苍,在五音为角,在五声为呼,在病变的表现为握,在七窍为目,在五味为酸,在情志的变动为怒。怒气能伤肝,悲能够抑制怒;风气能伤筋,燥能够抑制风;过食酸味能伤筋,辛味能抑制酸味。
南方应夏,阳气盛而生热,热甚则生火,火气能产生苦味,苦味能滋长心气,心气能化生血气,血气充足,则又能生脾,心气关联于舌。它的变化在天为热气,在地为火气,在人体为血脉,在五脏为心,在五色为赤,在五音为徵,在五声为笑,在病变的表现为忧,在窍为舌,在五味为苦,在情志的变动为喜。喜能伤心,以恐惧抑制喜;热能伤气,以寒气抑制热;苦能伤气,咸味能抑制苦味。
中央应长夏,长夏生湿,湿与土气相应,土气能产生甘味,甘味能滋养脾气,脾气能滋养肌肉,肌肉丰满,则又能养肺,脾气关联于口。它的变化在天为湿气,在地为土气,在人体为肌肉,在五脏为脾,在五色为黄,在五音为宫,在五声为歌,在病变的表现为哕,在窍为口,在五味为甘,在情志的变动为思。思虑伤脾,以怒气抑制思虑;湿气能伤肌肉,以风气抑制湿气,甘味能伤肌肉,酸味能抑制甘味。
西方应秋,秋天天气急而生燥,燥与金气相应,金能产生辛味,辛味能滋养肺气,肺气能滋养皮毛,皮毛润泽则又能养肾,肺气关联于鼻。它的变化在天为燥气,在地为金气,在人体为皮毛,在五脏为肺,在五色为白,在五音为商,在五声为哭,在病变的表现为咳,在窍为鼻,在五味为辛,在情志的变动为忧。忧能伤肺,以喜抑制忧;热能伤皮毛,寒能抑制热;辛味能伤皮毛,苦味能抑制辛味。
北方应冬,冬天生寒,寒气与水气相应,水气能产生咸味,咸味能滋养肾气,肾气能滋长骨髓,骨髓充实,则又能养肝,肾气关联于耳。它的变化在天为寒气,在地为水气,在人体为骨髓,在五脏为肾,在五色为黑,在五音为羽,在五声为呻,在病变的表现为战栗,在窍为耳,在五味为咸,在情志的变动为恐。恐能伤肾,思能够抑制恐;寒能伤血,燥(湿)能够抑制寒;咸能伤血,甘味能抑制咸味。
所以说:天地是在万物的上下;阴阳如血气与男女之相对待;左右为阴阳运行不息的道路;水性寒,火性热,是阴阳的象征;阴阳的变化,是万物生长的原始能力。所以说:阴阳是互相为用的,阴在内,为阳之镇守;阳在外,为阴之役使。
黄帝道:阴阳的法则怎样运用于医学上呢?岐伯回答说:如阳气太过,则身体发热,腠理紧闭,气粗喘促,呼吸困难,身体亦为之俯仰摆动,无汗发热,牙齿干燥,烦闷,如见腹部帐满,是死症,这是属于阳性之病,所以冬天尚能支持,夏天就不能耐受了。阴气胜则身发寒而汗多,或身体常觉冷而不时战栗发寒,甚至手足厥逆,如见手足厥逆而腹部胀满的,是死症,这是属于阴胜的病,所以夏天尚能支持,冬天就不能耐受了。这就是阴阳互相胜负变化所表现的病态。
黄帝问道:调摄阴阳的办法怎样?岐伯说:如果懂得了七损八益的养生之道,则人身的阴阳就可以调摄,如其不懂得这些道理,就会发生早衰现象。一般的人,年到四十,阴气已经自然的衰减一半了,其起居动作,亦渐渐衰退;到了五十岁,身体觉得沉重,耳目也不够聪明了;到了六十岁,阴气萎弱,肾气大衰,九窍不能通利,出现下虚上实的现象,会常常流着眼泪鼻涕。所以说:知道调摄的人身体就强健,不知到调摄的人身体就容易衰老;本来是同样的身体,结果却出现了强弱不同的两种情况。懂得养生之道的人,能够注意共有的健康本能;不懂得养生之道的人,只知道强弱的异形。不善于调摄的人,常感不足,而重视调摄的人,就常能有余;有余则耳目聪明,身体轻强,即使已经年老,亦可以身体强壮,当然本来强壮的就更好了。所以圣人不作勉强的事情,不胡思乱想,有乐观愉快的旨趣,常使心旷神怡,保持着宁静的生活,所以能够寿命无穷,尽享天年。这是圣人保养身体的方法。
天气是不足与西北方的,所以西北方属阴,而人的右耳也不及左边的聪明;地气是不足于东南方的,所以东南方属阳,而人的左手足也不及右边的强。黄帝问道,这是什麽道理?岐伯说:东方属阳,阳性向上,所以人体的精神集合于上部,集合于上部则上部强盛而下部虚弱,所以使耳目聪明,而手足不便利;西方属阴,阴性向下,所以人体的精气集合于下部,集合于下部则下部强盛而上部虚弱,所以耳目不聪明而手足便利。如虽左右同样感受了外邪,但在上部则身体的右侧较重,在下部则身体的左侧较重,这是天地阴阳之所不能全,而人身亦有阴阳左右之不同,所以邪气就能乘虚而居留了。
所以天有精气,地有形体;天有八节之纲纪,地有五方的道理,因此天地是万物生长的根本。无形的清阳上生于天,有形的浊阴下归于地,所以天地的运动与静止,是由阴阳的神妙变化为纲纪,而能使万物春生、夏长、秋收、冬藏,终而复始,循环不休。懂得这些道理的人,他把人体上部的头来比天,下部的足来比地,中部的五脏来比人事以调养身体。天的轻清之气通于肺,地的水谷之气通于嗌,风木之气通于肝,雷火之气通于心,溪谷之气通于脾,雨水之气通于肾。六经犹如河流,肠胃犹如大海,上下九窍以水津之气贯注。如以天地来比类人体的阴阳,则阳气发泄的汗,象天的下雨;人身的阳气,象天地疾风。人的暴怒之气,像天有雷霆;逆上之气,象阳热的火。所以调养身体而不取法于自然的道理,那麽疾病就要发生了。
所以外感致病因素伤害人体,急如疾风暴雨。善于治病的医生,于邪在皮毛的时候,就给予治疗;技术较差的,至邪在肌肤才治疗;又更差的,至邪在五脏才治疗。假如病邪传入到五脏,就非常严重,这时治疗的效果,只有半死半生了。
所以自然界中的邪气,侵袭了人体就能伤害五脏;饮食之或寒或热,就会损害人的六腑;地之湿气,感受了就能损害皮肉筋脉。
所以善于运针法的,病在阳,从阴以诱导之,病在阴,从阳以诱导之;取右边以治疗左边的病,取左边以治疗右边的病,以自己的正常状态来比较病人的异常状态,以在表的症状,了解里面的病变;并且判断太过或不及,就能在疾病初起的时候,便知道病邪之所在,此时进行治疗,不致使病情发展到危险的地步了。
所以善于诊治的医生,通过诊察病人的色泽和脉搏,先辨别病症的属阴属阳;审察五色的浮泽或重浊,而知道病的部位;观察呼吸,听病人发出的声音,可以得知所患的病苦;诊察四时色脉的正常是否,来分析为何脏何腑的病,诊察寸口的脉,从它的浮、沉、滑、涩,来了解疾病所产生之原因。这样在诊断上就不会有差错,治疗也没有过失了。
所以说:病在初起的时候,可用刺法而愈;及其病势正盛,必须待其稍微衰退,然后刺之而愈。所以病轻的,使用发散轻扬之法治之;病重的,使用消减之法治之;其气血衰弱的,应用补益之法治之。形体虚弱的,当以温补其气;精气不足的,当补之以厚味。如病在上的,可用吐法;病在下的,可用疏导之法;病在中为胀满的,可用泻下之法;其邪在外表,可用汤药浸渍以使出汗;邪在皮肤,可用发汗,使其外泄;病势急暴的,可用按得其状,以制伏之;实症,则用散法或泻法。观察病的在阴在阳,以辨别其刚柔,阳病应当治阴,阴病应当治阳;确定病邪在气在血,更防其血病再伤及气,气病再伤及血,所以血适宜用泻血法,气虚宜用导引法。

\chapter{阴阳离合论篇 第六}
黄帝问曰:余闻天为阳,地为阴,日为阳,月为阴,大小月三百六十日成一岁,人亦应之。今三阴三阳,不应阴阳,其故何也?岐伯对曰:阴阳者,数之可十,推之可百;数之可千,推之可万;万之大,不可胜数,然其要一也。天覆地载,万物方生,未出地者,命曰阴处,名曰阴中之阴;则出地者,命曰阴中之阳。阳予之下,阴为之主;故生因春,长因夏,收因秋,藏因冬。失常则天地四塞。阴阳之变,其在人者,亦数之可数。
帝曰:愿闻三阴三阳之离合也。岐伯曰:圣人南面而立,前曰广明,后曰太冲,太冲之地,名曰少阴,少阴之上,名曰太阳,太阳根起于至阴,结于命门,名曰阴中之阳。中身而上,名曰广明,广明之下,名曰太阴,太阴之前,名曰阳明,阳明根起于厉兑,名曰阴中之阳。厥阴之表,名曰少阳,少阳根起于窍阴,名曰阴中之少阳。是故三之离合也,太阳为开,阳明为阖,少阳为枢。三经者,不得相失也,搏而勿浮,命曰一阳。
帝曰:愿闻三阴。岐伯曰:外者为阳,内者为阴,然则中为阴,其冲在下,名曰太阴,太阴根起于隐白,名曰阴中之阴。太阴之后,名曰少阴,少阴根起于涌泉,名曰阴中之少阴。少阴之前,名曰厥阴,厥阴根起于大敦,阴之绝阳,名曰阴之绝阴。是故三阴之离合也,太阴为开,厥阴为阖,少阴为枢。三经者,不得相失也,搏而勿沉,名曰一阴。
阴阳雩重,重传为一周,气里形表而为相成也。
阴阳离合论篇第六参考译文
黄帝问道:我听说天属阳,地属阴,日属阳,月属阴,大月和小月合起来三百六十天而成为一年,人体也与此相应。如今听说人体的三阴三阳,和天地阴阳之数不相符合,这是什麽道理?岐伯回答说:天地阴阳的范围,极其广泛,在具体运用时,经过进一步推演,则可以由十到百,由百到千,由千到万,再演绎下去,甚至是数不尽的,然而其总的原则仍不外乎对立统一的阴阳道理。天地之间,万物初生,未长出地面的时候,叫做居于阴处,称之为阴中之阴;若已长出地面的,就叫做阴中之阳。有阳气,万物才能生长,有阴气,万物才能成形。所以万物的发生,因于春气的温暖,万物的盛长,因于夏气的炎热,万物的收成,因于秋气的清凉,万物的闭藏,因于冬气的寒冷。如果四时阴阳失序,气候无常,天地间的生长收藏的变化就要失去正常。这种阴阳变化的道理,在人来说,也是有一定的规律,并且可以推测而知的。
黄帝说:我愿意听你讲讲三阴三阳的离合情况。岐伯说:圣人面向南方站立,前方名叫广明,后方名叫太冲,行于太冲部位的经脉,叫做少阴。在少阴经上面的经脉,名叫太阳,太阳经的下端起于足小趾外侧的至阴穴,其上端结于晴明穴,因太阳为少阴之表,故称为阴中之阳。再以人身上下而言,上半身属于阳,称为广明,广明之下称为太阴,太阴前面的经脉,名叫阳明,阳明经的下端起于足大趾侧次趾之端的历兑穴,因阴阳是太阴之表,故称为阴中之阳。厥阴为里,少阳为表,故厥阴精之表,为少阳经,少阳经下端起于窍阴穴,因少阳居厥阴之表,故称为阴中之少阳。因此,三阳经的离合,分开来说,太阳主表为开,阴明主里为阖,少阳介于表里之间为枢。但三者之间,不是各自为政,而是相互紧密联系着的,所以合起来称为一阳。
黄帝说:愿意再听你讲讲三阴的离合情况。岐伯说:在外的为阳,在内的为阴,所以在里的经脉称为阴经,行于少阴前面的称为太阴,太阴经的根起于足大趾之端的隐白穴,称为阴中之阴。太阴的后面,称为少阴,少阴经的根起于足心的涌泉穴,称为阴中之少阴。少阴的前面,称为厥阴,厥隐经的根起于足大趾之端的大敦穴,由于两阴相合而无阳,厥阴又位于最里,所以称之为阴之绝阴。因此,三阴经之离合,分开来说,太阴为三阴之表为开,厥阴为主阴之里为阖,少阴位于太、厥表里之间为枢。但三者之间,不能各自为政,而是相互协调紧密联系着的,所以合起来称为一阴。
阴阳之气,运行不息,递相传注于全身,气运于里,形立于表,这就是阴阳离合、表里相成的缘故。
\chapter{阴阳别论篇 第七}
黄帝问曰:人有四经十二从,何谓?岐伯对曰:四经应四时,十二从应十二月,十二月应十二脉。脉有阴阳,知阳者知阴,知阴者知阳。凡阳有五,五五二十五阳。所谓阴者,真藏也,见则为败,败必死也;所谓阳者,胃脘之阳也。别于阳者,知病处也;别于阴者,知死生之期。三阳在头,三阴在手,所谓一也。别于阳者,知病忌时;别于阴者,知死生之期。谨熟阴阳,无与众谋。
所谓阴阳者,去者为阴,至者为阳;静者为阴,动者为阳;迟者为阴,数者为阳。
凡持真脉之藏脉者,肝至悬绝急,十八日死;心至悬绝,九日死;肺至悬绝,十二日死;肾至悬绝,七日死;脾至悬绝,四日死。
曰:二阳之病发心脾,有不得隐曲,女子不月;其传为风消,其传为息贲者,死不治。
曰:三阳为病,发寒热,下为痈肿,及为痿厥;其传为索泽,其传为息贲者,其传为颓疝。
曰:一日发病,少气,善咳,善泄。其传为心掣,其传为隔。
二阳一阴发病,主惊骇,背痛,善噫,善欠,名曰风厥。二阴一阳发病,善胀,心满善气。三阳三阴发病,为偏枯痿易,四支不举。
鼓一阳曰钩,鼓一阴曰毛,鼓阳胜急曰弦,鼓阳至而绝曰石,阴阳相过曰溜。
阴争于内,阳扰于外,魄汗未藏,四逆而起,起则熏肺,使人喘鸣。阴之所生,和本曰和。是故刚与刚,阳气破散,阴气乃消亡;淖则刚柔不和,经气乃绝。
死阴之属,不过三日而死;生阳之属,不过四日而死。所谓生阳、死阴者,肝之心谓之生阳,心之肺谓之死阴,肺之肾谓之重阴,肾之脾谓之辟阴,死不治。
结阳者,肿四支;结阴者,便血一升,再结二升,三结三升;阴阳结斜,多阴少阳曰石水,少腹钟。二阳结谓之消,三阳结为之隔,三阴结谓之水,一阴一阳结谓之喉痹。
阴搏阳别,谓之有子;阴阳虚,肠辟死;阳加于阴谓之汗;阴虚阳搏谓之崩。
三阴俱搏,二十日夜半死;二阴俱搏,十三日夕时死;一阴俱搏,十日死;三阳俱搏且鼓,三日死;三阴三阳俱搏,心腹满,发尽,不得隐曲,五日死;二阳俱搏,其病温,死不台,不过十日死。
阴阳别论篇第七参考译文
黄帝问道:人有四经十二从,这是什麽意思?岐伯回答说:四经,是指与四时相应的正常脉象,十二从,是指与十二个月相应的十二经脉。
脉有阴有阳,能了解什麽是阳脉,就能知道什么是阴脉,能了解什么是阴脉,就能知道什么是阳脉。阳脉有五种,就是春微弦,夏微钩,长夏微缓,秋微毛,冬微石。五时各有五脏的阳脉,所以五时配合五脏,则为二十五种阳脉。所谓阴脉,就是脉没有胃气,称为真脏脉象。真脏脉是胃气已经败坏的象征,败象已见,就可以断其必死。所谓阳脉,就是指有胃气之脉。辨别阳脉的情况,就可以知道病变的所在;辨别真脏脉的情况,就可以知道死亡的时期。三阳经脉的诊察部位,在结喉两旁的人迎穴,三阴经脉的诊察部位,在手鱼际之后的寸口。一般在健康状态之下,人迎与寸口的脉象是一致的。辨别属阳的胃脉,能知道时令气候和疾病的宜忌;辨别属阴的真脏脉,能知道病人的死生时期。临证时应谨慎而熟练地辨别阴脉与阳脉,就不致疑惑不绝而众议纷纭了。
凡诊得无胃气的真藏脉,例如:肝脉来的形象,如一线孤悬,似断似绝,或者来得弦急而硬,十八日当死;心脉来时,孤悬断绝,九日当死;脉脉来时,孤悬断绝,十二日当死;肾脉来时,孤悬断绝,七日当死;脾脉来时,孤悬断绝,四日当死。
一般地说:胃肠有病,则可影响心脾,病人往往有难以告人的隐情,如果是女子就会月经不调,甚至经闭。若病久传变,或者形体逐渐消瘦,成为“风消”,或者呼吸短促,气息上逆,成为“息贲”,就不可治疗了。
一般地说:太阳经发病,多有寒热的症状,或者下部发生痈肿,或者两足痿弱无力而逆冷,腿肚酸痛。若病久传化,或为皮肤干燥而不润泽,或变为颓疝。
一般的说:少阳经发病,生发之气即减少,或易患咳嗽,或易患泄泻。若病久传变,或为心虚掣痛,或为饮食不下,阻塞不通。
阳明与厥隐发病,主病惊骇,背痛,常常嗳气、呵欠,名曰风厥。少阴和少阳发病,腹部作胀,心下满闷,时欲叹气。太阳和太阴发病,则为半身不遂的偏枯症,或者变易常用而痿弱无力,或者四肢不能举动。
脉搏鼓动于指下,来时有力,去时力衰,叫做钩脉;稍无力,来势轻虚而浮,叫做毛脉;有力而紧张,如按琴瑟的弦,叫做弦脉;有力而必须重按,轻按不足,叫做石脉;既非无力,又不过于有力,一来一去,脉象和缓,流通平顺,叫做滑脉。
阴阳失去平衡,以致阴气争盛于内,阳气扰乱于外,汗出不止,四肢厥冷,下厥上逆,浮阳熏肺,发生喘鸣。
阴之所以不能生化,由于阴阳的平衡,是谓正常。如果以刚与刚,则阳气破散,阴气亦必随之消亡;倘若阴气独盛,则寒湿偏胜,亦为刚柔不和,经脉气血亦致败绝。
属于死阴的病,不过三日就要死;属于生阳的病,不过四天就会痊愈。所谓生阳、死阴:例如肝病传心,为木生火,得其生气,叫做生阳;心病传肺,为火克金,金被火消亡,叫做死阴,肺病传肾,以饮传阴,无阳之候,叫做重阴;肾病传脾,水反侮土,叫做辟阴,是不治的死症。
邪气郁结于阳经,则四肢浮肿,以四肢为诸阳之本;邪气郁结于阴经,则大便下血,以阴络伤则血下溢,初结一升,再结二升,三结三升;阴经阳经都有邪气郁结,而偏重于阴经方面的,就会发生“石水”之病,少腹肿胀;邪气郁结于二阳(足阳明胃、手阳明大肠),则肠胃俱热,多为消渴之症;邪气郁结于三阳(足太阳膀胱、手太阳小肠),则多为上下不通的隔症;邪气郁结于三阴(足太阴脾、手太阴肺),多为水肿膨胀的病;邪气郁结于一阴一阳(指厥阴和少阳)多为喉痹之病。
阴脉搏动有力,与阳脉有明显的区别,这是怀孕的现象;阴阳脉(尺脉、寸脉)具虚而患痢疾的,是为死症;阳脉加倍于阴脉,当有汗出,阴脉虚而阳脉搏击,火迫血行,在妇人为血崩。
三阴(指手太阴肺、足太阴脾)之脉,俱搏击于指下,大约到二十天半夜时死亡;二阴就(指手少阴心、足少阴肾)之脉俱搏击于指下,大约到十三天傍晚时死亡;一阴(指手厥阴心胞络、足厥阴肝)之脉俱搏击于指下,而鼓动过甚的,三天就要死亡;三阴三阳之脉俱搏,心腹胀满,阴阳之气发泄已尽,大小便不通,则五日死;三阳(指足阳明胃、手阳明大肠)之脉俱搏击于指下,患有温病的,无法治疗,不过十日就要死了。
\chapter{灵兰秘典论篇 第八}

黄帝问曰:愿闻十二脏之相使,贵贱何如?
岐伯对曰:悉乎哉问也!请遂言之。心者,君主之官也,神明出焉。肺者,相傅之官,治节出焉。肝者,将军之官,谋虑出焉。胆者,中正之官,决断出焉。膻中者,臣使之官,喜乐出焉。脾胃者,仓廪之官,五味出焉。大肠者,传道之官,变化出焉。小肠者,受盛之官,化物出焉。肾者,作强之官,伎巧出焉。三焦者,决渎之官,水道出焉。膀胱者,州都之官,津液藏焉,气化则能出矣。凡此十二官者,不得相失也,故主明则下安,以此养生则寿,殁世不殆,以为天下则大昌;主不明则十二官危,使道闭塞而不通,形乃大伤,以此养生则殃,以为天下者,其宗大危。戒之戒之!
至道在微,变化无穷,孰知其原?窘乎哉!消者瞿瞿,孰知其要?闵闵之当,孰者为良?恍惚之数,生于毫厘,起于度量,千之万之,可以益大,推之大之,其形乃制。
黄帝曰:善哉!余闻精光之道,大圣之业,而宣明大道。非斋戒择吉日,不敢受也。黄帝乃择吉日良兆,而藏灵兰之室,以传保焉。
灵兰秘典论篇第八参考译文
黄帝问道:我想听你谈一下人体六脏六腑这十二个器官的责任分工,高低贵贱是怎样的呢?岐伯回答说:你问的真详细呀!请让我谈谈这个问题。心,主宰全身,是君主之官,人的精神意识思维活动都由此而出。肺,是相傅之官,犹如相傅辅佐着君主,因主一身之气而调节全身的活动。肝,主怒,像将军一样的勇武,称为将军之官,谋略由此而出。膻中,维护着心而接受其命令,是臣使之官,心志的喜乐,靠它传佈出来。脾和胃司饮食的受纳和布化,是仓廪之官,五味的营养靠它们的作用而得以消化、吸收和运输。大肠是传导之官,它能传送食物的糟粕,使其变化为粪便排除体外。小肠是受盛之官,它承受胃中下行的食物而进一步分化清浊。肾,是作强之官,它能够使人发挥强力而产生各种伎巧。三焦,是决渎之官,它能够通行水道。膀胱是州都之官,蓄藏津液,通过气化作用,方能排除尿液。以上这十二官,虽有分工,但其作用应该协调而不能相互脱节。所以君主如果明智顺达,则下属也会安定正常,用这样的道理来养生,就可以使人长寿,终生不会发生危殆,用来治理天下,就会使国家昌盛繁荣。君主如果不明智顺达,那麽,包括其本身在内的十二官就都要发生危险,各器官发挥正常作用的途径闭塞不通,形体就要受到严重伤害。在这种情况下,谈养生续命是不可能的,只会招致灾殃,缩短寿命。同样,以君主之昏聩不明来治理天下,那政权就危险难保了,千万要警惕再警惕呀!
至深的道理是微渺难测的,其变化也没有穷尽,谁能清楚地知道它的本源!实在是困难得很呀!有学问的人勤勤恳恳地探讨研究,可是谁能知道它的要妙之处!那些道理暗昧难明,就象被遮蔽着,怎能了解到它的精华是什麽!那似有若无的数量,是产生于毫蹻的微小数目,而毫蹻也是起于更小的度量,只不过把它们千万倍地积累扩大,推衍增益,才演变成了形形色色的世界。黄帝说:好啊!我听到了精纯明彻的道理,这真是大圣人建立事业的基础,对于这宣畅明白的宏大理论,如果不专心修省而选择吉祥的日子,把这些著作珍藏在灵台兰室,很快地保存起来,以便流传后世。

\chapter{六节藏象论篇 第九}

黄帝问曰:余闻天以六六之节,以成一岁;人以九九制会,计人亦有三百六十五节,以为天地,久矣,不知其所谓也?
岐伯对曰:昭乎哉问也!请遂言之。夫六六之节、九九制会者,所以正天之度、气之数也。天度者,所以制日月之行也;气数者,所以纪化生之用也。天为阳,地为阴,日为阳,月为阴,行有分纪,周有道理,日行一度,月行十三度而有奇焉,故大小月三百六十五日而成岁,积气余而盈闰矣。立端于始,表正于中,推余于终,而天度毕矣。
帝曰:余已闻天度矣,愿闻气数何以合之?
岐伯曰:天以六六为节,地以九九制会;天有十日,日六竟而周甲,甲六复而终岁,三百六十日法也。夫自古通天者,生之本,本于阴阳。其气九州、九窍,皆通乎天气,故其生五,其气三,三而成天,三而成地,三而成人,三而三之,合则为九,九分为九野,九野为九脏,故形脏四,神脏五,合为九脏以应之也。
帝曰:余已闻六六九九之会也,夫子言积气盈闰,愿闻何谓气?请夫子发蒙解惑焉。
岐伯曰:此上帝所秘,先师传之也。
帝曰:请遂闻之。
岐伯曰:五日谓之候,三候谓之气,六气谓之时,四时谓之岁,而各从其主治焉。五运相袭,而皆治之,终期之日,周而复始;时立气布,如环无端,候亦同法。故曰:不知年之所加,气之盛衰,虚实之所起,不可以为工矣。
帝曰:五运之始,如环无端,其太过不及何如?
岐伯曰:五气更立,各有所胜,盛虚之变,此其常也。
帝曰:平气何如?
岐伯曰:无过者也。
帝曰:太过不及奈何?岐伯曰:在经有也。
帝曰:何谓所胜?
岐伯曰:春胜长夏,长夏胜冬,冬胜夏,夏胜秋,秋胜春,所谓得五行时之胜,各以气命其脏。帝曰:何以知其胜?
岐伯曰:求其至也,皆归始春。未至而至,此谓太过,则薄所不胜,而乘所胜也,命曰气淫,不分邪僻内生,工不能禁;至而不至,此谓不及,则所胜妄行,而所生受病,所不胜薄之也,命曰气迫。所谓求其至者,气至之时也。谨候其时,气可与期,失时反候,五治不分,邪僻内生,工不能禁也。
帝曰:有不袭乎?
岐伯曰:苍天之气,不得无常也。气之不袭,是谓非常,非常则变矣。
帝曰:非常而变奈何?
岐伯曰:变至则病,所胜则微,所不胜则甚,因而重感于邪则死矣,故非其时则微,当其时则甚也。
帝曰:善。余闻气合而有形,因变以正名。天地之运,阴阳之化,其于万物,孰少孰多,可得闻乎?
岐伯曰:悉乎哉问也!天至广不可度,地至大不可量,大神灵问,请陈其方。草生五色,五色之变,不可胜视;草生五味,五味之美,不可胜极。嗜欲不同,各有所通。天食人以五气,地食人以五味。五气入鼻,藏于心肺,上使五色修明,音声能彰;五味入口,藏于肠胃,味有所藏,以养五气,气和而生,津液相成,神乃自生。
帝曰:脏象何如?
岐伯曰:心者,生之本神之变也;其华在面,其充在血脉,为阳中之太阳,通于夏气。肺者,气之本,魄之处也;其华在毛,其充在皮,为阳中之太阴,通于秋气。肾者,主蛰,封藏之本,精之处也;其华在发,其充在骨,为阴中之少阴,通于冬气。肝者,罢极之本,魂之居也;其华在爪,其充在筋,以生血气,其味酸,其色苍,此为阳中之少阳,通于春气。脾、胃、大肠、小肠、三焦、膀胱者,仓廪之本,营之居也,名曰器,能化糟粕,转味而入出者也;其华在唇四白,其充在肌,其味甘,其色黄,此至阴之类,通于土气。凡十一藏,取决于胆也。
故人迎一盛,病在少阳,二盛病在太阳,三盛病在阳明,四盛已上为格阳。寸口一盛,病在厥阴,二盛病在少阴,三盛病在太阴,四盛以上为关阴。人迎与寸口俱盛四倍已上为关格,关格之脉赢,不能极于天地之精气,则死矣。
六节脏象论篇第九参考译文
黄帝问道:我听说天体的运行是以六个甲子构成一年,人则以九九极数的变化来配合天道的准度,而人又有三百六十五穴,与天地相应,这些说法,已听到很久了,但不知是什麽道理?岐伯答到:你提的问题很高明啊!请让我就此问题谈谈看法。六六之节和九九制会,是用来确定天度和气数的。天度,是计算日月行程的。气数,是标志万物化生之用的。天属阳,地属阴,日属阳,月属阴。它们的运行有一定的部位和秩序,其环周也有一定的道路。每一昼夜,日行一度,月行十三度有余,所以大月、小月和起来三百六十五天成为一年,由于月份的不足,节气有盈余,于是产生了闰月。确定了岁首冬至并以此为开始,用圭表的日影以推正中气的时间,随着日月的运行而推算节气的盈余,直到岁尾,整个天度的变化就可以完全计算出来了。
黄帝说:我已经明白了天度,还想知道气数是怎样与天度配合的?岐伯说:天以六六为节制,地以九九之数,配合天道的准度,天有十干,代表十日,十干循环六次而成一个周甲,周甲重复六次而一年终了,这是三百六十日的计算方法。自古以来,都以通于天气而为生命的根本,而这个根本不外天之阴阳。地的九州,人的九窍,都与天气相通,天衍生五行,而阴阳又依盛衰消长而各分为三。三气合而成天,三气合而成地,三气合而成人,三三而合成九气,在地分为九野,在人体分为九脏,形脏四,神脏五,合成九脏,以应天气。
黄帝说:我已经明白了六六九九配合的道理,先生说气的盈余积累成为闰月,我想听您讲一下是什麽气?请您来启发我的蒙昧,解释我的疑惑!岐伯说:这是上帝秘而不宣的理论,先师传授给我的。黄帝说:就请全部讲给我听。岐伯说:五日称为候,三候称为气,六气称为时,四时称为岁,一年四时,各随其五行的配合而分别当旺。木、火、土、金、水五行随时间的变化而递相承袭,各有当旺之时,到一年终结时,再从头开始循环。一年分力四时,四时分布节气,逐步推移,如环无端,节气中再分候,也是这样的推移下去。所以说,不知当年客气加临、气的盛衰、虚实的起因等情况,就不能做个好医生。
黄帝说:五行的推移,周而复始,如环无端,它的太过与不及是怎样的呢?岐伯说:五行之气更迭主时,互有胜克,从而有盛衰的变化,这是正常的现象。黄帝说:平气是怎样的呢?岐伯说:这是没有太过和不及。黄帝说:太过和不及的情况怎样呢?岐伯说:这些情况在经书中已有记载。
黄帝说:什麽叫做所胜?岐伯说:春胜长夏,长夏胜冬,冬胜夏,夏胜秋,秋胜春,这就是时令根据五行规律而互相胜负的情况。同时,时令又依其五行之气的属性来分别影响各脏。黄帝说:怎样知道它们之间的相胜情况呢?岐伯说:首先要推求气候到来的时间,一般从立春开始向下推算。如果时令未到而气候先期来过,称为太过,某气太过就会侵侮其所不胜之气,欺凌其所胜之气,这就叫做气淫;时令以到而气候未到,称为不及,某气不及,则其所胜之气因缺乏制约而妄行,其所生之气因缺乏资助而困弱,其所不胜则更会加以侵迫,这就叫做气迫。所谓求其至,就是要根据时令推求气候到来的早晚,要谨慎地等候时令的变化,气候的到来是可以预期的。如果搞错了时令或违反了时令与气候相合的关系,以致于分不出五行之气当旺的时间,那麽,当邪气内扰,病及于人的时候,好的医生也不能控制了。
黄帝说:五行之气有不相承袭的吗?岐伯说:天的五行之气,在四时中的分布不能没有常规。如果五行之气不按规律依次相承,就是反常的现象,反常就会使人发生病变,如在某一时令出现的反常气候,为当旺之气之所胜者,则其病轻微,若为当旺之气之所不胜者,则其病深重,而若同时感受其他邪气,就会造成死亡。所以反常气候的出现,不在其所克制的某气当旺之时令,病就轻微,若恰在其所克制的某气当旺之时令发病,则病深重。
黄帝说:好。我听说由于天地之气的和合而有万物的形体,又由于其变化多端以至万物形态差异而定有不同的名称。天地的气运,阴阳的变化,它们对于万物的生成,就其作用而言,哪个多,哪个少,可以听你讲一讲吗?岐伯说:问的实在详细呀!天及其广阔,不可测度,地极其博大,也很难计量,像您这样伟大神灵的圣主既然发问,就请让我陈述一下其中的道理吧。草木显现五色,而五色的变化,是看也看不尽的;草木产生五味,而五味的醇美,是尝也尝不完的。人们对色味的嗜欲不同,而各色味是分别与五脏相通的。天供给人们以五气,地供给人们以五味。五气由鼻吸入,贮藏于心肺,其气上升,使面部五色明润,声音洪亮。五味入于口中,贮藏于肠胃,经消化吸收,五味精微内注五脏以养五脏之气,脏气和谐而保有生化机能,津液随之生成,神气也就在此基础上自然产生了。
黄帝说:脏象是怎样的呢?岐伯说:心,是生命的根本,为神所居之处,其荣华表现于面部,其充养的组织在血脉,为阳中的太阳,与夏气相通。肺是气的根本,为魄所居之处,其荣华表现在毫毛,其充养的组织在皮肤,是阳中的太阴,与秋气相通。肾主蛰伏,是封藏经气的根本,为精所居之处,其荣华表现在头发,其充养的组织在骨,为阴中之少阴,与冬气相通。肝,是罢极之本,为魄所居之处,其荣华表现在爪甲,其充养的组织在筋,可以生养血气,其味酸,其色苍青,为阳中之少阳,与春气相通。脾、胃、大肠、小肠、三焦、膀胱,是仓廪之本,为营气所居之处,因其功能象是盛贮食物的器皿,故称为器,它们能吸收水谷精微,化生为糟粕,管理饮食五味的转化、吸收和排泄,其荣华在口唇四旁的白肉,其充养的组织在肌肉,其味甘,其色黄,属于至阴之类,与土气相通。以上十一脏功能的发挥,都取决于胆气的升发。
人迎脉大于平时一倍,病在少阳;大两倍,病在太阳;大三倍,病在阳明;大四倍以上,为阳气太过,阴无以通,是为格阳。寸口脉大于平时一倍,病在厥阴;大两倍,病在少阴;大三倍,病在太阴;大四倍以上,为阴气太过,阳无以交,是为关阴。若人迎脉与寸口脉俱大与常时四倍以上,为阴阳气俱盛,不得相荣,是为关格。关格之脉盈盛太过,标志着阴阳极亢,不再能够达于天地阴阳经气平调的胜利状态,会很快死去。

\chapter{五藏生成篇第十}

心之和、脉也,其荣、色也,其主肾也。肺之合、筋也,其荣、爪也,其主肺也。脾之合、肉也,其荣、唇也,其主肝也。肾之合、骨也,其荣、发也,其主脾也。
是故多食咸,则脉凝泣而变色;多食苦,则皮槁而毛拔;多食辛,则筋急而爪枯;多食酸,则肉胝×而唇揭;多食甘,则骨痛而发落。此五味之所伤也。故心欲苦,肺欲辛,肝欲酸,脾欲甘,肾欲咸。此五味之所合也。
五藏之气:故色见青如草兹者死,黄如枳实者死,黑如台者死,赤如血者死,白如枯骨者死,此五色之见死也;青如翠羽者生,赤如鸡冠者生,黄如蟹腹者生,白如豕膏者生,黑如乌羽者生,此五色之见生也。生于心,如以稿裹朱;生于肺,如以缟裹红;生于肝,如以缟裹甘;生于脾,如以缟裹栝楼实;生于肾,如以缟裹紫。此五藏所生之外荣也。
色味当五藏:白当肺、辛,赤当心、苦,青当肝、酸,黄当脾、甘,黑当肾、咸。故白当皮,赤当脉,青当筋,黄当肉,黑当骨。
诸脉者,皆属于目;诸髓者,皆属于脑,诸筋者,皆属于节;诸血者,皆属于心;诸气者,皆属于肺。此四支八溪之朝夕也。故人卧血归于肝,肝受血而能视,足受血而能步,掌受血而能握,指受血而能摄。卧出而风吹之,血凝于肤者为痹,凝于脉者为泣,凝于足者为厥,此三者,血行而不得反其空,故为痹厥也。人有大谷十二分,小溪三百五十四名,少十二俞,此皆卫气之所留止,邪气之所客也,针石缘而去之。
诊病之始,五决为纪,欲知其始,先建其母。所谓五决者,五脉也。
是以头痛巅,疾,下虚上实,过在足少阴、巨阳,甚则入肾。徇蒙招尤,目冥耳聋,下实上虚,过在足少阳、厥阳,甚则入肝,腹满月真胀,支鬲月去胁,下厥上冒,过在足太阴、阳明。咳嗽上气,厥在胸中,过在手阳明、太阴。心烦头痛,病在鬲中,过在手巨阳、少阴。
夫脉之小、大、滑、涩、浮、沉,可以指别;五藏之象,可以类推;五藏相音可以意识;五色微诊,可以目察。能合脉色,可以万全。
赤脉之至也,喘而坚,诊曰有积气在中,时害于食,名曰心痹,得之外疾,思虑而心虚,故邪从之。白脉之至也喘而浮,上虚下实,惊,有积气在胸中,喘而虚,名曰肺痹,寒热,得之醉而使内也。青脉之至也长而左右弹,有积气在心下支月去,名曰肝痹,得之寒湿,与疝同法,腰痛,足清,头痛。黄脉之至也,大而虚,有积气在腹中,有厥气,名曰厥疝,女子同法,得之疾使四支汗出当风。黑脉之至也上坚而大,有积气在小腹与阴,名曰肾痹,得之沐浴清水而卧。
凡相五色之奇脉,面黄目青,面黄目赤,面黄目白,面黄目黑者,皆不死也。面青目赤,面赤目白,面青目黑,面黑目白,面赤目青,皆死也。
五藏生成篇第十参考译文
所以过食咸味,则使血脉凝塞不畅,而颜面色泽发生变化。过食苦味,则使皮肤枯槁而毫毛脱落。过食辛味,则使筋脉劲急而爪甲枯干。过食酸味,则使肌肉粗厚皱缩而口唇掀揭。过食甘味,则使骨骼疼痛而头发脱落。这是偏食五味所造成的损害。所以心欲得苦味,肺欲得辛味,肝欲得酸味,脾欲得甘味,肾欲得咸味,这是五味分别与五脏之气相合的对应关系。
面色出现青如死草,枯暗无华的,为死症。出现黄如枳实的,为死症;出现黑如烟灰的,为死症;出现红如凝血的,为死症;出现白如枯骨的,为死症;这是五色中表现为死症的情况。面色青如翠鸟的羽毛,主生;红如鸡冠的,主生;黄如蟹腹的,主生;白如猪脂的,主生;黑如乌鸦毛的,主生。这是五色中表现有生机而预后良好的情况。心有生机,面色就象细白的薄绢裹着朱砂;肺有生机,面色就象细白的薄绢裹着粉红色的丝绸;肝有生机面色就象细白的薄绢裹着天青色的丝绸;脾有生机,面色就象细白的薄绢裹着栝蒌实;肾有生机,面色就象细白的薄绢裹着紫色的丝绸。这些都是五脏的生机显露于外的荣华。
色、味与五脏相应:白色和辛味应于肺,赤色和苦味应于心,青色和酸味应于肝,黄色和甘味应于脾,黑色和咸味应于肾。因五脏外合五体,所以白色应于皮,赤色应于脉,青色应于筋,黄色应于肉,黑色应于骨。
各条脉络,都属于目,而诸髓都属于脑,诸筋都属于骨节,诸血都属于心,诸气都属于肺。同时,气血的运行则朝夕来往,不离于四肢八溪的部位。所以当人睡眠时,血归藏于肝,肝得血而濡养于目,则能视物;足得血之濡养,就能行走;手掌得血之濡,就能握物;手指得血之濡养,就能拿取。如果刚刚睡醒就外出受风,血液的循环就要凝滞,凝于肌肤的,发生痹证;凝于经脉的,发生气血运行的滞涩;凝于足部的,该部发生厥冷。这三种情况,都是由于气血的运行不能返回组织间隙的孔穴之处,所以造成痹厥等症。全身有大谷十二处,小溪三百五十四处,这里面减除了十二脏腑各自的俞穴数目。这些都是卫气留止的地方,也是邪气客居之所。治病时,可循着这些部位施以针石,以祛除邪气。
诊病的根本,要以五决为纲纪。想要了解疾病的要关键,必先确定病变的原因。所谓五决,就是五脏之脉,以此诊病,即可决断病本的所在。比如头痛等巅顶部位的疾患,属于下虚上实的,病变在足少阴和足太阳经,病甚的,可内传于肾。头晕眼花,身体摇动,目暗耳聋,属下实上虚的,病变在足少阳和足厥阴经,病甚的,可内传于肝。腹满瞋胀,支持胸膈协助,属于下部逆气上犯的,病变在足太阴和足阳明经。咳嗽气喘,气机逆乱于胸中,病变在手阳明和手太阳经。心烦头痛,胸膈不适的,病变在手太阳和手少阴经。
脉象的小、大、滑、浮、沉等,可以通过医生的手指加以鉴别;五脏功能表现于外,可以通过相类事物的比象,加以推测;五脏各自的声音,可以凭意会而识别,五色的微小变化,可以用眼睛来观察。诊病时,如能将色、脉两者合在一起进行分析,就可以万无一失了。外现赤色,脉来急疾而坚实的,可诊为邪气积聚于中脘,常表现为妨害饮食,病名叫做心痹。这种病得之于外邪的侵袭,是由于思虑过度以至心气虚弱,邪气才随之而入的。外现白色,脉来急疾而浮,这是上虚下实,故常出现惊骇,病邪积聚于胸中,迫肺而作喘,但肺气本身是虚弱的,这种病的病名叫做肺痹,它有时发寒热,常因醉后行房而诱发。青色外现,脉来长而左右搏击手指,这是病邪积聚于心下,支撑协肋,这种病的病名叫做肝痹,多因受寒湿而得,与疝的病理相同,它的症状有腰痛、足冷、头痛等。外现黄色,而脉来虚大的,这是病邪积聚在腹中,有逆气产生,病名叫做厥疝,女子也有这种情况,多由四肢剧烈的活动,汗出当风所诱发。外现黑色,脉象尺上坚实而大,这是病邪积聚在小腹与前阴,病名叫做肾痹,多因冷水沐浴后睡卧受凉所引起。
大凡观察五色,面黄目青、面黄目赤、面黄目白、面黄目黑的、皆为不死,因面带黄色,是尚有土气。如见面青目赤、面赤目白、面青目黑、面黑目白、面赤木青的,皆为死亡之征象,因面无黄色,是土气以败。

\chapter{五藏别论篇第十一}
黄帝问曰:余闻方土,或以脑髓为脏,或以肠胃为脏,或以为腑。敢问更相反,皆自谓是。不知其道,愿闻其说。
岐伯对曰:脑、髓、骨、脉、胆、女子胞,此六者,地气之所生也,皆藏于阴而象于地,故藏而不泻,名曰奇恒之府。夫胃、大肠、小肠、三焦、膀胱,此五者,天气之所生也,其气象天,故泻而不藏,此受五藏浊气,名曰传化之腑。此不能久留,输泻者也。魄门亦为五藏使,水谷不得久藏。所谓五脏者,藏精气而不泻也,故满而不能实。六府者,传化物而不藏,故实而不能满也。所以然者,水谷入口,则胃实而肠虚;食下,则肠实而胃虚,故曰实而不满,满而不实也。
帝曰:气口何以独为五藏主?
岐伯曰:胃者,水谷之海,六府之大源也。五味入口,藏于胃,以养五藏气;气口亦太阴也,是以五脏六腑之气味,皆出于胃,变见于气口。故五气入鼻,藏于心肺;心肺有病,而鼻为之不利也。凡治病必察其下,适其脉,观其志意,与其病也。
拘于鬼神者,不可与言至德;恶于针石者,不可与言至巧;病不许治者,病必不治,治之无功矣。
五藏别论篇第十一参考译文
皇帝问道:我听说方士之中,有人以脑髓为脏,有人以肠胃为脏,也有的把这些都称为腑,如果向它们提出相反的意见,却又都坚持自己的看法,不知哪种理论是对的,希望你谈一谈这个问题。岐伯回答说:脑、髓、骨、脉、胆、女子胞,这六种是禀承地气而生的,都能贮藏阴质,就象大地包藏万物一样,所以它们的作用是藏而不泻,叫做奇恒之腑。胃、大肠、小肠、三焦、膀胱,这五者是禀承天气所生的,它们的作用,像天一样的健运周转,所以是泻而不藏的,它们受纳五脏的浊气,所以称为传化之腑。这是因为浊气不能久停其间,而必须及时转输和排泄的缘故。此外,肛门也为五脏行使输泻浊气,这样,水谷的糟粕就不会久留于体内了。所谓五脏,它的功能是贮藏经气而不向外发泻的,所以它是经常地保持精神饱满,而不是一时地得到充实。六腑,它的功能是将水谷加以传化,而不是加以贮藏,所以它有时显的充实,但却不能永远保持盛满。所以出现这种情况,是因为水谷入口下行,胃充实了,但肠中还是空虚的,食物再下行,肠充实了,而胃中就空虚了,这样依次传递。所以说六腑是一时的充实,而不是持续的盛满,五脏则是持续盛满而不是一时的充实。
黄帝问道:为什麽气口脉可以独主五脏的病变呢?岐伯说:胃是水谷之海,为六腑的泉源,饮食五味入口,留在胃中,经足太阴脾的运化输转,而能充养五脏之气。脾为太阴经,主输布津液,气口为手太阴肺经过之处,也属太阴经脉,主朝百脉,所以五脏六腑的水谷精微,都出自胃,反映于气口的。而五气入鼻,藏留于心肺,所以心肺有了病变,则鼻为之不利。凡治病并观察其上下的变化,审视其脉侯的虚实,查看其情志精神的状态以及痴情的表现。对那些拘守鬼神迷信观念的人,是不能与其谈论至深的医学理论的,对那些讨厌针石治疗的人,也不可能和他们讲什麽医疗技巧。有病不许治疗的人,他的病是治不好的,勉强治疗也收不到应有的功效。
\chapter{异法方宜论篇第十二}

黄帝曰:医之治病也,一病而治各不同,皆愈,何也?
岐伯对曰:地势使然也。故东方之域,天地之所始生也,鱼盐之地,海滨傍水。其民食鱼而嗜咸,皆安其处,美其食。鱼者使人热中,盐者胜血,故其民皆黑色疏理,其病皆为痈疡,其治宜砭石。故砭石者,亦从东方来。
西方者,金玉之域,沙石之处,天地之所收引也。其民陵居而多风,水土刚强,其民不衣而褐荐,其民华食而脂肥,故邪不能伤其形体,其病生于内,其治宜毒药。故毒药者,亦从西方来。
北方者,天地所闭藏之域也,其地高陵居,风寒冰洌。其民乐野处而乳食,藏寒生满病,其治宜炙芮。故灸芮者,亦从北方来。
南方者,天地所长养,阳之所盛处也,其地下,水土弱,雾露之所聚也。其民嗜酸而食腐,故其民皆致理而赤色,其病挛痹,其治宜微针。故九针者,亦从南方来。
中央者,其地平以湿,天地所以生万物也众。其民食杂而不劳,故其病多瘘厥寒热,其治宜导引按蹻。故导引按蹻者,亦从中央出也。
故圣人杂合以治,各得其所宜。故治所以异而病皆愈者,得病之情,知治之大体也。
异法方宜论篇第十二参考译文
黄帝问道:医生医疗疾病,同病而采取各种不同的治疗方法,但结果都能痊愈,这是什麽道理?岐伯回答说:这是因为地理形式不同,而治法各有所宜的缘故。
例如东方的天地始生之气,气候温和,是出产鱼和盐的地方。由于地处海滨而接近于水,所以该地方的人们多吃鱼类而喜欢咸味,他们安居在这个地方,以鱼盐为美食。但由于多吃鱼类,鱼性属火会使人热积于中,过多的吃盐,因为咸能走血,又会耗伤血液,所以该地的人们,大都皮肤色黑,肌理松疏,该地多发痈疡之类的疾病。对其治疗,大都宜用砭石刺法。因此,砭石的治病方法,也是从东方传来的。
西方地区,是多山旷野,盛产金玉,遍地沙石,这里的自然环境,象秋令之气,有一种收敛引急的现象。该地的人们,依山陵而住,其地多风,水土的性质又属刚强,而他们的生活,不堪考究衣服,穿毛巾,睡草席,但饮食都是鲜美酥酪骨肉之类,因此体肥,外邪不容易侵犯他们的形体,他们发病,大都属于内伤类疾病。对其治疗,宜用药物。所以药物疗法,是从西方传来的。
北方地区,自然气候如同冬天的闭藏气象,地形较高。人们依山陵而居住,经常处在风寒冰冽的环境中。该地的人们,喜好游牧生活,四野临时住宿,吃的是牛羊乳汁,因此内脏受寒,易生胀满的疾病。对其治疗,宜用艾火炙灼。所以艾火炙灼的治疗方法,是从北方传来的。
南方地区,象自然界万物长养的气候,阳气最盛的地方,地势低下,水土薄弱,因此雾露经常聚集。该地的人们,喜欢吃酸类和腐熟的食品,其皮肤腠理致密而带红色,易发生筋脉拘急、麻木不仁等疾病。对其治疗,宜用微针针刺。所以九针的治病方法,是从南方传来的。
中央之地,地形平坦而多潮湿,物产丰富,所以人们的食物种类很多,生活比较安逸,这里发生的疾病,多是痿弱、厥逆、寒热等病,这些病的治疗,宜用导引按蹻的方法。所以导引按蹻的治法,是从中央地区推广出去的。
从以上情况来看,一个高明的医生,是能够将这许多治病方法综合起来,根据具体情况,随机应变,灵活运用,使患者得到适宜治疗。所以治法尽管各有不同,而结果是疾病都能痊愈。这是由于医生能够了解病情,并掌握了治疗大法的缘故。
\chapter{移精变气论篇第十三}
黄帝问曰:余闻古之治病,惟其移精变气,可祝由而已。今世治病,毒药治其内,针石治其外,或愈或不愈,何也?
岐伯对曰:往石人居禽兽之间,动作以避寒,阴居以避暑,内无眷慕之累,外无伸宦之形,此恬淡之世,邪不能深入也。故毒药不能治其内,针石不能治其外,故可移精祝由而已。当今之世不然,忧患缘其内,苦形伤其外,又失四时之从,逆寒暑之宜,贼风数至,虚邪朝夕,内至五脏骨髓,外伤空窍肌肤;所以小病必甚,大病必死,故祝由不能已也。
帝曰:善。余欲临病人,观死生,决嫌疑,欲知其要,如日月光,可得闻乎?岐伯曰:色脉者,上帝之所贵也,先师之所传也。上古使僦贷季,理色脉而通神明,合之金木水火土、四时、八风、六合,不离其常,变化相移,以观其妙,以知其要。欲知其要,则色脉是矣。色以应日,脉以应月,常求其要,则其要也。夫色之变化,以应四时之脉,此上帝之所贵,以合于神明也,所以远死而近生。生道以长,命曰圣王。
中古之治,病至而治之汤液,十日,以去八风五痹之病,十日不已,治以草苏草亥之枝,本末为助,标本已得,邪气乃服。暮世之治病也则不然,治不本四时,不知日月,不审逆从,病形已成,乃欲微针治其外,汤液治其内,粗工凶凶,以为可攻,故病未已,新病复起。
帝曰:愿闻要道。
岐伯曰:治之要极,无失色脉,用之不惑,治之大则。逆从倒行,标本不得,亡神失国!去故就新,乃得真人。
帝曰:余闻其要于夫子矣!夫子言不离色脉,此余之所知也。岐伯曰:治之极于一。
帝曰:何谓一?
岐伯曰:一者因问而得之。
帝曰:奈何?
岐伯曰:闭户塞牖,系之病者,数问其情,以从其意,得神者昌,失神者亡。
帝曰:善。
移精变气论篇第十三参考译文
黄帝问道:我听说古时治病,只要对病人移易精神和改变气的运行,用一种“祝由”的方法,病就可以好了。现在医病,要用药物治其内,针石治其外,疾病还是有好、有不好,这是什麽缘故呢?岐伯回答说:古时候的人们,生活简单,巢穴居处,在禽兽之间追逐生存,寒冷到了,利用活动以除寒冷,暑热来了,就到阴凉的地方避免暑气,在内没有眷恋羡慕的情志牵挂,在外没有奔走求官的劳累形役,这里处在一个安静淡薄、不谋势利、精神内守的意境里,邪气是不可能深入侵犯的。所以既不须要药物治其内,也不须要针石治其外。即使有疾病的发生,亦只要对病人移易精神和改变气的运行,用一种“祝由”的方法,病就可以好了。现在的人就不同了,内则为忧患所牵累,外则为劳苦所形役,又不能顺从四时气候的变化,常常遭受到“虚邪贼风”的侵袭,正气先馁,外邪乘虚而客袭之,内犯五脏骨髓,外伤孔窍肌肤,这样轻病必重,重病必死,所以用祝由的方法就不能医好疾病了。
黄帝道:很好!我想要临诊病人,能够察其死生,决断疑惑,掌握要领,如同日月之光一样的心中明了,这种诊法可以讲给我听吗?岐伯曰:在诊法上,色和脉的诊察方法,是上帝所珍重,先师所传授的。上古有位名医叫僦贷季,他研究色和脉的道理,通达神明,能够联系到金木水火土以及四时、八风、六合,从正常的规律和异常的变化,来综合分析,观察它的变化奥妙,从而知道其中的要领。我们如果要能懂得这些要领,就只有研究色脉。气色是象太阳而有阴晴,脉息是象月亮而有盈亏,从色脉中得其要领,正是诊病的重要关键。而气色的变化,与四时的脉象是相应的,这是上古帝王所十分珍重的,若能明白原理,心领神会,便可运用无穷。所以他能从这些观察中间,掌握情况,知道去回避死亡而达到生命的安全。要能够做到这样就可以长寿,而人们亦将称奉你为“圣王”了。
中午时候的医生治病,多在疾病一发生就能及时治疗,先用汤液十天,以祛除“八风”、“五痹”的病邪。如果十天不愈,再用草药治疗。医生还能掌握病情,处理得当,所以邪气就被征服,疾病也就痊愈。至于后世的医生治病,就不是这样了,治病不能根据四时的变化,不知道阴阳色脉的关系,也不能够辨别病情的顺逆,等到疾病已经形成了,才想用微针治其外,汤液治其内。医术浅薄、工作粗枝大叶的医生,还认为可以用攻法,不知病已形成,非攻可愈,以至原来的疾病没有痊愈,又因为治疗的错误,产生了新的疾病。
黄帝道:我愿听听有关临证方面的重要道理。岐伯说:诊治疾病极重要的关键在于不要搞错色脉,能够运用色脉而没有丝毫疑惑,这是临证诊治的最原则。假使色脉的诊法不能掌握,则对病情的顺逆无从理解,而处理亦将有倒行逆施的危险。医生的认识与病情不能取得一致,这样去治病,会损害病人的精神,若用以治国,是要使国家灭亡的!因此暮世的医生,赶快去掉旧习的简陋知识,对崭新的色脉学问要钻研,努力进取,是可以达到上古真人的地步的。黄帝道:我已听到你讲的这些重要道理,你说的主要精神是不离色脉,这是我已知道的。岐伯说:诊治疾病的主要关键,还有一个。黄帝道:是一个什麽关键?岐伯说:一个关键就是从与病人接触中问得病情。黄帝道:怎样问法?岐伯说:选择一个安静的环境,关好门窗,与病人取得密切联系,耐心细致的询问病情,务使病人毫无顾虑,尽情倾诉,从而得知其中的真情,并观察病人的神色。有神气的,预后良好;没有神气的,预后不良。黄帝说:讲得很好。
\chapter{汤液醪醴论篇第十四}
黄帝问曰:为五谷汤液及醪醴,奈何?
岐伯对曰:必以稻米,炊之稻薪,稻米者完,稻薪者坚。
帝曰:何以然?
岐伯曰:此得天地之和,高下之宜,故能至完;伐取得时,故能至坚也。
帝曰:上古圣人作汤液醪醴,为而不用,何也?
岐伯曰:自古圣人之作汤液醪醴者,以为备耳,夫上古作汤液,故为而弗服也。中古之世,道德稍衰,邪气时至,服之万全。帝曰:今之世不必已,何也?
岐伯曰:当今之世,必齐毒药攻其中,镵石、针艾治其外也。
帝曰:形弊血尽而功不立者何?
岐伯曰:神不使也。
帝曰:何谓神不使?
岐伯曰:针石,道也。精神不进,志意不治,故病不可愈。今精坏神去,荣卫不可复收。何者?嗜欲无穷,而忧患不止,精气驰坏,荣泣卫除,故神去之而病不愈也。
帝曰:夫病之始生也,极微极精,必先入结于皮肤。今良工皆称曰病成,名曰逆,则针石不能治,良药不能及也。
今良工皆得其法,守其数,亲戚兄弟远近,音声日闻于耳,五色日见于目,而病不愈者,亦何暇不早乎?
岐伯曰:病为本,工为标,标本不得,邪气不服,此之谓也。
帝曰:其有不从毫毛而生,五藏阳以竭也,津液充郭,其魄独居,孤精于内,气耗于外,形不可与衣相保,此四极急而动中,是气拒于内,而形施于外,治之奈何?
岐伯曰:平治于权衡,去宛陈莝,微动四极,温衣,缪刺其处,以复其形。开鬼门,洁净府,精以时限,五阳已布,疏涤五脏。故精自生,形自盛,骨肉相保,巨气乃平。
帝曰:善。
汤液醪醴论篇第十四参考译文
黄帝问道:用五谷来做成汤液及醪醴,应该怎样?岐伯回答说:必须要用稻米作原料,以稻杆作燃料,因为稻米之气完备,稻杆又很坚劲。黄帝问道:何以见得?岐伯说:稻禀天地之和气,生长于高下适宜的地方,所以得气最完;收割在秋时,故其杆坚实。
黄帝道:上古时代有学问的医生,制成汤液和醪醴,但虽然制好,却备在那里不用,这是什麽道理?岐伯说:古代有学问的医生,他做好的汤液和醪醴,是以备万一的,因为上古太和之世,人们身心康泰,很少疾病,所以虽制成了汤液,还是放在那里不用的。到了中古代,养生之道稍衰,人们的身心比较虚弱,因此外界邪气时常能够乘虚伤人,但只要服些汤液醪醴,病就可以好了。黄帝道:现在的人,虽然服了汤液醪醴,而病不一定好,这是什麽缘故呢?岐伯说:现在的人和中古时代又不同了,一有疾病,必定要用药物内服,砭石、针炙外治,其病才能痊愈。
黄帝道:一个病情发展到了形体弊坏、气血竭尽的地步,治疗就没有办法见效,这里有什麽道理?岐伯说:这是因为病人的神气,已经不能发挥他的应有作用的关系。黄帝道:什麽叫做神气不能发生他的应有作用?岐伯说:针石治病,这不过是一种方法而已。现在病人的神气已经散越,志意已经散乱,纵然有好的方法,神气不起应有作用,而病不能好。况且病人的严重情况,是已经达到精神败坏,神气离去,荣卫不可以再恢复的地步了。为什麽病情会发展到这样的地步的呢?由于不懂得养生之道,嗜好欲望没有穷尽,忧愁患难又没有止境,以致于一个人的精气败坏,荣血枯涩,卫气作用消失,所以神气失去应有的作用,对治疗上的方法以失却反应,当然他的病就不会好。
黄帝道:凡病初起,固然是精微难测,但大致情况,是必先侵袭于皮肤,所谓表证。现在经过医生一看,都说是病已经成,而且发展和预后很不好,用针石不能治愈,吃汤药亦不能达到病所了。现在医生都能懂得法度,操守术数,与病人象亲戚兄弟一样亲近,声音的变化每日都能听到,五色的变化每日都能看到,然而病却医不好,这是不是治疗得不早呢?岐伯说:这是因为病人为本,医生为标,病人与医生不能很好合作,病邪就不能制服,道理就在这里。
黄帝道:有的病不是从外表毫毛而生的,是由于五脏的阳气衰竭,以致水气充满于皮肤,而阴气独盛,阴气独居于内,则阳气更耗于外,形体浮肿,不能穿原来的衣服,四肢肿急而影响到内脏,这是阴气格拒于内,而水气弛张于外,对这种病的治疗方法怎样呢?岐伯说:要平复水气,当根据病情,衡量轻重,驱除体内的积水,并叫病人四肢做些轻微运动,令阳气渐次宣行,穿衣服带温暖一些,助其肌表之阳,而阴凝易散。用缪刺方法,针刺肿处,去水以恢复原来的形态。用发汗和利小便的方法,开汗孔,泻膀胱,使阴精归于平复,五脏阳气输布,以疏通五脏的郁积。这样,精气自会生成,形体也强盛,骨骼与肌肉保持着常态,正气也就恢复正常了。黄帝道:讲得很好。
\chapter{玉版论要篇第十五}
黄帝问曰:余闻揆奇恒所指不同,用之奈何?
岐伯曰:揆度者,度病之浅深也。奇恒者,言奇病也。请言道之至数,五色脉变,揆度奇恒,道在于一。神转不回,回则不转,乃失其机。至数之要,迫近以微,著之玉版,命曰合玉机。
容色见上下左右,各在其要。其色见浅者,汤液主治,十日已;其见深者,必齐主治,二十一日已;其见大深者,醪酒主治,百日已;色夭面脱,不治,百日尽已。脉短气绝死;病温虚甚死。
色见上下左右,名在其要。上为逆,下为从;女子右为逆,左为从;男子左为逆,右为从。易,重阳死,重阴死。阴阳反他,治在权衡相夺,奇恒事也,揆度事也。
搏脉痹躄,寒热之交。脉孤为消气;虚泄为夺血。孤为逆,虚为从。行奇恒之法,以太阴始,行所不胜曰逆,逆则死;行所胜曰从,从则活。八风四时之胜,终而复始,逆行一过,不复可数。论要毕矣。
玉版论要篇第十五参考译文
黄帝问道:我听说揆度、奇恒的诊法,运用的地方很多,而所指是不同的,究竟怎样运用呢?岐伯回答说:一般来讲,《揆度》是用以衡量疾病的深浅。奇恒是辨别异于正常的疾病。请允许我从诊病的主要理数说起,五色、脉变、揆度、奇恒等,虽然所指不同,但道理只有一个,就是色脉之间有无神气。人体的气血随着四时的递迁,永远向前运转而不回折。如若回折了,就不能运转,就失却生机了!这个道理很重要,诊色脉是浅近的事,而微妙之处却在于察神机。把它记录在玉版上,可以与《玉机真藏论》合参的。
面容的五色变化,呈现在上下左右不同的部位,应分别其深浅顺逆之要领。如色见浅的,其病轻,可用五谷汤液调理,约十天就可以好了;其色见深的,病重,就必须服用药剂治疗,约二十一天才可以恢复;如果其色过深,则其病更为严重,必定要用药酒治疗,须经过一百天左右,才能痊愈;假如神色枯槁,面容瘦削,就不能治愈,到一百天就要死了。除此以外,如脉气短促而阳气虚脱的,必死;温热病而正气虚极的,亦必死。
面色见于上下左右,必须辨别观察其要领。病色向上移的为逆,向下移的为顺;女子病色在右侧的为逆,在左侧的为顺;男子病色在左侧的为逆,在右侧的为顺。如果病色变更,倒顺为逆,那就是重阳、重阴了,重阳、重阴的预后不好。假如到了阴阳相反之际,应尽快衡量其病情,果断的采用适当的治法,使阴阳趋于平衡,这就在于揆度、奇恒的运用了。
脉象搏击于指下,是邪盛正衰之象,或为痹证,或为躄证,或为寒热之气交合为病。如脉见孤绝,是阳气损耗;如脉见虚弱,而又兼下泄,为阴血损伤。凡脉见孤绝,预后都不良;脉见虚弱,预后当好。在诊脉时运用奇恒之法,从手太阴经之寸口脉来研究。就所见之脉在四时、五行来说,不胜现象(如春见秋脉,夏见冬脉),为逆,预后不良;如所见之脉是所胜现象(如春见长夏脉,夏见秋脉),为顺,预后良好。至于八风、四时之间的相互胜复,是循环无端,终而复始的,假如四时气候失常,就不能用常理来推断了。至此,则揆度奇恒之要点都论述完了
\chapter{诊要经终论篇第十六}
黄帝问曰:诊要何如?
岐伯对曰:正月、二月,天气始方,地气始发,人气在肝;三月、四月,天气正方,地气定发,人气在脾;五月、六月,天气盛,地气高,人气在头;七月、八月,阴气始杀,人气在肺;九月、十月,阴气始冰,地气始闭,人气在心;十一月、十二月,冰复,地气合,人气在肾。
故春刺散俞,及与分理,血出而上,甚者传气,间者环也。夏刺络俞,见血而止,尽气闭环,痛病必下。秋刺皮肤,循理,上下同法,神变而止。冬刺俞窍于分理,甚者直下,间者散下。春夏秋冬,各有所刺,法其所在。
春刺夏分,脉乱气微,入淫骨髓,病不能愈,令人不嗜食,又且少气;春刺秋分,筋挛逆气,环为咳嗽,病不愈,令人时惊,又且哭;春刺冬分,邪气著藏,令人胀,病不愈,又且欲言语。
夏刺春分,病不愈,令人解堕;夏刺秋分,病不愈,令人心中欲无言,惕惕如人将捕之;夏刺冬分,病不愈,令人少气,时欲怒。
秋刺春分,病不已,令人惕然欲有所为,起而忘之;秋刺夏分,病不已,令人益嗜卧,又且善梦;秋刺冬分,病不已,令人洒洒时寒。
冬刺春分,病不已,令人欲卧不能眠,眠而有见;冬刺夏分,病不愈,气上,发为诸痹;冬刺秋风,病不已,令人善渴。
凡刺胸腹者,必避五脏。中心者,环死;中脾者,五日死;中肾者,七日死;中肺者,五日死;中鬲者,皆为伤中,其病虽愈,不过一岁必死。刺避五脏者,知逆从也。所谓从者,鬲与脾肾之处,不知者反之。刺胸腹者,必以布巾著之,乃从单布上刺,刺之不愈,复刺。刺针必肃,刺肿摇针,经刺勿摇。此刺之道也。
帝曰:愿闻十二经脉之终奈何?
岐伯曰:太阳之脉,其终也,戴眼,反折瘛疭,其色白,绝汗乃出,出则死矣。少阳终者,耳聋,百节皆纵,目圜绝系,绝系一日半死,其死也,色先青白,乃死矣。阳明终者,口目动作,善惊,妄言,色黄,其上下经盛,不仁,则终矣。少阴终者,面黑,齿长而垢,腹胀闭,上下不通而终矣。太阴终者,腹胀闭不得息,善噫,善呕,呕则逆,逆则面赤,不逆则上下不通,不通则面黑,皮毛焦而终矣。厥阴终者,中热嗌干,善溺心烦,甚则舌卷,卵上缩而终矣。此十二经之所败也。
诊要经终论篇第十六参考译文
黄帝问道:诊病的重要关键是什麽?岐伯回答说:重要点在于天、地、人相互之间的关系。如正月、二月,天气开始有一种升发的气象,地气也开始萌动,这时候的人气在肝;三月、四月,天气正当明盛,地气也正是华茂而欲结实,这时候的人气在脾;五月、六月,天气盛极,地气上升,这时候的人气在头部;七月、八月,阴气开始发生肃杀的现象,这时候的人气在肺;九月、十月,阴气渐盛,开始冰冻,地气也随着闭藏,这时候的人气在心;十一月、十二月,冰冻更甚而阳气伏藏,地气闭密,这时候的人气在肾。由于人气与天地之气皆随顺阴阳之升沉,所以春天的刺法,应刺经脉俞穴,及于分肉腠理,使之出血而止,如病比较重的应久留其针,其气传布以后才出针,较轻的可暂留其针,候经气循环一周,就可以出针了。夏天的刺法,应刺孙络的俞穴,使其出血而止,使邪气尽去,就以手指扪闭其针孔伺其气行一周之顷,凡有痛病,必退下而愈。秋天的刺法应刺皮肤,顺着肌肉之分理而刺,不论上部或下部,同样用这个方法,观察其神色转变而止。冬天的刺法应深取俞窍于分理之间,病重的可直刺深入,较轻的,可或左右上下散布其针,而稍宜缓下。
春夏秋冬,各有所宜的刺法,须根据气之所在,而确定刺的部位。如果春天刺了夏天的部位,伤了心气,可使脉乱而气微弱,邪气反而深入,浸淫于骨髓之间,病就很难治愈,心火微弱,火不生土,有使人不思饮食,而且少气了;春天刺了秋天的部位,伤了肺气,春病在肝,发为筋挛,邪气因误刺而环周于肺,则又发为咳嗽,病不能愈,肝气伤,将使人时惊,肺气伤,且又使人欲哭;春天刺了冬天的部位,伤了肾气,以致邪气深着于内脏,使人胀满,其病不但不愈,肝气日伤,而且使人多欲言语。
夏天刺了春天的部位,伤了肝气,病不能愈,反而使人精力卷怠;夏天刺了秋天的部位,伤了肺气,病不能愈,反而使人肺气伤而声不出,心中不欲言,肺金受伤,肾失其母,故虚而自恐,惕惕然好象被人逮捕的样子;夏天刺了冬天的部位,伤了肾气,病不能愈,反而使精不化气而少气,水不涵木而时常要发怒。
秋天刺了春天的部位,伤了肝气,病不能愈,反而使人血气上逆,惕然不宁,且又善忘;秋天刺了夏天的部位,伤了心气,病不能愈,心气伤,火不生土,反而使人嗜卧,心不藏神,又且多梦;秋天刺了冬天的部位,伤了肾气,病不能愈,反使人肾不闭藏,血气内散,时时发冷。
冬天刺了春天的部位,伤了肝气,病不能愈,肝气少,魂不藏,使人困倦而又不得安眠,即便得眠,睡中如见怪异等物;冬天刺了夏天的部位,伤了心气,病不能愈,反使人脉气发泄,而邪气闭痹于脉,发为诸痹;冬天刺了秋天的部位,伤了肺气,病不能愈,化源受伤,凡使人常常作渴。
凡于胸腹之间用针刺,必须注意避免刺伤了五脏。假如中伤了心脏,经气环身一周便死;假如中伤了脾脏,五日便死;假如中伤了肾脏,七日便死;假如中伤了肺脏,五日便死;假如中伤隔膜的,皆为伤中,当时病虽然似乎好些,但不过一年其人必死。刺胸腹注意避免中伤五脏,主要是要知道下针的逆从。所谓从,就是要明白膈和脾肾等处,应该避开;如不知其部位不能避开,就会刺伤五脏,那就是逆了。凡刺胸腹部位,应先用布巾覆盖其处,然后从单布上进刺。如果刺之不愈,可以再刺,这样就不会把五脏刺伤了。在用针刺治病的时候,必须注意安静严肃,以候其气;如刺脓肿的病,可以用摇针手法以出脓血;如刺经脉的病,就不要摇针。这是刺法的一般规矩。
黄帝问道:请你告诉我十二经气绝的情况是怎样的?岐伯回答说:太阳经脉气绝的时候,病人两目上视,身背反长,手足抽掣,面色发白,出绝汗,绝汗一出,便要死亡了。少阳经脉气绝的时候,病人耳聋,遍体骨节松懈,两目直视如惊,到了目珠不转,一日半便要死了;临死的时候,面色先见青,再由青色变为白色,就要死亡了。阳名经脉气绝的时候,病人口眼牵引歪斜而瞤动,时发惊惕,言语胡乱失常,面色发黄,其经脉上下所过的部分,都表现出盛躁的症状,由盛躁而渐至肌肉麻木不仁,便死亡了。少阴经脉气绝的时候,病人面色发黑,牙龈收削而牙齿似乎变长,并积满污垢,腹部胀闭,上下不相通,便死亡了。太阴经脉气绝的时候,腹胀闭塞,呼吸不利,常欲嗳气,并且呕吐,呕则气上逆,气上逆则面赤,假如气不上逆,又变为上下不通,不通则面色发黑,皮毛枯樵而死了。厥阴经脉气绝的时候,病人胸中发热,咽喉干燥,时时小便,心胸烦躁,渐至舌卷,睾丸上缩,便要死了。以上就是十二经脉气绝败坏的症候。
\chapter{脉要精微论篇第十七}
黄帝问曰:诊法何如?
岐伯对曰:诊法常以平旦,阴气未动,阳气未散,饮食未进,经脉未盛,络脉调匀,气血未乱,故乃可诊有过之脉。
切脉动静,而视精明,察五色,观五藏有余不足,六腑强弱,形之盛衰。以此参伍,决死生之分。
夫脉者,血之府也。长则气治,短则气病,数则烦心,大则病进,上盛则气高,下盛则气胀,代则气衰,细则气少,涩则心痛,浑浑革至如涌泉。病进而色弊,绵绵其去如弦绝,死。
夫精明五色者,气之华也。赤欲如白裹朱,不欲如赭;白欲如鹅羽,不欲如盐;青欲如苍壁之泽,不欲如蓝;黄欲如罗裹雄黄,不欲如黄土;黑欲如漆色,不欲如地苍。五色精微象见矣,其寿不久也。夫精明者,所以视万物、别白黑,审短长。以长为短,以白为黑,如是则精衰矣。
五藏者,中之守也。中盛脏满,气胜伤恐者,声如从室中言,是中气之湿也;言而微,终日乃复言者,此夺气也;衣被不敛,言语善恶,不避亲疏者,此神明之乱也;仓廪不藏者,是门户不要也;水泉不止者,是膀胱不藏也。得守者生,失守者死。夫五脏者,身之强也。头者,精明之府,头倾视深,精神将夺矣;背者,胸中之府,背曲肩随,府将坏矣;腰者,肾之府,转摇不能,肾将惫矣;膝者,筋之府,屈伸之能,行则偻附,筋将惫矣;骨者,髓之府,不能久立,行则振掉,骨将惫矣。得强则生,失强则死。
岐伯曰:反四时者,有余为精,不足为消。应太过,不足为精;应不足,有余为消。阴阳不相应,病名曰关格。
帝曰:脉其四时动奈何?知病之所在奈何?知病之所变奈何?知病乍在内奈何?知病乍在外奈何?请问此五者,可得闻乎?岐伯曰:请言其与天运转大也。万物之外,六合之内,天地之变,阴阳之应,彼春之暖,为夏之暑,彼秋之忿,为冬之怒。四变之动,脉与之上下,以春应中规,夏应中矩,秋应中衡,冬应中权。是故冬至四十五日,阳气微上,阴气微下;夏至四十五日,阴气微上,阳气微下。阴阳有时,与脉为期,期而相失,知脉所分,分之有期,故知死时。微妙在脉,不可不察,察之有纪,从阴阳始,始之有经,从五行生,生之有度,四时为宜,补写勿失,与天地如一,得一之情,以知死生。是故声合五音,色合五行,脉合阴阳。
是知阴盛则梦涉大水恐惧,阳盛则梦大火燔灼,阴阳俱盛则梦相杀毁伤;上盛则梦飞,下盛则梦堕;甚饱则梦予,甚饥则梦取;肝气盛则梦怒,肺气盛则梦哭;短虫多则梦聚众,长虫多则相击毁伤。
是故持脉有道,虚静为保。春日浮,如鱼之游在波;夏日在肤,泛泛乎万物有余;秋日下肤,蛰虫将去;冬日在骨,蛰虫周密,君子居室。故曰:知内者按而纪之,知外者终而始之。此六者,持脉之大法。
心脉搏坚而长,当病舌卷不能言;其软而散者,当消环自己。肺脉搏坚而长,当病唾血;其软而散者,当病灌汗,至今不复散发也。肝脉搏坚而长,色不青,当病坠若搏,因血在胁下,令人喘逆;其软而散,色泽者,当病溢饮。溢饮者,渴暴多饮,而易入肌皮肠胃之外也。胃脉搏坚而长,其色赤,当病折髀;其软而散者,当病食痹。脾脉搏坚而长,其色黄,当病少气;其软而散,色不泽者,当病足胫肿,若水状也。肾脉搏坚而长,其色黄而赤者,当病折腰;其软而散者,当病少血,至今不复也。
帝曰:诊得心脉而急,此为何病?病形何如?岐伯曰:病名心疝,少腹当有形也。帝曰:何以言之?岐伯曰:心为牡藏,小肠为之使,故曰少腹当有形也。帝曰:诊得胃脉,病形何如?岐伯曰:胃脉实则胀,虚则泄。
帝曰:病成而变何谓?岐伯曰:风成为寒热;瘅成为消中;厥成为巅疾;久风为飧泄;脉风成为疠。病之变化,不可胜数。
帝曰:诸痈肿筋挛骨痛,此皆安生?岐伯曰:此寒气之肿,八风之变也。帝曰:治之奈何?岐伯曰:此四时之病,以其胜治之愈也。
帝曰:有故病五藏发动,因伤脉色,各何以知其久暴至之病乎?岐伯曰:悉乎哉问也!徵其脉小色不夺者,新病也;徵其脉不夺,其色夺者,此久病也;徵其脉与五色俱夺者,此久病也;徵其脉与五色俱不夺者,新病也。肝与肾脉并至,其色苍赤,当病毁伤,不见血,已见血,湿若中水也。
尺内两傍,则季胁也,尺外以候肾,尺里以候腹。中附上,左外以候肝,内以候鬲;右外以候胃,内以候脾。上附上,右外以候肺,内以候胸中;左外以候心,内以候膻中。前以候前,后以候后。上竟上者,胸喉中事也;下竟下者,少腹腰股膝胫足中事也。
粗大者,阴不足,阳有余,为热中也。来疾去徐,上实下虚,为厥巅疾;来徐去疾,上虚下实,为恶风也,故中恶风者,阳气受也。有脉俱沉细数者,少阴厥也。沉细数散者,寒热也。浮而散者,为眩仆。诸浮不躁者,皆在阳,则为热;其有躁者在手。诸细而沉者,皆在阴,则为骨痛;其有静者在足。数动一代者,病在阳之脉也,泄及便脓血。诸过者,切之涩者,阳气有余也;滑者,阴气有余也。阳气有余为身热无汗;阴气有余为多汗身寒;阴阳有余则无汗而寒。推而外之,内而不外,有心腹积也;推而内之,外而不内,身有热也;推而上之,上而不下,腰足清也;推而下之,下而不上,头项痛也。按之至骨,脉气少者,腰脊痛而身有痹也。
脉要精微论篇第十七参考译文
黄帝问道:诊脉的方法是怎样的呢?岐伯回答说:诊脉通常是以清晨的时间为最好,此时人还没有劳于事,阴气未被扰动,阳气尚未耗散,饮食也未曾进过,经脉之气尚未充盛,络脉之气也很匀静,气血未受到扰乱,因而可以诊察出有病的脉象。在诊察脉搏动静变化的同时,还应观察目之精明,以候神气,诊察五色的变化,以审脏腑之强弱虚实及形体的盛衰,相互参合比较,以判断疾病的吉凶转归。
脉是血液汇聚的所在。长脉为气血流畅和平,故为气治;短脉为气不足,故为气病;数脉为热,热则心烦;大脉为邪气方张,病势正在向前发展;上部脉盛,为邪壅于上,可见呼吸急促,喘满之症;下部脉盛,是邪滞于下,可见胀满之病;代脉为元气衰弱;细脉,为正气衰少;涩脉为血少气滞,主心痛之症。脉来大而急速如泉水上涌者,为病势正在进展,且有危险;脉来隐约不现,微细无力,或如弓弦猝然断绝而去,为气血已绝,生机已断,故主死。
精明见于目,五色现于面,这都是内脏的精气所表现出来的光华。赤色应该象帛裹朱砂一样,红润而不显露,不应该象赭石那样,色赤带紫,没有光泽;白色应该象鹅的羽毛,白而光泽,不应该象盐那样白而带灰暗色;青色应该青而明润如璧玉,不应该象蓝色那样青而带沉暗色;黄色应该象丝包着雄黄一样,黄而明润,不应该象黄土那样,枯暗无华;黑色应该象重漆之色,光彩而润,不应该象地苍那样,枯暗如尘。假如五脏真色暴露于外,这是真气外脱的现象,人的寿命也就不长了。目之精明是观察万物,分别黑白,审察长短的,若长短不明,黑白不清,这是精气衰竭的现象。
五脏主藏精神在内,在体内各有其职守。如果邪盛于腹中,脏气壅满,气盛而喘,善伤于恐,讲话声音重浊不清,如在室中说话一样,这是中气失权而有湿邪所致。语音低微而气不接续,语言不能相继者,这是正气被劫夺所致。衣服不知敛盖,言语不知善恶,不辩亲疏远近的,这是神明错乱的现象。脾胃不能藏纳水谷精气而泄利不禁的,是中气失守,肛门不能约束的缘故。小便不禁的,是膀胱不能闭藏的缘故。若五脏功能正常,得其职守者则生;若五脏精气不能固藏,失其职守则死。五脏精气充足,为身体强健之本。头为精明之府,若见到头部低垂,目陷无光的,是精神将要衰败。背悬五脏,为胸中之府,若见到背弯曲而肩下垂的,是胸中脏气将要败坏。肾位居于腰,故腰为肾之府,若见到不能转侧摇动,是肾气将要衰惫。膝是筋汇聚的地方,所以膝为筋之府,若曲伸不能,行路要曲身附物,这是筋的功能将要衰惫。骨为髓之府,不能久立,行则震颤摇摆,这是髓虚,骨的功能将要衰惫。若脏气能够恢复强健,则虽病可以复生;若脏气不能复强,则病情不能挽回,人也就死了。
岐伯说:脉气与四时阴阳之气相反的,如相反的形象为有余,皆为邪气盛于正气,相反的形象为不足,为血气先己消损。根据时令变化,脏气当旺,脉气应有余,却反见不足的,这是邪气盛于正气;脉气应不足,却反见有余的,这是正不胜邪,邪气盛,而血气消损。这种阴阳不相顺从,气血不相营运,邪正不相适应而发生的疾病名叫关格。
黄帝问道:脉象是怎样应四时的变化而变动的呢?怎样从脉诊上知道病变的所在呢?怎样从脉诊上知道疾病的变化呢?怎样从脉诊上知道病忽然发生在内部呢?怎样从脉诊上知道病忽然发生在外部呢?请问这五个问题,可以讲给我听吗?岐伯说:让我讲一讲人体的阴阳升降与天运之环转相适应的情况。万物之外,六合之内,天地间的变化,阴阳四时与之相应。如春天的气候温暖,发展为夏天的气候暑热,秋天得劲急之气,发展为冬天的寒杀之气,这种四时气候的变化,人体的脉象也随着变化而升降浮沉。春脉如规之象;夏脉如矩之象;秋脉如秤衡之象,冬脉如秤权之象。四时阴阳的情况也是这样,冬至到立春的四十五天,阳气微升,阴气微降;夏至到立秋的四十五天,阴气微升,阳气微降。四时阴阳的升降是有一定的时间和规律的,人体脉象的变化,亦与之相应,脉象变化与四时阴阳不相适应,即是病态,根据脉象的异常变化就可以知道病属何脏,再根据脏气的盛衰和四时衰旺的时期,就可以判断出疾病和死亡的时间。四时阴阳变化之微妙,都在脉上有所反应,因此,不可不察。诊察脉象,有一定的纲领,就是从辨别阴阳开始,结合人体十二经脉进行分析研究,而十二经脉应五行而有生生之机;观测生生之机的尺度,则是以四时阴阳为准则;遵循四时阴阳的变化规律,不使有失,则人体就能保持相对平衡,并与天地之阴阳相互统一;知道了天人统一的道理,就可以预决死生。所以五声是和五音相应合的;五色是和五行相应合的;脉象是和阴阳相应合的。
阴气盛则梦见渡大水而恐惧;阳气盛则梦见打火烧灼;阴阳俱盛则梦见相互残杀毁伤;上部盛则梦飞腾;下部盛则梦下堕;吃的过饱的时候,就会梦见送食物给人;饥饿时就会梦见去取食物;肝气盛,则做梦好发怒气,肺气盛则做梦悲哀啼哭;腹内短虫多,则梦众人集聚;腹内长虫多则梦打架损伤。
所以诊脉是有一定方法和要求的,必须虚心静气,才能保证诊断的正确。春天的脉应该浮而在外,好象鱼浮游于水波之中;夏天的脉在肤,洪大而浮,泛泛然充满于指下,就象夏天万物生长的茂盛状态;秋天的劢处于皮肤之下,就象蛰虫将要伏藏;冬天的脉沉在骨,就象冬眠之虫闭藏不出,人们也都深居简出一样。因此说:要知道内脏的情况,可以从脉象上区别出来;要知道外部经气的情况,可从经脉循行的经络上诊察而知其终始。春、夏、秋、冬、内、外这六个方面,乃是诊脉的大法。
心脉坚而长,搏击指下,为心经邪盛,火盛气浮,当病舌卷而不能言语;其脉软而散的,当病消渴,待其胃气来复,病自痊愈。肺脉坚而长,搏击指下,为火邪犯肺,当病痰中带血;其脉软而散的,为肺脉不足,当病汗出不止,在这种情况下,不可在用发散的方法治疗。肝脉坚而长,搏击指下,其面色当青,今反不青,知其病非由内生,当为跌坠或搏击所伤,因淤血积于肋下,阻碍肺气升降,所以使人喘逆;如其脉软而散,加之面目颜色鲜泽的,当发溢饮病,溢饮病口渴暴饮,因水不化气,而水气容易流入肌肉皮肤之间、肠胃之外所引起。胃脉坚而长,搏击指下,面色赤,当病髌痛如折,如脉软而散的,则胃气不足,当病食痹,脾脉坚而长,搏击指下,面部色黄,乃脾气不运,当病少气;如其脉软而散,面色不泽,为脾虚,不能运化水湿,当病足胫浮肿如水状。肾脉坚长,搏击指下,面部黄而带赤,是心脾之邪盛侵犯于肾,肾受邪伤,当病腰痛如折;如其脉软而散者,当病精血虚少,使身体不能恢复健康。
黄帝说:诊脉时,其心脉劲急,这是什麽病?病的症状是怎样的呢?岐伯说:这种病名叫心疝,少腹部位一定有形征出现。黄帝说:这是什么道理呢?嘁伯说:心为阳脏,心与小肠为表里,今与病传于腑,小肠受之,为疝而痛,小肠居于少腹,所以少腹当有病形。黄帝说:诊察到胃脉有病,会出现什麽病变呢?岐伯说:胃脉实则邪气有余,将出现腹胀满病;胃脉虚则胃气不足,将出现泄泻病。黄帝说:疾病的形成及其发展变化又是怎样的呢?岐伯说:因于风邪,可变为寒热病;瘅热既久,可成为消中病;气逆上而不己,可成为癫痫病;风气通于肝,风邪经久不愈,木邪侮土,可成为飧泻病;风邪客于脉,留而不去则成为疠风病;疾病的发展变化是不能够数清的。黄帝说:各种痈肿、筋挛、骨痛的病变,是怎样产生的呢?岐伯说:这都是因为寒气聚集和八风邪气侵犯人体后而发生的变化。黄帝说:怎样进行治疗呢?岐伯说:由于四时偏胜之邪气所引起的病变,根据五行相胜的规律确定治则去治疗就会痊愈。
黄帝说:有旧病从五脏发动,都会影响到脉色而发生变化,怎样区别它是久病还是新病呢?岐伯说:你问的很详细啊!只要验看它脉色就可以区别开来:如脉虽小而气色不失于正常的,是为新病;如脉不失于正常而色失于正常的,乃是久病;如脉象与气色均失于正常状态的,也是久病;如脉象与面色都不失于正常的,乃是新病。脉见沉弦,是肝脉与肾脉并至,而外部没有血,或外部已见血,其经脉必滞,血气必凝,血凝经滞,形体必肿,有似乎因湿邪或水气中伤的现象,成为一种淤血肿胀。
迟脉两旁的内侧候于季胁部,外侧候于肾脏,中间候于腹部。尺肤部的中段、左臂的外侧候于肝脏,内侧候于膈部;右臂的外侧候于胃腑,内侧候于脾脏。尺肤部的上段,右臂外侧候于肺脏,内侧候于胸中;左臂外侧候于心脏,内侧候于膻中。尺肤部的前面,候身前即胸腹部;后面,候身后即背部。从尺肤上段直达鱼际处,主胸部与喉中的疾病;从尺肤部的下段直达肘横纹处,主少腹、腰、股、膝、胫、足等处的疾病。
脉象洪大的,是由于阴精不足而阳有余,故发为热中之病。脉象来时急疾而去时徐缓,这是由于上部实而下部虚,气逆于上,多好发为癫仆一类的疾病。脉象来时徐缓而去时急疾,这是由于上部虚而下部实,多好发为疠风之病。患这种病的原因,是因为阳气虚而失去捍卫的功能,所以才感受邪气而发病。有两手脉均见沉细数的,沉细为肾之脉体,数为热,故发为少阴之阳厥;如见脉沉细数散,为阴血亏损,多发为阴虚阳亢之虚劳寒热病。脉浮而散,好发为眩晕仆倒之病。凡见浮脉而不躁急,其病在阳分,则出现发热的症状,病在足三阳经;如浮而躁急的,则病在手三阳经。凡见细脉而沉,其病在阴分,发为骨节疼痛,病在手三阴经;如果脉细沉而静,其病在足三阴经。发现数动,而见一次歇止的脉象,是病在阳分,为阳热郁滞的脉象,可出现泄利或大便带脓血的疾病。诊察到各种有病的脉象而切按时,如见涩脉是阳气有余则多汗而身寒,阴气阳气均有余,则无汗而身寒。按脉浮取不见,沉取则脉沉迟不浮,是病在内而非在外,故知其心腹有积聚病。按脉沉取不显,浮取则脉浮数不沉,是病在外而不在内,当有身发热之症。凡诊脉推求于上部,只见于上部,下部脉弱的,这是上实下虚,故出现腰足清冷之症。凡诊脉推求于下部,只见于下部,而上部脉弱的,这是上虚下实,故出现头项疼痛之症。若重按至骨,而脉气少的,是生阳之气不足,故可以出现腰脊疼痛及身体痹证。
\chapter{平人气象论篇第十八}
黄帝问曰:平人何如?
岐伯对曰:人一呼脉再动,一吸脉亦再动,呼吸定息脉五动,闰以太息,命曰平人。平人者不病也。常以不病调病人,医不病,故为病人平息以调之为法。
人一呼脉一动,一吸脉一动,曰少气。人一呼脉三动,一吸脉三动而躁,尺热曰病温;尺不热脉滑曰病风;脉涩曰痹。人一呼脉四动以上曰死;脉绝不至曰死;乍疏乍数曰死。
平人之常气禀于胃,胃者平人之常气也;人无胃气曰逆,逆者死。春胃微弦曰平,弦多胃少曰肝病,但弦无胃曰死;胃而有毛曰秋病,毛甚曰今病。藏真散于肝,肝藏筋膜之气也。夏胃微钩曰平,钩多胃少曰心病,但钩无胃曰死;胃而有石曰冬病,石甚曰今病。藏真通于心,心藏血脉之气也。长夏胃微弱曰平,弱多胃少曰脾病,但代无胃曰死;软弱有石曰冬病,弱甚曰今病。藏真濡于脾,脾藏肌肉之气也。秋胃微毛曰平,毛多胃少曰肺病,但毛无胃曰死;毛而有弦曰春病,弦甚曰今病。藏真高于肺,以行荣卫阴阳也。冬胃微石曰平,石多胃少曰肾病,但石无胃曰死;石而有钩曰夏病,钩甚曰今病。藏真下于肾,肾藏骨髓之气也。
胃之大络,名曰虚里。贯鬲络肺,出于左乳下,其动应衣,脉宗气也。盛喘数绝者,则病在中;结则横,有积矣;绝不至,曰死。乳之下,其动应衣,宗气泄也。
欲知寸口太过与不及。寸口之脉中手短者,曰头痛。寸口脉中手长者,曰足胫痛。寸口脉中手促上击者,曰肩背病。寸口脉沉而坚者,曰病在中。寸口脉浮而盛者,曰病在外。寸口脉沉而弱,曰寒热及疝瘕、少腹痛。寸口脉沉而横,曰胁下有积,腹中有横积痛。寸口脉沉而喘,曰寒热。脉盛滑坚者,曰病在外。脉小实而坚者,病在内。脉小弱以涩,谓之久病。脉滑浮而疾者,谓之新病。脉急者,曰疝瘕少腹痛。脉滑曰风。脉涩曰痹。缓而滑曰热中。盛而紧曰胀。脉从阴阳,病易已;脉逆阴阳,病难已。脉得四时之顺,曰病无他;脉反四时及不间脏,曰难已。
臂多青脉,曰脱血。尺脉缓涩,谓之解惰,安卧。脉盛,谓之脱血。尺涩脉滑,谓之多汗。尺寒脉细,谓之后泄。脉尺粗常热者,谓之热中。肝见庚辛死,心见壬癸死,脾见甲乙死,肺见丙丁死,肾见戊已死,是谓真藏见皆死。颈脉动喘疾咳,曰水。目裹微肿,如卧蚕起之状,曰水。溺黄赤,安卧者,黄疸。已食如饥者,胃疸。面肿曰风。足胫肿曰水。目黄者曰黄疸。
妇人手少阴脉动甚者,妊子也。
脉有逆从四时,未有脏形,春夏而脉瘦,秋冬而脉浮大,命曰逆四时也。风热而脉静,泄而脱血脉实,病在中脉虚,病在外脉涩坚者,皆难治,命曰反四时也。
人以水谷为本,故人绝水谷则死,脉无胃气亦死。所谓无胃气者,但得真脏脉,不得胃气也。所谓脉不得胃气者,肝不弦,肾不石也。太阳脉至,洪大以长;少阳脉至,乍数乍疏,乍短乍长;阳明脉至,浮大而短。
夫平心脉来,累累如连珠,如循琅玕,曰心平,夏以胃气为本;病心脉来,喘喘连属,其中微曲,曰心病;死心脉来,前曲后居,如操带钩,曰心死。
平肺脉来,厌厌聂聂,如落榆荚,曰肺平,秋以胃气为本;病肺脉来,不上不下,如循鸡羽,曰肺病;死肺脉来,如物之浮,如风吹毛,曰肺死。
平肝脉来,弱弱招招,如揭长竿末梢,曰肝平,春以胃气为本;病肝脉来,盈实而滑,如循长竿,曰肝病;死肝脉来,急益劲,如新张弓弦,曰肝死。
平脾脉来,和柔相离,如鸡践地,曰脾平,长夏以胃气为本;病脾脉来,实而盈数,如鸡举足,曰脾病;死脾脉来,锐坚如鸟之喙,如鸟之距,如屋之漏,如水之流,曰脾死。
平肾脉来,喘喘累累如钩,按之而坚,曰肾平,冬以胃气为本;病肾脉来,如引葛,按之益坚,曰肾病;死肾脉来,发如夺索,辟辟如弹石,曰肾死。
平人气象论篇第十八参考译文
黄帝问道:正常人的脉象是怎样的呢?岐伯回答说:人一呼脉跳动两次,一吸脉也跳动两次,呼吸之余,是为定息,若一吸劢跳动五次,是因为有时呼吸较长以尽脉跳余数的缘故,这是平人的脉象。平人就是无病之人,通常以无病之人的呼吸为标准,来测候病人的呼吸至数及脉跳次数,医生无病,就可以用自己的呼吸来计算病人脉搏的至数,这是诊脉的法则。如果一呼与一吸脉各跳动三次而且急疾,尺之皮肤发热,乃是温病的表现;如尺肤不热,脉象滑,乃为感受风邪而发生的病变;如脉象涩,是为痹证。人一呼一吸脉跳动八次以上是精气衰夺的死脉;脉气断绝不至,亦是死脉;脉来忽迟忽数,为气血已乱,亦是死脉。
健康人的正气来源于胃,胃为水谷之海,乃人体气血生化之源,所以胃气为健康人之常气,人若没有胃气,就是危险的现象,甚者可造成死亡。
春天有胃气的脉应该是弦而柔和的微弦脉,乃是无并之平脉;如果弦象很明显而缺少柔和之胃气,为肝脏有病;脉见纯弦而无柔和之象的真脏脉,主死;若虽有胃气而兼见轻虚以浮的毛脉,是春见秋脉,故预测其到了秋天就要生病,如毛脉太甚,则木被金伤,现时就会发病。肝旺于春,春天脏真之气散于肝,以养筋膜,故肝藏筋膜之气。夏天有胃气的脉应该是钩而柔和的微心脏有病;脉见纯钩而无柔和之象的真脏脉,主死;若虽有胃气而兼见沉象的石脉,是夏见冬脉,故预测其到了冬天就要生病;如石脉太甚,则火被水伤,现时就会发病。心旺于夏,故夏天脏真之气通于心,心主血脉,而心之所藏则是血脉之气。长夏有胃气的脉应该是微耎弱的脉,乃是无病之平脉,如果若甚无力而缺少柔和之胃气,为脾脏有病;如果见无胃气的代脉,主死;若软弱脉中兼见沉石,是长夏见冬脉,这是火土气衰而水反侮的现象,故预测其到了冬天就要生病;如弱火甚,现时就会发病。脾旺于长夏,故长夏脏真之气濡养于脾,脾主肌肉,故脾藏肌肉之气。秋天有胃气的脉应该是轻虚以浮而柔和的微毛脉,乃是无病之平脉;如果是脉见轻虚以浮而缺少柔和之胃气,为肺脏有病;如见纯毛脉而无胃气的真脏脉,就要死亡;若毛脉中兼见弦象,这是金气衰而木反侮的现象,故预测其到了春天就要生病;如弦脉太甚,现时就会发病。肺旺于秋而居上焦,故秋季脏真之气上藏于肺,肺主气而朝百脉,营行脉中,卫行脉外,皆自肺宣布,故肺主运行营卫阴阳之气。冬天有卫气的脉应该是沉石而柔和的微石脉,乃是无病之平脉;如果脉见沉石而缺少柔和的胃气,为肾脏有病;如脉见纯石而不柔和的真脏脉,主死;若沉石脉中兼见钩脉,是水气衰而火反侮的现象,故预测其到了夏天就要生病;如钩脉太甚,现时就会发病。肾旺于冬而居人体的下焦,,冬天脏真之气下藏与肾,肾主骨,故肾藏骨髓之气。
胃经的大络,名叫虚里,其络从胃贯膈而上络于肺,其脉出现于左乳下,搏动时手可以感觉得到,这是积于胸中的宗气鼓舞其脉跳动的结果。如果虚里脉搏动急数而兼有短时中断之象,这是中气不守的现象,是病在膻中的征候;如脉来迟而有歇止兼见长而竖直位置横移的主有积滞,如脉断绝而不至,主死。如果虚里跳动甚剧而外见于衣,这是宗气失藏而外泄的现象。
切脉要知道寸口脉的太过和不及。寸口脉象应指而短,主头痛。寸口脉应指而长,主足胫痛。寸口应指急促而有力,上搏指下,主肩背痛。寸口脉沉而坚硬,主病在内。寸口脉浮而盛大,主病在外。寸口脉沉而弱,主寒热、疝少腹疼痛。寸口脉沉而横居,主胁下有积病,或腹中有横积而疼痛。寸口脉沉而急促,主病寒热。脉盛大滑而坚,主病在外。脉小实而坚,主病在内。脉小弱而涩,是为久病。脉来滑利浮而疾数,是为新病。脉来紧急,主疝瘕少腹疼痛。脉来滑利,主病风。脉来涩滞,主痹证。脉来缓而滑利,为脾胃有热,主病热中。脉来盛紧,为寒气痞满,主胀病。脉与病之阴阳想一致,如阳病见阳脉,阴病见阴脉,病易愈;脉与病之阴阳相反,如阳病见阴脉,阴病见阳脉,病难愈。脉与四时相应为顺,如春弦、夏钩、秋毛、冬石,即使患病,亦无什麽危险;如脉与四时相反,及不间脏而传变的,病难愈。臂多青脉,乃血少脉空,乃由于失血。尺肤缓而脉来涩,主气血不足,多为倦怠懈惰,但欲安卧。尺肤发热而脉象盛大,是火盛于内,主脱血。尺肤涩而脉象滑,阳气有余于内,故为多汗。尺肤寒而脉象细,阴寒之气盛于内,故为泄泻。脉见粗大而尺肤常热的,阳盛于内,为热中。
肝的真脏脉出现,至庾辛日死;心的真脏脉出现,至任癸日死;脾的真脏脉出现,至甲乙日死;肺的真脏脉出现,至丙丁日死;肾的真脏脉出现,至戊已日死。这是说的真脏脉见,均主死亡。
颈部之脉搏动甚,且气喘咳嗽,主水病。眼睑浮肿如卧蚕之状。也是水病。小便颜色黄赤,而且嗜卧,是黄疸病。饮食后很快又觉得饥饿,是胃疸病。风为阴邪,下先受之,面部浮肿,为风邪引起的风水病。水湿为阴邪,下先受之,足胫肿,是水湿引起的水肿病。眼白睛发黄,是黄疸病。妇人手少阴心脉搏动明显,是怀孕的征象。
脉与四时有相适应,也有不相适应的,如果脉搏不见本脏脉的正常脉象,春夏而不见弦、洪,而反见沉、涩;秋冬而不见毛、石,而反见浮大,这都是与四时相反的脉象。风热为阳邪脉应浮大,今反沉静;泄利脱血,津血受伤,脉应虚细,今反实大;病在内,脉应有力,乃正气尚盛足以抗邪,今反脉虚;病在外,脉应浮滑,乃邪气仍在于表,今反见脉强坚,脉证像反,都是难治之病,着就叫做“反四时”。
人依靠水谷的营养而生存,所以人断绝水谷后,就要死亡;胃气化生于水谷,如脉无胃气也要死亡。所谓无胃气的脉,就是单见真脏脉,而不见柔和的胃气脉。所谓不得胃气的脉,就是肝脉见不到微弦脉,肾脉见不到微石脉等。
太阳主时,脉来洪大而长;少阳主时,脉来不定,忽快忽慢,忽长忽短;阳明主时,脉来浮大而短。
正常的心脉来时,圆润象珠子一样,相贯而至,又象安抚琅杆美玉一样的柔滑,这是心脏的平脉。夏天以胃气为本,脉当柔和而微钩。如果脉来时,喘急促,连串急数之中,带有微曲之象,这是心的病脉。将死的心脉来时,脉前曲回,后则端直,如摸到革带之钩一样的坚硬,全无和缓之意,这是心的死脉。
正常的肺脉来时,轻虚而浮,像榆荚下落一样的轻浮和缓,这是肺的平脉。秋天以胃气为本,脉当柔和而微毛。有病的肺脉来时,不上不下,如抚摩鸡毛一样,这是肺的病脉。将死的肺脉来时,轻浮而无根,如物之漂浮,如风吹毛一样,飘忽不定,散动无根,这是肺的死脉。
正常的肝脉来时,柔软而弦长,如长竿之末梢一样的柔软摆动,这是肝的平脉。春天以胃气为本,脉当柔和而微弦。有病的肝脉来时,弦长硬满而滑利,如以手模长竿一样的长而不软,这是肝的病脉。将死的肝脉来时,弦急而坚劲,如新张弓弦一样紧绷而强劲,这是肝的死脉。
正常的脾脉来时,从容和缓,至数匀净分明,好象鸡足缓缓落地一样的轻缓而从容不迫,这是脾的平脉。长夏以胃气为本,脉当和缓。有病的脾脉来时,充实硬满而急数,如鸡举足一样急疾,这是脾的病脉。将死的脾脉来时,或锐坚而无柔和之气,如乌之嘴,鸟之爪那样坚硬而锐,或时动复止而无规律,或脉去而无不至,如屋之漏水点滴无伦,或如水之流逝,去而不返,这是脾的死脉。
正常的肾脉来时,沉石滑利连续不断而又有曲回之象,按之坚实,有如心之钩脉这是肾的平脉。冬天以胃气为本,脉当柔软而微石。有病的肾脉来时,坚搏牵连如牵引葛藤一样,愈按愈坚硬,这是肾的病脉。将死的肾脉来时,象夺索一般,长而坚硬劲急,或坚实如以指弹石,这是肾的死脉。
\chapter{玉机真藏论篇第十九}
黄帝问曰:春脉如弦,何如而弦?
岐伯对曰:春脉者肝也,东方木也,万物之所以始生也,故其气来,软弱轻虚而滑,端直以长,故曰弦,反此者病。
帝曰:何如而反?
岐伯曰:其气来实而强,此谓太过,病在外;其气来不实而微,此谓不及,病在中。
帝曰:春脉太过与不及,其病皆何如?
岐伯曰:太过则令人善忘,忽忽眩冒而巅疾;其不及,则令人胸痛引背,下则两胁胠满。
帝曰:善。夏脉如钩,何如而钩?
岐伯曰:夏脉者心也,南方火也,万物之所以盛长也,故其气来盛去衰,故曰钩,反此者病。帝曰:何如而反?岐伯曰:其气来盛去亦盛,此谓太过,病在外;其气来不盛去反盛,此谓不及,病在中。
帝曰:夏脉太过与不及,其病皆何如?岐伯曰:太过则令人身热而肤痛,为浸淫;其不及,则令人烦心,上见咳唾,下为气泄。帝曰:善。
秋脉如浮,何如而浮?岐伯曰:秋脉者肺也,西方金也,万物之所以收成也,故其气来,轻虚以浮,来急去散,故曰浮,反此者病。
帝曰:何如而反?岐伯曰:其气来毛而中央坚,两傍虚,此谓太过,病在外;其气来毛而微,此谓不及,病在中。
帝曰:秋脉太过与不及,其病皆何如?
岐伯曰:太过则令人逆气,而背痛,慢慢然;其不及,则令人喘,呼吸少气而咳,上气见血,下闻病音。
帝曰:善。冬脉如营,何如而营?岐伯曰:冬脉者肾也,北方水也,万物之所以合藏也,故其气来沉以搏,故曰营,反此者病。帝曰:何如而反?岐伯曰:其气来如弹石者,此谓太过,病在外;其去如数者,此谓不及,病在中。帝曰:冬脉太过与不及,其病皆何如?
岐伯曰:太过则令人解怠,脊脉痛而少气,不欲言;其不及则令人心悬如病饥,肋中清,脊中痛,少腹满,小便变。帝曰:善!
帝曰:四时之序,逆从之变异也,然脾脉独何主?
岐伯曰:脾脉者土也,孤脏以灌四傍者也。帝曰:然则脾善恶,可得见之乎?岐伯曰:善者不可得见,恶者可见。帝曰:恶者何如可见?岐伯曰:其来如水之流者,此谓太过,病在外;如鸟之喙者,此谓不及,病在中。帝曰:夫子言脾为孤脏,中央土以灌四傍,其太过与不及,其病皆何如?岐伯曰:太过则令人四支不举;其不及则令人九窍不通,名曰重强。帝瞿然而起,再拜稽首曰:善!吾得脉之大要。天下至数,五色脉变,揆度奇恒,道在于一。神转不回,回则不转,乃失其机。至数之要,迫近以微,著之玉版,藏之脏腑,每旦读之,名曰玉机。
五藏受气于其所生,传之于其所胜,气舍于其所生,死于其所不胜,病之且死,必先传行至其所不胜,病乃死。此言气之逆行也,故死。肝受气于心,传之于脾,气舍于肾,至肺而死。心受气于脾,传之于肺,气舍于肝,至肾而死。脾受气于肺,传之于肾,气舍于心,至肝而死。肺受气于肾,传之于肝,气舍于脾,至心而死。肾受气于肝,传之于心,气舍于肺,至脾而死。此皆逆死也。一日一夜五分之,此所占死生之早暮也。
黄帝曰:五藏相通,移皆有次。五藏有病,则各传其所胜;不治,法三月,若六月,若三日,若六日,传五藏而当死,是顺传所胜之次。故曰:别于阳者,知病从来;别于阴者,知死生之期。言知至其所困而死。
是故风者百病之长也。今风寒客于人,使人毫毛毕直,皮肤闭而为热,当是之时,可汗而发也;或痹不仁肿痛,当时之时,可汤熨及火灸刺而去之。弗治,病入舍于肺,名曰肺痹,发咳上气;弗治,肺即传而行之肝,病名曰肝痹,一名曰厥,胁痛,出食,当是之时,可按若刺耳;弗治,肝传之脾,病名曰脾风,发瘅,腹中热,烦心,出黄,当此之时,可按、可药、可浴;弗治,脾传之肾,病名曰疝瘕,少腹冤热而痛,出白,一名曰蛊,当此之时,可按、可药;弗治;肾传之心,病筋脉相引而急,病名曰瘛,当此之时,可灸、可药;弗治,满十日法当死。肾因传之心,心即复反传而行之肺,发寒热,法当三岁死,此病之次也。然其卒发者,不必治于传;或其传化有以次,不以次入者,忧恐悲喜怒,令不得以其次,故令人有大病矣。因而喜大虚,则肾气乘矣,怒则肝气乘矣,悲则肺气乘矣,恐则脾气乘矣,忧则心气乘矣,此其道也。故病有五,五五二十五变,及其传化。传,乘之名也。
大骨枯槁,大肉陷下,胸中气满,喘息不便,其气动形,期六月死,真脏脉见,乃予之期日。大骨枯槁,大肉陷下,胸中气满,喘息不便,内痛引肩项,期一月死,真脏见,乃予之期日。大骨枯槁,大肉陷下,胸中气满,喘息不便,内痛引肩项,身热,脱肉囷破,真脏见,十月之内死。大骨枯槁,大肉陷下,肩髓内消,动作益衰,真藏来见,期一岁死,见其真藏,乃予之期日。大骨枯槁,大肉陷下,胸中气满,腹内痛,心中不便,肩项身热,破囷脱肉,目眶陷,真脏见,目不见人,立死;其见人者,至其所不胜之时则死。急虚身中卒至,五脏绝闭,脉道不通,气不往来,譬于堕溺,不可为期。其脉绝不来,若人一息五六至,其形肉不脱,真脏虽不见,犹死也。
真肝脉至,中外急,如循刀刃,责责然,如按琴瑟弦,色青白不泽,毛折乃死;真心脉至,坚而搏,如循薏苡子累累然,色赤黑不泽,毛折乃死;真肺脉至,大而虚,如以毛羽中人肤,色白赤不泽,毛折乃死;真肾脉至,搏而绝,如指弹石,辟辟然,色黑黄不泽,毛折乃死;真脾脉至,弱而乍数乍疏,色黄青不泽,毛折乃死。诸真脏脉见者,皆死不治也。
黄帝曰:见真脏曰死,何也?岐伯曰:五脏者,皆禀气于胃,胃者五脏之本也;藏气者,不能自致于手太阴,必因于胃气,乃至于手太阴也。故五脏各以其时,自为而至于手太阴也。故邪气胜者,精气衰也;故病甚者,胃气不能与之俱至于手太阴,故真脏之气独见,独见者,病胜藏也,故曰死。帝曰:善。
黄帝曰:凡治病察其形气色泽,脉之盛衰,病之新故,乃治之,无后其时。形气相得,谓之可治;色泽以浮,谓之易已;脉从四时,谓之可治;脉弱以滑,是有胃气,命曰易治,取之以时。形气相失,谓之难治;色夭不泽,谓之难已;脉实以坚,谓之益甚;脉逆四时,为不可治。必察四难,而明告之。
所谓逆四时者,春得肺脉,秋得心脉,冬得肾脉,秋得心脉,冬得脾脉,其至皆悬绝沉涩者,命曰逆四时。未有脏形,于春夏而脉沉涩,秋冬而脉浮大,名曰逆四时也。
病热脉静,泄而脉大,脱血而脉实,病在中,脉实坚,病在外,脉不实坚者,皆难治。
黄帝曰:余闻虚实以决死生,愿闻其情?岐伯曰:五实死,五虚死。帝曰:愿闻五实、五虚。岐伯曰:脉盛,皮热,腹胀,前后不通、闷瞀,此谓五实。脉细,皮寒,气少,泄利前后,饮食不入,此谓五虚。帝曰:其时有生者,何也?岐伯曰:浆粥入胃,泄注止,则虚者活;身法得后利,则实者活。此其候也。
玉机真藏论篇第十九参考译文
黄帝问道:春时的脉象如弦,怎样才算弦?岐伯回答说:春脉主应肝脏,属东方之木。在这个季节里,万物开始生长,因此脉气来时,软弱轻虚而滑,端直而长,所以叫做弦,假如违反了这种现象,就是病脉。黄帝道:怎样才称反呢?岐伯说:其脉气来,应指实而有力,这叫做太过,主病在外;如脉来不实而微弱,这叫做不及,主病在里。黄帝道:春脉太过与不及,发生的病变怎样?岐伯说:太过会使人记忆力衰退,精神恍惚,头昏而两目视物眩转,而发生巅顶疾病;其不及会使人胸部作痛,牵连背部,往下则两侧胁助部位胀满。黄帝道:讲得对!
夏时的脉象如钩,怎样才算钩?岐伯说:夏脉主应心脏,属南方之火,在这个季节里,万物生长茂盛,因此脉气来时充盛,去时轻微,犹如钩之形象,所以叫做钩脉,假如违反了这种现象,就是病脉。黄帝道:怎样才称反呢?岐伯说:其脉气来盛去亦盛,这叫做太过,主病在外;如脉气来时不盛,去时反充盛有余,这叫做不及,主病在里。黄帝道:夏脉太过与不及,发生的病变怎样?岐伯说:太过会使人身体发热,皮肤痛,热邪侵淫成疮;不及会使人心虚作烦,上部出现咳嗽涎沫,下部出现失气下泄。黄帝道:讲得对!
秋天的脉象如浮,怎样才算浮?岐伯说:秋脉主应肺脏,属西方之金,在这个季节里,万物收成,因此脉气来时轻虚以浮,来急去散,所以叫做浮。假如违反了这种现象,就是病脉。黄帝道:怎样才称反呢?岐伯说:其脉气来浮软而中央坚,两旁虚,这叫做太过,主病在外;其脉气来浮软而微,这叫做不及,主病在里。黄帝道:秋脉太过于不及,发生的病变怎样?岐伯说:太过会使人气逆,背部作痛,愠愠然郁闷而不舒畅;其不及会使人呼吸短气,咳嗽气喘,其上逆而出血,喉间有喘息声音。黄帝道:讲得对!
冬时的脉象如营,怎样才算营?岐伯说:冬脉主应肾脏,属北方之水,在这个季节里,万物闭藏,因此脉气来时沉而搏手,所以叫做营。假如违反了这种现象,就是病脉。黄帝道:怎样才称反呢?岐伯说:其脉来如弹石一般坚硬,这叫做太过,主病在外;如脉去虚数,这叫做不及,主病在里。黄帝道:冬脉太过与不及,发生的病变怎样?岐伯说:太过会使人精神不振,身体懈怠,脊骨疼痛,气短,懒于说话;不及则使人心如悬,如同腹中饥饿之状,季胁下空软部位清冷,脊骨作痛,少腹胀满,小便变常。黄帝道:讲得对!
黄帝道:春夏秋冬四时的脉象,有逆有从,其变化各异,但独未论及脾脉,究竟脾脉主何时令?岐伯说:脾脉属土,位居中央为孤脏,以灌溉四旁。黄帝道:脾脉的正常与异常可以得见吗?岐伯说:正常的脾脉不可能见到,有病的脾脉是可以见到的。黄帝道:有病的脾脉怎样?岐伯说:其来如水之流散,这叫做太过,主病在外;其来尖锐如鸟之喙,这叫做不及,主病在中。黄帝道:先生说脾为孤脏,位居中央属土,以灌溉四旁,他的太过和不及各发生什麽病变?岐伯说:太过会使人四肢不能举动,不及则使人九窍不通,名叫重强。黄帝惊悟书肃然起立,敬个礼道:很好!我懂得诊脉的要领了,这是天下极其重要的道理。《五色》、《脉变》、《揆度》、《奇恒》等书,阐述的道理都是一致的,总的精神在于一个“神”字。神的功用运转不息,向前而不能回却,倘若回而不转,就失掉它的生机了。极其重要的道理,往往迹象不显而尽于微妙,把它著录在玉版上面,藏于枢要内府,每天早上诵读,称它为《玉机》。
五脏疾病的传变,是受病气于其所生之脏,传于其所胜之脏,病气留舍于生我之脏,死于我所不胜之脏。当病到将要死的时候,必先传行于相克之脏,病者乃死。这是病气的逆传,所以会死亡。例如,肝受病气于心脏,而又传行于脾脏,其病气留舍于肾脏,传到肺脏而死。心受病气于脾脏,传行于肺脏,病气留舍于肝脏,传到肾脏而死。脾受气于肺脏,传行于肾脏,病气留舍于心脏,传到肝脏而死。肺受病气于肾脏,传行于肝脏,病气留舍于脾脏,传到心脏而死。肾受气于肝脏,传行于内脏,病气留舍于肺脏,传到脾脏而死。凡此都是病气之逆传,所以死。以一日一夜划分为五个阶段,分属五脏,就可以推测死候的早晚时间。
黄帝道:五脏是相通连的,病气的转移,都有一定的次序。假如五脏有病,则各传其所胜;若不能掌握治病的时机,那麽三个月或六个月,或三天,或六天,传遍五脏就当死了,这是相克的顺传次序。所以说:能辨别三阳的,可以知道病从何经而来;能辨别三阴的,可以知道病的死生日期,这就是说,知道他至其所不胜而死。
风为六淫之首,所以说它是百病之长。风寒中人,使人毫毛直竖,皮肤闭而发热,在这个时候,可用发汗的方法治疗;至风寒入于经络,发生麻痹不仁或肿痛等症状,此时可用汤熨(热敷)及火罐、艾炙、针刺等方法来祛散。如果不及时治疗,病气内传于肺,叫做肺痹,又叫做肝厥,发生胁痛、吐食的症状,在这个时候,可用按摩或者针刺的方法;如不及时治疗,就会传行于脾,叫做脾风,发生黄,腹中热,烦心,小便黄色等症状,在这个时候,可用按摩、药物或热汤沐浴等方法;如再不治,就会传行于肾,叫做瘕,少腹烦热疼痛,小便色白而混浊,又叫做盅病,在这个时候,可用按摩、或用药物;如再不治,病就由肾传心,发生筋脉牵引拘挛,叫做瘛病,在这个时候,可用炙法,或用药物;如再不治,十日之后,当要死亡。倘若病邪由肾传心,心又复反传于肺脏,发为寒热,发当三日即死,这是疾病传行的一般次序。假如骤然爆发的病,就不必根据这个相传的次序而治。有些病不依这个次序传变的,如忧、恐、悲、喜、怒情志之病,病邪就不能依照这个次序相传,因而使人生大病了。如因喜极伤心,心虚则肾气相乘;或因大怒,则肝气乘脾;或因悲伤,则肺气乘肝;或因惊恐,则肾气虚,脾气乘肾;或因大忧,则肺气内虚,心气乘肺。这是五志激动,使病邪不以次序传变的道理。所以病虽有五,及其传化,就有五五二十五变。所谓传化,就是相乘的名称。
大骨软弱,大肉瘦削,胸中气满,呼吸困难,呼吸困难,呼吸时身体振动,为期六个月就要死亡。见了真脏脉,就可以预知死日,大骨软弱,大肉瘦削,胸中气满,呼吸困难,呼吸困难,胸中疼痛,牵引肩项,为期一个月就要死亡,见了真脏脉,就可以预知死日,大骨软弱,大肉瘦削,胸中气满,呼吸困难,胸中疼痛,上引肩项,全身发热,脱肉破腘,真脏脉现,十个月之内就要死亡。大骨软弱,大肉瘦削,两肩下垂,骨髓内消,动作衰颓,真脏脉未出现,为期一年死亡,若见到真脏脉,就可以预知死日。大骨软弱,大肉瘦削,胸中气满,腹中痛,心中气郁不舒,肩项身上俱热,破腘脱肉,目眶下陷,真脏脉出现,精脱目不见人,立即死亡;如尚能见人,是精未全脱,到了它所不声胜之时,便死亡了。如果正气暴虚,外邪陡然中人,仓卒获病,五脏气机闭塞,周身脉道不通,气不往来,譬如从高堕下,或落水淹溺一样,猝然的病变,就无法预测死期了。其脉息绝而不至,或跳动异常疾数,一呼脉来五、六至,虽然形肉不脱,真脏不见,仍然要死亡的。
肝脏之真脏脉至,中外劲急,如象按在刀口上一样的锋利,或如按在琴弦上一样硬直,面部显青白颜色而不润泽,毫毛枯焦,就要死亡。心脏的真脏脉至,坚硬而搏手,如循薏苡子那样短而圆实,面部显赤黑颜色而不润泽,毫毛枯焦乃死。肺脏的真脏脉至,大而空虚,好象毛羽着人皮肤一般地轻虚,面部显白赤颜色而不润泽,毫毛枯焦,就要死亡。肾脏的真脏脉至,搏手若索欲断,或如以指弹石一样坚实,面部显黑黄颜色而不润泽,毫毛枯焦,就要死亡。脾脏的真脏脉至,软弱无力,快慢不匀,面部显黄青颜色而不润泽,毫毛枯焦,就要死亡。凡是见到五脏真脏脉,皆为不治的死侯。
黄帝道:见到真脏脉象,就要死亡,是什麽道理?岐伯说:五脏的营养,都赖于胃腑水谷之精微,因此胃是五脏的根本。故五脏之脏脉气,不能自行到达于手太阴寸口,必须赖借胃气的敷布,才能达于手太阴。所以五脏之气能够在其所主之时,出现于手太阴寸口,就是有了胃气。如果邪气胜,必定使精气衰。所以病气严重时,胃气就不能与五脏之气一起到达手太阴,而为某一脏真脏脉象单独出现,真脏独见,是邪气胜而脏气伤,所以说是要死亡的。黄帝道:讲得对!
黄帝道:大凡治病,必先诊察形体盛衰,气之强弱,色之润枯,脉之虚实,病之新久,然后及时治疗,不能错过时机。病人形七相称,是可治之症;面色光润鲜明,病亦易愈;脉搏与四时相适应,亦为可治;脉来弱而流利,是有胃气的现象,病亦易治,必须抓紧时间,进行治疗。形气不相称,此谓难治;面色枯槁,没有光泽,病亦难愈;脉实而坚,病必加重;脉与四时相逆,为不可治。必须审察这四种难治之证,清楚地告诉病家。
所谓脉与四时相逆,是春见到肺脉,夏见到肾脉,秋见到心脉,冬见到脾脉,其脉皆悬绝无根,或沉涩不起,这就叫做逆四时。如五脏脉气不能随着时令表现于外,在春夏的时令,反见沉涩的脉象,秋冬的时令,反见浮大的脉象,这也叫做逆四时。
热病脉宜洪大而反静;泄泻脉应小而反大;脱血脉应虚而反实;病在中而脉不实坚;病在外而脉反坚实。这些都是症脉相反,皆为难治。
黄帝道:我听说根据虚实的病情可以预决死生,希望告诉我其中道理!岐伯说:五实死,五虚亦死。黄帝道:请问什麽叫做五实、五虚?岐伯说:脉盛是心受邪盛,皮热是肺受邪盛,腹胀是脾受邪盛,二便不通是肾受邪盛,闷瞀是肝受邪盛,这叫做五实。脉细是心气不足,皮寒是肺气不足,气少是肝气不足,泄利前后是肾气不足,饮食不入是脾气不足,这叫做五虚。黄帝道:五实、五虚,有时亦有痊愈的,又是什麽道理?岐伯说:能够吃些粥浆,慢慢地胃气恢复,大便泄泻停止,则虚者也可以痊愈。如若原来身热无汗的,而现在得汗,原来二便不通的,而现在大小便通利了,则实者也可以痊愈。这就是五虚、五实能够痊愈的机转。
\chapter{三部九候论篇第二十}
黄帝问曰:余闻九针于夫子,众多博大,不可胜数。余愿闻要道,以属子孙,传之后世,著之骨髓,藏之肝肺,歃血而受,不敢妄泄,令合天道,必有终始,上应天光星辰历纪,下副四时五行。贵践更立,冬阴夏阳,以人应之奈何?愿闻其方。
岐伯对曰:妙乎哉问也!此天地之至数。
帝曰:愿闻天地之至数,合于人形血气,通决死生,为之奈何?
岐伯曰:天地之至数,始于一,终于九焉。一者天,二者地,三者人。因而三之,三三者九,以应九野。故人有三部,部有三候,以决死生,以处百病,以调虚实,而除邪疾。
帝曰:何谓三部?
岐伯曰:有下部,有中部,有上部;部各有三候,三候者,有天有地有人也。必指而导之,乃以为真。上部天,两额之动脉;上部地,两颊之动脉;上部人,耳前之动脉;中部天,手太阴也;中部地,手阳明也;中部人,手少阴也;下部天,足厥阴也;下部地,足少阴也;下部人,足太阴也。故下部之天以候肝,地以候肾,人以候脾胃之气。
帝曰:中部之候奈何?
岐伯曰:亦有天,亦有地,亦有人。天以候肺,地以候胸中之气,人以候心。
帝曰:上部以何候之?
岐伯曰:亦有天,亦有地,亦有人。天以候头角之气,地以候口齿之气,人以候耳目之气。三部者,各有天,各有地,各有人;三而成天,三而成地,三而成人。三而三之,合则为九。九分为九野,九野为九脏;故神脏五,形脏四,合为九脏。五脏已败,其色必夭,夭必死矣。
帝曰:以候奈何?
岐伯曰:必先度其形之肥瘦,以调其气之虚实,实则泻之,虚则补之。必先去其血脉,而后调之,无问其病,以平为期。
帝曰:决死生奈何?
岐伯曰:形盛脉细,少气不足以息者危;形瘦脉大,胸中多气者死。形气相得者生;参伍不调者病;三部九候皆失者死;上下左右之脉相应如参舂者,病甚;上下左右相失不可数者死;中部之候虽独调,与众脏相失者死;中部之候减者死;目内陷者死。
帝曰:何以知病之所在?
岐伯曰:察九候独小者病,独大者病,独疾者病,独迟者病,独热者病,独寒者病,独陷下者病。以左手足上,上去踝五寸按之,庶右手足当踝而弹之,其应过五寸以上,蠕蠕然者,不病;其应疾,中手浑浑然者病;中手徐徐然者病;其应上不能至五寸,弹之不应者死。是以脱肉、身不去者死。中部乍疏乍数者死。其脉代而钩者,病在络脉。九候之相应也,上下若一,不得相失。一候后则病;二候后则病甚;三候后则病危。所谓后者,应不俱也,察其腑脏,以知死生之期。必先知经脉,然后知病脉,真脏脉见者,胜死。足太阳气绝者,其足不可屈伸,死必戴眼。
帝曰:冬阴夏阳奈何?
岐伯曰:九候之脉,皆沉细悬绝者为阴,主冬,故以夜半死;盛躁喘数者为阳,主夏,故以日中死。是故寒热病者,以平旦死;热中及热病者,以日中死;病风者,以日夕死;病水者,以夜半死;其脉疏乍数、乍迟乍疾者,日乘四季死;形肉已脱,九修虽调,犹死;七诊虽见,九候皆从者,不死。所言不死者,风气之病及经月之病,似七诊之病而非也,故言不死。若有七诊之病,其脉候亦败者死矣,必发哕噫。必审问其所始病,与今之所方病,而后各切循其脉,视其经络浮沉,以上下逆从循之。其脉疾者,不病;其脉迟者病;脉不往来者死;皮肤著者死。
帝曰:其可治者奈何?
岐伯曰:经病者,治其经;孙络病者,治其孙络血;血病身有痛者,治其经络。其病者在奇邪,奇邪之脉,则缪刺之。留瘦不移,节而刺之。上实下虚,切而从之,索其结络脉,刺出其血,以见通之。瞳子高者,太阳不足。戴眼者,太阳已绝。此决死生之要,不可不察也。手指及手外踝上五指留针。
三部九候论篇第二十参考译文
黄帝问道:我听先生讲了九针道理后,觉得丰富广博,不可尽述。我想了解其中的主要道理,以嘱咐子孙,传于后世,铭心刻骨,永志不忘,并严守誓言,不敢妄泄。如何使这些道理符合于天体运行的规律,有始有终,上应于日月星辰周历天度之标志,下符合四时五行阴阳盛衰的变化,人是怎样适应这些自然规律的呢?希望你讲解这方面的道理。岐伯回答说:问得多好啊!这是天地间至为深奥的道理。
黄帝道:我愿闻天地的至数,与人的形体气血相通,以决断死生,是怎样一回事?岐伯说:天地的至数,开始于一,终止于九。一奇数为阳,代表天,二偶数为阴代表地,人生天地之间,故以三代表人;天地人合而为三,三三为九,以应九野之数。所以人有三部,每部各有三候,可以用它来决断死生,处理百病,从而调治虚实,祛除病邪。
黄帝道:什麽叫做三部呢?岐伯说:有下部,有中部,有上部。每部各有三候,所谓三候,是以天、地、人来代表的。必须有老师的当面指导,方能懂得部候准确之处。上部天,即两额太阳脉处动脉;上部地,即两颊大迎穴处动脉;上部人,即耳前耳门穴处动脉;中部天,即两手太阴气口、经渠穴处动脉;中部地,即两手阴明经合谷处动脉;中部人,即两手少阴经神门处动脉;下部天,即足厥阴经五里穴或太冲穴处动脉;下部地,即足少阴经太溪穴处动脉;下部人,即足太阴经箕门穴处动脉。故而下部之可以天候肝脏之病变,下部之地可以候肾脏之病变,下部之人可以候脾胃之病变。
黄帝道:中部之侯怎样?岐伯说:中部亦有天、地、人三侯。中部之天可以候肺脏之病变,肿不之地可以候胸中之病变。中部之人可以候心脏之病变。黄帝道:上部之侯又怎样?岐伯说:上布也有天、地、人三候。上部之天可以候头角之病变,上部之地可以候口齿之病变,上部之人可以候耳目之病变。三部之中,各有天、各有地、各有人。三侯为天,三侯为地,三侯为人,三三相乘,合为九侯。脉之九候,以应地之九野,以应人之九脏。所以人有肝、肺、心、脾、肾五神脏和膀胱、胃、大肠、小肠四形脏,合为九脏。若五脏以败,必见神色枯槁,枯槁者是病情危重,乃至死亡征象。
黄帝道:诊察的方法怎样?岐伯说:必先度量病人的身形肥瘦,了解它的正气虚实,实证用泻法,虚症用补法。但必先去除血脉中的凝滞,而后调补气血的不足,不论治疗什麽病都是以达到气血平调为准则。
黄帝道:怎样决断死生?岐伯说:形体盛,脉反细,气短,呼吸困难,危险;如形体瘦弱,脉反大,胸中喘满而多气的是死亡之症。一般而论;形体与脉一致的主生;若脉来三五不调者主病,三部九候之脉与疾病完全不相适应的,主死;上下左右之脉,相应鼓指如春杵捣谷,参差不齐,病必严重;若见上下之脉相差甚大,而又息数错乱不可计数的,是死亡征候;中部之脉虽然独自调匀,而与其他众脏不相协调的,也是死候;目内陷的为正气衰竭现象,也是死候。
黄帝道:怎样知道病的部位呢?岐伯说:从诊察九候脉的异常变化,就能知病变部位。九候之中,有一部独小,或独大,或独疾,或独迟,或独热,或独寒,或独陷下(沉伏),均是有病的现象。
以左手加于病人的左足上,距离内踝五寸处按着,以右手指在病人足内踝上弹之,医者之左手即有振动的感觉,如其振动的范围超过五寸以上,蠕蠕而动,为正常现象;如其振动急剧而大,应手快速而浑乱不清的,为病态;若振动微弱,应手迟缓,应为病态;如若振动不能上及五寸,用较大的力量弹之,仍没有反应,是为死候。
身体极度消瘦,体弱不能行动,是死亡之征。中部之脉或快或慢,无规律,为气脉败乱之兆,亦为死征。如脉代而钩,为病在络脉。九侯之脉,应相互适应,上下如一,不应该有参差。如九候之中有一候不一致,就是病态;二候不一致,则病重;三候不一致,则病必危险。所谓不一致,就是九侯之间,脉动的不相适应。诊察病邪所在之脏腑,以知死生的时间。临症诊察,必先知道正常之脉,然后才能知道有病之脉;若见到真脉脉象,胜己的时间,便要死亡。足太阳经脉气绝,则两足不能屈伸,死亡之时,必目睛上视。
黄帝道:冬为阴,夏为阳,脉象与之相应如何?岐伯说:九候的脉象,都是沉细悬绝的,为阴,冬令死于阴气极盛之夜半;如脉盛大躁动喘而疾数的,为阳,主夏令,所以死于阳气旺盛之日中;寒热交作的病,死于阴阳交会的平旦之时;热中及热病,死于日中阳极之时;病风死于傍晚阳衰之时;病水死于夜半阴极之时。其脉象忽疏忽数,忽迟忽急,乃脾气内绝,死于辰戌丑未之时,,也就是平旦、日中、日夕、夜半、日乘四季的时候;若形坏肉脱,虽九候协调,犹是死亡的征象;假使七诊之脉虽然出现,而九候都顺于四时的,就不一定是死候。所说不死的病,指心感风病,或月经之病,虽见类似七诊之病脉,而实不相同,所以说不是死候。若七诊出现、其脉候有败坏现象的,这是死征,死的时候,必发呃逆等证候。所以治病之时,必须详细询问他的起病情形和现在症状,然后按各部分,切其脉搏,以观察其经络的浮沉,以及上下逆顺。如其脉来流利的,不病;脉来迟缓的,是病;脉不往来的,是死候;久病肉脱,皮肤干枯着于筋骨的,亦是死候。
黄帝道:那些可治的病,应怎样治疗呢?岐伯说:病在经的,刺其经;病在孙络的,刺其孙络使它出血;血病而有身痛症状的,则治其经与络。若病邪留在大络,则用右病刺左、左病刺右的缪刺法治之。若邪气久留不移,当于四肢八溪之间、骨节交会之处刺之。上实下虚,当切按气脉,而探索气脉络郁结的所在,刺出其血,以通其气。如目上视的,是太阳经气不足。目上视而又定直不动的,是太阳经气已绝。这是判断死生的要诀,不可不认真研究。

\chapter{经脉别论篇第二十一}

黄帝问曰:人之居处、动静、勇怯,脉亦为之变乎?岐伯曰:凡人之惊恐恚劳动静,皆为变也。是以夜行则喘出于肾,淫气病肺;有所堕恐,喘出于肝,淫气害脾;有所惊恐,喘出于肺,淫气伤心;度水跌仆,喘出于肾与骨。当是之时,勇者气行则已;怯者则着而为病也。故曰:诊病之道,观人勇怯、骨肉、皮肤,能知其情,以为诊法也。
故饮食饱甚,汗出于胃;惊而夺精,汗出于心;持重远行,汗出于肾;疾走恐惧,汗出于肝;摇体劳苦,汗出于脾。故春秋冬夏,四时阴阳,生病起于过用,此为常也。
食气入胃,散精于肝,淫气于筋。食气入胃,浊气归心,淫精于脉;脉气流经,经气归于肺,肺朝百脉,输精于皮毛;毛脉合精,行气于腑;腑精神明,留于四脏,气归于权衡;权衡以平,气口成寸,以决死生。
饮入于胃,游溢精气,上输于脾;脾气散精,上归于肺;通调水道,下输膀胱;水精四布,五经并行,合于四时五脏阴阳,揆度以为常也。
太阳脏独至,厥喘虚气逆,是阴不足、阳有余也,表里当俱泻,取之下俞。阳明脏独至,是阳气重并也,当泻阳补阴,取之下俞。少阳脏独至,是厥气也,足乔前卒大,取之下俞。少阳独至者,一阳之过也。太阴脏搏者,用心省真,五脉气少,胃气不平,三阴也,宜治其下俞,补阳泻阴。一阳独啸,少阳厥也,阳并于上,四脉争张,气归于肾,宜治其经络,泻阳补阴。一阴至,厥阴之治也,真虚痛心,厥气留薄,发为白汗,调食和药,治在下俞。
帝曰:太阳脏何象?岐伯曰:象三阳而浮也。帝曰:少阳脏何象?岐伯曰:象一阳也。一阳脏者,滑而不实也。
帝曰:阳阴脏何象?
岐伯曰:象大浮也。太阳脏搏,言伏鼓也;二阴搏至,肾沉不浮也。
经脉别论篇第二十一参考译文
黄帝问道:人们的居住环境、活动、安静、勇敢、怯懦有所不同,其经脉血气也随着变化吗?岐伯回答说:人在惊恐、忿怒、劳累、活动或安静的情况下,静脉血气都要受到影响而发生变化。所以夜间远行劳累,就会扰动肾气,使肾气不能闭藏而外泄,则气喘出于肾脏,其偏胜之气,就会侵犯肺脏。若因坠堕而受到恐吓,就会扰动肝气,而喘出于肝,其偏胜之气就会侵犯脾脏。或有所惊恐,惊则神越气乱,扰动肺气,喘出于肺,其偏胜之气就会侵犯心脏。渡水而跌仆,跌仆伤骨,肾主骨,水湿之气通于肾,致肾气和骨气受到扰动,气喘出于肾和骨。在这种情况下,身体强盛的人,气血畅行,不会出现什麽病变;怯弱的人,气血留滞,就会发生病变。所以说:诊察疾病,观察病人的勇怯及骨骼、肌肉、皮肤的变化,便能了解病情,并以此作为诊病的方法。在饮食过饱的时候,则食气蒸发而汗出于胃。惊则神气浮越,则心气受伤而汗出于心。负重而远行的时候,则骨劳气越,肾气受伤而汗出于肾。疾走而恐惧的时候,由于疾走伤筋,恐惧伤魂,则肝气受伤而汗出于肝。劳力过度的时候,由于脾主肌肉四肢,则脾气受伤而汗出于脾。春、夏、秋、冬四季阴阳的变化都有其常度,人在这些变化中所发生疾病,就是因为对身体的劳用过度所致,这是通常的道理。
五谷入胃,其所化生的一部分精微之气输散到肝脏,再由肝将此精微之气滋养于筋。五谷入胃,其所化生的精微之气,注入于心,再由心将此精气滋养于血脉。血气流行在经脉之中,到达于肺,肺又将血气输送到全身百脉中去,最后把精气输送到皮毛。皮毛和经脉的精气汇合,又还流归入于脉,脉中精微之气,通过不断变化,周流于四脏。这些正常的生理活动,都要取决于气血阴阳的平衡。气血阴阳平衡,则表现在气口的脉搏变化上,气口的脉搏,可以判断疾病的死生。水液入胃以后,游溢布散其精气,上行输送与脾,经脾对精微的布散转输,上归于肺,肺主清肃而司治节,肺气运行,通调水道,下输于膀胱。如此则水精四布,外而布散于皮毛,内而灌输于五脏之经脉,并能合于四时寒暑的变易和五脏阴阳的变化。作出适当的调节,这就是经脉的正常生理现象。
太阳经脉偏盛,则发生厥逆、喘息、虚气上逆等症状,这是阴不足而阳有余,表里两经俱当用泻法,取足太阳经的束骨穴和足少阴经的太溪穴。阳明经脉偏盛,是太阳、少阳之气重并于阳明,当用泻阳补阴的治疗方法,当泻足阳明经的陷谷穴,补太阴经的太白穴。少阳经脉偏盛,是厥气上逆,所以阳蹻脉前的少阳劢猝然盛大,当取足少阳经的临泣穴。少阳经脉偏盛而独至,就是少阳太过。太阴经脉鼓搏有力,应当细心的审查是否真脏脉至,若五脏之脉均气少,胃气又不平和,这是足太阴脾太过的缘过,应当用补阳泻阴的治疗方法,补足阳明之陷谷穴,泻足太阴之太白穴。二阴经脉独盛,是少阴厥气上逆,而阳气并越于上,心、肝、脾、肺四脏受其影响,四脏之脉争张于外,病的根源在于肾,应治其表里的经络,泻足太阳经的经穴昆仑、络穴飞扬,补足少阴的经穴复溜,络穴大钟。一阴经脉偏盛,是厥阴所主,出现真气虚弱,心中痠痛不适的症状,厥气留于经脉与正气相搏而发为白汗,应该注意饮食调养和药物的治疗,如用针刺,当取决阴经下部的太冲穴,以泄其邪。
黄帝说:太阳经的脉象是怎样的呢?岐伯说:其脉象似三阳之气浮盛于外,所以脉浮。黄帝说:少阳经的脉象是怎样的呢?岐伯说:其脉象似一阳之初生,滑而不实。黄帝说:阳明经的脉象是怎样的呢?岐伯说:其脉象大而浮。太阴经的脉象搏动,虽沉伏而指下仍搏击有力;少阴经的脉象搏动,是沉而不浮。

\chapter{藏气法时论篇}

黄帝问曰:合人形以法四时五行而治,何如而从?何如而逆?得失之意,愿闻其事。
岐伯对曰:五行者,金、木、水、火、土也,更贵更贱,以知死生,以决成败,而定五脏之气、间甚之时、死生之期也。
帝曰:愿卒闻之。
岐伯曰:肝主春,足厥阴、少阳主治,其日甲乙;肝苦急,急食甘以缓之。心主夏,手少阴、太阳主治,其日丙丁;心苦缓,急食酸以收之。
脾主长夏足太阴、阳明主治,其日戊己;脾苦湿,急食苦以燥之。
肺主秋,手太阴、阳明主治,其日庚辛;肺苦气上逆,急食苦以泄之。
肾主冬,足少阴、太阳主治,其日壬癸;肾苦燥,急食辛以润之。开腠理,致津液,通气也。
病在肝,愈于夏;夏于愈,甚于秋;秋不死,持于冬,起于春,禁当风。肝病者,愈在丙丁;丙丁不愈,加于庚辛;庚辛不死,持于壬癸,起于甲乙。肝病者,平旦慧,下晡甚,夜半静。且欲散,急食辛以散之,用辛补之,酸泻之。
病在心,愈在长夏;长夏不愈,甚于冬;冬不死,持于春,起于夏,禁温食热衣。心病者,愈在戊己,戊己不愈,加于壬癸;壬癸不死,持于甲乙,起于丙丁。心病者,日中慧,夜半甚,平旦静。心欲弱,急食咸以弱之,用咸补之,甘泻之。
病在脾,愈在秋;秋不愈,甚于春;春不死,持于夏,起于长夏,禁温食饱食、湿地濡衣。脾病者,愈在庚辛;庚辛不愈,加于甲乙;甲乙不死,持于丙丁,起于戊己。脾病者,日昳慧,日出甚,下晡静。脾欲缓,急食甘以缓之,用苦泻之,甘补之。
病在肺,愈在冬;冬不愈,甚于夏;夏不死,持于长夏,起于秋,禁寒饮食寒衣。肺病者,愈在壬癸;壬癸不愈,加于丙丁;丙丁不死,持于戊己,起于庚辛。肺病者,下哺慧,日中甚,夜半静。肺欲收,急食酸以收之,用酸补之,辛泻之。
病在肾,愈在春;春不愈,甚于长夏;长夏不死,持于秋,起于冬,禁犯焠燱!热食温炙衣。肾病者,愈在甲乙;甲乙不愈,甚于戊己;戊己不死,持于庚辛,起于壬癸。肾病者,夜半慧,四季甚,下晡静。肾欲坚,急食苦以坚之,用苦补之,咸泻之。
夫邪气之客于身也,以胜相加,至其所生而愈,至其所不胜而甚,至于所生而持,自得其位而起。必先定五脏之脉,乃可言间甚之时,死生之期也。
肝病者,两胁下痛引少腹,令人善怒;虚则目疎疎无所见,耳无所闻,善恐,如人将捕之。取其经,厥阴与少阳。气逆则头痛,耳聋不聪,颊肿,取血者。
心病者,胸中痛,胁支满,胁下痛,膺背肩甲间痛,两臂内痛;虚则胸腹大,胁下与腰相引而痛,取其经,少阴、太阳、舌下血者。其变病,刺郄中血者。
脾病者,身重,善肌,肉痿,足不收行,善瘛,脚下痛;虚则痛满肠鸣,飧泄食不化。取其经,太阴、阳明、少阴血者。
肺病者,喘咳逆气,肩背痛,汗出,尻阴股膝、髀足皆痛;虚则少气不能报息,耳聋嗌干。取其经,太阴、足太阳之外厥阴内血者。肾病者,腹大胫肿,喘咳身重,寝汗出,憎风;虚则胸中痛,大腹、小腹痛,清厥,意不乐。取其经,少阴、太阳血者。
肝色青,宜食甘,粳米、牛肉、枣、葵皆甘。心色赤,宜食酸,小豆、犬肉、李、韭皆酸。肺色白,宜食苦,麦、羊肉、杏、薤皆苦。脾色黄,宜食咸,大豆、豕肉、栗、藿皆咸。肾色黑,宜食辛,黄黍、鸡肉、桃、葱皆辛。辛散、酸收、甘缓、苦坚、咸软。毒药攻邪,五谷为养,五果为助,五畜为益,五菜为充,气味合而服之,以补精益气。此五者,有辛、酸、甘、苦、咸,各有所利,或散、或收、或缓、或急、或坚、或软,四时五脏,病随五味所宜也。
脏气法时论篇第二十二参考译文
黄帝问道:结合人体五脏之气的具体情况,取法四时五行的生克制化规律,作为救治疾病的法则,怎样是从?怎样是逆呢?我想了解治法中的从逆和得失是怎麽一会事。岐伯回答说:五行就是金、木、水、火、土,配合时令气候,有衰旺盛克的变化,从这些变化中可以测知疾病的死生,分析医疗的成败,并能确定五脏之气的盛衰、疾病轻重的时间,以及死生的日期。
黄帝说:我想听你详尽地讲一讲。岐伯说:肝属木、旺于春,肝与胆为表里,春天是足厥阴肝和足少阳胆主治的时间,甲乙属木,足少阳胆主甲木,足厥阴肝主乙木,所以肝胆旺日为甲乙;肝在志为怒,怒则气急,甘味能缓急,故宜急食甘以缓之。心属火,旺于夏,心与小肠为表里,夏天是手少阴和手太阳小肠主治的时间;丙丁属火,手少阴心主丁火,手太阳小肠主丙火,所以心与小肠的旺日为丙丁;心在志为喜,喜则气缓,心气过缓则心气虚而散,酸味能收敛,故宜急食酸以收之。脾属土,旺于长夏(六月),脾与胃为表里,长夏是足太阴脾和足阳明胃主治的时间;戊已属土,主太阴脾主已土,主阳明胃主戊土,所以脾与胃的旺日为戊已;脾性恶湿,湿盛则伤脾,苦味能燥湿,故宜急食苦以燥之。肺属金,旺于秋;肺与大肠为表里,秋天是手太阴肺和手阳明大肠主治的时间;庚辛属金,手太阴肺主辛金,手阳明大肠主庚金,所以肺与大肠的旺日为庚辛;肺主气,其性清肃,若气上逆则肺病,苦味能泄,故宜急食苦以泄之。肾属水,旺于冬,肾与膀胱为表里,冬天是足少阴肾与足太阴膀胱主治的时间;壬癸属水,足少阴肾主癸水,足太阳膀胱主壬水,所以肾与膀胱的旺日为壬癸;肾为水脏,喜润而恶燥,故宜急食辛以润之。如此可以开发腠理,运行津液,宜通五脏之气。
肝脏有病,在夏季当愈,若至夏季不愈,到秋季病情就要加重;如秋季不死,至冬季病情就会维持稳定不变状态,到来年春季,病即好转。因风气通于肝,故肝并最禁忌受风。有肝病的人,愈于丙丁日;如果丙丁日不愈,到庚辛日病就加重;如果庚辛日不死,到壬癸日病情就会维持稳定不变状态,到了甲乙日病即好转。患肝病的人,在早晨的时候精神清爽,傍晚的时候病就加重。到半夜时便安静下来。肝木性喜条达而恶抑郁,故肝病急用辛味以散之,若需要补以辛味补之,若需要泻,以酸味泻之。
心脏有病,愈于长夏;若至长夏不愈,到了冬季病情就会加重;如果在冬季不死,到了明年的春季病情就会维持稳定不变状态,到了夏季病即好转。心有病的人应禁忌温热食物,衣服也不能穿的太暖。有心病的人,愈于戊已日;如果戊已日不愈,到壬癸日病就加重;如果在壬癸日不死,到甲已日病情就会维持稳定不变状态,到丙丁日病即好转。心脏有病的人,在中午的时候神情爽慧,半夜时病就加重,早晨时便安静了。心病欲柔软,宜急食咸味以软之,需要补则以咸味补之,以肝味泻之。
脾脏有病,愈于秋季;若至秋季不愈,到春季病就加重;如果在春季不死,到夏季病情就会维持稳定不变状态,到长夏的时间病即好转。脾病应禁忌吃温热性食物即饮食过饱、居湿地、穿湿衣等。脾有病的人,愈于庚辛日;如果在庚辛日不愈,到甲已日加重;如果在甲已日不死,到丙丁日病情就会维持稳定不变状态,到了戊已日病即好转。脾有病的人,在午后的时间精神清爽,日出时病就加重,傍晚是便安静了。脾脏病需要缓和,甘能缓中,故宜急食甘味以缓之,需要泻则用苦味药泻脾,以甘味补脾。
肺脏有病,愈于冬季;若至冬季不愈,到夏季病就加重;如果在夏季不死,至长夏时病情就会维持稳定不变状态,到了秋季病即好转。肺有病应禁忌寒冷饮食及穿的太单薄。肺有病的人,愈于壬癸日;如果在壬癸日不愈,到丙丁日病就加重;如果在丙丁日不死,到戊已日病情就会维持稳定不变状态,到了庚辛日,病即好转。肺有病的人,傍晚的时候精神爽慧,到中午时病就加重,到半夜时变安静了。肺气欲收敛,宜急食酸味以收敛,需要补的,用酸味补肺,需要泻的,用辛味泻肺。
肾脏有病,愈于春季;若至春季不愈,到长夏时病就加重;如果在长夏不死,到秋季病情就会维持稳定不变状态,到冬季病即好转。肾病禁食炙过热的食物和穿经火烘烤过的衣服。肾有病的人,愈于甲已日;如果在甲已日不愈,到戊已日病就加重;如果在戊已日不死,到庚辛日病情就会维持稳定不变状态,到壬癸日病即好转。肾有病的人,在半夜的时候精神爽慧,在一日当中辰、戌、丑、未四个时辰病情加重,在傍晚时便安静了。肾主必藏,其气欲坚,需要补的,宜急食苦味以坚之,用苦味补之,需要泻的,用咸味泻之。
凡是邪气侵袭人体,都是以胜相加,病至其所生之时而愈,至其所不胜之时而甚,至其所生之时而病情稳定不变,至其自旺之时病情好转。但必须先明确五脏之平脉,然后始能推测疾病的轻重时间及死生的日期。
肝脏有病,则两肋下疼痛牵引少腹,使人多怒,这是肝气实的症状;如果肝气虚,则出现两目昏花而视物不明,两耳也听不见声音,多恐惧,好象有人要逮捕他一样。治疗时,取用厥阴肝经和少阳胆经的经穴。如肝气上逆,则头痛、耳聋而听觉失灵、颊肿,应取厥阴、少阳经脉,刺出其血。
心脏有病,则出现胸中痛,肋部支撑胀满,肋下痛,胸膺部、背部及肩胛间疼痛,两臂内侧疼痛,这是心实的症状。心虚,则出现胸腹部胀大,肋下和腰部牵引作痛。治疗时,取少阴心经和太阳小肠经的经穴,并刺舌下之脉以出其血。如病情有变化,与初起不同,刺委中穴出血。
脾脏有病,则出现身体沉重,易饥,肌肉痿软无力,两足弛缓不收,行走时容易抽搐,脚下疼痛,这是脾实的症状;脾虚则腹部胀满,肠鸣,泄下而食物不化。治疗时,取太阴脾经、阳明胃经和少阴肾经的经穴,刺出其血。
肺脏有病,则喘咳气逆,肩背部疼痛,出汗,尻、阴、股、膝、髀骨、足等部皆疼痛,这是肺实的症状;如果肺虚,就出现少气,呼吸困难而难于接续,耳聋,咽干。治疗时,取太阴肺经的经穴,更取足太阳经的外侧及足厥阴内侧,即少阴肾经的经穴,刺出其血。
肾脏有病,则腹部胀大,胫部浮肿,气喘,咳嗽,身体沉重,睡后出汗,恶风,这是肾实的症状;如果肾虚,就会出现胸中疼痛,大腹和小腹疼痛,四肢厥冷,心中不乐。治疗时,取足少阴肾经和足太阳膀胱经的经穴,刺出其血。
肝合青色,宜食甘味,粳米、牛肉、枣、葵菜都是属于味甘的。心合赤色,宜食酸味,小豆、犬肉、李、韭都是属于酸味的。肺合白色,宜食苦味,小麦、羊肉、杏、薤都是属于苦味的。脾合黄色,宜食咸味,大豆、猪肉、栗、藿都是属于咸味的。肾合黑色,宜食辛味,黄黍、鸡肉、桃、葱都是属于辛味的。五味的功用:辛味能发散,酸味能收敛,甘味能缓急,苦味能坚燥,咸味能坚。凡毒药都是可用来攻逐病邪,五谷用以充养五脏之气,五果帮助五谷以营养人体,五畜用以补益五脏,五菜用以充养脏腑,气味和合而服食,可以补益精气。这五类食物,各有辛、酸、甘、苦、咸的不同气味,各有利于某一脏气,或散,或收,或缓,或急,或坚等,在运用的时候,要根据春、夏、秋、冬四时和五脏之气的偏盛偏衰及苦欲等具体情况,各随其所宜而用之。

\chapter{宣明五气篇第二十三}

五味所入:酸入肝,辛入肺,苦入心,咸入肾,甘入脾,是谓五入。
五气所病:心为噫,肺为咳,肝为语,脾为吞,肾为欠、为嚏,胃为气逆、为哕、为恐,大肠、小肠为泄,下焦溢为水,膀胱不利为癃,不约为遗溺,胆为怒,是谓五病。
五精所并:精气并于心则喜,并于肺则悲,并于肝则忧,并于脾则畏,并于肾则恐,是谓五并,虚而相并者也。
五脏所恶:心恶热,肺恶寒,肝恶风,脾恶湿,肾恶燥,是为谓恶。
五脏化液:心为汗,肺为涕,肝为泪,脾为涎,肾为唾,是为五液。
五味所禁:辛走气,气病无多食辛;咸走血,血病无多食咸;苦走骨,骨病无多食苦;甘走肉,肉病无多食甘;酸走筋,筋病无多食酸,是谓五禁,无令多食。
五病所发:阴病发于骨,阳病发于血,阴病发于肉,阳病发于冬,阴病发于夏,是谓五发。
五邪所乱:邪入于阳则狂,邪入于阴则痹,搏阳则为巅疾,搏阴则为,阳入阴则静,阴出之阳则怒,是谓五乱。
五邪所见:春得秋脉,夏得冬脉,长夏得春脉,秋得夏脉,冬得长夏脉,名曰阴出之阳,病善怒,不治。是谓五邪,皆同命,死不治。
五脏所藏:心藏神,肺藏魄,肝藏魄,脾藏意,肾藏志,是谓五脏所藏。
五脏所主:心主脉,肺主皮,肝主筋,脾主肉,肾主骨,是谓五主。
五带所伤:久视伤血,久卧蓖气,久坐伤肉,久立伤骨,久行伤筋,是谓五带所伤。
五脉应象:肝脉弦,心脉钩,脾脉代,肺脉毛,肾脉石,是谓五脏之脉。
宣明五气篇第二十三参考译文
五脏之气失调后所发生的病变:心气失调则嗳气;肺气失调则咳嗽;肝气失调则多言;脾气失调则吞酸;肾气失调则为呵欠、喷嚏;胃气失调则为气逆为哕,或有恐惧感;大肠、小肠病则不能泌别清浊,传送糟粕,而为泄泻;下焦不能通调水道,则水液泛溢与皮肤而为水肿;膀胱之气化不利,则为癃闭,不能约制,则为遗尿;胆气失调则易发怒。这是五脏之气失调而发生的病变。
五脏之精气相并所发生的疾病:精气并与心则喜,精气并于肺则悲,精气并于肝则忧,精气并于脾则畏,精气并于肾则恐。这就是所说的五并,都是由于五脏乘虚相并所致。
五脏化生的液体:心之液化为汗,肺之液化为涕,肝之液化为泪,脾之液化为涎,肾之液化为唾。这是五脏化生的五液。
五味所禁:辛味走气,气病不可多食辛味;咸味走血,血病不可多食咸味;苦味走骨,骨病不可多食苦味;甜味走肉,肉病不可多食甜味;酸味走筋,筋病不可多食酸味。这就是五味的禁忌,不可使之多食。
五种病的发生:阴病发生于骨,阳病发生于血,阴病发生于肉,养病发生于冬,阴病发生于夏。这是五病所发。
五邪所乱:邪入于阳分,则阳偏盛,而发为狂病;邪入于阴分,则阴偏盛,而发为痹病;邪搏于阳则阳气受伤,而发为癫疾;邪搏于阴侧则阴气受伤,而发为音哑之疾;邪由阳而入于阴,则从阴而为静;邪由阴而出于阳,则从阳而为怒。这就是所谓五乱。
五脏克贼之邪所表现的脉象:春天见到秋天的毛脉,是金克木;夏天见到冬天的石脉,是水克火;长夏见到春天的弦脉,是木克土;秋天见到夏天的洪脉,是火克金;冬天见到长夏的濡缓脉,是土克水。这就是所谓的五邪脉。其预后相同,都属于不治的死证。
五种过度的疲劳可以伤耗五脏的精气:如久视则劳于精气而伤血,久卧则阳气不伸而伤气,久坐则血脉灌输不畅而伤肉,久立则劳于肾及腰、膝、胫等而伤骨,久行则劳于筋脉而伤筋。这就是五劳所伤。
五脏应四时的脉象:肝脏应春,端直而长,其脉象弦;心脉应夏,来盛去衰,其脉象钩;脾旺于长夏,其脉弱,随长夏而更代;肺脉应秋,轻虚而浮,其脉象毛;肾脉应冬,其脉沉坚象石。这就是所谓的应于四时的五脏平脉。

\chapter{血气形志篇第二十四}

夫人之常数,太阳常多血少气,少阳常少血多气,阳明常多气多血,少阴常少血多气,厥阴常多血少气,太阴常多气少血。此天之常数。
足太阳少阴为表里,少阳与厥阴为表里,阳明与太阴为表里,是为足阴阳也。手太阳与少阴为表里,少阳与心主为表里,阳明与太阴为表里,是为手之阴阳也。今知手足阴阳所苦。凡治病必先去其血,乃去其所苦,伺之所欲,然后泻有余,补不足。
欲知背俞,先度其两乳间,中折之,更以他草度去半已,即以两隅相拄也,乃举以度其背,令其一隅居上,齐脊大椎,两隅在下,当其下隅者,肺之俞也;复下一度,心之俞也;复下一度,左角肝之俞也,右角脾之俞;复下一度,肾之俞也。是谓五脏之俞,灸刺之度也。
形乐志苦,病生于脉,治之以灸刺;形乐志乐,病生于肉,治之以针石;形苦志乐,病生于筋,治之以熨引;形苦志苦,病生于咽嗌,治之以百药;形数惊恐,经络不通,病生于不仁,治之以按摩醪药。是谓五形志也。
刺阳明,出血气;刺太阳,出血恶气;刺少阳,出气恶血;刺太阴,出气恶血;刺少阴,出气恶血;刺厥阴,出血恶气也。
血气形志篇第二十四参考译文
人身各经气血多少,是有一定常数的。如太阳经常多血少气,少阳经常少血多气,阳明经常多气多血,少阴经常少血多气,厥阴经常多血少气,太阴经常多气少血,这是先天禀赋之常数。
足太阳膀胱经与足少阴肾经为表里,足少阳胆经与足厥阴肝经为表里,足阳明胃经与足太阴脾经为表里。这是足三阳经和足三阴经之间的表里配合关系。手太阳小肠经和手太阴心经为表里,手少阳三焦经与手厥阴心包经为表里,手阳明大肠经与手太阴肺经为表里,这是手三阳经和手三阴经之间的表里配合关系。现已知道,疾病发生在手足阴阳使二经脉的那一经,其治疗方法,血脉雍盛的,必须先刺出其血,以减轻其病苦;再诊察其所欲,根据病情的虚实,然后泻其有余之实邪,补其不足之虚。
要想知道背部五脏俞穴的位置,先用草一根,度量两乳之间的距离,再从正中对折,另一草与前草同样长度,折掉一半之后,拿来支撑第一根草的两头,就成了一个三角形,然后用它量病人的背部,使其一个角朝上,和脊背部大椎穴相平,另外两个角在下,其下边左右两个角所指部位,就是肺俞穴所在。再把上角移下一度,方在两肺俞连线的中点,则其下左右两角的位置是心俞的部位。再移下一度,左角是肝俞,右角是脾俞。再移下一度,左右两角是肾俞。这就是五脏俞穴的部位,为刺炙取穴的法度。
形体安逸但精神苦闷的人,病多发生在经脉,治疗时宜用针炙。形体安逸而精神也愉快的人,病多发生在肌肉,治疗时宜用针刺或砭石。形体劳苦但精神很愉快的人,病多发生在筋,治疗时宜用热熨或导引法。形体劳苦,而精神又很苦恼的人,病多发生在咽喉部,治疗时宜用药物。屡受惊恐的人,经络因气机紊乱而不通畅,病多为麻木不仁,治疗时宜用按摩和药酒。以上是形体和精神方面发生的五种类型的疾病。
刺阳明经,可以出血出气;刺太阳经,可以出血,而不宜伤气;刺少阳经,只宜出气,不宜出血;刺太阳经,只宜出气,不宜出血;刺少阴经,只宜出气,不宜出血;刺厥阴经,只宜出血,不宜伤气。

\chapter{宝命全形论篇第二十五}

黄帝问曰:天覆地载,万物悉备,莫贵于人。人以天地之气生,四时之法成,君王众庶,尽欲全形,形之疾病,莫知其情,留淫日深,著于骨髓,心私虑之。余欲针除其疾病,为之奈何?
岐伯曰:夫盐之味咸者,其气令器津泄;弦绝者,其音嘶败;木敷者,其叶发;病深者,其声哕。人有此三者,是为坏府,毒药无治,短针无取,此皆绝皮伤肉,血气争黑。
帝曰:余念其痛,心为之乱惑,反甚其病,不可更代,百姓闻之,以为残贼,为之奈何?
岐伯曰:夫人生于地,悬命于天,天地合气,命之曰人。人能应四时者,天地为之父母;知万物者,谓之天子。天有阴阳,人有十二节;天有寒暑,人有虚实。能经天地阴阳之化者,不失四时;知十二节之理者,圣智不能欺也;能存八动之变,五胜更立,能达虚实之数者,独出独入,呿吟至微,秋毫在目。
帝曰:人生有形,不离阴阳,天地合气,别为九野,分为四时,月有大小,日有短长,万物并至,不可胜量,虚实呿吟,敢问其方?
岐伯曰:木得金而伐,火得水而灭,土得木而达,金得火而缺,水得土而绝。万物尽然,不可胜竭。做针有悬布天下者五,黔首共余食,莫知之也。
一曰治神,二曰知养身,三曰知毒药为真,四曰制砭石小大,五曰知府藏血气之诊。五法俱立,各有所先。今末世之刺也,虚者实之,满者泄之,此皆众工所共知也。若夫法天则地,随应而动,和之者若响,随之者若影,道无鬼神,独来独往。
帝曰:愿闻其道。岐伯曰:凡刺之真,必先治神,五脏已定,九候已备,后乃存针;众脉不见,众凶弗闻,外内相得,无以形先,可玩往来,乃施于人。人有虚实,五虚勿近,五实勿远,至其当发,间不容瞬。手动若务,针耀而匀,静意视义,观适之变。是谓冥冥,莫知其形,见其乌乌,见其稷稷,从见其飞,不知其谁,伏如横弩,起如发机。
帝曰:何如而虚?何如而实?岐伯曰:刺虚者须其实,刺实者须其虚;经气已至,慎守勿失。深浅在志,远近若一,如临深渊,手如握虎,神无营于众物。
宝命全形论篇第二十五参考译文
黄帝问道:天地之间,万物俱备,没有一样东西比人更宝贵了。人依靠天地之大气和水谷之精气生存,并随着四时生长收藏的规律而生活着,上至君主,下至平民,任何人都愿意保全形体的健康,但是往往有了病,却因病轻而难于察知,让病邪稽留,逐渐发展,日益深沉,乃至深入骨髓,我为之甚感忧虑。我要想解除他们的痛苦,应该怎样办才好?岐伯回答说:比如盐味是咸的,当贮藏在器具中的时候,看到渗出水来,这就是盐气外泄;比如琴弦将要断的时候,就会发出嘶败的声音;内部已溃的树木,其枝叶好象很繁茂,实际上外盛中空,极容易萎谢;人在疾病深重的时候,就会产生呃逆。人要是有了这样的现象,说明内脏已有严重破坏,药物和针炙都失去治疗作用,因为皮肤肌肉受伤败坏,血气枯槁,就很难挽回了。
黄帝道:我很同情病人的痛苦,但思想上有些慌乱疑惑,因治疗不当反使病势加重,又没有更好的方法来替代,人们看起来,将要认为我残忍粗暴,究竟怎麽好呢?岐伯说:一个人的生活,和自然界是密切相关联的。人能适应四时变迁,则自然界的一切,都成为他生命的泉源。能够知道万物生长收藏的道理的人,就有条件承受和运用万物。所以天有阴阳,人有十二经脉;天有寒暑,人有虚实盛衰。能够顺应天地阴阳的变化,不违背四时的规律,了解十二经脉的道理,就能明达事理,不会被疾病现象弄糊涂了。掌握八风的演变,五行的衰旺,通达病人虚实的变化,就一定能有独到的见解,哪怕病人的呵欠呻吟极微小的动态,也能够明察秋毫,洞明底细。
黄帝道:人生而有形体,离不开阴阳的变化,天地二气相合,从经纬上来讲,可以分为九野,从气候上来讲,可以分为四时,月行有小大,日行有短长,这都是阴阳消长变化的体现。天地间万物的生长变化更是不可胜数,根据患者微细呵欠及呻吟,就能判断出疾病的虚实变化。请问运用什麽方法,能够提纲挈领,来加以认识和处理呢?岐伯说:可根据五行变化的道理来分析:木遇到金,就能折伐;火受到水,就能熄灭;土被木殖,就能疏松;金遇到火,就能熔化;水遇到土,就能遏止。这种变化,万物都是一样,不胜枚举。所以用针刺来治疗疾病,能够嘉惠天下人民的,有五大关键,但人们都弃余不顾,不懂得这些道理。所谓五大关键:一是要精神专一,二是要了解养身之道,三是要熟悉药物真正的性能,四要注意制取砭石的大小,五是要懂得脏腑血气的诊断方法。能够懂得这五项要道,就可以掌握缓急先后。近世运用针刺,一般的用补法治虚,泻法制满,这是大家都知道的。若能按照天地阴阳的道理,随机应变,那末疗效就能更好,如响之应,如影随形,医学的道理并没有什麽神秘,只要懂得这些道理,就能运用自如了。
黄帝道:希望听你讲讲用针刺的道理。岐伯说:凡用折的关键,必先集中思想,了解五脏的虚实,三部九侯脉象的变化,然后下针。还要注意有没有真脏脉出现,五脏有无败绝现象,外形与内脏是否协调,不能单独以外形为依据,更要熟悉经脉血气往来的情况,才可施针于病人。病人有虚实之分,见到五虚,不可草率下针治疗,见到五实,不可轻易放弃针刺治疗,应该要掌握针刺的时机,不然在瞬息之间就会错过机会。针刺时手的动作要专一协调,针要洁净而均匀,平心静意,看适当的时间,好象鸟一样集合,气盛之时,好象稷一样繁茂。气之往来,正如见鸟之飞翔,而无从捉摸他形迹的起落。所以用针之法,当气未至的时候,应该留针侯气,正如横弩之待发,气应的时候,则当迅速起针,正如弩箭之疾出。
黄帝道:怎样治疗虚症?怎样治疗实症?岐伯说:刺虚症,须用补法,刺实症,须用泻法;当针下感到经气至,则应慎重掌握,不失时机地运用补泻方法。针刺无论深浅,全在灵活掌握,取穴无论远近,侯针取气的道理是一致的,针刺时都必须精神专一,好象面临万丈深渊,小心谨慎,又好象手中捉着猛虎那样坚定有力,全神贯注,不为其他事物所分心。

\chapter{八正神明论篇第二十六}
黄帝问曰:用针之服,必有法则焉,今何法何则?
岐伯对曰:法天则地,合以天光。
帝曰:愿卒闻之。
岐伯曰:凡刺之法,必候日月星辰,四时八正之气,气定乃刺之。是故天温日明,则人血淖液,而卫气浮,故血易泻,气易行;天寒日阴,则人血凝泣,而卫气沉。月始生,则血气始精,卫气始行;月郭满,则血气实,肌肉坚;月郭空,则肌肉减,经络虚,卫气去,形独居。是以因天时而调血气也。是以天寒无刺,天温无疑,月生无泻,月满无补,月郭空无治。是谓得时而调之。因天之序,盛虚之时,移光定位,正立而待之。故曰:月生而泻,是谓脏虚;月满而补,血气扬溢,络有留血,命曰重实:月郭空而治,是谓乱经。阴阳相错,真邪不别,沉以留止,外虚内乱,淫邪乃起。
帝曰:星辰八正何候?
岐伯曰:星辰者,所以制日月之行也。八正者,所以候八风之虚邪,以时至者也。四时者,所以分春秋冬夏之气所在,以时调之也,八正之虚邪而避之勿犯也。以身之虚而逢天之虚,两虚相感,其气至骨,入则伤五脏。工候救之,弗能伤也。故曰:天忌不可不知也。
帝曰:善!
其法星辰者,余闻之矣,愿闻法往古者。
岐伯曰:法往古者,先知《针经》也。验于来今者,先知日之寒温,月之虚盛,以候气之浮沉,而调之于身,观其立有验也。观其冥冥者,言形气荣卫之不形于外,而工独知之,以日之寒温,月之虚盛,四时气之浮沉,参伍相合而调之,工常先见之,然而不形于外,故曰观于冥冥焉。通于无穷者,可以传于后世也,是故工之所以异也。然而不形见于外,故俱不能见也。视之无形,尝之无味,故谓冥冥,若神仿佛。
虚邪者,八正之虚邪气也。正邪者,身形若用力,汗出腠理开,逢虚风。其中人也微,故莫知其情,莫见其形。上工救其萌牙,必先见三部九候之气,尽调不败而救之,故曰上工。下工救其已成,救其已败。救其已成者,言不知三部九候之相失,因病而败之也。知其所在者,知诊三部九候之病脉处而治之,故曰守其门户焉,莫知其情,而见邪形也。
帝曰:余闻补泻,未得其意。
岐伯曰:泻必用方。方者,以气方盛也,以月方满也,以日方温也,以身方定也,以息方吸而内针,乃复候其方吸而转针,乃复候其方呼而徐引针。故曰泻必用方,其气而行焉。补必用员。员者,行也;行者,移也,刺必中其荣,复以吸排针也。故员与方,非针也。故养神者,必知形之肥瘦,荣卫血气之盛衰。血气者,人之神,不可不谨养。
帝曰:妙乎哉论也!合人形于阴阳四时,虚实之应,冥冥之期,其非夫子,孰能通之!然夫子数言形与神,何谓形?何谓神?愿卒闻之。
岐伯曰:请言形。形乎形,目冥冥,问其所病,索之于经,慧然在前,按之不得,不知其情,故曰形。
帝曰:何谓神?岐伯曰:请言神。神乎神,耳不闻,目明心开而志先,慧然独悟,口弗能言,俱视独见,适若错,昭然独明,若风吹云,故曰神。三部九候为之原,九针之论,不必存也。
八正神明论篇第二十六参考译文
黄帝问道:用针的技术,必然有他一定的方法准则,究竟有什麽方法,什麽准则呢?岐伯回答说:要在一切自然现象的演变中去体会。黄帝道:愿详尽的了解一下。岐伯说:凡针刺之法,必须观察日月星辰盈亏消长及四时八正之气候变化,方可运用针刺方法。所以气候温和,日色晴朗时,则人的血液流行滑润,而卫气浮于表,血容易泻,气容易行;气候寒冷,天气阴霾,则人的血行也滞涩不畅,而卫气沉于里。月亮初生的时候,血气开始流利,卫气开始畅行;月正圆的时候,则人体血气充实,肌肉坚实;月黑无光的时候,肌肉减弱,经络空虚,卫气衰减,形体独居。所以要顺着天时而调血气。因此天气寒冷,不要针刺;天气温和,不要迟缓;月亮初生的时候,不可用泻法;月亮正圆的时候,不可用补法;月黑无光的时候,不要针刺。这就是所谓顺着天时而调治气血的法则。因天体运行有一定顺序,故月亮有盈亏盛虚,观察日影的长短,可以定四时八正之气。所以说:月牙初生时而泻,就会使内脏虚弱;月正圆时而补,使血气充溢于表,以致络脉中血液留滞,这叫做重实;月黑无光的时候用针刺,就会扰乱经气,叫做乱经。这样的治法必然引起阴阳相错,真气与邪气不分,使病变反而深入,致卫外的阳气虚竭,内守的阴气紊乱,淫邪就要发生了。
黄帝道:星辰八正观察些什麽?岐伯说:观察星辰的方位,可以定出日月循行的度数。观察八节常气的交替,可以测出异常八方之风,是什麽时候来的,是怎样为害于人的。观察四时,可以分别春夏秋冬正常气候之所在,以便随时序来调养,可以避免八方不正之气候,不受其侵犯。假如虚弱的体质,再遭受自然界虚邪贼风的侵袭,两虚相感,邪气就可以侵犯筋骨,再深入一步,就可以伤害五脏。懂得气候变化治病的医生,就能及时挽救病人,不至于受到严重的伤害。所以说天时的宜忌,不可不知。黄帝道:讲得好!
关于取法于星辰的道理,我已经知道了,希望你讲讲怎样效法于前人?岐伯说:要取法和运用前人的学术,先要懂得《针经》。要想把古人的经验验证于现在,必先要知道日之寒温,月之盈亏,四时气候的浮沉,而用以调治于病人,就可以看到这种方法是确实有效的。所谓观察其冥冥,就是说荣卫气血的变化虽不显露于外,而医生却能懂得,他从日之寒温,月之盈亏,四时气候之浮沉等,进行综合分析,做出判断,然后进行调治。因此医生对于疾病,每有先见之明,然而疾病并未显露于外,所以说这是观察于冥冥。能够运用这种方法,通达各种事理,他的经验就可以流传于后世,这是学识经验丰富的医生不同于一般人的地方。然而病情是不显露在表面,所以一般人都不容易发现,看不到形迹,尝不出味道,所以叫做冥冥,好象神灵一般。
虚邪,就是四时八节的虚邪贼风。正邪,就是人在劳累时汗出腠理开,偶而遭受虚风。正邪伤人轻微,没有明显的感觉,也无明显病状表现,所以一般医生观察不出病情。技术高明的医生,在疾病初起,三部九侯之脉气都调和而未败坏之时,就给以早期救治,所以称为“上工”。“下工”临证,是要等疾病已经形成,甚或至于恶化阶段,才进行治疗。所以说下工要等到病成阶段才能治疗,是因为不懂得三部九侯的相得相失,致使疾病发展而恶化了。要明了疾病之所在,必须从三部九侯的脉象中详细诊察,知道疾病的变化,才能进行早期治疗。所以说掌握三部九侯,好象看守门户一样的重要,虽然外表尚未见到病情,而医者已经知道疾病的形迹了。
黄帝道;我听说:针刺有部泻二法,不懂得它的意义。岐伯说:泻法必须掌握一个“方”字。所谓“方”,就是正气方盛,月亮方满,天气方温和,身心方稳定的时候,并且要在病人吸气的时候进针,再等到他吸气的时候转针,还要等他呼气的时候慢慢的拔出针来。所以说泻必用方,才能发挥泻的作用,使邪气泻去而正气运行。补法必须掌握一个“圆”字。所谓“圆”,就是行气。行气就是导移其气以至病所,刺必要中其荥穴,还要在病人吸气时拔针。所谓“圆”与“方”,并不是指针的形状。一个技术高超有修养的医生,必须明了病人形体的肥瘦,营卫血气的盛衰。因为血气是人之神的物质基础,不可不谨慎的保养。
黄帝道:多麽奥妙的论述啊!把人身变化和阴阳四时虚实联系起来,这是非常微妙的结合,要不是先生,谁能够弄得懂呢!然而先生屡次说道形如神,究竟什麽叫形?什麽叫神?请你详尽的讲一讲。岐伯说:请让我先讲形。所谓形,就是反映于外的体征,体表只能察之概况,但只要问明发病的原因,再仔细诊察经脉变化,则病情就清楚的摆在面前,要是按寻之仍不可得,那麽便不容易知道他的病情了,因外部有形迹可察,所以叫做形。黄帝道:什麽叫神?岐伯说:请让我再讲神。所谓神,就是望而知之,耳朵虽然没有听到病人的主诉,但通过望诊,眼中就明了它的变化,亦已心中有数,先得出这一疾病的概念,这种心领神会的速度独悟,不能用言语来形容,有如观察一个东西,大家没有看到,但他能运用望诊,就能够独自看到,有如在黑暗之中,大家都很昏黑,但他能运用望诊,就能够昭然独明,好象风吹云散,所以叫做神,诊病时,若以三不九侯为之本原,就不必拘守九针的理论了。

\chapter{离合真邪论篇第二十七}

黄帝问曰:余闻九针九篇,夫子乃因而九之,九九八十一篇,余尽通其意矣。经言气之盛衰,左右倾移,以上调下,以左调右,有余不足,补泻于荥输,余知之矣。此皆荣卫之倾移,虚实之所生,非邪气从外入于经也。余愿闻邪气之在经也,其病人何如?取之奈何?
岐伯对曰:夫圣人之起度数,必应于天地。故天有宿度,地有经水,人有经脉。天地温和,则经水安静;天寒地冷,则经水凝泣;天暑地热,则经水沸溢;卒风暴起,则经水波涌而陇起。夫邪之入于脉也,寒则血凝泣,暑则气淖泽,虚邪因而入客,亦如经水之得风也,经之动脉,其至也亦时陇起。其行于脉中循循然,其至寸口中手也,时大时小,大则邪至,小则平,其行无常处,在阴与阳,不可为度,从而察之,三部九候,卒然逢之,早遏其路。吸则内针,无令气忤;静以久留,无令邪布;吸则转针,以得气为故;候呼引针,呼尽乃去。大气皆出,故命曰泻。
帝曰:不足者补之奈何?
岐伯曰:必先扪而循之,切而散之,推而按之,弹而怒之,抓而下之,通而取之,外引其门,以闭其神。呼尽内针,静以久留,以气至为故。如待所贵,不待日暮,其气以至,适而自护,候吸引针,气不得出;各在其处,推阖其门,令神气存,大气留止,故命曰补。
帝曰:候气奈何?
岐伯曰:夫邪去络入于经也,舍于血脉之中,其寒温未相得,如涌波之起也,时来时去,故不常在。故曰方其来也,必按而止之,止而取之,无逢其冲而泻之。真气者,经气也。经气太虚,故曰其来不可逢,此之谓也。故曰候邪不审,大气已过,泻之则真气脱,脱则不复,邪气复至,而病益蓄。故曰其往不可追,此之谓也。不可挂以发者,待邪之至时,而发针泻矣,若先若后者,血气已尽,其病不可下。故曰知其可取如发机,不知其取如扣椎。故曰知机道者不可挂以发,不知机者扣之不发,此之谓也。
帝曰:补泻奈何?
岐伯曰:此攻邪也。疾出以去盛血,而复其真气,此邪新客,溶溶未有定处也,推之则前,引之则止,逆而刺之,温血也,刺出其血,其病立已。
帝曰:善!然真邪以合,波陇不起,候之奈何?
岐伯曰:审扪循三部九候之盛虚而调之。察其左右上下相失及相减者,审其病脏以期之。不知三部者,阴阳不别,天地不分,地以候地,天以候天,人以候人,调之中府,以定三部。故曰:刺不知三部九候病脉之处,虽有大过且至,工不能禁也。诛罚无过,命曰大惑,反乱大经,真不可复,用实为虚,以邪为真,用针无义,反为气贼,夺人正气,以从为逆,荣卫散乱,真气已失,邪独内著,绝人长命,予人天殃。不知三部九候,故不能久长;因不知合之四时五行,因加相胜,释邪攻正,绝人长命。邪之新客来也,未有定处,推之则前,引之则止,逢而泻之,其病立已。
离合真邪论篇第二十七参考译文
黄帝问道:我听说九针有九篇文章,而先生又从九篇上加以发挥,演绎成为九九八十一篇,我已经完全领会它的精神了。《针经》上说的气之盛衰,左右偏胜,取上以调下,去左以调右,有余不足,在荥输之间进行补泻,我亦懂得了。这些变化,都是由于容卫的偏胜、气血虚实而形成的,并不是邪气从侵入经脉而发生的病变。我现在希望知道邪气侵入经脉之时,病人的症状怎样?又怎样来治疗?岐伯回答说:一个有修养的医生,在制定治疗法则时,必定体察于自然的变化。如天有宿度,地有江河,人有经脉,其间是互相影响,可以比类而论的。如天地之气温和,则江河之水安静平稳;天气寒冷,则水冰地冻,江河之水凝涩不流;天气酷热,则江河之水沸腾扬溢;要是暴风骤起,则使江河之水,波涛汹涌。因此病邪侵入了经脉,寒则使血行滞涩,热则使血气滑润流利,要是虚邪贼风的侵入,也就象江河之水遇到暴风一样,经脉的搏动,则出现波涌隆起的现象。虽然血气同样依次在经脉中流动,但在寸口处按脉,指下就感到时大时小,大即表示病邪盛,小即表示病邪退,邪气运行,没有一定的位置,或在阴经或在阳经,就应该更进一步,用三部就侯的方法检查,一旦察之邪气所在,应急早治疗,以阻止它的发展。治疗时应在吸气时进针,进针时勿使气逆,进针后要留针静侯其气,不让病邪扩散;当吸气时转捻其针,以得气为目的;然后等病人呼气的时候,慢慢地起针,呼气尽时,将针取出。这样,大邪之气尽随针外泄,所以叫做泻。
黄帝道:不足之虚症怎样用补法?岐伯说:首先用手抚摸穴位,然后以指按压穴位,再用手指揉按周围肌肤,进而用手指弹其穴位,令脉络怒张,左手按闭孔穴,不让正气外泄。进针方法,是在病人呼气将尽时进针,静侯其气,稍久留针,以得气为目的。进针侯气,要象等待贵客一样,忘掉时间的早晚,当得气时,要好好保护,等病人吸气时候,拔出其针,那末气就不至外出了;出针以后,应在其孔穴上揉按,使针孔关闭,真气存内,大经之气留于营卫而不泄,这便叫做补。
黄帝道:对邪气怎样诊侯呢?岐伯说:当邪气从络脉而进入经脉,留舍于血脉之中,这时邪正相争,或寒或温,真邪尚未相合,所以脉气波动,忽起忽伏,时来时去,无有定处。所以说诊得泄气方来,必须按而止之,阻止它的发展,用针泻之,但不要正当邪气冲突,遂用泻法,因为真气,就是经脉之气,邪气冲突,真气大虚,这时而用泻法,反使经气大虚,所以说气虚的时候不可用泻,就是指此而言。因此,诊侯邪气而不能审慎,当大邪之气已经过去,而用泻法,则反使真气虚脱,真气虚脱,则不能恢复,而邪气益甚,那病更加重了。所以说,邪气已经随经而去,不可再用泻法,就是指此而言。阻止邪气,使用泻法,是间不容发的事,须待邪气初到的时候,随即下针去泻,在邪至之前,或在邪去之后用泻法,都是不适时的,非但不能去邪,反使血气受伤,病就不容易退了。所以说,懂得用针的,象拨动弩机一样,机智灵活,不善于用针的,就象敲击木椎,顽钝不灵了。所以说,识得机宜的,一霎那时毫不迟疑,不知机宜的,纵然时机已到,亦不会下针,就是指此而言。
黄帝道:怎样进行补泻呢?岐伯说:应以攻邪为主。应该及时刺出盛血,以恢复正气,因为病邪刚刚侵入,流动未有定处,推之则前进,引之则留止,迎其气而泻之,以出其毒血,血出之后,病就立即会好。黄帝道:讲得好!假如到了病邪和真气并合以后,脉气不现波动,那麽怎样诊察呢?岐伯说:仔细审察三部九候的盛衰虚实而调治。检查的方法,在它左右上下各部分,观察有无不相称或特别减弱的地方,就可以知道病在那一脏腑,待其气至而刺之。假如不懂得三部九侯,则阴阳不能辨别,上下也不能分清,更不知道从上部脉以诊察下,从上部脉以诊察上,从中部脉以诊察中,结合胃气多少有无来决定疾病在那一部。所以说,针刺而不知三部九侯以了解病脉之处,则虽然有大邪为害,这个医生也没有办法来加以事先防止的。如果诛罚无过,不当泻而泻之,这就叫做“大惑”,反而扰乱脏腑经脉,使真气不能恢复,把实症当作虚症,邪气当作真气,用针毫无道理,反助邪气为害,剥夺病人正气,使顺症变成逆症,使病人荣卫散乱,真气散失,邪气独存于内,断送病人的性命,给人家带来莫大的祸殃。这种不知三部九侯的医生,是不能够久长的,因为不知配合四时五行因加相胜的道理,会放过了邪气,伤害了正气,以致断绝病人性命。病邪新侵入人体,没有定着一处,推它就向前,引它就阻止,迎其气而泻之,其病是立刻可以好的。

\chapter{通评虚实论篇第二十八}

黄帝问曰:何谓虚实?
岐伯曰:邪气盛则实,精气夺则虚。
帝曰:虚实何如?岐伯曰:气虚者,肺虚也;气逆者,足寒也。非其时则生,当其时则死。余脏皆如此。
帝曰:何谓重实?
岐伯曰:所谓重实者,言大热病,气热,脉满,是谓重实。
帝曰:经络俱实何如?何以治之?
岐伯曰:经络皆实,是寸脉急而尺缓也,皆当治之。故曰:滑则从,涩则逆也。夫虚实者,皆从其物类始,故五脏骨肉滑利,可以长久也。
帝曰:络气不足,经气有余,何如?
岐伯曰:络气不足,经气有余者,脉口热而尺寒也。秋冬为逆,春夏为从,治主病者。
帝曰:经虚络满何如?
岐伯曰:经虚络满者,尺热满,脉口寒涩也。此春夏死,秋冬生也。
帝曰:治此者奈何?
岐伯曰:络满经虚,灸阴刺阳;经满络虚,刺阴灸阳。
帝曰:何谓重虚?
岐伯曰:脉气上虚尺虚,是谓重虚。
帝曰:何以治之?
岐伯曰:所谓气虚者,言无常也;尺虚者,行步框然;脉虚者,不象阴也。如此者,滑则生,涩则死也。
帝曰:寒气暴上,脉满而实,何如?
岐伯曰:实而滑则生,实而逆则死。
帝曰:脉实满,手足寒,头热何如?
岐伯曰:春秋则生,冬夏则死。脉浮而涩,涩而身有热者死。
帝曰:其形尽满何如?
岐伯曰:其形尽满者,脉急大坚,尺涩而不应也。如是者,故从则生,逆则死。
帝曰:何谓从则生,逆则死?
岐伯曰:所谓从者,手足温也;所谓逆者,手足寒也。
帝曰:乳子而病热,脉悬小者何如?岐伯曰:手足温则生,寒则死。
帝曰:乳子中风热,喘鸣肩息者,脉何如?
岐伯曰:喘鸣肩息者,脉实大也。缓则生,急则死。
帝曰:肠澼便血,何如?岐伯曰:身热则死,寒则生。
帝曰:肠澼下白沫,何如?岐伯曰:脉沉则生,脉浮则死。
帝曰:肠澼下脓血,何如?岐伯曰:脉悬绝则死,滑大则生。
帝曰:肠澼之属,身不热,脉不悬绝,何如?岐伯曰:滑大者曰生,悬涩者曰死,以脏期之。
帝曰:癫疾何如?岐伯曰:脉搏大滑,久自己;脉小坚急,死不治。
帝曰:癫疾之脉,虚实何如?岐伯曰:虚则可治,实则死。
帝曰:消瘅虚实何如?岐伯曰:脉实大,病久可治;脉悬小坚,病久不可治。
帝曰:形度、骨度、脉度、筋度,何以知其度也?
帝曰:春亟治经络;夏亟治经俞;秋亟治六府;冬则闭塞,闭塞者,用药而少针石也。所谓少针石者,非痈疽之谓也,痈疽不得顷时回。痈不知所,按之不应手,乍来乍已,刺手太阳傍三痏,与缨脉各二。掖痈大热,刺足少阳五;刺而热不止,刺手心主三,刺手太阴经络者,大骨之会各三。暴痛筋软,随分而痛,魄汗不尽,胞气不足,治在经俞。
腹暴满,按之不下,取手太阳经络者,胃之募也,少阴俞去脊椎三寸傍五,用员利针。霍乱,刺俞傍五,足阳明及上傍三。刺痫惊脉五,针手太阴各五,刺经,太阳五,刺手少阴经络傍者一,足阳明一,上踝五寸,刺三针。
凡治消瘅、仆击、偏枯、痿厥、气满发逆,肥贵人则高梁之疾也。隔塞、闭绝、上下不通,则暴忧之病也。暴厥而聋,偏塞闭不通,内气暴薄也。不从内,外中风之病,故瘦留著也。蹠跛,寒风湿之病也。
黄帝曰:黄疸暴痛,癫疾厥狂,久逆之所生也。五脏不平,六腑闭塞之所生也。头痛耳鸣,九窍不利,肠胃之所生也。
通评虚实论篇第二十八参考译文
黄帝问道:什麽叫虚实?岐伯回答说:所谓虚实,是指邪气和正气相比较而言的。如邪气方盛,是为实证;若精气不足,就为虚证了。黄帝道:虚实变化的情况怎样?岐伯说:以肺脏为例:肺主气,气虚的,是属于肺脏先虚;气逆的,上实下虚,两足必寒。肺虚若不在相克的时令,其人可生;若遇克贼之时,其人就要死亡。其他各脏的虚实情况亦可类推。
黄帝道:什麽叫重实?岐伯说:所谓重实,如大热病人,邪气甚热,而脉象又盛满,内外俱实,便叫重实。
黄帝道:经络俱实是怎样情况?用什麽方法治疗?岐伯说:所谓经络俱实,是指寸口脉急而尺肤弛缓,经和络都应该治疗。所以说:凡是滑利的就有生机为顺,涩滞的缺少生机为逆。因为一般所谓虚实,人与物类相似,如万物有生气则滑利,万物欲死则枯涩。若一个人的五脏骨肉滑利,是精气充足,生气旺盛,便可以长寿。
黄帝道:络气不足,经气有余的情况怎样?岐伯说:所谓络气不足,经气有余,是指寸口脉滑而尺肤却寒。秋冬之时见这样现象的为逆,在春夏之时就为顺了,治疗必须结合时令。黄帝道:经虚络满的情况怎样?岐伯说:所谓经虚络满,是指尺肤热而盛满,而寸口脉象迟而涩滞。这种现象,在春夏则死,在秋冬则生。黄帝道:这两种病情应怎样治疗呢?岐伯说:络满经虚,炙阴刺阳;经满络虚,刺阴炙阳。
黄帝道:什麽叫重虚?岐伯说:脉虚,气虚,尺虚,称为重虚。黄帝道:怎样辨别呢?岐伯说:所谓气虚,是由于精气虚夺,而语言低微,不能接续;所谓尺虚,是尺肤脆弱,而行动怯弱无力;所谓脉虚,是阴血虚少,不似有阴的脉象。所有上面这些现象的病人,可以总的说一句,脉象滑利的,随病可生,要是脉象涩滞,就要死亡了。
黄帝道:有一种病症,寒气骤然上逆,脉象盛满而实,它的预后怎样呢?岐伯说:脉时而有滑利之象的生;脉实而涩滞,这是逆象,主死。黄帝道:有一种病证,脉象实满,手足寒冷,头部热的预后又怎样呢?岐伯说:这种病人,在春秋之时可生,若在冬夏便要死了。又一种脉象浮而涩,脉涩而身有发热的,亦死。黄帝道:身形肿满的将会怎样呢?岐伯说:所谓身形肿满的脉象急而大坚,而尺肤却涩滞,与脉不相适应。象这样的病情,从则生,逆则死。黄帝道:什麽叫从则生,逆则死?岐伯说:所谓从,就是手足温暖;所谓逆,就是手足寒冷。
黄帝道:乳子而患热病,脉象悬小,它的预后怎样?岐伯说:手足温暖的可生,若手足厥冷,就要死亡。黄帝道:乳子而感受风热,出现喘息有声,张口抬肩症状,它的脉象怎样?岐伯说:感受风热而喘气有声,张口抬肩的,脉象应该实大。如果实大中具有缓和之气的,尚有胃气,可生;要是实大而弦急,是胃气已绝,就要死亡。
黄帝道:赤痢的变化怎样?岐伯说:痢兼发热的,则死;身寒不发热的,则生。黄帝道:痢疾而下白沫的变化怎样?岐伯说:脉沉则生,脉浮则死。黄帝道:痢疾而下脓血的怎样?岐伯说:脉悬绝者死;滑大者生。黄帝道:痢疾病,身不发热,脉搏也不悬绝,预后如何?岐伯说:脉搏滑大者生;脉搏悬涩者死。五脏病各以相克的时日而预测死期。
黄帝道:癫疾的预后怎样?岐伯说:脉来搏而大滑,其病慢慢的会自己痊愈;要是脉象小而坚急,是不治的死证。黄帝道:癫脉象虚实变化怎样?岐伯说:脉虚的可治,脉实的主死。
黄帝道:消渴病脉象的虚实怎样?岐伯说:脉见实大,病虽长久,可以治愈;假如脉象悬小而坚,病拖长了,那就不可治疗。
黄帝道:形度,骨度,脉度,筋度,怎样才测量的出来呢?
黄帝道:春季治病多取各经的络穴;夏季治病多取各经的俞穴;秋季治病多取六腑的合穴;冬季主闭藏,人体的阳气也闭藏在内,治病应多用药品,少用针刺砭石。但所谓少用针石,不包括痈疽等病在内,若痈疽等病,是一刻也不可徘徊迟疑的。
痈毒初起,不知他发在何处,摸又摸不出,时有疼痛,此时可针刺手太阴经穴三次,和颈部左右各二次。生腋痈的病人,高热,应该针足少阳经穴五次;针过以后,热仍不退,可针手厥阴心包经穴三三次,针手太阴经的络穴和大骨之会各三次。急性的痈肿,筋肉挛缩,随着痈肿的发展而疼痛加剧,痛得厉害,汗出不止,这是由于膀胱经气不足,应该刺其经的俞穴。
腹部突然胀满,按之不减,应取手太阳经的络穴,即胃的募穴和脊椎两傍三寸的少阴肾于穴各刺五次,用员利针。霍乱,应针肾俞旁志室穴五次,和足阳明胃俞及胃仑穴各三次。治疗惊风,要针五条经上的穴位,取手太阴的经穴各五次,太阳的经穴各五次,手少阴通里穴傍的手太阳经支正穴一次,足阳明经之解溪穴一次,足踝上五寸的少阴经筑宾穴三次。
凡诊治消瘅、仆击、偏枯、痿厥、气粗急发喘逆等病,如肥胖权贵人患这种病,则是由于偏嗜肉食厚味所造成的。凡是郁结不舒,气粗上下不通,都是暴怒或忧郁所引起的。突然厥逆,不知人事,耳聋,大小便不通,都是因为情志骤然激荡,阳气上迫所致。有的病不从内发,而由于外中风邪,因风邪留恋不去,伏而为热,消烁肌肉,着于肌肉筋骨之间。有的两脚偏跛,是由于风寒湿侵袭而成的疾病。
黄帝道:黄疸、骤然的剧痛、癫疾、劂狂等证,是由于经脉之气,久逆于上而不下行所产生的。五脏不和,是六腑闭塞不通所造成的。头痛耳鸣,九窍不利,是肠胃的病变所引起的。

\chapter{太阴阳明论篇第二十九}

黄帝问曰:太阴阳明为表里,脾胃脉也,生病而异者何也?
岐伯对曰:阴阳异位,更虚更实,更逆更从,或从内,或从外,所以不同,故病异名也。
帝曰:愿闻其异状也。岐伯曰:阳者,天气也,主外;阴者,地气也,主内。故阳道实,阴道虚。故犯贼风虚邪者,阳受之;食饮不节,起居不时者,阴受之。阳受之则入六府;阴受之,则入五脏。入六腑,则身热,不时卧,上为喘呼;入五脏,则脘满闭塞,下为飧泄,久为肠澼。故喉主天气,咽主地气。故阳受风气,阴受湿气。故阴气从足上行至头,而下地循臂至指端;阳气从手上行至头,而下行至足。故曰:阳病者,上行极而下;阴病者,下行极而上。故伤于风者,上先受之;伤于湿者,下先受之。
帝曰:脾病而四支不用,何也?岐伯曰:四支皆禀气于胃,而不得到经,必因于脾,乃得禀也。今脾病不能为胃行其津液,四支不得禀水谷气,气日以衰,脉道不利,筋骨肌肉皆无气以生,故不用焉。
帝曰:脾不主时,何也?岐伯曰:脾者土也,治中央,常以四时长四脏,各十八日寄治,不得独主于时也。脾脏者,常著胃土之精也。土者生万物而法天地。故上下至头足,不得主时也。
帝曰:脾与胃以膜相连耳,而能为之行其津液,何也?岐伯曰:足太阴者,三阴也,其脉贯胃、属脾、络嗌,故太阴为之行气于三阴;阳明者,表也,五脏六腑之海也,亦为之行气于三阳。脏腑各因其经而受气于阳明,故为胃行其津液。四支不得禀水谷气,日以益衰,阴道不利,筋骨肌肉无气以生,故不用焉。
太阴阳明论篇第二十九参考译文
黄帝问道:太阴、阳明两经,互为表里,是脾胃所属的经脉,而所生的疾病不同,是什麽道理?岐伯回答说:太阴属阴经,阳明属阳经,两经循行的部位不同,四时的虚实顺逆不同,病或从内生,或从外入,发病原因也有差异,所以病名也就不同。黄帝道:我想知道它们不同的情况。岐伯说:人身的阳气,犹如天气,主卫互于外;阴气,犹如地气,主营养于内。所以阳气性刚多实,阴气性柔易虚。凡是贼风虚邪伤人,外表阳气先受侵害;饮食起居失调,内在阴气先受损伤。阳分受邪,往往传入六腑;阴气受病,每多累及五脏。邪入六腑,可见发热不得安卧,气上逆而喘促;邪入五脏,则见脘腹胀满,闭塞不通,在下为大便泄泻,病久而产生痢疾。所以喉司呼吸而通天气,咽吞饮食而连地气。因此阳经易受风邪,阴经易感湿邪。手足三阴经脉之气,从足上行至头,再向下沿臂膊到达指端;手足三阳静脉之气,从手上行至头,再向下行到足。所以说,阳经的病邪,先上行至极点,再向下行;阴经的病邪,先下行至极点,再向上行。故风邪为病,上部首先感受;湿邪成疾,下部首先侵害。
黄帝道:脾病会引起四肢功能丧失,这是什麽道理?岐伯说:四肢都要承受胃中水谷精气以濡养,但胃中精气不能直接到达四肢经脉,必须依赖脾气的传输,才能营养四肢。如今脾有病不能为胃输送水谷精气,四肢失去营养,则经气日渐衰减,经脉不能畅通,筋骨肌肉都得不到濡养,因此四肢便丧失正常的功能了。
黄帝道:脾脏不能主旺一个时季,是什麽道理?岐伯说:脾在五行中属土,主管中央之位,分旺于四时以长养四脏,在四季之末各寄旺十八日,故脾不单独主旺于一个时季。由于脾脏经常为胃土传输水谷精气,譬如天地养育万物一样无时或缺的。所以它能从上到下,从头到足,输送水谷之精于全身各部分,而不专主旺于一时季。
黄帝道:脾与胃仅以一膜相连,而脾能为胃转输津液,这是什麽道理?岐伯说:足太阴脾经,属三阴,它的经脉贯通到胃,连属于脾,环绕咽喉,故脾能把胃中水谷之精气输送到手足三阴经;足阳明胃经,为脾经之表,是供给五脏六腑营养之处,故胃也能将太阴之气输送到手足三阳经。五脏六腑各通过脾经以接受胃中的精气,所以说脾能为胃运行津液。如四肢得不到水谷经气的滋养,经气便日趋衰减,脉道不通,筋骨肌肉都失却营养,因而也就丧失正常的功用了。

\chapter{阳明脉解篇第三十}
黄帝问曰:足阳明之脉病,恶人与火,闻木音,则惕然而惊,钟鼓不为动,闻木音而惊,何也?愿闻其故。岐伯对曰:阳明者,胃脉也,胃者,土也。故闻木音而惊者,土恶木也。帝曰:善!其恶火何也?岐伯曰:阳明主肉,其脉血气盛,邪客之则热,热甚则恶火。帝曰:其恶人何也?岐伯曰:阳明厥则喘而惋,惋则恶人。帝曰:或喘而死者,或喘而生者,何也?岐伯曰:厥逆连脏则死,连经则生。帝曰:善!病甚则弃衣而走,登高而歌,或至不食数日,逾垣上屋,所上之处,皆非其素所能也,病反能者何也?岐伯曰:四支者,诸阳之本也。阳盛则四支实,实则能登高也。帝曰:其弃衣而走者何也?岐伯曰:热盛于身,故弃衣欲走也。帝曰:其妄言骂詈,不避亲疏而歌者,何也?岐伯曰:阳盛则使人妄言骂詈,不避亲疏,而不欲食,不欲食,故妄走也。
阳明脉解篇第三十参考译文
黄帝问道:足阳明的经脉发生病变,恶见人与火,听到木器响动的声音就受惊,但听到敲打钟鼓的声音却不为惊动。为什麽听到木音就惊惕?我希望听听其中道理。岐伯说:足阳明是胃的经脉,属土。所以听到木音而惊惕,是因为土恶木克的缘故。黄帝道:好!那麽恶火是为什麽呢?岐伯说:足阳明经主肌肉,其经脉多血多气,外邪侵袭则发热,热甚则所以恶火。黄帝道:其恶人是何道理?岐伯说:足阳明经气上逆,则呼吸喘促,心中郁闷,所以不喜欢见人。黄帝道:有的阳明厥逆喘促而死,有的虽喘促而不死,这是为什麽呢?岐伯说:经气厥逆若累及于内脏,则病深重而死;若仅连及外在的经脉,则病轻浅可生。黄帝道:好!有的阳明病重之时,病人把衣服脱掉乱跑乱跳,登上高处狂叫唱歌,或者数日不进饮食,并能够越墙上屋,而所登上之处,都是其平素所不能的,有了病反能够上去,这是什麽原因?岐伯说:四肢是阳气的根本。阳气盛则四肢充实,所以能够登高。黄帝道:其不穿衣服而到处乱跑,是为什么?岐伯说:身热过于亢盛,所以不要穿衣服而到处乱跑。黄帝道:其胡言乱语骂人,不避亲疏而随便唱歌,是什麽道理?岐伯说:阳热亢盛而扰动心神,故使其神志失常,胡言乱语,斥骂别人,不避亲疏,并且不知道吃饭,所以便到处乱跑。

\chapter{热论篇第三十一}
黄帝问曰:今夫热病者,皆伤寒之类也。或愈或死,其死皆以六、七日之间,其愈皆以十日以上者何也?不知其解,愿闻其故。岐伯对曰:巨阳者,诸阳之属也,其脉连于风府,故为诸阳主气也。人之伤于寒也,则为病热,热虽甚不死;其两感于寒而病者,必不免于死。
帝曰:愿闻其状。岐伯曰:伤寒一日,巨阳受之,故头项痛,腰脊强;二日阳明受之,阳明主肉,其脉挟鼻,络于目,故身热,目疼而鼻干,不得卧也;三日少阳受之,少阳主胆,其脉循胁络于耳,故胸胁痛而耳聋。三阳经络皆受其病,而未入于脏者,故可汗而已。四日太阴受之,太阴脉布胃中,络于嗌,故腹满而嗌干;五日少阴受之,少阴脉贯肾,络于肺,系舌本,故口燥舌干而渴;六日厥阴受之,厥阴脉循阴器而络于肝,故烦满而囊缩。三阴三阳、五脏六腑皆受病,荣卫不行,五藏不通,则死矣。
其不两感于寒者,七日巨阳病衰,头痛少愈;八日阳明病衰,身热少愈;九日少阳病衰,耳聋微闻;十日太阴病衰,腹减如故,则思饮食;十日少阴病衰,渴止不满,舌干已而嚏;十二日厥阴病衰,囊纵,少腹徽下,大气皆去,病日已矣。
帝曰:治之奈何?岐伯曰:治之各通其藏脉,病日衰已矣。其未满三日者,可汗而已;其满三日者,可泄而已。
帝曰:热病已愈,时有所遗者,何也?岐伯曰:诸遗者,热甚而强食之,故有所遗也。若此者,皆病已衰而热有所藏,因其谷气相薄,两热相合,故有所遗也。帝曰:善!治遗奈何?岐伯曰:视其虚实,调其逆从,可使必已矣。帝曰:病热当何禁之?岐伯曰:病热少愈,食肉则复,多食则遗,此其禁也。
帝曰:其病两感于寒者,其脉应与其病形如何?岐伯曰:两感于寒者,病一日,则巨阳与少阴俱病,则头痛,口干而烦满;二日则阳明与太阴俱病,则腹满,身热,不欲食,谵言;三日则少阳与厥阴俱病,则耳聋,囊缩而厥,水浆不入,不知人,六日死。
帝曰:五脏已伤,六腑不通,荣卫不行,如是之后,三日乃死,何也?岐伯曰:阳明者,十二经脉之长也,其血气盛,故不知人。三日,其气乃尽,故死矣。
凡病伤寒而成温者,先夏至日者为病温,后夏至日者为病暑。暑当与汗皆出,勿止。
热论篇第三十一参考译文
黄帝问道:现在所说的外感发热的疾病,都属于伤寒一类,其中有的痊愈,有的死亡,死亡的往往在六七日之间,痊愈的都在十日以上,这是什麽道理呢?我不知如何解释,想听听其中的道理。岐伯回答说:太阳经为六经之长,统摄阳分,故诸阳皆隶属于太阳。太阳的经脉连于风府,与督脉、阳维相会,循行于巅背之表,所以太阳为诸阳主气,主一身之表。人感受寒邪以后,就要发热,发热虽重,一般不会死亡;如果阴阳二经表里同时感受寒邪而发病,就难免于死亡了。
黄帝说:我想知道伤寒的症状。岐伯说:伤寒病一日,为太阳经感受寒邪,足太阳经脉从头下项,侠脊抵腰中,所以头项痛,腰脊强直不舒。二日阳明经受病,阳明主肌肉,足阳明经脉挟鼻络于目,下行入腹,所以身热目痛而鼻干,不能安卧。三日少阳经受病,少阳主骨,足少阳经脉,循胁肋而上络于耳,所以胸肋痛而耳聋。若三阳经络皆受病,尚未入里入阴的,都可以发汗而愈。四日太阴经受病,足太阴经脉散布于胃中,上络于咽,所以腹中胀满而咽干。五日少阴经受病,足少阴经脉贯肾,络肺,上系舌本,所以口燥舌干而渴。六日厥阴经受病,足厥阴经脉环阴器而络于肝,所以烦闷而阴囊收缩。如果三阴三阳经脉和五脏六腑均受病,以致营卫不能运行,五脏之气不通,人就要死亡了。
如果病不是阴阳表里两感于寒邪的,则第七日太阳病衰,头痛稍愈;八日阳明病衰,身热稍退;九日少阳病衰,耳聋将逐渐能听到声音;十日太阴病衰,腹满已消,恢复正常,而欲饮食;十一日少阴病衰,口不渴,不胀满,舌不干,能打喷嚏;十二日厥阴病衰,阴囊松弛,渐从少腹下垂。至此,大邪之气已去,病也逐渐痊愈。黄帝说:怎麽治疗呢?岐伯说:治疗时,应根据病在何脏和经,分别予以施治,病将日渐衰退而愈。对这类病的治疗原则,一般病未满三日,而邪犹在表的,可发汗而愈;病已满三日,邪已入里的,可以泻下而愈。
黄帝说:热病已经痊愈,常有余邪不尽,是什麽原因呢?岐伯说:凡是余邪不尽的,都是因为在发热较重的时候强进饮食,所以有余热遗留。象这样的病,都是病逝虽然已经衰退,但尚有余热蕴藏于内,如勉强病人进食,则必因饮食不化而生热,与残存的余热相薄,则两热相合,又重新发热,所以有余热不尽的情况出现。黄帝说:好。怎样治疗余热不尽呢?岐伯说:应诊察病的虚实,或补或泻,予以适当的治疗,可使其病痊愈。黄帝说:发热的病人在护理上有什麽禁忌呢?岐伯说:当病人热势稍衰的时候,吃了肉食,病即复发;如果饮食过多,则出现余热不尽,这都是热病所应当禁忌的。
黄帝说:表里同伤于寒邪的两感证,其脉和症状是怎样的呢?岐伯说:阴阳两表里同时感受寒邪的两感证,第一日为太阳与少阴两经同时受病,其症状既有太阳的头痛,又有少阴的口干和烦闷;二日为阳明与太阴两经同时受病,其症状既有阳明的身热谵言妄语,又有太阳的腹满不欲食;三日为少阳与厥阴两经同时受病,其症状既有少阳之耳聋,又有厥阴的阴囊收缩和四肢发冷。如果病逝发张展至水浆不入,神昏不知人的程度,到第六天便死亡了。
黄帝说:病已发展至五脏已伤,六腑不通,荣卫不行,象这样的病,要三天以后死亡,是什麽道理呢?岐伯说:阴阳为十二经之长,此经脉的气血最盛,所以病人容易神识昏迷。三天以后,阳明的气血已经竭尽,所以就要死亡。
大凡伤于寒邪而成为温热病的,病发于夏至日以前的就称之为温病,病发于夏至日以后的就称之为暑病。暑病汗出,可使暑热从汗散泄,所以暑病汗出,不要制止。

\chapter{刺热篇第三十二}
肝热病者,小便先黄,腹痛多卧,身热。热争则狂言乃惊,胁满痛,手足躁,不得安卧;庚辛甚,甲乙大汗,气逆则庚辛死。刺足厥阴、少阳。其逆则头痛员员,脉引冲头也。
心热病者,先不乐,数日乃热。热争则卒心痛,烦闷善呕,头痛面赤,无汗;壬癸甚,丙丁大汗,气逆则壬癸死。刺手少阴、太阳。
脾热病者,先头重,颊痛,烦心,颜青,欲呕,身热。热争则腰痛,不可用俯仰,腹满泄,两颔痛;甲乙甚,戊已大汗,气逆则甲乙死。刺足太阴、阳明。
肺热病者,先淅然厥,起毫毛,恶风寒,舌上黄,身热。热争则喘咳,痛走胸膺背,不得太息,头痛不堪,汗出而寒;丙丁甚,庚辛大汗,气逆则丙丁死。刺手太阴、阳明,出血如大豆,立已。
肾热病者,先腰痛胫酸,苦渴数饮,身热。热争则项痛而强,胫寒且酸,足下热,不欲言,其逆则项痛员员澹澹然;戊已甚,壬癸大汗,气逆则戊已死。刺足少阴、太阳。诸汗者,至其所胜日汗出也。
肝热病者,左颊先赤;心热病者,颜先赤;脾热病者,鼻先赤;肺热病者,右颊先赤;肾热病者,颐先赤。病虽未发,见赤色者刺之,名曰治未病。热病从部所起者,至期而已;其刺之反者,三周而已;重逆则死。诸当汗者,至其所胜日汗大出也。诸治热病,以饮之寒水,乃刺之;必寒衣之,居止寒处,身寒而止也。
热病先胸胁痛,手足躁,刺足少阳,补足太阴,病甚者为五十九刺。热病始手臂痛者,刺手阳明、太阴而汗出止。热病始于头首者,刺项太阳而汗出止。热病始于足胫者,刺足阳明而汗出止。热病先身重,骨痛,耳聋好瞑,刺足少阴,病甚为五十九刺。热病先眩冒而热,胸胁满,刺足少阴、少阳。
太阳之脉,色荣颧骨,热病也,荣未夭,曰今且得汗,待时而已;与厥阴脉争见者,死期不过三日,其热病内连肾,少阳之脉色也。少阳之脉,色荣颊前,热病也,荣未交,曰今且得汗,待时而已;与少阴脉争见者,死期不过三日。
热病气穴:三椎下间主胸中热;四椎下间主鬲中热;五椎下间主肝热;六椎下间主脾热;七椎下间主肾热。荣在骶也。项上三椎,陷者中也。颊下逆颧为大瘕,下牙车为腹满,颧后为胁痛。颊上者鬲上也。
刺热篇第三十二参考译文
肝脏发生热病,先出现小便黄,腹痛,多卧,身发热。当气邪入脏,与正气相争时,则狂言惊骇,胁部满痛,手足躁扰不得安卧;逢到庚辛日,则因木受金克而病重,若逢甲已日木旺时,便大汗出而热退若将在庚辛日死亡。治疗时,应刺足厥阴肝和足少阳胆经。若肝气上逆,则见头痛眩晕,这是因热邪循肝脉上冲于头所致。
心脏发热病,先觉得心中不愉快,数天以后始发热,当热邪入脏与正气相争时,则突然心痛,烦闷,时呕,头痛,面赤,无汗;逢到壬癸日,则因火受水克而病重,若逢丙丁日火旺时,便大汗出而热退,若邪气胜脏,病更严重将在壬癸日死亡。治疗时,应刺手少阴心和手太阳小肠经。
脾脏发生热病,先感觉头重,面颊痛,心烦,额部发青,欲呕,身热。当热邪入脏,与正气相争时,则腰痛不可以俯仰,,腹部胀满而泄泻,两颌部疼痛,逢到甲已日木旺时,则因土受木克而病重,若逢戊日土旺时,便大汗出而热退,若邪气胜脏,病更严重,就会在甲已日死亡。治疗时,刺足太阴脾和足阳明胃经。
肺脏发生热病,先感到体表淅淅然寒冷,毫毛竖立,畏恶风寒,舌上发黄,全身发热。当热邪入脏,与正气相争时,则气喘咳嗽,疼痛走窜于胸膺背部,不能太息,头痛的很厉害,汗出而恶寒,逢丙丁日火旺时,则因金受火克而病重,若逢庚辛日金旺时,便大汗出而热退,若邪气胜脏,病更严重,就会在丙丁日死亡。治疗时,刺手太阴肺和手阳明大肠经,刺出其血如大豆样大,则热邪去而经脉和,病可立愈。
肾脏发生热病,先觉腰痛和小腿发痠,口渴的很厉害,频频饮水,全身发热。当邪热入脏,与正气相争时,则项痛而强直,小腿寒冷痠痛,足心发热,不欲言语。如果肾气上逆,则项痛头眩晕而摇动不定,逢利戊已日土旺时,则因水受土克而病重,若逢壬癸日水旺时,便大汗出而热退,若邪气胜脏,病更严重,就会在戊已日死亡。治疗时,刺足少阴肾和足太阳膀胱经。以上所说的诸脏之大汗出,都是到了各脏器旺之日,正胜邪却,即大汗出而热退病愈。
肝脏发生热病,左颊部先见赤色;心脏发生热病,额部先见赤色;脾脏发生热病,鼻部先见赤色;肺脏发生热病,右颊部先见赤色,肾脏发生热病,颐部先见赤色。病虽然还没有发作,但面部已有赤色出现,就应予以刺治,这叫做“治未病”。热病只在五脏色部所在出现赤色,并未见到其他症状的,为病尚轻浅,若予以及时治疗,则至其当旺之,病即可愈;若治疗不当,应泻反补,应补反泻,就会延长病程,虚通过三次当旺之日,始能病愈;若一再误治,势必使病情恶化而造成死亡。诸脏热病应当汗出的,都是至其当旺之日,大汗出而病愈。
凡治疗热病,应在喝些清凉的饮料,以解里热之后,再进行针刺,并且要病人衣服穿的单薄些,居住于凉爽的地方,以解除表热,如此使表里热退身凉而病愈。
热病先出现胸胁痛,手足躁扰不安的,是邪在足少阳经,应刺足少阳经以泻阳分之邪,补足太阴经以培补脾土,病重的就用“五十九刺”的方法。热病先手臂痛的,是病在上而发于阳,刺手阳明、太阴二经之穴,汗出则热止。热病开始发于头部的,是太阳为病,刺足太阳颈项部的穴位,汗出则热止。热病开始发于足胫部的,是病发于阳而始于下,刺足阳明经穴,汗出则热止。热病先出现身体重,骨节痛,耳聋,昏倦嗜睡的,是发于少阴的热病,刺足少阴经之穴,病重的用“五十九刺”的方法。热病先出现头眩晕昏冒而后发热,胸胁满的,是病发于少阳,并将传入少阴,使阴阳枢机失常,刺足少阴和足少阳二经,使邪从枢转而外出。
太阳经脉之病,赤色出现于颧骨部的,这是热病,若色泽尚未暗晦,病尚轻浅,至其当旺之时,可以得汗出而病愈。若同时又见少阴经的脉证,此为木盛水衰的死证,死期不过三日,这是因为热病已连于肾。少阳经脉之病,赤色出现于面颊的前方,这是少阳经脉热病,若色泽尚未暗晦,是病邪尚浅,至其当旺之时,可以得汗出而病愈。若同时又见少阴脉色现于颊部,是母胜其子的死证,其死期不过三日。
治疗热病的气穴:第三脊椎下方主治胸中的热病,第四脊椎下方主治膈中的热病,第五脊椎下方主治肝热病,第六脊椎下方主治脾热病,第七脊椎下方主治肾热病。治疗热病,即取穴于上,以泻阳邪,当再取穴于下,以补阴气,在下取穴在尾骶骨处。项部第三椎以下凹陷处的中央部位是大椎穴,由此向下便是脊椎的开始。诊察面部之色,可以推知腹部疾病,如颊部赤色由下向上到颧骨部,为有“大瘕泄”病;见赤色自颊下行至颊车部,为腹部胀满;赤色见于颧骨后侧,为胁痛;赤色见于颊上,为病在膈上。

\chapter{评热病论篇第三十三}
黄帝问曰:有病温者,汗出辄复热,而脉躁疾,不为汗衰,狂言不能食,病名为何?岐伯对曰:病名阴阳交,交者死也。帝曰:愿闻其说?岐伯曰:人所以汗出者,皆生于谷,谷生于精。今邪气交争于骨肉而得汗者,是邪却而精胜也。精胜,则当能食而不复热。复热者,邪气也。汗者,精气也。今汗出而辄复热者,是邪胜也。不能食者,精无俾也。病而留者,其寿可立而倾也。且夫<<热论>>曰:汗出而脉尚躁盛者死。今脉不与汗相应,此不胜其病也,其死明矣。狂言者,是失志,失志者死。今见三死,不见一生,虽愈必死也。
帝曰:有病身热,汗出烦满,烦满不为汗解,此为何病?岐伯曰:汗出而身热者,风也;汗出而烦满不解者,厥也,病名曰风厥。帝曰:愿卒闻之?岐伯曰:巨阳主气,故先受邪,少阴与其为表里也,得热则上从之,从之则厥也。帝曰:治之奈何?岐伯曰:表里刺之,饮之服汤。帝曰:劳风为病何如?岐伯曰:劳风法在肺下。其为病也,使人强上冥视,唾出若涕,恶风而振寒,此为劳风之病。帝曰:治之奈何?岐伯曰:以救俯仰。巨阳引精者三日,中年者五日,不精者七日。咳出青黄涕,其状如脓,大如弹丸,从口中若鼻中出,不出则伤肺,伤肺则死也。
帝曰:有病肾风者,面浮然壅,害于言,可刺否?岐伯曰:虚不当刺,不当刺而刺,后五日其气必至。帝曰:其至何如?岐伯曰:至必少气时热,时热从胸背上至头,汗出手热,口干苦渴,小便黄,目下肿,腹中鸣,身重难以行,月事不来,烦而不能食,不能正偃,正偃则咳,病名曰风水,论在《刺法》中。
帝曰:愿闻其说。岐伯曰:邪之所凑,其气必虚。阴虚者阳必凑之,故少气时热而汗出也,小便黄者,少腹中有热也。不能正偃者,胃中不和也。正偃则咳甚,上迫肺也。诸有水气者,微肿先见于目下也。帝曰:何以言?岐伯曰:水者阴也,目下亦阴也,腹者至阴之所居,故水在腹者,必使目下肿也。真气上逆,故口若舌干,卧不得正偃,正偃则咳出清水也。诸水病者,故不得卧,卧则惊,惊则咳甚也。腹中鸣也,病本于胃也。薄脾则烦不能食。食不下者,胃脘隔也。身重难以行者,胃脉在足也。月事不来者,胞脉闭也。胞脉者,属心而络于胞中。今气上迫肺,心气不得下通,故月事不来也。帝曰:善!
评热病论篇第三十三参考译文
黄帝问道:有的温热病患者,汗出以后,随即又发热,脉象急疾躁动,其病势不仅没有因汗出而衰减,反而出现言语狂乱,不进饮食等症状,这叫什麽病?岐伯回答说:这种病叫阴阳交,阴阳交是死症。黄帝说:我想听听其中的道理。岐伯说:人所以能够出汗,是依赖于水谷所化生的精气,水谷之精气旺盛,便能胜过邪气而出汗,现在邪气与正气交争于骨肉之间,能够得到汗出的是邪气退而精气胜,精气胜的应当能进饮食而不在发热。复发热是邪气尚留,汗出是精气胜邪,现在汗出后又复发热,是邪气胜过精气。不进饮食,则精气得不到继续补益,邪热又逗留不去,这样发展下去,病人的生命就会立即发生危险。《热论》中也曾说:汗出而脉仍躁盛,是死证。现在其脉象不与汗出相应,是精气已经不能胜过邪气,死亡的征象已是很明显的了。况且狂言乱语是神志失常,神志失常是死证。现在已出现了三种死证,却没有一点生机,病虽可能因汗出而暂时减轻,但终究是要死亡的。
黄帝说:有的病全身发热,汗出,烦闷,其烦闷并不因汗出而缓解,这是什麽病呢?岐伯说:汗出而全身发热,是因感受了风邪;烦闷不解,是由于下气上逆所致,病名叫风厥。黄帝说:希望你能详尽地讲给我听。岐伯说:太阳为诸阳主气,主人一身之表,所以太阳首先感受风邪的侵袭。少阴与太阳相为表里,表病则里必应之,少阴手太阳发热的影响,其气亦从之而上逆,上逆便称为厥。黄帝说:怎麽治疗呢?岐伯说:治疗时应并刺太阳、少阴表里两经,即刺太阳以泻风热之邪,刺少阴以降上逆之气,并内服汤药。
黄帝说:劳风的病情是怎样的呢?岐伯说:劳风的受邪部位常在肺下,其发病的症状,使人头项强直,头昏眩而视物不清,唾出粘痰似涕,恶风而寒栗,这就是劳风病的发病情况。黄帝说:怎样治疗呢?岐伯说:首先应使其胸中通畅,俯仰自如。肾经充盛的青年人,太阳之气能引肾经外布,则水能济火,经适当治疗,可三日而愈;中年人精气稍衰,须五日可愈;老年人精气已衰,水不济火,须七日始愈。这种病人,咳出青黄色粘痰,其状似脓,凝结成块,大小如弹丸,应使痰从口中或鼻中排出,如果不能咳出,就要伤其肺,肺伤则死。
黄帝说:有患肾风的人,面部浮肿,目下壅起,妨害言语,这种病可以用针刺治疗吗?岐伯说:虚证不能用刺。如果不应当刺而误刺,必伤其真气,使其脏气虚,五天以后,则病气复至而病势加重。黄帝说:病气至时情况怎样呢?岐伯说:病气至时,病人必感到少气,时发热,时常觉得热从胸背上至头,汗出手热,口中干渴,小便色黄,目下浮肿,腹中鸣响,身体沉重,行动困难。如患者是妇女则月经闭止,心烦而不能饮食,不能仰卧,仰卧就咳嗽的很厉害,此病叫风水,在《刺法》中有所论述。
黄帝说:我想听听其中的道理。岐伯说:邪气之所以能够侵犯人体,是由于其正气先虚。肾脏属阴,风邪属阳。肾阴不足,风阳便乘虚侵入,所以呼吸少气,时时发热而汗出。小便色黄,是因为腹中有热。不能仰卧,是以内水气上乘于胃,而胃中不和。仰卧则咳嗽加剧,是因为水气上迫于肺。凡是有水气病的,目下部先出现微肿。黄帝说:为什麽?岐伯说:水是属阴的,目下也是属阴的部位,腹部也是至阴所在之处,所以腹中有水的,必使目下部位微肿。水邪之气上泛凌心,迫使脏真心火之气上逆,所以口苦咽干,不能仰卧,仰卧则水气上逆而咳出清水。凡是有水气病的人,都因水气上乘于胃而不能卧,卧则水气上凌于心而惊,逆于肺则咳嗽加剧。腹中鸣响,是胃肠中有水气窜动,其病本在于胃。若水迫于脾,则心烦不能进食。饮食不进,是水气阻隔于胃脘。身体沉重而行动困难,是因为胃的经脉下行于足部,水气随经下流所致。妇女月经不来,是因为水气阻滞,胞脉闭塞不通的缘故。胞脉属于心而下络于胞中,现水气上迫于肺,使心气不得下通,所以胞脉闭而月经不来。黄帝说:好。

\chapter{逆调论篇第三十四}
黄帝问曰:人身常温也,非常热也,为之热而烦满者,何也?岐伯曰:阴气少而阳气胜,故热而烦满也。帝曰:人身非衣寒也,中非有寒气也,寒从中生者何?岐伯曰:是人多痹气也,阳气少,阴气多,故身寒如从水中出。
帝曰:人有四支热,逢风寒如炙如火者,何也?岐伯曰:是人者,阴气虚,阳气盛。四支者,阳也。两阳相得,而阴气虚少,少水不能灭盛火,而阳独治。独治者,不能生长也,独胜而止耳。逢风而如炙如火者,是人当肉烁也。
帝曰:人有身寒,汤火不能热,厚衣不能温,然不冻栗,是为何病?岐伯曰:是人者,素肾气胜,以水为事,太阳气衰,肾脂枯不长,一水不能胜两火。肾者水也,而生于骨,肾不生,则髓不能满,故寒甚至骨也。所以不能冻栗者,肝一阳也,心二阳也,肾孤藏也,一水不能胜二火,故不能冻栗,病名曰骨痹,是人当挛节也。
帝曰:人之肉苛者,虽近衣絮,犹尚苛也,是谓何疾?财伯曰:荣气虚,卫气实也。荣气虚则不仁,卫气虚则不用,荣卫俱虚,则不仁且不用,肉如故也,人身与志不相有,曰死。
帝曰:人有逆气,不得卧而息有音者,有不得卧而息无音者;有起居如故而息有音者;有得卧,行而喘者;有不得卧,不能行而喘者;有不得卧,卧而喘者。皆何藏使然?愿闻其故。岐伯曰:不得卧而息有音者,是阳明之逆也。足三阳者下行,今逆而上行,故息有音也。阳明者,胃脉也,胃者,六腑之海,其气亦下行。阳明逆,不得从其道,故不得卧也。《下经》曰:胃不和则卧不安。此之谓也。夫起居如故而息有音者,此肺之络脉逆也,络脉不得随经上下,故留经而不得。络脉之病人也微,故起居如故而息有音也。夫不得卧,卧则喘者,是水气之客也。夫水者,循津液而流也。肾者,水脏,主津液,主卧与喘也。帝曰:善!
逆调论篇第三十四参考译文
黄帝道:有的病人,四肢发热,遇到风寒,热得更加厉害,如同炙于火上一般,这是什麽原因呢?岐伯回答说:这是由于阴气少而阳气胜,所以发热而烦闷。黄帝说:有的人穿的衣服并不单薄,也没有为寒邪所中,却总觉得寒气从内而生,这是什麽原因呢?岐伯说:是由于这种人多痹气,阳气少而阴气多,所以经常感觉身体发冷,象从冷水中出来一样。
黄帝说:有的人四肢发热,一遇到风寒,便觉得身如热火熏炙一样,这是什麽原因呢?岐伯说:这种人多因素体阴虚而阳气胜。四肢属阳,风邪也属阳,属阳的四肢感受属阳的风邪,是两阳相并,则阳气更加亢盛,阳气益盛则阴气日益虚少,致衰少的阴气不能熄灭旺盛的阳火,形成了阳气独旺的局面。现洋气独旺,便不能生长,因阳气独生而生机停止。所以这种四肢热逢风而热的如炙如火的,其人必然肌肉逐渐消瘦。
黄帝说:有的人身体寒凉,虽进汤火不能使之热,多穿衣服也不能使之温,但却不恶寒战栗,这是什麽病呢?岐伯说:这种人平素即肾水之气盛,又经常接近水湿,致水寒之气偏盛,而太阳之阳气偏衰,太阳之阳气衰则肾之枯竭不长。肾是水脏,主生长骨髓,肾脂不生则骨髓不能充满,故寒冷至骨。其所以不能战栗,是因为肝是一阳,心是二阳,一个独阴的肾水,胜不过心肝二阳之火,所以虽寒冷,但不战栗,这种病叫“骨痹”,病人必骨节拘挛。
黄帝说:有的人皮肉麻木沉重,虽穿上棉衣,仍然如故,这是什麽病呢?岐伯说:这是由于营气虚而卫气实所致。营气虚弱则皮肉麻木不仁,卫气虚弱,则肢体不能举动,营气与卫气具虚,则既麻木不仁,又不能举动,所以皮肉更加麻木沉重。若人的形体与内脏的神志不能相互为用,就要死亡。
黄帝说:人病气逆,有的不能安卧而呼吸有声;有的不能安卧而呼吸无声;有的起居如常而呼吸有声;有的能够安卧,行动则气喘;有的不能安卧,也不能行动而气喘;有的不能安卧,卧则气喘。是哪些脏腑发病,使之这样呢?我想知道是什麽缘故。岐伯说:不能安卧而呼吸有声的,是阳明经脉之气上逆。足三阳的经脉,从头到足,都是下行的,现在足阳明经脉之气上逆而行,所以呼吸不利而有声。阳明是胃脉,胃是六腑之海,胃气亦以下行为顺,若阳明经脉之气逆,胃气便不得循常道而下行,所以不能平卧。《下经》曾说:“胃不和则卧不安。”就是这个意思。若起居如常而呼吸有声的,这是由于肺之脉络不顺,络脉不能随着经脉之气上下,故其气留滞于经脉而不行于络脉。但络脉生病是比较轻微的,所以虽呼吸不利有声,但起居如常。若不能安卧,卧则气喘的,是由于水气侵犯所致。水气是循着津液流行的道路而流动的。肾是水脏,主持津液,如肾病不能主水,水气上逆而犯肺,则人即不能平卧而气喘。黄帝说:好。
\chapter{疟论篇第三十五}
黄帝问曰:夫痎疟皆生于风,其蓄作有时者何也?岐伯曰:疟之始发也,先起于毫毛,伸欠乃作,寒栗鼓颔,腰脊俱痛;寒去则内皆热,头痛如破,渴欲冷饮。
帝曰:何气使然?愿闻其道。岐伯曰:阴阳上下交争,虚实更作,阴阳相移也。阳并于阴,则阴实而阳虚,阳明虚则寒栗鼓颔也;巨阳虚则腰背头项痛;三阳俱虚,则阴气胜,阴气胜则骨寒而痛,寒生于内,故中外皆寒。阳盛则外热,阴虚则内热,外内皆热,则喘而渴,故欲冷饮也。此皆得之夏伤于暑,热气盛,藏于皮肤之内,肠胃之外,此荣气之所舍也。此令人汗空疏,腠理开,因得秋气,汗出遇风,及得之以浴,水气舍于皮肤之内,与卫气并居;卫气者,昼日行于阳,夜行于阴,此气得阳而外出,得阴而内薄,内外相薄,是以日作。
帝曰:其间日而作者何也?岐伯曰:其气之舍深,内薄于阴,阳气独发,阴邪内著,阴与阳争不得出,是以间日而作也。帝曰:善!
其作日晏与其日早者,何气使然?岐伯曰:邪气客于风府,循膂而下,卫气一日一夜大会于风府,其明日日下一节,故其作也晏,此先客于脊背也。每至于风府,则腠理开,腠理开则邪气入,邪气入则病作,以此日作稍益晏也。其出于风府,日下一节,二十五日下至骶骨;二十六日入于脊内,注于伏膂之脉;其气上行,九日出于缺盆之中。其气日高,故作日益早也。其间日发者,由邪气内薄于五藏,横连募原也,其道远,其气深,其行迟,不能与卫气俱行,不得皆出,故间日乃作也。
帝曰:夫子言卫气每至于风府,腠理乃发,发则邪气入,入则病作。今卫气日下一节,其气之发也,不当风府,其日作者奈何?岐伯曰:此邪气客于头项,循膂而下者也,故虚实不同,邪中异所,则不得当其风府也。故邪中于头项者,气至头项而病;中于背者,气至背而病;中于腰脊者,气至腰脊而病;中于手足者,气至手足而病;卫气之所在,与邪气相合,则病作。故风无常府,卫气之所发,必开其腠理,邪气之所合,则其府也。帝曰:善!
夫风之与疟也,相似同类,而风独常在,疟得有时而休者,何也?岐伯曰:风气留其处,故常在;疟气随经络,沉以内薄,故卫气应乃作。帝曰:疟先寒而后热者,何也?岐伯曰:夏伤于大暑,其汗大出,腠理开发,因遇夏气凄沧之水寒,藏于腠理皮肤之中,秋伤于风,则病成矣。夫寒者,阴气也;风者,阳气也。先伤于寒而后伤于风,故先寒而后热也,病以时作,名曰寒疟。
帝曰:先热而后寒者,何也?岐伯曰:此先伤于风,而后伤于寒,故先热而后寒也,亦以时作,名曰温疟。
其但热而不寒者,阴气先绝,阳气独发,则少气烦冤,手足热而欲呕,名曰瘅疟。
帝曰:夫经言有余者写之,不足者补之。今热为有余,寒为不足。夫疟者之寒,汤火不能温也,及其热,冰不能寒也。此皆有余不足之类。当此之时,良工不能止,必须其自衰乃刺之,其故何也?愿闻其说。岐伯曰:经言无刺火高。
之热,无刺浑浑之脉,无刺漉漉之汗,故为其病逆,未可治也。夫疟之始发也,阳气并于阴,当是之时,阳虚而阴盛,外无气,故先寒栗也;阴气逆极,则复出之阳,阳与阴复并于外,则阴虚而阳实,故先热而渴。夫疟气者,并于阳则阳胜,并于阴则阴胜;阴胜则寒,阳胜则热。疟者,风寒之气不常也,病极则复。至病之发也,如火之热,如风雨不可当也。故经言曰:方其盛时必毁,因其衰也,事必大昌。此之谓也。夫疟之未发也,阴未并阳,阳未并阴,因而调之,真气得安,邪气乃亡。故工不能治其已发,为其气逆也。帝曰:善。
攻之奈何?早晏何如?岐伯曰:疟之且发也,阴阳之且移也,必从四末始也。阳已伤,阴从之,故先其时坚束其处,令邪气不得入,阴气不得出,审候见之,在孙络盛坚而血者,皆取之,此真往而未得并者也。
帝曰:疟不发,其应何如?岐伯曰:疟气者,必更胜更虚。当气之所在也,病在阳,则热而脉躁;在阴,则寒而脉静;极则阴阳俱衰,卫气相离,故病得休;卫气集,则复病也。
帝曰:时有间二日或至数日发,或渴或不渴,其故何也?岐伯曰:其间日者,邪气与卫气客于六府,而有时相失,不能相得,故休数日乃作也。疟者,阴阳更胜也,或甚或不甚,故或渴或不渴。
帝曰:论言夏伤于暑,秋必病疟,今疟不必应者,何也?岐伯曰:此应四时者也。其病异形者,反四时也。其以秋病者寒甚,以冬病者寒不甚,以春病者恶风,以夏病者多汗。
帝曰:夫病温疟与寒疟,而皆安舍,舍于何藏?岐伯曰:温疟者,得之冬中风,寒气藏于骨髓之中,至春则阳气大发,邪气不能自出,因遇大暑,脑髓烁,肌肉消,腠理发泄,或有所用力,邪气与汗皆出。此病藏于肾,其气先从内出之于外也。如是者,阴虚而阳盛,阳盛则热矣,衰则气复反入,入则阳虚,阳虚则寒矣,故先热而后寒,名曰温疟。帝曰:瘅疟何如?岐伯曰:瘅疟者,肺素有热,气盛于身,厥逆上冲,中气实而不外泄,因有所用力,腠理开,风寒舍于皮肤之内,分肉之间而发,发则阳气盛,阳气盛而不衰,则病矣。其气不用于阴,故但热而不寒,气内藏于心,而外舍于分肉之间,令人消烁脱肉,故命曰瘅疟。帝曰:善。
疟论篇第三十五参考译文
黄帝问道:一般来说,疟疾都由于感受了风邪而引起,他的修作有一定时间,这是什麽道理?岐伯回答说:疟疾开始发作的时候,先起于毫毛竖立,继而四体不舒,欲的引伸,呵欠连连,乃至寒冷发抖,下颌鼓动,腰脊疼痛;及至寒冷过去,便是全身内外发热,头痛有如破裂,口渴喜欢冷饮。
黄帝道:这是什麽原因引起的?请说明它的道理。岐伯说:这是由于阴阳上下相争,虚实交替而作,阴阳虚实相互移易转化的关系。阳气并入于阴分,使阴气实而阳气虚,阳明经气虚,就寒冷发抖乃至两颌鼓动;太阳经气虚便腰背头项疼痛;三阳经气都虚,则阴气更胜,阴气胜则骨节寒冷而疼痛,寒从内生,所以内外都觉寒冷。如阴气并入阳分,则阳气实而阴气虚。阳主外,阳盛就发生外热;阴主内,阴虚就发生内热,因此外内都发热,热甚的时候就气喘口渴,所以喜欢冷饮。这都是由于夏天伤于暑气,热气过盛,并留藏于皮肤之内,肠胃之外,亦即荣气居留的所在。由于暑热内伏,使人汗孔疏松,腠理开泄,一遇秋凉,汗出而感受风邪,或者由于洗澡时感受水气,风邪水气停留于皮肤之内,与卫气相合并居于卫气流行的所在;而卫气白天行于阳分,夜里行于阴分,邪气也随之循行于阳分时则外出,循行于阴分时则内搏,阴阳内外相搏,所以每日发作。
黄帝道:疟疾有隔日发作,为什麽?岐伯说;因为邪气舍留之处较深,向内迫近与阴分,致使阳气独行于外,而阴分之邪留着于里,阴与阳相争而不能即出,所以隔一天才发作一次。黄帝道:讲得好!
疟疾发作的时间,有逐日推迟,或逐日提前的,是什麽缘故?岐伯说:邪气从风府穴侵入后,循脊骨逐日逐节下移,卫气是一昼夜会于风府,而邪气却每日向下移行一节,所以其发作时间也就一天迟一天,这是由于邪气先侵袭于脊骨的关系。每当卫气会于风府时,则腠理开发,腠理开发则邪气侵入,邪气侵入与卫气交争,病就发作,因邪气日下一节,所以发病时间就日益推迟了。这种邪气侵袭风府,逐日下移一节而发病的,约经二十五日,邪气下行至骶骨;骶骨;二十六日,又入于脊内,而流注于伏肿脉;再沿冲脉上行,至九日上至于缺盆之中。因为邪气日渐上升,所以发病的时间也就一天早一天。至于隔一天发病一次的,是因为邪气内迫与五脏,横连与膜原,它所行走的道路较远,邪气深藏,循行迟缓,不能和卫气并行,邪气与卫气不得同时皆出,所以隔一天才能发作一次。
黄帝道:您说卫气每至于风府时,腠理开发,邪气乘机袭入,邪气入则病发作。现在又说卫气与邪气相余的部位每日下行一节,那麽发病时,邪气就不恰在于风府,而能每日发作一次,是何道理?岐伯说:以上是指邪气侵入于头项,循着脊骨而下者说的,但人体各部分的虚实不同,而邪气侵犯的部位也不一样,所以邪气所侵,不一定都在风府穴处。例如:邪中于头项的,卫气行至头顶而病发;邪中于背部的,卫气行至背部而病发;邪中于腰脊的,卫气行至腰脊而病发;邪中于手足的,卫气行至手足而病发;凡卫气所行之处,和邪气相合,那病就要发作。所以说风邪侵袭人体没有一定的部位,只要卫气与之相应,腠理开发,邪气得以凑合,这就是邪气侵入的地方,也就是发病的所在。黄帝道:讲得好!
风病和疟疾相似而同属一类,为什麽风病的症状持续常在,而疟疾却发作有休止呢?岐伯说:风邪为病是稽留于所中之处,所以症状持续常在;疟邪则是随着经络循行,深入体内,必须与卫气相遇,病才发作。
黄帝道:疟疾发作有先寒而后热的,为什麽?岐伯说:夏天感受了严重的暑气,便留藏在腠理皮肤之中,到秋天又伤了风邪,就成为疟疾了。所以水寒,是一种阴气,风邪是一种阳气。先伤于水寒之气,后伤于风邪,所以先寒而后热,病的发作有一定的时间,这名叫寒疟。
黄帝道:有一种先热而后寒的,为什麽?岐伯说:这是先伤于风邪,后伤于水寒之气,所以先热而后寒,发作也有一定的时间,这名叫温疟。
还有一种只发热而不恶寒的,这是由于病人的阴气先亏损于内,因此阳气独旺于外,病发作时,出现少气烦闷,手足发热,要想呕吐,这名叫瘅疟。
黄帝道:医经上说有余的应当泻,不足的应当补。今发热是有余,发冷是不足。而疟疾的寒冷,虽然用热水或向火,亦不能使之温暖,及至发热,即使用冰水,也不能使之凉爽。这些寒热都是有余不足之类。但当其发冷、发热的时候,良医也无法制止,必须待其病势自行衰退之后,才可以施用刺法治疗,这是什麽缘故?请你告诉我。岐伯说:医经上说过,有高热时不能刺,脉搏纷乱时不能刺,汗出不止时不能刺,因为这正当邪盛气逆的时候,所以未可立即治疗。疟疾刚开始发作,阳气并于阴分,此时阳虚而阴盛,外表阳气虚,所以先寒冷发抖;至阴气逆乱已极,势必复出于阳分,于是阳气与阴气相并于外,此时阴分虚而阳分实,所以先热而口渴。因为疟疾并与阳分,则阳气胜,并于阴分,则阴气胜;阴气胜则发寒,阳气胜则发热。由于疟疾感受的风寒之气变化无常,所以其发作至阴阳之气俱逆极时,则寒热休止,停一段时间,又重复发作。当其病发作的时候,象火一样的猛烈,如狂风暴雨一样迅不可当。所以医经上说:当邪气盛极的时候,不可攻邪,攻之则正气也必然受伤,应该乘邪气衰退的时候而攻之,必然获得成功,便是这个意思。因此治疗疟疾,应在未发的时候,阴气尚未并于阳分,阳气尚未并于阴分,便进行适当的治疗,则正气不至于受伤,而邪气可以消灭。所以医生不能在疟疾发病的时候进行治疗,就是因为此时正当正气和邪气交争逆乱的缘故。黄帝道:讲得好!
疟疾究竟怎样治疗?时间的早晚应如何掌握?岐伯说:疟疾将发,正是阴阳将要相移之时,它必从四肢开始。若阳气已被邪伤,则阴分也必将受到邪气的影响,所以只有在未发病之先,以索牢缚其四肢末端,使邪气不得入,阴气不得出,两者不能相移;牢缚以后,审察络脉的情况,见其孙络充实而郁血的部分,都要刺出其血,这是当真气尚未与邪气相并之前的一种“迎而夺之”的治法。
黄帝道:疟疾在不发作的时候,它的情况应该怎样?岐伯说:疟气留舍于人体,必然使阴阳虚实,更替而作。当邪气所在的地方是阳分,则发热而脉搏躁急;病在阴分,则发冷而脉搏较静;病到极期,则阴阳二气都以衰惫,卫气和邪气互相分离,病就暂时休止;若卫气和邪气再相遇合,则病又发作了。
黄帝道:有些疟疾隔二日,或甚隔数日发作一次,发作时有的口渴,有的不渴,是什麽缘故?岐伯说:其所以隔几天再发作,是以为邪气与卫气相会于风府的时间不一致,有时不能相遇,不得皆出,所以停几天才发作。疟疾发病,是由于阴阳更替相胜,但其中程度上也有轻重不同,所以有的口渴,有的不渴。
黄帝道:医经上说夏伤于暑,秋必病疟,而有些疟疾,并不是这样,是什麽道理?岐伯说:夏伤于暑,秋必病疟,这是指和四时发病规律相应的而言。亦有些疟疾形症不同,与四时发病规律相反的。如发于秋天的,寒冷较重;发于冬天的,寒冷较轻;发于春天的,多恶风;发于夏天的,汗出得很多。
黄帝道:有病温疟和寒疟,邪气如何侵入?逗留在哪一脏?岐伯说:温疟是由于冬天感受风寒,邪气留藏在骨髓之中,虽到春天阳气生发活波,邪气仍不能自行外出,乃至夏天,因夏热炽盛,使人精神倦怠,脑髓消烁,肌肉消瘦,腠理发泄,皮肤空疏,或由于劳力过甚,邪气才乘虚与汗一齐外出。这种病邪原是伏藏与肾,故其发作时,是邪气从内而于外。这样的病,阴气先虚,而阳气偏盛,阳盛就发热,热极之时,则邪气又回入于阴,邪入于阴则阳气又虚,阳气虚便出现寒冷,所以这种病是先热而后寒,名叫温疟。黄帝道:瘅疟的情况怎样?岐伯说:瘅疟是由于肺脏素来有热,肺气壅盛,气逆而上冲,以致胸中气实,不能发泄,适因劳力之后,腠理开泄,风寒之邪便乘机侵袭于皮肤之内、肌肉之间而发病,发病则阳气偏盛,阳气盛而不见衰减,于是病就但热不寒了。为什麽不寒?因邪气不入于阴分,所以但热而不恶寒,这种病邪内伏于心脏,而外出则留连于肌肉之间,能使人肌肉瘦削,所以名叫瘅疟。黄帝道:讲得好!
\chapter{刺疟篇第三十六}
足太阳之疟,令人腰痛头重,寒从背起,先寒后热,熇熇暍暍然,热止汗出,难已,刺郄中出血。足少阳之疟,令人身体解倦,寒不甚,热不甚,恶见人,见人心惕惕然,热多,汗出甚,刺足少阳。足阳明之疟,令人先寒,洒淅洒淅,寒甚久乃热,热去汗出,喜见日月光火气,乃快然,刺足阳明跗上。足太阴之疟,令人不乐,好大息,不嗜食,多寒热汗出,病至则善呕,呕已乃衰,即取之。足少阴之疟,令人呕吐甚,多寒热,热多寒少,欲闭户牖而处,其病难已。足厥阴之疟,令人腰痛,少腹满,小便不利,如癃状,非癃也,数便,意恐惧,气不足,腹中悒悒,刺足厥阴。肺疟者,令人心寒,寒甚热,热间善惊,如有所见者,刺手太阴、阳明。心疟者,令人烦心甚,欲得清水,反寒多,不甚热,刺手少阴。肝疟者,令人色苍苍然,太息,其状若死者,刺足厥阴见血。脾疟者,令人寒,腹中痛,热则肠中鸣,鸣已汗出,刺足太阴。肾疟者,令人洒洒然,腰脊痛宛转,大便难,目眴眴然,手足寒,刺足太阳、少阴。胃疟者,令人且病也,善饥而不能食,食而支满腹大,刺足阳明、太阴横脉出血。疟发身方热,刺跗上动脉,开其空,出其血,立寒;疟方欲寒,刺手阳明太阴、足阳明太阴。疟脉满大急,刺背俞,用中针傍伍俞各一,适肥瘦,出其血也。疟脉小实急,灸胫少阴,刺指井。疟脉满大急,刺背俞,用五胠俞、背俞各一,适行至于血也。疟脉缓大虚,便宜用药,不宜用针。凡治疟,先发如食顷,乃可以治,过之则失时也。诸疟而脉不见,刺十指间出血,血去必已;先视身之赤如小豆者,尽取之。十二疟者,其发各不同时,察其病形,以知其何脉之病也。先其发时如食顷而刺之,一刺则衰,二刺则知,三刺则已;不已,刺舌下两脉出血;不已,刺郄中盛经出血,又刺项已下侠脊者,必已。舌下两脉者,廉泉也。刺疟者,必先问其病之所先发者,先刺之。先头痛及重者,先刺头上及两额、两眉间出血。先项背痛者,先刺之。先腰脊痛者,先刺郄中出血。先手臂痛者,先刺手少阴、阳明十指间。先足胫酸痛者,先刺足阳明十指间出血。风疟,疟发则汗出恶风,刺三阳经背俞之血者。胫酸痛甚,按之不可,名曰跗髓病,以镵针针绝骨出血,立已。身体小痛,刺至阴。诸阴之井,无出血,间日一刺。疟不渴,间日而作,刺足太阳;渴而间日作,刺足少阳;温疟汗不出,为五十九刺。
刺疟篇第三十六参考译文
足太阳经的疟疾,使人腰痛头重,寒冷从脊背而起先寒后热,热势很盛,热止汗出,这种疟疾,不易痊愈,治疗方法,刺委中穴出血。足少阳经的疟疾,使人身倦无力,恶寒发热都不甚厉害,怕见人,看见人就感到恐惧,发热的时间比较长,汗出亦很多,治疗方法,刺足少阳经。足阳明经的疟疾,使人先觉怕冷,逐渐恶寒加剧,很久才发热,退热时便汗出,这种病人,喜欢亮光,喜欢向火取暖,见到亮光以及火气,就感到爽快,治疗方法,刺足阳明经足背上的冲阳穴。足太阴经的疟疾,使人闷闷不乐,时常要叹息,不想吃东西,多发寒热,汗出亦多,病发作时容易呕吐,吐后病势减轻,治疗方法,取足太阴经的孔穴。足少有病的疟疾,使人发生剧烈呕吐,多发寒热,热多寒少,常常喜欢紧闭门窗而居,这种病不易痊愈。足厥有病的疟疾,使人腰痛,少腹胀满,小便不利,似乎癃病,而实非癃病,只是小便频数不爽,病人心中恐惧,气分不足,腹中郁滞不畅,治疗方法,刺足厥有病。
肺疟,使人心里感到发冷,冷极则发热,热时容易发惊,好象见到了可怕的事物,治疗方法,刺手太阴,手阳明两经。心虐,使人心中烦热得很厉害,想喝冷水,但身上反觉寒多而不太热,治疗方法,刺手少阴经。肝疟,使人面色苍青,时欲太息,厉害的时候,形状如死,治疗方法,刺足厥有病出血。脾疟,使人发冷,腹中痛,待到发热时,则脾气行而肠中鸣响,肠鸣后阳气外达而汗出,治疗方法,刺足太阴经。肾疟,使人洒淅寒冷,腰脊疼痛,难以转侧,大便困难,目视眩动不明,手足冷,治疗方法,刺足太阳、足少阴两经。胃疟,发病时使人易觉饥饿,但又不能进食,进食就感到腕腹胀满膨大,治疗方法,取足阳明、足太阴两经横行的络脉,刺出其血。
治疗疟疾,在刚要发热的时候,刺足背上的动脉,开其孔穴,刺出其血,可立即热退身凉;如疟疾刚要发冷的时候可刺手阳明、太阴和足阳明、太阴的俞穴。如疟疾病人的脉搏满大而急,刺背部的俞穴,用中等针按五胠俞各取一穴,并根据病人形体的胖瘦,确定针刺出血的多少。如疟疾病人的脉搏小实而急的,炙足胫部的少有病穴,并刺足趾端的井穴。如疟疾病人的脉搏满大而急,刺背部俞穴,取五胠俞、背俞各一穴,并根据病人体质,刺之出血。如疟疾病人的脉搏缓大而虚的,就应该用药治疗,不宜用针刺。大凡治疗疟疾,应在病没有发作之前约一顿饭的时候,予以治疗,过了这个时间,就会失去时机。凡疟疾病人脉沉伏不见的,急刺十指间出血血出病必愈;若先见皮肤上发出象赤小豆的红点,应都用针刺去。上述十二种疟疾,其发作各有不同的时间,应观察病人的症状,从而了解病属于那一经脉。如在没有发作以前约一顿饭的时候就给以针刺,刺一次病势衰减,刺二次病就显著好转,刺三次病即痊愈;如不愈,可刺舌下两脉出血;如再不愈,可取委中血盛的经络,刺出其血,并刺项部以下挟脊两旁的经穴,这样,病一定会痊愈。上面所说的舌下两脉,就是指的廉泉穴。
凡刺疟疾,必先问明病人发作时最先感觉症状的部位,给以先刺。如先发头痛头重的,就先刺头上及两额、两眉间出血。先发倾项脊背痛的,就先刺颈项和背部。先发腰脊痛的,就先刺委中出血。先发手臂痛的,就先刺手少阴、手阳明的十指见的孔穴。先发足胫痠痛的,就先刺足阳明十趾间出血。风疟,发作时是汗出怕风,可刺三阳经背部的俞穴出血。小腿痠疼剧烈而拒按的,名叫跗髓病,可用鑱针刺绝骨穴出血,其痛可以立止。如身体稍感疼痛,刺至阴穴。但应注意,凡刺诸有病的井穴,皆不可出血,并应隔日刺一次。疟疾口不渴而间日发作的,刺足太阳经;如口渴而间日发作的,刺足少阳经;温疟而汗不出的,用“五十九刺”的方法。

\chapter{气厥论篇第三十七}
黄帝问曰:五脏六腑,寒热相移者何?岐伯曰:肾移寒于脾,痈肿,少气。脾移寒于肝,痈肿,筋挛。肝移寒于心,狂,隔中。心移寒于肺,肺消;肺消者,饮一溲二,死不治。肺移寒于肾,为涌水;涌水者,按腹不坚,水气客于大肠,疾行则鸣濯濯,如囊裹浆,水之病也。脾移热于肝,则为惊衄。肝移热于心,则死。心移热于肺,传为鬲消。肺移热于肾,传为柔庢。肾移热于脾,传为虚,肠澼死,不可治。胞移热于膀胱,则癃溺血。膀胱移热于小肠,鬲肠不便,上为口麋。小肠移热于大肠,为虑瘕,为沉。大肠移热于胃,善食而瘦,以谓之食亦。胃移热于胆,亦曰食亦。胆移热于脑,则辛頞鼻渊,鼻渊者,浊涕下不止也,传为衄蔑瞑目。故得之气厥也。
气厥论篇第三十七参考译文
皇帝问道:五脏六腑的寒热互相转移的情况是怎样的?岐伯说:肾移寒于脾,则病痈肿和少气。脾移寒于肝,则痈肿和筋挛。肝移寒于心,则病发狂和胸中隔塞。心移寒于肺,则为肺消;肺消病的症状是饮水一分,小便要排二分,属无法治疗的死证。肺移寒于肾,则为涌水;涌水病的症状是腹部按之不甚坚硬,但因水气留居于大肠,故快走时肠中濯濯鸣响,如皮囊装水样,这是水气之病。脾移热于肝,则病惊骇和鼻衄。肝移热于心,则引起死亡。心移热于肺,日久则为鬲消。肺移热于肾,日久则为柔庢。肾移热于脾,日久渐成虚损;若再患肠澼,便宜成为无法治疗的死症。胞移热于膀胱,则病小便不利和尿血。膀胱移热于小肠,使肠道隔塞,大便不通,热气上行,以至口舌糜烂。小肠移热于大肠,则热结不散,成为伏瘕,或为痔痔。大肠移热于胃,则使人饮食增加而体瘦无力,病称为食亦。胃移热于胆,也叫做食亦。胆移热于脑,则鼻梁内感觉辛辣而成为鼻渊,鼻渊症状,是常鼻流浊涕不止,日久可至鼻中流血,两目不明。以上各种病症,皆由于寒热之气厥逆,在脏腑中互相移传而引起的。
\chapter{咳论篇第三十八}
黄帝问曰:肺之令人咳,何也?岐伯曰:五脏六腑皆令人咳,非独肺也。帝曰:愿闻其状。岐伯曰:皮毛者,肺之合也;皮毛先受邪气,邪气以从其合也。其寒饮食入胃,从肺脉上至于肺则肺寒,肺寒则外内合邪,因而客之,则为肺咳。五脏各以其时受病,非其时,各传以与之。
人与天地相参,故五脏各以治时,感于寒则受病,微则为咳,甚则为泄、为痛。乘秋则肺先受邪,乘春则肝先受之,乘夏则心先受之,乘至阴则脾先受之,乘冬则肾先受之。
帝曰:何以异之?岐伯曰:肺咳之状,咳而喘息有音,甚则唾血。心咳之状,咳则心痛,喉中介介如梗状,甚则咽肿喉痹。肝咳之状,咳则两胁下痛,甚则不可以转,转则两胠下满。脾咳之状,咳则右胁下痛,阴阴引肩背,甚则不可以动,动则咳剧。肾咳之状,咳则腰背相引而痛,甚则咳涎。
帝曰:六府之咳奈何?安所受病?岐伯曰:五脏之久咳,乃移于六府。脾咳不已,则胃受之;胃咳之状,咳而呕,呕甚则长虫出。肝咳不已,则胆受之;胆咳之状,咳呕胆汁。肺咳不已,则大肠受之;大肠咳状,咳而遗失。心咳不已,则小肠受之;小肠咳状,咳而失气,气与咳俱失。肾咳不已,则膀胱受之;膀胱咳状,咳而遗溺。久咳不已,则三焦受之;三焦咳状,咳而腹满,不欲食饮。此皆聚于胃,关于肺,使人多涕唾而面浮肿气逆也。
帝曰:治之奈何?岐伯曰:治脏者,治其俞;治腑者,治其合;浮肿者,治其经。帝曰:善。
咳论篇第三十八参考译文
黄帝问道:肺脏有病,都能使人咳嗽,这是什麽道理?岐伯回答说:五脏六腑有病,都能使人咳嗽,不单是肺病如此。黄帝说:请告诉我各种咳嗽的症状。岐伯说:皮毛与肺是相配合的,皮毛先感受了外邪,邪气就会影响到肺脏。再由于吃了寒冷的饮食,寒气在胃循着肺脉上于肺,引起肺寒,这样就使内外寒邪相合,停留于肺脏,从而成为肺咳。这是肺咳的情况。至于五脏六腑之咳,是五脏各在其所主的时令受病,并非在肺的主时受病,而是各脏之病传给肺的。
人和自然界是相应的,故五脏在其所主的时令受了寒邪,使能得病,若轻微的,则发生咳嗽,严重的,寒气入里就成为腹泻、腹痛。所以当秋天的时候,肺先受邪;当春天的时候,肝先受邪;当夏天的时候,心先受邪;当长夏太阴主时,脾先受邪;当冬天的时候,肾先受邪。
黄帝道:这些咳嗽怎样鉴别呢?岐伯说:肺咳的症状,咳而气喘,呼吸有声,甚至唾血。心咳的症状,咳则心痛,喉中好象有东西梗塞一样,甚至咽喉肿痛闭塞。肝咳的症状,咳则两侧胁肋下疼痛,甚至痛得不能转侧,转侧则两胁下胀满。脾咳的症状,咳则右胁下疼痛,并隐隐然疼痛牵引肩背,甚至不可以动,一动就会使咳嗽加剧。肾咳的症状,咳则腰背互相牵引作痛,甚至咳吐痰涎。
黄帝道:六腑咳嗽的症状如何?是怎样受病的?岐伯说:五脏咳嗽日久不愈,就要传移于六腑。例如脾咳不愈,则胃就受病;胃咳的症状,咳而呕吐,甚至呕出蛔虫。肝咳不愈,则胆就受病,胆咳的症状是咳而呕吐胆汁。肺咳不愈,则大肠受病,大肠咳的症状,咳而大便失禁。心咳不愈,则小肠受病,小肠咳的症状是咳而放屁,而且往往是咳嗽与失气同时出现。肾咳不愈,则膀胱受病;膀胱咳的症状,咳而遗尿。以上各种咳嗽,如经久不愈,则使三焦受病,三焦咳的症状,咳而腹满,不想饮食。凡此咳嗽,不论由于那一脏腑的病变,其邪必聚于胃,并循着肺的经脉而影响及肺,才能使人多痰涕,面部浮肿,咳嗽气逆。
黄帝道:治疗的方法怎样?岐伯说:治五脏的咳,取其俞穴;治六腑的咳,取其合穴;凡咳而浮肿的,可取有关脏腑的经穴而分治之。黄帝道:讲得好!
\chapter{举痛论篇第三十九}
黄帝问曰:余闻善言天者,必有验于人;善言古者,必有合于今;善言人者,必有厌于己。如此则道不惑而要数极,所谓明也。今余问于夫子,令言而可知,视而可见,扪而可得,令验于己而发蒙解惑,可得而闻乎?岐伯再拜稽首对曰:何道之问也?帝曰:愿闻人之五脏卒痛,何气使然?岐伯对曰:经脉流行不止,环周不休。寒气入经而稽迟,泣而不行,客于脉外则血少,客于脉中则气不通,故卒然而痛。
帝曰:其痛或卒然而止者,或痛甚不休者,或痛甚不可按者,或按之而痛止者,或按之无益者,或喘动应手者,或心与背相引而痛者,或胁肋与少腹相引而痛者,或腹痛引阴股者,或痛宿昔而成积者,或卒然痛死不知人,有少间复生者,或痛而呕者,或腹痛而后泄者,或痛而闭不通者。凡此诸痛,各不同形,别之奈何?
岐伯曰:寒气客于脉外则脉寒,脉寒则缩踡,缩踡则脉绌急,绌急则外引小络,故卒然而痛,得炅则痛立止;因重中于寒,则痛久矣。寒气客于经脉之中,与炅气相薄则脉满,满则痛而不可按也。寒气稽留,炅气从上,则脉充大而血气乱,故痛甚不可按也。寒气客于肠胃之间,膜原之下,血不得散,小络急引,故痛;按之则血气散,故按之无益也。寒气客于冲脉,冲脉起于关元,随腹直上,寒气客则脉不通,脉不通则气因之,故喘动应手矣。寒气客于背俞之脉,则脉泣,脉泣则血虚,血虚则痛,其俞注于心,故相引而痛。按之则热气至,热气至则痛止矣。寒气客于厥阴之脉,厥阴之脉者,络阴器,系于肝,寒气客于脉中,则血泣脉急,故胁肋与少腹相引痛矣。厥气客于阴股,寒气上及少腹,血泣在下相引,故腹痛引阴股。寒气客于小肠膜原之间,络血之中,血泣不得注于大经,血气稽留不得行,故宿昔而成积矣。寒气客于五藏,厥逆上泄,阴气竭,阳气未入,故卒然痛死不知人,气复反则生矣。寒气客于肠胃,厥逆上出,故痛而呕也。寒气客于小肠,小肠不得成聚,故后泄腹痛矣。热气留于小肠,肠中痛,瘅热焦渴,则坚干不得出,故痛而闭不通矣。
帝曰:所谓言而可知者也。视而可见奈何?岐伯曰:五藏六府,固尽有部,视其五色,黄赤为热,白为寒,青黑为痛,此所谓视而可见者也。帝曰:扪而可得奈何?岐伯曰:视其主病之脉,坚而血及陷下者,皆可扪而得也。帝曰:善。
余知百病生于气也。怒则气上,喜则气缓,悲则气消,恐则气下,寒则气收,灵则气泄,惊则气乱,劳则气耗,思则气结,九气不同,何病之生?岐伯曰:怒则气逆,甚则呕血及飧泄,故气上矣。喜则气和志达,荣卫通利,故气缓矣。悲则心系急,肺布叶举,而上焦不通,荣卫不散,热气在中,故气消矣。恐则精却,却则上焦闭,闭则气还,还则下焦胀,故气不行矣。寒则腠理闭,气不行,故气收矣。灵则腠理开,荣卫通,汗大泄,故气泄。惊则心无所倚,神无所归,虑无所定,故气乱矣。劳则喘息汗出,外内皆越,故气耗矣。思则心有所存,神有所归,正气留而不行,故气结矣。
举痛论篇第三十九参考译文
黄帝问道:我听说善于谈论天道的,必能应验于人事;善于谈论历史的,必能应合于今事;善于谈论人事的,必能结合自己的情况。这样,才能掌握事物的规律而不迷惑,了解事物的要领极其透彻,这就是所谓明达事理的人。现在我想请教先生,将问诊所知,望诊所见,切诊所得的情况告诉我,使我有所体验,启发蒙昧,解除疑惑,你能告诉我呢?岐伯再次跪拜回答说:你要问的是哪些道理呢?黄帝说:我想听听人体的五脏突然作痛,是什麽邪气造成的呢?岐伯回答说:人体经脉中的气血流行不止,如环无端,如果寒邪侵入了经脉,则经脉气血的循行迟滞,凝涩而不畅行,故寒邪侵袭于经脉内外,则使经脉凝涩而血少,脉气留止而不通,所以突然作痛。
黄帝说:其疼痛有突然停止的,有疼得很剧烈而不停止的,有痛得很剧烈而不能按压的,有按压而疼痛停止的,有按压也不见缓解的,有疼痛跳动应手的,有心和背部相互牵引而痛的,有胁肋和腹相互牵引而痛的,有腹痛牵引阴股的,有疼痛日久而成积聚的,有突然疼痛昏厥如死不知人事稍停片刻而又清醒的,有痛而呕吐的,有腹痛而后泄泻的,有痛而大便闭结不通的,以上这些疼痛的情况,其病形各不相同,如何加以区别呢?岐伯说:寒协侵袭于脉外,则经脉受寒,经脉受寒则经脉收缩不伸,收缩不伸则屈曲拘急,因而牵引在外的细小脉络,内外引急,故突然发生疼痛,如果得到热气,则疼痛立刻停止。假如再次感受寒邪,卫阳受损就会久痛不止。寒邪侵袭经脉之中,和人体本身的热气相互搏争,则经脉充满,脉满为实,不任压迫,故痛而不可按。寒邪停留于脉中,人体本身的热气则随之而上,与寒邪相搏,使经脉充满,气血运行紊乱,故疼痛剧烈而不可触按。寒协侵袭于肠胃之间,膜原之下,以致血气凝涩而不散,细小的脉络拘急牵引,所以疼痛;如果以手按揉,则血气散行,故按之疼痛停止。寒邪侵袭于侠脊之脉,由于邪侵的部位较深,按揉难以达到病所,故按揉也无济于事。寒邪侵袭于冲脉之中,冲脉是从小腹关员穴开始,循腹上行,如因寒气侵入则冲脉不通,脉不通则气因之鼓脉欲通,故腹痛而跳动应手。寒邪侵于背俞足太阳之脉,则血脉流行滞涩,脉涩则血虚,血虚则疼痛,因足太阳脉背俞与心相连,故心与背相引而痛,按揉能使热气来复,热气来复则寒邪消散,故疼痛即可停止。寒邪侵袭于足厥阴之脉,足厥阴之脉循股阴入毛中,环阴器抵少腹,布胁肋而属于肝,寒邪侵入于脉中,则血凝涩而脉紧急,故胁肋与少腹牵引作痛。寒厥之气客于阴股。寒邪侵袭于小肠膜原之间、络血之中,使络血凝涩不能流注于大经脉,血气留止不能畅行,故日久便可结成积聚。寒邪侵袭于五脏,迫使五脏之气逆而上行,以致脏气上越外泄,阴气竭于内,阳气不得入,阴阳暂时相离,故突然疼痛昏死,不知人事;如果阳气复返,阴阳相接,则可以苏醒。寒协侵袭于肠胃,迫使肠胃之气逆而上行,故出现疼痛而呕吐。寒协复袭于小肠,小肠为受盛之腑,因寒而阳气不化,水谷不得停留,故泄泻而腹痛。如果是热邪留蓄于小肠,也可发生肠中疼痛,由于内热伤津而唇焦口渴,粪便坚硬难以排出,故腹痛而大便闭结不通。
黄帝说:以上所说从问诊中可以了解。至于望诊可见又是怎样的呢?岐伯说:五脏六腑在面部各有所属部位,望面部五色的变化就可以诊断疾病,如黄色赤色主热,白色主寒,青色黑色主痛,这就是通过望诊可以了解的。
黄帝说:用手切诊而知病情是怎样的呢?岐伯说:看他主病的经脉,然后以手循按,如果脉坚实的,是有邪气结聚;属气血留滞的,荦脉必充盛而高起;如果脉陷下的,是气血不足,多属阴证。这些都是可以用手扪切切按循而得知的。
黄帝说:好。我已知道许多疾病的发生,都是由气机失调引起的,如暴怒则气上逆,喜则气舒缓,悲哀则所消沉,恐惧则气下却,遇寒则气收敛,受热则气外泄,受惊则气紊乱,过劳则气耗散,思虑则气郁结。这九种气的变化各不相同,会发生怎样的疾病呢?岐伯说:大怒则使肝气上逆,血随气逆,甚则呕血,或肝气乘脾发生飧泄所以说是气上。喜则气和顺而志意畅达,容卫之气通利,所以说是气缓。悲哀太过则心系急迫,但悲为肺志,悲伤肺则肺叶张举,上焦虽之闭塞不通,营卫之气得不到布散,热气喻闭于中而耗损肺气,所以说是气消。恐惧则使精气下却,精气下却则升降不交,故上焦闭塞,上焦闭塞则气还归于下,气郁于下则下焦胀满,所以说“恐则气下”。寒冷之气侵袭人体,则使腠理闭密,容卫之气不得畅行而收敛于内,所以说是气收。火热之气能使人腠理开放,容卫通畅,汗液大量外出,致使气随津泄,所以说是气泄。受惊则心悸动无所依附,神志无所归宿,心中疑虑不定,所以说是气乱。劳役过度则气动喘息,汗出过多,喘则内气越,汗出过多则外气越,内外之气皆泄越,所以说是气耗。思则精力集中,心有所存,神归一处,以致正气留结而不运行,所以说是气结。
\chapter{腹中论篇第四十}
黄帝问曰:有病心腹满,旦食则不能暮食,此为何病?岐伯对曰:名为鼓胀。帝曰:治之奈何?岐伯曰:治之以鸡矢醴,一剂知,二剂已。帝曰:其时有复发者,何也?岐伯曰:此饮食不节,故时有病也。虽然其病且已,时故当病,气聚于腹也。
帝曰:有病胸胁支满者,妨于食,病至则先闻腥臊臭,出清液,先唾血,四支清,目眩,时时前后血,病名为何?何以得之?岐伯曰:病名血枯。此得之年少时有所大脱血;若醉入房中,气竭肝伤,故月事衰少不来也。帝曰:治之奈何?复以何术?岐伯曰:以四乌鰂骨一藘茹二物并合之,丸以雀卵,大如小豆;以五丸为后饭,饮以鲍鱼汁,利肠中及伤肝也。
帝曰:病有少腹盛,上下左右皆有根,此为何病?可治不?岐伯曰:病名曰伏梁。帝曰:伏梁何因而得之?岐伯曰:裹大脓血,居肠胃之外,不可治,治之每切按之致死。帝曰:何以然?岐伯曰:此下则因阴,必下脓血,上则迫胃脘,生鬲,侠胃脘内痈。此久病也,难治。居齐上为逆,居齐下为从,勿动亟夺。论在刺法中。
帝曰:人有身体髀股胫皆肿,环脐而痛,是为何病?岐伯曰:病名伏梁,此风根也。其气溢于大肠,而著于肓,肓之原在脐下,故环脐而痛也。不可动之,动之为水溺涩之病。
帝曰:夫子数言热中、消中,不可服高梁、芳草、石药,石药发癫,芳草发狂。夫热中、消中者,皆富贵人也,今禁高梁,是不合其心,禁芳草、石药,是病不愈,愿闻其说。岐伯曰:夫芳草之气美,石药之气悍,二者其气急疾坚劲,故非缓心和人,不可以服此二者。帝曰:不可以服此二者,何以然?岐伯曰:夫热气栗悍,药气亦然,二者相遇,恐内伤脾。脾者土也,而恶木,服此药者,至甲乙日更论。帝曰:善。
有病膺肿颈痛,胸满腹胀,此为何病?何以得之?岐伯曰:名厥逆。帝曰:治之奈何?岐伯曰:灸之则喑,石之则狂,须其气并,乃可治也。帝曰:何以然?岐伯曰:阳气重上,有余于上,灸之则阳气入阴,入则喑;石之则阳气虚,虚则狂。须其气并而治之,可使全也。帝曰:善。何以知怀子之且生也?岐伯曰:身有病无邪脉也。
帝曰:病热而有所痛者,何也?岐伯曰:病热者,阳脉也。以三阳之动也,人迎一盛少阳,二盛太阳,三盛阳明。入阴也,夫阳入于阴,故病在头与腹,乃脘胀而头痛也。帝曰:善。
腹中论篇第四十参考译文
黄帝问道:有一种心腹胀满的病,早晨吃了饭晚上就不能再吃,这是什麽病呢?岐伯回答说:这叫鼓胀病。黄帝说:如何治疗呢?岐伯说:可用鸡失醴来治疗,一刺就能见效,两济病就好了。黄帝说:这种病有时还会复发是为什麽呢?岐伯说:这是因为饮食不注意,所以病有时复发。这种情况多是正当疾病将要痊愈时,而又复伤于饮食,使邪气复聚于腹中,因此鼓胀就会再发。
黄帝说:有一种胸胁满的病,妨碍饮食,发病时先闻到腥臊的气味,鼻流清涕,先唾血,四肢清冷,头目眩晕,时常大小便出血,这种病叫什麽名字?是什麽原因引起的?岐伯说:这种病的名字叫血枯,得病的原因是在少年的时候患过大的失血病,使内脏有所损伤,或者是醉后肆行房事,使肾气竭,肝血伤,所以月经闭止而不来。黄帝说:怎样治疗呢?要用什麽方法使其恢复?岐伯说:用四份乌贼骨,一份藘菇,二药混合,以雀卵为丸,制成如小豆大的丸药,每次服五丸,饭前服药,饮以鲍鱼汁。这个方法可以通利肠道,补益损伤的肝脏。
黄帝说:病有少腹坚硬盛满,上下左右都有根蒂,这是什麽病呢?可以治疗吗?岐伯说:小腹部裹藏着大量脓血,居于肠胃之外,不可能治愈的。在诊治时,不宜重按,每因重按而致死。黄帝说:为什麽会这样呢?岐伯说;此下为小腹及二阴,按摩则使脓血下出;此上是胃脘部,按摩则上迫胃脘,能使横膈与胃脘之间发生痈此为根深蒂固的久病,故难治疗。一般地说,这种病生在脐上的为逆症,生在脐下的为顺症,切不可急切按摩,以使其下夺。关于本病的治法,在《刺法》中有所论述。黄帝说:有人身体髀、股、小腿等部位都发肿,且环绕脐部疼痛,这是什麽病呢?岐伯说:病的名字叫伏梁,这是由于宿受风寒所致。风寒之气充溢于大肠而留着于肓,肓的根源在脐下气海,所以绕脐而痛。这种病不可用攻下的方法治疗,如果误用攻下,就会发生小便涩滞不利的病。
黄帝说:先生屡次说患热中、消中病的,不能吃肥甘厚,也不能吃芳香药草和金石药,因为金石药物能使人法癫,芳草药物能使人发狂。患热中、消中病的,多是富贵之人,现在如禁止他们吃肥甘厚味,则不适合他们的心理,不使用芳草石药,又治不好他们的病,这种情况如何处理呢?我愿意听听你的意见。岐伯说:芳草之气多香窜,石药之气多猛悍,这两类药物的性能都是疾坚劲的,若非性情和缓的人,不可以服用这两类药物。黄帝说:不可以服用这两类药物,是什麽道理呢?岐伯说:因为这种人平素嗜食肥甘而生内热,热气本身是慓悍的,药物的性能也是这样,两者遇在一起,恐怕会损伤人的脾气,脾属木而恶土,所以服用这类药物,在甲日和已日肝木主令时,病情就会更加严重。
黄帝说:好。有人患膺肿颈痛,胸满腹胀,这是什麽病呢?是什麽原因引起的?岐伯说:病名叫厥逆。黄帝说:怎样治疗呢?岐伯说:这种病如果用灸法便会失音,用针刺就会发狂,必须等到阴阳之气上下相合,才能进行治疗。黄帝说:为什麽呢?岐伯说:上本为阳,阳气又逆于上,重阳在上,则有余于上,若再用灸法,是以火济火,阳极乘阴,阴不能上承,故发生失音;若用砭石针刺,阳气随刺外泄则虚,神失其守,故发生神志失常的狂证;必须在阳气从上下降,阴气从下上升,阴阳二气交并以后再进行治疗,才可以获得痊愈。
黄帝说:好。妇女怀孕且要生产是如何知道的呢?岐伯说:其身体似有某些病的征候,但不见有病脉,就可以诊为妊娠。
黄帝说:有病发热而兼有疼痛的是什麽原因呢?岐伯说:阳脉是主热证的,外感发热是三阳受邪,故三阳脉动甚。若人迎一倍于寸口,是病在太阳;大三倍于寸口,是病在阳明。三阳既毕,则传入于三阴。病在三阳,则发热头痛,今传入于三阴,故又出现腹部胀满,所以病人有腹胀和头痛的症状。黄帝说:好。
\chapter{刺腰痛篇第四十一}
足太阳脉令人腰痛,引项脊尻背如重状,刺其郄中太阳正经出血,春无见血。少阳令人腰痛,如以针刺其皮中,循循然不可以俯仰,不可以顾,刺少阳成骨之端出血,成骨在膝外廉之骨独起者,夏无见血。
阳明令人腰痛,不可以顾,顾如有见者,善悲,刺阳明于胫前三痏,上下和之出血,秋无见血。
足少阴令人腰痛,痛引脊内廉,刺少阴于内踝上二痏,春无见血,出血太多,不可复也。
厥阴之脉,令人腰痛,腰中如张弓弩弦,刺厥阴之脉,在腿踵鱼腹之外,循之累累然,乃刺之,其病令人善言,默默然不慧,刺之三痏。
解脉令人腰痛,痛引肩,目@@然,时遗溲,刺解脉,在膝筋肉分间郄外廉之横脉出血,血变而止。解脉令腰痛如引带,常如折腰状,善恐;刺解脉,在郄中结络如黍米,刺之血射以黑,见赤血而已。
同阴之脉令人腰痛,痛如小锤居其中,怫然肿,刺同阴之脉,在外踝上绝骨之端,为三痏。
衡络之脉令人腰痛,不可以俯仰,仰则恐仆,得之举重伤腰,衡络绝,恶血归之,刺之在郄阳筋之间上郄数寸衡居,为二痏出血。
会阴之脉令人腰痛,痛上漯漯然,汗干令人欲饮,饮已欲走,刺直阳之脉上三痏,在蹻上郄下五寸横居,视其盛者出血。
飞阳之脉令人腰痛,痛上怫怫然,甚则悲以恐,刺习阳之脉,在内踝上五寸,少阴之前,与阴维之会。
昌阳之脉令人腰痛,痛为膺,目疏疏然,甚则反折,舌卷不能言,刺内筋为二痏,在内踝上大筋前,太阴后上踝二寸所。
散脉,令人腰痛而热,热甚生烦,腰下如横木居其中,甚则贵溲;刺散脉,在膝前骨肉分间,络外廉束脉,为三痏。
肉里之脉令人腰痛,不可以咳,咳则筋缩急,刺肉里之脉为二痏,在太阳之外,少阳绝骨之后。
腰痛侠脊而痛至头几几然,目疏疏欲僵仆,刺足太阳郄中出血。腰痛上寒,刺足太阳、阳明;上热,刺足厥阴;不可以俯仰,刺足少阳;中热而喘,刺足少阴,刺郄中出血。
腰痛,上寒不可顾,刺足阳明;上热,刺足厥阴;不可以俯仰,刺足少阳;中热而喘,刺足少阴.
大便难,刺足少阴。少腹满,刺足厥阴。如折不可以俯仰,不可举,刺足太阳。引脊内廉,刺足少阴。腰痛引少腹控肋,不可以仰,刺腰尻交者,两髁胂上,以月生死为痏数,发针立已,左取右,右取左。
刺腰痛篇第四十一参考译文
足太阳经脉发病使人腰痛,痛时牵引项脊尻背,好象担负着沉重的东西一样,治疗时应刺其合穴委中,即在委中穴处刺出其恶血。若在春季不要刺出其血。
足少阳经脉发病使人腰痛,痛如用针刺于皮肤中,逐渐加重不能前后俯仰,并且不能左右回顾。治疗时应刺足少阳经成骨的起点出血,成骨即膝外侧高骨突起处,若在夏季则不要刺出其血。
阳明经脉发病而使人腰痛,颈项不能转动回顾,如果回顾则神乱目花犹如妄见怪异,并且容易悲伤,治疗时应刺足阳明经在胫骨前的足三里穴三次,并配合上、下巨虚穴刺出其血,秋季则不要刺出其血。
足少阴脉发病使人腰痛,痛时牵引到脊骨的内侧,治疗时应刺足少阴经在内踝上的复溜穴两次,若在春季则不要刺出其血。如果出血太多,就会导致肾气损伤而不易恢复。
厥阴经脉发病使人腰痛,腰部强急如新张的弓弩弦一样,治疗时应刺阻厥阴的经脉,其部位在腿肚和足根之间鱼腹之外的蠡沟穴处,摸之有结络累累然不平者,就用针刺之,如果病人多言语或沉默抑郁不爽,可以针刺三次。
解脉发病使人腰痛,痛时会牵引到肩部,眼睛视物不清,时常遗尿,治疗时应取解脉在膝后大筋分肉间(委中穴)外侧的委阳穴处,有血络横见,紫黑盛满,要刺出其血直到血色由紫变红才停止。
解脉发病使人腰痛,好象有带子牵引一样,常好象腰部被折断一样,并且时常有恐惧的感觉,治疗时应刺解脉,在郄中有络脉结滞如黍米者,刺之则有黑色血液射出,等到血色变红时即停止。
同阴之脉发病使人腰痛,痛时胀闷沉重,好象有小锤在里面敲击,病处突然肿胀,治疗时应刺同阴之脉,在外踝上绝骨之端的阳辅穴处,针三次。
衡络之脉发病使人腰痛,不可以前俯和后仰,后仰则恐怕跌倒,这种病大多因为用力举重伤及腰部,使横络阻绝不通,淤血滞在里。治疗时应刺委阳大筋间上行数寸处的殷门穴,视其血络横居满者针刺二次,令其出血。
会阴之脉发病使人腰痛,痛则汗出,汗止则欲饮水,并表现着行动不安的状态,治疗时应刺直阳之脉上三次,其部位在阳蹻申脉穴上,足太阳郄中穴下五寸的承筋穴处,视其左右有络脉横居、血络盛满的,刺出其血。
本条经文,注家说法亦颇不一,姑从王冰之说以释之。脱阴为腰痛之文,待考。
昌阳之脉发病使人腰痛,疼痛牵引胸膺部,眼睛视物昏花,严重时腰背向后反折,舌卷短不能言语,治疗时应取筋内侧的复溜穴刺二次,其穴在内踝上大筋的前面,足太阴经的后面,内踝上二寸处。
散脉发病使人腰痛而发热,热甚则生心烦,腰下好象有一块横木梗阻其中,甚至会发生遗尿,治疗时应刺散脉下俞之巨虚上廉和巨虚下廉,其穴在膝前外侧骨肉分间,看到有青筋缠束的脉络,即用针刺三次。
肉里之脉发病使人腰痛,痛得不能咳嗽,咳嗽则筋劢拘急挛缩,治疗时应刺肉里之脉二次,其穴在阻太阳的外前方,阻少阳绝骨之端的后面。
腰痛挟脊背而痛,上连头部拘强不舒,眼睛昏花,好象要跌倒,治疗时应刺足太阳经的委中穴出血。
腰痛时有寒冷感觉的,应刺足太阳经和足阳明经,以散阳分之阴邪;有热感觉的,应刺足厥阴经,以去阴中之风热;腰痛不能俯仰的,应刺足少阳经,以转枢机关;若内热而喘促的,应刺足少阴经,以壮水制火,并刺委中的血络出血。
腰痛时,感觉上部寒冷,头项强急不能回顾的,应刺足阳明经;感觉上部火热的,应刺足太阴经;感觉内里发热兼有气喘的,应刺足少阴经。大便困难的,应刺足少阴经。少腹胀满的,应刺足厥阴经。腰痛犹如折断一样不可前后俯仰,不能举动的,应刺足太阳经。腰痛牵引脊骨内侧的,应刺足少阴经。
腰痛时牵引少腹,引动季胁之下,不能后仰,治疗时应刺腰尻交处的下髎穴,其部位在两踝骨下挟脊两旁的坚肉处,针刺时以月亮的盈亏计算针刺的次数,针后会立即见效,并采用左痛刺右侧、右痛刺左侧的方法。
\chapter{风论篇第四十二}
黄帝问曰:风之伤人也,或为寒热,或为热中,或为寒中,或为疠风,或为偏枯,或为风也;其病各异,其名不同,或内至五脏六腑,不知其解,愿闻其说。
岐伯对曰:风气藏于皮肤之间,内不得通,外不得泄;风者善行而数变,腠理开则洒然寒,闭则热而闷,其寒也则衰食饮,其热也则消肌肉,故使人颤栗而不能食,名曰寒热。
内气与阳明入胃。循脉而上至目内眦,其人肥则风气不得外泄,则为热中而目黄;人瘦,则外泄而寒,则为寒中而泣出。
风气与太阳俱入,行诸脉俞,散于分肉之间,与卫气相干,其道不利,故使肌肉愤胀而有疡;卫气有凝而不行,故其肉有不仁也。疠者,有荣气热腐,其气不清,故使其鼻柱坏而色败,皮肤疡溃。风寒客于脉而不去,名曰疠风,或名曰寒热。
以春甲乙伤于风者为肝风;以夏丙丁伤于风者为心风;以季夏戊己伤于邪者为脾风;以秋庚辛中于邪者为肺风;以冬壬癸中于邪者为肾风。
风中五脏六腑之俞,亦为脏腑之风,各入其门户所中,则为偏风。风气循风府而上,则为脑风;风入系头,则为目风眼寒;饮洒中风,则为漏风;入房汗出中风,则为内风;新沐中风,则为道风;久风入中,则为肠风飧泄;外在腠理,则为泄风。故风者百病之长也,至其变化乃为他病也,无常方,然致有风气也。
帝曰:五脏风之形状不同者何?愿闻其诊及其病能。岐伯曰:肺风之状,多汗恶风,色鉼然白,时咳短气,昼日则差,暮则甚,诊在眉上,其色白。心风之状,多汗,恶风,焦绝,善怒吓,赤色,病甚则言不可快,诊在口,其色赤。
肝风之状,多汗恶风,善悲,色微苍,嗌干善怒,时憎女子,诊在目下,其色青。
脾风之状,多汗恶风,身体怠惰,四支不欲动,色薄微黄,不嗜食,诊在鼻上,其色黄。
肾风之状,多汗恶风,面疣然浮肿,脊痛不能正立,其色炱,隐曲不利,诊在肌上,其色黑。
胃风之状,颈多汗恶风,食饮不下,鬲塞不通,腹善满,失衣则脘胀,食寒则泄,诊形瘦而腹大。
首风之状,头面多汗恶风,当先风一日则病甚,头痛不所以出内,至其风日,则病少愈。漏风之状,或多汗,常不可单衣,食则汗出,甚则身汗,喘息恶风,衣常濡,口干善渴,不能劳事。
泄风之状,多汗,汗出泄衣上,口中干上渍,其风不能劳事,身体尽痛则寒。帝曰:善。
风论篇第四十二参考译文
黄帝问道:风邪侵犯人体,或引起寒热病,或成为热中病,或成为寒中病,或引起疠风病,或引起偏枯病,或成为其他风病。由于病变表现不同,所以病名也不一样,甚至侵入到五脏六腑,我不知如何解释,愿听你谈谈其中的道理。岐伯说:风邪侵犯人体常常留滞于皮肤之中,使腠理开合失常,经脉不能通调于内,卫气不能发泄于外;然而风邪来去迅速,变化多端,若使腠理开张则阳气外泄而洒淅恶寒,若使腠理闭塞则阳气内郁而身热烦闷,恶寒则引起饮食减少,发热则会使肌肉消瘦,所以使人振寒而不能饮食,这种病称为寒热病。风邪由扬名经入胃,循经脉上行到目内眦,假如病人身体肥胖,腠理致密,则风邪不能向外发泄,稽留体内郁而化热,形成热中病,症见目珠发黄;假如病人身体瘦弱,腠理疏松,则阳气外泄而感到畏寒,形成寒中病,症见眼泪自出。风邪由太阳经侵入,偏行太阳经脉及其腧穴,散布在分肉之间,与卫气相搏结,使卫气运行的道路不通利,所以肌肉肿胀高起而产生疮疡;若卫气凝涩而不能运行,则肌肤麻木不知痛痒。疠风病是营气因热而腐坏,血气污浊不清所致,所以使鼻柱蚀坏而皮色衰败,皮肤生疡。病因是风寒侵入经脉稽留不去,病名叫疠风。
在春季或甲日、已日感受风邪的,形成肝风;在夏季或丙日、丁日感受风邪的,形成心风;再长夏或戊日、己日感受风邪的,形成脾风;在秋季或庾日、辛日感受风邪的,形成肺风;在冬季或壬日、癸日感受风邪的,形成肾风。
风邪侵入五脏六腑的俞穴,沿经内传,也可成为五脏六腑的风病。逾穴是机体与外界相通的门户,若风邪从其血气衰弱场所入侵,或左或右;偏着于一处,则成为偏风病。
风邪由风府穴上行入脑,就成为脑风病;风邪侵入头部累及目系,就成为目风病,两眼畏惧风寒;饮酒之后感受风邪,成为漏风病;行房汗出时感受风邪,成为内风病;刚洗过头时感受风邪成为首风病;风邪久留不去,内犯肠胃,则形成肠风或飧泄病;风邪停留于腠理,则成为泄风病。所以,风邪是引起多种疾病的首要因素。致于它侵入人体后产生变化,能引起其他各种疾病,就没有一定常规了,但其病因都风邪入侵。
黄帝问道:五脏风证的临床表现有何不同?希望你讲讲诊断要点和病态表现。岐伯回答道:肺风的症状,是多汗恶风,面色淡白,不时咳嗽气短,白天减轻,傍晚加重,诊察时要注意眉上部位,往往眉间可出现白色。心风的症状,是多汗恶风,唇舌焦躁,容易发怒,面色发红,病重则言语謇涩,诊察时要注意舌部,往往舌质可呈现红色。肝风的症状,是多汗恶风,常悲伤,面色微青,易发怒,有时厌恶女性,诊察时要注意目下,往往眼圈可发青色。脾风的症状,是多汗恶风,身体疲倦,四肢懒于活动,面色微微发黄,食欲不振,诊察时要注意鼻尖部,往往鼻尖可出现黄色。肾风的症状,是多汗恶风,颜面疣然而肿,腰脊痛不能直立,面色黑加煤烟灰,小便不利,诊察时要注意颐部,往往颐部可出现黑色。胃风的症状,是项部多汗恶风,腰脊痛不能直立,面色黑加煤烟灰,小便不利,诊察时要注意颐部,往往颐部可出现黑色。胃风的症状,是颈部多汗,恶风,吞咽饮食困难,隔塞不通,腹部易作胀满,如少穿衣,腹即脘胀,如吃了寒凉的食物,就发生泄泻,诊察时可见形体瘦削而腹部胀大。首风的症状,是头痛,面部多汗,恶风,每当起风的前一日病情就加重,以至头痛得不敢离开室内,待到起风的当日,则痛热稍轻。漏风的症状,是汗多,不能少穿衣服,进食即汗出,甚至是自汗出,喘息恶风,衣服常被汗侵湿,口干易渴,不耐劳动。泄风的症状,是多汗,汗出湿衣,口中干燥,上半身汗出如水渍一样,不耐劳动,周身疼痛发冷。黄帝道:讲得好!
\chapter{痹论篇第四十三}
黄帝问曰:痹之安生?岐伯对曰:风寒湿三气杂至,合而为痹也。其风气胜者为行痹;寒气胜者为痛痹;湿气胜者为著痹也。帝曰:其有五者何也?岐伯曰:以冬遇此者为骨痹;以春遇此者为筋痹;以夏遇此者为脉痹;以至阴遇此者为肌痹;以秋遇此者为皮痹。帝曰:内舍五脏六腑,何气使然?岐伯曰:五脏皆有合,病久而不去者,内舍于其合也。故骨痹不已,复感于邪,内舍肾;筋痹不已,复感于邪,内舍于肝;脉痹不已,复感于邪,内舍于心;肌痹不已,复感于邪,内舍于脾;皮痹不已,复感于邪,内舍于肺。所谓痹者,各以其时重感于风寒湿之气也。
凡痹之客五脏者:肺痹者,烦满喘而呕。心痹者,脉不通,烦则心下鼓,暴上气而喘。嗌干善噫,厥气上则恐。肝痹者,夜卧则惊,多饮数小便,上为引如怀。肾痹者,善胀,尻以代踵,脊以代头。脾痹者,四支解堕,发咳呕汁,上为大塞。肠痹者,数饮而出不得,中气喘争,时发飧泄。胞痹者,少腹膀胱按之内痛,若沃以汤,涩于小便,上为清涕。
阴气者,静则神藏,躁则消亡。饮食自倍,肠胃乃伤。淫气喘息,痹聚在肺;淫气忧思,痹聚在心;淫气遗溺,痹聚在肾;淫气乏竭,痹聚在肝;淫气肌绝,痹聚在脾。诸痹不已,亦益内也。其风气胜者,其人易已也。
帝曰:痹,其时有死者,或疼久者,或易已者,其故何也?岐伯曰:其入脏者死,其留连筋骨间者疼久,其留皮肤间者易已。
帝曰:其客于六腑者,何也?岐伯曰:此亦其食饮居处,为其病本也。六腑亦各有俞,风寒湿气中其俞,而食饮应之,循俞而入,各舍其腑也。
帝曰:以针治之奈何?岐伯曰:五藏有俞,六腑有合,循脉之分,各有所发,各随其过,则病瘳也。
帝曰:荣卫之气。亦令人痹乎?岐伯曰:荣者。水谷之精气也,和调于五藏,洒陈于六腑,乃能入于脉也,故循脉上下,贯五藏,络六腑也。
卫者,水谷之悍气也,其气剽疾滑利,不能入于脉也,故循皮肤之中,分肉之间,熏于盲膜,散于胸腹。逆其气则病,从其气则愈。不与风寒湿气合,故不为痹。帝曰:善!
痹,或痛,或不痛,或不仁,或寒,或热,或燥,或湿,其故何也?岐伯曰:痛者,寒气多也,有寒,故痛也。其不痛、不仁者,病久入深,荣卫之涩,经络时疏,故不通,皮肤不营,故为不仁。其寒者,阳气少,阴气多,与病相益,故寒也。其热者,阳气多,阴气少,病气胜,阳遭阴,故为痹热。其多汗而濡者,此其逢湿甚也,阳气少,阴气盛,两气相感,故汗出而濡也。
帝曰:夫痹之为病,不痛何也?岐伯曰:痹在于骨则重;在于脉则凝而不流;在于筋则屈不伸;在于肉则不仁;在于皮则寒。故具此五者,则不痛也。凡痹之类,逢寒则虫,逢热则纵。帝曰:善。
痹论篇第四十三参考译文
黄帝问道:痹病是怎样产生的?是多汗恶风回答说:由风、寒、湿三种邪气杂合伤人而形成痹病。其中风邪偏胜的叫行痹,寒邪偏胜的叫痛痹,诗协偏胜的叫着痹。
黄帝问道:痹病又可分为五种,为什麽?岐伯说:在冬天得病称为骨痹;在春天得病的称为筋痹;在夏天得病的称为脉痹;在长夏得病的称为肌痹;在秋天得病的称为皮痹。
黄帝问道:痹病的冰鞋又有内侵而累及五脏六腑的,是什麽道理?岐伯五脏都有与其相合的组织器官,若冰鞋久留不除,就会内犯于相合的内脏。所以,骨痹不愈,再感受邪气,就会内舍于心;肌痹不愈,再感受邪气,就会内舍于脾;皮痹不愈,再感受邪气,就会内舍于肺。总之,这些痹证是各脏在所主季节里重复感受了风、寒、湿气所造成的。
凡痹病侵入到五脏,症状各有不同:肺痹的症状是烦闷胀满,喘逆呕吐,心痹的症状是血脉不通畅,烦躁则心悸,突然气逆上壅而喘息,咽干,易暖气,厥阴上逆则引起恐惧。肝痹的症状是夜眠多惊,饮水多而小便频数,疼痛循肝经由上而下牵引少腹如怀孕之状。肾痹的症状是腹部易作胀,骨萎而足不能行,行步时臀部着地,脊柱曲屈畸行,高耸过头。脾痹的症状是四肢倦怠无力,咳嗽,呕吐清水,上腹部痞塞不通。肠痹的症状是频频饮水而小便困难,腹中肠鸣,时而发生完谷不化的泄泻。膀胱痹的症状是少腹膀胱部位按之疼痛,如同灌了热水似的,小便涩滞不爽,上部鼻流青涕。
五脏精气,安静则精神内守,躁动则易于耗散。若饮食过量,肠胃就要受损。致痹之邪引起呼吸喘促,是痹发生在肺;致痹之邪引起忧伤思虑,是痹发生在心;致痹之痹引起遗尿,是痹发生在肾;致痹之邪引起疲乏衰竭,是痹发生在肝;致痹之邪引起肌肉瘦削,是痹发生在脾。总之,各种痹病日久不愈,病变就会进一步向内深入。其中风邪偏胜的容易痊愈。
黄帝问道:患了痹病后,有的死亡,有的疼痛经久不愈,有的容易痊愈,这是什麽缘故?岐伯说:痹邪内犯到五脏则死,痹邪稽留在筋骨间的则痛久难愈,痹邪停留在皮肤间的容易痊愈。
黄帝问道:痹邪侵犯六腑是何原因?岐伯说:这也是以饮食不节、起居失度为导致腹痹的根本原因。六腑也各有俞穴,风寒湿邪在外侵及它的俞穴,而内有饮食所伤的病理基础与之相应,于是冰鞋就循着俞穴如里,留滞在相应的腑。
黄帝问道:怎样用针刺治疗呢?岐伯说:五脏各有输穴可取,六腑各有合穴可取,循着经脉所行的部位,各有发病的征兆可察,根据冰鞋所在的部位,取相应的输穴或合穴进行针刺,病就可以痊愈了。
黄帝问道:营卫之气亦能使人发生痹病吗?岐伯说:营是水谷所化生的精气,它平和协调地运行于五脏,散布于六腑,然后汇入脉中,所以营卫气循着经脉上下运行,起到连贯五脏,联络六腑的作用。胃是水谷所化生的悍气,它流动迅疾而滑利,不能进入脉中,所以循行于皮肤肌肉之间,熏蒸于肓膜之间,敷布于胸腹之内。若营卫之气的循行逆乱,就会生病,只要营卫之气顺从调和了,病就会痊愈。总的来说,营卫之气若不于风寒湿邪相合,则不会引起痹病。黄帝说:讲得好!
痹病,有的疼痛,有的不痛,有的麻木不仁,有的表现为寒,有的表现为热,有的皮肤干燥,有的皮肤湿润,这是什麽缘故?岐伯说:痛是寒气偏多,有寒所以才痛。不痛而麻木不仁的,系患病日久,冰鞋深入,营卫之气运行涩滞,致使经络中气血空虚,所以不痛;皮肤得不到营养,所以麻木不仁。表现为寒象的,是由于机体阳气不足,阴气偏盛,阴气助长寒邪之势,所以表现为寒象。表现为热象的,是由于机体阳气偏盛,阴气不足,偏胜的阳气与偏胜的风邪相结合而乘阴分,所以出现热象。多汗而皮肤湿润的,是由于感受邪湿太甚,加之机体阳气不足,阴气偏盛,湿邪与偏盛的阴气相结合,所以汗出而皮肤湿润。
黄帝问道:痹病而不甚疼痛是什麽缘故?岐伯说:痹发生在骨则身重;发生在脉则血凝涩而不畅;发生在筋则曲屈不能伸;发生在肌肉则麻木不仁;发生在皮肤则寒冷。如果有这五种情况,就不甚疼痛。凡痹病一类疾患,遇寒则筋脉拘急,遇热则筋脉弛缓。黄帝道:讲得好!
\chapter{}
\chapter{}
\chapter{}
\chapter{}
\chapter{}
\chapter{}
\chapter{}
\chapter{}
\chapter{}
\chapter{}
\chapter{}
\chapter{}
\chapter{}
\chapter{}
\chapter{}
\chapter{}
\chapter{}
\chapter{}
\chapter{}
\chapter{}
\chapter{}
\chapter{}
\chapter{}
\chapter{}
\chapter{}
\chapter{}
\chapter{}
\chapter{}
\chapter{}
\chapter{}
\chapter{}
\chapter{}
\chapter{征四失论篇第七十八}
黄帝在明堂,雷公侍坐,黄帝日:夫子所通书受事众多矣,试育得失之意,所以得之,所以失之。雷公对日:循经受业,皆言十全,其时有过失者,请闻其事解也。帝日:子年少智未及邪,将言以杂合耶?夫经脉十二,络脉三百六十五,此皆人之所明知,工之所循用也。所以不十全者,精神不专,志意不理,外内相失,故时疑殆。诊不知阴阳逆从之理,此治之一失矣。受师不卒,妄作杂术,谬言为道,更名自功,妄用范石,后遗身咎,此治之二失也。不适贫富贵贱之居,坐之薄厚,形之寒温,不适饮食之宜,不别人之勇怯,不知比类,足以自乱,不足以自明,此之三失也。诊病不问其始,忧患饮食之失节,起居之过度,或伤于毒,不先言此,卒持寸口,何病能中,妄言作名,为粗所穷,此治之四失也。是以世人之语者,驰于里之外,不明尺寸之论,诊无人事。治数之道,从容之熊,坐持寸口,诊不中五脉,百病所起,始以自怨,遗师其咎。是政治不能循理,弃术于市,妄治时愈,愚心自得。呜呼!窈窈冥冥,孰知其道?道之大者,拟于天地,配于四海,汝不知道之谕,受以明为晦。
徵四失论篇第七十八参考译文
黄帝坐在明堂,雷公侍坐于旁,黄帝说:先生所通晓的医书和所从事的医疗工作,已经是很多的了,你试谈谈对医疗上的成功与失败的看法,为什麽能成功,为什麽会失败。雷公说:我遵循医经学习医术,书上都说可以得到十全的效果,但在医疗中有时还是有过失的,请问这应该怎样解释呢?
黄帝说:这是由于年岁轻智力不足,考虑不及呢?还是对众人的学说缺乏分析呢?经脉有十二,络脉有三百六十五,这是人们所知道的,也是医生所遵循应用的。治病所以不能收到十全的疗效,是由于精神不能专一,志意不够条理,不能将外在的脉证与内在的病情综合一起分析,所以时常发生疑惑和危殆。
诊病不知阴阳逆从的道理,这是治病失败的第一个原因。随师学习没有卒业,学术未精,乱用杂术,以错误为真理,变易其说,而自以为功,乱施砭石,给自己遗留下过错,这是治病失败的第二个原因。治病不能适宜于病人的贫富贵贱生活特点、居处环境的好坏、形体的寒温,不能适合饮食之所宜,不区别个性的勇怯,不知道用比类异同的方法进行分析,这种作法,只能扰乱自己的思想,不足以自明,这是治病失败的第三个原因。诊病时不问病人开始发病的情况,及是否曾有过忧患等精神上的刺激,饮食是否失于节制,生活起居是否超越正常规律,或者是否曾伤于毒,如果诊病时不首先问清楚这些情况,便仓促去诊视寸口。怎能诊中病情,只能是乱言病名,使病为这种粗律治疗的作风所困,这是治病失败的第四个原因。
所以社会上的一些医生,虽学道于千里之外,但却不明白尺寸的道理,诊治疾病,不知参考人事。更不知诊病之道应以能作到比类从容为最宝贵的道理,只知诊察寸口。这种作法,既诊不中五脏之脉,更不知疾病的起因,开始埋怨自己的学术不精,继而归罪于老师传授不明。所以治病如果不能遵循医理,必为群众所不信任,乱治中偶然治愈疾病,不知是侥幸,反自鸣得意。啊!医道之精微深奥,有谁能彻底了解其中的道理?!医道之大,可以比拟于天地,配于四海,你若不能通晓道之教谕,则所接受之道理,虽很明白,必反成暗晦不明。
\chapter{阴阳类论篇第七十九}
孟春始至,黄帝燕坐,临观八极,正八风之气,而问雷公曰:阴阳之类,经脉之道,五中所主,何藏最贵?雷公对日:春甲乙青,中主肝,治七十二日,是①热;原作“熟”O
脉之主时,臣以其藏最贵。帝曰:却念上下经,阴阳从容,于所言资,最其下也。雷公致斋七比旦复传坐。帝曰:三阳为经,二阳为维,一阳为游部,此知五藏终始。三阴①为表,二阴为里,一明至绝作朔晦,却具合以正其理。雷公日:受业未能明。帝曰:所谓三阳考,太阳为经,三阳脉至手太阴,弦浮而不沉,决以度,察以心,合之阴阳之论;所谓二阳者,阳明也,至手太阴,弦而沉急不鼓;灵至以病皆死;一阳者,少阳也,至手太阴,上连人迎,弦急是不绝,此少阳之病也,专阴则死。三明者,六经之所主也,交于太阳,伏鼓不浮,上空志心;二阴至肺,其气归膀脱,外连脾胃;一阴独至,经绝,气浮不鼓,钩而滑。此六脉者,乍阴乍阳,交属相并,缀通五藏,合于阴阳,先至为主,后至为客。雷公日:臣悉尽意,受传经脉,颂得从容之道,以合从容,不知阴阳,不知雌雄。帝日:三阳为父,二阳为卫,一阳为纪,三阴为母,二明为雌,一阴为独使。二阳一阴,阳明主病,不胜一阴,耍③而动,九窍皆沉。三阳一阴,太阳脉胜,一阴不能止,内乱五藏,外为惊骇。二阴二阳,病在脑,少阴脉沉,胜肺伤牌,外伤四支。二阴二阳皆交至,病在肾,骂署妄行,巅疾为狂。二阴一阳,病出于肾,阴气客游于心院,下空窍提,闭塞不通,四支别离。一阴一阳代绝,此阴气全心,上下无常,出入不知,喉咽干燥,病在立碑。二阳王阴,至阴皆在,阴不过阳,阳气不能止阴,阴阳非绝,浮为血痕,沉为脓附。阴阳皆壮,下至阴阳,上合昭昭,下合冥冥,诊决死生之期,遂合岁首。雷公日:请问短期。黄帝不应。雷公复问。黄帝曰:在经论中。雷公日:请闻短期。黄帝日:冬王月之病,病合于阳者,至春正月脉有死征,皆归于②春。冬三月之病,在理已尽,草与柳叶皆杀,春阴阳皆绝,期在孟春。春三月之病,日阳杀,阴阳皆绝,期在草于。夏三月之病,至明不过十日,阴阳交,期在燃水。秋三月之病,三阳仅起,不治自已。阴阳交合者,方不能坐,坐不能起。王阳独至,期在石水。二阴独至,期在盛水
阴阳类论篇第七十九参考译文
在立春的这一天,黄帝很安闲地坐者,观看八方的远景,侯察八风的方向,向雷公问道:按照阴阳的分析方法和经脉理论,配合五脏主时,你认为哪一脏最贵?雷公回答说:春季为一年之首,属甲乙木,其色青,五脏中主肝,肝旺于春季七十二日,此时也是肝脉当令的时候,所以我认为肝脏最贵。黄帝到:我依据《上、下经》阴阳比例分析的理论来体会,你认为最贵的,却是其中最贱下的。
雷公斋戒了七天,早晨又侍坐于黄帝的一旁。黄帝到:三阳为经,二阳为维,一阳为游部,懂得这些,可以知道五脏之气运行的终始了。三阴为表,二阴为里,一阴为阴气之最终,是阳气的开始,有如朔侮的交界,都符合于天地阴阳终始的道理。雷公说:我还没有明白其中的意义。黄帝到:所谓“三阳”,是指太阳,其脉至于手太阴寸口,见弦浮不沉之象,应当根据常度来判断,用心体察,并参合阴阳之论,以明好坏。所谓“二阳”,就是阳明,其脉至于手太阴寸口,见弦浮不沉之急,不鼓击于指,火热大至之时而由此病脉,大都有死亡的危险。“一阳”就是少阳,其脉至于手太阴寸口,上连人迎,见弦急悬而不绝,这是少阳经的病脉,如见有阴而无阳的真脏脉象,就要死亡。“三阴”为手太阴肺经,肺朝百脉,所以为六经之主,其气交于太阴寸口,脉象沉浮鼓动而不浮,是太阴之气陷下而不能升天,以致心志空虚。“二阴”是少阴,其脉至于肺,其气归于膀胱,外与脾胃相连。“一阴”是厥阴,其脉独至于太阴寸口,经气已绝,故脉气浮而不鼓,脉象如钩而滑。以上六种脉象,或阳脏见阴脉,或阴脏见阳脉,相互交错,会聚于寸口,都和五脏相通,与阴阳之道相合。如出现此种脉象,凡先见于寸口的为主,后见于寸口的为客。
雷公说;我已经完全懂得您的意思了,把您以前传授给我的经脉道理,以及我自己从书本上读到的从容之道,和今天您所讲的从容之法相结合的话,我还不明白其中阴阳雌雄的意义。黄帝道:三阳如父亲那样高尊,二阳如外卫,一阳如枢纽;三阴如母亲那样善于养育,二阴如雌雄那样内守,一阴如使者一般,能交通阴阳。
二阳一阴是阳明主病,二阳不胜一阴,则阳明脉软而动,九窍之气沉滞不利。三阳一阴为病,则太阳脉胜,寒水之气大盛,一阴肝气不能制止寒水,故内乱五脏,外现惊骇。二阴二阳则病在肮,少阴脉沉,少阴之气胜肺伤脾,在外伤及四肢。二阴与二阳交互为患,则土邪侮水,其病在肾,骂詈妄行,癫疾狂乱。二阴一阳,其病出于肾,阴气上逆于心,并使脘下空窍如被堤坝阻隔一样闭塞不通,四肢好象离开身体一样不能为用。一阴一阳为病,其脉代绝,这是厥阴之气上至于心发生的病变,或在上部,或在下部,而无定处,饮食无味,大便泄泻无度,咽喉干部,病在脾土。二样三阴为病,包括至阴脾土在内,阴气不能至于阳,阳气不能达于阴,阴阳相互隔绝,阳浮于外则内成血瘕,阴沉于里外成脓肿;若阴阳之气都盛壮,而病变趋向于下,再男子则阳道生病,女子则阴器生病。上观天道,下察地理,必以阴阳之理来决断病者死生之期,同时还要参合一岁之中何气为首。
雷公说:请问疾病的死亡日期。黄帝没有回答。雷公又问。黄帝道:在医书上有说明。雷公又说:请问疾病的死亡日期。黄帝道:冬季三月的病,如病症脉象都属阳盛,则春季正月见脉有死征,那麽到初春交夏,阳盛阴衰之时,便会有死亡的危险。冬季三月的病,根据地理,势必将尽,草和柳叶都苦死了,如果到春天阴阳之气都绝,那麽其死期就在正月。春季三月的病,名为“阳杀”。阴阳之气都绝,死期在冬天草木苦干之时。夏季三月的病,若不痊愈,到了至阴之时,那么死期在至阴后不超过十日;若脉见阴阳交错,则死期在初冬结薄冰之时。冬季三月的病,表现了手足三阳的脉证,不给治疗也会自愈。若是阴阳叫错和而为病,则立而不能坐,坐而不能起。若三阳脉独至,则独阳无阴,死期在冰结如石之时。三阴脉独至,则独阴无阳,死期在正月雨水节。
\chapter{方盛衰论篇第八十}
雷公请问:气之多少,何者为逆,何着为从?黄帝答曰:阳从左,阴从右,①阴:原为“阳’O类经》张注作“阴”O②四:拥乙经》及王在“耍’上并有“脉”字。③于:原为“出”O据押己经》改。老从上,少从下,是以春夏归阳为生,归秋冬为死,反之,则归秋冬为生,是以气多少逆皆为厥。问曰:有余者厥耶?答曰:一上不下,寒厥到膝,少者秋冬死,老者秋冬生。气上不下,头痛巅疾,求阳不得,求阴不审,五部隔无征,若居旷野,若伏空室,绵绵乎属不满目。是以少气之厥,令人妄梦,其极至迷。三阳绝,三明微,是为少气。是以肺气虚则使人梦见白物,见人斯血藉藉,得其时则梦见兵战。肾气虚则使人梦见舟船溺人,得其时则梦伏水中,若有畏恐。肝气虚则梦见菌香生草,得其时则梦伏树下不敢起。心气虚则梦救火阳物,得其时则梦活灼。脾气虚则梦饮食不足,得其时则梦筑垣盖屋。此皆五藏气虚,阳气有余,阴气不足,合之五诊,调之阴阳,以在经脱诊有十度度人:脉度,藏度、肉度、筋度、俞度。阴阳气尽,人病自具。脉动无常,散明颇阳,脉脱不具,诊无常行,诊必上下,度民君卿,受师不卒,使术不明,不察逆从,是为妄行,持雌失雄,弃阴附阳,不知并合,诊故不明,传之后世,反论自章。至阴虚,天气绝,至阳盛,地气不足。阴阳并交,至人之所行。阴阳并交者,阳气先至,阴气后至。是以圣人持诊之道,先后阴阳而持之,符恒之如乃六十首,诊合微之事,连阴阳之变,章五中之情,其中之论,取虚实之要,定五度之事,知此乃足以诊。是以切阴不得阳,诊消亡,得阳不得阴,守学不湛,知左不知右,知右不知左,知上不知下,知先不知后,故治不久。知丑知善,知病知不病,知高知下,知坐知起,知行知止,用之有纪,诊道乃具,万世不殆。起所有余,知所不足,度事上下,脉事因格。是以形弱气虚,死;形气有余,脉气不足,死;脉气有余,形气不足,生。是以读有大方,坐起有常,出入有行,以转神明,必清必净,上观下观,司八正邪,别五中部,按脉动静,循尺滑涩,寒温之意,视其大小,合之病能,逆从以得,复知病名,诊可十全,不失人情,故诊之,或视总规意,故不失条理,道甚明察,故能长久。不知此道,失经绝理,亡①言妄期,此谓失道。
方盛衰论篇第八十参考译文
雷公请问道:气的盛衰,哪一种是逆?哪一种是顺?黄帝回答道:阳气主升,其气从左而右;阴气主降,其气从右而左老年之气先衰于下;少年之气先盛于下,其气从下而上。因此春夏之病见阳证阳脉,一阳归阳,则为顺为生,若见阴证阴脉,如秋冬之令,则为逆为死。反过来说,秋冬之病见阴证阳脉,以阴归阴,则为顺为生。所以不论气盛或气衰,逆则都成为厥。雷公又问:气有余也能成为厥吗?黄帝答道:阳气一上而不下,阴阳两气不相顺接,则足部厥冷至膝,少年在秋冬见病则死,而老年在秋冬见病却可生。阳气上而不下,,则上实下虚,为头痛癫顶疾患,这种厥病,谓其属阳,本非阳盛,谓其属阴,则又非阴盛,五脏之气隔绝,没有显著征象可,好象置身于旷野,负居于空窒,无所见闻,而病势绵绵一息,视其生命,一不满一天了。
所以,气虚的厥,使人梦多荒诞;厥逆盛极,则梦多离奇迷乱。三阳之脉悬绝,三阴之脉细微,就是所谓少气之侯。肺气虚则梦见悲惨的事物,或梦见人被杀流血,尸体狼籍,当金旺之时,则梦见战争。肺气虚则梦见悲惨的事物,或梦见人被杀流血,尸体狼籍,当金旺之时,则梦见战争。肾气虚则梦见舟船淹死人,当水旺之时,则梦见大火燔灼。脾气虚则梦饮食不足,得其土旺之时,则梦见作恒盖屋。这些都是五脏气虚,阳气有余,阴气不足所致。当参合五脏见证,调其阴阳,其内容已在《经脉》篇中论述过了。
诊法有十度,就是衡量人的脉度、脏度、肉度、筋度、俞度、揆度它的阴阳虚实,对病情就可以得到全面了解。脉息之动本无常体,或则出现阴阳散乱而有偏颇,或则脉象搏动不明显,所以诊察时也就没有固定的常规。诊病时必须知道病人身份上下,是平民还是君卿。如果对老师的传授不能全部接受,医术不高明,不仅不能辨别逆从,而且会使诊治带有盲目性和片面性,看到了一面,看不到另一面,抓住了一点,放弃了另一点,不知道结合全面情况,加以综合分析,所以诊断就不能明确,如以这种诊断方法授给后人的话,在实际工作中自会明显地暴露出它的错误。
至阴虚,则天之阳气离绝;至阳盛,则地之阴气不足。能使阴阳互济交通,这是有修养的医生的能事。阴阳之气互济交通,是阳气先至,阴气后至。所以,高明的医生诊病,是掌握阴阳先后的规律,根据奇恒之势六十首辩明正常和异常,把各种诊察所得的点滴细微的临床资料综合起来,追寻阴阳的变化了,了解五脏的病情,作出中肯的结论,并根据虚实纲要及十度来加以判断,知道了这些,方可以诊病。所以切其阴而不能了解其阳,这种诊法是不能行于世上的;切其阳而不能了解其阴,其所学的技术也是不高明的。知左而不知其右,知右而不知其左,知上而不知其下,知先而不知其后,他的医道就不会长久。要知道不好的,也要知道好的;要知道有病的,也要知道无病的;既知道高,亦知道下,既知道坐,也要知道起;既知道行,也要知道止。能做到这样有条不紊,反复推求,诊断的步骤,才算全备,也才能永远不出差错。
疾病的初期,见到邪气有余,就应考虑其正气不足,因虚而受邪;检查病者的上下各部,脉证参合,以穷究其病理。例如形弱气虚的;主死;形气有余的,脉气不足的,亦死;脉气有余,形气不足的,主生。所以,诊病有一定的大法,医生应该注意起坐有常,一举一动,保持很好的品德;思维敏捷,头脑清静,上下观察,分别四时八节之邪,辨别邪气中于五脏的何部;触按其脉息的动静,探切尺部皮肤滑涩寒温的概况;视其大小便的变化,与病状相参合,从而知道是逆是顺,同时也知道了病名,这样诊察疾病,可以十不失一,也不会违背人情。所以诊病之时,或视其呼吸,或看其神情,都能不失于条理,技术高明,能保持永久不出差错;假如不知道这些,违反了原则真理,乱谈病情,妄下结论,这是不符合治病救人的医道的。
\chapter{解精微论篇第八十一}
黄帝在明堂,雷公请曰:臣授业,传之行教以经论,从容形法,阴阳刺灸,汤药所滋,行治有贤不肖,未必能十全。若先言悲哀喜怒,噪湿寒暑,阴阳妇女,请问其所以然者,卑贱富贵,人之形体所从,群下通使,临事以适道术,谨①亡:疑作“妄”。
且闻命矣。请问有是愚仆漏之间,不在经者,欲闻其状。帝曰:大矣。公请问:哭泣而泪不出者,若出而少涕,其故何也?帝曰:在经有也。复问:不知水所从生,涕所从出也。帝曰:若问此者,无益于治也,工之所知,道之所生也。夫心者,五藏之专精也,目者其窍也,华色者其荣也,是以人有德也,则气和于目,有亡,忧知于色。是以悲哀则泣下,泣下水所由生。水宗者积水也,积水者至阴也,至阴者肾之精也,宗精之水所以不出者,是精持之也,辅之裹之,放水不行也。夫水之精为志,火之精为神,水火相感,神志俱悲,是以目之水生也。故谚言回:心悲名曰志悲,志与心精共凑于目也。是以俱悲则神气传于心精,上不传于志而志独悲,故泣出也。泣涕者脑也,脑者阻也,髓者骨之充也,故脑渗为涕。志者骨之主也,是以水流而涕从之者,其行类也。夫涕之与泣者,譬如人之兄弟,急则俱死,生则仅生,其志以早悲,是以涕泣仅出而横行也。夫人涕泣俱出而相认者,所属之类也。雷公曰:大矣。请问人哭泣而泪不出者,若出而少,涕不从之何也?帝曰:夫泣不出者,哭不悲也。不泣者,神不慈也。神不想则志不悲,阴阳相持,泣安能独来?夫志悲者惋,惋则冲阴,冲阴则志去目,志去则神不守精,精神去目,涕泣出也。且于独不诵不念夫经言乎,厥则目无所见。夫人厥则阳气并于上,阴气并于下。阳并于上,则火独光也;阴并于下,则足寒,足寒则胀也。夫一水不胜五火,放目毗盲。是以冲风,泣下而不止。夫风之中目也,阳气内守于精,是火气潘自,故见风则泣下也。有以比之,夫火疾风生乃能雨,此之类也。
解精微论篇第八十一参考译文
黄帝在明堂里,雷公请问说:我接受了您传给我的医道,再教给我的学生,教的内容是经典所论,从容形法,阴阳刺灸,汤药所滋。然而他们在临症上,因有贤愚之别,所以未必能十全。至于教的方法,是先告诉他们悲哀喜怒,燥湿寒暑,阴阳妇女等方面的问题,再叫他们回答所以然的道理,并向他们讲述贱富贵及人之形体的适从等,使他们通晓这些理论,再通过临症适当地运用,这些起在过去我已经听您讲过了。现在我还有一些很愚陋的问题,在经典中找不到,要请您解释。黄帝道:你钻研的问题真实深而大啊!
雷公请问:有哭泣而泪涕皆出,或泪出而很少有鼻涕的,这是什麽道理?黄帝说:在医经中有记载。雷公又问:眼泪是怎样产生的?鼻涕是从哪里来的?黄帝道:你问这些问题,对治疗上没有多大帮助,但也是医生应该知道的,因为他是医学中的基本知识。心为五脏之专精,两目是它的外窍,光华色泽是它的外荣。所以一个人在心里有得意的事,则神气和悦于两目;假如心有所失意,则表现忧愁之色。因此悲哀就会哭泣,泣下的泪水所产生的。水的来源,是体内积聚的水液;积聚的水液,是至阴;所谓至阴,就是肾藏之精。来源于肾精的水液,平时所以不出,是受着精的约制,是神,水火相互交感,神志俱悲,因而泪水就出来了。所以俗语说:心悲叫做志悲,因为肾志与心精,同时上凑于目,所以心肾俱悲,则神气传于心精,而不传于肾志,肾志独悲,水失去了精的约制,故而泪水就出来了。哭泣而涕出的,其故在脑,脑属阴,贿充于骨并且藏于脑,而鼻窍通于脑,所以脑髓渗漏而成涕。肾志是骨之主,所以泪水出而鼻涕也随之而出,是因为鼻涕泪是同类的关系。涕之与泪,譬如兄弟,危急则同死,安乐则共存,肾志先悲而脑髓随之,所以涕随泣出而涕泪横流。涕泪所以俱出而相随,是由于涕泪同属水类的缘故。雷公说:你讲的道理真博大!
请问有人哭泣而眼泪不出的,或虽出而量少,且涕不随出的,这是什麽道理?黄帝道:哭而没有眼泪,是内心上并不悲伤。不出眼泪,是心神没有被感动;神不感动,则志亦不悲,心神与肾志相持而不能相互交感,眼泪怎麽能出来呢?大凡志悲就会有凄惨之意。凄惨之意冲动于脑,则肾志去目凄;肾志去目,则神不守精;精和神都离开了眼睛,眼泪和鼻涕才能出来。你难道没有读过或没有想到医经上所说的话吗?厥则眼睛一无所见。当一个人在厥的时候,阳气并走于上部,阴气并走于下部,阳并于上,则上部亢热,阴并与下则足冷,足冷则发胀。因为一水不胜五火,所以眼目就看不见了。所以迎风就会流泪不止的,因风邪中于目而流泪,是由于阳气内守于精,也就是火气燔目的关系,所以遇到风吹就会流泪了。举一个比喻来说:火热之气炽甚而风生,风生而有雨,与这个情况是相类同的。

\end{document}