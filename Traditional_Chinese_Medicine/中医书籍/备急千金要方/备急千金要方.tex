%-*- coding: UTF-8 -*-
% 备急千金要方

\documentclass{ctexbook}

\title{备急千金要方}
\author{孙思邈}
\date{}

\begin{document}
	
	\maketitle
	\tableofcontents
	
	\chapter{前言}
	《备急千金要方》是被誉为中国最早的临床百科全书,世简称为《千金方》。唐孙思邈(581—682年)撰于公元652年,共30卷。《道藏》收入时析为93卷。孙氏以为“人命至重,有贵千金,一方济之,德逾于此”,故以“千金”命名。该书撰成后在国内外有着极广泛之影响,现存日本之《真本千金方》可能系未经宋校正医书局校正之传抄本,经宋校正医书局校刊之《备急千金要方》,中、日翻刻影印者达30余次,又有刻石本、节选本、改编本、《道藏》本等刻印者亦数十种。
	《千金要方》首篇所列的《大医精诚》、《大医习业》,是中医伦理学的基础;其妇、儿科专卷的论述,奠定了宋代妇、儿科独立的基础;其治内科病提倡以脏腑寒热虚实为纲,与现代医学按系统分类有相似之处,其中将飞尸鬼疰(类似肺结核病)归入肺脏证治,提出霍乱因饮食而起,以及对附骨疽(骨关节结核)好发部位的描述、消渴(糖尿病)与痈疽关系的记载,均显示了很高的认识水平;针灸孔穴主治的论述,为针灸治疗提供了准绳,阿是穴的选用、“同身寸”的提倡,对针灸取穴的准确性颇有帮助。因此,《千金要方》素为后世医学家所重视。
	
	\chapter{前言}
	\chapter{前言}
	\chapter{前言}
	\chapter{前言}
	论曰:人年四十已下多有放恣,四十已上即顿觉气力一时衰退。衰退既至,众病蜂起。久而不治,遂至不救。所以彭祖曰:以人疗人,真得其真。故年至四十,须识房中之术。
	
	夫房中术者,其道甚近,而人莫能行。其法,一夜御十女,闭固而已,此房中之术毕矣。兼之药饵,四时勿绝,则气力百倍,而智慧日新。然此方之作也,非欲务于淫佚,苟求快意,务存节欲,以广养生也。
\end{document}