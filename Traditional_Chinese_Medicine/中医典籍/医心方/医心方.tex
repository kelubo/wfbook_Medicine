%-*- coding: UTF-8 -*-
% 医心方.tex

\documentclass[a4paper,12pt,UTF8,twoside]{ctexbook}

% 设置纸张信息。
\RequirePackage[a4paper]{geometry}
\geometry{
	%textwidth=138mm,
	%textheight=215mm,
	%left=27mm,
	%right=27mm,
	%top=25.4mm, 
	%bottom=25.4mm,
	%headheight=2.17cm,
	%headsep=4mm,
	%footskip=12mm,
	%heightrounded,
	inner=1in,
	outer=1.25in
}

% 目录 chapter 级别加点(.)。
\usepackage{titletoc}
\titlecontents{chapter}[0pt]{\vspace{3mm}\bf\addvspace{2pt}\filright}{\contentspush{\thecontentslabel\hspace{0.8em}}}{}{\titlerule*[8pt]{.}\contentspage}

% 设置 part 和 chapter 标题格式。
\ctexset{
	part/name= {第,卷},
	part/number={\chinese{part}},
	chapter/name={第,篇},
	chapter/number={\chinese{chapter}}
}

% 设置古文原文格式。
\newenvironment{yuanwen}{\noindent\bfseries\zihao{4}}

\title{\heiti\zihao{0} 医心方}
\author{日本·丹波康赖}
\date{北宋(公元984年)}

\begin{document}

\maketitle
\tableofcontents

\frontmatter

\chapter{刻医心方序}
\begin{yuanwen}
	《医心方》三十卷,每卷首题:从五位下行针博士兼丹波介丹波宿檷康赖撰。谨按:臣等远祖记合也,意盖数百命臣等曾祖臣元德使以仁和王府所藏抄本誊写,储之医学,当时称为希观。顾其为书,残脱知今命臣旧金石遗则帙根证博之协之于古今之异与风土之宜,参伍而错综之,然后其道乃始完,可以模楷后学矣。求之前人之着,能具斯道者,其唯王氏之书足以当之。而是书则直驾而上之,岂不更伟乎。况其所征引宋字则是书在天壤间凡以裨补后学有匪细故者,不仅为医家鸿宝也。
	
	臣等窍幸今日衣冠文物之化施及吾然上元拜
\end{yuanwen}

\mainmatter

\part{卷第一}
\chapter{治病大体第一}
《千金方》云∶张湛曰∶夫经方之难精,由来尚矣。今病有内同而外异,亦有内异而外同沉于兹而彻艺能又云∶大医治病,必当安神定志,无欲无求,先发大慈恻隐之心,誓愿普救含灵之疾。

若有等,怆,纵有所又云∶夫为医之法,不得多语调笑;谈谑喧哗,道说是非;议论人物,炫耀声名,訾毁诸医老子神报之药又云∶自古名医治病,多用生命以济交急,虽曰贱畜贵人,至于爱命,人畜一也。损彼益己先死者能不又云∶仲景曰∶不须汗而强汗之者,出其津液,枯竭而死;须汗而不与汗之者,使诸毛孔闭塞,令人闷绝而死。又须(余吕反,又羊汝反,善也,待也,参与也)下而不与下之者,使人心内懊恼,胀满烦乱,浮肿而死;不须下而强与下之者,令人开肠洞泄不禁而死。又不须灸而强与灸之者。令人火邪入肠,干错五脏,重加其烦而死。须灸而不与灸之者,使冷结重凝《本草经》云∶凡欲治病,先察其源,先候病机;五脏未虚,六腑未竭,血脉未乱,精神又云∶复应观人之虚实、补泻、男女、老少、苦乐、荣悴、乡壤、风俗,并各不同。褚澄治《太素经》云∶黄帝问于岐伯曰∶医之治病也,一病而治各不同,皆愈,何也?岐伯曰∶地人来西方者,金玉之域,沙石之处也,天地之所收引也;其民陵居而多风,水土刚强,其民不衣其治宜北方者,天地所闭藏之域也,其地高陵居,风寒冰冻;其民乐野处而乳食,脏寒生病,其治南方者,天地所养长阳气之所盛之处也。其地洼下,水土弱,雾露之所聚也;其民嗜酸而食中央者,其地平以湿,天地所生物色者众。其民食杂而不劳,故其病多痿厥寒热,其治宜导之情又云∶凡诊病者,必问尝贵后贱,虽不中于外邪,病从内生,名曰脱荣;尝富后贫,名曰失之辱最胜王经云∶病有四种别,谓风热痰,及以总集病,应知发动时。春中痰动,夏内风病生,秋时黄热服药病。

应变顺时病。

问其多汗及多。聪明梦见火,斯人是热性。心定身平整,虑审头津腻。梦见水白物,是性应知。

无死舌黑生。

慈悯《南海传》云∶夫四大违和,生灵共有,八节交竟,发动无恒。凡是病生,即须将息。

故《令身病源二《太素经》云∶病先起于阴者,先治其阴,而后治其阳;先起于阳者,先治其阳,而后治其阴,皆疗其本也。又云∶形乐志苦,病生于脉,治之以灸刺;形苦忘乐,病生于筋,治之以熨引;形乐志乐,病生于肉,治之以针石;形苦志苦,病生咽喝,治之以药;形数惊恐,筋于不仁,治之按摩醪药。

又云∶病有生于风寒、暑湿、饮食、男女,非心病者,可以针石汤药去之。喜怒忧思,伤神不可又云∶伯高曰∶兵法曰∶无迎逢逢之气,无击堂堂之阵。刺法曰∶无刺之热,无刺漉漉兵大《针灸经》云∶十岁小儿,七十老人,不得针,宜灸及甘药。

《医门方》云∶大法春夏宜发汗。凡服汤药发汗,中病便止,不必尽剂。凡发汗,欲令手足身体,凡大汗出复后,脉洪大,形如疟,一日再发,汗出便解,更与桂枝麻黄汤∶麻黄(四两,去节)桂心(二两)甘草(一两,炙)杏仁(八十枚,去尖)以水七升,先煮麻黄三两,沸,撩去沫,纳诸药,煮取二升半,分三服,服相去七八里,覆不须啜粥,余如桂枝法。

凡发汗后,汗遂漏不止,其人恶风;小便难,四肢微急,难以屈伸,桂枝加附子汤主之∶桂心(三两)夕药(三两)甘草(二两,炙)生姜(三两)大枣(十二枚,擘,切)以水七升,微火煮取三小升,去滓,分温三服。服汤已须臾,啜一升热粥,以助药力。

温覆复服又云∶冬可热药,不可发汗;汗退场门中疮生,吐痢。

凡衄(女鞠反,鼻出血也)家不可发其汗。汗出即直视不得眠睡。

凡淋家不可发汗,汗出即便血。

凡咽中闭塞、咽燥者不可发其汗。

凡大下后发汗即胀满。凡发汗后恶寒者,虚也;不恶寒但热者,实也;当和其胃气。

凡疮家虽身疼痛,不可攻其表。汗出必致痉(久至反,恶病反)。

凡新大吐下、衄血、鼻失血、得欧楗之后。妇人新伤产,皆不可汗。

凡咳而小便利,若失小便,不可发汗,汗出即呕逆厥冷。

凡发汗后饮水多,其人必喘,以水灌之亦喘。

凡发汗以后,其人脐下悸(其季反,心动也)。欲作奔豚(徒昆反,豕子也)气,茯苓桂心甘草凡发汗以后,腹胀满,浓朴生姜半夏甘草人参汤主之。

又云∶大法春宜吐。

凡诸病在胸中者,宜吐之。

凡服汤吐,中病便止,不必尽剂。须吐者虚及伤寒胸中满,及积痰干呕。又胸膈痰热转嗽(凡宿食在胃管,当吐下之。

又云∶诸四逆者,不可吐之。凡诸虚羸者,不可吐之。凡新产者,不可吐之。

凡香港脚上冲心者,不宜吐之。

凡病者恶寒而不欲近衣,不可吐之。

又云∶大法秋宜下。

凡服汤下,中病便止,不可尽剂。

凡病发作汗多,急下之。

凡病五日六日,腹满不大便,急下之。

凡大下后六七日,不大便,烦不解,腹痛而满,此有燥屎。所以然者,本有宿食,宜承气汤凡病者小便不利,大便乍难乍易,时有微热,沸冒不能卧,此有燥屎故也,宜下之。

凡可下者,汤胜丸。胁下偏痛发热,此寒也。当以温药下。其寒胀须下。

凡诸病大便涩,诸伤寒腹满,疟,腹满鼓胀,水胀,大便不通,须利小便者;黄病、水病、凡病腹中满痛者,为寒,当下之。腹满不减,减不足言,常下之。脉数而滑者,有宿食,下凡可下者,以承气汤方∶大黄(四两,别渍一宿)浓朴(二两,炙)切,以水五升,煮取一升半,下大黄,更一二沸,去滓,分二服。当利二三行愈。

又云∶夏月不可下。凡病喘而胸满,不可下之。凡病心下坚,颈项强而眩,勿下之。凡厥逆凡病欲吐者,不可下。凡病有外证,外证未解,不可下之。凡病腹满吐食,下之益甚。

凡大者不




合药料理法第六

《千金方》云∶凡捣药法,烧香,洒扫,洁净,勿得杂语,当使童子捣之。务令细熟,杵数菩相日造。所求皆得。攘灾减恶,病者得瘥,死者更生。

凡合肾气,署预及诸大五虫,大麝香丸、金牙散、大酒煎膏等。合时勿令妇女、小儿、产妇在禁等诸难捣之药,费人功力作者,悉盗弃之。又为尘埃秽气,皆入药中罗筛,粗恶随风飘扬,众口反加凡百药皆不欲数数晒曝,多见风日,气力即薄歇无用,宜熟知之。诸药未即用者,候天晴时中风。虽经年亦如新也。其丸散以瓷器贮,蜡封之勿泄,则三十年不坏。诸杏仁及子等药瓦器贮则凡贮药法,皆去地四尺,则土湿之气不中之。

凡药冶择熬炮讫,然后称之,以依方。不得生称。

凡药治,凡诸果子仁皆去尖头赤皮,双仁仍切之。

凡茯苓、夕药须白者,泻药唯赤者。

凡石斛皆以硬推打,令碎,乃入臼;不尔,捣,不可熟。牛膝、石斛等入汤,擘碎用之。

凡渍凡枳壳、浓朴、甘草、桂、黄柏诸木皮及诸毛、羽、贝、齿、牙、蹄、甲、龟、鳖、鲭、鲤凡诸汤用酒者,皆临熟下。凡汤中用麝香、犀角、鹿茸、羊角、牛黄、蒲黄、丹沙者,汤熟凡生麦门冬、生姜,入汤皆切,三捣三绞,取汁,汤成去滓,纳煮五六沸,依如升数,不可凡银屑,皆以水银和成泥。

凡用乳等诸石,以玉锤水研三日三夜,漂练,务令极细。凡用麦、曲米、大豆、黄卷、泽兰、芜荑、僵(纪良反)蚕、干漆、蜂房,皆微炒。

《本草》云∶凡汤酒膏药,旧方皆云咀者,谓称毕捣之如大豆。又使吹去细末,此于事殊义《葛氏方》∶咀者,皆应细切。)今皆细切之较略,令如咀者瘥,得无末而粒片调于药丸辈之日干三分一分耗。)其汤酒中不须如此。凡筛丸药用重密绢,令细于密丸,易熟;若筛散,草药用轻疏绢。于酒服则不泥,其石药亦用细绢,筛如丸者。凡筛丸散药竟,皆更合于臼中,以杵研冶之数百过,视色理和同为佳。(今按∶《范汪方》云∶乃□着蜜熟捣之。)凡汤酒膏中用诸石药,皆细捣如粟米。亦以葛布筛令调,并新绵别裹纳中。其雄黄、朱砂细末如粉。凡煮汤欲熟,微火令小沸。其水数依方多少大略升两,药用水一斗,煮取四升,以此为率。然则利汤欲生,少水而多取。补汤欲熟,多水而少取。好详视所得,宁令水少,多用新布。两人以尺木绞之,澄去浊,纸覆令密,温汤勿令器中有水气,于热汤上煮令暖,亦好服汤,宁令小热易下,冷则呕涌,云分再服三服者,要令力势足相及,并视人之强羸,病之轻重,以为进退增减之,不必悉依方说也。(今按∶《新注》云∶但看病之轻重,形之羸疲,除加乃至四五服亦好,故曰不依方。)凡渍药酒,随寒暑,日数视其浓烈,便可漉出,不必待至酒尽也。(今按∶《新注》云∶凡渍药酒,冬日七宿,春秋五宿,夏日三宿。辛贞日、冬日者必须七宿一时,药酒皆盛瓷瓶以纸盖口,勿令气泄。二月、三月、八月、九月五四宿,四月、五月三宿,六月、七月若二十两药酒一斗五升。渍者先五升酒,渍药二日宿。初合,日再服,但视酒尽,更增一升酒日别,日别添着一升。如此法者,不必酒回酢,春秋日如常法用。)凡建中肾沥诸补汤,滓合两剂加水煮,竭饮之。亦敌一剂新药,贫人当依用,皆应先曝令燥。凡合膏,初以苦酒渍取,令淹溲浸,不用多汁,密覆勿泄。(今按∶注》云∶淹溲者,酢药得相和,不用多汁。)云时者,周时从今旦至明旦,亦有止一(今按∶《新注》云∶有言半日者,从旦至暮也。又一宿意同半日。)三上三下以泻其令药味得出。上之使币(祖合反),币沸仍下,下之取沸,静良久乃上,宁欲小生。其中有薤白者,以两头微焦黄为候,有白芷附子者,亦令小黄色也。(今按∶《新注》云∶膏得焦微生,若过焦者气力小也。)猪肪皆勿令经水,腊月弥佳。绞膏亦以新布。若是之膏。膏滓亦堪酒煮。稍饮之,可摩之膏,膏滓则宜以薄病上。

凡膏中有雄黄朱砂辈,皆别捣细碎如面,须绞膏竟乃投中,以物疾搅至干凝强,勿使沉聚在凡汤酒中用大黄,不须细锉(粗卧反),作汤者先水渍令淹溲(疏柳反),覆一宿。明旦煮汤,今凡汤中用麻黄,皆先别煮两三沸,断去其沫。更益水如本数,乃纳余药,不尔,令人烦。

麻则凡汤中用完物皆擘破,干枣、栀子、栝萎子之类是也。用细核物亦打碎,山茱萸、五味子、子之类薄切。芒凡用麦门冬皆微润汤,抽去心。(今按∶《新注》云∶于微熬,抽去心,今时不润抽心也毛取有犀凡汤丸散用天雄、附子、乌头、鸟喙、侧子,皆灰中炮之,令微坼,削去黑皮乃称之。

唯凡不尔洗便毕云∶方》炮之如建法,削去焦皮。)凡丸散用胶皆先炙,便使通体沸起燥,乃可捣,有不浃处更炙之。(今按《千金方》云∶断下汤直尔用之,勿炙也。又云∶既细碎,不炙,于子熬亦得筛凡用蜜,皆先火上煎断去沫,令色微黄,则丸经久不坏。克之多少,随蜜精粗。(今按∶《用凡丸散用巴豆、杏仁、桃仁、葶苈、胡麻,诸有膏脂药,皆先熬令黄黑,别捣。令如膏脂,尽。(数百凡用桂、浓朴、杜仲、秦皮、木兰辈,皆削去上虚软甲错皮,取里有味者称之。茯苓、猪苓白、根毛凡野狼毒、枳实、橘皮、半夏、麻黄、吴茱萸皆欲得陈久者。其余唯须精新。(《范汪方》云《录验方》云∶蜜腊膏髓类者皆成汤,纳烊令和调也。又,合汤用血及酒者,临熟纳之。

然《葛氏方》云∶凡直云末者,皆是捣筛。
药斤两升合法第七

《本草经》云∶古秤唯有铢两而无分名,今则以十黍为一铢,六铢为一分,四分成一两,十又云∶凡方有云分等者,非分两之分,谓诸药斤两多少皆同耳。

又云∶凡散药有云刀圭、十分、方寸匕∶(必履反,匙也)之一准如梧子大也。方寸匕者,作此为方又云∶钱五匕者,今五铢钱边五字者。(今按∶《葛氏方》云∶五铢钱重五铢也。)又云∶一撮(粗活反)者,四刀圭也。十撮为一夕(之药反),十夕为一合,(今按∶《千金方为一又云∶药升方作上径一寸,下径六分,深八分。

又云∶凡丸药有云如细麻者,即胡麻也。又以十六黍为一大豆也。如大麻者,即大麻子准三以如梧子十丸为度。如弹丸及鸡子黄者,以十梧子准之。(今按∶方寸匕散为丸如梧子,得十六丸,如弹丸一枚。若鸡子黄者准四十丸,今以弹丸同鸡子黄,此甚不等也。)又云∶巴豆如干枚者,粒有大小,当先去心皮竟,称之,以一分准十六枚。又云∶附子乌头重又云∶枳实如干枚者,去核竟,以一分准二枚。橘皮一分准三枚。枣有大小,以三枚准一两。干姜一累者以一两为正。(今按∶《千金方》∶干姜一累以半两为正。《录验方》云∶干姜、生姜一累数者,其一支为累,取肥大者。《范汪方》云∶凡无生姜,可用干姜一两当二两。)又云∶桂一尺者,削去皮竟,重半两为正。(今按∶《范汪方》云∶桂一尺若五寸者,以广六分,浓三分为正。《录验方》∶桂一尺若数寸是,以浓二分、广六分为准。)又云∶甘草一尺者,重二两为正。(今按∶《范汪方》∶甘草一尺若五寸者大小,以径一寸为正。《录验方》∶甘草一尺若数寸者,以径半寸为准,去赤皮炙之,令不吐。《短剧方》云∶以径头一寸为准。)又方∶凡方云半夏一升者,洗竟称五两为正。(今按∶苏敬云∶半夏一升以八两为正。)又云∶凡椒一升三两为正,吴茱萸一升,五两为正;蛇床子一升,三两半;地肤子一升,四两;菟丝子一升,重有九两;(于炎反)(力鱼反)子一升四两。(今按∶苏敬云∶三又云∶凡方云某草一束者,以重三两为正;云一把者,重二两为正。(今按∶《范汪方》∶麻黄若他草一者,以重三两为正。《录验方》∶麻黄一把一握者,并以重三两为准。)又云∶蜜一斤者,有七合;猪膏一斤者,一升二合。《范汪方》云∶胶一廷如三指大,长三一KT,长一尺,径三寸是也。《经心方》云∶胡粉十二棋。(博棋者,大小方寸是也。按∶棋者,牙棋子。)《短剧方》云∶凡黄柏一斤者,以重二两为准。人参一枚者以重二分为准。

又云∶凡浓朴一尺及数寸者,以浓三分、广一寸半为准。又云∶服汤云一杯者,以三合酒杯子
药不入汤酒法第八

《本草经》云∶药有宜丸者,宜散者,宜水煮者,宜酒渍者,宜膏煎者,亦有一物兼宜者,朱砂雌黄云母阳起石矾石硫黄钟乳(入酒)孔公孽(入酒)誉石银屑铜镜上十七种石类。

冶葛野狼毒鬼臼毒公莽草巴豆踯躇(入酒)蒴(入酒)皂荚菌藜芦蛇衔陟厘上四十七种草木类。

蜂子蜜蜡白马茎狗阴雀卵鸡子雄鹊伏翼鼠妇樗鸡萤火KT强蚕魁虾蟆上二十九种虫兽类。
药畏恶相反法第九

《本草经》云∶药有单行者,有相须者,有相畏者,有相恶者,有相使者,有相反者,有相

石上

玉泉(畏款冬花。)玉屑(恶鹿角。)丹砂(恶磁石,畏碱水。)水银〔恶磁石。(今按∶《范汪方》∶杀铜金毒。《药辨决》∶畏玄石。)〕曾青(恶菟丝子。)石胆(水英为之使,畏芫花、辛夷、白薇、牡桂、菌桂。)云母〔泽泻为之使,畏鳝甲,反流水。(今按∶《极要方》∶恶除长卿。)〕硝石〔萤火为之使,恶苦参、苦菜,畏女苑。(《药辨决》∶术为之使。《千金方》∶畏牡桂、芫花。)〕朴硝〔畏麦句姜。(今按∶《千金方》∶恶麦勺姜。)〕芒硝〔石苇为之使,畏麦勺姜。(今按∶《药决》云∶滑石为之使。《千金方》∶恶曾青。)〕矾石〔甘草为之使,恶牡蛎。(今按∶《范汪方》∶铅为之使。)〕滑石〔石苇为之使。恶曾青。(今按∶《药决》∶恶空青。)〕紫石英(长石为之使。畏扁青、附子,不欲鳝甲、黄连、麦勺姜。)白石英(恶马目毒公。)赤石脂〔恶大黄、畏芜花。(今按∶《药辨决》∶畏黄芩,反甘草。)〕黄石脂(曾青为之使,恶细辛,畏蜚廉。)太一禹余粮(杜仲为之使,畏贝母、菖蒲、铁落。)白石脂〔鸡矢为之使,恶松脂,畏黄芩。(今按∶《药辨决》∶恶柏脂布。)〕

石中

钟乳〔蛇床子为之使。恶牡丹、玄石、牡蒙,畏紫石英、草。(《千金方》∶菟丝子为凝水石(畏地榆,解巴豆毒。)石膏(鸡子为之使,恶莽草、毒公。)阳起石〔桑螵蛸(上瓢音,下消音)为之使,恶泽泻、菌桂、雷丸、蛇蜕皮,畏菟丝子。〕玄石(恶松柏脂子,菌桂。)理石(滑石为之使,畏麻黄。)殷孽(恶木防己孽。)孔公孽(木兰为之使,恶细辛。)磁石(茈胡为之使,畏黄石脂,恶牡丹、莽草,杀铁毒。)

石下

青琅〔得水银良,畏乌头,杀锡毒。(今按∶《千金方》∶畏鸡骨。)〕誉石〔得火良。棘针为之使,恶虎掌、毒公、细辛,畏水蛭也。(今按∶《范汪方》∶甘方解石(恶巴豆。)代赭(畏天雄。)大盐(漏芦为之使。)特生誉(石火练之良,畏水。)

草上

六芝(薯蓣为之使,得发良,恶恒山,畏扁青、茵陈蒿。)茯苓茯神(马间为之使,恶白蔹,畏牡蒙、地榆、雄黄、秦胶、龟甲。)柏子〔牡蛎、桂、瓜子为之使。恶菊花、羊蹄、硝石。(今按∶《范汪方》∶恶白菊。)天门冬(垣衣地黄为之使,畏曾青。)麦门冬〔地黄、车前为之使,恶款冬、苦瓠、苦参、青。(《短剧方》∶垣衣为使。)〕术〔防风、地榆为之使。〕女萎(一名阿米尔)(畏卤碱。)干地黄(得麦门冬、清酒良,恶贝母,畏芜荑。)菖蒲(秦皮为之使,恶地胆,麻黄。)远志(得茯苓、冬葵子、龙骨良,杀天雄、附子毒,畏真珠、蜚廉、藜芦、齐蛤。)泽泻(畏海蛤、文蛤。)薯蓣〔紫芝为之使,恶甘遂。(今按∶《极要方》∶紫石为之使,恶远志。)〕菊花(术、枸杞根、桑根白皮为之使)。

甘草(术、干漆、苦参为之使,恶远志,反甘遂、大戟、芫花、海藻。)人参(茯苓为之使,恶搜疏,反藜芦。)石斛〔陆英为之使,恶凝水石、巴豆,畏僵蚕、雷丸。(《药辨决》∶不欲菌桂、蛇蜕皮石龙芮(大戟为之使,畏蛇蜕、茱萸。)落石(杜仲、牡丹为之使,恶铁落、菖蒲、贝母。)龙胆(贯众为之使,恶防葵、地黄。)牛膝(恶萤火、龟甲,畏白前。)杜仲(畏蛇皮、玄参。)干漆(半夏为之使,畏鸡子。)细辛(曾青、枣根为使,恶野狼毒、山茱萸、黄花,畏滑石、硝石,反藜芦。)独活(蠡实为之使。)茈胡(半夏为之使,恶皂荚、女菀、藜芦。)酸枣(恶防己。)槐子(景天为之使。)子(荆子、薏苡为之使。)蛇床子(恶牡丹、巴豆、贝母。)菟丝子〔得酒良。薯蓣、松脂为之使。恶菌。(今按∶《药辨决》∶恶菌桂、雷丸。)〕析子〔得荆实、细辛良,恶干姜、苦参。(今按∶《范汪方》云∶细辛为使。)〕蒺藜(乌头为之使。)天名精(恒衣为之使。)茜根(畏鼠姑。)蔓荆实(恶乌头、石膏。)牡荆实(防风为之使,恶石膏。)秦椒(恶栝萎、防葵,畏雌黄。)辛夷(芎为之使,恶五石脂,畏菖蒲、黄连、石膏、黄环。)

草中

当归(恶茹,畏菖蒲、海藻、牡蒙。)防风(不欲干姜、藜芦、白、芫花,杀附子毒。)秦艽(菖蒲为之使。)黄(恶龟甲。)吴茱萸〔蓼实为之使。恶丹参、硝石、白恶土,畏紫石英。(《药辨决》∶不欲诸石。)〕黄芩(茱萸、龙骨为之使。恶葱实,畏丹砂、牡丹、药芦。)黄连〔黄芩、龙骨为使,恶菊花、芫花、玄参、白藓,畏款冬花,胜乌头,解巴豆毒。

(《药辨决》∶恶茯苓。)〕五味(苁蓉为之使,恶葳蕤,胜乌头。)决明(蓍实为之使,恶大麻子。)夕药(须丸为使,恶石斛、芒硝,畏硝石、鳖甲、山筋,反藜芦,恶葵菜。)桔梗(秦皮为之使,畏白芨、龙胆、龙眼。)芎〔白芷为之使,得细辛,牡蛎良。(《极要方》云∶恶黄连。)〕本(恶茹。)麻黄(浓朴为之使,恶辛夷、石苇。)葛根(杀野葛、巴豆百药毒。)前胡(半夏为之使,恶皂荚,畏藜芦。)贝母(浓朴、白薇为使,恶桃花,畏秦艽、誉石、莽草,反乌头。)栝蒌(枸杞为之使,恶干姜,畏牛膝、干漆,反乌头。)丹参(畏咸水,反藜芦。)浓朴〔干姜为使。恶泽泻、寒水石、硝石。(《药辨决》云∶恶细辛。)〕玄参〔恶黄、干姜、大枣、山茱萸,反藜芦。〕沙参〔恶防己,反藜芦。(《药辨决》云∶不欲防己、术,使人渍。)〕苦参(玄参为之使,恶贝母、漏芦、菟丝子,反藜芦。)续断(地黄为之使,恶雷丸。)山茱萸〔蓼实为之使,恶桔梗、防风、防己。〕桑根白皮(续断、桂心、麻子为之使。)狗脊〔萆为之使,恶败酱。(《药辨决》云∶不欲大黄、前胡、牡蛎。)〕萆(薏苡为之使,畏葵根、大黄、茈胡、牡蛎、前胡。)石苇(杏仁为之使,得菖蒲良。)瞿麦(草、牡丹为之使,恶桑螵蛸。)秦皮〔大戟为使,恶茱萸。(《药辨决》云∶陆英为之使。)〕白芷(当归为之使,恶旋复花。)杜若(得辛夷、细辛良,恶茈胡、前胡。)柏木〔恶干漆。(今按∶《药辨决》云∶不欲干漆,反伤人腹。)〕栀子(解踯躅毒。)紫菀(款冬为之使,恶天雄、瞿麦、雷丸、远志,畏茵陈蒿。)白藓(恶螵蛸、桔梗、茯苓、萆。)白薇(恶黄、干姜、干漆、大枣、山茱萸。)薇衔(得秦皮良。)海藻(反甘草。)干姜(秦椒为之使,恶黄连、黄芩、天鼠矢,杀半夏、莨菪毒。)

草下

大黄(黄芩为使,无所畏,得夕药、黄芩、牡蛎、细辛、茯苓、硝石、紫石、桃仁良。)蜀椒(杏仁为之使,畏橐吾。)巴豆〔芫花为使。恶草,畏大黄、黄连、藜芦。(《药辨决》云∶得大良。)〕甘遂(瓜蒂为之使,恶远志,反甘草。)葶苈〔榆皮为使,得酒良。恶僵蚕、石龙芮。(《范汪方》云∶畏僵蚕。)〕大戟(反甘草。)泽漆(小豆为之使,恶薯蓣。)芫花(决明子为之使,反甘草。)钩吻(半夏为之使,恶黄芩。)野狼毒(大豆为之使,恶麦勺姜。)鬼臼(畏垣衣。)天雄(远志为之使,恶腐婢。)乌头乌喙(莽草为之使,反半夏、栝蒌、贝母、白、白芨,恶藜芦。)附子(蛇胆为使,恶蜈蚣,畏防风、甘草、黄、人参、乌韭、大豆。)皂荚(柏子为之使,恶麦门冬,畏空青、人参、苦参。)恒山(畏玉札。)蜀漆(栝蒌为之使,恶贯众。)半夏(射干为使,恶皂荚,畏雄黄、生姜、秦皮、龟甲,反乌头。)款冬(杏仁为使,得紫菀良。恶皂荚,硝石,畏辛夷、麻黄、黄芩、黄连、黄、青葙。)牡丹〔畏菟丝子。(今按∶《药辨决》云∶不欲大黄、贝母。)〕防己(殷孽为之使。恶细辛,畏萆,杀雄黄毒。)巴戟天(覆盆为之使,恶朝生雷丸、丹参。)石南草〔五茄为之使,(今按∶《范汪方》∶恶山蓟。)〕女菀(畏卤碱。)地榆(得发良,恶麦门冬。)五茄(远志为之使,畏蛇皮、玄参。)泽兰(防己为之使。)黄环(鸢尾为之使,恶茯苓。)紫参(畏辛夷。)菌(得酒良,畏鸡子。)贯众(菌为之使。)野狼牙〔芜荑为之使,恶地榆,枣肌。(今按∶《药辨决》∶一说云恶地胆。)〕藜芦(黄连为之使,反细辛、夕药、五参,恶大黄。)茹(甘草为之使,恶麦门冬。)白蔹〔代赭为之使,反乌头。(今按∶《范汪方》云∶恶乌头。又云∶杀火毒。)〕白芨(紫石为之使,恶理石、李核仁、杏仁。)占斯(解野狼毒毒。)溲流(漏芦为之使。)淫羊藿(薯蓣为之使。)蜚廉〔(一名布)得乌头良。要恶麻黄。(今按∶《药辨决》云∶不欲麻黄、酸枣、防己。)栾华(决明为之使。)虎掌(蜀漆为之使,恶莽草。)蕈草〔矾石为之使。(今按∶《药辨决》云∶得发良。)〕荩草(畏鼠姑。)夏枯草(土为之使。)弋共〔畏玉丸、蜚廉。(今按∶《药辨决》云∶不欲蜚廉。)〕雷丸(荔实,浓朴为之使,恶葛根。)

虫上

龙骨(得人参、牛黄良,畏石膏。)龙齿角(畏干漆、蜀椒、理石。)牛黄(人参为之使,恶龙骨、地黄、龙胆、飞廉,畏牛膝。)蜂子(畏黄芩、夕药、牡蛎。)蜡蜜(恶芫花、文蛤。)白胶(得火良,畏大黄。)阿胶〔得大良,恶大黄。(今按∶《极要方》∶恶白胶、大黄。)〕牡蛎(贝母为使,得其草、牛膝、远志良,恶麻黄、茱萸、辛夷。)

虫中

犀角(松脂为之使,恶菌、雷丸。)羚羊角羊角(菟丝子为之使。)鹿茸(麻勃为之使。)鹿角(杜仲为之使。)伏翼(苋实、云实为之使。)皮(得酒良,畏桔梗、麦门冬。)蜥蜴(恶硫黄、斑蝥、芜荑。)蜂房(恶干姜、丹参、黄芩、夕药、牡蛎。《药辨决》云∶杀蜂毒。)桑螵蛸(得龙骨疗泄精虫(畏皂荚、菖蒲。)蛴螬(蜚虻为之使,恶附子。)海蛤(蜀漆为之使,畏狗胆、甘遂、芫花。)龟甲(恶沙参、蜚廉。)鳖甲(恶矾石。)鳝甲(蜀漆为之使,畏狗胆、甘遂、芫花。)乌贼鱼骨(恶白蔹、白芨。)蟹(杀莨菪毒。)

虫下

麋脂〔畏大黄。(今按∶《极要方》∶畏大黄、甘草。)〕蛇蜕(畏磁石,反酒。)蜣螂(畏羊角、石膏。)蛇胆(恶甘草。)马刀(得水良。)天鼠矢〔恶白蔹、白薇。(今按∶《药辨决》云∶不欲沙参。)〕斑蝥(马刀为之使,畏巴豆、空青。)

果上

大枣(杀乌头毒。)

果下

杏核〔得火良,恶黄、黄芩、葛根,解锡胡粉,畏草。(《范汪方》云∶猪膏为使。)

菜上

冬葵子(黄芩为之使。)

米上

麻麻子〔畏牡蛎、白薇,恶茯苓。(《药辨决》云∶虫为之使。)〕

米中

大豆及黄卷(恶五参、龙胆,得前胡、乌喙、杏仁、牡蛎良,杀乌头毒。)大麦(食蜜为之使。)
诸药和名第十

本草内药八百五十种

第三卷玉石上二十二种

玉泉(唐)玉屑(唐)丹砂(唐又出,伊芳势饭高郡日向。)空青(唐又出。近江国慈贺郡。)绿青(和名安宁仁。出长门国。)曾青(唐)白青(唐)扁(补典反)青(唐)石胆(出备中国)云母(和名歧良良,出近江陆奥国。)石钟乳(和名伊芳之乃知。出备中英贺郡。)朴(普角反)硝(出信浓若挟备中国。)硝石(出赞歧国。)芒硝(出大宰。)矾石(出飞国肥后国阿苏神社。)滑石(出纪伊芳国。)紫石英(出伯耆国。)白石英(出近江备中大宰。)青石脂(唐)赤石脂(出备后国大宰。)黄石脂(唐)白石脂(出伊芳豆大宰。)黑石脂(唐)太一余粮(唐)石中黄子(唐)禹余粮(出大宰。)

第四卷玉石中三十种

金屑〔(先结反)和名古加檷,出陆奥国。〕银屑(和名之吕加檷,出对马长门飞国。)水银(和名美都加檷,出伊芳势国。)雄黄(和名歧尔,出伊芳势国。)雌黄(出备中国。)殷孽(钟乳根也,出备中英贺郡。)孔公孽(今钟乳床也,出备中国。)石脑〔(乃道反)唐〕石硫黄(和名由乃阿和,出大宰。)阳起石(唐)凝水石(唐。一名寒水石。)石膏(和名之良从之。出大宰备中若狭国。)磁石(唐。吸针石。)玄石(唐)理石(唐)长石(唐)肤青(唐)铁落(和名久吕加檷及波太。)铁(和名阿良加檷。)生铁(是不被釜之类者。)刚铁(和名布介留加檷,是杂练生作刀者。)铁精(和名加奈久曾。又加檷及佐比)光明盐(唐)绿盐(唐)密陀僧(唐)紫〔(古猛反)骐竭(唐)〕桃花石(唐)珊瑚(唐)石花(唐)石床(出钟乳中)

第五卷玉石下三十一种

青(唐)(音预)石(唐)。(又出长门美牟郡)。

特生石(出长门国。唐)握雪石(唐)方解石(唐)苍石(唐)土阴孽(出钟乳中。)代赭(和名阿加都知。出大宰。)卤(音鲁)碱(音咸)〔和名阿和之保。(陶云∶煎盐釜下凝滓也。)〕大盐(和名之保。苏云人常食者。)戎盐(唐)白垩〔(音恶)和名之良都知。〕铅丹(和名多尔。)粉锡〔(先历反)和名已布尔。〕锡铜镜鼻(和名奈末利。)铅弩牙(和名于保由美乃波须。)金牙(出但马上野国。)锻石(和名以之波比。)冬灰(和名阿加佐乃波比。)灶灰(和名加知酒(须)留所乃都知。)伏龙肝(和名加末都知。)东壁土(陶云屋之东壁土耳。)(乃交反)沙(唐)胡桐泪〔(唐。)(敬云∶是桐树滋沦入地作之。)〕姜石(唐)赤铜屑(和名安加之檷。)铜〔(古猛反,正作矿)石(唐)〕白瓷〔(自夷反)瓦屑〕乌古瓦(屋上年久者)石燕(唐)梁上尘

第六卷草上之上四十一种

青芝(唐)赤芝(唐)黄芝(唐)白芝(黄)黑芝(唐)紫芝(唐)赤箭〔(子贼反)和名平止平止之。又加美乃也。出知泉。〕天门冬(和名须末吕久佐。)麦门冬(和名也末须介。)木〔(直伟反)和名平介良。〕女萎(于么反)葳蕤〔(人□反)和名惠美久佐。又阿末尔。〕黄精(和名于保惠美,又阿末奈。一名也万惠美。)干地黄菖蒲(和名阿也女久佐。)远志(唐)泽泻〔(音昔,又扬和也反)和名奈末为。又于毛多加。〕薯蓣(和名也末都以毛。)菊花(和名加波良于波歧。又歧久。)甘草(和名阿末歧。出陆奥国。)人参(所金反)(和名加乃尔介久佐。一名尔已太。)石斛(胡木反)(和名须久奈比古乃久须檷。又以波久须利。)牛膝(息七反)(和名为乃久都知,又以奈歧久佐。)卷柏(和名伊芳波久美,又伊芳波古介。)细辛(和名美良乃檷久佐。又比歧乃比太比久佐。又美也末好奈波。)独活(和名宇止。又都知多良。)升麻(和名止利乃阿之久佐。又宇多加久佐。又止利乃檷久佐。)茈胡(和名乃世利。又波末阿加奈。)房葵(和名也末奈须比。)蓍实(和名女止久佐。)(音淹)芦子(和名比歧与毛歧,一名波波古。)薏(音忆)苡(音以)子(和名都之多末。)车前子(和名于保波已。)菥(先历反)(音觅)子(和名都波比良久佐。)茺(音充)蔚(音尉)子(和名女波之歧。)木香(和名佐宇毛久。)龙胆(和名衣也美久佐。一名尔加奈。)菟丝子(和名檷奈之久佐。)巴戟天(和名也末比比良歧。)白莫(和名保吕久。又都久美乃以比檷。)白蒿(和名之吕与毛歧。一名加波良与毛歧。)

第七卷草上之下三十八种

肉苁蓉(唐)地肤子(和名尔波久佐。又末歧久佐。)忍冬(和名须比都良。)蒺藜子(和名波末比之。)防风(和名波末须加奈。又波末尔加奈。)石龙刍(和名宇之乃比多比。又多都乃比介。)落石(和名都多。)千岁汁(和名阿末都良。一名止止歧。)黄连(和名加久末久佐。)沙参(唐)丹参(唐。又殖养浓国。)王不留行(和名须须久佐。一名加佐久佐。)蓝实(和名阿为乃美。)景天(和名伊芳歧久佐。)天名精(和名波末多加奈。一名波末布久良佐。)蒲黄(和名加末乃波奈。)香蒲(和名女加末。)兰草(和名布知波加末。)决明(和名衣比须久佐。)芎(和名于无奈加都良久佐。)蘼芜(芎苗也。)续断(和名于尔乃也加良。又波美久佐。)云实(和名波末佐佐介。)黄(和名也波良久佐。又加波良佐佐介。)徐长卿(和名比女加加美。又加加毛。)杜若(唐)蛇床子(和名比留无之吕。一名波未世利。)茵陈蒿(和名比歧与毛歧。)漏芦(和名久吕久佐。一名阿利久佐。)茜根(和名阿加檷。)飞廉(和名曾曾歧。又之保天。)营实(和名宇波良乃美。)薇衔(唐)五味(和名佐檷加都良。)旋(徐兜反)花(和名波也比止久佐。)白兔藿(唐)鬼督邮(和名乎止乎止之。又太止太止之乃奈。)白花藤(唐)

第八卷草中之上三十七种

当归(和名宇末世利。一名也末世利。又于保世利。一名加波佐久。)秦艽(音交,俗作胶非)(和名都加利久佐。又波加利久佐。)黄芩(渠今反)(和名比比良歧。又波比之波。)夕药(时药反)(和名衣比须久须利。一名奴美久须利。)干姜(和名久礼乃波之加美。)本(和名加佐毛知。又佐波曾良之。)麻黄(和名加都檷久佐。一名阿末奈。出赞歧国。)葛根(和名久须乃檷。)前胡(和名宇多奈。又乃世利。)知母(和名也末止已吕。又也未志。)大青(和名波止久佐。一名久留久佐。)贝母(和名波波久利。)栝(古活反)萎(和名加良须宇利。)玄参(和名于之久佐。)苦参(和名久良良。一名丁止利久佐。)石龙芮(和名之之乃比太比久佐。又布加都美。)石苇(和名伊芳波乃加波宇知。又伊芳波之。一名伊芳波久佐。)狗脊(和名于尔和良比。一名伊芳奴和良比,一名久末和良比。)萆(必檷反)(和名于尔止已吕。)拔(蒲八反)(苦八反)(和名宇久比须乃佐留加歧。又佐留止利。)通草(和名阿介比加都良。)瞿麦(和名奈天之古。)败酱(和名于保都知。又夕知女久佐。又加末久佐。)白芷(和名加佐毛知。一名与吕比久佐。又佐波宇止。一名佐波曾良之。)杜蘅(和名不多末加美。又都不檷久佐。)紫草(和名牟良佐歧。)紫菀(和名乃之。)白藓(和名比都之久佐。)白薇(和名美奈之古久佐。一名阿万奈。又久吕女久佐。)耳(和名奈毛美。)茅根(和名知乃檷。)百合(和名由利。)酸浆(和名保保都歧。一名奴加都歧。)紫参(和名知知乃波久佐。)女萎(和名惠美久佐。)淫羊藿(和名宇牟歧奈。又也末止利久佐。)蠡实(和名加歧都波太。)

第九卷草中之下三十九种

款冬(和名也未布布歧。又于保波。)牡丹(和名布加美久佐。又也末多知波奈。)防己(和名阿乎加都良。又佐檷加都良。)女菀(和名惠美乃檷。)泽兰(和名佐波阿良良歧。一名阿加末久佐。)地榆(和名阿也女牟,又衣比须檷,又衣比须久佐。)王孙(和名奴波利久佐,又乃波利。)爵床(和名乃加加毛。)白前(和名加加牟。)百部根(和名布止都良。)王(和名比佐久。)荠(和名佐歧久佐奈,一名美乃波。)高良姜(和名久礼乃波之加美乃宇止。又加波檷久佐。)马先蒿(和名波波古久佐。)蜀羊泉(唐)积雪草(和名都保久佐。)恶实(和名歧多歧须。又宇末布布歧,牛蒡也。)莎草(和名美久利。又佐久。)大小蓟(音计)根(和名阿佐美。)垣(音表)衣(和名之乃布久佐。)艾(五盖反)叶(和名与毛歧。)水萍(和名宇歧久佐。又以乎女。)海藻(和名之末毛。又尔歧女。又于古。)昆布(和名比吕女。又衣比须女。)荭(音红)(和名伊芳奴多天。)陟厘(和名阿乎乃利。)井中苔及萍(和名为乃美止利。)(胡木反)(和名奈歧。)凫葵(和名阿佐佐。)菟葵(和名以倍尔礼。)鳢肠(和名宇末歧多之,莲子草也。)酱(和名和多多比。)百脉根(唐)萝摩子(和名加加美。)白药(唐)香子(和名久礼乃于毛。)郁金(唐)姜黄(唐)阿魏(唐)

第十卷草下之上三十五种

大黄(和名于保之。)桔梗(和名阿利乃比布歧,又乎加止止歧。)甘遂(和名尔坡曾,又尔比曾。)葶苈(和名波末多奈。又波万世利。又阿之奈都奈。)芫花(唐,和名加尔比。)泽漆(和名波也比止久佐乃女。)大戟(和名波也比止久佐。)荛(人摇反)花(和名波末尔礼。)旋(似泉反)复花(和名加未都保。又加未保。)钩(古侯反)吻(唐)藜(力奚反)芦(和名也未宇波良。一名之之久比久佐。)赭魁(和名乌乃止止歧。)及(芨)己(音以)(和名都歧檷之佐。一名于宇。)乌头(和名于宇。)天雄(和名于宇。)附子(和名于宇。)侧子(和名于宇。)羊踯躇(和名以波都都之。又毛知都都之。一名之吕都都之。)茵芋(和名尔都都之。)射(音夜,考寒反)干(和名加良须安不歧。)鸢(音缘)尾(和名古也须久佐。)贯众(和名于尔和良比。)半夏(和名保曾久美。)由跋(和名加支都波奈多。)虎掌(和名于保保曾美。)莨菪子(和名于保美留久佐,又于尔保美久佐。)蜀漆叶(和名久佐歧也万宇都支乃波。恒山苗也。)恒山(和名久佐支。又宇久比须乃伊芳比檷。)青葙(和名宇末佐久。一名阿末佐久。)牙子(和名宇末都奈支。)白蔹(和名也末加加美。)白芨(音及)(和名加加美。)蛇全(和名宇都末女。)草蒿(和名于波歧。)菌(贝殒反)(唐)

第十一卷草下之下六十七种

连翘(和名从多知波二十,又从多知久佐。)白头(和名于歧奈久佐,又奈加久佐。)茹(和名檷阿佐美。又尔比末久佐。)苦(乌老反)(和名加未奈。又加美于吉之奈。)羊桃(和名从良良久佐。)羊蹄(和名之乃檷。)鹿藿(和名久须加都良乃波江。)牛扁(和名太知末知久佐。)陆英(和名曾久止久蒴也。)荩草(和名加歧奈。又阿之乃阿乌。)夏枯草(和名宇留比。)乌韭(和名知比佐歧古介。)蚤休(敬云草甘遂。)虎杖根(和名从多止利。)石长生(敬云KT筋草。)鼠尾草(和名美曾波歧。)马鞭草(和名久末都都良。)马勃(和名于尔不须倍。)鸡肠草(和名波久倍良。)蛇莓汁(和名倍美乃从知古。)苎根(和名宁乃檷。又加良牟之乃檷。)菰根(和名古毛乃檷。)野狼跋子(和名布知乃美。)蒴(和名曾久止久,敬云是陆英。)弓弩弦(和名于保由美乃都留。)舂杵头细糠(和名支檷乃波之乃奴加。)败蒲席(和名布留歧加末古毛。)败船茹(和名布檷乃阿久。)败鼓皮(和名都都美乃也礼加波。)败天公(和名多加佐乃也礼。)半天河(和名歧乃宇都保乃美都。)地浆(和名都知宁保利天都久留美都,是土浆。)屋游(和名也乃宇倍乃古介。陶云瓦屋上青苔衣。)赤地利(唐)赤车使者(唐)刘寄奴草(唐)三白草(和名加太之吕久佐。)牵牛子(和名阿佐加保。)猪膏莓紫葛(和名衣比加都良乃檷。)萆(卜继反)麻子(和名加良加之波乃三)(音律反)草(和名牟久良。)格注(主句反)草(唐)独行根苟(古浓反)舌草(唐)乌蔹莓(和名比佐古都良。)野狼毒(和名也末佐久。)鬼臼(和名奴波乃美。)芦根(和名阿之乃檷。)甘蕉根(和名波世乎波乃檷。)蓄(和名多知末知久佐。又宇之久佐。)酢浆草(和名加多波美。)实(和名以知比。)蒲公草(和名布知奈。一名多奈。)商陆(和名以乎须歧。)女青(和名加波檷久佐。)水蓼(和名美都多天。)角蒿昨叶何草(和名无。敬云瓦松。)白附子(唐)鹤虱(唐)甑带灰(和名古之支和良乃波比。)屐鼻绳灰(和名阿之太乃乎乃波比。)故麻鞋底(和名布留歧乎久都乃之歧。)雀麦(和名加良须牟支。又燕麦。)笔头灰和名不留支不氐乃都加乃波比。

第十二卷木上二十七种

茯苓(和名末都保止。)虎魄(和名阿加多末。)松脂(和名乎加末都乃也尔。)柏实(和名比乃美。)菌桂(唐)牡桂(唐)桂(唐)杜仲(和名波比末由三)枫香脂(和名加都良乃安不良。)干漆(和名如宇。)蔓荆实(和名波末波比。)牡荆实(殖近江国)女贞(和名太以乃歧乃三。)桑上寄生(和名久波乃支乃保也。)蕤核(唐)五茄(和名牟古支。)沉香(唐。诸香同是一树也。)柏木(和名歧波太。)辛夷(和名也末阿良良支。又古不波之加三。)木兰(和名毛久良尔。出大宰。)榆皮(和名也尔礼。又以倍尔礼。)酸枣(和名须支奈都女。一名佐檷布止。又宇奈以女。)槐实(和名惠须乃支乃三。)柠实(和名加知乃支。)枸杞(和名奴美久须檷。)苏合(和名加波美止利。唐)橘柚(和名太知波奈。又由。)

第十三卷木中二十八种

龙眼(和名佐加支乃三。)浓朴(和名保保加之波乃支。)猪苓(和名加之波支。又久奴支。一名也未加加波。)竹叶芹竹叶(和名久礼多介。又加波多介。)枳实(和名加良多知。)山茱萸(和名从多知波之加三。又加利波乃三。)吴茱萸(和名加良波之加三。)秦皮(和名止檷利古乃支。又太牟木。)栀子(和名久知奈之。)槟榔(和名阿知末佐。)合欢(和名檷布利乃支。)秦椒(和名加波波之加三,又古不之波之加三。)卫茅(和名加波久末都都良。)紫葳(和名乃宇世宇加都良。又末加也支。)芜荑(和名比歧佐久良。又也尔礼乃美。)食茱萸(和名于保多良乃三。)椋(力将反)子木(和名牟久乃支。)每始王木(唐)折伤木(和名从多比。)茗苦荼茗(和名荼。)桑根白皮(和名久波乃檷乃加波。)松萝(和名末都乃古介。)白棘(和名奈都女乃波利。)棘刺花(唐)安息香(唐)龙脑香(唐)庵摩勒(唐)毗黎勒(唐)

第十四卷木下四十五种

黄环(和名布知加都良。)石南草(和名止比良乃歧。)巴豆(唐)蜀椒(和名布佐波之加美。)莽草(和名之支美乃支。)郁核(和名宇倍。)鼠李(和名须毛毛乃支。)栾华(和名牟久礼之乃波奈。)杉材(和名须支乃支。)楠材(和名久须乃歧。)榧实(和名加倍乃美。)蔓椒(和名保曾歧。又从多知波之加美。)钓樟根皮(和名奈美久奴歧。)雷丸(唐。敬云是竹之苓也。)溲疏(和名宇都歧。)举树皮(和名之良久奴歧。又美久奴支。)白杨树皮(和名也奈歧。又波古歧。)水杨叶(和名由也奈歧。又加波也奈歧。)栾荆小柏〔(补草反)和名加波宇须歧歧波多。〕荚(古叶反)(唐)钓(丁叫反)藤(唐)药实根(唐)皂荚(和名加波良布知乃支。)楝实(和名阿布知乃美。)柳华(和名之多利也奈支。)桐叶(和名支利乃支。)梓白皮(和名阿都佐乃支。)苏方木(唐)接骨木(和名美也都古支。)枳(居纸反)(俱禹反)(唐)木天蓼(和名和多多比。)乌臼木(唐)赤爪(侧绞反)草(唐)诃黎勒(唐)风柳皮(唐)卖子木(和名加波知佐乃支。)大空(唐)紫真檀木(唐)椿(耻伦反)木叶(和名都波歧)胡椒(唐)橡实(和名都留波美乃美。)每食子(唐)杨庐木(和名宇都歧。)槲若叶(和名加之波歧。又久奴支。)

第十五卷兽禽五十六种

龙骨(和名多都乃保檷。)牛黄(唐)麝香(唐)人乳汁发皮〔(走孔反,又尸润反)和名人乃加美。〕乱发(和名介都利加美。)头垢(和名加之良乃安加。)人屎马乳牛乳羊乳酪苏熊脂(和名久末乃阿布良。)白胶(和名加乃都乃乃尔加波。)阿胶(和名尔加波。)醍醐(唐,苏之精液也。百练者也。好苏一石有三四升。)底野迦(唐)酪犀角(唐)羚羊角(和名加末之之乃以乃。)羊角(唐)牛角(和名宇之乃古以乃。)白马茎(和名安乎支马乃万良。)牡狗阴茎鹿茸(和名加乃和加都乃。)獐骨(和名乎之加乃保檷。)虎骨(唐)豹肉(唐,和名奈加以加三。)狸骨(和名多多介。)菟头骨(和名宇佐支。)六畜毛蹄甲(力佳反)鼠(和名毛美。)麋脂(和名于保之加乃阿布良。)豚卵(和名乃布久利。)鼹鼠(和名宇古吕毛知。)獭肝(和名乎曾。)狐阴茎(和名支以檷。)膏(和名美。)野猪黄(和名久佐为奈支。)驴屎(唐,和名宇佐支宇米。)豺皮(和名于保加美。)丹雄鸡(和名尔波止利。)白鹅膏(唐)肪(和名加毛。)雁肪(和名加利。)鹧鸪鸟雉肉(和名歧之。)鹰矢白(和名多加乃久曾。)雀卵(和名须须美。)鹳骨(和名于保止利。)雄鹊(和名加佐佐歧。)鸲鹆肉(和名尔波久奈不利。)燕矢(和名都波久良女。)孔雀矢(唐)鸬(和名宇。)头(和名止比。)

第十六卷虫鱼类七十二种

石蜜(敬云可除石字。)腊蜜(敬云可除蜜字。)蜂子(和名波知乃古。)牡蛎(和名乎加歧乃加比。)桑螵蛸(和名于保知加布久利。)海蛤(和名宇牟歧乃加比。)文蛤(和名从多也加比。)魁蛤(表有文。)石决明(和名阿波比乃加比。)秦龟(和名从之加女。)龟甲(和名宇三加女。)鲤鱼(和名古比。)彖鱼(和名波牟。)鲍鱼(唐,和名阿波比。)鱼(和名阿由。)鳝(和名牟奈歧。)鲫鱼(和名布奈。)伏翼(和名加波保利。)皮(和名久佐不。)石龙子(和名止加介。)露蜂房(和名于保波知乃须。)樗鸡(和名奴天乃支乃牟之。)蚱蝉(和名奈波世美。)白僵蚕(和名加比古。)木虻(和名于保安不。)蜚虻(和名古阿布。)蜚蠊(和名阿久多牟之。又都乃牟之。)虫(和名于女牟之。)蛴螬(和名须久毛牟之。)蛞(音舌)蝓(音腴)(和名奈女久知。)水蛭(之日反)(和名比留。)鳖甲(和名加波加女。)鳝(徒何反)鱼甲(和名古女,又江比。)乌贼鱼(和名以加。)蟹(和名加尔。)拥钮(和名加佐女。)天鼠矢(和名加波保利乃久曾。)螈(音元)蚕蛾(和名比比留乃布多古毛利。)鳗(莫安反)鲡(力号反)鱼(和名波之加美以乎。)鲛鱼(和名佐女。)紫贝(和名牟未乃久得保加比。)虾蟆(音麻)(和名比支。)蛙(和名加倍留。)牡鼠(和名乎檷须美。)蚺(而占反)蛇胆(唐)蝮蛇胆(和名波美。)陵鲤甲(唐)蜘蛛(和名久毛。)蜻蛉(和名加支吕布,又加介吕布。又加太千。)石蚕(唐)蛇蜕皮(和名倍美乃毛奴介。)蛇黄(蛇腹中得之。)蜈蚣(和名牟加天。)马陆(和名阿末比古。)(于佶反)(于公反)(和名佐曾利。)雀KT(于恭反)(和名须须美乃都保。)彼子(和名加加乃三。宜在木部。)鼠妇(和名于女牟之。)萤火(和名保多留。)衣鱼(和名之三)白颈蚯蚓(和名美美须。)蝼蛄(和名介良。)蜣螂(和名久曾牟之。)斑蝥(唐)芫青(唐)葛上亭长(和名久须加以良之牟之,又云唐。)地胆(唐)马刀〔(音凋)和名末天乃加比又都尧反。〕贝子(和名牟末乃都保加比。)田中螺〔(力弋反)汁和名多都比。〕蜗(古华反)牛(和名加多都布利。)甲香(和名阿支乃布多。)珂(唐)

第十七卷果二十五种

豆蔻(和名加宇礼牟加宇乃美。)蒲陶(和名于保衣比加都良。)蓬〔(力水反)(和名以知古。)〕覆盆(和名加宇布利以知古。陶云根名蓬,实名覆盆。)大枣(和名于保奈以女。)藕实(和名波知须乃美。)鸡头实(和名美以不不支乃美。)芰(奇寄反)。

实(和名比之。)栗(和名久利。)樱桃(和名加尔波久良乃三。又波波乃三。)梅实(和名牟女。兼名菀。一名同心。)枇杷(和名比波。)柿(和名加支。)木(和名毛介。)甘蔗(之夜反)(唐)石蜜(唐)沙糖(唐)芋〔(于付反)和名以以乇。〕乌芋(和名久吕久和乌。)杏核(和名加良毛毛。)桃核(和名毛毛。)李核(和名须毛毛。)梨(和名奈之,兼名菀。一名六俗。)柰(和名奈以。)安久榴(和名佐久吕。)

第十八卷菜三十八种

白瓜子(和名宇利乃佐檷。)白冬瓜(和名加毛宇利。)瓜蒂(和名尔加宇利乃保曾。)冬葵子(和名阿布比乃美。)葵根(和名阿不比乃檷。)苋(胡弁反)实(和名比由。)苦菜(和名尔加奈,又都波比良久久佐。)荠(和名奈都奈。)芜青(和名阿乎奈。)莱〔(音来)菔(蒲北反)和名于保檷。〕龙葵(和名古奈须比。)菘(和名多加奈。)芥(和名加良之。)苜(恩六反)蓿(和名于保比乃美。)荏(而枕反)子(和名于保衣乃美。)蓼(音了)实(和名多天。)葱实(和名歧乃三。)薤(下成反)(和名于保美良。)韭(音九)(和名古美良。)白(而羊反)荷(和名女加。)菜(和名布都久佐。)苏(和名以奴衣。又乃良衣。)水苏(和名知比佐支衣。)假苏(和名乃乃衣。又以奴衣。)香薷(而由反)(和名从奴衣,又从奴阿良良支。)薄荷〔(音哥)(唐)〕秦荻(徒历反)梨(唐)苦瓠(和名尔加比佐古。)水靳(又作芹)(和名厥利巨斤反。)马芹子(和名宇末世利。)(和名奴奈波。)落葵和名加良阿布比。

蘩(音烦)蒌(绿珠反)(和名波久倍良。)蕺(侧六反)(和名之布支。)葫(和名于保比留。)蒜(和名古比留。)堇(和名须美礼。)芸苔(和名宇知。)

第十九卷米谷二十八种

胡麻(和名宇古末。)青〔(私羊反)巨胜苗也,胡麻淳黑者名巨胜。〕麻(音坟)(和名阿佐乃美。)饴糖(和名阿女。)大豆黄卷(和名末女乃毛也之。)赤小豆(和名阿加阿都支。)豉(和名久支。)大麦(和名布止牟支。)(古猛反)麦(和名加良须牟支。)小麦(和名古牟支。)青粱米(和名阿波乃与檷。)黄粱米(和名支奈留支美。)白粱米(和名之吕支阿波。)粟米(和名阿波乃宇留之檷。)丹黍米(和名阿加支支美。)孽(鱼列反)米(和名毛也之。)秫米(和名阿波乃毛知。)陈廪(力甚反)米(和名布留支与檷。)酒(和名佐介。)腐婢(和名阿都支乃波奈。)扁豆(和名阿知末女。)黍米(和名支美。)粳(古行反)米(和名宇留之檷。)稻米(和名多多与檷。)稷米(和名支美乃毛知。)酢(和名须。)酱(和名比之保。)盐(和名之保。)

第二十卷有名无用药百九十三种无和名

本草外药七十种鬼皂荚(和名久久佐。)江浦草(和名都久毛。)茭弱(和名古毛乃古。)鹿毛菜(和名都之毛。)茭郁(和名古毛布都良。)鸭头草(和名都支久佐。)鸡冠草(和名加良阿为。)(和名古尔也久已。)上八种出《新撰食经》。

砺石〔(一名磨石)和名止。〕温石(今烧火熨人腰脚者。)鼠场土〔(一名鼠壤土)和名檷须美乃都知。〕仰天皮(是停污水干地皮卷起者。)土槟榔(此蟾蜍屎也,和名比支乃久曾。)啄木头〔(一名斫木鸟)和名天良都都支。〕百劳〔(一名)和名毛须。〕蒿雀(和名加也久支。)百舌鸟(一名莺)。

目(一名KT)(和布久吕布。)姑获(一名乳母鸟,一名钓鸟)。

鸟死蚕(蚕在簇上鸟死者。)蚕布KT(和名加比古乃以天加良。)鬼齿(一名鬼针。此腐竹根入地者。)阿勒勃(一名波罗皂荚。)榈木(一名棕榈)和名须吕乃支。)赤柽(和名牟吕乃支。)红蓝花(作燕支者。和名久礼乃阿为。)零陵香(一名燕草。)甘松香艾纳香兜纳香零余子(此薯蓣子,和名奴加古。)灯心草甜糟(和名阿末加须。)以上二十五种出《本草拾遗》。

石骨(出赤白石脂桃花石中,状如骨玉辈,故以以名之。)铁屑(此环铤镔等屑非生者。)蓝子(有小毒,岭南来。)石荆(一名生茵芋,花子似茺蔚。)牡蒙(一名白马茎,出山谷阴处。)木占斯(形如浓朴,有纵横纹理。)赤赫树(一名木藜芦,似郁李而少。)KT虱(似蓬蒿子而细。)蒿瘿(是蒿茎间毛瘿也,和名与毛支乃和多。)薰草(一名萱草。)三棱草(本草所谓莎草也。和名美久利。)练石草(苗细似苗蔓。)刀圭草(一名无心草。苗叶似小草,根似瞿麦。)漆姑草(一名苟尿珠。)槎牙草(一名慈菇。)瓦松(生尾瓦上似松。)胡葱青桴木(其木大者尺。)续骨木(缘树木叶如落石。)不灰木(生萧丘,虽燃而不糜。)朝菌雉口(治疮。)鸭头(治水肿。)牡鼠卵(治卵。)黄白赤(獐皮,治金疮。)猫屎(治疮,和名檷古末乃久曾。)野狼血(治久疥,和名于保加美乃知。)鳆鱼(治咳味。)陈久蚬壳(治胃反。)蚕砂(治胃反和名加比古乃久曾。)绿茧汁(治脚和名末由比介留之留。)桑蠹(和金疮肉生不足。)KT(煮糯米孽作之。)以上三十三种出《本草稽疑》。

仙沼子(和名之多都支。)续随子(一名百两金。)益蓝柒朴奈(和名久留倍支奈。)以上四种世用多验,但所出未详医心方卷第一医心方卷第一背记厥《病源论》曰∶尸厥者,阴气逆也寒热逆厥侯,夫厥者逆也。谓阴阳二气卒有衰绝,逆脐下《八十一难经》曰∶脉有三部,上部法天,主胸以上至头之有疾也。中部法人,主膈下奔豚气《病源论》曰∶贲豚气者,肾之积气。起于惊恐忧思所生云云。神志伤,动气积于肾以上第九叶。

疟《病源论》曰∶夏日伤寒,秋必病疟。

伤寒《病源论》曰∶冬时寒毒藏于肌骨中,至春变为温病;夏变为暑病,皆由冬时触冒之所干呕《病源论》曰∶干呕者,胃气逆故也。但呕而欲吐,吐而无所出,故谓之干呕也。

肺痈吐脓又云肺痈者,由风寒伤于肺,其气结聚所成也。又云肺痈有脓而呕者。

以上第十叶。

厥逆《病源论》曰∶厥者,逆也。谓阴气来于阳气也。
卷第二

夫《黄帝明堂经》、《华》、《扁》针灸法,或繁文奥义,卷轴各分;或上孔下穴,次第相违,既而去圣绵邈,后学暗昧。披篇按文之间急疾,又《经》治取艾作炷之处要穴易迷,是以头面手足胸胁腹背各随其处,尽抄其穴,主治之法,略注穴下,针灸之例,详附条末,专依轩宫之正经,兼拾诸家之别说,唯恐轻以愚憨之思,猥乱圣贤之踪,庸误乱圣旨,譬犹夏蛾之自迷灯,秋蝉之不知雪矣。
孔穴主治法第一

合六百六十穴(《明堂经》穴六百四十九;诸家方穴十一。)

头部诸穴六十八∶

头上五行行五,五五二十五穴∶第一行五穴∶囟会一穴∶(一名天窗,在上星后一寸陷者中,刺入四分灸五壮。督脉。主∶风眩,头痛,前顶一穴∶(在囟会后一寸半骨陷中,刺入四分灸五壮。督脉。主∶风眩,目瞑痛,恶风寒百会一穴∶(一名三阳五会,在前顶后一寸半顶中央旋毛中,刺入三分灸五壮。主∶疟,阳膀胱后顶一穴∶(一名交冲,在百会后一寸半,刺入四分,灸五壮。督脉。主∶风眩,目,强间一穴∶(一名大羽,在后顶后一寸半,刺入三分灸五壮。主∶癫疾狂走,螈,摇头,第二行左右十穴∶五处二穴∶(在督脉旁去上星一寸五分,刺入三分留七呼,灸三壮。此以泻诸阳气热,衄,承光二穴∶(在五处后一寸,不可灸,刺入三分。足太阳膀胱腑。主∶风眩头痛,欲呕,烦通天二穴∶(一名天臼,在承光后一寸半,刺入三分留七呼,灸三壮。主∶头痛,项痛,僵络却二穴∶(一名强阳,一名脑盖,一名反行。在通天后一寸半,刺入三分留五呼,灸五壮玉枕二穴∶(在络却后七分半,侠脑户旁一寸三分起肉枕骨上,入发际五寸,刺入二分留三痛。灸第三行左右十穴∶临泣二穴∶(在当目上、目直、目上、入发际五分陷者中,刺入三分留七呼,灸三壮。

主目窗二穴∶(一名至荣,在临泣后一寸,注云∶目上一寸五分是目之窗牖,故曰之,刺三分正营二穴∶(在目窗后一寸,刺入三分,灸五壮。主∶上齿痛,恶寒。足少阳胆腑,又阳维承灵二穴∶(在正营后一寸半,刺入三分,灸五壮。主∶脑风头痛,恶见风寒,鼽衄窒鼻,脑空二穴∶(一名颞,在承灵后一寸半,侠玉枕旁枕骨陷者中,刺入四分,灸五壮。

主∶头上五行外四十三穴∶头维二穴∶(在额角发际本神旁一寸五分,刺入五分,禁不可灸。足少阳胆腑,又足阳明胃脑户一穴∶(一名迎风,一名合颅。在枕骨上强间后一寸五分,不可灸,刺入二分,留二呼汗出颔厌二穴∶(在曲周颞上廉,刺入三分留七呼,灸三壮。足少阳胆,足太阳膀胱腑,足阳天冲二穴∶(在耳上如前三寸,刺入三分灸九壮。又足少阳胆,足太阳膀胱。主∶头痛、痉蟀谷二穴∶(在耳上入发际一寸半,嚼而取之,刺入四分,灸三壮。足太阳膀胱腑,又入足曲鬓二穴∶(在耳上发际曲隅陷者中,刺入四分灸三壮。足太阳膀。又入足少阳胆腑。

主∶浮白二穴∶(在耳后入发际一寸,刺入三分灸三壮。足太阳膀胱腑,如上又入足少阳胆腑。

完骨二穴∶(在耳后入发际四分,刺入二分留七呼,灸三壮。足小阳胆腑,又足太阳膀胱同窍阴二穴∶(在完骨上枕骨下,摇动手而取之。刺入四分灸五壮。足少阳胆腑,又足太阳膀悬颅二穴∶(在曲周颞中,注云∶在曲颔骨上,刺入三分,留三呼,灸三壮。足阳明胃腑悬厘二穴∶(在曲周颞下廉,刺入三分留七呼,灸三壮。足阳明胃,足少阳胆腑。主∶发上关二穴∶(一名客主人,在耳前上廉起骨,开口有空,刺入三分留七呼,灸三壮。足阳明寒耳门二穴∶(在耳前起肉当耳缺者中,刺入二分留三呼,灸三壮。足阳明脉胃。主∶耳鸣,听宫二穴∶(在耳中珠子大如赤小豆,刺入一分灸三壮。足少阳胆腑,又手少阳三焦。

主∶听会二穴∶(在耳前陷者中张口得之,刺入分灸三壮。手太阳又手少三焦。主∶聋、齿痛、角孙二穴∶(在耳廓中间上开口有空,刺入三分灸三壮。足少阳胆腑。主∶牙齿不可嚼,龈下关二穴∶(在客主人下耳前动脉下空下廉,合口有空,张口而闭。刺入三分灸三壮。

足少和二穴∶(在耳前兑发下动脉,刺入三分灸三壮。足少阳胆。主∶头重,颔痛,引耳中之KTKTKTKT。又手太阳脉小肠腑,又手少阳三焦。)颅息二穴∶(在耳后间青脉,刺入一分出血多杀人,灸三壮。手少阳三焦。主∶身热,头胁螈脉二穴∶(一名资脉,在耳本鸡足青脉,刺出血如豆,今按∶《千金方》不灸。主∶小儿翳风二穴∶(在耳后陷者中,按之引耳中,刺入四分各三壮。手少阳三焦,又足少阳胆。

主风府一穴∶(一名舌本,在项后入发际一寸大筋内宛中起肉,刺入四分留三呼,不可灸。

欲自喑门一穴∶(一名舌厌,一名舌横,在项中发际宛宛中,入系舌本,刺入四分不灸。主∶项

面部诸穴三十九∶

面一行从上星直下至承浆七穴∶上星一穴∶(在颅上鼻直上中央入发际一寸,刺入三分留六呼,灸五壮。督脉。主∶风眩,神庭一穴∶(在发际直鼻,不可刺,灸三壮。督脉,又足阳明胃,又足太阳膀胱。主∶寒热素一穴∶一名面王,在鼻柱,刺入三分。督脉。主∶鼽衄涕出,中有悬痈宿肉,窒洞不木沟一穴∶(在鼻柱下人中低唇取之,刺入三分留六呼灸三壮。督脉,又手阳明大肠。

主∶兑端一穴∶(在唇上尖锐之端,刺入二分留六呼,灸三壮。手阳明大肠。主∶癫疾,呕沫,龈交一穴∶(在唇内齿上龈缝,注云∶上齿龈间。刺入三分灸三壮。足阳明胃腑。主∶风寒承浆一穴∶(一名天地,在颐前,下唇之下,开口取之,刺入二分留六呼,灸三壮。任脉。

止。)面一行外左右三十二穴∶曲差二穴∶(一名鼻冲,在侠神庭旁一寸五分,在发际。刺入三分灸五壮。又足太阳脉膀本神二穴∶(在侠曲差旁一寸五分发际。刺入三分灸五壮。阳维脉,足少阳胆。主∶头痛,阳白二穴∶(在眉上一寸直瞳子。刺入三分。阳维脉。主∶头、目瞳子痛,不可以视,使项攒竹二穴∶(一名员柱,一名始光,一名夜光,一名明光。在眉头陷者中,刺入二分灸三壮丝竹空二穴∶(一名目,在眉后陷者中,刺入三分留三呼,禁不可灸。主∶头痛,目中赤精明二穴∶(一名泪孔,在目内,刺入一分留六呼,灸三壮。主∶目泪出,憎风寒,头痛瞳子二穴∶(在目外去五分,刺入三分,灸三壮。足少阳胆,又手太阳小肠,又手厥阴承泣二穴∶(一名鼷穴,一名面,在目下七分,直瞳子,刺入三分不灸。主∶目不明,泪四白穴∶(在目下一寸,刺入四分。主∶目痛,口,泪出,目不明。足阳明胃腑。)颧二穴∶(一名兑骨,在面鼽骨下廉陷骨下。刺入三分。手太阳小肠,又手。主∶口,巨二穴∶(在侠鼻旁八分直瞳子,刺入三分。主∶面目恶风,翳膜,口,青盲。足阳明迎香二穴∶(一名衡阳,在禾上,鼻下孔旁,刺入三分,参(灸)三壮。足阳明胃腑,又手禾二穴∶(一名,在直鼻孔下侠水沟旁五分,灸三壮。主∶鼻窒,口,清涕不止,鼽地仓二穴∶(一名胃维,侠口旁四分,刺入三分。足阳明胃腑,又阳跷脉,又手阳脉明大腹颊车二穴∶(在耳下曲颊端陷者中,开口有空,刺入三分,灸三壮。主∶牙车骨痛,齿不可大迎二穴∶(一名髓空,在曲颔前一寸二分陷者中,刺入三分,留七呼,灸三壮。主∶寒热颐下部穴二∶中矩一穴∶(一名垂矩,在颐下骨里曲骨中。此一穴出《华佗传》也。主∶中风舌强不能言廉泉一穴∶(一名本池。在颐下,结喉上舌本。刺入三分留三呼,灸三壮。任脉,又阴维脉

颈部左右诸穴二十∶

天牖二穴∶(在颈筋缺盆上,天容后,天柱前,完骨下,发际上。刺入一寸,留七呼,灸三鼻,喉天柱二穴∶(在侠项后发际,大筋外廉陷者中,刺入二分,留六呼,灸三壮。足太阳膀胱。

风池二穴∶(在颞后发际陷者中,灸三壮。足少阳胆,又阳维脉。主∶寒热,癫仆,狂,天窗二穴∶(一名窗聋。在曲颊下,扶突后,动脉应手陷者中,刺入六分,灸三壮。手太阳天容二穴∶(在耳下曲颊后。刺入一寸灸三壮。手少阳三焦,又足少阳胆。主∶寒热,喉痹人迎二穴∶(一名天五会,在颈大脉动应手,侠结喉旁,禁不可灸,刺四分。足阳明胃。

主水突二穴∶(一名水门,在颈大筋前,直人迎下。气舍上,刺入四分,灸三壮。足阳明胃。

气舍二穴∶(在颈,直人迎,侠天窗后陷者中。刺入四分,灸三壮。足阳明胃。主∶咳逆上天鼎二穴∶(在颈缺盆,直扶突,气舍后一寸半,刺入四分,灸三壮。手阳明大肠。主∶暴扶突二穴∶(一名水穴。在曲颊下一寸,人迎后,仰而取之。刺入四分,灸三壮。手阳明大

肩部左右诸穴二十六∶

秉风二穴∶(在侠天鼎外肩上后,举臂有空,举臂取之。刺入五分,灸五壮。主∶肩痛不肩井二穴∶(在肩上陷解中,缺盆上,大骨前。刺入五分,灸三壮。注云∶大骨,谓胛上廉眠卧。

巨骨二穴∶(在肩端上行两叉骨间陷者中,注云∶肩端大骨间。刺入一寸半,灸五壮。

主∶肩二穴∶(在肩端两骨间。刺入六分,留六呼,灸三壮。手阳明大肠,又阳跷脉。主∶肩肩中俞二穴∶(在肩胛内廉,去脊二寸陷者中。刺入三分,留七呼,灸三壮。足太阳膀胱。

肩外俞二穴∶(在肩胛上廉,去脊三寸陷者中。刺入六分,留六呼,灸三壮。手少阳上三焦缺盆二穴∶(一名天盖。在肩上横骨陷者中。刺入二分,留七呼,灸三壮。足少阳胆。

主∶天二穴∶(在肩缺盆中上,毖骨之陬陷者中。刺入八分,灸三壮。手少阳三焦,又足少阳天宗二穴∶(在秉风后,大骨下陷者中,刺入五分,留六呼,灸三壮。手太阳小肠。主∶肩肩贞二穴∶(在肩曲胛下两骨解间,肩后陷者中。刺入八分,灸三壮。手阳明大肠。

主∶肩二穴∶(在肩端上斜,举臂肩取之。刺入七分,灸三壮,手阳明大肠。主∶肩重不举俞二穴∶〔在侠肩后大骨下胛上廉陷者中,刺入八分,灸三壮。手太阳小肠,又阳维,曲垣二穴∶(在肩中央曲胛陷者中。刺入九分,灸十壮。手太阳小肠。主∶肩胛周痹。)

手部左右诸穴百二十∶

会二穴∶〔一名(肩)。在臂前廉,去肩头三寸。刺入五分,灸五壮。手阳明大肠。

主极泉二穴∶(在腋下两筋间动脉。注云∶腋下臂极处。刺入四分,灸三壮。手少阳心。

主∶天泉二穴∶(一名天湿,在曲腋下臂三寸,举腋取之。有本∶在腋下前偶二骨间陷者中,刺天府二穴∶(在腋下三寸,臂内廉动脉。禁不可灸,刺入四分留三呼。手太阴肺。主∶咳臂二穴∶(在肘上七寸KT肉端。刺入三分灸三壮。KT肉谓分肉块也。手阳明大肠。主∶寒侠白二穴∶(在天府下去肘五寸。刺入四分,灸五壮。手太阴肺。主心痛,咳,干呕,烦满消泺二穴∶(在肩下臂外关腋横斜肘分下行,刺入六分灸三壮。主∶寒热,痹,头痛,项背五里二穴∶(在肘上三寸半,不可刺,灸十壮。手阳明大肠。主∶嗜卧,四肢不欲动摇,身清冷渊二穴∶(在肘上三寸,伸肘举臂取之。刺入三分,留三呼,灸三壮。主∶肩不举,不天井二穴∶(在肘外大骨之后,肘后一寸,两节间陷者中,屈肘得之。刺入一寸,灸三壮。

肘二穴∶(在肘大骨外廉陷者中。刺入四分,灸三壮。手阳日大肠。主∶肘节酸重、痹痛小海二穴∶(在肘内大骨外,去肘端五分陷者中,屈肘乃得之。刺入二分,留七呼,灸五壮肢不举尺泽二穴∶(在肘中约上动脉。有本云∶在肘屈大横纹中。刺入三分,留三呼,灸三壮。

手曲泽二穴∶(在肘内廉下陷者中,屈肘得之。刺入三分,留七呼,灸三壮。手厥阴心主。

主少海二穴∶(一名曲节。在肘内廉节后陷者中。刺入五分,灸三壮。主∶身热,疟,逆气曲池二穴∶(在肘外辅,屈肘曲骨之中。刺入五分,留六呼,灸三壮。手阳明大肠。主∶肩三里二穴∶(在曲池下二寸,按之肉起兑肉之端。刺入三分,灸三壮。同上。主∶腹,肘上廉二穴∶(在三里下一寸。刺入五分,灸三壮。同上。主∶小便黄,腹鸣相追逐。)下廉二穴∶(在辅骨下去上廉一寸。刺入五分,灸三壮。同上。主∶眼痛,尿黄。)孔最二穴∶(在腕上七寸。刺入三分,灸五壮。手太阴肺。可以出汗,头痛,振寒,臂厥,四渎二穴∶(在肘前五寸外廉陷者中,刺入六分,留七呼,灸三壮。手少阳三焦。主∶卒气支正二穴∶(在腕后五寸,刺入二分,留七呼,灸三壮。手太阳小肠。主∶振寒,寒热,颈温溜二穴∶(一名逆注,一名蛇头。在腕后,大士六寸,小士五寸。刺入三分,灸三壮。

手门二穴∶(在腕五寸。刺入三分,灸三壮。手厥阴心主。主∶心痛,衄,哕,呕血,惊恐偏历二穴∶(在腕后三寸。刺入三分,留七呼,灸三壮。主∶寒热,汗不出,风疟,目茫茫三阳络二穴∶(在臂上大交脉,支沟上一寸,不可刺,灸九壮。手少阳三焦。主∶嗜卧,身会宗二穴∶(在腕后三寸空中,刺入三分,灸三壮。注云∶空中一寸有上、中、下,总为会支沟二穴∶(在腕后三寸两骨间陷者中。刺入二分,留七呼,灸三壮。手少阳三焦。主∶热间使二穴∶(在掌后三寸,两筋间陷者中。刺入六分,留七呼,灸七壮。手少阳三焦。

主∶外关二穴∶(在腕后二寸陷者中。刺入三分,灸三壮。手少阳三焦。主∶肘中濯濯,耳淳淳内关二穴∶(在掌后去腕二寸。刺入三分,主∶面赤,皮热,热病汗不出。心悲,心暴痛,列缺二穴∶(在腕上一寸半。刺入三分,留三呼,灸五壮。手太阴肺。主∶疟寒甚热,痫惊灵道二穴∶(在掌后一寸半,或曰∶一寸。刺入三分,灸三壮,正手取之,手少阴心。

主∶通里二穴∶(在腕后一寸,刺入三分,灸三壮。手少阴心。主∶热痛,心痛,苦吐,头痛,养老二穴∶(在踝骨上一空,在后一寸,灸三壮。手太阳小肠。主∶肩痛欲折,如拔,手阳池二穴∶(一名别阳。在手表腕上陷者中。刺入二分,留六呼,灸三壮。手少阳三焦。

主阳溪二穴∶(一名中魁。在腕中上侧两筋陷者中。刺入三分,留七呼,灸三壮。手阳明大肠阳谷二穴∶(在手外侧,腕中兑骨下陷者中。刺入二分,留二呼,灸三壮。同前小肠。

主∶腕骨二穴∶(在手外侧,腕前起骨下陷者中。刺入三分,留三呼,灸三壮。手太阳小肠,主经渠二穴∶(在寸口陷者中。注云∶从关至鱼一寸,故曰寸口。刺入三分,留三呼,不可灸阴二穴∶(在掌后脉中,去腕半寸,刺入三分,灸三壮。此空手少阴雄已,手少阴脉心脏太渊二穴∶(在手掌后陷者中。刺入二分,留二分,灸三壮。手太阴肺。主∶痹,逆气寒厥大陵二穴∶(在掌后两骨之间陷者中。刺入六分,留七呼,灸三壮。手太阴肺。主∶热痛,神门二穴∶(一名兑衡,一名中都。在掌后兑骨之陷端者中。刺入三分,留七呼,灸三壮。

劳宫二穴∶(一名五星。在掌中央。刺入三分,留六呼,灸三壮。手厥阴心主。主∶热病烦合谷二穴∶(一名虎口。在手大指歧骨之间。刺入三分,留六呼,灸三壮。手阳明大肠。

主鱼际二穴∶(在手大指本节后内侧散脉。刺入二分,留三呼,灸三壮。手阳明大肠。主∶虚少商二穴∶(在手大指端内侧,去爪甲角如韭叶。刺入一分,留一呼,灸一壮。手太阴肺。

三间二穴∶(一名少谷。在手大指次指本节后内侧陷者中。刺入三分,灸三壮。手阳明大肠二间二穴∶(一名间谷。在手大指次指本节前,内侧陷者中。刺入三分,留六呼,灸三壮。

口商阳二穴∶(一名而明,一名绝阳。在手大指次指内侧去爪甲角如韭叶,刺入一分,留一呼背痛,后溪二穴∶(在手小指外侧本节后陷者中。刺入一分,留一呼,灸一壮。手太阳小肠腑穴也前谷二穴∶(在手小指外侧本节后陷者中,刺入一分,留三呼,灸三壮。同前小肠。主∶热少泽二穴∶(在手小指之端,去爪甲下一分陷者中。刺入一分,留二呼,灸一壮。同前小肠中渚二穴∶(在手小指次指七节间陷者中。刺入三分,留三呼,灸三壮。手少阳三焦。

主∶腋门二穴∶(在手小指次指间陷者中。刺入二分,留二呼,灸三壮。手少阳三焦腑。主∶热关冲二穴∶(在手小指次指之端,去爪甲如韭叶,刺入一分,留二呼,灸三壮。手少阳三韭中衡二穴∶(在手中指端,去爪甲如韭叶陷者中。刺入一分,留三呼,灸一壮。手厥阴心主少府二穴∶(在手小指本节后陷者中。刺入三分,灸五壮。手少阴心脏。主∶阴痛、挺长,少衡二穴∶(一名经始。在小少指内廉之端,去爪甲如韭叶。刺入一分,留一呼,灸一壮。

痛。)

背部诸穴七十九∶

一行从大椎直下至骨端十一穴∶大椎一穴∶(在第一椎上陷者中。刺入五分,灸九壮。督脉,又手阳小肠,又足太阳膀胱,又手少阳三焦腑。主∶伤寒热盛,烦呕也。)陶道一穴∶(在项大椎节下间,俯而取之。刺入五分,留五呼,灸五壮。督脉,又足太阳膀身柱一穴∶(在第三椎节下间。刺入五分,灸五壮。督脉。主∶身热狂走,言,见鬼,螈,癫疾,怒欲杀人。)神道一穴∶(在第五椎节下间,仰而取之。刺入五分,灸三壮。督脉。主∶身热痛,进退往至阳一穴∶(在第七椎下间,俯而取之。刺入五分,灸三壮。督脉。主∶寒热解烂,淫泺胫筋缩一穴∶(在第九椎节下间,俯而取之。刺入五分,灸三壮。督脉。主∶惊痫,螈狂走脊中一穴∶(在第十一椎节下间。刺入五分,不可灸,令人偻也。督脉。主∶腰脊强,下得悬枢一穴∶(在第十三椎节下间。刺入三分,灸三壮。督脉。主∶腹中积气上下行,不仁。

命门一穴∶(一名属累。在第十四椎节下间。伏而取之。刺入五分,灸三壮。督脉。主∶头腰输一穴∶(一名背解,一名髓孔,一名腰柱,一名腰户。在第二十一椎节下间。刺入三寸长强一穴∶(一名气之阴。在脊端。刺入二寸,留七呼,灸三壮。督脉。主∶腰痛上寒二行左右四十二穴∶大抒二穴∶(在项第一椎下旁各一寸半陷者中。刺入三分留七呼,灸三壮。足太阳膀胱,又风门热府二穴∶(在第二椎下两旁各一寸半。刺入五分,灸五壮。足太阳膀胱,又督脉。

主肺输二穴∶(在第三椎下两旁各一寸半。刺入三分,留七呼,灸三壮。足太阳膀胱。主∶肺泣心输二穴∶(在第五椎下两旁各一寸半。刺入三分,留七呼,灸三壮。足太阳膀胱。主∶寒膏肓输二穴∶(《千金方》云∶主无所不治。羸瘦虚损,梦中失精,上气咳逆,狂惑妄误。

求穴大较以右手后右肩上往指头表所不及者也。左手亦然也。灸六百壮多至千壮。)膈输二穴∶(在第七椎下两旁各一寸半。刺入三分,灸三壮。足太阳膀胱。主∶咳,膈寒,肝输二穴∶(在第九椎下两旁各一寸半。刺入三分,留六呼,灸三壮。足太阳膀胱。主∶两胆输二穴∶(在第十椎下两旁各一寸半,正坐取之。刺入五分,灸三壮。同前膀。主∶胸满脾输二穴∶(在第十一椎下两旁各一寸半。刺入三分,留七呼,灸三壮。同前。主∶腹中气肢急烦胃输二穴∶(在第十二椎下两旁各一寸半。刺入三分,留七呼,灸三壮。同上。主∶胃中寒三焦输二穴∶(在第十三椎下两旁各一寸半,正坐取之。刺入五分,灸三壮。同上。主∶头肾输二穴∶(在第十四椎下两旁各一寸半,刺入三分,留七呼,灸三壮。同上。主∶腰痛,大肠输二穴∶(在第十六椎下两旁各一寸半,刺入三分,灸三壮。同上。主∶大肠转气,按小肠输二穴∶(在第十八椎下两旁各一寸半,刺入三分,灸三壮。同上。主∶少肠痛热控睾膀胱输二穴∶(在第十九椎下两旁各一寸半。刺入三分,灸三壮。同上。主∶脊痛强引背少中膂内输二穴∶(在第二十椎下两旁各一寸半。刺入三分,留十呼,灸三壮。同上。主∶腰白环输二穴∶(在第二十一椎下两旁各一寸半,伏而取之。刺入五分,灸三壮。同上。

主∶上二穴∶(在第一空腰下一寸,侠脊陷者中,谓腰髋骨而旁高处也。宽骨上有八穴,此出不次二穴∶(在第二空侠脊陷者中,刺入三寸,灸三壮。同上。主∶腰痛怏怏不可俯仰,腰中二穴∶(在第三空侠脊陷者中刺入二寸,灸三壮。同上。主∶腰痛,大便难,飧泄,少腹下二穴∶(在第四空侠脊陷者中,刺入二寸,灸三壮。主∶腰痛不可反侧,尻中痛,女三行左右二十六穴∶附分二穴∶(在第二椎下,附项内廉两旁各三寸。刺入八分,灸三壮。手太阳小肠,又足太魄户二穴∶(在第三椎下两旁各三寸,正坐取之。刺入五分,灸五壮。足太阳膀胱。主∶肩间急,凄厥恶寒,咳逆上气,呕吐,烦满,项背痛引颈。)神堂二穴∶(在第五椎下两旁各三寸陷者中,正坐取之。刺入三分,灸五壮。同前膀胱。

主噫嘻二穴∶(在肩内廉,侠第六椎下,两旁各三寸。刺入六分,灸五壮。同前膀胱。

主∶膈关二穴∶(在第七椎下两旁各三寸陷者中,阔肩取之。刺入五分,灸三壮。同前膀胱。

主魂门二穴∶(在第九椎下两旁各三寸陷者中,正坐取之。灸三壮。同前膀胱。主∶胸胁满,阳纲二穴∶(在第十椎下两旁各三寸陷者中,正坐取之。刺入五分,灸三壮。同前膀胱。

主意舍二穴∶(在第十一椎下两旁各三寸陷者中,正坐取之。刺入五分,灸三壮。主∶腹中满胃仓二穴∶(在第十二椎下两旁各三寸。刺入五分,灸三壮。同前膀胱。主∶胪胀水肿,食肓门二穴∶(在第十三椎下两旁各三寸叉肋间。刺入五分,灸三十壮。同前膀胱。主∶心下大志室二穴∶(在第十四椎下两旁各三寸陷者中,正坐取之。刺入五分,灸三壮。同前膀胱。

胞肓二穴∶(在第十九椎下两旁各三寸陷者中,伏而取之。刺入五分,灸三壮。同前膀胱。

秩边二穴∶(在第二十一椎下两旁各三寸陷者中,伏而取之。刺入五分,灸三壮。主∶腰痛

胸博诸穴四十三∶

一行从天突直下至中庭七穴∶天突一穴∶(一名五户,在颈结喉下五寸中央宛宛中。刺入一寸,留七呼,灸三壮。任脉,又热,心旋机一穴∶(在天突下一寸,仰头取之。刺入三分,灸五壮。任脉。主∶胸满痛,喉痹咽肿华盖一穴∶(在旋机下一寸陷者中,仰而取之。刺入三分,炙五壮。主∶胸胁支满,骨痛引紫宫一穴∶(在华盖下一寸六分陷者中,仰而取之。灸五壮。任脉。主∶胸中KT满,痹痛骨玉堂一穴∶(一名玉英,在紫宫下一寸六分陷者中。灸五壮。任脉。主∶胸满不得喘息,胁膻中一穴∶(一名元儿。在玉堂下一寸六分,直两乳间陷者中,仰而取之。灸五壮。任脉。

中庭一穴∶(在膻中下一寸六分陷者中。刺入三分,灸五壮。任脉。主∶胸胁KT满,膈塞,二行左右十二穴∶输府二穴∶(在巨骨下,旋机旁各二寸陷者中,仰卧而取之。刺入四分,灸五壮。足少阴肾或中二穴∶(在输府下一寸六分陷者中,仰卧而取之。刺入四分,灸五壮。足少阴肾。

主∶神藏二穴∶(在或中下一寸六分陷者中,仰而取之。刺入四分,灸五壮。足少阴肾。主∶胸灵墙二穴∶(在神藏下一寸六分陷者中,仰而取之。刺入四分,灸五一。足少阴肾。主∶胸神封二穴∶(在灵墙下一寸六分。刺入四分,灸五一。主∶胸满不得息,咳逆,乳痈,洒沂步郎二穴∶(在神封下一寸六分,陷者中,仰而取之。刺入四分,灸五壮。主∶胸胁KT满,三行左右十二穴∶气户二穴∶(在巨骨下,输府旁各二寸陷者中,仰而取之。刺入四分,灸五壮。足阳明胃。

库房二穴∶(在气户下一寸六分陷者中,仰而取之。刺入四分,灸五壮。主∶胸胁KT满,咳屋翳二穴∶(在库房下一寸六分陷者中,仰而取之。刺入四分,灸五壮。同上胃。主∶身肿膺(于陵反)窗二穴∶(在屋翳下一寸六分陷者中。刺入四分,灸五壮。同上胃。主∶胸胁肿乳中二穴∶(此穴居处当乳中央,故曰之禁不可灸。灸之不幸生蚀疮,疮中有脓血清汁者可乳根二穴∶(在乳下一寸六分陷者中,仰而取之。刺入四分,灸五壮。同前胃。主∶胸下满四行左右十二穴∶云门二穴∶(在巨骨下气户两旁各二寸陷者中,举臂取之。注云∶巨骨谓是缺盆下畔横大骨中府二穴∶(肺募也。一名膺中输。在云门下一寸,乳上三肋间动脉应手陷者中。刺入三分面腹周荣二穴∶(在中府下一寸六分陷者中,仰而取之。刺入四分,灸五壮。主∶胸胁KT满,不胸乡二穴∶(在周荣下一寸六分陷者中,仰而取之。刺入四分,灸五壮。足太阴脾。主∶胸天溪二穴∶(在胸乡下一寸六分陷者中,仰而取之。刺入四分,灸五壮。同上。主∶胸中满食窦二穴∶(在天溪下一寸六分陷者中,举臂取之。刺入四分,灸五壮。足太阴脾。主∶胸

腹部诸穴七十四∶

一行从鸠尾直下至曲骨十四穴∶鸠尾一穴∶(一名尾翳,一名,在臆前蔽骨下五分,人有无鸠尾者,当从臆前骨歧下一血痂巨阙一穴∶(心募也。在鸠尾穴下五分,去鸠尾骨端一寸。灸五壮,刺入六分,留七呼。

任呕吐上脘一穴∶(在巨阙下一寸。刺入八分,灸五壮。主∶胃脘中伤,饱食不化,五脏肠胀,心中脘一穴∶(一名太仓,胃募也。在上管下一寸,居蔽骨脐中。刺入一寸二分,灸七壮。

主烦明建里一穴∶(在中管下一寸,刺入五分,留十呼,灸五壮。主∶心痛上抢心,不欲食,支痛下管一穴∶(在建里下一寸,刺入五分,灸五壮。任脉。主∶食欲不化,入肠还出。注云∶水分一穴∶(在下管下,脐上一寸。刺入一寸,灸五壮。主∶痉,脊强,里急,腹中拘痛。

脐中一穴∶(禁不可刺,刺之使人脐中恶疡溃,矢出者死。灸三壮。任脉。主∶疝,绕脐痛阴交一穴∶(一名少因,一名横户。在脐下一寸。刺入五分,灸五壮。任脉,又冲脉。

主∶坚痛。

气海一穴∶(一名脖注云∶,脐也。一名下肓,在脐下一寸半。刺入一寸二分,灸五壮石门一穴∶(三焦募也。一名利机,一名精露,一名丹田,一名命门。在脐下二寸。刺入五足少关元一穴∶(小肠募也。一名门。在脐下三寸,刺入二寸。留七呼,灸七壮。又任脉,足胁下中极一穴∶(膀胱募也。一名气原,一名玉泉。在脐下四寸。刺入三寸,留七呼,灸三壮。

腹苦寒曲骨一穴∶(在横骨上,中极下一寸阴毛际陷者中。刺入一寸半,灸三壮。足厥阴肝,又任恶合阴二行左右二十二穴∶幽门二穴∶(一名上门,在巨阙旁半寸陷者中。刺入五分,灸五壮。又冲脉。主∶胸背相引通谷二穴∶(在幽门下一寸陷者中。刺入五分,灸五壮。足少阴肾,又冲脉。主∶失欠,口阴都二穴∶(一名食宫,在通谷下一寸。刺入五分,灸五壮。足少阴肾,又冲脉,主∶身寒石关二穴∶(在阴都下一寸。刺入一寸,灸五壮。主∶痉,脊强,口不可开,多唾,大便难高曲二穴∶(在石关下一寸,刺入一寸,灸五壮。足少阴肾,又冲脉。主∶大腹积聚,时切肓输二穴∶(在高曲下一寸,直脐旁五分。刺入一寸,灸五壮。足少阴肾,又冲脉。主∶大中注二穴∶(在肓输下五分。刺入一分,灸五壮。冲脉,又足少阴肾。主∶少腹有热,大小四满二穴∶(一名髓府。在中注下一寸。刺入一寸,灸五壮。足少阴肾,又冲脉。主∶脐下气穴二穴∶(一名胞门,一名子户。在四满下一寸。刺入一寸,灸五壮。足少阴肾,又冲脉大赫二穴∶(一名阴关,一名阴维。在气穴下一寸。刺入一寸,灸五壮。足少阴肾,又冲脉横骨二穴∶(一名下极,在大赫下一寸。刺入一寸,灸五壮。足少阴肾,又冲脉。主∶小三行左右二十四穴∶不容二穴∶(在幽门旁各一寸五分。刺入五分,灸五壮。足阳明胃。主∶呕血,肩息,胁下承满二穴∶(在不容下一寸,去上管各二寸,刺入八分,灸五壮。同前胃。主∶胁下痛,腹梁门二穴∶(在承满下一寸,去中管各二寸。刺入八分,灸五壮。同前胃。主∶胁下积气结关明二穴∶(在梁门下,太一上,去建里各二寸。刺入八分,灸五壮。同前胃。主∶身肿、太一二穴∶(在梁门下一寸。刺入八分,灸五壮。同前胃腑。主∶狂,癫疾,吐舌。)滑肉门二穴∶(在太一下一寸,去水分二寸。刺入八分,灸五壮,同前胃。主∶狂,痫,癫天枢二穴∶(大肠募也,一名长溪,一名谷门。去肓输一分五分,侠脐两旁各二寸陷者中。刺不休止外陵二穴∶(在天枢下,大巨上。刺入八分,灸五壮。同上胃。主∶肠中尽痛也。)大巨二穴∶(一名液门。在长溪下二寸。刺入八分,灸五壮。同上胃。主∶腹满痛,善烦,水道二穴∶(在大巨下三寸。刺入二寸半,灸五壮。同上胃。主∶少腹胀满,痛引阴中,月归来二穴∶(一名溪穴。在水道下二寸,注云∶有本∶侠曲骨相去五寸。刺入八分,灸五壮气街二穴∶(在归来下,鼠鼷上一寸刺入三分,留七呼,灸三壮。主∶腹中有大气,私使,四行左右十四穴∶期门二穴∶(肝募也。在第二肋端,不容旁各一寸五分上,直两乳,去巨阙各三寸五分,举食日月二穴∶(胆募也,在期门下五分。刺入七分,灸五壮。足少阳胆,足太阴脾。主∶太息肠哀二穴∶(在日月下一寸五分,去中管各三寸五分,刺入七分,灸五壮。主∶便脓血,寒大横二穴∶(在肠哀下三寸,直脐旁,刺入七分,灸五壮。主∶大风,逆,多寒,善悲也。

肠结二穴∶(一名肠屈。在大横下一寸三分。刺入七分,灸五壮。足太阴脾,又足厥阴肝。

府舍二穴∶(在肠结下三寸。刺入七分,灸五壮。主∶疝瘕,髀中痛,循胁上下,抢心,腹衡门二穴∶(一名慈宫,去大横五寸,在府舍下,横骨两端约中。刺入七分,灸五壮。

足太

侧胁部左右二十穴∶

大包二穴∶(脉出腋下三寸。刺入三分,留五呼,灸三壮。足太阴脾。主∶胸胁痛,身寒,辄筋二穴∶(在腋下三寸,复前行一寸,着胁。刺入六分,灸三壮。足少阳胆,手厥阴心主天池二穴∶(一名天会。在乳复一寸,胁下三寸,着胁直胁撅肋间。刺入七分,灸三壮。

手喉渊腋二穴∶(一名泉腋。在腋下三寸宛宛中,举臂得之。刺入三分,不灸。足少阳胆。

主∶章门二穴∶(脾募也。一名长平,一名胁廓。在大横外,直脐,季肋端,侧卧伸下足,屈上心痛腰京门二穴∶(肾募也。一名气府,一名气输。在监骨腰中季肋本侠脊。注云∶穴当十一椎,洞泄带脉二穴∶(在季肋端下一寸八分,刺入六分,灸五壮。主∶妇人少腹坚痛,月水不通。)维道二穴∶(一名外枢。在章门下五寸三分。刺入八分,灸三壮。主∶呕,咳逆不止,三焦五枢二穴∶(在带脉下三寸,刺入一寸,灸五壮。主∶男子阴疝,两丸上入少腹;妇人下赤居二穴∶(在长平下八寸三分,监骨上陷者中。刺入八分,灸三壮。主∶腰痛引少腹。)

足部左右诸穴百六十九∶

阴廉二穴∶(在羊矢下,去气街二寸。刺入八分。注云∶羊矢亦曰鼠鼷,阴之两廉,腹与股会阴二穴∶(一名屏翳。在大便前,小便后,两阴之间,刺入二寸,留七呼,灸三壮。

任脉阴寒。

会阳二穴∶(一名利夷,在阴尾骨两旁。刺入八分,灸五壮。主∶五肠有寒,泄注,肠,扶承二穴∶(一名肉,一名阴开,一名皮部。在尻臀下股阴下横纹中。注云∶扶承其身,胱。)五里二穴∶(在下去气街三寸,阴股中动脉也。注云∶去腹五寸,故曰之。刺入六分,灸五环铫二穴∶(在髀枢中。刺入一寸,留二十呼,灸十壮。足少阳胆。主∶髀枢中痛,腰胁相殷门二穴∶(在肉下六寸。刺入五分,灸三壮。主∶腰痛得俯不得仰,仰则仆痛。足太阳风市二穴∶(《千金方》云∶令病患起正身平立,垂两臂直下舒十指掩着两髀上,盒饭手中三伏兔二穴∶(在膝上六寸起肉。禁不可灸刺。足阳明胃。无主治。)髀关二穴∶(在膝上伏兔后交分中。刺入六分,灸三壮。同上胃。主∶膝寒痹不仁,痿不得中渎二穴∶(在髀外,膝上五寸分肉间陷者中。刺入五分,灸五壮。足少阳胆。主∶寒气在阴包二穴∶(在膝上四寸,股内廉两筋间,刺入六分,灸三壮。足厥阴肝。主∶腰、少腹痛阴市二穴∶(一名阴鼎,在膝上三寸。刺入三分,留七呼,灸三壮。足阳明胃。主∶寒疝下梁丘二穴∶(在膝上二寸两筋间,刺入三分,灸三壮。同上胃上。主∶大惊,乳痛,胫苕苕箕门二穴∶(在鱼股上越筋间动应手,阴市内。刺入三分,灸三壮。足太阴脾。主∶癃,遗血海二穴∶(在膝膑上内廉白肉际二寸中,刺入五分,灸五壮。足太阴脾。主∶妇人漏下,犊鼻二穴∶(在膝膑下上侠解大筋中,刺入六分,灸三壮。足阳明胃。主∶犊鼻肿,可灸膝目四穴∶(华佗云∶在膝盖下两边宛宛中。主∶膝弱痉疼冷,胫痛矣。《短剧方》云∶膝阳关二穴∶(在阳陵泉上三寸,犊鼻外廉,刺入七分,灸三壮。足厥阴肝。主∶膝外廉痛,委中二穴∶(在中央约文中动脉。刺入五分,灸三壮。足太阳膀胱。主∶腰痛侠脊至头,曲泉二穴∶(在膝内辅骨下,大筋上,小筋下陷者中,屈膝而得之。刺入六分,灸三壮。

足阴谷二穴∶(在膝内辅骨之后,大筋之下,小筋之上,按之应手,屈膝而得之。刺入四分,少阴肾脉。

(阳陵泉二穴∶(在膝下一寸,外廉陷者中,刺入六分,灸三壮。足少阳胆。主∶太息口苦,阴陵泉二穴∶(在膝下内侧辅骨下陷者中,伸足乃得之。刺入五分,灸三壮。足太阴脾脉。

膝关二穴∶(在犊鼻下二寸陷者中,刺入四分,灸五壮。足厥阴肝脏。主∶膝内廉痛引髌,浮二穴∶(在委阳上一寸,展膝得之。刺入五分。足太阳膀胱。主∶不得卧,出汗,不得委阳二穴∶(在足太阳之前,少阳之后,出于中外廉两筋间,扶承下六寸。刺入七分,灸合阳二穴∶(在膝约中央下二寸,刺入六分,灸五壮。同前膀胱。主∶踝厥,癫疾,螈,地机二穴∶(一名脾舍。在别走上一寸一空,在膝下五寸。刺入三分,灸五壮。足太阴脾脉三里二穴∶(在膝下三寸,骨外廉。刺入一寸,留七呼,灸三壮。足阳明胃。主∶腹中寒巨虚上廉二穴∶(在三里下三寸。刺入八分,灸三壮。同前胃。主∶飧泄,大肠痛,胸胁KT条口二穴∶(在下廉上一寸。刺入八分,灸三壮。同上胃。主∶寒胫疼,足缓失履,湿痹,巨虚下廉二穴∶(在上廉下三寸。刺入三分,灸三壮。足阳明胃。主∶少腹痛,飧泄,次指阳辅二穴∶(在足外踝上辅骨前绝骨端,如前三分,去丘虚上七寸,刺入五分,灸三壮。

足丰隆二穴∶(在踝上八寸,下廉下,外廉陷者中。刺入三分,灸三壮。足阳明胃。主∶厥中二穴∶(一名中都。在内踝上七寸胫骨中。刺入三分,灸三壮。足厥阴脉。主∶颓疝,承筋二穴∶(一名肠,一名直肠。在肠中央陷者中。不刺,灸三壮。足太阳膀胱。

主∶仁。)承山二穴∶(一名鱼肠,一名肠山,一名肉柱。在兑肠下分肉间陷者中。刺入七分,灸五三阴交二穴∶(在内踝上八寸,骨下陷者中。刺入三分,留七呼,灸三壮。足太阴脾,又飞扬二穴∶(一名厥阳,在外踝上七寸。刺入三寸,留十呼,灸三壮。足太阳膀胱。主∶下阳交二穴∶(一名别阳,一名足。在踝上七寸,斜属三阳分肉间。刺入六分,灸三壮。

主筑宾二穴∶(在内踝上分中。刺入三分,灸五壮。主∶大疝,绝子,狂,癫疾。)外丘二穴∶(在外踝上七寸,刺入三分,灸三壮。足少阳胆。主∶肤痛,痿痹,胁,头痛,漏谷二穴∶(在内踝上六寸骨下陷者中。刺入三分,灸三壮。主∶腹中热若寒,肠善鸣,膝蠡沟二穴∶(在内踝上五寸,刺入二分,留三呼,灸三壮。足厥阴肝。主∶女子疝,少腹肿光明二穴∶(在外踝上五寸。刺入六分,留十呼,灸五壮。足少阳胆。主∶身解,寒少热交信二穴∶(在内踝上二寸。刺入四分,灸三壮。又阴维脉也。主∶气癃,颓疝,阴急,股悬钟二穴∶(在外踝上三寸动者,刺入六分,灸五壮。足阳胃,又足太阳膀胱,又足少阳胆付阳二穴∶(在外踝上三寸。刺入六分,灸三壮。阳跷脉。主∶痿厥,风头重,痛,四肢复留二穴∶(一名昌阳,一名伏白。在内踝上二寸陷者中。刺入三分,留三呼,灸五壮。

足中对二穴∶(在足内踝前一寸,仰足而取之,陷者中。刺入四分,留七呼,灸三壮。足厥阴昆仑二穴∶(在足外踝后踝骨上陷者中。刺入五分,留十呼,灸三壮。足太阳膀胱脉。

主∶金门二穴∶(在足外踝上,名曰关梁。刺入三分,灸三壮。足太阳膀胱脉。主∶尸厥暴死,照海二穴∶(在足内踝下,刺入四分,留六呼,灸三壮。阴跷脉。主∶卒疝,少腹痛,四肢曲尺二穴∶(《短剧方》云∶在一脚趺上,胫之下,接腕曲屈处,对大指歧当踝前两筋中央申脉二穴∶(在外踝下陷中容爪甲,刺入三分,留六呼,灸三壮。阳跷脉。主∶腰痛不能举商丘二穴∶(在足内踝下微前陷者中,刺入三分,留七呼,灸三壮。足太阴脾脉。主∶疟寒然谷二穴∶(一名龙渊,在足内踝前,起大骨下陷者中。刺入三分,留三呼,灸三壮。

足少黄胆。

太溪二穴∶(在足内踝后跟骨上动脉陷者中。刺入三分灸三壮。足少阴肾脉。主∶闷,呕,水原(泉)二穴∶(去太溪下一寸,在内踝下。刺入四分,灸五壮。足少阴肾脉。主∶月经不仆参二穴∶(一名安邪,在跟骨下陷者中,供足得之。刺入三分,留六呼,灸三壮。阳跷脉丘虚二穴∶(在足外踝下如前陷者中,去临泣三寸。刺入五分,留七呼,灸三壮。足少阳胆解溪二穴∶(在足衡阳后一寸半,腕上陷者中,刺入五分,留五呼,灸三壮。足阳明胃。

主大钟二穴∶(在足踝后街中,有本作踵中。刺入三分,灸三壮。足少阴肾脉。主∶腰脊痛,冲阳二穴∶(一名会原。在足跗上五寸骨间动脉上,去陷谷三寸。刺入三分,灸三一。

足阳太冲二穴∶(在足大指本节后二寸或一寸半陷者中。刺入三分,留十呼,灸三壮。足厥阴肝遗尿。)公孙二穴∶(在足大指本节之后一寸。刺入四分,留二十呼,灸三壮。主∶寒热汗出,不嗜太白二穴∶(在足内侧核骨下陷者中。刺入三分,留七呼,灸三壮。足太阴脾脉。主∶热病隐白二穴∶(在足大指端内侧,去爪甲如韭叶,刺入一分,留三呼,灸三壮。同上,井木。

大都二穴∶(在足大指本节后陷者中,刺入三分,留七呼,灸三壮。主∶热病汗出且出厥,大敦二穴∶(在足大指端,去爪甲如韭叶及三毛中。刺入三分,留十呼,灸三壮。足厥阴肝行间二穴∶(在足大指间动应手陷者中。刺入六分,留十呼,灸三壮。足厥阴肝。主∶尿难,痛,白浊,卒疝,腰腹痛,咳逆,面热,口,咽痛,短气,足下热,胸背痛。)陷谷二穴∶(在足大指次指外间,本节后陷者中,去内庭二寸。刺入五分,灸三壮。足阳明内庭二穴∶(在足大指次指外间陷者中。刺入三分,留二十呼,灸三壮。同前胃。主∶四厥厉兑二穴∶(在足大指次指之端,去爪甲角如韭叶。刺入一分,留一呼,灸一壮。同前胃。

临泣二穴∶(在足小指次指本节后间陷者中,去侠溪一寸半。刺入二分,留五呼,灸三壮。足利。)地五会二穴∶(在足小指次指本节后陷者中。刺入三分,不可灸。足少阳胆。主∶内伤唾血侠溪二穴∶在足小指次指歧骨间,本节前陷者中。刺入三分,留三呼,灸三壮。足少阳胆。

窍阴二穴∶(在足小指次指之端,去爪甲如韭叶。刺入一分,留三呼,灸三壮。主∶胁痛,京骨二穴∶(在足外侧大骨下,赤白肉际陷者中。刺入三分,留七呼,灸三壮。主∶喘,头束骨二穴∶(在足小指外侧本节后陷者中。刺入二分,留三呼,灸三壮。足太阳膀胱脉。

主通骨二穴∶(在足小指外侧本节前陷者中。刺入二分,留五呼,灸三壮。同上。主∶身疼痛至阴二穴∶(在足小指外侧去爪甲角如韭叶。刺入一分,灸三壮。同上。主∶疟,头重,鼻涌泉二穴∶(一名地冲。在足心陷者中,屈足卷指宛宛中。刺入三分,留三壮,灸三壮。

主难。
诸家取背输法第二

杨玄操曰∶黄帝正经椎有二十一节。华佗、扁鹊、曹翕、高KT力之徒,或云二十四椎,或云二十二,或云长人二十四椎,短人二十一椎。此并两失,其衷大致或疑。夫人感天地之精,受五行之性,骨节孔窍一禀无KT,长短粗细乃因成育,是以人长则骨节亦长,人短则骨节亦短,其分段机关无盈缩也。今云长人二十四椎者,其肢节宁即多矣。短人二十椎者,其肢节便少乎?是知骨法常定,肢节无差。时人穿凿,互生异见,宜取轩后正经,勿视杂术之浅法也。然华佗,扁鹊并往代名医,遗文旧迹岂应如此?当是后人传录失其本意也。

又云∶诸输皆两穴,侠脊相去三寸,诸家杂说多有不同。或云∶肺输第五椎,心输第七椎;或云∶相去二寸半;或云∶二寸;或云∶三寸三分;或云∶诸输皆有三穴。此又谬矣。《明堂》者,黄帝之正经,圣人之遗教,所注孔穴靡不指的。又皇甫士安,晋朝高秀,洞明医术,撰次《甲乙》,并取三部为定。如此,则《明堂》、《甲乙》是圣人之秘宝,后世学人宜遵用之,不可苟从异说,致乖杨上善曰∶取背输法,诸家不同者,但人七尺五寸之躯虽小,法于天地,无一经不尽也。

故日取实等所承别本处,所及名亦皆有异。而除遣疾又复不少,正可以智量之适为用不可全,言非也而并为非者,不知大方之论,所以此之量法,圣人设教有异,未足怪也。

《黄帝明堂经》输椎法曰∶大抒在第一椎下傍。风门在第二椎下傍。肺输在第三椎下。心输在第五椎下。膈输在第七椎下。

椎下凡侠脊椎下间傍相去三寸也。

《扁鹊针灸经》曰∶第二椎名大抒(各一寸半,又名风府。)第四椎名关输。第五椎名督脉输三椎名悬十八椎名三治。)第二十二凡十九椎应治其病灸之,诸输侠脊左右各一寸半或一寸二分。但肝输一椎灸其节。其第十三《华佗针灸经法》∶第一椎名大椎。第三椎名云门。第四椎名神输。第五椎名脉输(又云厥第椎名名大少(又云重下第二十三椎名下凡诸椎侠脊相去一寸也。

《龙衔素针经》曰∶热府大椎上去发一寸横三间寸。心输第三椎横相去三寸,一名身枢。

风(正凡人身长短肥瘦,骨节各有大小,故不可以一法取,宜各以其自夫、尺、寸为度。横度手四僧匡及彻公二家与上件四经不同者别出∶风门第三节。心输第七节。膈输第八节。脾输第十二节(又云十六节。)胃输第十一节。

小肠督脉上件,五家背输椎法与《明堂经》不合者别出如上,与其经不异者不在上例。

《背输度量法》曰∶凡人有长短肥瘦,随形量之,不得同量。脏腑十二输欲令详审者,宜以《黄帝素问》曰∶欲知背输,先度其两乳间,中折之,更以他草度去其半已,即以两禺相柱一度之度《黄帝九卷》曰∶若取人节解者,可从大椎骨头直下至尻尾骨端度取,分为二十二分,还约《金腾灸经》曰∶脏腑十二输,经连经,法从大椎直穷骨中折度去其半,乃取余半四折之,输;从肝点名凡脏腑输皆侠脊相去二寸半,唯肝输正脊中央也。
针禁法第三

孙思邈云∶经云∶云门刺不可深。今则都忌不刺,学人宜详悉之。

大寒无刺;月生无泻;月满无补;月廓空无治;新内无刺;已刺无内;大怒无刺;已刺无怒无饥乘车来者,卧而休之如食顷,乃刺之;步行来者,坐而休之如行十里顷乃刺之;大惊大怒,五日死,其动为欠;刺中脾,十五日死,其动为吞;刺中肾,三日死,其动为嚏;刺中胆,一日半死,其动为呕;刺中膈为伤中,不过一岁必死;刺中趺大脉,血出不止死,趺上大脉,动脉也;刺阴股中大脉,血出不止死;刺面中溜脉,不辜为盲;刺客主人内陷中脉,为内漏为聋;刺项中脑户,入脑立死;刺膝膑出液为跛;刺舌下中脉大过,血不止为喑;刺臂太阴脉,出血多立死,手太阴经渠不可出血,血出立死。按∶此臂之太阴脉总不得出血也;刺足下布络中脉,血不出为肿,布络是足少阴脉皮部络也;刺足少阴脉重虚出血,为舌难以言,足少阴至舌本,若其脉先虚又刺出血即为重虚,故为语难也;刺中大脉令人仆、脱色,刺诸当空刺之,不可中于大络也;刺膺中陷中肺,为喘逆仰息,一名中府,肺募也;刺街中脉,血不出为肿鼠鼷。

刺肘中内陷气归之,为不屈伸;刺脊间中髓为伛;刺阴股下阴三寸内陷,令人遗尿,阴股下刺刺神庭禁大不可刺;上关刺不可深;缺盆刺不可深;颅息刺不可多出血。

左角刺不可久留;云门刺不可深;五里禁不可刺。

脐中禁不可刺。

伏兔禁不可刺;三阳络不可刺;复留刺无多见血;承筋禁不可刺;然谷刺无多见血。
灸禁法第四

陈延之云∶《黄帝经禁》曰∶不可灸者有十八处,而《明堂》说便不禁之,今别之记如下∶中有不可天又云∶曹氏说不可灸者如下(陈延之同∶)玉枕者人音声之所经从,无病不可灸,灸则声不能语;若有疾可灸五十壮。维角者在眼后发精赤寸气涩也。

胡脉在颈本边主乳中脉上是也,一名荣听,人五脏血气之注处也,无病不可多多灸,熟则血天突者,名为天瞿,复名身道,是体精之衢路也,无病不可灸,灸则伤与声反喑;有疾可灸壮。此能举臂有疾可血海者名为冲使,在膝内骨上一夫陷中,人阴阳气之所由从也,无病不可灸,灸男则气衰,灸,也,上二十穴,曹氏说云∶无病不可灸,灸则为害也。寻不病者则不应徒然而灸,以痛苦为玩者头病即此为病皆灸之在
针例法第五

《素问》曰∶夫九针者,天地之大数,始于一而终于九。故曰∶一以法天,二以法地,三以又曰∶夫圣人之起天地之数也,一而九之,故以立九野,九而九之,九九八十一,以起黄钟针者,取法布针,去末半寸,卒兑之,长一寸六分,主热在头身也。二曰员针,取法于絮长三寸主痈热(利)针曰毫针,深邪远痹者。九曰大针,取法于锋针,其锋微圆,长四寸,主取大气不出关节者。针形毕此九针小大德贞常曰∶凡刺竟不得即灸,若拔针即灸者,内外热气相击,必变为异病也。若针处有肿核人也用也及风,皆用大员利针如KT也,亦量肥瘦大小之宜。皆烧针过热紫色为佳,深浅量病大小至病为度。针讫以烧钉赤,灸上七过佳也,毋钉灸上七壮而以引之佳也,不则大气伏留以为肉痈也。若肉薄之处不灸,亦得大禁水入也。禁冷冻饮料食。疮不发者,欲不作瘢者,脓时去之。乍寒孙思邈曰∶火针用锋针,以油为烧之,务在猛热,不热即于人有损也。隔日一报,三报之后
灸例法第六

陈延之曰∶经说∶夫病以汤药救其内,针灸营其外。夫针术须师乃行。其灸则凡人便施,为所张仲景述∶夫病其脉大者不宜灸也。

凡欲灸者,当详所宜,审应灸处疏孔穴名,应灸壮数,出之以疏临图像,依注明寸数量度灸凡灸之腥说熟,宜视其人盛衰所在,大熟则伤衰腥,少则不能愈疾。是以宜节度随盛衰也。

凡男女之体,同以腰上为阳,以腰下为阴也。男以背为阴,腹为阳。女以腹为阴,背为阳。

凡灸法当先发于上,然后灸下。先发于阳,然后灸阴,则为顺也。

凡灸诸俞皆令如经也,不如经者徒病无益。灸得脓坏,风寒乃出,不坏病则不除也。

凡肾气有风冷,令人如邪鬼状,但数报灸令熟,风寒除自愈。

凡头者人神所治气之精也,病则气虚精散。夫灸头必令当病使火气足,却邪则止火也。

足而不止,则神出不得入,伤精明营卫衰损也;未足而止,则邪微有余,喜因天阴阳而发也。

四肢者,身之枝KT也,其气系于五脏随血脉出入养四肢也,其分度浅易达也。是以灸头精神气微,得火则冷气散。且背膂重浓,灸宜熟,务多善也。

愈皆令如经也,不如经者徒痛无益。灸得脓坏风寒乃出,不坏病则不除也。凡肾气有风冷,《太素经》云∶手中指本节至其末,长四寸半。注云∶从本节端至中指未合四寸半,今人取孙思邈曰∶凡孔穴在身此皆脏腑营卫血脉流通,表里往来各有所生,临时救难,必在审详,寸之取手又以肌肉纹理节解缝会宛臼之中,又以手按之,病者快然,如此仔细安详用心者,乃能得之凡经言横三间寸者,则是三灸而间。间一寸有三灸,灸有三分,三壮之处即为一寸。凡言数有至百壮直,之,凡灸当先阳后阴,先上后下,皆以日正午后,乃可下火。午前,平旦谷气令人癫眩,不可针杨玄操曰∶灸疮得脓坏,其病乃出;不坏则病不除。《甲乙·丙卷》云∶灸不发者灸KT熨之《苏敬香港脚论》云∶灸疮瘥后,KT色赤者风毒尽;青黑者犹有毒瓦斯,仍灸勿止,待身体轻利《扁鹊针灸经》云∶凡灸因火生疮,长润,久久不瘥。变成火疽,取桃树东边皮一寸以上煮
针灸服药吉凶日第七

合服药吉曰∶《大清经》云∶凡欲合服神仙药者,以天清无风雨,欲得王相日,上下相生日合之,神良。

王相日者∶春甲乙寅卯王,丙丁巳午相;夏丙丁巳午王,戊己辰戌丑未相;四季戊己辰戌丑相生日者∶春甲午、乙巳、丙寅、丁卯;夏丙辰、丁丑、丙戌、丁未;四季戊辰、己丑、戊又云∶凡作药,始以甲子、开、除之日为之,甲申、己卯次之。

又云∶服药良日常以建,开日晨服为阳,暮服为阴,多其阳少其阴。

《虾蟆经》云∶凡服药吉曰∶甲辰、乙巳、丙辰、丁巳。(今按∶丙辰五不生也。甲辰此天又云∶服药吉时∶甲乙日∶(鸡鸣,日入、维时、晡时。)丙丁日吉时∶(晡时、日入、人定定、今按∶凡甲子、丙子、戊子、壬子、甲午、丙午、庚午、壬午、甲戌、丙戌、壬戌、乙巳、今检件曰∶避诸禁合药服药针灸治病皆吉。但可避节气,月忌并生年衰日等。

合服药忌日∶《大清经》云∶正月亥、二月寅、三月巳、四月亥(一曰申)、五月亥、六月寅、七月巳、八上日,常不可和长生药。

又云∶六绝曰∶正月辰、二月卯、三月寅、四月丑、五月子、六月亥、七月戌、八月酉、九月申、十月未、十一月午、十二月巳。

上日,不可服药治病。

又云∶月建、月杀、反支、天季、上朔、自刑日,此不可用。自刑日者,如寅生人不得用寅《湛余经》云∶天季日,正月子、二月卯、三月午、四月酉、五月子、六月卯、七月午、八上日,不可用。今按∶《耆婆方》云∶天狱日也。《大清经》云∶不得和药、服药。

又云∶凡除日可服药治病,满日不可服药,病患难起。

张仲景《药辨决》云∶凡春戊辰、己巳、戊午,夏丁亥、戊甲、乙酉(今按∶《虾蟆经》作辛上日,《虾蟆经》云∶皇帝禁合药日四时忌。今古传讳,不合药服药也。

又云∶天有五不生日,不可合药、服药。乙丑、丁卯、己巳、癸未、乙酉、庚戌、戊申、丁上十二日,扁鹊不治病,大凶。(今按∶《虾蟆经》同之,但《大清经》加甲寅、甲午并十《虾蟆经》云∶四激曰∶春戌、夏丑、秋辰、冬卯。

上四时忌日,古今传讳,不合药服药也。今按∶《开元天一循甲经》曰∶此为四极所破,故又云∶凡夏至,冬至日天地阴阳前后各七日,皆不可服药灸刺。

又云∶凡五月辛,己日不可针,针灸服药,出血致死。

又云∶凡反支日∶子丑在朔六日反支,寅卯在朔五日反支,辰巳在朔四日反支,午未在朔三上日,不可灸刺、服药,大凶。

又云∶凡天医以辛巳日死,扁鹊以癸未日死(一本作辛未,)师旷以辛卯日死(一本作辛未,)上日,不可服药、治病,凶。

又云∶凡大圣服药,治病,避五未寅申,此三日大凶。

又云∶凡五寅六戊辛未,上三日,不可合药、服药。

又云∶凡甲寅、乙卯、庚辰、丙寅、辛巳,上五日,不可灸刺、服药,凶,三年死。

又云∶凡己巳、丁亥、壬辰、庚戌,上五日,不可灸刺,服药,凶。

又云∶凡日出,日中时(一云日入,)上二时,不可服药、治病,大凶。

《养生要集》云∶四激、破、除、未日时,不中合药、服药。(今按∶《大清经》同之。)《真人集辨方》∶春忌戊辰、己巳,夏忌戊申、己未,秋忌戊戌、己亥,冬忌戊寅、己卯。

又云∶和合神仙药法,常避四孟,正月寅,二月巳,三月申,四月未(亥),五月寅。

以右行不可和合长生药也。当以其时王相作之,服药用意。

《服石论》云∶凡服药之本,必须命其病者正意深信,不得于中持疑更怀他念。但想其药入又云∶凡服药,先首于吉日清旦具服严,净嗽其口,面向东立再拜,一心发愿,愿服神药以后千殃散灭,百病消除,志求长生,无违其愿,愿一切大圣如护去老还年。发此愿已,又服药颂∶《新罗法师方》云∶儿服药咒曰∶南无东方药师琉光佛,药王药上菩萨耆婆,医王灵山童子惠施阿竭,以疗病者,邪气消除川,针灸忌日∶《华佗法》云∶凡诸月朔晦、节气、上下弦望日、血忌、反支日皆不可针灸,治久病滞疾记又云∶冬至、夏至、岁旦,此前三日、后二日,皆不可针灸及房室,杀人,大忌。

又云∶立春、春分、立夏、夏至、立秋、秋分、立冬、冬至。

上日,忌不可针灸治病也。

又云∶男忌壬申、戊戌、丁未。女忌甲申、乙酉,又甲辰、壬辰。忌服药刺灸,此天地四时《虾蟆经》云∶凡天阴雾、疾风豪雨、雷鸣、地动、四时月节前后三日、晦朔日、月薄蚀无阳分日月夏王王庚辛日,无治肺募,俞及手太阴。冬肾王壬癸日,无治肾募,前(俞)及足少阴。

上四时五脏王日,禁之无治。

又云∶建日不治两足(禁晡时。)除日不治阴孔(一本尻,禁日入时。)满日不治腹(禁黄昏时鸣时。)破禁食时。)收日又云∶凡除日可以服药治病,满日不可服药,病患难起。

又云∶凡甲子旬乙丑、丁卯、己巳,上三日,不用治病,凶。

甲戊旬癸未,上一日,不用治病,凶。

甲申旬乙酉、丁亥、庚辰,上三日,不用治病。凶。

甲午旬甲午、庚子,上二日,不用治病。凶。

甲辰旬戊甲、庚戌,上二日,不用治病,凶。

甲寅旬甲寅、丙辰、戊午,上三日,不用治病,凶。

又云∶凡甲不治头,乙不治眉(一云颈头,)丙不治心(一云肩,)丁不治胸,戊不治胁,己不治足。

又云∶凡月一日、五日、六日、七日、八日、十五日、十六日、十八日、二十三日、又云∶凡血忌日∶正月丑,二月未,三月寅,四月申,五月卯,六月酉,七月辰,八月戌,上十二日,是血忌也,一名杀忌,一名禁忌,其日不可灸刺,见血凶。

又云∶凡四绝日∶戊申、戊寅、癸亥、癸巳,上日,不可针灸。

又云∶凡五离日∶戊申、巳酉(天地离),壬申、癸酉(鬼神离,)甲申、乙酉(人氏离,)丙申今按∶针灸服药通忌五不生日等在服药禁忌条中。

针灸吉日∶《虾蟆经》云∶凡阳日可治男,阴日可治女。甲丙戊庚壬皆阳日也,乙丁己辛癸皆阴日也。

又云∶六甲日可治男病,六己(乙)日可治女病。
人神所在法第八

年神《虾蟆经》云∶黄帝问于歧伯曰∶人有九部,何谓也?歧伯曰∶九部者,神所藏行有神宫部终而复始。故必慎神之所在,前后不可灸刺。当其年神而发伤及兵创病者,致死也。

年一、十、十九、二十八、三十七、四十六、五十五、六十四、七十三、八十二、九十一、百上年神在神宫部,一名气鱼,在脐下四寸,当于中极。

年二、十一、二十、二十九、三十八、四十七、五十六、六十五、七十四、八十三、九十二、上年神在大敦部,一名五户,当天突,在颈结喉下五寸中央宛宛中。

年三、十二、二十一、三十、三十九、四十八、五十七、六十六、七十五、八十四、九十三、百上神在巨部,当于巨骨,在肩隅上两骨陷中。

年四、十三、二十二、三十一、四十、四十九、五十八、六十七、七十六、八十五、九十四、百三。

上神在领部,一名本地,当于廉泉,在领下结喉上。

年五、十四、二十三、三十二、四十一、五十、五十九、六十八、七十七、八十六、九十五、上神在下承部,一名承浆,在唇下交脉中。

年六、十五、二十四、三十三、四十二、五十一、六十、六十九、七十八、八十七、九十六、上神在天部,当于额上下行,在神庭。

年七、十六、二十五、三十四、四十三、五十二、六十一、七十、七十九、八十八、九十七、上神在阙庭部,当于伏兔上。

年八、十七、二十六、三十五、四十四、五十三、六十二、七十一、八十、八十九、九十八、百上神在胫部,当于膝下三里。

年九、十八、二十七、三十六、四十五、五十四、六十三、七十二、八十一、九十、九十九、上神在地部,当于太冲。

日神《范汪方》云∶凡月生一日,虾蟆生喙,人气在足少阴。(《虾蟆经》云∶虾蟆生头喙,人二日,虾蟆生左股,人气在股里。(《虾蟆经》云∶虾蟆生左肩,人气在内踝后。)三日,虾蟆生右股,人气在足踝后。(《虾蟆经》云∶虾蟆生右肩,人气在股里。)四日,虾蟆生左胁,人气在肾中。(《虾蟆经》云∶虾蟆生左胁,人气在肾输。)五日,虾蟆生右胁,人气在舌本。(《虾蟆经》云∶虾蟆生右胁,人气在承浆舌本。)六日,虾蟆生后左股,人气在足太阴。(《虾蟆经》云∶虾蟆生左股,人气在太冲。)七日,虾蟆生后右股,人气在口中。(《虾蟆经》云∶虾蟆生右股,人气在足内踝厥阴交。)八日,虾蟆生大形,人气在腰中。(《虾蟆经》云∶虾蟆生尻身,人气在鱼际。)九日,兔生头,人气在尻上。(《虾蟆经》云∶兔生头,人气在足趺交脉。)十日,兔生左股,人气在肩中。(《虾蟆经》云∶兔生左肩,人气在趺上五寸、腰、目。)十一日,兔生右股,人气在鼻上。(《虾蟆经》云∶兔生右肩,人气在鼻柱。)十二日,兔生左胁,人气在发际。(《虾蟆经》云∶兔生左胁,人气在人迎。)十三日,兔生右胁,人气在股本。(《虾蟆经》云∶兔生右胁,人气在颈,当两乳间。)十四日,兔生后左股,人气在人迎。(《虾蟆经》云∶兔生左股,人气在阳陵泉、胃脘。)十五日,兔生身,人气在胃脘。(《虾蟆经》云∶兔生尻身,人气在巨虚上、下廉。)月殿十六日,虾蟆始省头,人气在胸中。(《虾蟆经》云∶虾蟆省头,人气在目。)十七日,虾蟆省左股,人气在太冲。(《虾蟆经》云∶虾蟆省肩,人气在脊膂。)十八日,虾蟆省右股,人气在右胁里。(《虾蟆经》云∶虾蟆省右肩,人气在肾募下至股。)十九日,虾蟆省后左股,人气在四肢脉。(《虾蟆经》云∶虾蟆省左胁,人气在痿阳。)二十日,虾蟆省后右股,人气在巨阙下。(《虾蟆经》云∶虾蟆省右胁,人气在外踝后京骨二十一日,虾蟆省左胁,人气在足小指次指。(《虾蟆经》云∶虾蟆省左股,人气在目外二十二日,虾蟆省右胁,人气在足外踝上。(《虾蟆经》云∶虾蟆省右股,人气在缺盆、腋二十三日,虾蟆省身成,人气在足外踝。(《虾蟆经》云∶虾蟆省尻身,人气在髀厌。)二十四日,兔省左股,人气在腰胁。(《虾蟆经》云∶兔省头,人气在脚外踝。)二十五日,兔省右股,人气在儿骨。(《虾蟆经》云∶兔省左肩,人气在太阴绝骨。)二十六日,兔省左胁,人气在胸中。(《虾蟆经》云∶兔省右肩,人气在大丛毛。)二十七日,兔省右胁,人气在膈中。(《虾蟆经》云∶兔省左胁,人气在内踝上。)二十八日,兔省后左股,人气在阴中。(《虾蟆经》云∶兔省右胁在脚内廉。)二十九日,兔省后右股,人气在内荣。(《虾蟆经》云∶兔省左股,人气在鼠仆、环阴、气三十日,兔省身尽,人气在踝上。(《虾蟆经》云∶兔省右股身形,人气在关元。)上三十日人气所在,不可灸刺。

《华佗法》云∶凡人月一日神在足(《虾蟆经》云在两足下。)二日神在踝(《虾蟆经》云在外踝后。)三日神在股(《虾蟆经》云在腹里。)四日神在腰中(《虾蟆经》云同之。)五日神在口齿、膺、舌本(《虾蟆经》云同之。)六日神在两足小指小阳(《虾蟆经》云同之。)七日神在踝上(《虾蟆经》云在足内踝。)八日神在手腕中(《虾蟆经》云同之。)九日神在尻尾(《虾蟆经》云同之。)十日神在腰目(《虾蟆经》云同之。)十一日神在鼻柱(《虾蟆经》)云同之。)十二日神在发际(《虾蟆经》同之。八素注云∶发际在上星下一寸毛内之。)十三日神周匝蟆经之。

指。

在肝。)二十二十六日神在日神在阴中(《虾(《虾蟆经》云在两足。

上三十日神所在,不可灸刺。

《虾蟆经》云∶六甲日神游舍避灸刺法∶甲子头上正中,乙丑头上左太阳,丙寅头上左角,丁卯左耳,戊辰左曲颊,己巳左颊,庚午右角壬午右髀,己丑右膝,庚寅右膝下五寸,辛卯右踝上三寸,壬辰右足中指本节,癸巳右足心,甲午左乳,乙未左肘里,丙申左季肋,丁酉左髀上,戊戌左髀,己亥左膝,庚子左膝下五寸,辛丑踝上三寸,壬寅左足中指本节,癸卯左足心,甲辰踝上,乙巳左KT下三寸,丙午左脚中,丁未左股阴中,戊申阴中,已酉右股阴中,庚戌右脚中,辛亥右KT肠,壬子右KT下五寸,癸丑右足踝上,甲寅膺中,乙卯直两乳间,丙辰心鸠尾下,丁巳胃脘,戊午胃脘左,已未胃脘右,庚申右气街,辛酉左气街,壬戌左股阴中太阴,癸亥右股阴中太阴。

上六十日神所在之处,宜避针灸,不避致害。

又云∶凡子日神在目,丑日在耳,寅日在口,卯日在鼻,辰日在腰,巳日在舌,午日在心,上十二日神所在,不可灸刺。

时神《虾蟆经》云∶凡鸡鸣神舍头(丑),平旦舍目(寅),日出舍耳面(卯),食时舍口(辰),禺中黄昏舍阴(上十二时神所舍处,慎禁之。

又云∶凡平旦至食时(魂在中府,魂在目,神在膀胱,志在天窗,意在人中。)禺中(魂在气阴迎,魄尺泽,神在上五神所处,不可刺灸,禁之。
天医扁鹊天德所在法第九

天医扁鹊法《虾蟆经》云∶正月(天医在卯,扁鹊在酉。)二月(天医在戌,扁鹊在辰。)三月(天医在巳医在寅,扁医在亥,扁鹊月(天医在申,扁又云∶凡推日天医法∶甲己在卯,乙庚在亥,丙辛在酉,壬丁在未,戊癸在巳。

又云∶推月天医法∶正月在酉,二月在亥,三月在午,四月在未,五月在申,六月在卯,七月在戌,八月在丑,又云∶推行年天医法∶行年在子(天医在卯,)行年在丑(天医在辰,)行年在寅(天医在巳,)行年在卯(天医在子,)(天医在亥,)又云∶凡天医阳月以大吉,加月建功曹下为天医传送下为扁鹊。阴月以小吉,加月建功曹又云∶天医常以神后加今日时,功曹下为天医传送下为天巫。从魁下为天师,以神后加大岁又云∶凡病患不瘥,当从天医治之,不避众忌,所治之处百鬼不敢当天医所在。虽有凶神不又云∶凡病者,向生气坐。治其人,背天医坐而治也。火置扁鹊上,作艾,人背天医坐也,又云∶推月生气法∶正月在子(死气在午,)二月在丑(死气未,)三月在寅(死气申,)四月在卯(死气酉,)五月在在申(死气上,向生气所在可服药,莫向死气。

又云∶推天德法∶《虾蟆经》云∶正月(在丁,)二月(在西南角,)三月(在壬,)四月(在申,)五月(在西北角十一月(在东
月杀厄月KT日法第十

月杀所在法∶《耆婆方》云∶正月杀鬼在丑不向东,治病者死。二月杀鬼在戌不向北。三月杀鬼在戌不向不向未不上月杀在之处,勿向治病,病患死。

《虾蟆经》云∶正、五、九月东北向治病,病者死。二、六、十月西北向治病,病者死。

三厄月法∶《虾蟆经》云∶凡子年生人,大厄在未,小厄在丑,KT六月、十二月。

丑年生人,大厄在午,小厄在子,KT五月、十一月。

寅年生人,大厄在巳,小厄在亥,KT四月、十月。

卯年生人,大厄在辰,小厄在戌,KT三月、九月。

辰年生人,大厄在卯,小厄在酉,KT二月、八月。

巳年生人,大厄在寅,小厄在申,KT正月、七月。

午年生人,大厄在丑,小厄在未,KT六月、十二月。

未年生人,大厄在子,小厄在午,KT五月、十一月。

申年生人,大厄在巳,小厄在亥,KT四月、十月。

酉年生人,大厄在辰,小厄在戌,KT三月、九月。

戊年生人,大厄在酉,小厄在卯,KT二月、八月。

亥年生人,大厄在申,小厄在寅,KT正月、七月。

上,黄帝曰∶以此大小厄日月及大小厄方地向以厄日不可灸刺,灸刺则死。又以此日服药大八卦法(出《考命书》)KT离。(年一、八、十六、二十四、三十二、四十、四十一、四十八、五十六、六十四、七十游年立离,祸害艮,绝命干,鬼吏坎,墓在亥,生气震,养者坤,天医兑,绝体坎,游魂在坤,祸德巽,五鬼艮。小KT正月、五月、十二月(忌五日、十二、二十八日。)大厄十月(忌二日、九日、十七日、二十五日,不可北行。)一说厄正月、三月、十月、十二月(忌二日、十KT坤。(年二、九、十七、二十五、三十三、四十二、四十九、五十七、六十五、七十三、八游年坤,祸害震,绝命在坎,鬼吏震,墓在辰,生气艮,养者离,五鬼震,绝体干,游魂离,天医巽,祸德兑。小KT六月、十二月(忌十三日、二十九日。)大厄二月七月(忌三日、八日、十日,不可北行。)一说月厄二月、八月(忌十五日、二十四日,不可北行。)KT日时卯、KT兑卦。(年三、十、十八、二十六、三十四、四十三、五十、五十八、六十六、七十四、八游年兑,祸害坎,绝命震,鬼吏午,墓在丑,生气干,养者艮,五鬼午,绝体艮,游魂巽,祸德坤。小KT七月(忌十日,十四日、二十三日。)大厄正月、五月、十一月(忌一日、五日、不可东行。)一说厄月,十一月、五月(忌十六日,二十六日,不可东行。)KT日时子午。

KT干。(年四、十一、十九、二十七、三十五、四十四、五十一、五十九、六十七、七十五、游年干,祸害巽,绝命离,鬼吏离,墓在丑,生气兑,祸德艮;养者坎,五鬼巽,绝体坤,游魂坎,天医震。小KT五月(忌十五日、二十二日。)大厄二月、三月、四月、九月(忌六日、十二日、十四日、十九日,不可南行。)一说月厄正月、三月(忌二日、十二日,不可南行。

坎。(年五、十二、二十、二十八、三十六、四十五、五十二、六十、六十八、七十六、八游年坎,祸害兑,绝命坤,鬼吏坤,墓在辰,生气巽,养者干,五鬼兑,绝体离,游魂干,福德震,天医艮。小KT正月、六月、七月(忌十六日、二十日。)大厄三月、十月、十二月(忌七日、十日、二十日。不可南行及起土。或本云∶不可西行起土。)一说月厄八月,二月(忌KT艮。(六、十三、二十一、二十九、三十七、四十六、五十三、六十一、六十九、七十七、游年艮,祸害午,绝命巽,鬼吏午,墓在辰,生气坤,养者兑,五鬼午;绝体兑,游魂震,天医,福德干。小KT三月、四月、九月、十月(忌二日、九日、二十五日。)大厄四月、十二月(忌五日、二十三日,不可南行。)一说月厄五月、十一月(忌十五日、二十三日,不可南行KT震。(七、十四、二十二、三十、三十八、四十七、五十四、六十二、七十、七十八、八十七游年震,祸害坤,绝命兑,鬼吏干,墓在未;生气午,养者坎,五鬼坤,天医干,绝体巽,八月、十六日、KT巽。(年十五、二十三、三十一、三十九、五十五、六十三、七十一、七十九、九十五、百三游年巽,祸害干,绝命艮,鬼吏兑,墓在丑,生气坎,养者离;五鬼干,天医坤,绝体震,游魂兑,祸德离。小KT四月、十一月忌四日、十一日、十七日、不可东北行。大厄二月、六月、九月忌(忌十日、二十五日,不可东北行。)一说月厄十月、十二月(忌六日、十七日、
作艾用火(法)灸治颂第十一

作艾法∶《短剧方》云∶黄帝曰∶灸不三分,是谓徒哑。解曰∶此为作炷欲令根下广三分为适也,减还极《千金方》云∶凡新生儿七日以上,周年以还,不过七壮,炷如雀矢大。

用火法∶《虾蟆经》云∶松木之火以灸即根难愈。柏木之火以灸即多汁。竹木之火以灸即伤筋,多壮枯。

之火上八木之火以灸人,皆伤血肌肉骨髓。大上阳燧之火以为灸,上次以KT石之火,大常槐木之《短剧方》云∶凡八木之火害人肌血筋脉骨髓,不可以灸也。大上用阳燧之火,其次KT石之火也。今世之皆良也。相传是可用也。

灸治颂∶《虾蟆经》云∶灸时咒曰∶天师天医,愿我守来疗治百病,我当针灸疾病,不治神明,恶毒又咒云∶天地开张,禁之越王,俱摄金刚,针不当神,利不伤损,疾病速去,急急如律令。
明堂图第十二

《千金方》云∶夫病源所起,本于脏腑。脏腑之脉,并手足,循环腹背无所不至,往来出没其外其源阙,不足为图。

五色又咒云∶赫同赫同,日出于东,左王后、西王母、前朱雀、后玄武,厝鼓织女。使我灸汝,《耆婆脉决经》云∶壬午、辛卯、庚戌、辛酉、壬寅、乙卯。

上六日允病患代死,善善忌之。

凡不问见病者曰∶正月(巳午),二月(午未),三月(戌亥),四月(戌寅),五月(亥子),六月(丑寅),七月(丑凡戊日不见病患,巳日不问病者。

天狗下食日∶子岁(丁丑),丑岁(庚寅),寅岁(丁卯),卯岁(壬辰),辰岁(丁巳),巳岁(丙申),午岁(丁上不可看病及合药作服也。

凡甲乙日(平旦),丙丁日(食禺中),戊己日(日中、日),庚辛日(晡时),壬癸日(黄昏、上日时不可诣看病者。

医心方卷第二背记∶《玉篇》云∶名聘反,也。《宋韵》云∶目或单作名,弥正反。

丁浪也,言中也。《宋韵》如《玉篇》。

宇治本在里,而此脚本书面仍留之。

今按∶甲子丙子戊子壬子丙午庚午壬午甲戌丙戌壬戌乙巳丁巳乙亥辛亥丁壬己壬辛壬癸壬癸壬癸卯今捡件日,避诸禁合药服针灸治病皆吉。但可避节气用忌并生年KT日等。
卷第三
风病证候第一

《黄帝太素经》云∶风者,百病之长也。至其变化为他病也,无常方。(杨上善云∶百病因寓经,长养万物。若风从南方来向中宫,为冲后来虚风,贼伤人者也。)《素问经》∶千病万病,无病非风。

《医门方》云∶凡人性禀五行,因风气而生长。风气虽能长物,还能为害伤人。如水浮舟,亦能覆舟。

《病源论》云∶中风者,风气中于人也。风是四时之气,分布八方,主长养万物。从其乡来不得心中风,但得偃卧,不得倾侧,汗出。若肤赤汗流者可治,急灸心俞百壮。若唇或青或白,肝中风,但踞坐,不得低头,若绕两目连额上色微有青,唇色青,面黄可治,急灸肝俞百壮。若大青黑,面一黄一白者,是肝已伤,不可复治,数日而死。

脾中风,踞而腹满,身通黄,吐咸汁出者可治,急灸脾俞百壮。若手足青者,不可复治也。

肾中风,踞而腰痛,视胁左右,未有黄色如麦饼粢大者可治,急灸肾俞百壮。若齿黄赤,鬓肺中风,偃卧而胸满,短气,冒闷汗出;视目下鼻上下两边下行至口,色白可治,急灸肺俞此数日而死。

《短剧方》云∶说曰∶风者,四时五行之气也,分布八方,顺十二月,终三百六十日。

各以时从其乡来为正风,在天地为五行,在人为五脏之气也。万物生成之所顺,非毒厉之气也。人当触之过,不胜其气乃病之耳。虽病,然有自瘥者也,加治则易愈。其风非时至者,则为毒风也。不治则不能自瘥焉。今则列其证如下∶春甲乙木,东方清风。伤之者为肝风,入头颈肝俞中。为病多汗,恶风,喜怒。两胁痛恶血夏丙丁火,南方汤风。伤之者为心风,入胸胁腑脏心俞中。为病多汗恶风,憔悴喜悲,颜色仲夏戊己土,同南方阳风。伤之者为脾风,入背脊脾俞中。为病多汗恶风,肌肉痛,身体怠在秋庚辛金,西方凉风。伤之者为肺风,入肩背肺俞中。为病多汗恶风,寒热咳动肩背,颜色冬壬癸水,北方寒风。伤之者为肾风,入腰股四肢肾俞中。为病多汗恶风,腰脊骨肩背颈项颜色上四时正气之风,平人当触之,过得病,证候如此。

又云∶四时风物名∶春,九十日,清风;夏,九十日,汤风;秋,九十日,凉风;冬,九十日,寒风。其气分布八方,亦各异名也。太一之神随节居其乡各四十五日,风云皆应之。

今东北方∶艮之气,立春王,为条风,一名凶风,王四十五日;东方∶震之气,春分王,为明庶风,一名婴儿风,王四十五日;东南方∶巽之气,立夏王,为清明风,一名弱风,王四十五日;南方∶离之气,夏至王,为景风,一名大弱风,王二十七日,合仲夏也;仲夏中央之气,主立八方之气,戊己王十八日,合夏至合四十五日,风名同;西南方∶坤之气,立秋王,为凉风,一名谋风,王四十五日;西方∶兑之气,秋分王,为闾阖风,一名刚风,王四十五日;西北方∶干之气,立冬王,为不周之风,一名折风,王四十五日;北方∶坎之气,冬至王,为广莫风,一名大刚风,王四十五日。

上八方之风,各从其乡来,主长养万物,民众少死病也。

又云∶八方风不从其乡来,而从冲后来者,为虚邪。贼害万物,则民众多死病也。故圣人说凶风之气,内舍大肠。外在胁肌骨下,四肢节解中。书本遗其病证,今无也。

婴儿风为病,令人筋纽湿。其气内舍肝中,外在筋中。

弱风为病,令人体重。其气内舍胃中,外在肉中。

大弱风为病,令人发热。其气内舍心中,外在脉中。

谋风为病,令人弱,四肢缓弱也。其气内舍脾中,外在肌中。

刚风为病,令人燥,燥者枯燥瘦瘠也。其气内舍肺中,外在皮中。

折风为病,则因人脉绝时而泄利,脉闭时则结不通,喜暴死也。其气内舍小肠中,外在右手大刚风为病,令人寒,寒者患冷不能自温也。其气内舍肾中,外在骨中脊膂筋中也。

上八风从其冲后来者,为病如此。

又云∶新食竟取风,为冒风。其状恶风,颈多汗,膈下塞不通;食饮不下,胀满形瘦,腹大因醉取风,为漏风。其状恶风多汗,少气,口干渴,近衣则身热如火烧,临食则汗流如雨,新沐浴竟取风,为首风。其状恶风,面多汗,头痛。

新房室竟取风,为泄风。其状恶风,汗流沾衣。

劳风之为病,喜在肺。使人强上恶风,寒战,目脱涕唾出,候之三日中及五日中,不精明者《又云∶风者,其气喜行而数变。在人肌肤中内不得泄,外不得散,因人动静乃变其性,其证如下∶有风遇寒,则食不下;遇热,则肌肉消,寒热。

有风遇阳盛,则不得汗;遇阴盛,则汗自出。

肥人有风,肌肉浓则难泄,喜为热中,目黄。瘦人有风,肌肉薄则恒外行,身中寒,目泪出有风遇实,则腠理闭,则内伏,令人热闷。若因热食,汗欲通,腠理得开,其风自出。

则觉有风遇虚,腠理开则外出,凄然如寒状,觉身中有如水淋时,如竹管吹处。

《录验方》云∶风者,天地山川之气也。所发近远有二焉∶其一是天地八方四时五行之气,为远风也。其风KTKT鼓振者,此则山川间气为近风耳。譬由鼓肩动于手握之间便能致风又云∶经言诸取风者,非是时行永节之风,亦非山川鼓振之风也。此人间庭巷门户窗牖之迳人五亦古雒阳市有一上贴家,最要货卖倍集,但货主周年中必得病致死,遂成空废,无复坐者。

有主儿复今有一人家作北向听事,在南架下。主人四月中温病,逐凉开辟正首,卧乃诊脉,脉作住寸之后也永不肯还斋中避风,而脉冷。汤以除温,遂喑绝而死矣。是以明知迳气之风,不可久当也。

有一家作三间屋,开中央一间,南北对作都户。安一床当中央,夫妇便坐其中监看事。

经一
治一切风病方第二

《耆婆方》治一切风病日月散方∶秦胶(八分)独活(八分)二味,切,捣筛为散,以酒服一方寸匕,日二。还遂四时之四季作服之,春散、夏汤、秋丸又云∶治男女老小一切风病。病风之状,头重痛,眼暗,四肢沉重,不奉不随,头闷心闷烦色,房事转弱,渐自瘦,不能劳动,劳动万病即发,万病并主之,方∶人参白藓防风防己芎秦胶独活(老小各一两,小壮二两。)上七味,切,以水一斗二升,煮取二升,分六服。一方以水六升,煮取一升半,分为三服,服之相去十里。分六服者,相去三十里。令了勘,无相恶,宜久服之,延年益智聪慧。

汤服讫,散服方寸匕,酒服。酒三斗,渍之一宿,少少饮之;煎服,少少服之;丸服,蜜和为丸,丸如大气,昼瘥夜剧;八曰入肝,头眩,目视不明;九曰入脾,令人肠鸣,舌上疮,两胁下芎(四分)蜀椒(三分,一方无)贝母(三分)防风(二分,一方九分)当归(二分,一一方三分)凡八物,冶下筛,丸以蜜,如弹丸。顿服一丸。先食。禁食生鱼、猪肉、生菜。

服药十三日服三丸,《短剧方》小续命汤治卒中风欲死,身体绞急,口目不正,舌强不能语,奄奄惚惚,精神闷甘草(一两)麻黄(一两)防风(一两半)防己(一两)人参(一两)黄芩(一两)桂心(两)上十一物,以水九升,煮取三升,分三服。甚良。不瘥,更合三四剂,必佳。取汗,随人风(于《极要方》云∶疗风病多途,有失音不得语,精神如醉,人手足俱不得运用者。有能言语,不异常,醒后久方候,以劳役。既极于事,能不败乎?常量己所归,而舍割之,静思息事,兼助以药物,亦有可复之理汤服生葛根(一挺,长一尺,径三寸)生姜汁(二大合)竹沥(二大升,如不可得,宜筋竹根大叶细切一以水一上,先取生葛根,净洗刷,便捣碎且空,迮取汁令尽,尽讫。又捣即竹沥,酒KT迮取汁,汁尽为度。用和生姜汁,绵滤之。细细暖服之,不限回数及食前食后,如觉腹内转作声又似痛。,即以食后温服之,如经七日以后,服附子等汤之。

《杂酒方》治一切风病独活酒方∶独活(五两)黑大豆(三升,熬令无音。)凡二物,以酒一斗渍之,五日始服,日三,多少任意。但大豆者渍之二日出去。

《录验方》云帝释六时服诃黎勒丸方∶上诃黎勒者,具五种,味辛酸苦咸甘,服无忌。治一切病,大消食,益寿补益,令人有威德痔;痢不思食疗声破无涌;健忘诃黎勒皮(八分)槟榔仁(八分)人参(三分)橘皮(六分)茯苓(四分)芒硝(四分)狗牛子(十三两)桂心(凡十三味,咀,下筛,以蜜丸如梧子,服二十丸,食前以温酒若薄粥汁服,平旦得下利良。
治偏风方第三

《病源论》云∶偏风者,风邪偏客于身一边也。人体有偏虚者,风邪乘虚而伤之,故为偏风《千手经》曰∶一边偏风,耳鼻不通,手脚不随者。取胡麻油,煎青木香,咒三七遍,摩拭《龙门方》治卒偏风方∶以草火灸,令遍身汗流,立瘥。

又方∶大麻子捣,以酒和,绞取汁,温服。熬蒸亦佳。

又方∶黑胡麻捣末,酒渍,服七日后瘥验。

《极要方》疗偏风服之三日内能起方∶羌活(三两)桂心(二两)干姜(二两)附子(二颗,炮)上,以水一升半,煮取半大升,分二服。
治半风方第四

《病源论》云∶风半身不遂者,脾胃气弱,血气偏虚,为风邪所乘故也。

《千金方》治大风半身不遂方∶蒸鼠壤土,袋盛,熨之瘥。

又方∶蚕砂熟蒸,作练袋三枚,各受七升热盛一袋,着患处。如冷,即取余袋,一依前法,又方∶灸风池、肩、曲池、阳陵泉、巨虚下廉等穴。
治风痉方第五

《病源论》云∶风痉(充至反)者,口噤不开,背强而直,如发痫之状。其重者耳中策策痛;《效验方》治风痉身强方∶蒸大豆,熨之。(《千金方》同之。)又方∶蒸鼠壤土,熨之取汗。

《新录方》治风痉身强方∶薄荷三枚,以水六升,煮取二升,分二服。

《葛氏方》苦身直不得屈伸反复者方∶取槐皮黄白者,切,以酒若水六升,煮得二升,去滓,稍服。

《苏敬本草注》治风痉方∶铜屑熬令极热,投酒中,服五合,日三。

又方∶铁屑炒使极热,投酒中,饮酒,良。

《本草稽疑》治风痉方∶蒸蚕砂,熨之。
治柔风方第六

《病源论》云∶柔风者,血气俱虚,风邪并入,在于阳,则皮肤缓。在于阴,则腹里急。

柔《短剧方》治中柔风身体疼痛四肢缓弱欲作不遂方∶羌活(三两)桂肉(三两)生姜(六两)干地黄(三两)葛根(三两)夕药(三两)麻黄(凡八物,以清酒三升,水五升,煮取三升,温服五合,日三。(今按∶《千金方》云,酒二《葛氏方》治中缓风四肢不收者方∶豉三升,水九升,煮取三升,分三服,日二作。亦可酒渍煮饮。
治头风方第七

《病源论》云∶风头眩者,由血气虚,风邪入于脑,而引目系(胡计反)故也。

《养生方》云∶饱食仰卧,久成气病,头风。

又云∶饱食沐浴,作头风。

《耆婆方》治人一切风气风眩病三光散方∶秦胶(十二分)茯神(十二分)独活(八分)三味,切,捣筛为散,以酒服方寸匕,日三。依日月法。

又云∶治人风气,风眩,头面病四时散方∶秦胶独活茯神薯蓣四味,切,捣筛为散,以酒服一方寸匕,日二。依日月法。

春各四分,夏各二分,秋各八分,冬各十二分。

又云∶治人风气,风眩,头面风病五脏散方∶秦胶独活茯神薯蓣山茱萸(分两,依四时散)五味,切,捣筛为散,以酒服一方寸匕,日二。依日月散法。

又云∶治人风气,风眩,头面风,头中风病六时散方∶秦胶独活茯神薯蓣山茱萸本(依四时散分两)六味,切,捣筛以为散,以酒服一方寸匕,日二。依日月散法。

又云∶治人风气,风眩,头中风病,中风脚弱,风湿痹病七星散方∶秦胶独活茯神薯蓣天雄山茱萸本七味,切,捣筛为散,以酒服方寸匕,日二。依四时散法。

又云∶治人风气,风眩,头面风,中风;湿痹脚弱,房少精八风散方∶秦胶独活茯神薯蓣山茱萸本天雄钟乳(研七日)八味,切,捣筛为散,以酒服方寸匕,日二。依四时散分两,依日月散法。

又云∶治人风气,风眩,头面风,中风脚弱,风湿痹弱,房少精,伤寒心痛,中恶冷病十善秦胶独活茯神薯蓣山茱萸本天雄钟乳(研七日)夕药干姜十味,切,捣筛为散,以酒服一方寸匕,日二。依四时散分两。

《千金方》治风头眩,口目痛,耳聋大三五七散方∶天雄(三两)细辛(三两)山茱萸(五两)干姜(五两)薯蓣(七两)防风(七两)六味,下筛,为散渍酒,服五分匕,日再,不知稍加。

又云∶小三五七散主头风目眩方∶天雄(三两)山茱萸(五两)薯蓣(七两)三味,下筛,渍酒,服五分匕,日再,不知稍增,以知为度。

又云∶治头风方∶常以九月九日取菊花作枕袋枕头。

又方云∶芥子末,酢和,敷头一周时覆之。

又方∶葶苈子煮,沐,不过三四度,愈。

又方∶菊花,独活,草,防风,细辛,蜀椒,皂荚,桂心,杜蘅,可作汤沐及熨之。

(《《录验方》桃花散治风头眩倒及身体风痹走在皮肤中方∶石南(五两)薯蓣(四两)黄(三两)山茱萸(三两)桃花(半升)菊花(半升)真朱(凡八物,合冶下筛,食竟,酒服半钱匕,日三,稍增之。

《集验方》治风头眩欲倒眼旋屋转头脑痛防风枳实汤方∶防风(三两)枳实(三两,炙)茯神(四两)麻黄(四两,去节)细辛(二两)芎(三两)升)十一物,切,以水六升合竹沥,煮取二升七合,分三服,频服两三剂尤良。

《葛氏方》治患风头每天阴辄发眩冒者方∶取盐一升,以水半升和,涂头絮巾,裹一宿。当黄汁出,愈。附子屑一合,纳盐中尤良。

又方∶以桂屑和苦酒,涂顶上。

《范汪方》治鼻孔偏塞,中有脓血。此乃是头风所作,兼由肺疾。宜服此散方∶天雄(八分,炮)干姜(五分)薯蓣(四分)通草(六分)山茱萸(六分)天门冬(八分)凡六物,冶下筛,为散,酒服方寸匕,日再,稍加至二匕。

《极要方》疗风头痛,眼眩心闷,阴雨弥甚方∶防风(二两)当归(一两)山茱萸(一两)柴胡(二两)薯蓣(二两)鸡子(二两,去白黄上,为散,用鸡子黄和散令调,酒服方寸匕,日三。

《僧深方》∶治头风方∶吴茱萸三升。以水五升,煮取三升,以绵染汁,以拭发根,数用。

灸头风方∶《千金方》云∶灸天窗穴,在上星后一寸。

灸后顶穴,在百会后一寸。

《百病针灸》云∶灸百会穴,在顶上旋毛中。

又灸前顶穴,在亚会后一寸五分。

又灸五处穴,在当两眼入发际一寸。
治中风口噤方第八

《病源论》云∶诸阳经筋,皆在于头。三阳之筋,并络入于颔颊,夹于口,诸阳为风寒所客《葛氏方》治口噤不开者方∶取大豆五升,熬令黄黑,以五升酒,渍取汁。奏强发口,以灌之。

又方∶独活四两,桂二两,以酒水各二升,合煮取一升半,分二服,温卧。

《千金方》治中风口噤方∶术(四两)酒(三升)煮取一升,顿服之。

又方∶服淡竹沥一升。

《极要方》疗中风口噤不知人方∶豉(五升)茱萸(一升)以水七升,煮三沸,饮之。(《千金方》同之。)《广利方》理中风口噤不开方∶独活一大两。切,以清酒二大升,煮取一升半。即大豆五大合,熬取煎酒热,投豆中,密盖《效验方》治人卒中风欲死口不开身不得着席大豆散方∶大豆(二两,熬令焦)姜(二两)蜀椒(二两,去目,汗。)凡三物,捣下筛,酒服一钱匕,日一愈。

《新录方》治口噤方∶灸承浆穴,在颐前下,唇之下。

又方∶灸颐尖七壮。
治中风口方第九

《病源论》云∶风邪入于足阳明,手太阳之经,遇寒则筋急引颊,故使口,言语不正,《养生方》云∶夜卧当耳,勿令有孔,风入耳口喜。

《太素经》云∶颊筋有寒,则急引颊移口;有热,则筋弛纵缓不胜,故。

治之以马膏膏,其急者,以白酒和桂以涂;其缓者,以桑钩钩之,即以生桑炭置之坎中,高拊《录验方》治口眼相引僻者方∶以生鳖血涂之,以桑钩钩吻边,挂着耳也。血干复涂之,用白酒胜血。(《短剧方》同之。)《极要方》疗风口面兼暴风半身不遂语不转方∶以酒煮桂心,取汁湿故布,拓病上则止。左拓右,右拓左。秘不传,常用大效。

《僧深方》治风着人面引口偏着牙车急舌不得转方∶竹沥(一升)独活(三两)生地黄汁(一升)凡三物,合,煮取一升,顿服之。

又方∶翳风穴灸三壮,主耳聋,口眼为不正;牙车引口噤不开,喑不能言。甚神良。

穴在《短剧方》云∶眼动,口唇动,偏,皆风入脉故也。

急服小续命汤,摩神明眼膏。

又方∶灸吻边横纹赤白际,逐左右风乘不收处,灸随年壮。日日报之,三报且息,三日不效《千金方》治中风口方∶炒大豆三升令焦,以酒三升淋取汁,顿服之,日一。(《令李方》同之。)又方∶皂荚大者一枚,去皮、子,灸。一方一两下筛,三年苦酒和涂之,左涂右,右涂《集验方》治中风口不正方∶取空青如桑者,着口中含咽之,即愈。(《千金方》云∶如豆一枚,含之。)《葛氏方》治口者方∶衔奏灸口吻中横纹间,觉大热便去艾,即愈。勿尽艾,尽艾则大过,左灸右,右灸左。

又方∶鳖血和乌头涂之,欲止即拭去。

《范汪方》治中风口噤方∶豉(五升)茱萸(一升)合煮三沸,去滓饮汁,神验。

又方∶两手叉于头上,随左右,灸肘头三四壮。

《经心方》治口方∶青松叶一斤,捣令出汁。清酒一升,渍二宿,近火一宿。初服半升,渐至一升。头面汗即止又方∶取衣鱼摩发边,即正。
治中风舌强方第十

《病源论》云∶脾脉络胃,夹咽,连舌本,散舌下;心之别脉系舌本。今心脾二脏受风邪,《范汪方》治风舌强不语方∶豉煮汁,渐服,一日可数十过,不顿多。

又方∶新好桂,削去皮,捣下筛。以三指撮着舌下,咽之。

又方∶灸廉泉穴,在颐下结喉上舌本。(今按∶《华佗传》云∶中矩穴主中风舌强不语,在《集验方》治卒不得语方∶煮大豆取汁,稍含,咽之。

又方∶取桂一尺,水三升,煮取一升,顿服之。

《葛氏方》治中风不语方∶豉、茱萸各一升,水五升,煮取二升,稍服之。

又方∶灸第二、第三椎上百五十壮。

《录验方》治舌强不能语言舌下药矾石散方∶矾石(二两)桂心(二两)。

凡二物,下筛,置舌下,便能言。
治中风失音方第十一

《病源论》云∶喉咙(力公反,《尔雅》曰元鸟龙谓咙)者,气之所以上下也。喉厌者,声之语。

《范汪方》治失音大豆紫汤方∶大豆一升,熬令焦。好酒二升,合煮令沸。随人多少服,取令醉。

《效验方》治卒风失音大豆散方∶大豆(熬令焦)蜀椒(去目汗)干姜(各三两)凡三物,合,下筛,酒服一钱匕,日一,汗出即愈。

《千金方》治卒失音方∶浓煮桂汁,服一升,覆取汗。亦可末桂,着舌下,渐咽汁。

又方∶浓煮大豆汁,含之。豉亦良。

又方∶灸天窗,百会穴。(《新录方》同之。)《孟诜食经》治失音方∶杏仁(三分,去皮,熬,捣作脂)桂心末(一分)和如泥,取李核七个许,绵裹,少咽之。日五夜一。

又方∶捣梨汁一合,顿服之。
治中风声嘶方第十二

《病源论》云∶声嘶者风冷伤于肺所为也。

《葛氏方》治卒中冷声嘶哑方∶甘草(一两)桂心(二两)五味(二两)杏仁(三十枚)生姜(八两)切,以水七升,煮取二升,三服。

《耆婆方》治人声嘶喉中不利方∶桂心杏仁干姜芎甘草(各一分)上五味,捣筛,以蜜和,为丸如梧子,口中餐,咽汁。
治声噎不出方第十三

《葛氏方》治卒失声声噎不出方∶橘皮五具,水三升,煮取一升,顿服。

又方∶针大椎旁一寸五分。

又方∶浓煮苦竹叶,服之。

又方∶捣荷根,酒和,绞饮其汁。

矾石、桂末,绵裹如枣,纳舌下,有唾吐出之。

《耆婆方》治人风噎方∶羚羊角(五两,炙)通草(二两半)防风(二两)升麻(二两)甘草(四两,炙)五味,捣筛为散,以白饮服一方寸匕,日二。
治中风惊悸方第十四

《病源论》云∶风惊悸者,由体虚,心气不足。心之经为风邪所乘也,或恐惧忧恚,迫令心而不《极要方》四神镇心丸,疗男子读诵健忘,心神不定,心风虚弱,补骨髓方∶茯神(十二分)天门冬(十二分)干地黄(十二分)人参(八分)远志皮(八分)以上蜜丸,饮服十五丸,日再,加至三十丸。

《博济安众方》云∶治因重病虚损后;或因忧虑失心,惊悸心忪;或夜间狂言,恒常忧怕。

虎睛(一双,炙)金薄(五十片)银薄(五十片)光明珠(二分)雄黄(二分)牛黄(二分)上,如法研如面,以枣肉为丸,如绿豆大。每日空心以井花水下,三丸,或五丸,或七丸,《短剧方》远志汤,治中风心气不定,惊悸,言语谬误,恍恍惚惚,心中烦闷,耳鸣方∶远志(三两,去心)茯苓(二两)独活(四两)甘草(二两)夕药(三两)当归(二两)桂子(一两,炮)黄(三凡十二物,以水一斗二升,煮取四升,服八合。人羸可服五合。日三夜一。

《葛氏方》治人心下虚悸方∶麻黄、半夏分等,捣蜜丸,服如大豆三丸,日三。

《千金方》云∶补心汤主心气不足,多汗心烦,喜独语,多梦不自觉,喉咽痛,时吐血,舌本强,水浆不通方∶紫石英(二两)麦门冬(三两)茯苓(二两)人参(二分)紫菀(一两)桂心(二两)赤小九味,咀,水八升,煮取二升半,分三服,宜春夏服。

又云∶定志汤主心气不足心痛惊恐方∶远志(四两)菖蒲(四两)人参(四两)茯苓(四两)四味,咀,以,水一斗,煮取三升分三服。

《僧深方》云∶定志丸治恍惚忆忘胸中恐悸,志不定,风气干脏方∶人参(二两)茯苓(二两)菖蒲(二两)远志(二两)防风(二两)独活(二两)凡六物,冶下筛,以蜜丸,丸如梧子,服五丸,日再。(今按《范汪方》加铁精一合,细辛
治中风四肢不屈伸方第十五

《病源论》云∶四肢拘挛,不得屈伸者,由体虚腠理开,风邪在于筋故也。

《短剧方》云∶张仲景三黄汤。治中风手足拘挛,百节疼烦,发作心乱,恶寒引日,不欲饮麻黄(五分,去节)独活(五分)细辛(一分)黄(二分)黄芩(三分)凡五物,以水五升,煮取二升,分再服。一服即小汗出,两服大汗出,即愈。

《葛氏方》云∶若骨节疼痛,不得屈伸,近之则痛,短气自汗出,或欲肿者方∶附子(二两)桂(四两)术(三两)甘草(二两)水六升,煮取三升,分三服,汗出愈。

又云∶若手足不遂者方∶取青布烧作烟,于小口器中,熏痛处佳。

又方∶豉三升,水九升,煮取三升,分三服。
治中风身体不仁方第十六

《病源论》云∶风不仁者,由营气虚,卫气实。风寒入于肌肉,使血气行不宣流。其状,搔《葛氏方》云∶若身中有掣痛,不仁不随处者方∶取干艾叶一斛许,丸之。纳瓦甑下,塞余目,唯一目。以痛处着甑目上,烧艾以熏之,一时又方∶好术,削之,以水煮令浓热的的尔,以渍痛处良。
治中风身体如虫行方第十七

《病源论》云∶风大,虚风邪中于营卫溢于皮肤之间,与虚热并,故游弈遍体,状若虫行。

《千金方》治风身体如虫行方∶盐一升,水一石,煎减半,澄清,温洗三四遍,亦治一切风。

又方∶以大豆渍饭浆中,旦旦温洗面头,发不净加少面,勿以水濯之。

又方∶成练雄黄,松脂分等,蜜和,饮服十丸如梧子,日三。慎酒、肉、盐、豉。神秘不传
治中风隐疹方第十八

《病源论》云∶邪气客于皮肤,复逢风寒相折,则起风瘙隐疹。

若赤疹者,由凉湿折于肌中之热,热结成赤疹也。得天热则剧,取冷则减也。

白疹者,由风气折于肌中热,热与风相搏为白疹也。得天阴雨冷则剧出风中亦剧,得晴温则《素问》云∶赤疹忽起如蚊(亡云反,啮人虫也)蚋(竹合反,斑身小虻也),烦痒重沓垄起,《短剧方》白疹方∶宜煮蒴汤,与(羊洳反,参也)少酒,以浴佳。

又方∶以酒煮石南草,拭之。

又方∶水煮矾石汁,拭之。

又云∶赤疹方∶宜生蛇衔草涂之,最验。大法如治丹诸方。

《千金方》治隐疹百治不瘥方∶景天一斤,一名慎火,捣绞取汁,涂上。热灸手摸之,再三度即瘥。

又方∶矾石二两,末,酒三升渍令洋,拭上立瘥。

又方∶白芷叶煮汤洗之。

又方∶芒硝八两,水一斗,煮取四升,湿绵拭。

又∶淋锻石汁洗。

又方∶大豆三升,酒六升,煮四五沸,服一杯,日三。

《如意方》治隐疹术∶漏芦作汤,以洗浴。

《效验方》治风搔茱萸汤方∶茱萸(一升)酒清(五升)凡二物,合,煮得一升五合,绞去滓,适寒温洗,日三。

《录验方》黄连汤治风搔隐疹方∶芒硝(五两)黄连(五两)凡二物,以水八升,煮取四升,去滓,洗风痒处,日二。良。

《本草稽疑》风搔隐疹方∶煮蚕砂汁,渍之。

又方∶柠茎单煮洗之。

又方∶茺蔚叶可作浴汤。

《孟诜食经》风搔隐疹方∶煮赤小豆取汁,停冷洗之。
治中风隐疹疮方第十九

《病源论》云∶人皮肤虚,为风寒所折,则起隐疹。寒多则色赤,风多则色白,甚者痒痛,《葛氏方》卒得风搔隐疹,搔之生疮汁出。初痒后痛,烦闷不可勘方∶烧石令赤,以少水中,纳盐数合,及热的以洗渍之。

又方∶锉桑皮二斗许,煮令浓及热,以自洗浴。

又方∶以盐汤洗之,(大廉反)菜涂之。

又方∶以慎火合豉,捣以敷之。

《医门方》云∶煮蒴汤和少酒以浴之,极佳。

《刘涓子》方治耳体隐疹发疮方∶地榆根(三两)黄连(二两)大黄(四两)黄芩(四两)苦参(八两)甘草(六两)芎(上七物,以水六升,煮取三升半,浴洗。

《孟诜食经》云∶拧茎单煮,洗浴之。

又方∶茺蔚可作浴汤。

又方∶煮赤小豆取汁,停冷洗,不过三四。

又方∶捣蘩蒌,封上。
治中风癞病方第二十

《病源论》云∶凡癞病,皆是恶风及犯触忌害得之。初觉皮肤不仁,或淫淫苦痒如虫行;或最又有乌癞、白癞,论诸癞(诸癞,或本无此字)形状在本书依繁不载。

又云∶酒醉露卧,不幸生癞。

《录验方》云∶有人五癞八风之法∶一木癞、二石癞、三风癞、四水癞、五沸癞。

《葛氏方》云∶癞病乃有八种云云。

治白癞乌癞方∶苦参根皮干之粗捣,以好酒三斗,渍二十一日,去滓,服三合,日三。

又方∶干艾叶随多少,浓煮以渍面及饭酿如常法,酒熟随意饮,恒使熏。

又方∶取马薪蒿,一名马矢蒿,一名烂石草,捣末,末服方寸匕。日三,百日更赤起,一年又方∶捣好雌黄末,苦酒和,鸡羽染以涂疮上,干复涂之。

《僧深方》治癞方∶水中荷浓煮以自渍半日,用此方多愈。

又方∶水中浮青萍浓煮,自渍之。

《范汪方》治大风癞疮方∶取(音律)草一担,二石水,煮取一石汁,以渍疮,不过三渍愈。

《集要方》治癞方∶留黄,苦酒和涂之。
治中风言语错乱方第二十一

《病源论》云∶风邪者,谓风气伤于人也。人以身内血气为正,外风气为邪。若其居处失宜暑,悲《僧深方》五邪汤,治风邪入人体中,鬼语妄有所说,闷乱,恍惚不足,意志不定,发来往人参(三两)茯苓(三两)茯神(三两)白术(三两)菖蒲(三两)凡五物,水一斗,煮取二升半,去滓,先食,服八合,日三。

《范汪方》茯神汤,五邪气入体中,鬼语妄言,有所见闻说,心悸恸,恍惚不定,发作有时茯神(三两)菖蒲(三两)赤小豆(三十枚)人参(三两)茯苓(三两)凡五物,水一斗,煮取二升半,分三服。

《经心方》葱利汤,治邪发无常,骂詈与鬼语方∶乌头(一分,炮)恒山(一分)甘草(一分)葱利(一分)桃花(一分)五味,好酒四升,煎取一升,顿服,大吐。
治中风癫病方第二十二

《病源论》云∶五癫者,一阳癫,二阴癫,三风癫,四湿癫,五马癫也。人有血气少则心虚气并《千金方》云∶治癫疾者,视之,有过者即泻之,置其血于瓠壶之中,至其发时,血独动矣《范汪方》治五癫方∶铁精(一合)芎(一两)蛇床子(五合)防风(一两)凡四物,合和,捣下筛,日三服,日用一钱,有验即愈。

又方∶灸尺泽穴,在肘中动脉。

《效验方》高风散治癫劳方∶蜀天雄(一分,炮)山茱萸(一分)薯蓣(十四分)独活(八两,一方八分)石南草(二分(十二分)干姜(八凡十二物,冶下筛,酒服方寸匕,日三。不知稍增,以知为度。

《短剧方》治癫疾发作,僵仆不知人,言语妄见鬼方∶野狼HT子三升,清酒五升,渍出曝干。复纳汁尽,曝干,捣冶末,空腹服四分匕,日三。

《葛氏方》云癫病方∶灸阴茎上宛宛中三壮,得小便通便愈。

又方∶灸足大指丛毛中七壮。

又断鸡冠血,沥口中。

《千金方》云∶风癫∶灸天窗,百会各三百壮狂癫吐舌∶灸胃脘百壮狂走癫疾∶灸大幽百壮。

狂走癫痫∶灸季肋三十壮。

狂言妄语∶灸间使三十壮。

狂走喜怒悲泣∶灸巨揽随年壮。
治中风狂病方第二十三

《病源论》云∶狂病者,由风邪入并于阳所为也,风邪入人血脉使人阴阳二气虚实不调,若一实一虚,则令血气相并。并于阳则为狂发,则欲走。或自高贤,称神圣是也。

《千金方》云∶狂发少卧,不饥,自高贤,自辨知,自贵,大善骂詈,目夜不休。

《短剧方》治卒发狂方∶用其人着地,以冷水淋其面,终日淋之。

又云∶卒狂言鬼语方∶以甑带急合缚两手父指,便灸左右胁下对屈肘头,两火俱起,灸七壮。须臾鬼语,自云姓名又云∶狂骂詈掷打人方∶灸口两吻边燕丸处赤白际各一壮,并灸。

背胛间名臣揽三壮,三日一报之。

又方∶灸阴囊下缝三十壮,女人者灸阴会也。

《葛氏方》治卒发狂方∶烧虾蟆捣末,服方寸匕,日三。

又方∶煮三年陈蒲,去滓,服之。

又云∶狂言鬼语方∶针其足大拇指爪甲下,入小许即止。
治虚热方第二十四

《病源论》云∶虚劳而生热者,是阴气不足,阳气有余,故内外生于热,非邪气从外来乘之《极要方》疗一切虚热气壅滞结不通三黄丸方∶黄连(二两)大黄(二两)黄芩(三两)上件三物,捣蜜,和丸如梧子,食后服三丸,日三。

又云∶疗心膈间虚热气上迫咽喉口干方∶茯苓(五两)麦门冬(三升二合,去心)乌梅肉(二两)。

上,蜜丸如酸枣大,含消咽之。日夜含六七枚。若因食即口若者,加升麻三两。

《广济方》疗虚热呕逆不下食即烦闷地黄饮方∶生地黄汁(六合)芦根(一掘)生麦门冬(一升)人参(八分)橘皮(六分)生姜(八分)白蜜(三合)切,以水六升,煮取二升,去滓,下地黄汁,蜜,分温三服,如人行四五里进一服。不痢。

《经心方》大黄丸,治虚热食饮不消化,头眩引胸胁,喉仲介介口中烂伤,不嗜食方∶大黄(一两)黄芩(一两)黄连(三两)苦参(二两)龙胆(二两)五味,蜜丸如梧子,服五丸,日三。

又云∶生地黄煎治虚热及血利方∶生地黄汁三升。上,纳汁铜器中,于微火上煎令如饴服二合。

《效验方》龙胆丸,治朝寒暮热,手足烦,鼻张血青,不能饮食方∶龙胆(二分)黄连(二分)黄芩(二分)人参(二分)芒硝(二两)大黄(二分)凡六物,冶下筛,蜜丸如梧子,服五丸,日三。不知可至七丸。

《葛氏方》云∶若胸中热结,烦满闷乱,狂言起走者方∶以芫(音元)花一升,水三升,煮取升半,以布渍汤中,拓胸中上,燥复易。
治客热方第二十五

《病源论》云∶客热者由人腑脏不调,生于虚热。热客于上焦,则胸膈生痰实,口苦舌干;《耆婆方》治人客热方∶生地黄根一握,净洗,捣绞取汁,纳少许蜜,少少服之。

又方∶以竹沥待冷,少少饮之。

又云∶治季夏月客热方∶升麻(一两)甘草(一分,灸)蓝(二分)人参(一分)粟米(一升,一方一合)以水五升,煮取半升,去滓,夜露之,平旦一服之。

又云,治人舌涩不能食方∶荠(十二分)人参(二分)防己(二分)切,捣筛为散,以饮服一方寸匕,日二。服此方至夏月。恒须早服之,无此病之。

《录验方》竹茹汤,治胸中客热,口生疮烂,不得食方∶生竹茹(四两,去上青)生姜(四两)甘草(二两)前胡(二两)茯苓(二两)橘皮(一两)凡六物,水六升,煮取二升,分服,半日尽。

《范汪方》阳逆汤,治胸中有热,喘逆肩息方∶半夏(半升)人参(一两)石膏(如鸡子者一枚)生姜(四两)饴(四两)凡五物,以水一斗五升,煮得七升,服一升,日三夜二。

医心方卷第三背记忘多∶(《病源论》曰∶多忘者,心虚也。心神虚损而多忘。《养生方》云∶丈夫头勿北首∶(《病源论》云∶风喉者,风邪之气先中于阴。病发于五脏者,其状奄忽不知人,喉(《病源论》曰∶五脏六腑之精气皆上注于目,血气与脉并上,系上属于脑后,出于项中。)
卷第四
治发令生长方第一

《病源论》云∶发是足少阴之经血所荣也,血气盛,则发长美;若血虚少,则发不长,故须《僧深方》生发泽兰膏方∶细辛(二两)蜀椒(三升)续断(二两)杏仁(三升)乌头(二两)皂荚(二两)泽兰(二凡十一物,咀,以淳苦酒三升渍铜器中一宿,以不中水猪肪成煎四斤,铜器中东向灶炊以生发膏生长发,白黄者令黑,魏文帝秘方∶黄(二两)当归(二两)独活芎白芷夕药草辛夷防风生地黄大黄凡十六物,切,微火煎三上三下,白芷黄,膏成,去滓,敷头。(一方加麝香二分。)《千金方》治发令生长方∶麻子一升,熬令黑,压。取脂,敷头。

又方∶麻叶、桑叶。泔煮去滓,沐发七遍,长六尺。

又方∶多取乌麻花,瓷瓮盛,密盖,埋之,百八十日出,用涂发,长而黑。

《葛氏方》治发令长方∶术一升,锉之,水五升,煮以沐,不过三即长。

《新录方》治发令长方∶乌麻花末之,以生油和泥,涂之。

又方∶每暮好蜜涂如上,七日亦生。

《本草经》云∶鳢肠汁涂发眉,生速而繁。(注云∶一名莲子草。)《如意方》云∶长发术∶东行枣根直者,长三尺,以中央当甑饭蒸之。承两头汁以涂头,发长七尺。

又方∶白芷四两,煮沐头,长发。

又方∶麻子仁(三升)白桐叶(一把)米汁煮,去滓,适寒温以沐,二十日发长。

又方∶麻子仁(三升)秦椒(二升)合研,渍之一宿以沐头,日一,长发二尺。

又方∶乙卯丙辰日沐浴,令人发长。
治发令光软方第二

《如意方》软发术∶沐头竟,以酒更濯,日一,发即软。

又方∶新生乌鸡子三枚,先作五升麻沸汤,出扬之令温,破鸡子悉纳汤中,搅令和,复煮令又云∶光发术∶捣大麻子蒸令熟,以汁润发,令发不断生光泽,大良。
治发令竖方第三

《延寿赤书》云∶《太极经》曰∶理发宜向壬地,当数易栉,栉处多而不使痛。亦可令侍者《如意方》云竖发术∶马蔺灰(一升)紫宁灰(五升)胡麻灰(七升)凡三灰,各各淋之,先用马蔺灰汁,次用紫宁灰汁,后用胡麻灰汁。
治白发令黑方第四

《病源论》云∶血气虚则肾气弱,肾气弱则骨髓枯竭,故发变白也。

又云∶千过梳发发不白。

又云∶正月一日取五香煮作汤,沐头不白。

《隋炀帝后宫香药方》染白发大豆煎∶酢浆大豆上二物,以浆煮大豆以染之,黑便如漆。

《葛氏方》治白发方∶先沐头发令净,取白灰,胡粉分等,浆和温之,夕卧涂敷讫,油衣抱裹。明旦洗去,便黑。

又方∶拔白毛,仍以好蜜敷孔处,即生黑。

《千金方》治发白方∶正月四日,二月八日,三月十三日,四月二十日,五月二十日,六月二十四日,七月二十八上日拔之,不复白。

又方∶乌麻九蒸九曝,末,以枣膏丸,久服之。

又方∶生油渍乌梅,当用敷头良。

《灵奇方》令白发还黑术方∶陇西白芷(一升)旋复(一升)秦椒(一升)好桂心(一尺)合捣筛,井花水服方寸匕,日三,三十日白发悉黑,禁房内。以此药食白犬子,二十日皆变为《僧深方》欲令发黑方∶八角附子一枚,淳苦酒半升,于铜器中煎令再沸,纳好矾石大如博其石一枚;矾石消尽,纳《极要方》染鬓发白方∶数用大麻子泔浴之,极佳。

《龙门方》治发白方∶用皂荚汤净洗,干拭,以陈久油滓涂之,日三。(《千金方》同之。)《孟诜食经》治白发方∶胡桃烧令烟尽,研为泥,和胡粉。拔白发毛敷之,即生毛。(今按《本草拾遗》为泥拔白发《如意方》染发白术∶取谷实捣取汁,和水银以拭发,皆黑。

又方∶熟桑椹以水渍,服之,令发黑。

又云∶反白发术∶以五八午日烧白发。

又方∶癸亥日除白发,甲子日烧之,自断。
治须发黄方第五

《病源论》云∶足太阳之经血外荣于发,血气盛,则须发美而长。若虚少不足,不能荣润于《葛氏方》治须发黄方∶烧梧桐作灰,乳汁和,以涂其肤及须发,即黑。

《如意方》治须黄术∶胡粉,白灰分等以水和,涂须。

一方∶浆和,夕涂,明日洗去,便黑。

《录验方》染须发神验如柒方∶胡粉(三两)锻石(三升)以泔和粉灰等煮一两沸,及暖,揩洗发令遍,急痛水以濯之,经宿旦还直暖涂泔洗濯。

又以
治须发秃落方第六

《病源论》云∶血盛则荣于头发,故须发美;若血气衰弱不能荣润,故鬓发秃落也。

《经心方》治中风发落不生方∶铁生衣下筛,腊月猪脂合,煎三沸,涂,日三良。亦治眉落。

《葛氏方》治鬓发秃落不生长方∶麻子(三升)秦椒(二升)合研置沈汁中一宿,去滓,日一沐,一月长二尺。

又方∶生柏叶一斗,附子四枚,捣末,以猪肪三斤合和为三十丸。布裹一丸,着沐汁中,间日《医门方》治发落方∶油磨铁衣,涂之即生。

又方∶桑根白皮(二升)大麻子(二升)白桐叶(切,一升半)上,以米或泔九升,浸经一宿,煮五六沸,去滓,以沐浴发。

《千金方》治鬓发堕落方∶麻子(三升碎)白桐叶(切,一把)二味,以米泔汁二升,煮五六沸,去滓以洗沐,则头鬓发不落,二十日验。(《葛氏方》同《如意方》治鬓发秃落术∶桑树皮,削去黄黑取白,锉二三升,以水淹煮五沸,去滓,以洗沐鬓发,数为不落。

又方∶甘草二两,咀,渍一升汤中,沐头,不过再三,则不落。
治头白秃方第七

《病源论》云∶凡人有九虫在腹内,值血气虚则侵食。而蛲虫发动,最能生疮。仍成疽、癣并《千金方》治秃头方∶芜菁子,末,酢和,敷之,日一。

又方∶油磨铁衣,涂之,即生。

又方∶麻子三升,末,研,纳泔中一宿,去滓,日一沐,一月长二尺。

又云∶白秃方∶煮桃皮汁饮之,并洗上。

又方∶曲、豉两种下筛,酢和,敷上。

又方∶炒大豆黑末和猪脂,热暖匙抄封上遍即裹,勿见风。

又方∶桃花和猪脂封上。

《极要方》疗头风痒多白屑方∶大麻子仁(三升,研)秦椒(二升)柏叶(切,三升)上,并置于泔汁中一宿,明旦温之,去滓,用已沐发。(今按《集验方》无柏叶。)
治头赤秃方第八

《病源论》云∶赤秃(此由头疮,虫食发秃落)者,无白痂,有汁,皮赤而痒,故谓之赤秃。

《千金方》治赤秃∶桑灰汁洗头,捣椹封之,日中曝头。

又方∶马蹄灰,末,猪脂和,敷之。

又方∶烧牛羊角灰,和猪脂,敷之。
治鬼舐头方第九

《病源论》云∶人有风邪在于头,有偏虚处,则发秃落,肌肉枯死。或如钱大,或如指大,《千金方》治鬼舐头方∶烧猫儿矢,腊月猪脂和,敷之。

又方∶烧麝香,研,敷之。

又方∶赤砖末,和蒜捣,敷之。
治头烧处发不生方第十

《病源论》云∶夫发之生,血气所润养也。火烧之处,疮痕致密,则气血下沉,不能荣宣腠理,故发不生也。

《如意方》生毛发术∶取鸟内器,中埋于丙丁土入三尺,百日以涂人肉,即生毛。

又方∶涂好蜜。

《千金方》治火烧疮发毛不生方∶蒲灰,正月苟脑和,敷,毛生。又方∶芜菁子,末,酢和,涂毛生。
治眉脱令生方第十一

《病源论》云∶血气盛,则养眉有豪,血少则眉恶。又为风邪所伤,眉脱,皆是血气损伤,《千金方》生眉毛方∶垆上青衣,铁精分等,和水涂之。

又方∶七月乌麻花,阴干,以生乌麻油和,三日一涂,眉发。

《如意方》眉中无毛方∶以针挑伤,敷蜜,生毛。

《新录单方》生眉毛方∶油和铁精研,涂眉。

又方∶每暮好蜜涂之,七日亦生。

又方∶铁汁数洗之。
治毛发妄生方第十二

《病源论》云∶若风邪乘经络,血气改变。则异毛恶发妄生,则须以药敷令不生之。

《新录方》∶拔去毛,以蚌灰和鳖脂涂之,永不生。(《千金方》同之。)又方∶去毛,用狗猪等胆涂,即永不生。(《千金方》同之。)又方∶拔去毛,以伏翼血涂之,不生。

《千金方》∶除日拔毛,以鳖脂涂之。

又方∶狗乳涂之。

又方∶东行枣根灰,水和,涂之。
治头面疮方第十三

《病源论》云∶内热外虚,为风湿所乘,湿热相搏,故头面身体皆生疮。

《如意方》治面上恶疮术∶胡粉(五两,熬)黄柏(五两)黄连(五两)三物,冶下筛,粉面疮上,日三。(《短剧方》同之。)《极要方》疗面上疮,极痒,搔即生疮黄脂出,名曰肥疮方∶上,煮苦参汁,洗去痂,故烂帛淹,即涂白蜜,自当汁出如胶,即敷雄黄末,不过一两度,又云∶治头面恶疮胡粉膏方∶胡粉(三两)松脂(二两)水银(三两)猪脂(六合)凡四物,松脂,猪脂合煎去滓,以水银,胡粉着中,搅使和,涂疮上,日三。

《膏药方》治头面生疮痒黄连膏方∶黄连(四两)白蔹(二两)大黄(三两)黄柏(二两)胡粉(二两)上五物,下筛,以猪膏和涂之,时以盐汤洗之。(今按∶藜芦膏可敷之。在第二十五卷小儿
治面疮方第十四

《病源论》云∶面者,谓面上有风热气生,或如米大,亦如谷大,白色者是也。

又云∶《养生方》云,醉不可露卧,令人面发疮。(和名尔支养。)《养生要集》云∶酒醉热未解,勿以冷水洗面,发疮轻者。

《如意方》治术∶荠(二分)桂肉(一分)下筛,以酢浆服方寸匕,日三,晚即服栀子散相参也。

栀子散方∶栀子仁(一斤)捣下筛,先食,以酢浆服方寸匕,日三。先服荠桂散,次后服栀子散,即以同日服之。

《录验方》治男女面生疮黄连粉方∶黄连(二两)牡蛎(二两)凡二物,下筛,有脓汁以散粉之。

《葛氏方》治年少气盛面生疮方∶鹰矢白(二分)胡粉(一分)蜜和涂上,日二。

又方∶以三岁苦酒,渍鸡子三宿,当软破,取以涂。良。

《极要方》面癣肿方∶白附子(二两)青木香(二两)麝香(二两)拔(二两)并为散,以水和,涂面,日三。

《短剧方》治面方∶土瓜,冶,以水银、胡粉、青羊脂分等,和敷面上,日二。有效。

又方∶胡粉二分,水银四分,以猪膏和研,敷面,天晓以布拭去,勿洗水。

《千金方》治面甚者方∶冬葵子柏子仁茯苓瓜子凡四味,分等服方寸匕,日三。(今按∶《极要方》云∶二十日面目光泽,气尽去。)《刘涓子方》治方∶鸬屎一升,下筛,以腊月猪膏和,敷之。(《千金方》同之。)《新录方》治面方∶捣杏仁为泥,和浆若酪,涂之。

又方∶取兔系上秋露洗之,最佳。

又方∶大麻子研,和猪脂,涂。

又方∶鹿脂涂拭面上,自瘥。
治面方第十五

《病源论》云∶面者,谓面皮上,或有如乌麻,或如雀卵上之色是也。此由风邪客于皮《养生要集》云∶凡远行途中逢河水,勿洗面,生鸟如鸟卵之色斑也。

《葛氏方》治面多或如雀卵色方∶苦酒渍术,恒以拭面,稍稍自去。

又方∶桃花,瓜子分等,捣以敷面。

《千金方》治面方∶捣生兔系草汁,涂,不过三。

又方∶李子仁,末,和鸡子白,敷一宿,即落。

又方∶杏仁,酒渍皮脱,捣,绢囊盛,夜拭面。

《短剧方》治面方∶白蜜和茯苓,涂,满之七日便瘥。

又方∶杏仁去皮,冶令细,鸡子白和之,敷经宿,拭去。

《极要方》治面上黑粉泽等方白蔹(二分)生石(一分)白石脂(一分)杏仁(半分去皮)上为散,以鸡子白和,夜卧涂之,明晓以井花水洗之,老若更少黑者,白润。

《如意方》治术∶以鸬白矢敷之。

又方∶以树穴中水洗之。

又方∶茯苓,白石脂分等,末,蜜和涂之,日三。

《新录方》取蒺藜末,蜜和涂之。

又方∶蛴螬汁涂面。

《僧深方》∶桃仁冶下筛,鸡子白和以涂面,日四五。

《苏敬本草注》∶以桑薪灰洗之。

《陶景本草注》∶取蜂子未成头足时以酒渍,敷面,令悦白。
治鼻方第十六

《病源论》云∶此由饮酒,热势冲面,而遇风冷之气相搏所生也,故令鼻面间生,赤匝《葛氏方》面及鼻宿酒方∶鸬矢末,以腊月猪膏和涂之,鹤矢亦佳。

《僧深方》治蒺藜散方∶蒺藜子栀子仁香豉(各一升)木兰皮(半斤)凡四物,下筛,酢浆和如泥,暮卧涂病上,明旦汤洗去。

《千金方》治鼻栀子丸方∶芎(四两)大黄(六两)栀子仁(三升)好豉(三升熬)木兰(半斤)甘草(四两)上六味,蜜和,服十丸如梧子,日稍稍加至二十五丸。(《僧深方》云∶栀子仁二升,香豉《短剧方》治面木兰散方∶木兰皮一斤,渍以着三年酢中。趣令没之百日,出木兰皮,曝燥,捣为散,服方寸匕,日三《新录方》治鼻方∶木兰皮栀子仁豉等分为酢和如泥,涂上,日一。

《刘涓子方》木兰膏治鼻方∶木兰(二两)栀子(三两)凡二物,细切,渍苦酒一宿,明旦以猪膏一升,煎去滓,稍以摩之。

《如意方》治面术云∶前治荠桂肉方,亦治之在面方。
治饲面方第十七

《病源论》云∶饲面者,面皮上有滓如米粒者也。此由肤腠受于风邪,搏于津液,津液之气《葛氏方》治卒病饲面如米料敷者方∶十月霜初下,取以洗拭面,乃敷诸药为佳。

又方∶白蔹(二分)生石(一分)白石脂(一分)杏仁(半分)捣末。鸡子白和,暮卧涂一方无石、白脂,有鸡子白、蜜和新水以拭之。

《范汪方》治饲面方∶熬矾石,以酒和,涂之。不过三。

又方∶捣生菟丝取汁,涂之不过三,皆尽。
治疡方第十八

《病源论》云∶人颈边及胸前,腋下自然斑剥,点相连,色微白而圆;亦有乌色者。无痛痒《葛氏方》云∶面颈忽生白驳、状如癣、世名为疠疡方∶以新布揩令赤,苦酒摩巴豆涂之,勿广。

又方∶取生树木孔中汁拭之,末桂且唾和,敷之,日二三。

《千金方》治疠疡方∶酢磨留黄涂之,最上。

又方∶以三年酢摩乌贼骨,先布摩肉赤,敷之。

又方∶取途中自死蜣螂,捣烂,涂之。当揩令热封,一宿瘥。

《如意方》治疠疡术∶半天河水洗之。

又方∶荷叶上水洗之。

《极要方》疗面上生白驳名疠疡风方∶雄黄硫黄矾石以上等分为末,以猪膏和涂之。

又方∶取蛇脱皮磨之数过令热,乃弃之于草中,勿反顾。

《僧深方》治疠疡方∶硫黄(一分)矾石(一分)水银(一分)灶黑(一分)上四物,冶末,以葱涕和研,临卧以敷上。

又方∶糜脂数摩上。

又云∶疗身体易斑剥方∶女萎(一分)附子(一枚炮)鸡舌香(二分)青木香(二分)麝香(二分)白芷(一分)。

以上以腊月猪膏七合煎五味令小沸,急下去滓,纳麝香绞调,复煎三上三下,膏成。

磨令小伤,以敷之。

又方∶三淋灰取汁,重淋之。洗历易讫,醋研木防己涂之,即愈。

又方∶茵陈蒿两握。上,以水一斗,煮取七升,先以皂荚汤洗历易令伤,然以汤洗之。

《广济方》疗疠疡风方∶雄黄(一两)囟砂(二两)附子(三两,生)上为散,苦酒和如泥,涂之。

《经心方》治疠疡方∶取屋瓦上癣,先拭令赤,敷之。

《龙门方》疗疠易风方∶取皂荚子半升,细研,和生麻油,先用生布揩患处,复敷之。良。
治白癜方第十九

《病源论》云∶面及颈项,身体皮肉色变白,与肉色不同,亦不痛痒,谓之白癜。此亦风邪《千金方》治白癜方∶矾石、硫黄分等,末,酢和,敷之。

又方∶酒服生胡麻油一合,日三,稍加至五合。慎生冷、猪、鸡、鱼、蒜,百日服五升瘥。

又方∶揩上令破罗摩白汁涂之,日日涂之,取瘥。又煮以拭之。

《新录方》白癜方∶捣常思草汁涂,日三。

又方∶捣杏仁如泥,和鸡子白涂上,日三。

《葛氏方》云∶白癜风,一名白癞,或谓龙舐。此大难疗。取苦瓠经冬干者,穿头圆如钱许《录验方》治白癜方∶荷裹令叶相和,更裹臭烂,先拭令热,敷即瘥。

《极要方》疗白癜膏方∶附子(三两)天雄(三两)防风(二两)乌头(三)上,以猪膏三升煎之,敷上《刘涓子方》治白定方∶树穴中水汁向东者,熟刮洗白定二三过,即愈。枫树胜也。

又方∶生鸡卵一枚,纳苦酒中淹渍,令没鸡卵壳,壳欲消破之。先以白敷,次以黄敷,燥便又云∶疗颈及面上白驳浸淫渐长有似癣但无疮方∶上,取燥鳗鲡鱼,炙脂出,以涂之。先拭驳上,外把刮之,令小燥痛,然以鱼脂涂,便愈。
治赤疵方第二十

《病源论》云∶面及身体及肉变赤,与肉色不同,或如手大,或如钱大,亦不痒痛,谓之赤《千金方》治赤疵方∶用墨、大蒜、鳝血合和,敷之。

又方∶以银拭之令热即消,不瘥,数数拭之乃止。

《如意方》治白癜赤疵术∶用竹中水如马尿者洗之。

《徐伯方》治疵痧方∶独秃根凡一物,以苦酒研之,涂痧上,立即瘥。
治黑子方第二十一

《病源论》云∶黑痣者,风邪搏血气,变化所生。夫人血气充盛,则皮肤润悦,不生疵瑕,《录验方》五灰煎方∶锻石灰桑灰炭灰(各一升)蕈灰(五升)以水溲蒸令气匝,仍取釜汤淋之,取清汁五升许,于铜器纳东向灶煎之,不用鸡狗、小儿、《集验方》去黑子及赘方∶生梨灰(五升)锻石(二升半)生姜灰(五升)凡三物,合令调和,蒸令气溜下甑,取下汤一升从上淋之,尽其汁于铁器中,煎减半,更闲《如意方》治KT痣术∶鸬白尿敷之。

又方∶灰锻石醇苦酒煎,以簪涂黑,须臾灭去。

《葛氏方》去KT痣方∶桑灰艾灰各三斗,水三石,淋取汁,重复淋三过止。以五色帛纳中,合煎令可丸,以敷上《千金方》治疣赘疵痣方∶雄黄硫黄真珠矾石茹巴豆藜芦(各一两)七味为散,和合如泥,涂上,贴病上,须成疮,及去面点,皮中紫赤疵痣,靥秽。
治疣目方第二十二

《病源论》云∶人手足边忽生如豆,或如结筋。或五个,或十个,相连肌裹,粗强于肉,谓《葛氏方》∶以盐涂疣上,令牛舐之不过三。

又方∶作艾炷如疣大,灸上三壮。

又方∶以硫黄揩其上,二七过佳。

又方∶蒴赤子坏,刮目上令赤,以涂之,即去。

《千金方》∶每月十五日,月正中时,望月,以秃条帚扫二七遍,瘥。

又方∶松、柏脂合和,涂之,一宿失矣。

又方∶取牛涎数涂,自落。

《经心方》∶苦酒,渍锻石六七日,滴取汁沾疣上小作疮即落。良验。

《范汪方》∶月晦夜于厕前取取故草二七枚,枚二七过砭目上记祝曰∶今日月晦,尤惊或明又方∶杏仁烧令黑,研,涂,良。

《苏敬本草注》∶捣马苋揩之。(今按∶倍用赤苋,良。)又方∶以桑薪灰洗之。

又方∶缠蜘蛛网七日,消烂,甚效。

《如意方》∶取故拂床帚向青虹咒曰∶某甲患疣子,就青虹乞瘥,青虹没,疣子脱。意仍送帚,置都路口又方∶雷时以手疣,掷与雷二七过,即脱。

《集验方》∶七月七日,以大豆一合,拭疣目上,三过讫。使病疣目人种豆,着南向屋东头第三流中。

豆
治疮瘢方第二十三

《刘涓子方》治诸伤灭瘢膏方∶衣中白鱼鸡尿白蔹夕药白蜂白鹰矢上六物分等,合乳汁,和以涂伤上,日三。良。

《极要方》∶鹰矢白下筛,白蜜和,涂瘢上,日三。良。

《本草》∶白瓷瓦水摩,涂之。

《新录方》∶衣鱼摩上,日一。

又方∶胡粉敷,日一。

又方∶白疆蚕末,敷。

又方∶单用蜜涂之。

又方∶桑白汁和鸡子白,涂之。

又方∶榆白皮灰敷之。

又方∶涂鼠脂之。

《耆婆方》∶胡粉和白蜜,敷之。

《范汪方》∶以人精和鹰矢白敷之。(《医门方》云∶瘥后不知疮处,神验。)
治狐臭方第二十四

《病源论》云∶人腋下臭如葱豉之气者,亦言如狐狸之气者。故谓之狐臭也,此皆血气不和《葛氏方》云∶人身体及腋下状如狐气,世谓之狐臭,治之方∶正旦以小便洗腋下。

又方∶炊甑饭及热,丸之,以拭腋下,仍与犬食之,七旦如此,即愈。

又方∶青木香(一斤)锻石(半斤)合末恒以粉身。

《千金方》云∶有天生狐臭,有为人所染臭者,天生臭者难治,有为人所染者,易治也。

凡水银胡粉和涂之,大良验。

又方∶牛脂、胡粉各等分合煎,和,涂腋下。一宿即愈,不过两三《短剧方》云∶治漏腋下及足心,手掌、阴下、股里恒如汗湿致臭者,六物胡粉膏方∶干商陆(一两)干枸杞白皮(半两)干姜(半两)滑石(一两)甘草(半两)胡粉(一两)上六物,冶末,以苦酒和涂腋下,微汗出,易衣复更着之,不过三便愈。或一岁复发,发复《灵奇方》∶常以矾石熬末,敷两腋下。

《新录方》∶取白马尿洗之。

又方∶酢和胡粉涂腋下,日一。

《枕中方》治人气臭方∶丑时取井华水,口含吐着厕中,良。

《经心方》∶取白马蹄煮取汁,拭腋下,日二。

又方∶苦酒和白灰涂,燥复易。

《本草云》∶裹铁精以熨之。

又方∶铁屑和酢封腋,铜屑又佳。

《效验方》治腋臭鸡舌散∶鸡舌香(二两)藿香(二两)青木香(二两)胡粉(一两)凡四物,冶下筛,绵裹纳腋下押之,拊须着乃止。

《范汪方》治腋下臭方∶干姜白芷胡粉白灰凡四物分等,合粉腋下。

又方∶青木香散∶青木香(二两)附子(一两)白灰(一两)矾石(半两)凡四物,合捣,着粉中汁出,粉粉之愈。

《删繁论》治狐臭方∶杜衡本辛夷芎细辛(各二分)胡粉(十分)凡六物,咀,以苦酒二升渍,煎取三合,去滓,和胡粉临卧涂腋下。

《集验方》治狐臭方∶辛夷细辛芎青木香四物分等,捣筛为散,粉之。

《隋炀帝后宫诸香药方》治腋下臭方∶雄黄(五分)麝香(五分)石硫黄(六分)薰六香(五分)青矾石(五分)马齿草(一握)上件药总和,捣熟出泔瓦上曝令干,更捣下筛为散。以酢浆洗臭处,以生布揩令破,以粉之本卷接纸凡十七叶,每叶有二草字,字体难辩识。或者尚翰苟用此为记号欤?是未可知也,医心方卷第四背记治白发令黑方∶胡分(三两)锻石(六两,绢下熬令黄。)二味,以榆皮作汤,和之如粉粥,先以皂荚汤净洗发,令极净必好,于夜卧以药涂发上,令发。(秘
卷第五
治耳聋方第一

《病源论》云∶耳聋者,肾(时忍反)为足少阴之精而藏精,其气通耳。耳,宗脉所聚也。若者则耳宗脉《养生方》云∶勿塞故井及水渎,令人耳聋目盲。

《葛氏方》云∶聋有五种∶风聋者,挚痛;劳聋者,黄汁出;干聋者,耵聍生;虚聋者,萧鲤鱼脑,以竹筒盛蒸之,炊下熟,热气以灌耳,绵塞莫动,半日乃拔塞。用胆亦良,蒸毕塞裹塞良。)又方∶灸手掌后第二横纹中央,随聋左右,依年壮。

又方∶伏翼血纳耳中甚良,脑中血尤妙。

又方∶鼠脑绵裹纳中良。(今按∶《博济安众方》云∶取猫伤了鼠胆一枚,侧卧沥耳中,一《千金方》治耳聋方∶绵裹蛇膏,塞耳,神良。

又方∶雄黄,硫黄分等,绵裹,塞数日,闻。

又方∶作泥饼,浓薄如馄饨皮,覆耳上四边,勿令泄气。当耳孔上,以草刺泥饼,穿作一小干,《短剧方》治耳聋方∶巴豆(十四枚,去心皮)松脂(半两,炼去滓)凡二物,合捣,取如黍米粒大,着簪头,着耳中,风聋即愈。劳聋当汗出,痒后乃愈,数用又方∶灸听会穴,在耳前陷中。

《范汪方》治耳聋方∶鸡子一枚,渍苦酒七日,塞耳,当取其黄汁用注中。神良。

又方∶以淳苦酒微煎附子五六宿,削令可入耳中,裹以絮,塞耳。

《新录方》云∶治耳聋方∶雀脑绵裹如杏仁,塞耳中,日一易。

又方∶生地黄燠软,绵裹塞耳。

又方∶燠石上菖蒲,塞耳。

《录验方》云∶菖蒲散治耳聋方∶菖蒲附子(分等)下筛,以酒和如枣核,绵裹,卧时塞耳,夜易之,十日愈。(今案∶《博济安众方》∶菖蒲《效验方》云∶杏仁丸治耳聋方∶杏仁(十分)桂(二分)和丸如鼠矢,绵裹塞耳中,日三。

又云∶菖蒲丸方∶菖蒲根(一寸)巴豆(一枚,去心皮)凡二物,捣合,分作七丸,绵裹如大豆,塞耳中,朝一夕一,良。

《极要方》疗三十年聋方∶杏仁、葶苈、盐等分,以猪脂煎,绵裹塞耳,良验。

《救急单验方》疗耳聋方∶捣鹅膏,沥耳中。数数着,瘥。
治耳鸣方第二

《病源论》云∶耳者,宗脉之所聚。宗脉虚,则风邪乘虚随脉入耳,与气相击,故为耳鸣。

《千金方》治耳鸣如流水声不治久成聋方∶生乌头,蒸削如枣核大,塞耳,日一夜一易,不过三日,愈。亦治痒及风聋。

《短剧方》治风聋耳中鸣方∶但用鲤鱼脑竹筒盛,塞头蒸令烊,冷以灌耳,即愈。

又方∶附子,菖蒲分等,捣,以绵裹,塞两耳,甚良。

《葛氏方》云∶耳田中恒鸣方∶生地黄切断,仍以塞耳之,日夜数十易。亦治聋。

又云∶卒得风耳中吼吼者方∶急取盐七升,甑中蒸使热,以耳枕盐上,冷易之。
治耳卒痛方第三

《病源论》云∶凡患耳中策策痛者,皆是风入于肾之经也。

《葛氏方》治耳疼痛方∶蒸盐熨。又云∶痛有汁出者方∶熬杏仁令赤黑,熟捣如膏,赤裹又云∶耳卒肿出脓者方∶末矾石,着管中吹入耳,三四过当愈。

《医门方》治耳痛方∶菖蒲,附子分等,末,以乌麻油和如泥,取如豆灌耳中,立愈。(今按∶《龙门方》∶绵裹
治耳方第四

《病源论》云∶耳者,宗脉之所聚,肾气之所通。足少阴,肾之经也。劳伤血气,风热乘虚《葛氏方》耳耳中痛脓血出方∶釜月下灰吹满耳令深入,无苦即自丸出。

又方∶捣桂,以鱼膏和塞耳,不过三四。

又方∶桃仁熟捣,以赤裹塞耳中。(今按∶《博济安众方》∶杏仁炒如膏塞之。)《短剧方》治耳出脓汁散方∶矾石(三两,烧令汁出尽)黄连(一两)乌贼骨(一两)上三物,捣,冶下筛如枣核大,绵裹塞耳,日二。

又云∶耳中脓血出作耳,治之不愈,是有虫也。治之方∶鲤鱼肠一具,细锉之,以三升,合捣布裹以塞两耳。食顷当闻痛,痛则者应有白虫出。

着《博济安众方》疗耳出脓∶杏仁炒令赤,捣如膏,绵裹塞耳。

又方∶细辛末、附子末,以葱涕和,灌耳中。

又方∶松脂为末,挑安耳中,再安极妙。

又方∶石首鱼脑中枕子为末,安耳中。

《华佗方》治耳方∶雄黄,矾石分等,以绵缠箸头拭脓,如大豆着耳中,湿者以药敷之。

《广济方》疗耳脓血出方∶取车辖脂塞耳中,出脓血愈。

《极要方》疗耳出脓水方∶白矾(一分,烧令沸)白龙骨(一分)乌贼鱼骨(一分)蒲黄(二分)上为散,绵裹纳耳中,日夜五遍,于耳中着十日内必瘥。

《救急单验方》疗胝耳脓血出方∶取成练白矾石如小豆纳耳中,不过三,瘥。

《录验方》治耳中痛脓血出菖蒲散方∶椒(二两)当归(二两)姜(二两)菖蒲(二两)附子(二两)凡五物,冶合下筛,绵裹塞耳孔,时时易之。
治耳耵聍方第五

《病源论》云∶耳耵聍者,是耳里津液结聚所成。人耳皆有之,轻者不能为患,若加以风热《葛氏方》治耵聍塞耳而强坚不可得挑出方∶捣曲蚯蚓,取汁以灌耳中,不过数灌,摘之皆出。(《千金方》同之。)
治百虫入耳方第六

《葛氏方》治百虫入耳方∶以好苦酒渍椒灌之,以起行便出。

又方∶绵裹猪肪塞耳,须臾虫死,出着绵。

又方∶闭气,令人以芦管吹耳。

又方∶捣生姜汁灌之,韭汁亦佳。

又方∶以两刀于耳前相敲作声,虫即出。

又方∶温汤令的的(都洒反)尔,以灌之。

又方∶烧干鳝头屑,绵裹塞耳,立出。

又方∶以草带钩草向耳孔,即诸虫皆出,勿令钓罗耳孔中,内虫即死耳中。

《短剧方》虫入耳者方∶取椒一撮(七活反)末之,以半升酢浆渍取汁,温灌耳中,行十四步,虫则出。

又方∶酱、苦酒、浆汁灌之。

又方∶绵裹白膏塞耳,虫则死,着绵出。

又方∶用车毂脂涂耳孔,虫则出。

《极要方》百虫入耳方∶以苦酒灌之。

又方∶生姜汁灌之。

又方∶韭汁亦佳。

《千金方》虫入耳方∶桃叶。塞耳,立出。

又方∶以葱涕灌耳,即出,大验。

又方∶车脂涂耳孔,虫自出。

《范汪方》虫入耳方∶水银如大豆置耳中,须臾令耳向下,以铜物击齿数十,即出。

又方∶捣(KT)菜,以汁灌之。

《新录方》治虫入耳方∶干姜末吹耳中,出。

又方∶绵裹铜屑塞耳。

《龙门方》疗百虫入耳方∶熬胡麻,以疏布裹作枕,枕头即出。

又方∶铜器近耳边打作声,即出。
治蜈蚣入耳方第七

《葛氏方》蜈蚣入耳方∶取新热豚肉若炙猪肉以当耳孔中安之,即出。

《医门方》治蜈蚣入耳方∶炙猪肉令香,掩耳,立出。(《千金方》同之。)《极要方》治蜈蚣入耳方∶以椒叶裹盐,炙令热以掩耳,冷即易,立验。
治蚰蜓入耳方第八

《葛氏方》云∶蚰蜒者,世呼为土蛩,似蜈蚣,黄色而细长,治入耳方∶以水银如大豆一枚,泻耳中。

又方∶熬胡麻,以葛囊盛,枕之,虫闻香觉出,即瘥。(今按∶《医门方》捣碎用之。)《短剧方》治蚰蜒入耳方∶炒麻子,葛囊盛之,倾耳枕之,虫闻香则出。

《极要方》治蚰蜒入耳方∶灌油即出。

又方∶桃叶汁灌之。

又方∶铜器近耳边,即出。(今按∶《医门方》∶打作声。)《千金方》治蚰蜒入耳方∶牛乳灌之。
治蚁入耳方第九

《葛氏方》治蚁入耳方∶炙脂膏香物,安耳孔边,则自出。

又方∶烧陵鲤甲末,水和灌耳中,出。

《龙门方》云∶耳边炙肉,即出。

《医门方》云∶以猪头炙令香,安孔边,立出。
治飞蛾入耳中方第十

《葛氏方》治飞蛾入耳中方∶以苇管吹之,立走出。
治水入耳方第十一

《新录方》治水入耳方∶取鱼目为灰,纳水中便出。

《龙门方》治水入耳方∶取水银豆许,安耳边,水出。
治耳中有物不出方第十二

《千金方》云∶耳中有物不可出方∶以麻,从一头令散,涂好胶柱,着耳中物令上,停之令相着,徐徐引之令出。
治目不明方第十三

《病源论》云∶夫目者,五脏六腑阴阳精气,皆上注于目。若为风邪所侵,则令目暗不明也《养生方》云∶恣乐伤魂,魂通于目,损肝则目暗。

《靳邵服石论》云∶凡洗头勿使头垢汁入目中,令人目痛。

《养生要集》云∶以冷水洗目,引热气令人目早瞑。

《治眼方》云∶治眼七病,一伤于房,精气虚竭;二伤大风;三伤于大寒;四伤于大热;五翳,或时苦疼痛,或但冥无所见。治之方∶决明子(四分)车前子(六分)白术(六分)地肤子(六分)细辛(四分)柏子仁(六分)凡八物,下筛,服方寸匕,日三。百日后眼疾除,远视明。

又云∶治眼失精,一岁二岁至三四岁,或目中无他病,但无所见,如绢中视。决明散方∶马蹄决明(二斗)凡一物,冶下筛,以粥清服方寸匕,日三。禁食生鱼猪肉辛菜。

《葛氏方》治目失明三十年不识人钟乳云母散方∶钟乳(四分)茯苓(四分)远志(四分)细辛(四分)云母(四分)上五物,捣下筛为散,服半钱匕,稍增至一钱。

又云∶治目不明方∶决明子(一分)蕤核仁(分)黄连(二分)秦皮(二分)上四物,切,以水八合,煎取三合,沾绵洗目中。

又方∶三岁雄鸡冠血,数数敷之自瘥。

《大唐延年方》治目茫茫无所见芜菁散方∶芜菁子(小二升,以水一大斗,煮取令尽,汁出日干,熬散)炼胡麻(小三升,熬为散)二味冶合,以饮苦酒服之。

《短剧方》治目卒不所见方∶锉梓木,煮以洗目,日三。(《葛氏方》同之。)《千金方》神曲丸主明目百岁可读细书方∶神曲(四两)磁石(二两)光明砂(一两)三味,饮服如梧子三丸,不禁。常服益眼力。众方不及,学人宜此知方神验,不可言。

又方∶芜菁子三升,净淘,醋味清酒三升煮令熟,曝干,冶下筛,以井花水和服方寸匕,日又方∶三月三日采蔓菁花,阴干,末之,空腹井花水服方寸匕。久服长生目明,可夜书。

又方∶胡麻一石,蒸三十遍,末之,每日酒服一升。

又云∶治目茫茫不明如年老方∶鲤鱼胆一枚,取汁染绵,拭目。

又云∶治眼暗灸方∶灸大椎下,数取第十节。正当脊中央二百壮,唯多为佳,至验,不须方《录验方》黄连太一丸治肝气热冲目令视瞻方∶黄连(二斤)凡一物,以好清酒一升,淹一宿,出曝之,干,复纳酒中。如是十过,酒尽为度。干捣筛,蜜和丸如梧子,一服七丸,日再。禁、猪、鱼、犬、马、鸡肉、五辛、生冷,余依药法。

《苏敬本草注》云∶捣绞地肤汁洗之。(今按∶《录验方》∶捣筛,酒服方寸匕,日三。)《范汪方》治目冥茫茫方∶蕤核(三分)黄连(二分)干姜细辛(各一分)凡四物,咀,蜜三合,水三合渍之一宿,煎得二合,如米注目中,日三四。

《集验方》治目不明苦泪出方∶用乌鸡胆,临眠敷之,良。

又方∶摘小酸模茎汁注四,数为之。

《僧深方》治目盲十岁,百医不能治,郁金散方∶郁金(二两)黄连(二两)矾石(二两)凡三物,冶令筛,卧时着目中,如黍米,日一。
治目清盲方第十四

《病源论》云∶清盲者,谓眼本无异,瞳子黑白分明,直不见物耳。若脏虚有风邪痰饮乘之谓之《眼论》云∶夫人苦眼无所因起,忽然幕幕,不痛不痒,渐渐不明,经历年岁,遂致失明。

要针又云∶夫清盲之为病,发在于内。有障状似凝膏,大如楮子,浮在眼内,游泊水中,正障瞳虚谈,辞妄说,徒施千万,竟不收一,虽复卢医起骨,华佗解KT,此皆偏学一边,各善一术。至于清盲内障,则自拱手。

《治眼方》治眼清盲无所见斑浮鸠散方∶斑浮鸠(一头,冶如食法,炙令熟)决明子(半升)细辛(二两)防风(二两)凡四物,咀,合封十五日,干之,冶下筛,酒服方寸匕,日三夜二。

又云∶治清盲无所见三十年方∶细辛(一分)荧火(十二枚)芜菁子(一升)鲤鱼胆(三枚)凡四物,冶芜菁子,细辛,荧火下筛,以鱼胆和之,不足,人乳汁益之,服如梧子三丸,七又方∶猪胆(一枚)凡一物,微火上煎之,令可丸以如黍米,纳眼中食顷有验,如方始生翳宜少敷,不可过多。

《耆婆方》治人目清盲昼夜不见物方∶秦皮、升麻、黄芩分等,以水三升,煮取一升半,沾绵,敷目中。

《短剧方》目清盲无所见方∶以赤鲤鱼胆并脑,杂真珠合和,绵取注中。
治雀盲方第十五

《病源论》云∶人有昼而睛明,至暮则不见物,世谓之为雀目。言如鸟雀,至暮无所见也。

《千金方》治雀目术∶令雀盲人至黄昏时,看雀宿处,打惊之,雀起飞,乃咒曰∶“紫公紫公,我还汝盲,汝还我明《葛氏方》治雀盲方∶以生雀头血敷目,可比夕作之。

又方∶鼠胆敷之,最良。

《新录单方》治雀盲方∶鲤鱼、鲋鱼胆敷如粟并良。(今按∶《葛氏方》∶鲤胆若脑敷。)《耆婆方》治雀盲方∶取猪肝去上白幕,切作脍,以淡姜齑,三朝空腹食之,瘥。

《录验方》治雀盲方∶小蒜一升,咀,以水四升,煮令蒜熟,着小口器中,以目临上,当小辛,可无苦。
治目肤翳方第十六

《病源论》云∶阴阳之气,皆注于目。若风邪痰气乘腑脏,腑脏之气虚实不调,故气冲于目《眼论》云∶若因时病后眼痛生白障,此为翳也;若因病后生赤肉者,此为肤障也。

又云∶若已生翳者,当镰之。其中有赤脉处,当以钩钩甘刀割断也。日日针镰、敷散,若钩者,更甚割断甘刀势也《治眼方》治眼急生肤翳及赤肉上黑精上服此泻肝汤方∶黄芩(二两)夕药(二两)芒硝(一两)甘草(一两半)大黄(二两)大枣(十二枚)凡六物,以水六升,煮取二升五合,分为三服。

又方∶卧时以胡粉注翳上,治三十年翳,甚良。

又云∶眼中卒生肤覆瞳子赤白者盐散敷方∶盐(一铢)头垢(一铢)干姜(一铢)凡三物,盐烧合冶,取如小豆大着眼中,不过二。

又云∶胡粉散治眼卒冥及生翳方∶胡粉(一铢)干姜(一铢)凡二物,末之,以筒吹,少少着眼中,大良。

又云∶治眼中生息肉并白赤障翳散方∶贝齿(一分)真珠(一分)凡二物,冶下筛,爪取如小豆着翳上。病患正仰眼,他人与之,可再三。经一食久拭之。

神又云∶治眼生淫肤覆瞳子上方∶随眼痛左右,灸眉当中瞳子七壮,便愈。

《葛氏方》治卒生翳方∶灸手大指节上横理三壮,左目灸右,右目灸左。

又方∶烧贝齿,细筛,仰卧令人以着翳上,日二三,一时拭去。

《千金方》治目翳方∶雄雀屎,人乳汁熟和,敷肤上,自消烂坏尽。

《僧深方》治目白翳方∶牡蛎乌贼鱼骨(分等)下筛以粉目,日三。亦可治马翳。

又方∶煮露蜂房,以汁洗之,数数洗良。

《范汪方》治目有热卒生翳方∶取书中白鱼,曝令干,末,少少注翳上,一注便愈。

又方∶捣枸杞汁,洗翳上,日五六良。

《录验方》治目翳干姜散方∶干姜雄黄(分等)下筛取如米,着翳上,日二。

《集验方》治白翳覆瞳子黑精龙骨散方∶龙骨(一分)贝齿(三枚,烧)矾石(一分,烧)凡三物,冶下筛,着头,日二。

《崔禹锡食经》云∶纳鸳鸯肪良。

《极要方》记曰∶有三人眼后肤肉生,前覆黑瞳子上,使割之,三日辄复生,不可止,有直
治目赤白膜方第十七

《治眼方》治眼卒生肤翳赤白幕方∶取薤白,弱刀截以注肤上,注之其使周遍,幕皆着薤头,去眼不耐辛,不过得再三注也。

又云∶治目白膜覆瞳子无所见方∶以鲤鱼胆涂铁镜面,一宿令干,刮取之,曝干。末涂目翳上,日再,神验。

《葛氏方》治目热生淫肤赤白膜方∶取生瓜牛一枚,去其厌,纳朱于中,着火上,令沸,绵注取,以敷中。

又方∶取雀矢细直者,以人乳和,敷膜上,自消烂尽也。

又方∶捣枸杞汁洗之,日五六。(《集验方》同之。)
治目息肉方第十八

《病源论》云∶息肉淫肤者,此由邪热在脏,熏于目,热气加于血脉,蕴积不散,结而成息《葛氏方》治目中生肉稍长欲满目及生珠管方∶捣贝齿,绢筛真丹分等,搅令和,以注肉上,日三四。

《广利方》理目久风赤,生息肉,痛,开不得方∶黄连(八分,碎)大枣肉(四分)竹叶(两握,切)蜜(半合)切,以水二大升先煎竹叶,取一大升。去竹叶,下枣肉,黄连,蜜半合,煎取四合,去滓,《苏敬本草注》云∶雀矢和首生男乳如泥,点目中。怒肉赤脉贯上瞳子者即消,神效。
治目珠管方第十九

《病源论》云∶目珠管者,风热痰饮积于脏腑,使肝脏血气蕴积,冲发于眼,津液变生结聚《范汪方》治目卒生珠管方∶以蜜涂目中,仰卧须臾,当汁出,随拭去之。半日乃可渗之,生蜜尤良。

又方∶鲤鱼若鲭鱼胆注中,以少真丹和胆缚尤佳。

《千金方》治目生珠管方∶冷石手爪甲龙骨三味,分等末为散,以新笔着上,日三。

《葛氏方》治目卒生珠管方∶捣牛膝根叶,取汁以洗目,亦入目中佳。
治目珠子脱出方第二十

《病源论》云∶风热痰饮积腑脏,则阴阳不和,肝气蕴积生热,热冲于目,使目睛疼痛,热《治眼方》目卒珠子脱出及青翳方∶越燕矢(一分)真丹(一分)干姜(一分)凡三物,捣细末,以管吹痛上,即愈。(今按∶《本草》∶越燕紫胸轻小也,胡燕胸斑黑,声大也。)《医门方》眼睛无故突出一二寸者方∶急以冷水灌渍眼中,数数易水,须臾睛当自入,平复如故也。
治眼肿痛方第二十一

《眼论》云∶若初患眼肿痛者,不可以物薄熨之。恐热势归里,当时虽好,久之不佳。

深可《治眼方》治眼肿痛方∶大黄(八两)以水五升渍之一宿,明旦绞取汁,分二服。

又方∶以酢浆作盐汤洗之,日可十反。

又方∶秦皮(二两)黄连(一两)苦竹叶(切一升)以水二升,煮取七合,洗眼也。

又云∶治眼暴天行风肿痒痛方∶地骨白皮(三斤)水三斗,煮取三升,绞去滓,更纳盐二合,取煎一升,敷目也。或加干姜一两。

又云∶治眼风肿痒痛方∶防风(二两)地骨白皮(二两)细辛(一两)干姜(一两)以水煮取七合,洗眼。

又云∶治酒后热毒肿痛方∶栀子仁(一升)茈胡(三两)石膏(三两)芒硝(二两)大黄(二两)黄芩(一两)甘草(以水五升,煮取一升半,洗眼。
治目赤痛方第二十二

《病源论》云∶目赤痛候,肝气通于目,言肝气有热,热冲于目,故令赤也。

《眼论》云∶治目赤痛涩不得开方∶鲤鱼胆(一枚)黄连(二七枚)凡二物,合和,淹于二斗米下蒸之,饭熟去滓,涂目,立愈。

又云∶治眼卒掣痛方∶灸当瞳子上入发际一寸七壮,痛即止。两眼痛,灸两眼处。甚良。

又云∶治目中白精赤如血栀子煎方∶黄连(一两,下筛)肥栀子中仁(二分,研)二物,研如上法,和以鸡子黄如泥,更熟研,以绢绞去滓,用注目,日七八。鸡子易臭,唯《博济安众方》治赤眼肿痛热泪下立验方∶黄连为末,绵裹以甘蔗汁浸良,久点之。

又云∶热毒风攻两眼并睑忽浮肿眼赤复欲上肉等方∶黑豆(一升,择)上,分作十分,将软绵帛子逐分裹于沸汤,蘸过蒸热,慢慢熨之,每分三度入汤,尽其十分又云∶治积年风赤眼方∶上,取长明灯油盏,纳油少许。以一铜钱于覆钵内细细磨之,令油凝钵底,覆却,以艾烟微《葛氏方》治目卒赤痛方∶捣荠菜根汁,洗之。

又方∶当灸耳叶上七壮。

又方∶鸡舌(二七枚)黄连(一两)大枣(一枚)上三物,切,以水一升,煮取三合,先以冷水洗,染绵拭目,日三,大良。

《千金方》治目赤痛方∶甘竹叶(二七枚)乌梅(三枚)古钱(三枚)。

凡三味,水二升,先渍药半日,东向灶煮三沸,三上三下,得二合,注目。

《范汪方》治目赤痛方∶干姜(二分)黄连(四分)凡二物,冶合已,乳汁和,如黍米,注四,昼夜无所在。

又方∶黄连(一两)丁香(二十枚)以水八合,渍之三日,去滓,洗眼。

《短剧方》治目痛方∶以盐汤洗之良。

又方∶以荷根取汁,着竹筒中,着目中,即愈。

《录验方》治目赤痛黄连汤方∶黄连(二分)大枣(十枚)凡二物,切,以水五合,煮取一合半,注目中,日十夜二。

《耆婆方》治人眼赤痛方∶秦皮(二两)升麻(三两)黄连(二两)三味,以水三升,煮取二升,去滓,少少纳目中,洗之。

《集验方》治目痛三十年方∶取虫螺一枚,以水洗之。纳燥杯中,使螺口开。以黄连一枚,纳螺口中,螺饮黄连,黄连苦
治目胎赤方第二十三

《龙门方》疗大赤眼胎赤方∶以绳从顶旋,量至前发际中,屈绳头,灸三百炷,验。

又方∶青荆烧令出汁,点眼,验。

《广济方》疗目赤痛及胎赤方∶以蚌蛤裹置蜜二分,绿盐一分,和,夜卧时火灸暖,着目,三四日愈。

又方∶猪胆和绿盐敷,亦效。
治目痒痛方第二十四

《疗眼方》治目茫茫痛痒泪出方∶用新熟米酒,正仰面卧,以酒灌目令满;便急闭目,须臾开之。别使年少明目人看视,虫即自愈。

《治眼方》治目痒痛方∶黄连(半两)大枣(一枚)凡二物,以水五合,煎取一合,绵缠簪纳煎中,敷目。日十过。

《集验方》治目中卒痒痛方∶削干姜令圆滑,纳目中,有顷复纳之,辛竭者更易。

《葛氏方》治风目常苦痒泪出方∶以盐注中。
治目赤烂方第二十五

《病源论》云∶目赤烂候。风热伤于目,则赤烂。其风热不去,故常烂赤,积年不瘥。

《葛氏方》治数十岁铁眼烂方∶摘胡叶中心一把,着中,以五升水煮之,小覆上,穿作孔目,临孔上痒痛,当饭顷泪出一二升,便瘥。

《千金方》治风眼烂方∶竹叶(四分)柏树白皮(六分)黄连(四分)上,并切,以水二升,煎取五六合,稍用滴眼又方∶三指撮盐置钱上,炭烧赤,投少醋中足淹钱,以绵沾汁注中。

《录验方》治烂神验方∶黄连干姜雄黄凡三物,分等为散,着,日二。

《治眼方》眼铁烂赤方∶淳苦酒(一升)大钱(二七枚)凡二物,烧钱令赤,投苦酒中,以着铜器中,覆头。着屋北阴地埋二十一日,出爆干,可丸又云∶治目风烂赤眵恒湿神明膏方∶蜀椒(一升半)吴茱萸(半升)术(五合)芎(五合)当归(五合)附子(十五枚,去皮)十物,咀,渍着苦酒中一宿,明旦内药膏中,微火上煎之,三上三下之。留定之冷乃上也痛向《医门方》疗目赤痛如刺,不得开,肝实热所致或障翳方∶苦竹沥(五合)黄连(二分)捣碎薄绵裹纳竹沥中一宿许。卧以沥点眼中,日数度,泪下即瘥。涩痛加大枣五颗。忌热面
治目泪出方第二十六

《病源论》云∶目为肝之外候,若风邪伤肝,肝气不足,故令目泪出。

《眼论》云∶若眼赤痒泪出,名为风眼也。

《治眼方》治眼中风寒,赤痛泪出,乳汁煎方∶乳汁(一升)黄连(三分)干姜(一分)蕤核(二分)凡四物,以乳汁渍药一宿,明旦于微火上,煎取三合,如黍米注眼四。

又云治眼风泪出痒痛散方∶决明子(一分)黄连(一分)细辛(一分)干姜(一分)凡四物,冶下筛,以爪取如麻子,注中,日可再三。

又云∶治中风泪出方∶乌雄鸡三岁者,刀割冠取血,旦一敷,暮卧又一敷。良验。

《葛氏方》治目泪出不止方∶黄连(四两)以水二升,煮取一升,绵半两纳中曝;复纳尽汁,恒以拭目。

又云∶风目常苦痒泪出方∶以盐注中。

又方∶末黄连和乳汁敷中。

又方∶虎杖根煮汁,以洗目。

《范汪方》治目泪出不止方∶烧马矢,细末,绢筛,以少少敷中。

《食经》云∶目涕出不止方∶蒸煮百合,食止涕泣也。
治目为物所中方第二十七

《疗眼方》治眼为物所触中疼痛肿赤结热甘草汤方∶甘草(一分)黄柏(一分)苦参(一分)当归(一分)水一升二合,煎取七合,待冷洗眼,日五六夜一。

《千金方》治目为物所撞青黑方∶炙羊肉熨之,勿令甚热,无羊肉用猪肝。

《葛氏方》治目为物所中伤有热痛而暗方∶断生地肤草汁注之,冬日煮干,取汁注也。(《短剧方》同之。)又方∶以水和雀矢,以笔注之。

又方∶乳汁和胡粉注,日五。(以上《范汪方》同之。)《短剧方》治目为物所中方∶羊胆、鸡胆、鱼胆,皆可用注之。(《葛氏方》同之。)《广利方》疗眼目筑损肉出方∶生杏仁七枚,(去皮尖。)细嚼吐于掌中,反暖以绵缠箸头,点肉上,不过三四度,即瘥。
治竹木刺目方第二十八

《葛氏方》治竹木刺目不出方∶取鲍鱼头二枚,合绳贯。以人溺煮令烂,取汁灌目中,即出。

《龙门方》疗眼刺不出方∶烧甑带灰,少少服,立出。

又方∶摩好书墨,以笔注目瞳子上,出。
治稻麦芒入目方第二十九

《广利方》疗麦芒入目不出方∶煮取大麦汁,注目中,即出。

《治眼方》治稻麦芒入目中方∶取生蛴螬,以新布覆目上,以蛴螬从布上摩之,芒出着布已,效。

又云∶麦芒入目方∶破蝼蛄背,着眯上半日,则出眯物入目中也。

《龙门方》麦芒入眼方∶以甑带汁洗出。

《范汪方》稻麦芒入目方∶取麦汁注目中。

《苏敬本草注》云∶稻麦芒入目中不出者,取荷根汁注目中,即出。
治芒草沙石入目方第三十

《葛氏方》目卒芒草沙上辈眯不出方∶磨好书墨,以新笔染注瞳子上。

又方∶盐豉各少少着水中,临视之,即出。

《广济方》疗眯目方∶取少许甑带,烧作灰,水服方寸匕,立出。

《治眼方》治目中眯方∶旦起对户门再拜已言∶户门狭小,不足宿客,便愈。(《集验方》同之。)又云∶治芒物草沙辈落目中眯不出方∶以鸡肝注之。

又方∶吞蚕矢一枚,良。

又,灸足中指节上,随目左右。(以上《千金方》同之。)《千金方》治目中眯方∶书中白鱼,以乳汁和,注之。

《医门方》疗眯目不出视不见方∶以酥纳鼻孔中,随左右垂头,淋前令流入眼,眼中泪出,眯逐出,无酥,猪脂亦佳。

《范汪方》治目眯不去生淫肤方∶瞿麦干姜凡二物,分等为散,以井花水服一刀圭,日三。
治鼻塞涕出方第三十一

《病源论》云∶夫津液涕唾,得热则干燥,得冷则流溢,不能自收。肺气通于鼻,其脏有冷《极要方》疗鼻不闻香臭方∶细辛,瓜蒂分等为末,以吹鼻中,须臾大涕出,恒能久自通。(今按∶《经心方》∶瓜蒂二《效验方》治鼻中不利干姜散方∶干姜(二分)桂心(一分)凡二物,合下筛,取如大豆许。以绵裹塞鼻中,药行热物,便去之。

《录验方》治鼻塞不得喘息皂荚散方∶皂荚(五分)菖蒲根(五分)凡二物,下筛,以绵裹塞鼻孔中,暮卧时着,良。

又云∶治鼻孔偏塞中,有脓血,此乃是头风所作,兼由蔽疾。宜服此散方∶天雄(八分,炮)干姜(五分)薯蓣(六分)通草(六分)山茱萸(六分)天门冬(八分)凡六物,冶下筛为散,酒服方寸匕,日再。

又云∶治鼻(乌贡反)有息肉,及中风有浊浓汁出,细辛散∶文姜(四分)细辛(五分)皂荚(二分)椒(四分)附子(二分)凡五物,下筛以绵裹,如杏仁大,着鼻孔中,日一,五日浊脓尽。

《千金方》治鼻窒气息不通方∶水三升,煮小蓟一把,取一升,服之又方∶绵裹瓜丁末,塞鼻。

《范汪方》治鼻中多清涕方∶细辛(二分)椒(二分)干姜(二分)皂荚(一分)桂心(二分)凡五物,冶筛,和以青羊脂,裹以帛,塞鼻中,良。

《医门方》疗久鼻塞清涕出不止方∶附子(六分)细辛蜀椒(各八分)杏仁(四分)细切,以苦酒淹一宿,以成炼猪脂一升,微微煎之,三上三下,附子色黄,去滓,以绵裹纳
治鼻中息肉方第三十二

《病源论》云∶肺气通于鼻。肺藏为风冷所乘,则鼻气不和,津液壅塞,而为鼻。冷搏于《范汪方》治鼻中息肉通草散方∶通草(半两)矾石〔一两(熬)〕真珠(一铢)凡三物,合冶下筛,展绵如枣核,取药如小豆,着绵头,纳鼻中,日再。(今按∶《集验方《千金方》治鼻息肉方∶矾石末,以面脂和,绵缠着鼻中,数日息肉随药出。

又方∶灸上星穴二百壮。又上星相去三寸,各百壮。

《葛氏方》治鼻中生息肉不通利方∶矾石,胡粉,分等末,以青羊脂和涂肉上,数佳。

又方∶末陈瓜蒂,注息肉。

《博济安众方》疗鼻塞息肉不通方∶上,以细辛末少许,吹入鼻中,自通。

《效验方》治鼻内肉方∶胡麻,成炼矾石等分末,以针刺息肉令破,以末敷之,日二。以瘥为限。
治鼻中生疮方第三十三

《病源论》云∶鼻是肺之候也,肺气通于鼻,其脏有热,气冲于鼻,故生疮也。

《千金方》治鼻中生疮方∶烧祀灶饭,末,涂鼻中。

又方∶烧故马鞍,末,敷之。

又方∶捣杏仁,和乳敷之。

又方∶马牛耳垢,敷之。

又云∶治蚶(呼该反)虫食鼻方∶烧铜箸,纳酢中,涂之。
治鼻痛方第三十四

《千金方》治鼻痛方∶恒以油涂鼻内外。

又方∶涂酥亦佳。
治鼻中燥方第三十五

《耆婆方》治人热风鼻中燥脑中方∶杏仁一小升(去皮炙),苏二升,纳杏仁于苏中煎之,杏仁黄,沥出之,纳臼中捣作末,还纳
治鼻衄方第三十六

《病源论》云∶肺开窍于鼻,热乘于肺,则气亦热也。血气俱热,血随气发出于鼻,为鼻衄《医门方》云∶上实下虚,其人必衄,衄发从春至夏,为大阳衄;从秋至冬为阳明衄。

《短剧方》治鼻衄血出数斗,眩(胡蠲反)冒,剧者不知人方∶干姜屑,龙骨末,吹之即止。

又方∶取乱发五两烧之,冶末,取如枣核着筒头,吹着鼻孔中。不止,益末吹之。并服方寸又云∶治鼻衄积年,夜卧起而肩头有凝血数升,众治不瘥方∶舂叶绞取汁,日饮三升,不过四五饮愈,神良。

《千金方》治鼻血出不止方∶冷水净漱口,含水,以苇管中吹二孔中,即止。

又方∶葱白一把,切,捣,绞取汁,沥鼻中三两滴,入即止。

又方∶地黄汁五合,酒一合,煮取四合,空腹服之。

禁酒炙肉。旦旦服粳米饮。

又方∶湿布敷胸上。

又方∶灸上星穴五十壮,在当鼻入发际一寸。

又方∶灸涌泉二穴各百壮,在足心陷者中。

《葛氏方》治鼻卒衄方∶苦酒渍绵,塞鼻孔。

又方∶釜底黑末,以吹纳鼻中。

又方∶水和粉如粥状,以墨和,服之多少在意,立愈。

又方∶以绵裹白马矢塞鼻。杂文马矢悉可用,若大甚者绞马矢汁,饮一二升。可用干者绞取又云∶大衄口耳皆血出不止方∶蒲黄五合,以水一升和,一顿服。

又方∶铧以柱鼻下。

又方∶熬盐三指撮,以酒服之,不止,更服也。

《极要方》疗鼻衄出数升,令人眩冒,剧者不知人方∶桂心(三两)干姜(一两)乱发灰(一两)上,为散,先食浆水,粥服方寸匕,日二。

《博济安众方》疗鼻衄不止方∶上,以糯米二合细研,以冷水顿服。

《广济方》治鼻衄出血不止方∶新汲水淋头顶上六七斗,并将浸脚立效。

又方∶童子小便三四灌入鼻中,立效。

又方∶干姜削令头尖,微煨,塞鼻中,立效。

《范汪方》卒衄出不止方∶书额上作“由”字。

又方∶浓融胶,胶额,胶燥血断已,用良。

又云∶热病鼻衄多者出血一二斛方∶蒲黄五合,水五升和,饮一顿尽,即愈。

《医门方》治鼻衄血出不止方∶生地黄汁服一升,须臾二三服,兼以冷水淋顶上,立愈。

《如意方》治鼻衄术∶取衄血以书其人额云“今某日,血忌”字,即止。当随今日甲乙也《广利方》疗鼻衄出血不止方∶浓研经墨点鼻中,立效。

《龙门方》疗鼻出血不止方∶捣刺蓟汁饮一升,验。

又方∶灸头顶上七壮。
治鼻中物入方第三十七

《千金方》治卒食物从鼻中落入头中,介介痛不出方∶牛脂若羊脂,如大豆大,纳鼻孔中。以手取脂,须臾脂消,则物逐脂俱出。(今按∶无牛羊
治紧唇生疮方第三十八

《病源论》云∶脾胃有热,气发于唇,则唇生疮。而重被风耶寒湿之气搏于疮,则微肿湿烂《集验方》治沈唇方∶烧矾石令沸,杂胡粉以敷之。

《千金方》治紧唇方∶腊贴一宿,瘥。

又方∶炙松脂粘贴,瘥。

又方∶灸虎口,男左女右,七壮。

又方∶先灸疮后,取蛇灰敷之,大验。

又方∶烧乱发、蜂房、六畜毛作灰,猪脂和,敷之。

又云∶治唇边生疮连年不瘥方∶取八月蓝叶十斤,绞取汁,洗之,日三。

《葛氏方》治审唇常疮烂方∶烧葵根敷之。

又方∶头垢敷之。

又方∶东壁土敷之。

《新录方》治审唇方∶荷汁和酒洗,日二三。

又方∶马苋捣汁洗之,日三。

又方∶槟榔KT灰敷上。

又方∶榆根白皮粘贴。

《龙门方》疗紧唇方∶取地黄叶于坏瓦器中捣之使烂,待干,刮取末,涂验。

《短剧方》治紧唇方∶俗谚言∶良方善伎,出于阿氏。是余少时,触风乘马行猎,数苦紧唇。人教缠白布作大灯,
治唇生核方第三十九

《病源论》云∶有风热邪气乘之,而冲发于唇,与血气相搏,则肿结。外为风冷乘,其结肿《葛氏方》治唇紧重忽生丸核稍大方∶以刀锋决去其脓血,即愈。

《千金方》治唇紧生核方∶取猪矢平量一升,以水绞取汁,温服。

又云∶唇舌忽生臼方∶烧鸡矢白作屑,以布裹,着病上,含,日三。
治唇黑肿硬方第四十

《千金方》治唇黑肿痛痒不可忍方∶取四文大钱,于磨石上,以腊月猪脂磨,取汁,涂之。

又方∶以竹弓弹之,出其恶血,亦瘥。

《医门方》疗人口唇皮黑,坚硬作痂,皮裂时血出,恒痛唇皮起落复生,历年不瘥方∶上,以山中黄泥和水,研令细熟以涂唇上,当有毛出,抽取烧之。又涂毛尽,瘥。其毛千得
治唇破方第四十一

《葛氏方》冬月唇干血出者方∶熬桃仁,捣猪脂和涂之。(《千金方》同之。)又云∶唇卒有伤缺破败处者方∶刀锋细割开,取新杀獐鹿肉,以锉补之。患兔缺又然,禁大语笑,百日。
治唇面KT方第四十二

《千金方》治远行唇口面KT裂方熟煎猪脂。将行夜,常涂面卧,行万里,野宿,不损。

《本草》云∶涂酥良。
治口舌生疮方第四十三

《病源论》云∶手少阴,心之经也,心气通于舌。足太阴,脾之经也,脾气通于口。腑脏热《葛氏方》治喉口中及舌生疮烂方∶含好淳苦酒即愈。

又方∶锉黄柏,恒含之。

又云∶若口表里皆有疮者方∶捣白荷根,酒渍含汁。

《录验方》治口中十二病,或肿;或有脓血;或如饭粒,青白黑起;或如鼠乳;或有根下断甘草桂心生姜细辛(各一两)凡四物,淳苦酒三升,煮取一升,适寒温含之。

《千金方》治口热生疮方∶升麻(六分)黄连(二分)上二味,筛,绵裹,含咽汁,亦可唾去之。

又云∶口中疮久不瘥,入胸中,并生疮三年以上不瘥方∶浓煮蔷薇根汁,冷,稍稍咽之。冬用根,夏用茎叶。

论云∶凡患口齿有疮,禁油、面、酒、酱、咸、酸、腻、干枣。瘥后七日慎弥佳。蔷薇根为入生蜜二合,旋旋含之吐之。)又云∶舌上疮方∶猪膏(一斤)蜜(二升)甘草(如指三寸)上三味,咀合煎,相得;含枣大,稍稍咽之,日三。

《经心方》治口疮久不瘥方∶枣膏三斤,以水三斗,煮取一斗五升。数洗愈。

《随时应验方》口疮方∶干姜火炙,口中含,吐热水尽,即瘥。

《龙花妙方》口疮方∶含矾石,吐去水,良。(今按∶《博济安众方》∶以白矾锻石涂之。)又方∶以井水,日三漱,弥好。

《崔禹锡食经》口疮方∶食石良。

《博济安众方》疗口疮舌硬语不得方∶白矾石(一分)桂心(一分)上为末,安舌上即语。

《范汪方》治人口生疮方∶杏子(一枚)黄连(一节)甘草(一寸)(今按∶《本草》∶甘草一尺者重二两为正,仍一凡三物,冶下筛,绵絮裹之,纳着口中含之,含汁稍咽之,已用验。

《集验方》治口中生疮方∶取黄柏削去皮,作如鸭舌含之,咽汁,弥好。蜜渍含亦佳。

《效验方》治口烂疮无复皮方∶黄连(三分)附子(一分)榆皮(三分)凡三物,冶筛,和蜜,绵裹如杏子和之;药味尽吐出,更含。
治口舌出血方第四十四

《病源论》云∶心主血脉而候于舌,若心藏有热,则舌上血出如涌泉。

《葛氏方》治口中忽出血不止方∶灸额上入发际一寸五十壮,便愈。

又云∶舌上出血如簪孔者方∶以戎盐敷之。

《千金方》治舌上出血,有四五孔,大如簪者,血出如涌泉,此心病也。治之方∶戎盐(五分)黄柏(五分)葵子(五分)人参(三分)桂肉(二分)大黄(二分)甘草(二分)炙上七味,丸如小豆,饮服三丸,日二,不知增至十丸。

《经心方》∶治舌上孔血出如泉,此心病也。烧铁熟烁孔中,良。
治九窍四肢出血方第四十五

《病源论》云∶九窍四肢下血者,营卫大虚,腑脏伤损,血脉空竭。因而喜怒失节,惊忿不《葛氏方》云∶人九窍四肢指岐间皆血出,此暴惊所致也。以井花水其面,当令卒至。

勿又方∶粉一升,水和如粥,饮之。

《千金方》九窍出血方∶又捣荆叶取汁,酒服二合。

又方∶灸上星穴五十壮。

又方∶龙骨末,吹鼻孔中,血断为度。

《范汪方》卒惊动七孔皆血出方∶盗以井花水洒其面,勿使知也。
治呕血方第四十六

《病源论》云∶呕血者。夫心主血,肝者藏血,愁忧思虑则伤心,恚怒气逆,上而不下则伤《千金方》治呕血方∶柏叶一斤,以水六升,煮取三升,分三服。

《葛氏方》治卒呕血,腹内绞急,胸中隐然痛,血色紫黑或从溺中出方∶灸脐左右各五分,四壮。(《集验方》同之。)又方∶末桂一尺,羊角一枚,炙焦捣末。分等合,服方寸匕,日三四。
治吐血方第四十七

《病源论》云∶夫吐血者,皆又大虚损及饮酒,劳伤所致也。

《医门方》经曰∶凡诸吐血呕血人,若兼右气喘咳不得卧者,多死难疗。

《葛氏方》治卒吐血方∶服蒲黄一升。

又方∶浓煮鸡苏饮汁。亦治下血漏血良。

《千金方》治吐血方∶服桂心末方寸匕,日夜可二十服。

又方∶烧乱发灰,水服方寸匕,日三。

又方∶熟艾三鸡子许,水五升,煮取二升,顿服。

又方∶灸胃脘三百壮。

又方∶灸胸堂穴五壮。

《短剧方》治吐血方∶用东向荷根,捣绞取汁一二升,顿服立愈。亦治蛄毒痔血,妇人腰腹痛,大起后出清血也《录验方》治血生姜汤方∶生姜(五两)人参(二两)甘草(三两)大枣(十枚)凡四物,咀,以水三升,煮取一升半,分再服。

《范汪方》治吐血下血不止方∶生地黄一升,咀,清酒五升,微火上合煎得二升半,去滓,强人顿服,老少分再服。

《令李方》治吐血便血方∶干地黄黄芩(各二两)凡二物,冶下筛,酒服方寸匕,日三。

《僧深方》治吐血方∶龙骨多少冶,温酒服方寸匕,日五六,可至二三匕,亦治小便血。

《医门方》疗吐血单神方∶生地黄汁一升二合,白胶一两,以铜器盛蒸之,令消,顿服之,三服,必瘥神效。

《龙门方》疗卒吐血不止方∶取灶底黄土一斤,以水一大升三合,研澄饮之。

《广利方》疗吐血不止方∶刺蓟菜及根捣汁半升,顿服之。

又方∶生葛根捣汁半大升,顿服之。(《僧深方》云∶治吐血欲死。)
治唾血方第四十八

《病源论》云∶唾血者,伤损肺所为。肺为五脏盖,易为伤损,若为热气所加则唾血。

如红《耆婆方》治人唾血及水涎不能食方∶干地黄人参蒲黄等分为散,以饮服一钱匕,日二,腹痛者加夕药八分。

又方∶生大豆五小升,以水二小斗,煎取二升豆汁,纳一小升酒,煎取一升,分为二服,三《僧深方》治唾血方∶干地黄(五两)桂心(一分)细辛(一分)干姜(一分)凡四物,散,酒服方寸匕,日三夜再。

《录验方》治唾血中有脓血牵胸胁痛方∶干地黄(五两)桔梗(三两)紫菀(三两)竹茹(三两)五味(三两)赤小豆(一升)续断凡九物,切,以水九升,煮取二升七合,分三服。

《葛氏方》治卒唾血方∶取茅根捣,服方寸匕。亦可绞取其汁,稍稍饮之,勿使顿多。(《极要方》同之。)又方∶服桂屑方寸匕,日夜令二十许服。亦治下血神方。

《千金方》治唾血方∶灸胃脘穴三百壮。

又方∶灸胸堂穴,肺俞。
治口中烂痛方第四十九

《千金方》治口中疮烂痛不得食方∶杏仁(二十枚)甘草(一寸)黄连(一分)上三味,下筛合和,绵裹杏仁许大,含勿咽,日三夜一。

《范汪方》治口中烂伤喉咽不利方∶矾石(二两)黄连(一分)冶筛,如大豆二枚,置口中含疮上,小儿疮石如小豆,日三。
治口吻疮方第五十

《葛氏方》治吻疮方∶烧栗敷之。

《千金方》治口吻疮方∶楸白皮及湿粘贴,三四度。

又方∶掘经年葵根欲腐者,作灰,及热着之。

又方∶取新炊甑下饭及热驻之二七下。

又云∶治口肥疮方∶熬灶上饭,末敷之。

《极要方》口吻白疮方∶烧槟榔KT为灰,敷上,良。
治口舌干焦方第五十一

《病源论》云∶腑脏虚热,气乘心脾,津液竭燥,故令口舌干焦也。

《葛氏方》治口中热干燥方∶乌梅、枣膏分等,以蜜和丸如枣,含之。

又方∶生姜汁(一合)甘草(二分)杏仁末(二分)枣(三十枚)蜜(五合)微火上煎,丸如李核,含一枚,日四。

《千金方》治口中热干方∶甘草(四分)人参(四分)半夏(三分)乌梅肉(四分)枣膏(四分)上五味,蜜和丸如弹丸,含咽汁,日三。

又云∶治虚劳口干方∶麦门冬(二两,末)干枣(三十枚,肉)蜜一升和,蒸之三升米下,服之。

《经心方》治口干方∶以水三升,煮石膏末五合,取二升,纳蜜二升,煎取二升,去滓,含枣核大,咽汁尽,复含又方∶生葛根汁服二升亦瘥。

《苏敬本草注》∶口干食软熟柿也。
治口臭方第五十二

《病源论》云∶口臭者,由五脏六腑不调,气上胸膈。然腑脏之气臊腐,因蕴积胸膈之间。

《养生方》云∶空腹不用见臭,尸气入脾,舌上白黑起,口常臭也。

《隋炀帝后宫诸香药方》疗口臭方∶桂心甘草细辛上三物,分等,捣筛,服三指撮,酒服之二十日,便瘥。

《录验方》治口中臭令还香方∶细辛(三分)当归(三分)桂心(一两)甘草(二两)凡四物,切,以水一升,煎取四合,含之,吐去之。

《千金方》治口臭方∶橘皮(五分)桂心(三分)木兰(四分)枣(三十枚)上四味,冶小筛,酒服方寸匕,日三。久服身香,亦可丸,含之日三。

又方∶熬大豆令焦及热泼酢取汁含之。

又方∶香薷一把,以水一升,煎取二升,稍含之。

《经心方》治口臭方∶水浓煮细辛汁,含久吐去。

又方∶以破除日井花水三升漱口,吐厕中,良。又平旦如此。

《极要方》治口臭方∶常宜含豆KT即愈。

又方∶常含细辛即愈。

《葛氏方》治口臭方∶蜀椒一升,桂心一尺,末三指撮,以酒服之。

又方∶煎柏子,含之。

《崔禹锡食经》云∶口臭方∶取薰蕖根汁果煎如膏,常食之。

又方∶亦可食怀()香。

又方∶末槟榔子含之。

《集验方》治口中臭散方∶甘草(五两)芎〔()四两〕白芷(三两)凡三物,冶下筛,酒服方寸匕,日三。
治张口不合方第五十三

《葛氏方》治卒失欠颔车蹉张口不得还方∶令人两手牵其颐已,暂推之,急出大指,或咋伤也。

《千金方》治张口不合方∶消蜡(蜜,又蜜滓也)和水,敷之。

《德贞常方》治张口不合方∶灸通谷穴,在上管旁半寸。

又方∶灸翳风穴,在耳后陷中,按之引耳中也。
治舌肿强方第五十四

《病源论》云∶心脾虚,为风热所乘,随脉至舌,热气留止,血气壅涩,故舌肿。

《千金方》治舌肿满口方∶半夏十二枚。

洗酢一升,煮取八合,含久,吐良。

又云∶治舌卒肿,起如吹猪胞状满口塞喉方∶急以指刮破舌两边,去汁即愈。亦可以刀锋决又方∶以指撞溃去汁,亦可以刀破以疮膏涂之。

又方∶釜底黑末和酢,浓涂舌上下,脱去更涂,须臾即消。若先决出汁,竟与弥佳,疮膏涂又方∶此患人皆不识,或错治益困,杀人甚急。但看其舌下,自有噤(渠金反)虫形状,或如蝼蛄。或似卧蚕子,细审看有头尾,其头少白,烧铁钉烙头上,使熟即自销。

《葛氏方》治舌卒肿起如吹猪胞状,满口塞喉,气息欲不复通,须臾不治则杀人方∶直以指出血愈者又方∶浓煮甘草汤,含少时,取釜底墨,苦酒和,浓涂舌上下,脱去更涂,须臾便消,若先《枕中方》治人肿舌方∶取水底石三七枚,熨之。
治重舌方第五十五

《病源论》云∶舌者,心之候也。脾之脉,起于足大指,入连于舌本。心脾有热,热气随脉《葛氏方》治卒重舌方∶末赤小豆,以苦酒涂和舌上。

又方∶乌贼鱼骨,蒲黄分等末,敷舌上。

又方∶灸两足外踝各三壮。

《千金方》治重舌方∶取蒲黄敷舌下,日三四。

又方∶灶中黄土,酒和,敷上。

《僧深方》治重舌方∶烧露蜂房,淳酒和敷喉下,立愈有验。

《集验方》治重舌方∶以铍针刺舌下肿者,令血出。愈。勿刺大脉也。
治悬雍卒长方第五十六

《病源论》云∶五脏六腑有伏热,上冲喉咽,热气乘于悬雍,或长或肿。

《葛氏方》治悬雍卒长数寸,随喉出入,不得食方∶捣盐,绵缠箸头注盐,就以揩之,日六又方∶开口以箸仰舌,乃烧小铁于管中,灼之,若不顿为者可三为之,毕,以盐涂烧灼处。

《千金方》悬膺咽中生息肉舌肿方∶羊蹄草,煮取汁,口含。

又方∶以盐豉和,涂之。

《范汪方》悬膺卒长方∶釜底墨以酒和涂舌上下,即愈。

又方∶以盐注其头,即自缩。
治风齿痛方第五十七

《病源论》云∶手阳明之支脉入于齿,齿是骨之所终,髓之所养。若风冷客于经络,伤于骨《葛氏方》治风齿疼痛颊肿方∶酒煮独活令浓及热,含之。

《范汪方》治风齿痛,根空肿痛引耳颊,昼呼夜啼,无聊赖方∶独活四两,切,以清酒三升《医门方》疗齿楚痛嚼食不得者方∶以生地黄,桂二味相和合,嚼之咽汁,瘥。

《录验方》治风齿痛方∶当归(三两)独活(一两)上二物,细切,绢囊盛,清酒五升渍三日,稍含渍齿,久久吐去,更含,日四五次。

《耆婆方》治风齿疼痛不可忍验方∶独活(一两)细辛(二分)椒(一勺)当归(一分)四味,以好酒大升半,微火煮令减半,稍稍含之吐出,更含,以瘥为度。
治龋齿痛方第五十八

《病源论》云∶手阳明之脉,入于齿。足太阳脉,有入于颊,遍于齿者。其经虚,风气客之亦曰风《养生方》云∶朝夕琢齿,齿不龋。

又云∶食必当漱口数过,不尔,使人病齿龋之。

《删繁方》云∶治齿龋方∶蜀椒(一两)矾石(半两)桂心〔一两(一方分等)〕凡三物,以水三升,煮取一升五合,细细漱口吐之。(今按∶《范汪方》以细辛代蜀椒。)《葛氏方》治龋齿方∶灸足外踝右三寸,随齿痛左右七壮。

又方∶鸡舌香置虫齿上,咋之。

又方∶取李枝,削取里白皮一把,以少水煮十沸。小冷含之。不过三,当吐虫长六七分,皆又方∶作竹针一枚,东向以钉柱,先咒曰∶冬多风寒,夏多暖暑,某甲病龋,七星北斗光鼓可《千金方》治龋齿方∶大酢一升,煮枸杞白皮一升,取半升含,虫出。

又方∶白杨叶,切,一升,水三升,煮取一升,含。

又云∶齿有孔,得食面肿方∶莽草七叶猪椒附根皮(长四寸者七枚)凡二味,以浆二升,煮得一升,适寒温,含满口,吐去《范汪方》治龋齿方∶有孔取细铁大小如孔中也,曲铁头,火烧令热,以纳孔中,不过四五《录验方》治龋齿方∶取丝杨柳细枝,除上皮取青皮,卷如梅李大;含嚼之含汁,渍齿根。

《医门方》治龋齿方∶以松脂捻令头尖,注孔中,虫当出也。

又方∶嚼薰陆香咽汁,立瘥。

《短剧方》治甘虫食齿根方∶伏龙肝置石上,着一撮盐须臾化为水,以展取,待凝浓,取又方∶皂荚去皮涂上,虫出。

又云∶治齿龋方∶腐棘针(二百枚,是枣树刺自朽落地者)凡一物,咀,以水二升,煎得一升,含漱出即愈。
治齿碎坏方第五十九

《葛氏方》治人病齿稍碎坏欲尽方恒能以绵裹矾石衔咋之,咽吐其汁。
治齿令坚方第六十

《录验方》云∶欲令齿坚方∶可用矾石,细辛分等煮,以漱口中。

又方∶煮枸杞根漱口,良。
治齿动欲脱方第六十一

《病源论》云∶经脉虚,风邪乘之,血气不能荣润,故令动摇。

《千金方》治齿根动欲脱方∶生地黄(二两)独活(三两)酒一升,渍一宿,含之。

《葛氏方》治齿根动欲脱方∶生地黄根,绵裹着齿上,咀,以汁渍齿根,日四五为之。能十日为之,长不复动。(今按∶《范汪方》∶取生地黄根肥大者一节,咽其汁,日三,其日即愈。可数十年不复发。)
治齿黄黑方第六十二

《病源论》云∶齿是骨之所终,髓之所养也。手阳明,足太阳之脉,皆入于齿。风邪冷气,《新录方》治齿黄黑方∶取桑黄皮,酢渍一宿,洗七遍。一云黄白皮,此方正月亦及五月五
治齿败臭方第六十三

《葛氏方》治风齿齿败口气臭方∶斑草细辛白芷当归独活(分等)水煎煮含之,吐去汁。(今按《广济方》∶芎二两,当归二两,独活四两,细辛四两,白
治齿龈肿方第六十四

《病源论》云∶头面有风,风气流入于阳明之脉,与龈间血气相搏,故成肿也。

《养生方》云∶水银不得近牙齿,发龈肿,喜落齿。

《经心方》云∶治齿根肿方松叶一虎口,盐一合,以好酒三升,煎取四五合,含之。

《食经》云∶齿根肿方∶郁根煮含之。

《极要方》疗牙齿动落,齿根宣露;或齿龈下血出;或龈肿,吃冷热物疼痛;或齿根下黑。

至明欲浆平坐日别
治齿龈间血出方第六十五

《病源论》云∶手阳明之支脉入于齿,头面风,而阳明脉虚,风挟热乘虚入齿龈,搏于血,《葛氏方》治齿间津液血出不止方∶矾石一两,以水三升,煮取一升,先拭血,乃含之漱吐。

《千金方》治齿龈间血出不止方∶生竹茹四两,酢渍一宿,含之。

又方∶细辛二两,甘草一两,酢二升,煮取一升,含之。

《经心方》齿龈间出血方∶取茗(茶也)草浓煮汁,勿与盐,适寒温,含漱竟,日为之,验。
治牙齿痛方第六十六

《病源论》云∶齿牙痛者,是牙齿相引痛。牙齿是骨之所终,髓之所养。若髓气不足,阳明《极要方》疗牙疼方∶鸡舌香(三分)斑草(三分)细辛(三分)附子(三分,炮)炮椒(二分,汗)独活(三分)上,捣五百杵,勿筛,以绢袋盛如枣核大,安牙疼处,涎满口吐之。含二袋终身不发,神验又方∶皂荚(一梃,肥浓者,剥去皮)上,以验酢半升,煎令极,调以桃枝如箸一枚,以绵裹头,承药沾之。当牙疼痛处灼之,冷《医门方》治牙疼痛方∶取杨白皮,切,以醋煎,含之即瘥。

又方∶取HT宕子,烧,以碗覆碗上,令碗底作孔,安竹筒,当牙熏,良。

《广济方》治牙痛方∶取槐白皮,切,一握,酢三升,煎取二升,去滓,纳盐,适寒温,含《集验方》治牙痛方∶取枯竹烧一头,以挂铁上,得汁着齿上,即瘥。

《德贞常方》牙疼方∶灸浮白穴,在耳后入发际一寸。

《博济安众方》疗热毒风攻牙齿疼痛方∶附子(一个,烧灰)白矾锻石(一分)上二味,为散,揩齿立瘥,极妙。

又云∶牙有孔疼痛方∶附子一个为末,以蜡为丸,纳孔中。

《龙门方》疗牙疼方∶湿柳枝每旦揩齿。不过三日,疼及口臭者亦瘥。
治牙齿后涌血方第六十七

《短剧方》治有饮酒醉,牙后涌血,射出不能禁者方∶取小钉烧令赤,正注血孔上一注,即
治齿方第六十八

《病源论》云∶齿者,骨之所终,髓之所养,髓弱骨虚,风气客之,则齿。

《养生要集》云∶治食酸果齿方∶含白蜜嚼之,立愈。

《千金方》治齿龈痛不可食生果方∶生地黄桂心上二味,合嚼,令味相得,咽之。
治齿方第六十九

《病源论》云∶齿者,是睡眠而相切也。此由血气虚,风邪客于牙车筋脉之间,故因眠气《录验方》治齿方∶是睡眠而相切有声也。令人取其卧席下土纳其口中,勿知之。
治喉痹方第七十

《病源论》云∶喉痹者,喉里肿塞痹痛,水浆不得入也。风毒客于喉间,气结蕴积而生热,《葛氏方》喉痹水浆不得入七八日则杀人,治之方∶随病所近左右,以刀锋裁刺手父指爪甲后半分中,令血出即愈。(今按∶《龙门方》云,以又方∶随病左右,刺手小指爪甲下,令出血,立愈。当先将缚,令向聚血,乃刺之。

又方∶熬杏仁,蜜丸如弹丸,含咽汁。

又方∶夜干三两,当归三两,水三升,煮取一升。稍稍含之,吐去,更含之。

又方∶菘子若芥子,捣苦酒和以薄。

《短剧方》治喉痹卒不得语方∶浓煮桂汁服一升,覆取汗。亦可末桂着舌下,大良。

又方∶取炊甑箅烧作屑,三指撮,少水服之,即效。

《经心方》治喉痹方∶生姜二斤,舂取汁。蜜五合,微火煎相得服一合,日五服。

又方∶浓煮大豆汁,含之,无豆煮豉亦良。

又方∶剥蒜塞耳鼻,日二易,有验。

《集验方》治喉痹方∶咀常陆根,苦酒熬令热,以敷喉上,冷复易。

《龙门方》治喉痹方∶取胡燕窠末水和,服,验。

《录验方》治喉痹方∶取芥子一升,舂碎,水和以敷喉下,干复易。

《千金方》治喉痹方∶末桂心枣核大,绵裹着舌下,须臾定。

又方∶煮大豆汁含之。(今按∶《短剧方》∶无豆煮豉亦良之。)《极要方》疗喉痹方∶马蔺子四十九枚,捣作末,和水服之,立愈。无子取根一大握,捣绞取汁,细细咽。今按∶《广利方》∶取根汁二大合,和蜜一匙含之。

《医门方》治喉痹方∶生艾叶,熟捣以敷肿处,随手即消,神验。无此,冬月以干艾水捣敷之。

《新录方》治喉痹方∶煮夜干,含其汁,吐出。

《僧深方》治卒喉痹咳痛不得咽唾方捣茱萸敷之,良。

《范汪方》治喉痹方∶烧秤锤令赤,着一杯酒沸,上出锤,适寒温尽饮之。

又方∶杏仁三分,熬,桂二分。合末着谷囊中,含之稍咽其汁。(今按∶《极要方》∶蜜丸《博济安众方》疗喉痹方∶生牛蒡研涂喉上。

又方∶生研糯米,入蜜服。又炒为末,贴喉上。

《崔禹锡食经》云∶喉痹方∶食虚羸子肠殊效。

又方∶食骨蓬甚良。
治马喉痹方第七十一

《病源论》云∶马喉痹者,谓热毒之气结于喉间,肿连颊而微壮热,烦满而数吐气,呼之为《千金方》曰∶喉痹深肿连颊,吐气数者,名马喉痹。治之方∶马衔一具,以水三升,煮取一升,分三服。

又方∶马鞭草根一握,勿中风,截去两头,捣取汁服。

《龙门方》治马痹方∶取马蔺草根,净洗,烧作灰一匙,烧枣枝取沥汁,和灰搅饮立瘥。
治喉咽肿痛方第七十二

《病源论》云∶脾胃有热,气上冲,则喉咽肿痛。

《葛氏方》治喉卒痈肿食饮不通方∶用韭一把,捣熬以敷肿上,冷复换之,苦酒和之亦佳。

又方∶吞薏苡子二枚。

又方∶烧荆(和)木,取其汁稍咽含之。

又方∶烧秤锤令赤,纳二升苦酒中,沸止,取饮之。

《僧深方》治喉咽卒肿痛,咽唾不得,消热下气升麻含丸方∶生夜干汁(六合)当归(一两)升麻(一两)甘草(三分)凡四物,下筛,以夜干汁丸之,绵裹如弹丸,含稍咽其汁,日三夜一。

《千金方》治喉卒肿食饮不通方∶含上好酢,口舌疮亦佳。

又方∶熬杏仁令黑,末服。

《录验方》五香汤治诸恶气喉肿结核方∶沉香(一两)熏陆香(一两)麝香(二分)青木香(二两)鸡舌香(三两)凡以水五升,煮取一升半,分三服。
治尸咽方第七十三

《病源论》云∶尸咽者,谓腹内尸虫,上食人喉咽生疮。其状,或痒或痛,如甘餐之候也。

《救急单验方》疗尸咽方∶捣干姜和盐末注,立瘥。

《龙门方》疗尸咽方∶灸两乳中间,随年壮,验。

又方∶张口吸烟,唾盆中水,见虫瘥。
治咽中如肉脔方第七十四

《医门方》疗咽中如肉脔咽不入吐不出方∶半夏生姜茯苓(各四两)浓朴(三两,炙)橘皮(二两)水七升,煮取二升半,去滓,分温三服,服相去八九里,不过两剂必瘥。
卷第六
治胸痛方第一

《病源论》云∶胸胁痛者,由胆(都敢反)与肝及肾(时忍反)之支脉虚,为寒气所乘故也。此胸胁相引《葛氏方》∶治胸痹之病,令人心中坚痞急痛,肌中苦痹,绞急如刺,不得俯仰。其胸前皮杀人橘皮(一升)枳实(四枚)生姜(半斤)水四升,煮取二升,分为再服。(今按∶《短剧方》∶枳实三两,水五升。)又云∶若已瘥复更发者∶取韭根五斤,捣绞取汁,饮之立愈。

《千金方》云∶胸痹之病,令人心中坚满痞急痛,肌中苦痹,绞急如刺,不得俯仰,其胸中短气枳实(四枚)浓朴(三两)薤白(一斤)栝蒌子(一枚)桂心(一两)五味,水七升,煮取二升半,分再服。

《录验方》治胸痛达背不得卧方∶大栝蒌实(一枚,捣)薤白(三斤,切)半夏(半升,洗)生姜(六两,)切凡四物,切,以清白浆一斗,煮取四升,一服一升。

《极要方》疗胸满气筑心腹中冷方∶半夏〔一升(洗)〕桂心(八两)生姜(一斤)上,以水七升,煮取二升七合,日三服。

《医门方》疗冷气胸中妨(害也)满或痛方∶半夏(四两)生姜(五两)浓朴(四两,炙)水六升,煮取二升,去滓,分温二服,服相去八九里,顿服二三剂。
治胁痛方第二

《病源论》云∶邪气客于足少阳之络,令人胁痛。

《葛氏方》治胁卒痛如得打方∶以绳横度两乳中间,屈绳从乳横度,以起痛胁下灸绳下屈处三十壮。

《短剧方》治胁下偏痛,发热,其脉弦,此寒也。当以温药下其寒。大黄附子汤方∶大黄(三两)附子(三枚)细辛(二两)凡三物,以水三升,煮取二升,分二服。

《千金方》治冷气胁下往来,冲胸膈(古核反,胸内也),痛引胁背闷当归汤方∶当归(二两)吴茱萸(二两)茯苓(一两)桂心(二两)干姜(三两)枳实(一两)人参(十味,以水八升,煮取二升半,一服八合,日三。治尸注亦佳。(今按∶《极要方》有干地又云∶治两胁下痛方∶热汤渍两足,冷则易。
治心痛方第三

《病源论》云∶心痛者,风冷邪气乘于心也。其病发,有死者,有不死成疹者。心为诸脏主其《葛氏方》治卒心痛方∶当力以自坐。若男子病者,令妇人以一杯水与饮之。若妇人病者,令男子以一杯水与饮之。

又方∶吴茱萸五合,桂一两,酒二升,煮取一升,分二服。

又方∶吴茱萸二升,生姜四两,豉一升,酒六升,煮取二升半,分三服。

又方∶切生姜,若干姜切半升,以水二升,煮得一升,去滓,顿服之。

又方∶取灶中热灰,筛去炭芥,燔熨心上,冷复易。

《录验方》治卒心痛方∶蒸大豆若煮之,以囊盛更燔,以熨心上,冷复易之。已熨之豆不可令畜生犬等食之,即死。

又云∶治人心痛懊(于报反)(于报反,恼也),(于缘反,忧也)闷(莫困反)筑筑(音竹),桂心(半两)茱萸(二升)夕药(三两)当归(二两)生姜(半斤)凡五物,切,水一斗二升,煮取四升,服一升,日三夜一。有验。无生姜用干姜五两代。

《备急方》治心痛方∶极咸作盐汤,饮三升,吐则即瘥。

《耆婆方》治卒心痛欲死方∶吴茱萸(三两)夕药(三两)桂心(三两)上,以淳酒大一升生煮之,令有半升在,顿服。

《僧深方》治卒心痛方∶当归(二两)夕药(一两)桂心(一两)人参(一两)栀子(二十一枚)五物,(弗禹反)咀,以水七升,煮取二升半,分服五服。

又云∶治三十年心痛附子丸方∶人参(二两)桂心(二两)干姜(二两)蜀附子(二两)巴豆(二两)凡五物,下筛,蜜丸如大豆,先食服三丸,日一。神良。

《新录方》治心痛方∶饮井花水二升。

又方∶以热汤渍手足,以瘥为度。

又方∶烧秤锤令赤,投二升酒中,分二服。

又方∶水服米粉一匙。

《集验方》治卒心痛方∶桂心(八两)以水四升,煮取一升半,分再服。

《鉴真方》治心痛方∶大验酢半升,切,葱白一茎,和煎顿服,立愈。

《范汪方》治卒心痛一物桂心散方∶桂心(一两)为散,温酒服方寸匕,日三,不饮酒,以米饮服之。

又云∶备急丸方∶巴豆(一分)大黄(二分)干姜(二分)捣筛,共冶巴豆合丸如大豆。有急,取二三丸,以水服之。

《如意方》治卒心痛术∶画地作五字,撮中央,以水一升,搅饮之。

《短剧方》解急蜀椒汤主寒疝心痛如刺,绕脐绞痛,腹中尽痛,白汗自出欲绝方∶蜀椒〔三百枚(一方二百枚)〕附子(一枚)粳米(半升)干姜(半两)大枣(三十枚)半夏凡七物,以水七升,煮取三升,汤成热服一升,不瘥复服一升。数用,治心痛最良。

《千金方》云∶九痛丸主九种心痛∶一、虫心痛;二、注心痛;三、风心痛;四、悸心痛;五、食心痛;六、饮心痛;七、冷心痛;八、热心痛;九、生来心痛。此方悉主之,并治附子(二两)巴豆仁(一两)生野狼毒(一两,炙令极香,称)人参(一两)干姜(一两)吴六味,蜜和,空腹服如梧子三丸。卒中恶腹痛,口不能言者,二日一服。连年积冷,流注心又方∶桃白皮煮汁,空腹服之。

又云,凡暴心痛,面无色欲死方∶以布裹盐如弹丸,烧令赤,置酒中消,服之。

又云∶灸心痛方∶心懊(恼也),彻痛烦逆,灸心俞百壮。

心痛如刀,刺气结∶(灸膈俞七壮。)心痛胸痹∶(灸膻中百壮。)心痛冷气上∶(灸龙头百壮,在心鸠尾头上行一十半。)心痛恶气上,胁急痛∶(灸通谷五十壮,在乳十二寸。)心痛暴绞急绝欲死∶(灸神府百壮,附心鸠尾,正心有忌。)心痛暴恶风∶(灸巨厥百壮。)心痛胸胁满∶(灸期门有年壮。)心痛坚烦气结∶(灸太仓百壮。)以上《短剧方》同之。

《救急单验方》冷心痛方∶吴茱萸(一升)桂心(三两)当归(三两)捣末蜜丸如梧子,酒服二十丸,日再,渐加三十丸,以知为度。

又云∶蛔心痛方∶取蛐(音曲)(是演反)粪,烧令赤,末之,酒服验。

又云∶一切心痛方∶生油半合,温服立愈。

又方∶服当归末方寸匕,和酒服瘥。

《孟诜食经》治心痛方∶酢研青木香服之。
治腹痛方第四

《病源论》云∶腹痛者,由腑脏虚,冷热之气客于肠胃募原之间,结聚不散,正气与邪气交《葛氏方》治卒腹痛方∶书舌上作“风”字。

又方∶捣桂下筛,服三方寸匕。苦参亦佳,干姜亦佳。

又方∶食盐一大握,多饮水送之,当吐即瘥。

又方∶掘土作小坎,以水满坎中,熟搅取汁饮之。

又方∶令人骑其腹,尿脐中之。

又方∶米粉一升,水二升,和饮之。

《如意方》治卒腹痛术∶书纸作两蜈蚣相交,吞之。(今按∶《葛氏方》同之。)《医门方》治腹痛方∶桂心三两,切,以水一升八合,煮得八合,去滓,顿服。无桂心,煮干姜亦佳。

《范汪方》云∶当归汤主寒腹痛方∶当归(三两)桂心(三两)甘草(二两)干姜(三两)凡四物,水八升,煮得三升,服一升,日三。

又云∶当归丸治寒腹痛如刀刺方∶当归(三分)夕药(三分)黄芩(四分)桂心(一分)凡四物冶筛,和蜜丸如梧子,服十丸,先食,日三。

《千金方》温脾(俾移反)汤治腹痛脐下绕结绕脐不止方∶当归(五两)干姜(五两)附子(三两)甘草(二两)大黄(五两)人参(三两)芒硝(三七味,以水七升,煮取三升,分三服,日三。

又云∶灸腹痛方∶灸巨阙穴,在鸠尾下一寸。

又方∶灸水分穴,在脐上一寸。

又方∶灸中极穴,在脐下四寸。
治心腹痛方第五

《病源论》云∶心腹痛者,由腑脏虚弱,风寒客于其间故也。邪气发作,与正气相击。

上冲于心则心痛,下攻于腹则腹痛,上下相攻。故心腹绞痛,气不得息。

《葛氏方》云∶凡心腹痛,若非中恶霍乱,则皆是宿结冷热所为也。治心腹俱胀痛,短气欲桂(三两,)切以水一升八合,煮得八合,去滓,顿服。无桂者,干姜亦佳。

又云∶若心腹痛急似中恶者方∶捣生菖蒲根汁,少少令下咽,即瘥。

《千金方》治心腹卒绞痛如刺,两胁支满,烦闷不可堪忍高良姜汤方∶高良姜(五两)原朴(二两,炙)当归(三两)桂心(二两)四味,切,以水八升八合,煮取一升八合,分为三服,一服六合,日二。若一服痛止,便停《短剧方》治寒疝心腹痛方∶夫寒疝腹中痛,逆冷,手足不仁,若一身疼痛,灸刺诸药所不治者,桂枝汤加乌头汤主之∶桂肉(三两)生姜(三两)甘草(二两)夕药(三两)大枣(十二枚)乌头(五枚,破之,凡五物,以水七升,煮取二升半,纳蜜煎,分服五合,日三。

《极要方》疗心腹久寒,卒发,绞如鬼刺,欲死。汤方∶吴茱萸(二升)香豉(三合)生姜(四两)上,以好酒五升,煮取二升,分服。服下喉,杯未去口,病便止。勿令咽水。

又云∶疗在心腹病不可忍方∶上,取桃东引枝,削去苍皮,取白皮一握,以水二升,煮取半升服之。

《医门方》疗卒心腹痛方∶浓朴(三两)桂心(二两)以水三升,煮取一升服之。若卒痛如刺者,但煮桂汁饮之佳。

《僧深方》治恶气心腹痛欲死方∶夕药(一两)甘草(二两)桂心(一两)当归(二两)凡四物,水五升,煮取二升,分再服。

《耆婆方》治人心腹绞痛不止方∶生姜(十两)桂心(三两)甘草(三两)人参(二两)四味,切,以水一斗,煮取二升,分三服。
治心腹胀满方第六

《病源论》云∶心腹胀者,脾虚而邪气客之,乘于心脾故也。

《葛氏方》治卒苦心腹烦满,又胸胁痛欲死方∶以热汤令的的尔,渍手足,冷复易。秘方。

又方∶灸乳下一寸七壮。

又方∶捣香汁,服一二升。

又方∶锉薏苡根,煮取汁,服三升。

《极要方》备急丸,疗忽然心腹胀满,急痛,气绝,大小便不通方∶大黄(五两)干姜(二两)巴豆(三两,去心,熬)芒硝(三两)上,蜜丸,平晓饮服四丸,不利更加一二丸。取得四五度利,利如不止取醋饭止之。

《新录方》治心腹烦满方∶桃仁去皮,捣如泥,热酒服如枣二枚,日三。

又云∶烦满吐逆方∶生姜(一斤)合皮切捣取汁,温服之。

又方∶生蓼捣取汁,服一合二合。

《医门方》云∶凡患胀满,频以利药下之,心腹中痛,妨(害也)不去气。又筑心者,缘胃中人参(三两)甘草(三两,炙)橘皮(二两)枣(三十颗)水七升,煮取二升半,去滓,分温三服。若心忪(织容反,心动也)加茯苓三两。

又云∶腹胀满,不能服药,导之方∶取独头蒜,烧令熟,去皮,及热以薄绵裹纳下部,佳。冷易之。

又方∶暖生姜,削,绵裹纳下部中,冷易之,佳。

《耆婆方》治人腹胀痛方∶浓朴(三两)高良姜(三两)切,以水三升煮,分取一升半,少少热饮之,乃止。

《僧深方》云∶浓朴汤,治腹满发数十日,脉浮数,食饮如故方∶浓朴(半斤)枳实(五枚)大黄(四两)凡三物,以水一斗二升,煮取五升,纳大黄;微火煎令得三升,先食服一升,日三。

《本草》云∶诃黎勒水摩或散,水服之。

《苏敬本草注》云∶槟榔子捣末,服之。

《孟诜食经》云∶薤可作宿菹,空腹食之。
治卒腰痛方第七

《病源论》云∶肾主腰,肾经虚损,风冷乘之,故腰痛也。又,邪客于足太阴之络,令人腰腰痛伤腰《养生方》云∶饮食勿即卧,久作气病,令腰痛疼。

又云∶笑多则肾转腰痛。

《千金方》杜仲酒方治五种腰痛方∶桑寄生杜仲鹿茸桂心四味分等,末,服方寸匕,日三。

又方∶治腰脊疼不随方∶鹿角去上皮取白者,熬黄,末,酒服方寸匕,日三。特禁生鱼,余不禁。新者良,陈者不任《录验方》治腰脚疼,不可忍,不能立。胡麻散方∶取胡麻熬令香,于臼内捣碎,即以纱罗筛之数之,筛之若不数筛,即脂出,不可筛,令皮日及饮。蜜汤、羹汁等并得服,亦无禁。

《短剧方》云∶灸腰痛法∶令病患正立,以竹杖注地,度至脐,以度注地,背正灸脊骨上,随年壮。灸竟,藏竹,勿带又,侠两旁各一寸,复灸之,为横三穴,间一寸也。

又,灸腰目小邪,在尻上左右陷处是也。

《葛氏方》治卒腰痛不得俯仰方∶正倚(于义反)立,竹以度其人足下至脐,断竹,反以度之背后,当脊中。灸竹上头处,追年又方∶去穷骨上一寸,灸七壮。其左右各一寸,灸七壮。

《范汪方》治腰卒痛,拘急不得喘息,若醉饱,得之欲死,大豆紫汤方∶大豆(一斗,熬令焦)好酒(二斗)二物,合煮令热沸,随人多少,服令醉。

《医门方》疗卒腰痛不得转侧方∶鹿角一枚,长五六寸,截之。烧鹿角令赤,纳酒中浸之。须臾又燥,还纳酒中。如此数度破
治概腰痛方第八

《病源论》云∶概腰者,谓卒然损伤于腰而致病也。此由损血搏于腰脊所为,久不已,令人《葛氏方》云∶概腰者,是反腰忽动转,而挽之。治概腰痛欲死方∶生葛根削之,嚼咽其汁,多多益佳。

又方∶生地黄捣绞取三升,煎得二升,纳白蜜一升,日三服。不瘥更作。

又云∶治反腰有血痛方∶捣桂,下筛三升许,以苦酒和以涂痛上,干复涂。

又方∶灸足踵白肉际三壮。

《范汪方》治概腰方∶寄生桑上者、鹿茸、桂心、牡丹分等为散,服方寸匕,日三。

又方∶灸足跟白肉际三丸。

又方∶灸腰目十丸。
治肾着腰痛方第九

《病源论》云∶肾经虚则受风冷,内有积水,风水相搏,渍于肾,肾气内着,不能宣通,故食如故。久变为水病。

《千金方》云∶肾着之为病,其人身体重,腰中冷,所以如水洗状,又不渴,小便自利,食肾着汤主之∶甘草(一两)干姜(二两)茯苓(四两)术(四两)四味,水五升,煮取三升,分三服,腰中即温。(今按∶《集验方》无术。)《僧深方》茯苓汤,治肾着之为病,从腰以下冷痛而重如五千钱腹肿方∶饴胶(八两)白术(四两)茯苓(四两)干姜(二两)甘草(二两)凡五物,以水一斗,煮取三升,去滓纳饴,令烊,分四服。
治肝病方第十

《病源论》云∶肝气盛,为血有余,则病目赤,两胁下痛引小腹,善怒。气逆则头眩,耳聋得太《千金方》治肝虚寒,胁下痛,胀满气急;眼昏浊,视物不明,槟榔汤方∶母姜(七两)附子(七两)槟榔(二十四枚)茯苓(三两)桔梗(四两)橘皮(三两)白术九味,以水九升,煮取三升,去滓,分温三服。

又云∶治肝实热,目痛,胸满气急塞,泻肝前胡汤方∶前胡(三两)秦皮(三两)细辛(三两)栀子仁(三分)黄芩(三两)蜀升麻(三两)蕤核升)芒硝(三两)十一味,以水九升,煮取三升,去滓,下芒硝,分二服。

《录验方》补肝汤,治肝气不足,胁下满,筋急,不得太息,厥疝抢心,脚中痛,两目不明牛黄(一两)乌头(四两)柏子(四两)大枣(二十枚)龙胆(四两)凡五物,切,以水五升,煮取一升,去滓,一服令尽。

《僧深方》泻肝汤,治肝气实,目赤若黄,胁下急,小便难方∶人参(三两)生姜(五两)黄芩(二两)半夏(一升,洗)甘草(二两)大枣(十四枚)凡六物,切,水五升,煮半夏令三四沸,纳药,后纳姜,煎取二升,去滓,分二服,羸人三
治心病方第十一

《病源论》云∶心气盛,为神有余,则病胸内痛,胁支满。膺、背、髀间痛,两臂内痛,恍惚,《千金方》治心虚寒,心中满胀,悲忧,或梦山丘平泽半夏补心汤方∶半夏(六两,洗)宿姜(三两)茯苓(三两)白术(四两)防风(二两)桂心(三两)远志九味,切,以水一斗,煮取三升,分三服。

《录验方》治心上虚热,胸中时痛,口生疮,四大羸乏少气方∶茈胡(四两)升麻(三两)黄芩(三两)生地黄(八两)夕药(四两)地骨白皮(五两)枳凡九物,切,以水八升,煮取二升七合,分三服。

《耆婆方》治人心中热风,见鬼来亲合阴阳,旦便力乏,黄瘦不能食,日日转羸方∶龙胆(三分)苦参(三分)上二味,为散,以白米饮一服一钱,日二服,忌猪肉酒面。
治脾病方第十二

《病源论》云∶脾气盛,为形有余,则病腹胀,溲不利;身重口苦饥,足痿不收,行善挚,肠鸣。是为脾气之虚也,则宜补之。

《删繁方》疗脾虚寒,劳损气胀,噫满,食不下通,噫消食方∶猪膏(三升)宿姜汁(五升)吴茱萸(一升)白术(一升)捣茱萸等二物为散,纳姜汁膏中,煎取六升,温酒进方寸匕,日再服。

《千金方》治脾实热,舌本强直,或梦歌乐而体重不能行,泻热汤方∶茈胡(三分)茯苓(三分)玄参(二分)大青(二分)龙胆(三两)细辛(三分)杏仁(四九味,水九升,煮取三升,分三服。

《极要方》疗脾热,内热唇燥渴,多肿身重方∶以生地黄五两,捣之以水,绞取汁渍梅,蜜和服之。

又方∶取芹菜汁一升,一服。

又方∶麦门冬汁服之。

《广济方》主脾胃中热,渴欲得饮冷水,不下食方∶茯苓(四两)甘草(二两)石膏(五两)地骨白皮(三两)茅根(切,一升)切,以水八升,煮取二升五合,绞去滓,分温三服,忌热面、热肉、海藻、猪、蒜。

《僧深方》温脾汤,治脾气不足,虚弱下利,上入下出方∶干姜(三两)人参(二两)附子(二两)甘草(三两)大黄(三两)凡五物,切,以水八升,煮取二升半,分三服,应得下去毒实,甚良。
治肺病方第十三

《病源论》云∶肺气盛,为气有余,则病喘咳上气,背痛,汗出。尻、阴、股、膝、(报息《千金方》治肺虚寒乏气,小腹拘急,腰痛,羸瘠(瘦也)百病,小建中汤方∶大枣(二十枚)干姜(三两)夕药(二两)甘草(三两)桂心(三两)五味,水八升,煮取三升,去滓,纳饴八两,煮三沸,分三服。

又云∶治肺实热则胸KT仰息泄气除热方∶枸杞根皮(二升,切)白前(三两)石膏(八两)杏仁(三两)橘皮(五两)白术(五两)六味水七升,煮取二升,去滓,下蜜,煎两三沸,分三服。
治肾病方第十四

《病源论》云∶肾气盛,为志有余,则病腹胀,飧泄,体肿,喘咳,汗出目黑,小便黄。

是也《千金方》治肾气虚寒,阴痿,腰脊痛,身重缓弱,言音混浊,阳气顿绝方∶生地黄(五斤,干)苁蓉(八两)白术(八两)巴戟天(八两)麦门冬(八两)茯苓(八两)两)干姜(五两)十二味,为散,食后酒服方寸匕,日三。

又云∶治肾实热,小腹胀满,四肢正黑,耳聋,梦腰脊离解,梦伏水等气急,泻肾汤方∶大黄(三两,以水一升,密器中宿渍,碎如雀头大)生地黄(五两)甘草(二两)茯苓(三硝(三两)十味,以水九升,煮取七味,取二升五合,去滓。别渍大黄,纳药汁中,更取二升三合,去《耆婆方》治肾气虚,则梦使人见舟船溺人。冬时梦见伏水中,及在水行,若有恐畏,恶人秦胶石斛泽泻防风人参(各一分)茯苓黄芩干地黄远志(各八分)十味,切,捣筛为散,以酒服方寸匕,日二。
治大肠病方第十五

《病源论》云∶大肠气盛则为有余,则病肠内切痛,如锥刀刺,无休息,腰背寒痹,挛(力大肠《千金方》治大肠虚寒,利下青白,肠虚雷鸣相逐黄连补汤方∶黄连(四两)茯苓(三两)芎(三两)酸石榴皮(五具)地榆(五两)伏龙肝(鸡子大一六味,切,水七升,煮取二升五合,去滓,下伏龙肝屑,搅挠调,分三服。

又云∶治大肠实热,腹胀不通,口生疮,生姜泄肠汤方∶宿姜(三两)橘皮(三两)大枣(十四枚)青竹茹(三两)生地黄(十两)桂心(一两)黄十一味,切,水七升,煮取三升,去滓,下芒硝,分三服。
治小肠病方第十六

《病源论》云∶小肠气盛为有余,则病小肠热,焦竭干涩,小腹胀,是为小肠之气实也,则《千金方》治小肠虚寒,痛下赤白,肠滑,腹中懊,补阳方∶干姜(三两)当归(四两)黄柏(四两)地榆皮(四两)黄连(三两)阿胶(三两)石榴皮七味,水七升,煮取二升五合,去滓,下胶煮烊,分三服。

又云∶治小肠实,热胀口疮紫胡泻阳方∶紫胡(三两)橘皮(三两)黄芩(三两)泽泻(三两)枳实(三两)旋复花(三两)升麻(九味,切,以水一斗,煮取三升,去滓,下芒硝,分三服。
治胆病方第十七

《病源论》云∶胆气盛为有余,则病腹内冒冒(干也,突也,莫报反)不安,身躯(去迂反,宿汁,心胆气之虚也《千金方》治胆腑实热,精神不守,泻热,半夏千里水汤方∶半夏(三两)酸枣仁(五合)黄芩(一两)远志(二两)茯苓(二两)宿姜(三两)秫米(凡八物,细切,取长流水五斗,先煮秫米,令蟹目沸,扬之三千过,澄清,取六升煮药,取
治胃病方第十八

《病源论》云∶胃气盛为有余,则病腹胀,气满,是为胃气之实也,则宜泻之。胃气饥而《千金方》治胃虚冷,少气口苦,身体无泽,补胃汤方∶防风(二两)柏子仁(二两)吴茱萸(三两)细辛(二两)甘草(一两)桂心(二两)橘皮凡九味,切,以水一斗,煮取三升,分三服。

又云∶治胃实泻胃热汤方∶栀子仁(三两)夕药(四两)白术(五两)茯苓(二两)生地黄汁(一升)夜干(三两)升八味,切,以水七升,煮取一升五合,去滓,下地黄汁,煮两沸;次下蜜,煎取三升,分三
治膀胱病方第十九

《病源论》云∶五谷五味之津液,悉归于膀胱,气化分入血脉,以成骨髓也;而津之余者,泻之。

《千金方》云∶膀胱病者;少腹满,肿而痛,以手按则欲小便而不得。

又云∶治膀胱虚冷,饥不欲食,面黑如炭,腰胁疼痛方∶磁石(六两)黄(三两)杜仲(四两)白石英(五两)五味(四两)茯苓(三两)术(五七味,水九升,煮取三升,分三服。

又云∶治膀胱实热方∶栀子仁(三两)石膏(八两)茯苓(三两)淡竹叶(切,一升)生地黄(切,一升)蜜(一七味,水七升,煮取二升,去滓,下蜜,煮两沸,分三服。须痢加芒硝三两。
治三焦病方第二十

《病源论》云∶三焦盛为有余,则胀,气满于皮肤内,轻轻然而不牢。或小便涩,或大便难或胸《千金方》云∶三焦病者,腹胀气满,小腹尤坚,不得小便。窘急,溢则为水,留则为腹胀又云∶治上焦虚寒,短气不续,语声不出,黄理中汤方∶黄(二两)桂心(二两)丹参(四两)桔梗(三两)干姜(三两)五味子(三两)茯苓(十味,切,以水九升,煮取三升,分三服。

又云∶治上焦实热,腹满而不欲食,或食先吐而后下,肘后胁挛痛,麦门冬理中汤方∶生麦门冬(一升)生姜(四两)甘草(二两)人参(三两)茯苓(二两)橘皮(三两)竹茹(一升)生芦根(切,一升)KT心(五合)廪米(一升)白术(一两)葳蕤(三两)十二味,切,水一斗五升煮取三升,分三服。

又云∶治中焦实热闭塞,上下不通,隔绝关格(关,闭也;格,相拒也),上吐不下,腹彭彭蜀大黄(三两,切,水一升,五合别渍)黄芩(三两)泽泻(三两)升麻(三两)羚羊角(九味,以水七升,煮八物,取二升三合,下大黄,更煎数沸,绞去黄滓,下芒硝,分三服。

又云∶治中焦虚寒,洞泄下利,或因霍乱后泻黄白无度,腹中虚痛,黄连煎方∶金色黄连(四两)黄柏(三两)当归(三两)浓朴(三两)干姜(三两)酸石榴皮(四具)八味,以水九升,煮七物取一升,去滓,下阿胶,更煎取烊,分三服。

又云∶治下焦虚寒损,或先便转后见血,此为远血。或利下,或不利,好因劳冷而发,续断续断(三两)当归(三两)干姜(四两)甘草(一两)干地黄(四两)桂心(三两)六味,切,以水九升煮八物,取三升五合,去滓,下阿胶,更烊,取胶烊尽,下蒲黄,分三又云∶治下焦热,大小便利不通,紫胡通塞汤方∶紫胡(三两)黄芩(三两)橘皮(三两)泽泻(三两)栀子仁(四两)石膏(六两,碎)羚两)十味,以水一斗,煮九物取三升,去滓,下芒硝,分三服。
治气病方第二十一

《千金方》治气极虚寒,皮毛焦,津液不通,虚劳百病,气力损乏,黄汤方∶黄(四两)人参(二两)干枣(十枚,去核)生姜(八分)白术(二两)桂心(二两)六味,切,以水八升,煮取三升,分四服。

又云∶理气丸治气不足方∶杏仁(一两)益智子(二两)廉姜(二两)桂心(一两)四味,丸如梧子,未食服三丸,以知为度。
治脉病方第二十二

《千金方》治脉虚,惊跳不定,乍来乍去,主小肠腑寒补虚调中防风丸方∶防风(十二分)桂心(十二分)通草(十二分)茯神(十二分)远志(十二分)甘草(十二九味,捣筛为散,白蜜和丸如梧子,酒服十丸,日再,加至二十为剂限也。

又云∶治脉实洪满,主心热病升麻汤方∶蜀升麻(三两)栀子仁(三两)生地黄(切,一升)子芩(三两)泽泻(三两)淡竹叶(三七味,以水九升,煮取三升,去滓,下芒硝,分三服。
治筋病方第二十三

《删繁方》治筋虚实,暴损绝极,或因霍乱转动腹满并转痛。或因服药吐利过瘥,脚手虚转人参(二两)浓朴(二两,炙)葱白(一虎口,切)白术(四两)蓼(一把长三寸)五物,切,以水五升,煮取二升,去滓,分再服。

又云∶治转筋霍乱后因而筋转方∶取絮巾若绵,炙暖以缚筋上。

又云∶胞(膀胱)转筋急方∶白术(四两)香豉(一升,熬)栀子仁(二两)榆白皮(三两)子芩(三两)通草(二两)七物,切,以水七升,煮取三升,去滓分三服。

又云∶治转筋阴囊卵缩入腹,腹中绞痛以交接极损所为方∶取豚子一头,杖撞三十六下,放于户中逐之,使喘极,刺胁下,取血一升,以酒一升,共和饮之,若无酒,单血亦好,勿令冷凝也。

又云∶治交接损缩,卵筋挛方∶烧妇人月经衣,服方寸匕。(以上《千金方》同之。)又云∶治转筋十指筋挛(吕员反,拘也,又缩也)急,不得屈伸别法∶灸手踝上七炷,大良。

又云∶治转筋,胫骨痛,不可忍方∶灸屈膝下廉横筋上三炷。

又云∶转筋方∶灸涌泉,涌泉在脚心下,当拇指大筋是,灸七壮。

又方∶灸大都,大都在足母指大节内侧,白肉际七壮。

又云∶腹肠转筋方∶灸脐上一寸十四壮。

又云∶治筋绝方∶蟹脑足髓熬,纳疮中,筋即续矣。

《千金方》治筋实极,手足爪甲或青,或黄,或黑,乌黯,四肢筋急,烦满。地黄煎方∶地黄汁(三升)生葛汁(一升)生玄参(一升)大黄(二两)栀子(三两)升麻(三两)石十味,水七升,煮七物取二升,去滓,下地黄汁,煎两沸;次下葛汁等,煎取三升,分三服
治骨病方第二十四

《删繁方》云∶凡骨虚实之应,主于肾,膀胱。若其腑脏有病,从骨生,热则应脏,寒则又云∶治骨实苦烦热鸡子白煎方(通四时用)∶鸡子(七枚,扣开取白)生地黄汁(一升)麦门冬(三合)赤蜜(一升)凡四汁,相和搅调,微火上煎之三沸,分三服。

《极要方》疗骨虚劳冷,骨节疼痛无力方∶豉(二升)地黄(八两)再遍蒸曝干,以食后以酒一升服之。并治虚热。
治髓病方第二十五

《删繁方》云∶凡髓虚实之应,主于肝,胆。若其腑脏有病,从髓生,热则应脏,寒则应腑又云∶治髓实,勇KT惊热,主肝热紫胡发泄汤方∶紫胡(三两)升麻(三两)黄芩(三两)泽泻(四两)细辛(三两)枳实(三两)淡竹叶(凡十物,以水九升,煮取三升,去滓,下芒硝,分三服。

《千金方》治髓虚,(恼)不安,胆腑中寒,羌活补髓丸方∶羌活(三两)桂心(三两)芎(三两)当归(三两)人参(四两)枣肉(一升,研为脂)大麻子仁(二升,熬,研为脂)羊髓(一升)蒜(一升)牛髓(一升)十味,先捣五种干药为散,下枣膏,麻仁更捣,相濡为一家,下二髓纳铜钵中,汤中煎之,
治皮病方第二十六

《删繁方》云∶皮虚实之应,主于肺、大肠。其病发于皮毛,热则应脏,寒则应腑。凡皮虚又云∶治皮虚,主大肠病,寒气关格(闭困也,拒也)蒴蒸阳方∶蒴根茎(切,三升)桃枝菜(锉,三升)细糠(一斗)秫米(五升)菖蒲根叶(锉,二升)凡五物,水一石五斗,煮取米熟为度,大盆器贮。盆上作少竹床子,置盆人身坐床中,四面可作两法,是又云∶治皮实,主肺病,热气所加,栀子煎方∶栀子仁(三两)生地黄(切,一升)枳实(三两)石膏(八两)大青(三两)杏仁(三两)凡十物,切,以水九升,煮取三升,去滓,下芒硝,平旦分三服。(以上《千金方》同之。)
治肉病方第二十七

《删繁方》云∶凡肉虚实之应,主脾胃。若其腑脏有病,从内生,热则应脏,寒则应腑。

凡肉虚者,坐不平席,身卮动。肉实者,坐平不动,喘气。

又云∶治肉虚坐不平席,好动,主胃病寒气所加,五茄酒方(通四时用∶)五茄皮(二升)枸杞皮(二升)干地黄(八两)丹参(八两)杜仲(一斤)干姜(四两)附凡八物,咀,橙子贮之。清酒二斗,渍之三宿,一服七合,日再服。

又云∶治肉实,坐平席,不动喘气,主脾病热气格,半夏汤除喘方(通四时用∶)半夏(八两,洗)宿姜(八两)细辛(三两)杏仁(五两)橘皮(四两)麻黄(三两)石膏凡八物,切,水九升,煮取三升,去滓,分三服。须利下,加芒硝三两。
卷第七
治阴疮方第一

《病源论》云∶肾(时忍反)营于阴。肾气虚,不能制津液,则汗湿;虚则为风邪所乘,邪客《葛氏方》治男子阴疮烂方∶削黄柏,煮以洗之,日十过。

又方∶野狼牙草根,煮以洗渍之,日五六过。(今按∶《极要方》∶野狼牙二把,水四升。)又方∶连黄、黄柏分等,捣,以肥猪肉汁,煮之,去滓,以渍之。复捣此二物,绢筛下,以又方∶煮地榆以洗渍之,合甘草尤佳。

《范汪方》治人阴头断生疮方∶芜菁一把,切,水煮令熟,食之。

《极要方》疗阴疮方∶黄连(三分)胡粉(一分)黄柏(三分)为散,敷疮上。

《集验方》云阴恶疮方∶以蜜煎甘草末,涂之。良。

又云∶阴头生疮如安石榴花、大者如卷方∶虎牙、犀角,刀刮末,以猪膏煎令变色,去滓,日三涂。

又方∶以乌贼鱼骨末粉之,良。

又方∶鳖甲烧末,以鸡子白和敷之。

《随时方》治阴恶疮方∶取薤白和苏敷之,当日即瘥。

《千金方》治阴生疮方∶地榆(八两)黄柏(八两)二味,以水一斗五升,煮取六升,去滓,适冷暖,用洗疮,日再。

又云∶妒精疮者,男子在阴头节下,妇人在玉门内,并似甘疮,治大痛方∶用银钗绵缠,以腊月猪脂熏黄,火上暖,以钗烙疮上,令熟,取干槐枝叶涂之。以麝香、黄《录验方》治阴头疮肿转困笃方∶黄连汁黄柏汁龙胆汁凡三物,合,得半升,别煮猪蹄汁二升,合和,着筒中热灰上温之,渍阴,日三。

《医门方》疗男子阴疮烂方∶黄柏野狼牙(各三两,或二两)水四升,煮取一升半,去滓,以浸疮数洗了,末蛇床子、黄连以敷疮中。

《令李方》治阴劳疮,生息肉烂破痛,医所不能治,矾石散方∶矾石(一分,烧)细辛(一分)白芷(一分)凡三物,冶筛,以温水洗疮,乃粉。
治阴蚀疮欲尽方第二

《葛氏方》治阴蚀疮欲尽方∶取虾蟆、菟矢分等,捣勃疮上。

《范汪方》治阴肿生疮至尽方∶KT胡粉涂良。

又方∶(荻)叶作灰,敷之。

又云∶男子阴头生疮精食啮欲尽方∶当归夕药黄芩术麝香白粉为汤,一洗之。

《令李方》治阴蚀蒲黄散方∶蒲黄(二两)桐皮(二两)甘草(二两)凡三物,捣筛粉疮上,不过三,愈。

《千金方》治阴蚀疮方∶以肥猪肉五斤,水三升,煮肉令极烂。去肉,以汤令极热,便以灌疮中,冷即愈。

又方∶雄黄(二分)矾石(二分,烧)麝香(半分)三味,冶下筛为粉,以粉疮上,即瘥。

治虫食人阴茎并囊欲尽方∶烧鲤骨为灰,和黄柏汁涂之。

又方∶烧鲋鱼,和酱汁涂之。
治阴痒方第三

《病源论》云∶大虚劳损,肾气不足,故阴冷,汗液自泄。风邪乘之,则瘙痒。

《录验方》治阴痒疮多少有汁者方∶煮黄柏汁,冷洗渍,敷蛇床子、黄连末。

《新录要方》治阴痒水出不能瘥者方∶干姜末粉之。

又方∶水煮芜菁子,洗并末,粉上。

又方∶杏仁烧取油,涂之。

又方∶水煮棘针,洗之。

又方∶水煮桃皮叶,洗之。

又方∶取薤白捣汁,涂之。

又方∶灸脊穷骨,名龟尾,依年壮,或七壮。

又,灸足大指丛毛中,多至七壮,并良。

《极要方》治阴痒湿生疮历年不瘥方∶桃核中人,烧末服七枚,日三服,二十一枚。三日愈。

又方∶嚼胡麻,敷疮,不过四五度,愈。

《龙华方》治男子阴下疮痒湿方∶白粉(一分)干姜(三分)牡蛎(三分)三物,下筛,欲卧时粉,夜三四,粉之。

《效验方》牡蛎散治男子阴下痒湿方∶牡蛎(三分)干姜(三分)凡二物,下筛,以粉,日二。

《僧深方》治阴下湿痒生疮方∶吴茱萸(一升)凡一物,以水三升,煮三沸,以去滓,洗疮愈。

又方∶蒲黄粉疮上,日三,过即愈。

又方∶甘草(一尺)凡一物,水五升,煮取三升,洗渍之。日三便愈,神良。

《耆婆方》治人阴下痒湿方∶蛇床子作末,和米粉,少少粉之。

《删繁方》治阴生湿疮包用∶石硫黄末敷之。

《集验方》治大人小儿阴茎痒,汁出方∶取生大豆刮去皮,熟嚼涂之。

《本草拾遗》云∶牡蛎壳和麻黄根、蛇床子、干姜为粉,去阴汗。
治阴茎肿痛方第四

《病源论》云∶阴肿候。此由风热客于肾经,流于阴器,肾虚不能宣散,故致肿。

《范汪方》治卒阴肿欲死方∶急服下药使大下,即佳。

又方∶末乌贼鱼骨,粉。

又方∶末蛇床子,和鸡子黄,敷之。

《葛氏方》治阴茎头急肿生疮汁出方∶浓煮黄柏汁,管中渍之。

又方∶浓煮水杨叶,管中温渍之。

又方∶当归(三分)黄连(三分)小豆(一分)凡三物,捣筛,以粉上。

又方∶杏仁、鸡子白和涂之。

又方∶烧豉三粒,末敷之。

又方∶以白蜜涂之。

又方∶烧牛矢末,和苦酒涂之。

又云∶阴茎忽肿痛不可忍方∶雄黄矾石(各二两)甘草(三尺)水二斗,煮取二升,以渍之。

又云∶治卒阴痛如刺汗出如雨方∶小蒜(一升)韭根(一斤)杨柳根(一斤)上三物,合烧,以酒灌之,及热,以气蒸阴。(《千金方》同之。)《千金方》治阴肿痛方∶捣苋菜根,敷之。

又方∶酒服桃仁,末,弹丸大,三服。

又云∶玉茎痛方∶甘草,石蜜。末,和乳,洗之。

《龙华方》治阴头肿溃败坏方∶甘草(一分)乌头〔(附子一名)一分〕夕药(一分)败酱(二分)四物,切,以水四升,煮取三升,洗之,日三。

治阴茎头肿生疮黄汁出方∶干姜末,捣敷上。不过再三即愈。

《新录单方》阴肿痛方∶捣桃仁为泥,和水苦酒,涂之,数易瘥止。

又方∶末蔓菁子并根,封之。

《枕中方》治男子阴肿方∶以灶中黄土,以酒和之,涂其上,立愈。有验。
治阴囊肿痛方第五

《千金方》治阴囊肿痛方∶酢和面,涂之。

又方∶酢和热灰,熨之。

又方∶釜月下土,鸡子白,和,敷之。

《葛氏方》治男子阴卵卒肿痛方∶烧牛矢末,以苦酒和敷之。

又方∶蛇床子末,鸡子黄和敷之。

又方∶捣芜菁涂之。

又方∶灸足大指第二节横纹理正中央五壮,佳。

《医门方》治阴卵肿方∶以桂心末涂之,佳。

又方∶末大黄和酢涂之,并佳。

又方∶取皂荚,炙,去皮子,末,和水涂之,立消。

又云∶疗卒阴卵肿疼痛不可忍方∶灸足大拇指头,去爪甲如韭叶,随年壮。右核肿灸右,左肿灸左,两核俱肿,俱灸之,一宿又方∶以玉茎头向下正囊缝点茎头,当缝上灸三壮(或七壮),即消。

《博济安众方》治久坐立卑湿冷处忽阴囊虚肿气上筑人方∶上,以米醋炒黑豆,青布裹熨心腹。

《集验方》治卒卵肿方∶熟捣桃仁,敷之,燥则易,亦治妇人阴肿。

《玄女经》云疗男子阴卵肿方∶取捣桃仁,去皮尖,并简除双人,熬令色变,作末,丸如弹丸,酒服之。
治阴卵入腹急痛方第六

《葛氏方》治阴丸卒缩入腹急痛欲死,名曰阴疝(音山)方∶野狼毒(四两)防葵(一两)附子(二两)上三物,蜜丸,服如梧子三丸,日夜三过。

《玄女经》疗房劳卵肿,或缩入腹,腹中绞痛或便气绝死方∶取妇人月经布衣有血者,汤洗取汁,服之。(今按∶《医门方》∶为灰酒服方寸匕。)又方∶取妇人阴上毛二七茎,烧作灰,以井华水服之。
治阴囊湿痒方第七

《葛氏方》治阴囊下湿痒皮剥方∶酸浆煮地榆根及黄柏汁,洗皆良。

又方∶柏叶、盐各一升,合煎以洗之毕,取蒲黄敷之。

又方∶煮槐枝以洗之。

又方∶嚼大麻子,敷之。

又方∶浓煮香洗之。

《医门方》疗阴囊下湿痒,搔破水出,干即皮剥起方∶地榆黄柏蛇床子(各三两)槐白皮(切,一升。)水七升,煮取三升,暖以洗疮,日三四度,勿食鱼。

《玉房秘诀》云∶治阴囊下湿散方∶麻黄(三两)蛇床子(二两)夕药(三两)黄连(三两)粢米(一升)捣末粉之。

又方∶干姜黄连牡蛎(各五分)粢米(八分)捣筛以粉之。

又方∶末子敷之。
治阴颓方第八

《病源论》云∶颓病之状,阴核肿大,有时小歇,歇时终大于常。劳冷阴雨便发,发则胀大气下胀不《短剧方》牡丹五痔散治颓疝,阴卵偏大,有气上下,胀大行走,肿大为妨,服此方良验∶牡丹(去心)防风桂心豉熬黄柏(各一分)凡五物,冶下筛,酒服一刀圭匕,二十日,愈。治少小颓疝最良,婴儿以乳汁和如大豆与之又云∶男颓有肠颓、卵胀,有水颓、气颓四种。肠颓,卵胀难瘥;气颓、水颓针灸则易瘥也又云∶男阴卵偏大颓方∶灸肩井,并灸关元百壮。

又方∶灸玉泉百壮,在关元下一寸。

又方∶灸足太阳五十壮,并灸足太阴五十壮,有验。

又云∶颓病阴卒肿方∶合并足,缚两大指,令爪相并,以艾丸灸两爪端方角处一丸,令顿在两爪角上也。令丸半上《千金方》云∶卵偏大上入肠方∶灸三阴交,在内踝上八寸,随年壮。

又云∶治颓方∶取杨柳枝如脚大指大,长三尺,二十枚,水煮令热,以故布及毡掩肿处,取热枝更牙柱之,又云∶颓疝卵偏大气上下胀方∶牡丹(一分)防风(一分)二味,酒服方寸匕,日二。

《葛氏方》治人超跃举重卒得阴颓方∶灸两足大指外白肉际陷中,令艾丸半在爪上,半在肉上,七壮。(《范汪方》同之。)又方∶以蒲度口广倍之,申度以约小腹中大横理,令中央正对脐,乃灸两头及中央三处,随又方∶白术(五分)地肤子(十分)桂心(三分)上三物,捣末,服一刀圭,日三。
治脱肛方第九

《病源论》云∶脱肛者,肛门脱出也,多由久利大肠虚冷所为。大肠虚而伤于寒,利而用气《千金方》云∶脱肛禁举重及急带衣,断房室周年,乃佳。

《梅略方》云∶脱肛慎举重,食滑物,急衣带。

《短剧方》治脱肛验方∶蒲黄(二两)猪膏(三合)凡二物,捣合和,敷肛上,当迫纳之,不过再三,便愈。

又方∶取女萎一升,火烧以烟熏肛,即愈。

《葛氏方》治卒大便脱肛方∶灸顶上回发中,百壮。

又方∶猪膏和蒲黄,抑纳,但以粉之,亦佳。

又方∶熬锻石令热,故绵裹之,坐其上。冷又易之,并豆酱渍,合酒涂之。

又云∶若肠随肛出转广不可入者,捣生栝蒌,取汁,温,以猪膏纳中,手洗,随按抑自得缩又方∶熬锻石令热,布裹以熨之,随按令入。

又方∶以铁精粉之。

《千金方》治肛滞出壁土散方∶故屋东壁土(一升碎)皂荚(三梃,各长一尺二寸)二味,捣土为散,挹粉肛头出处,取皂荚炙暖,更递熨之,取入为主。

又云∶炙麻履底按入方∶故败麻履底(一枚)鳖头(一枚)二味,烧,鳖头捣为散,敷肛门滞出头,将履底按入,即不出矣。

又云∶治脱肛历年不愈方∶死鳖头一枚,烧令烟绝,冶作屑,以敷肛门,上进,以手按之。(今按∶《本草拾遗》云∶有以似为药者∶蜗牛、鳖头,脱肛皆烧末敷之,自缩。此即以类为药也。)又方∶灸龟尾立愈,即后穷骨也。

又云∶治积冷利脱肛方∶枳实一枚,石上摩令滑泽,钻安把,蜜涂,炙令微暖,熨之。冷更易之。

又方∶铁精粉纳上,按令入即愈。

又云∶病寒冷脱肛方∶灸脐中,随年壮。

又云∶治脱肛出方∶磁石(四两)桂(一尺)皮(一枚,炙令黄)合捣下筛,服方寸匕,日一,十服即缩。

《范汪方》治脱肛方∶生铁三斤以水一斗,煮取五升,出铁,以汁洗上,日三。

《医门方》疗大肠寒则肛门洞泻凹出方∶鳖头(一枚,烧令烟绝)铁精(一两)捣筛为散,粉上令遍。

又方∶取破麻履底一枚,炙令微热,以履底按肛入,即更不出。
治谷道痒痛方第十

《病源论》云∶谷道痒者,由胃弱肠虚,则蛲虫下侵谷道。重者食于肛门,轻者但痒也。

《葛氏方》治下部痒痛如虫啮者∶胡粉、水银以枣膏和调,绵裹导之。

又方∶杏仁熬令黑,捣取膏涂之。

又方∶高鼻蜣螂烧末,绵裹纳孔中,当大蛘(弋掌反)虫出。

又方∶桃叶捣一斛,蒸之,纳小口器中,大孔布上坐,虫死。

《录验方》若下部痒痛如虫啮者∶小豆一升,好苦酒五升,煮豆熟出干,干复纳酒,酒尽止。末,酒服方寸匕,日三。

《徐伯方》治谷道忽痒痛,肿起欲生肉突,出方∶槐白皮(六两)甘草(三两)凡二物,以豆汁煮,渍故帛敷之,热即易。

《新录单方》治谷道中有虫痒者∶艾三升,水五升,煮取二升,二服。

又方∶诸肉炙令香,匝熨,虫皆出。

《极要方》卒暴冷下部疼闷方∶烧砖令热,大醋泼之,三重布覆,坐上取瘥止。(《千金方》同之。)《集验方》治虫食下部方∶胡粉、雄黄分等,末,着谷道中。
治谷道赤痛方第十一

《集验方》治谷道赤痛方∶菟丝子熬令黄黑,和以鸡子黄,以涂之,日三。

又方∶取杏仁熬令黄,捣作脂涂之。
治谷道生疮方第十二

《病源论》云∶谷道、肛门,大肠之候。大肠虚热,其气冲热结肛门,故令生疮。

《葛氏方》治下部卒有疮方∶捣蛴螬涂之。

又方∶煮豉以渍之。

又方∶豆汁以摩墨导之。

《范汪方》治下部卒有疮若转深者∶乌梅(五十枚)盐(五合)水七升,煮取三升,分三服。

又方∶常煮举皮饮之。(以上《葛氏方》同之。)
治湿(吕质反,小虫也)方第十三

《病源论》云∶湿餐病,由脾胃虚弱,为水湿所乘,腹内虫动,侵食成也。

《录验方》治湿下部生疮方∶胡粉水银黄柏凡三物,冶末粉等分,合研水银,散尽以敷疮上。

又方∶常炙猪胴肠,食之佳。

又方∶温尿令热,纳小矾石以洗之。

《集验方》治虫杏仁汤方杏仁(五十枚)苦酒(三升)盐(一合)煮取五合,顿服之。

《令李方》治虫及蛲虫侵食下部栝蒌散方∶栝蒌根(四两)葶苈子(四分)凡二物,冶合下筛,以艾汁浸,绵裹纳下部中,日三易。

《葛氏方》治谷道疮赤痛又痒方∶杏仁熬令黑,捣,以绵裹导之。

又方∶槐皮、桃皮、楝子合末,猪膏和捣。

又方∶菟丝子熬令黄黑,鸡子黄和涂,导之。

又方∶以枣膏和水银令相得,长三寸,绵裹宿导下部。

又方∶胡粉、雄黄分等末,导下部内。

《范汪方》治谷道疮赤痛又痒方∶捣鳝肠,涂绵如指以导之,虫出。
治疳(居酣反)湿方第十四

《病源论》云∶人有嗜甘味多,而动肠胃间诸虫,致令侵食腑脏,此犹是也。但虫因甘而喜忌上食血;肢节又云∶五甘,一是白甘,令人皮肤枯燥,面失颜色。二是赤甘,内食五脏,令人头发焦枯。

腰脊挛又云∶面青颜赤,眼无精光,唇口焦燥,腹胀有块,日日瘦损者是甘,食人五脏,至死不觉又云∶五甘缓者,则变成五蒸。

《千金方》云∶论曰∶凡疳湿之为病,皆由暑月多食肥浓油腻,取冷睡眠所得之。《礼》曰又云∶治疳湿下黑,医不能治,垂死方∶麝香(三分)丁子香(三分)甘草(三分)犀角(三分)四味,并细末之,合和,别以盐三合、椒三合、豉二合,以水二升,煮取一升,去滓,纳末又云∶忌生冷酢滑,但是油腻酱乳酪三十日慎之,大大佳。

又云∶凡疳,一切皆忌。唯白饭、盐、豉、苜蓿、苦豆、芜菁,不在禁限。

《录验方》治甘湿方∶青葙(二两)苦参(二两)雄黄(二两)石榴黄(二两)野狼牙(三两)芜荑(二两)雷丸凡八物,捣筛,取如杏仁大,纳下部中也。

《医门方》疗疳无问所在方取死虾蟆烧作灰,以好醋和涂疮上,即愈。

《救急单验方》疗急蚶方∶无食子末,腹内患者吹下部,立验。

又方∶灌白马尿一升,虫出,验。

又方∶取文蛤烧灰,和腊月猪脂,涂。

又方∶炼矾石、桂心、徐长卿各等分,末,涂验。
治诸痔方第十五

《病源论》云∶诸痔者,谓牡痔、牝痔、脉痔、肠痔、血痔也。

牡痔∶肛边生肉如鼠乳,时时出脓血。

牝痔∶肛边肿,生疮而血出。

脉痔∶肛边生疮,痒痛。

肠痔∶肛边肿核痛,发寒热血出。

血痔∶因便而清血随出。

又有酒痔,肛边生疮,有血出。又气痔,大便难而血出,肛亦出外,良久不肯入。

诸痔皆由伤风,房室不慎,醉饱合阴阳,致劳扰血气,而经脉流溢,渗(乞荫反)漏肠间,冲《养生方》云∶忍大便不出,久作气痔。

《龙门方》云∶一曰肿生,息肉状,如枣核,孔有脓血,名曰雄痔;二曰孔旁有疮,内引孔名曰过多痔病禁忌∶《千金方》云∶禁寒冷食、猪肉、生鱼、菜、房内。病瘥之后,百日今通房内。

又云∶通忌菜。

《极要方》云∶禁肥肉生鱼。

可食物∶《千金方》云∶得食干白肉。

《葛氏方》云∶作鲭鱼脍姜齑,食之多少任人。

《食经》云∶榧实(主五痔);鲷(主去痔虫);蠡鱼(主五痔);海鼠(疗痔为验);竹笋(主五《本草》云∶羊蹄主痔。

《拾遗》云∶鲫脍治赤白利及五痔。

《疗痔病经》云∶佛告阿难陀,汝可谛听。此疗痔病经,读诵受持系心勿忘。亦于他人广为齿受悉皆干燥、堕落、销灭,必瘥无疑。皆应习持如是神咒。即说咒曰∶怛侄他,阿烂帝,阿烂逮,室利HT,室里继,室里继,磨羯失质三婆,跋都婆诃。

《短剧方》五痔散,治酒客及劳损伤冷,下部中有旁孔,起居血纵横出者,及有肉者方悉主赤小豆(四分,熬)蜀黄(二分)蜀附子(一两炮)夕药(二分)白蔹(一分)黄芩(二七物,捣末下筛,以酒服方寸匕,日三,止血有验。

又云,治谷道痒痛痔疮槐皮膏方∶槐皮(五两)楝子(五十枚)桃仁(五十枚)甘草(二两)当归(二两)赤小豆(二合)白七物,咀,以猪膏二升煎令白芷黄,药成,绞去滓敷之,日再三,良。(今按∶《极要方《千金方》云∶五痔有气痔,温、寒、湿、劳即发。蛇脱主之。牡痔,生肉如鼠乳,在孔中肠筛五又云∶治痔下部出脓血,有虫旁出孔窍方∶槐白皮一担,,纳釜中煮,令味极出,置木盆中,适寒温,坐其中,欲大便状,虫悉出,又云∶五痔方∶涂熊胆,取瘥乃止。神良,一切方皆不及此方。

又方∶煮桃根洗之。

《僧深方》治痔神方∶槐耳为散,服方寸匕,亦粉谷道中,甚良。

《录验方》治痔方∶白蜜涂之,有孔,以纳孔中。张温表上使蜀所得也。

又方∶煮槐根洗之。

《耆婆方》治人下部热,风虚结成痔,久不瘥,令人血下、面黄、瘦无力方∶白饧糖,但少少空腹食,瘥乃止。若是秋月弥宜。

《集验方》治痔方∶生槐皮(十两,削去上皮)一物熟捣,丸如弹丸,绵裹之,纳谷道中。

《极要方》云∶疗痔神方∶鳢(公)鱼三头,破腹取肠,炙少许令香,以绵絮裹之,纳谷道中。一饭,虫当出。食鱼肠更《葛氏方》治患肠痔每大便恒去血方∶常服蒲黄方寸匕,日三,须瘥止。

《医门方》疗五痔下血不止无问冷热者方∶黄(十二分)猪后悬蹄(十四枚,炙)发灰青布灰绯布灰(各二分)本(五分)大蜜丸,空腹饮送三十丸,日二,加至知为度,甚效。

《救急单验方》疗五痔方∶五月五日,取苍耳茎叶阴干,末,水服二方寸匕,日三。

又方∶牛角鳃烧末,和酒服方寸匕,日三,秘验。

《传信方》疗野鸡方∶上,以槐枝汤洗痔上,便已,艾灸上七壮,以知为度。

王及充迁安抚到官,乘骡(吐禾反)马入骆谷数日。而宿有痔疾,其状如胡爪贯于肠,热如火为槐汤洗一时出
治九虫方第十六

《病源论》云∶九虫者,一曰伏虫,状长四分;二曰蛔虫,长一尺;三曰白虫,长一寸;四曰为《承祖方》云∶九虫丸治百虫方∶牙子贯众蜀漆芜荑雷丸橘皮凡六物,分等,捣筛,蜜丸如大豆,浆服三十丸,日二,令虫下。

又云∶九虫散∶(古溉反)芦(二两,炙)贯众(一两)干漆(二两,炙)野狼牙(一两)凡四物,下筛,以羊肉羹汁服一合,日三匕。
治三虫方第十七

《病源论》云∶三虫者,长虫、赤虫、蛲虫也。犹是九虫之数也。长虫,蛔虫也,长一尺,也,疽(七余为。

《极要方》治三虫方∶取茱萸东行根大者,长一尺,麻子八升,捣之,细削茱萸,以八升酒合渍一宿,布绞去滓,之虫不下。

《葛氏方》治三虫方∶捣桃叶绞取汁,饮一升。
治寸白方第十八

《病源论》云∶寸白者,九虫内之一虫是也。长一寸而色白,形小扁。因腑脏虚弱而能发动又云∶食生鱼后即饮乳酪,亦令生之。

又云∶此虫长,长一尺,则令人死之。

《葛氏方》治寸白方∶多食榧子亦佳。

又方∶煮猪血,宿不食,明旦饱食之。

又云∶浓煮猪槟榔,饮三升,虫则出尽。

又云∶治蛔虫方∶用龙胆根,多少任意,煮令浓,去滓,宿不食,清朝服一二升,不过二。

又方∶捣生艾绞取汁,宿不食,朝饮一升,常下蛔。

《集验方》治寸白方∶取茱萸根,洗去土,切,一升,渍一宿,平旦分再服。取树北阴地根。

又方∶桑根白皮,切,三升,以水七升,煮取二升,宿无食,一顿服之。

《耆婆方》云∶野狼牙丸治寸白方∶野狼牙(四分)芜荑(四分)白蔹(四分)枸杞(四分)干漆(四分)上五味,捣筛,丸如完豆,服十丸。

《医门方》疗寸白方∶橘皮、野狼牙、雷丸,分等,末,可以汤服方寸匕,日一,虫当尽出。

《录验方》治寸白方∶大槟榔(二十枚,碎)葱白(切,一升)豉(一合)凡三物,以水五升,煮取二升半,顿服即下。今按∶酢研雄黄,敷之。

又方∶酢研槟榔子,敷之。

又方∶研胡桃子,敷之,并食之。

又方∶芥子如上。
治蛔(胡恢反,人腹中长虫也)虫方第十九

《病源论》曰∶蛔虫者,是九虫之内一虫也。长一尺,亦有长五六寸。或因腑脏虚弱而动,涎及《新录方》治蛔心痛发吐水方∶取楝树东南下根不露者,切一升,以水二升,煮取一升,去滓,服七合。十里久更温余者服者。

《广济方》云∶治蛔虫寸白方∶取酸石榴根,切,二升,入土七寸东引者槟榔十枚,碎,切,水七升,煮取二升六合,绞去《录验方》治蛔薏苡汤方∶薏苡根二斤,洗,细切。以水七升,煮得三升,先食,尽饮之,人弱老分二服之,一宿蛔悉
治蛲(如消反,腹中虫也)虫方第二十

《病源论》云∶蛲虫者,犹是九虫内一虫也,形甚小,如今之蜗牛状。因腑脏虚弱而致发动无不《极要方》云∶疗长虫赤虫寸白方∶薏苡根,以水七升,煮取二升,分二服。(今按∶《新录方》∶薏苡根二斤。)《范汪方》治蛲虫方∶楝实,淳苦酒中再宿,以绵裹之,塞谷道中,令入二寸,日易。

《录验方》治蛲虫在谷道中痒或痛方∶附子干姜芦茹蜀椒(各二两)捣筛,绵裹,纳谷道中,不过再敷,良。

《新录方》治病蛔虫或攻心痛如刺口吐清汁方∶捣生艾,绞取汁,宿勿食,清朝饮一升,当下蛔。

又方∶取楝木根,锉之,以水煮令浓,赤黑色,以汁合米,煮作强麋,宿勿食,清朝食之,又方∶薏苡根二斤,细锉,水七升,煮取三升。分再服。亦可以作糜。
卷第八(手足)
香港脚所由第一

《病源论》云∶凡香港脚病,皆由感风毒所致也。初得此病,多不即觉,或先无他疾,而忽得苏敬论云∶夫香港脚为病,本因肾虚,多中肥溢、肌肤虚者。无问男女,若瘦而劳苦,肌肤薄亦唐侍中论云∶凡香港脚病者,盖由暑湿之气郁积于内,毒厉之风吹薄其外之所致也。

徐思恭论云∶此病多中闲乐人,亦因久立冷湿地,此病多或踏热来即冷水浸脚;或房室过度又云∶清湿袭虚,则病起于下;风雨袭虚,则病起于上。又身半以上者风中之,身半已下者《千金方》云∶凡四时之中,皆不得久立久坐湿冷之地,亦不得因酒醉汗出,脱衣靴帽,当又云∶夫风毒之气,皆起于地。地之寒暑风湿,皆作蒸气。足常履之,所以风毒之中人也,《极要方》云∶此病有数种,有饮气下流以成香港脚,饮气即水气之冲;亦有肾气先虚,暑月承以冷水洗脚,湿气不散;亦有肾气既虚,诸事不节,因居卑湿,湿气上冲,亦成香港脚。
香港脚形状第二

《病源论》云∶凡香港脚,此病之初甚微,饮食嬉戏,气力如故,当熟察之。其状,自膝至脚,有不仁,或若痹,或淫淫(流貌也)如虫行所缘,或脚指及膝胫洒洒(桑冷反,恶寒貌)尔饮食者反)腹,径上撞心,气上者;或有举体转筋者,或壮热头痛者;或心胸冲悸,寝处不欲见明者;或唐临论云∶此病形候大同小异,或脚冷疼痹;或行忽屈弱;或两胫肿满;或脚渐枯细;或心官反,苏敬论云∶香港脚复发,或似石发,恶寒壮热,头痛,手足逆冷;或似疟发,发作有时;又似徐思恭论云∶此病甚微,饮食、嬉戏、语笑、气力如故,卒不令人觉,当熟之。察其状,自行;之微。

又云∶初得之时,即或脚趺肿,或脚胫肿,渐渐向上;或由来不肿,唯缓弱顽痹,行卒屈倒《短剧方》云∶风毒中人,多不即觉,或因众病乃觉也。其状或有见食呕吐,憎闻食臭;或或病《千金方》云∶风毒之中人也,或见食呕吐,憎闻食臭;或有腹痛下利;或大小便秘涩不通体酷体挛(又云∶其人本黑瘦者易治,本肥大肉浓赤白者难愈。

《极要方》云∶香港脚皆令人脚胫大肿,趺肿重闷。甚者上冲心,肿满闷,气短。中间有干湿者二香港脚∶湿者脚肿;干者脚不肿,渐觉枯燥,皮肤甲错。须细察之。
香港脚轻重第三

苏敬论云∶凡香港脚病,多以春末夏初发动。春发如轻,夏发更重,入秋少轻,至冬自歇。

大略如此,亦时有异于此候者。

又云∶凡香港脚脉有三种∶以缓脉为轻,沉紧为次,洪数为下。沉紧者多死,洪数者并生,缓者不治自瘥。

又云∶又有不肿而缓弱,行卒屈倒,渐至不仁,毒瓦斯阴上攻心便死。急不旋踵,宽延岁月耳唐临论云∶此病胁满气上便杀人,急者不全日;缓者或一二月。初得此病,便宜速治之,不徐思恭论云∶凡香港脚,皆有阴阳,若两脚及髀(傍例反)已来肿满,按之应骨,骨疼又痛者,得,不废行,《千金方》云∶凡小觉病候有异,即须大怖畏,决意急治之,伤缓气上入腹,或肿或不肿,停;《短剧方》云∶脉浮大者,病在表;沉细者,病在里;其脉浮大紧快者,三品之中最恶脉也《葛氏方》云∶脚弱,满而痹至少腹,而小便不利,气上者死。
香港脚姑息法第四

唐临论云∶姑息香港脚法,依此消息,必得气愈。第一忌嗔,嗔即心腹烦,心腹烦即香港脚发动胫尤将息侵犀,七度,顶少似瘥。若洗之苏敬论云∶凡香港脚病患,不能永瘥,要至春夏,还复发动。夏时腠理开,不宜卧睡,睡觉令有身加浮
香港脚疗体第五

徐思恭论云∶香港脚之为病,疗乃百途,故须原始要终,察其形证,依穴针灸,当病用药,如又云∶香港脚之病,不同余病,一患以后,难瘥易发,诊候不同,诊病进药,随其冷热,旬月苏敬论云∶夫疗香港脚者,须顺四时,春秋二时,宜兼补泻。夏则疾盛,专须汗利,或十月以还者又云∶今略述病有数种,形证不同,每发瘥异,为疗亦殊。前用经效,后用便增剧,一旬之又云∶且香港脚为病,不同余病。风毒不退,未宜停药。比见病者,皆以轻疾致毙。或以病小又云∶凡香港脚病,虽苦虚羸,要不可补,补药唯宜冬月,酒中用之,丸散不可用补,服必胪法用非病又云∶夫有香港脚病,不可常服补药。多服(或本)令鼓胀坚实,则难救也。每月之中,须三五热烦常饮酒又云∶冷毒盛胀,即服金牙;热毒盛胀,须服紫雪;平平胀者,单用槟榔饮亦善;患脚气人又云∶诸毒瓦斯所攻,攻内则心急闷,不疗至死。若攻外毒出皮肤,出皮肤则不仁,不仁者,膏摩之瘥。若未出皮肤,在营卫刺痛者,随痛处急宜灸之三五炷,即瘥。不必要在孔穴也《医门方》云∶夫疗香港脚,或兼诸病者,则依证以当药对之。若乳石动发则以理石药疗之;又云∶香港脚毒盛,非热药不解。如热甚毒发忪(职容反)悸,头面热闷者,荆根汁或竹沥夜卧
香港脚肿痛方第六

徐豉酒方∶若能常饮此酒,极利腰脚。岭南常服此酒必佳。及卑湿处亦服弥好。又恐有香港脚好豉三升,以美酒一斗渍。先取豉,三蒸三曝干,纳酒中渍三宿,便可饮,随人多少。

用滓苏疗香港脚初发从足起至膝、胫肿骨疼方∶蓖麻(切,捣,一斗许。)上,蒸热,铺脚肿处,浓裹。旦二度易,二三日即消。(《医门方》同之。)又方∶秋月以前有蓖麻,冬月则无。宜取蒴根,切,捣碎,酒糟和蒸,令热,三分根,一唐方∶以酒糟一斗,和盐分作二分,炒令热,将故袜乳裹铺之,便易,以肿消为度。

唐疗香港脚挛不能行,及干疼不肿自渐枯消,或复肿满缓弱方∶取桃柳槐桑五木枝叶,各切一斗,以水一斛,盐五升,煮取五斗,浸将膝以下,一捋得,唐浸脚肿满及缓弱不仁疼痹等方∶柳树白皮,细锉如棋子三大升(斗,或本),以水一大石,煎取六大斗。取一小瓮可受一石者三度即唐又方∶如大肿不能行动者方∶取松木,锉三石,赤小豆二斗。以水六石,煮取一石七斗。取小瓮子依前柳树皮浸法。

若患唐又方∶如香港脚闷者方∶以水煮梓枝叶为汤,添冷水、盐等,和渍脚,气散少快使脚遂不闷,大验。

徐疗气肿方∶赤小豆一大升。

上,生研大麻子汁,煮前件小豆令烂,每服令尽此一升,甚疗肿。不限服数,多少以消为度唐熏香港脚法∶上,以笼两具,以锻石摩捣,泥裹,安二寸灰,灰上着炭火,火上着二寸灰,灰上着好盐,苏疗大肿不能行动者方∶以水煮松木作汤,渍捋,大神。

又方∶捣乌麻碎,水煮渍将脚,亦大验。

唐疗手足肿满洪直者大豆煎方∶大豆(一升,净择)树皮(一握)橘皮(三两)桑根白皮(一握)紫苏茎(一握)先以水四斗煮大豆,取二斗汁,去滓,待清,别以清酒七升,共豆汁合煮前件药,取七升,唐疗气肿上至腰小便涩诸药不效宜服方∶葶苈子(二两碎)大枣(十四枚,去核)上,以水三升,煮取二升分三服。

《经心方》治脚肿满、步行不能、众恶毒水肿牵牛子丸方∶大黄(二两)朴硝(三两,炼)牵牛子(七两,熬)桃仁(二两,去心熬)干姜(二两半)上七物,捣下筛,以蜜和,杵舂万杵,服如梧子二十丸,以微利为度,愈肿即止。不瘥尽剂《极要方》疗香港脚遍身肿方∶大豆(三大升,以水二斗,煮取五升汁,去滓)桑根白皮(一握,切)槟榔(三七颗,擘)茯苓(二两)上,以前豆汁,浸经一宿,煮取二升,去滓,添酒二合纳药中,随多少服之。利多量减服。

《葛氏方》云∶若胫已满捻之没指者方∶酒若水煮大豆饮汁。又恒食小豆。

又云∶若步行足痛不能复动方∶蒸大豆两囊盛,更燔以熨之。

《本草》云∶鲤鱼生煮食之,主水肿脚满。

《陶景本草经》疗香港脚满∶急服牵牛子,得小便利,无不瘥。

《崔禹锡食经》治脚肿方∶煮蔓荆根,蒸敷之,即消。

又方∶煮茄苗叶,涛脚尤验。
香港脚屈弱方第七

唐云∶若香港脚屈弱或不能语者,宜服此金牙酒,此酒最为香港脚之要。

金牙(碎,绵裹)细辛斑草(炙)干地黄干黄(姜)防风附子(炮)蛇床子蒴上十四味,以酒四斗,渍之七日,饮二三合,稍加之。以知为度。此酒最为香港脚之要,忌如苏白杨树皮渍酒疗香港脚肿、心胸满、手足缓弱不能起止方∶白杨树白皮(切,一大升。熬黄,绢袋盛)上,以酒一斗五升,渍三日,宿温服三四合,加日二三服,加至八合。能者加之;不能者减《千金方》八风散治风虚,面青黑土色,日月光不见,香港脚痹弱方∶苁蓉(八分)乌头(二分)钟乳(四分)薯蓣(四分)续断(四分)黄(四分)麦门冬(辛(四分)龙胆(四分)秦胶(四分)石苇(四分)柏子仁(四分)牛膝(四分)菖蒲(四分)杜仲(四分)茯天雄(六分)干地黄(四分)蛇床子(分)干姜(四分)萆(四分)人参(五分)三十三味,下筛,酒服方寸匕,日三,不知加至二匕。

《葛氏方》治香港脚疼痹,屈弱不仁,时冷时热方∶先取好豉一升,三蒸三曝干,以好酒三升渍之三宿便可,饮随人多少,以滓薄脚,其热得小又方∶以酒煮豉服之。

《本草》云∶豺皮熟之,以缠病上,瘥止。(主冷痹香港脚)
香港脚入腹方第八

苏方∶水研紫雪,服之立下。(今按∶紫雪方,鉴真云∶若香港脚冲心,取一少两,和水饮之唐气上急闷欲绝者,服生姜汁方∶上,生母姜合皮,捣取二升许汁,平旦温顿服之,立瘥。

《广利方》治卒香港脚冲心烦闷乱不识人方∶取大豆一大升,拭去土,以水三大升,浓煮取汁,顿服半升,不定,良久更服半升,即定。

苏徐木瓜汤,若毒瓦斯攻心,手足脉皆绝,此亦难济,不得已作此汤,疗十四五愈。方∶吴茱萸(六升)木瓜(二颗)上,以水一斗二升,煮取三升,去滓,分三服,相去十五里许,或利,或汗,便活。曾苦毒服之苏治气入人腹攻心,诸脉并绝不识人面。服牛尿方∶乌特沙尿(一大升)上,取新尿温者服令尽,即得利。因便眼明识人,虽不能大(太欤)疗气,然一时救死,余无徐治肿从脚始转上入腹则杀人猪肝散方∶生猪肝(一具,细切)。

上,以啖蒜齑,食令尽。大肝不尽,分脍二服即消。

又方∶以水煮单食之。

又方∶苏云,猪肉细切作脍,捣蒜齑食,日三,数日即气毒止,风除能行。

唐犀角汤,疗肿已消,犹遍身顽痹,毒瓦斯已入冲心,闷,吐逆,不下食,或肿未消,仍有犀角(二两)大枣(七升枚,碎)香豉(一升,绵裹)紫苏茎(一握)生姜(二两)上,以水八升,煮取二升八合,分三服,相去十里,频服三剂,以气下为度。

唐若气攻心此方甚散气极验∶大槟榔(七枚)生姜(二两)橘皮吴茱萸紫苏木瓜(各一两)上,以水三升,煮取一升三合,分再服。

《极要方》疗香港脚攻心、闷腹胀、气急欲死方∶吴茱萸(三升)木瓜(切,三合)槟榔(二十颗,碎)竹叶(切,二升)切,以水一斗二升,煮取三升,分温三服,得快痢即瘥,慎生菜、热面、乔麦、蒜。
香港脚胀满方第九

苏徐疗身体浮肿,心下胀满,短气小便涩,害饮食。方∶大豆一斗,以水三斗,煮取一斗七升,去豆,纳清酒一斗和煎之,令得一斗七升许,调适寒温,一服一升,日三服,甚佳。(今按∶《耆婆方》∶大豆三升,酒一升,水无升数。)苏桑根汤,主通身体满,小便涩,上气,心下痰水,不能食,食则胀满者方∶桑根白皮(五升)大豆(五升)上,以水三斗,煮取一升,去滓,分三服之。

苏徐疗肿已入股至腹胀小便涩少者方∶大麻子一斗,熬细研赤小豆五升上,以水三斗,煮豆烂,饮汁食豆,日三,不食余食。不瘥更为之。

又方云∶树白皮,切,一大斗,桑根白皮切一大斗入土深者。上以水五大斗,煮取二大斗烂,止徐唐葶苈丸疗小便涩少腹满不下食饮方∶葶苈子(五两,缓火熬令紫色)杏仁(二两半,去皮尖,熬令紫色)大枣(三十枚,去皮核取肉)先捣葶苈子一万杵,别取杏仁、枣肉,和捣一千杵,然后取葶苈子和捣一万杵,为丸。平五丸,丸
香港脚冷热方第十

唐治香港脚热烦,口干头面热闷。方∶好香豉一升,以水四升,煮取二升,停冷去滓,顿服之。(《极要方》同之。)苏治若觉飧气攻喉方∶当食生茱萸五十粒,即散。

徐疗冷气非冷热亦兼治方∶大蒜一升,去皮心,以酒三升,缓火煎汁一升许,去滓,每服一盏子,日三。大大去气,但唐云∶已觉着香港脚,宜服此方∶蒜(三升,去心,切,熬令黄色)豉(一大升,熬令香)桃仁(小一升,去皮,熬令紫色)上三味,合和,生绢袋盛,以酒一斗渍之,夏三日,冬七日,初服半升,渐加至二升。量增
香港脚转筋方第十一

论云∶凡香港脚初转筋者,灸承筋承山二穴。

《龙门方》疗脚转筋及入腹方∶取木瓜子根茎煮汤服,并验。

又云∶疗转脚筋及入腹方∶手构随所患脚大拇指,灸当脚心急筋上七壮。

又云∶筋已入腹者,令患人伏地,以绳绊两脚趺上踝下,两脚中间出系柱,去地稍高,患者《华佗方》治转筋方∶以白戢煮粉令一沸,因以洗足腓,至足立愈。(《短剧方》同之。)
香港脚灸法第十二

苏唐云∶凡香港脚发有阴阳表里,随状疗之。不可要依古方也,患阳疗阴,病表救里,皆为重先发下,足小泉、风摩之。若上下遍发,不知的处者,宜灸上廉、下廉、条口、三里,各灸一二处,以通泄之。

其用药色赤者三里、阳陵泉二穴∶在膝外侧骨下宛宛陷中是。(苏徐)。

绝骨二穴∶在外踝正上寻小骨头绝陷中是。(同前)。

风市二穴∶平立垂手当中指头中两筋间是。(同前)。

昆仑二穴∶在外踝后跟骨上陷中是。(同前)。

阳辅二穴∶在绝骨前半寸少下是。(苏徐云∶明堂无绝骨名,有阳辅)。

上廉二穴∶在三里下三寸是。(苏徐)。

下廉二穴∶在条口下一寸是。(苏徐)。

条口二穴∶在上廉下二寸是。(苏徐)。

大冲二穴∶在足大指本节后二寸是。或云一寸半。(苏徐)。

犊鼻二穴∶在膝盖上外角宛宛中是。(苏徐)。

膝目二穴∶在膝盖下两边宛宛中是。(苏)。

三里二穴∶在膝盖骨头侧下骨外三寸下宛宛。

曲泉二穴∶在膝内屈纹头是。

阴陵泉二穴∶在伸脚膝内侧骨下宛宛中是。(苏徐)。

中都二穴∶在阴陵泉三交中间是。(苏)。

三交二穴∶在内踝上三寸是。(苏徐云∶名大阴)。

复留二穴∶在内踝上三寸是。(苏徐云∶名承命)。

少阳二穴∶在内踝后一寸,动筋中是。(徐)。

三阴二穴∶在内踝上八寸,骨下陷中是。(徐)。

阴跷二穴∶在内踝下向前宛宛中是。(徐)。

委中二穴∶在膝后屈中央是。(苏徐)。

承筋二穴∶在当中心满中是。(苏徐)。

承山二穴∶在肠下际分肉间陷中是。(苏徐)。

涌泉二穴∶在脚心是。(苏徐)。

上件穴并要不能灸,其最要有三里、绝骨、承筋、大冲、昆仑、涌泉。患者不可不灸。

凡患香港脚,法皆是春发、夏甚、秋轻、冬歇,大法春秋宜灸,冬瘥可行,夏都不可。夏既疮灸一凡香港脚病大论毒从下上,亦从上向下者,或云灸上毒便止,误矣。比见毒瓦斯攻处,疼痛如刺凡所冲如贼欲出,得穴即出,岂在大门也。风气所攻,亦复如是,皆此经试,万不失一,必《葛氏方》云∶其灸法,孔穴亦甚多,恐人不能悉知。处今止疏要者,必先从上始,若直灸大椎一穴(灸百壮),肩井二穴(各灸百壮),膻中一穴(灸五十壮),巨阙一穴(灸百壮)。

凡灸此上部五穴,亦足以泄其气。若能灸百会、风府、胃脘及五藏俞亦佳。视病之宽急耳。

次∶风市二穴(灸百壮),三里二穴(灸二百壮),上廉二穴(灸百壮),下廉二穴(灸百壮),绝凡此下部十穴并至要,犹余伏菟、犊鼻耳。凡灸此壮数不必顿毕,三日中报之令竟。

《千金方》云∶凡病一脚,则灸一脚;病两脚,便灸两脚也。

又云∶初得脚弱便速灸之,并服竹沥汤,灸讫可服八风散,无不瘥者,唯急速治之。若人但灸而不服散,服散而不灸,如此者,有半瘥半死。虽得瘥者,或至一二年度复更发动。

觉得便依此法速灸之。服散者,治十十愈。竹沥汤八风散在上。
香港脚禁忌第十三

苏唐论云∶醉酒房室,久立冷湿地,船行水气,夏月屋中湿气,热蒸气,劳剧,哭泣忧愦,又云∶不用乘马,若能步行劳动,其香港脚自然渐瘥。

又云∶昼日莫多卧,须力起遨游,舒畅情性,勿恣睡也。

《千金方》云∶凡香港脚之病,极须慎房室,又忌大怒。

《极要方》云∶凡患香港脚,尤不宜眠睡,宜数数行散浮肿气,不宜服补药,每月须宣泻佳。
香港脚禁食第十四

苏唐论云∶不宜食面、羊肉、萝卜、葵、蔓菁、韭。

又云∶不得食酢饼。

《千金方》云∶羊肉、牛肉、鱼、蒜、蕺菜、菘菜、蔓菁、瓠子、酒、面、酥、油、乳糜、又云∶不得食诸生果子、酸酢之食,犯之者皆不可瘥。
香港脚宜食第十五

苏唐云∶常宜食犊肉、犊蹄、鲤鱼、鳢鱼、猪肉、兔肉、葱、芥、薤、韭等。及猪肝、肺。食法先汤中KT使才熟,切作脔,以酱汁和水,并着一抄半葱、姜、椒煮令极熟。每食下饭大《千金方》云∶唯得粳、粱、粟、米、酱、豉、葱、韭、薤、椒、姜、橘皮。

《医门方》云∶食生牛乳、生栗子,诸下气之物为佳。

《陶景本草》注云∶昔脚弱人往栗树下,生食数升,便能起行。

《食经》云∶昆布(崔禹云∶治手脚疼痹),鹿肉(苏注云∶主四肢不随),鲤(崔禹云∶主脚
治足肿方第十六

《病源论》云∶肿病者,自膝已下至踝及指,俱肿直是也,皆由血气虚弱,而邪伤之,经络痞涩而成。亦言江东诸山县人多病肿,云彼土有草,名肿草,人行误践触之,则令病肿之。

《录验方》治脚肿方∶掘坚实地坎,作可容脚深,没脚烧坎令赤,以尿投坎中,以绳缠膝下,仍以甘刀遍披肿上《葛氏方》治足忽得痹病,腓胫暴大如吹,头痛寒热,筋急。不即治之,至老不愈方∶随病痛所在左右足,对内踝直下白肉际,灸三壮即愈。后发更灸故处。

又陶氏∶初觉此病之始,股内间微有肿处;或大脉胀起;或胫中拘急。煎寒不决者,当检按若若数日不止,便以甘刀破足第四第五指间脉处,并踝下骨解,泄其恶血,血皆作赤色,去一令服大黄膏方∶大黄、附子、细辛、连翘、巴豆、水蛭炙一两,苦酒淹一宿,以腊月猪膏煎三上白头公酒方∶白头公(二两)甘草(一两)牛膝(二两)海藻(二两)石斛(一两)干地黄十物,以酒二斗渍五日,服一合,稍至三四合。

又,摩冶葛膏亦佳。

《耆婆方》刺内踝上大脉,血出即瘥。

又方∶灸外踝尖。
治尸脚方第十七

《病源论》云∶尸脚者,脚跟坼破也,亦是冬时触犯寒气,所以如然。又言脚踏死尸所卧地《葛氏方》治脚无冬夏恒坼裂者,名尸脚。踏死尸所致方∶取鸡矢一升,水二升,煮数沸,渍洗之半日乃出,数作瘥。(《千金方》同之。)《新录方》治尸脚方∶大麻子煮汤渍之。

又方∶捣马苋汁洗或煮渍之。

又方∶煮蔓菁根渍之。

又方∶车脂涂之。
治肉刺方第十八

《病源论》云∶肉刺者,脚趾间生肉如刺,谓之肉刺。由着靴急小,趾相揩而生。

《新录方》治肉刺方∶数数涂酥也。

又方∶封糖稍刮也。

又方∶盐汤温渍之。

又方∶酢摩至消。

又方∶浆汁涂刮去。

又方∶烧金银钗烁之。

又方∶薰陆香、硫黄等分,合研量大小,可刺着烙之。

《救急单验方》疗脚下肉刺方∶削令头破血出,取丹如米许附讫,裹,瘥。

又方∶以刀子割去刺,即以书墨研之,数十回,永大瘥验。
治手足冻肿疮方第十九

《病源论》云∶冻烂肿疮,严冬之月,触冒风雪寒毒之气,伤于肌肤,血气壅涩,因即瘃冻《葛氏方》治手足中寒雪冻肿疮烂方∶车膏,温,令热以灌之,洋腊杂蜜亦佳。

又方∶温咸菹汁渍之。

又方∶热煮小便以渍之。

又方∶烧黍粟若麦,作灰,以水和煮,令热穿器,盛于筒中,滴疮上半日中为之。

《千金方》云∶若冬月冒涉冻冰,面目手足缘坏,及始热痛欲瘃者∶取麦叶,令煮浓,汁,热洗之。

又云∶冻指欲堕方∶马矢三升,水三升,煮令沸,渍半日,愈。(《经心方》同之。)《新录方》∶熬曲散粉上。

又方∶煮松叶洗之。

《集验方》治冻疮方以腊洋灌之。
治手足皲裂方第二十

《病源论》云∶皲裂者,肌肉破也。言冬时触冒风寒,手足破,故谓之皲裂也。

《葛氏方》治冬天手足皲裂血出及瘃冻方∶取麦苗煮令浓热的的尔,以洗渍之。

《千金方》治手足皴劈破裂血出疼痛方∶猪脂着热酒以洗之,即瘥。

《集验方》手足瘃坏方∶蜀椒四合。以水一斗,煮三沸,去滓,以洗渍之。

《苏敬本草经》注∶嚼白芨,以填之,效。

《范汪方》∶取葱叶萎黄叶,煮以渍,洗之良。

《新录方》∶咋蒜封之。

又方∶车脂涂之。
治手足发胝方第二十一

《病源论》云∶人手足忽然皮浓涩而圆强如茧者,谓之胼(步田反)胝。此由血气沉行,不荣《葛氏方》手足忽发胝方∶取粢米粉,铁枪熬令赤,以众人唾和之以涂上,浓一寸,即消。(《范汪方》同之。)玉箸葙方∶以盐涂胝上,令牛舐之不过三。

又方∶作艾炷灸其上三壮。

《新录方》∶以温尿渍,瘥。
治手足逆胪方第二十二

《病源论》云∶逆胪者,手足爪甲际皮剥起,谓之逆胪。风邪入于腠理,血气不和故也。

《千金方》手足逆胪方∶亚青珠(一分)干姜(二分)捣,以粉疮上,日三。〔今按(一本无)∶《葛氏方》是名琅者,非真珠,亦以猪脂和涂之《枕中方》治人手足粗理方∶取榆树孔中水洗,即细如指。
治代指方第二十三

《病源论》云∶代指者,其指先肿,热痛,其色不黯,然后方缘爪甲边结脓,剧者爪甲故肿《短剧方》治代指法∶单煮甘草KT之,若无甘草,纳芒硝汁渍之,但卤得一种冷药药草菜汁KT渍之。〔今按∶(本《葛氏方》代指方∶煮地榆根作汤,渍之半日,甚良。

又方∶以指刺炊饭中二七遍,良。(以上《千金方》同之。)又方∶以泥泥指,令通匝浓一寸许,以纳热灰中炮之,泥燥候视,指皮绉者即愈也;不绉者《千金方》代指方∶麻沸汤渍之,即愈。

又方∶先刺去脓血,灸皮令温,以缠裹周匝,痛止便愈。

又方∶取萎黄葱叶煮沸,渍之。

又云∶割甲侵肉成漏不瘥方∶敷矾石末,裹之,以瘥为限。

又方∶捣鼠粘草根,和腊月猪脂敷,取瘥止。

《新录方》代指方∶酢和热气灰封,日二三。

又方∶盐汤渍之,良。

《集验方》代指方∶单煮甘草渍之。

又方∶用芒硝汁渍之。

《僧深方》代指方∶作艾炷正灸痛上七壮。
治指掣痛方第二十四

《葛氏方》指忽掣痛不可堪转上入方∶灸病指头七壮,立瘥。(《千金方》同之。)又云∶指端忽发疮方∶烧铁令赤,以灼之。

又云∶卒五指筋挛急不得屈伸方∶灸手踝骨上数壮。

《千金方》指掣痛方∶酱渍,和蜜,温涂之,愈。(《经心方》同之。)又云∶指疼故脱方∶猪脂和盐,煮令消,热纳指爪中,食久瘥。

医心方卷第八医心方卷第八背记天养二年二月以宇治入道太相国本移点。

移点∶(少内记藤原中光。)比校∶(助教清原定安。)移点比校之间所见及之不审,直讲中原师长医博士丹波知康重成等相共合医家本,异文殿所宇治本初下点∶(行盛朝臣朱星点墨假字)重加点∶(重基朝臣朱星点假字勘物又以朱点句于儒点)御本不改彼样,令点移之。

以上第一页秒(禾穗芒也,分秒定禾数,十二秒而当一粟,当一寸。《淮南子》作“KT”。)
卷第九
治咳嗽方第一

《病源论》云∶咳嗽者,肺感于寒,微者则成咳嗽也。肺主气,合于皮毛,邪之初伤,先客各喘息有音,甚则唾血;乘夏则心受之,心咳之状,咳则心痛,喉仲介介如哽,甚则咽肿喉痹;乘春则肝受之,肝咳之状,则两胁下痛,甚则不可以转,两脚下满;乘至阴则脾受之,脾咳之状,咳则右胁下痛,喑喑引于背,甚则不可动,动则咳;乘冬则肾受之,肾咳之状,咳则腰背相引而痛,甚则咳涎。此五脏之咳也。五脏咳久不已,传与六腑。脾咳不已,则胃受之,胃咳之状,咳而呕,呕甚则长虫出;肝咳不已,则胆受之,胆咳之状,呕胆汁;肺咳不已,大肠受之,大肠咳之状,咳而遗屎;心咳不已,则小肠受之,小肠咳之状,咳而失气,气者与咳俱;肾咳不已,膀胱受之,膀胱咳之状,咳而遗尿;久咳不已,三焦受之,三焦咳者,咳而肠满,不欲食饮,此皆聚于胃,关于肺,使人多涕唾而面浮肿,气逆也。

《千金方》云∶问云∶咳病有十,何谓?师曰∶有风咳,有寒咳,有支咳,有肝咳,有心咳异?咳则咳脐《僧深方》云∶热咳,唾粘而如饴;冷咳,唾清澄如水。

《医门方》云∶夫酒客咳者,其人必吐血,此为坐极饮过度所致,难疗。

《极要方》云∶此病有数种,有冷热咳嗽,有肺萎嗽,有肺痈嗽,有水气嗽。若有本性非热嗽,唾无出息,腹满闷,甚者头面有气过久,重者身体皆肿,此是水气嗽也。

《葛氏方》云∶上气喘嗽,肩息不得卧,手足逆冷,及面浮肿者,死。

《僧深方》紫菀丸,治咳嗽上气,喘息多唾方∶紫菀款冬花细辛甘皮(一名橘皮)干姜(各二两)上五物,丸如梧子三丸,先食服,日三。

又方∶如樱桃大,含一丸,稍咽其汁,日三。新久嗽,昼夜不得卧,咽中水鸡,声欲死者,《录验方》小紫菀丸,治上气夜咳逆多浊唾方∶干姜(二两)甘皮(二两)细辛(二两)紫菀(三分)款冬花(二两)附子(二两)凡六物,下筛,蜜和丸如梧子,先食,服五丸,日二。

大紫菀丸,治上气咳逆方∶紫菀(二两)五味子(二两)橘皮(二两)香豉(二两)干姜(二两)桂心(二两)杏仁(凡十一物,捣筛,蜜和丸如梧子,一服五丸,日二,夜含一丸如杏核大,咽汁,昼更含。

《承祖方》治上气咳嗽杏仁丸方∶杏仁(一升,熬)干姜(二两)细辛(二两)紫菀(二两)桂心(二两)捣下筛,杏仁别如脂,合和以蜜丸,服如枣核一枚,日三。

《广济方》疗咽喉干燥,咳嗽,语无声,桂心散方∶桂心(六两)杏仁(三两)捣筛,以绵裹一枣大,含,细细咽汁,日三夜二,忌生葱油腻。

《范汪方》治咳紫菀牙上丸方∶紫菀〔一分(一方一两)〕干姜(一分)附子(一分)桂心(一分)款冬花(一分)细辛(凡六物,冶筛,和蜜丸,丸如小豆,先食,以二丸着牙上,稍咽,日再,不知稍增。

又云∶投杯汤,治久咳上气,胸中寒冷,不能得食饮,卧不安床,牵绳而起,咽中如水鸡声款冬花〔四十枚(一方二十枚)〕细辛(一两)紫菀〔二两(一方一两)〕甘草〔二两(一方三两)〕凡十物,咀,以水八升,煮得二升,先食,适寒温,再服,温卧汗出即愈。(今按∶《录验方》∶麻黄三两,甘草三两,杏仁百枚。凡三物,切,以水六升,煮取二升五合,未食分三服。)《短剧方》治咳嗽上气,呼吸攀绳,肩息欲死覆杯汤方∶麻黄(四两)甘草(二两)干姜(二两)桂肉(二两)贝母(二两)凡五物,以水八升,煮取二升,再服即愈。(今按《范汪方》云∶苗诡士孙粟,男儿四岁,母又云∶泼雪汤,治上气不得息卧,喉中如水鸡声,气欲绝方∶麻黄(四两)细辛(二两)五味子(半升)干姜〔四两(一,或本)〕半夏(四两)桂心(凡六物,以水一斗,煮取三升,分服一升,投杯即得卧,一名投杯汤。令得汗,汗多喜,不杏仁三两细辛三两生姜十两半夏四两七物,以水一斗,煮取三升,分三服,亦可五合(七合)服,渐渐加之。〕《千金方》云∶夫酒客咳者必致吐血,坐极饮过多所致,浓朴大黄汤主之。

浓朴(一尺)大黄(六两)枳实(四两)三味,水五升,煮取二升,分再服之。

《本草》云∶咳逆,鹿髓,以酒服之甚良。

又云∶膏,酒和三合服之,日三。又云∶食鲤鱼肉也。

《孟诜食经》云∶疗卒嗽味方∶梨一颗,刺作五十孔,每孔中纳一粒椒,以面裹,于热灰中椒七枚。合煎含咽之。)又方∶梨去核,纳苏蜜,面裹,烧令熟食之,太良。

又方∶割梨肉于梨苏中,煎之,停冷食之。(今按∶《朱思简食经》云∶凡用梨治咳,皆须《葛氏方》治卒得咳嗽方∶皂荚、干姜、桂心分等捣丸,服三丸,日三。

又方∶生姜汁(一名干姜)、百部汁和煎,服二合。

《集验方》治忽暴气嗽奔喘,坐卧不得、并喉里KT声,气欲绝方∶麻黄(三两)杏仁(四两)干姜叶(二两)紫(茈)胡(四两)橘皮(二两)切,以水六升,煮取二升半,分三服。

《张仲景方》治三十年咳大枣丸方∶大枣(百枚,去核)杏仁(百枚,熬)豉(百二十枚)凡三物,豉、杏仁捣令相得,乃纳枣,捣令熟,和调丸如枣核一丸,含之,稍咽汁,日二《耆婆方》治三十年咳嗽方∶细辛紫菀麻黄甘草干姜(各四分)五味为散,白饮服一方寸匕,日三。

《效验方》款冬花分丸治三十年咳上气呕逆面肿方∶杏仁(三分,熬)干姜(三两)柑皮(一两)麻黄(三两)甘草(二两)款冬花(二两)凡六物,冶下筛,以蜜和丸如梧子,先食,服三丸,日三。

《僧深方》治新久嗽芫花煎方∶芫花(二两,末)干姜(二两)白蜜(二升)凡三物,纳于蜜中,微火煎,服如枣核一枚,日三。

熏咳嗽法∶《录验方》治久咳熏法∶蜡纸一张,熟艾薄遍布纸上熏,黄末一两,款冬花末二分。前遍布剂,一《千食方》治咳熏法∶细熟艾薄布纸上,纸广四寸,复以硫黄(《本草》∶硫黄一名)末薄布之,灸咳嗽法∶《僧深方》云∶灸近两乳下黑白肉际纹百壮,即日愈。(《范汪方》同之。)又方∶以绳当乳头围周身,令前后平正,当乳脊骨解中,灸之九十壮。

又方∶横度口中折绳,从脊灸绳两边,灸八十壮,三日报毕。

又方∶从大椎数,下行第五节下、第六节上穴间中一处灸,随年壮,并治上气秘方。

《短剧方》云∶灸肩井穴百壮,在肩上陷解中大骨前。

又方∶灸大杼穴随年壮,在项第一椎下两旁各一寸半陷者中。

又方∶灸肺俞,随年壮,在第三椎下两旁各一寸半。

又方∶灸风门热府穴百壮,在第二椎下两旁各一寸半。

又方∶灸天突穴五十壮,在结喉下五寸宛宛中。

又方∶灸玉堂穴百壮,在紫宫下一寸六分。

又方∶灸膻中穴五十壮,在玉堂下一寸六分,两乳间陷者中。

又方∶灸云门穴五十壮,在臣骨下气户两旁各二寸陷者中,横去旋机旁六寸。

又方∶灸中府穴五十壮,肺募也,在云门下一寸。

又方∶灸巨阙穴五十壮,在鸠尾穴下五分。

又方∶灸期门穴五十壮,在去巨阙五分举臂取之。(以上《千金》同之。)又方∶灸输府穴,在旋机旁各二寸。

又方∶灸或中穴,在输府下一寸六分。

又方∶灸气户穴,在去旋机旁各四寸。

《葛氏方》云∶度手拇指,中折以度心下,灸三壮即瘥。
治喘息方第二

《病源论》云∶肺主气,邪乘于肺,则胎胀,胀则肺管不利,不利则气道涩,故气上喘逆,《葛氏方》治卒上气鸣息便欲绝方∶捣韭,绞,饮汁一升许立愈。

又方∶人参末,服方寸匕,日三。

又方∶桑根白皮细切三升,生姜切半升,吴茱萸半升,酒五升,合煮三沸,去滓,尽服之。

《极要方》疗上气、气逆满,喘息不通,呼吸欲死,救命汤方∶麻黄(八两,去节)甘草(四两,炙)大枣(三十枚)夜干(如博子二枚)上,以井花水一斗,煮麻黄再沸,纳余药,煮取四升,分四服,入口即愈。

《医门方》治上气喘息不得卧,身面肿,小便涩方∶葶苈(一两,熬,捣如泥)大枣(三十枚,擘)水三升,煮取一升,纳葶苈,煮取五六沸,顿服,微利,瘥。

《效验方》游气汤,治上气一来一去无常,缓急不足,不得饮食,不得眠方∶生姜(八两)浓朴(四两)人参(二两)茯苓〔(一名松髓)四两〕桂心(五两)半夏〔(两)凡九物,切,以水一斗,煮取四升,服七合,日三。

《录验方》大枣汤,治上气胸塞、咽中如水鸡声方∶款冬花(三十枚)细辛(四分)桂心(四分)麻黄(四两)大枣(二十枚)甘草(四两)杏仁十味,以水八升,煮取二升,顿服,卧令汗。食糜粥数日,余皆禁,便愈。

《新录方》治上气、喉中水鸡鸣方∶桑根白皮(一升)生姜(合皮切一升)以水四升,煮取一升六合,二服。

又方∶冷水渍足,温易之,瘥。

又云∶上气,身面浮肿,小便涩,喘息不得卧方∶葶苈子(十分,熬)杏仁(四分,熬)大枣肉(五分)三物,合捣三四千杵,可丸饮服如梧子七丸,日二,加至十丸,以小便为度,此方大安稳,又方∶以桑根汁一斗,煮赤小豆三升,豆熟,啖豆饮汁。

又方∶大豆三升,以水一斗,煮取五升,去滓,纳桑根白皮,切一升,煮取一升六合,二服又方∶以水一斗,研麻子三升,取汁,煮赤小豆三升,豆熟,啖豆饮汁。

又云∶乏气喘息方∶桃仁去皮一升,捣为泥,分以酒若汤服之。
治短气方第三

《病源论》云∶短气者,平人无寒热,短气不足以息者,体实也。实则气盛,盛则气逆不通《僧深方》治短气欲绝、不足以息、烦扰,益气止烦竹根汤方∶竹根(一斤)麦门冬(一升)甘草(二两)大枣(十枚)粳米(一升)小麦(一升)凡六物,水一斗,煮麦米熟去之,纳药,煮取二升七合,服八合,日三,不能饮,以绵滴口《医门方》治胸中痞塞、短气者,或腹急痛方∶茯苓(四两)甘草(二两)半夏(三两)生姜(三两)杏仁(百颗)水七升,煮取二升半,去滓,分温三服,服相去八九里,泄气瘥。若气不下,加大黄、槟榔《千金方》治短气不得语方∶栀子(二七枚)豉(七合)水二升,煮豉,取一升半,去豉,纳栀子,煮取八合,服半升。

又云∶卒短气者方∶捣韭取汁,服一升,立愈。

又云∶冷气气短方∶椒五两,绢袋盛,酒一升,渍三七日,服之任性。
治少气方第四

《病源论》云∶少气者,此由脏气不足故也。肺主气而通呼吸,脏气不足,呼吸微弱而少气《广济方》疗腹冷气不能食及少气调中丸方∶人参(五两)茯苓(五两)甘草(五两)白术(五两)干姜(四两)捣筛,蜜和为丸,空腹温酒服如梧子三十丸,日二夜一,有益尽更合,不饮酒,煮大枣饮下,《千金方》治乏气方∶枸杞叶(二两)生姜(二两)以水三升,煮取一升,顿服。

《葛氏方》治卒乏气、气不复报、肩息方∶干姜三升,咀,以酒一升渍之,服一升,日三又方∶度手拇指折度心下,灸三壮即瘥。

又方∶麻黄三两,先以水五升,煮一沸,去沫,乃纳甘草二两,杏仁六十枚,煮取二升半,
治气噎方第五

《病源论》云∶气噎由阴阳不和,脏气不理,寒气填于胸膈,故气有咽塞不通,而谓之气噎又云∶噎者,一曰气噎,二曰忧噎,三曰食噎,四曰劳噎,五曰思噎。虽有五名,皆由阴阳《新录方》治气噎胸塞不达方∶水服盐末一大匙。

又方∶含咽美酒,取瘥止,酢亦得。

又方∶苏蜜合煎,令相得,细细含咽之。

又方∶捣韭取汁,服五合。

又方∶灸膻中穴。

又方∶灸第五椎。又灸内踝上三寸。

《集验方》通气噎汤方∶半夏(八两,洗)桂心(三两)生姜(八两)凡三物,以水八升,煮取三升,服半升,日二。

《千金方》治噎气不通、永不得食方∶杏仁(三两)桂心(三两)二味,丸如枣核,稍咽之,临食先含,弥佳。(《极要方》同之。)
治奔豚方第六

《病源论》云∶夫奔豚气者,肾之积气也,起于惊恐忧思所生也。若惊恐则伤神,心藏神也其妄言妄见,此惊恐奔豚之状也。若气满支心,心下烦乱,不欲闻人声,休作有时,乍瘥乍剧,吸吸短气,手足厥逆,内烦结痛,温温欲呕,此忧思奔豚之状也。

《医门方》云∶论曰,奔豚病者,从少腹起上冲喉咽,发作时欲死,皆从惊得之。

疗奔豚气方∶生葛根二十分,甘李根白皮切小一升,水九升,煮三升,分温三服,服相去八疗奔豚气在心胸中不下支满者方∶生姜(五两)半夏(四两,洗)桂心人参吴茱萸甘草(炙)茯苓(各二两)水七升,煮取二升半,分温三服,服相去八九里。

又云∶灸奔豚法∶恒灸气海、丹田、关元,皆当其穴灸之,穴在脐下一寸、二寸、三寸是也《短剧方》云∶师曰,病有奔豚,有吐脓,有惊怖,有火灸耶,此四部病皆从惊得之,所言如奔豚之状者,是病患气如豚奔走,气息喘迫,上逆之状也。汤方用KT猪者,谓雄豚、KT斗子是,先逐之,使奔之,然后杀取血及脏合药也。

葛根奔豚汤方∶葛根(八两,干者)生李根(一升,去皮)人参(三两)术肉(二两)半夏(一升,洗,炙)凡十物,以豚汁二斗,煮得五升,温服八合,日三。

牡蛎奔豚汤方∶牡蛎(三两)桂肉(八两)李根(一斤)甘草(三两)凡四物,煮豚令熟,取汁一斗七升,煮李根得七升汁,纳药,取三升,分服五合,日三,夜再。

《广济方》疗贲气在胸、心迫满支寄方∶生姜(一斤)半夏(四两,洗)桂心(三两)人参(二两)吴茱萸(一两)甘草(二两,炙)上,以水一斗,煮取三升,绞去滓,分温三服,忌生菜、面、粘食。

《集验方》奔豚茯苓汤,治虚气五脏不足、寒气厥逆、腹满、气奔冲胸膈、发作气欲绝不识生葛(八两)甘李根白皮(切,一升)生姜(五两)茯苓(四两)半夏(一升,洗)人参(肥豚一头三十斤者,逐走令口中沫出,刺取血,冶豚如食法,以水足淹豚,豚熟出之,澄取清《千金方》治气上下痞塞不能休息破气丸方∶桔梗(三分)胡椒(七分)荜茇(十卜)橘皮(三卜)干姜(三卜)椒(六分)乌头((五分)甘草(八分)细辛(苓(四分)前胡(四分)术(六分)茱萸(六分)二十四味,丸如梧子,酒服十丸,日二。有热者空腹服之。

又云∶治气上不得卧神秘方∶橘皮生姜紫苏人参五味子五味,各等分五两,水七升半,煮取三升,分三服。

又云∶治气满腹胀下气汤方∶半夏(一升)生姜(一升)人参(一两半)橘皮(三两)四味,切之,以水七升,煮取三升,分三服,一日令尽。
治痰饮方第七

《病源论》云∶痰饮者,由气脉闭塞,津液不通,水饮气停在胸腑,结而成痰。又其人素盛肋,气不《南海传》云∶若觉痰饮填胸,口中唾数,鼻流清水,KT(徒感反,糜和也)(作惨,息减反,杂也)咽开,户满抢喉,语声不转,饮食亡味,动历一旬。如此之流,绝食便瘥,不劳《千金方》云∶夫饮有四,其人素盛今瘦,水走肠间,沥沥有声,谓之痰饮;下后水流在胁其人痰饮者,当以温药和之;悬饮者,干枣汤主之∶甘草(四两)大枣(二十枚)干姜(二两)三味,以水一斗,煮取二升,分三服;溢饮者,青龙汤主之∶麻黄(六两)桂心(二两)甘草(二两,炙)石膏〔(一名细石)二两〕杏仁(三十枚)大枣凡七物,以水九升,煮麻黄,减二升,乃纳余药,得二升,去滓,服一升,温覆令汗。

汗出支饮者,木防己汤主之∶木防己(三两)石膏(鸡子大十二枚)桂心(二两)人参(四两)四味,水六升,煮取二升,分再服。

又云∶茯苓汤,主胸膈痰满方∶茯苓(四两)半夏(一两)生姜(一斤)桂心(八两)四味,水六升,煮取二升半,分四服。

又云∶治卒头痛如破,非冷又非中恶,其病是胸膈中痰厥气上冲所致,名为厥头痛,吐之即须吐又云∶灸留饮冷法∶灸通谷穴五十壮,在幽门下一寸,幽门在巨阙旁半寸。

《录验方》治胸膈痰饮,食啖经日并吐出方∶单服生姜汁一升,欲吐,吐之不吐,自向下出《葛氏方》治胸中多痰,头痛不欲食及饮酒人瘀僻菹痰方∶恒山(二两)甘草(一两)松萝(十两)瓜蒂(三七枚)以酒水各一升半,煮取升半,初服七合,取吐,吐不尽,余更分二服,后可服半夏汤。

又方∶先作一升汤,投一升水,名为生熟汤,乃餐三合盐,以此易送,须臾欲吐,便摘出,又云∶若胸中常有痰冷水饮、虚羸不足取吐者方(《范汪方》号半夏茯苓汤∶)半夏(一升,洗)生姜(半斤)茯苓(三两)水七升,煮取一升半,分再服。

《短剧方》茱萸汤治胸中积冷,心下痰水,烦满汪汪,不下饮食,心胸应背欲痛方∶生姜(三两)半夏(三两)桂心(三两)吴茱萸(三两)人参(一两)大枣(三十枚)甘草(凡七物,以水九升,煮取三升,纳白蜜五合,分三服。(今按∶《集验方》∶生姜五两。)《胡洽方》治痰冷癖气方∶生姜(八两)附子(四两)二物,以水三升,煮取一升半,分再服。

《僧深方》治五饮酒癖方∶术(一斤)桂(半斤)干姜(半斤)三物,冶下筛,和蜜丸如梧子,服十丸,不知稍增,初服当取下,先食服,日再。

《集验方》治胸中痰饮、腹中水鸣、食不消、呕吐水汤方∶大槟榔(三十口,令子碎)半夏(八两,洗)生姜(四两)杏仁(四两)白术(四两)茯苓(切,水一斗,煮取三升,分三服。

《效验方》甘草丸,治留饮方∶甘草(二分,炙)瓜蒂(一分)凡二物,冶下筛,蜜丸如梧子,欲下病,服三丸,日一。三丸不下,增之,以吐为度。

又云∶断膈散,治痰百病常用验方∶七月七日瓜蒂(二枚)赤小豆(二两)人参(二两)凡三物,冶合下筛,以温汤和服方寸匕,当吐病愈。(今按∶《新录方》∶瓜丁一两,赤小又云∶断膈丸,治胸膈间有痰水方∶蜀附子(一分)黎芦(一卜,熬)甘草(一卜,炙)赤小豆(一卜)瓜丁(一卜)凡五物,冶合下筛,蜜丸如小豆,一服五丸,当吐青黄汁,不知稍增。

《范汪方》病支饮不得息,葶苈大枣泻肺汤主之,方∶葶苈(熬令紫色,冶合自丸,丸如弹丸)大枣(二十枚)以水二升,煮枣,令得一升半,去枣,纳药一丸,复煎得一升,尽服之。

《极要方》疗痰气方∶橘皮(二两)上,以水三(二,或本)升,煮取一升二合,为一服,间日服之。

《新录方》治痰饮方∶苦瓠穣、赤小豆等分捣筛,蜜丸,饮服,如小豆三丸,大佳。

又方∶瓜丁、赤小豆各一两,捣筛,蜜丸,饮服如小豆七丸,吐痰瘥。
治癖食方第八

《病源论》云∶夫五脏调和则营卫气理,营卫气理则津液通流,虽复多饮水浆,不能为病。

寒气,《通玄》云∶癖之疾,亦不专于一。病者肝之所生,癖病者脾之所成。生于左,癖成脾矣。

《龙门方》治癖病、腹坚如石方∶取苦瓠开口盛大,严醋满中,密塞口,釜中煮令极热,出瓠以熨坚处,冷即更煮,煮时即作《极要方》治癖方∶鳖甲〔(一名青衣)六分,炙〕白术(六分)桔梗〔(一名荠)六分〕枳壳(六卜,炙)上,以水六升,煮取一升六合,分温三服,忌猪鱼。

《千金方》治心上痰荫僻气吞酸半夏汤方∶半夏(三两)生姜(六两)附子(一枚)吴茱萸(三两)四味,水五升,煮取二升半,分三服。

《葛氏方》治腹中冷癖、水谷结,心下停痰,两胁痞满,按之鸣转,逆害饮食方∶大黄(三两)甘草(二两)蜜(一升二合)枣(二十七枚)以水三升,煮枣,取一升,纳诸药,煮取一升七合,分再服。

又方∶茯苓一两,茱萸三两,捣蜜丸如梧子,服五丸,日三。

《医门方》疗气冷痛,吐酸水,或因出热吃水得此病方∶术干姜(各二两)橘皮细辛吴茱萸(各一两)茯苓人参(各二两半)上,捣筛,蜜和为丸,空腹暖水下十五丸,加至二十丸,日二。

《广利方》理癖气腹痛两肋胁胀满食少方∶柴胡(六分)桔梗(八分)通草(八分)茯苓(六分)赤夕药(四卜)郁李仁(四卜)鳖甲(四分,炙,碎)切,以水二大升,煎取九合,食后分温三服,如人行七八里进一服,忌生冷人苋。

《广济方》疗癖结心下硬痛巴豆丸方∶巴豆(三枚)杏仁(七枚)大黄(如鸡子大)捣筛,蜜丸,空腹以饮服如梧子七丸,日一服。
治胃反吐食方第九

《病源论》云∶胃反者,营卫俱虚,其血气不足,停水积饮在胃脘,则脏冷而脾不磨,脾不甚《医门方》云∶食已吐其食者,胃中虚冷所致。

《效验方》干姜丸,治胃反大吐逆,胸痛,羸瘦,不得食饮,温中下气、使人进食方∶吴茱萸(二两)小麦(二两,熬)杏仁(二两,去皮,熬)干姜(二两)好豉(二两,熬)凡六物,捣下筛,和蜜丸如梧子,服七丸,日三。

《葛氏方》治胃反不受食,食毕辄吐出方∶大黄(四两)甘草(二两)水三升,煮取一升半,分再服之。

《僧深方》治胃反吐逆不安谷,枳子汤方∶陈枳子(一枚)美豉(一升)茱萸(五合,去目,末)三物,枳、茱萸合冶为散,以水二升半,煮豉三四沸,漉去滓,汁着铜器中,乃纳散如鸡子《经心方》茯苓汤治胃反而渴方∶茯苓(四两)泽泻(四两)桂心(二两)半夏(四两)甘草(二两)五味,以水一升,煮取二升半,服八合,日三。

又云∶治胃反食辄吐方∶KT(徒党反,舂也)粟米,令极白,捣筛,下,作丸楮子大,熟煮,《范汪方》治胃反不受食、食已呕吐,四物当归汤方∶白蜜(一升)当归(二两)人参(二两)半夏(一升)凡四物,咀,以水二斗,合蜜,扬百四十过,纳药铜器中,煎得六升,分再服,加至一时又云∶橘皮汤,治呕吐反逆、食饮不下方∶人参(一两)橘皮(二两)白术(一两)生姜(三两)甘草(二两,炙)凡五物,切,以水一斗,煎取三升,先食,服一升,日三。

《千金方》治大虚胃反,食下喉便吐方∶人参(一两)泽泻(二两)甘草(二两)茯苓(四两)橘皮(三两)桂心(三两)干姜(三九味,水八升,煮取三升,服七合,日三夜一,已利者去大黄。

又方∶芦根、茅根,以水四升,煮取二升,顿服,下得食。(今按∶《新录方》∶切,各一又云∶灸胃反食吐方∶灸两乳下各一寸,以瘥为度。

又云∶灸脐上一寸,二十壮。

又方∶灸胃脘穴千壮,在鸠尾,脐中央。

《极要方》疗吐不得食,并胃反呕逆,食即吐方∶甘草(一两)橘皮(一两)生姜(八两)葱白(干,四枚)上,以水六升,煮取二升半,分三服,不止更作。(今按《广济方》∶葱白十茎,橘皮八分《医门方》疗胃反不受食,食讫呕吐方∶半夏人参生姜(各三两)橘皮(二两)大枣(十二枚)白蜜(五合)以东流水七升,煮取二升半,去滓,纳蜜,更烊三百下,煎三五沸,分温三服,服相去八九《救急单验方》疗反胃方∶捣生葛根,绞汁二升,服验。(今按《新录方》∶取六七合,日又方∶灸两乳下三寸,扁鹊云随年壮,华佗云三十壮,神验。
治宿食不消方第十

《病源论》云∶宿食不消者,由五脏气虚弱,寒气在于脾胃之间,故使谷不化也。旧谷未消复煎《集验方》治凡所食不消方∶取其余类烧作末,酒服方寸匕,便吐去宿食,即瘥。陆光录说子烧服《南海传》云∶若疑腹有宿食,又刺脐胸,宜须恣饮熟汤,指剔喉中,变吐令尽,更饮更决食。

《范汪方》治腹痛消谷止利胀大豆方∶取大豆,择貌好者服一合所,日四五服,一日中四五合,饭后辄服,虽非饭后可投,间服趋《范汪方》治食生冷之物,或寒时衣薄当风、食不消,或夜食以卧、不消化、心腹烦痛胀急《新录方》治宿食不消方∶薤白(切,一升)豉(一升)水四升,煮取二升,分二服。

又方∶生姜五大两,捣取汁,温服之。

又方∶捣蒜如泥,酒服如枣,日三。

又方∶曲末、干姜末一升,酒服一方寸匕,日二。

又方∶灸太仓穴二三百壮。

又方∶灸脐左右相去三寸,名魂舍,并依年壮,唯多益佳。

又方∶灸第五椎并左右相去一寸五分。

《录验方》治宿食不消,大便难练中丸方∶大黄(六分)葶苈子(四两,熬)杏仁(四两,熬)芒硝(四两,熬)凡四物,下筛,蜜和,食已服如梧子七丸,日三,不知稍增。

《千金方》消食丸,主数年不能食方∶小麦(一升)姜(四两)乌梅(四两)七月七日(或本无七月七日字)曲〔(音菊)一升〕四味,蜜和服十丸,日再,四十至丸,寒在胸中,及(反)胃翻心者皆瘥。

《极要方》治宿舍不消、心腹妨满胀痛须利方∶诃黎勒皮(八分)桔梗(六分)槟榔仁(八分)夕药(六分)大黄(十分)上,为散,空腹煮生姜,饮服三钱匕,日二服。

《葛氏方》治脾胃气弱,谷不得下,遂成不复受食方∶大麻子仁(一升)大豆黄卷(二升)并熬令黄香,捣筛,饮服一二方寸匕,日四五。(今按《僧深方》∶大麻子仁三升,大豆二
治寒冷不食方第十一

《千金方》消食断下丸,寒冷者常将之,方∶曲末(一升)大麦末(一升)吴茱萸(四两)三味,蜜和服十五丸如梧子,日三。

又云∶消食丸,主数年不能食方∶小麦(一升)干姜(四两)乌梅(四两)曲(一升)四味,蜜和服十丸,日再,至四十丸寒在胸中及反胃翻心者皆瘥。

《范汪方》治久寒不欲饮食数十岁方∶茱萸(八合)生姜(一斤,切)硝石(一升)凡三物,清酒一升,水五合,煮令得四升,绞去滓,温饮二升,病即下去,勿复服也。

《葛氏方》治胃中虚冷不能饮食,食辄不消,羸瘦乏、四肢弱、百疾因此牙生方∶薤白(一斤)枳实(三两)橘皮(一两)大枣(二十枚)粳米(二合)豉(七合)以水七升,先煮薤,得五升,纳诸药,煮取二升半,分三服,日日作之。

《广济方》疗冷气不能食及少气调中丸方∶人参(五两)茯苓(五两)甘草(五两)白术(五两)干姜(四两)捣,以蜜和丸,空腹温酒服如梧子三十丸,日二夜一,不饮酒,煮大枣饮下。

《录验方》治恶食人参汤方∶人参(四两)生姜(二斤)浓朴(二两)枳实(二两)甘草(二两)凡五物,切,以水六升,煮取二升,分三服。

《集验方》治久寒胸胁逆满不能食吴茱萸汤方∶吴茱萸(一升)人参(一两)生姜(八两)小麦(一升)甘草(一两)桂心(一两)半夏凡八物,咀,以清酒五升,水三升,煮取三升,绞去滓,适寒温,饮一升,日三。
治上热下冷不食方第十二

《耆婆方》治人上热下冷痰饮风气虚劳方∶独活茯苓白术泽泻浓朴黄升麻本紫菀甘草人参黄芩各(二两)上十四味,切,以水一斗二升,煮取三升,去滓,分三服。

又云∶因饮酒上热下寒不能食方∶人参(二分)甘草(二卜)升麻(二两)干蓝(二两)粟米(一合)凡五味,切,以水六升煮,取米去滓,分三服。

又方∶平且空腹服真酪一合即愈。

又方∶常食粟餐及粟粥之。

又云∶治虚上热下冷气上,头痛胸烦人参汤方∶人参(二两)茯苓(三两)麦门冬(一两)粟(二两)凡四物,水七升,煮取四升,分三服,日三夜二。

又云∶治内虚上热下冷,气不下,头痛胸烦头豉汤方∶豉一升,水二升(一方三升),令小沸,纳豉令三沸,顿服有验。

《僧深方》茱萸丸,治膈上冷膈下热,宿食癖饮积聚,食不消,寒在胸中,或反胃害食消瘦茱萸(二两)椒(一两半)黄芩(一两)前胡(一两)细辛(六分)皂荚(二枚)人参(三凡十一物,下筛,丸以蜜,服如梧子三丸,日三,不知稍增之。

《广济方》疗膈上热膈下冷,日西身体热疼,吃食不下,夜卧不安方∶苦参〔(一名火槐)六分〕龙胆(五分)夕药(四分)黄连(六分)栝蒌(四分)青葙子(捣筛,蜜丸,每食后以饮服丸如梧子十四丸,日二。
治谷劳欲卧方第十三

《病源论》云∶谷劳者,脾胃虚弱不能传消谷食,使腑脏气痞也,其状令人食已则卧,肢体《葛氏方》治饱食竟便卧,得谷劳病,令人四肢烦重欲卧、食毕辄甚方∶大麦(一斤)椒(一两)干姜(三两)捣末,服方寸匕,日三四服。(今按∶《范汪方》∶大麦一升,椒二升,干姜三两也。)又云∶治食过饱,烦闷但欲卧而腹胀方∶熬麦面令微香,捣服方寸匕,得大麦面益佳,无面《新录方》治谷劳食竟即因而睡方∶以酒二升,煎杏仁五十枚,取一升服之,覆取汗。

又方∶食伤饱为病,胃胀心满者∶灸胃脘七壮。

又方∶十沸汤,生水共三升饮之,当吐食出。
治恶心方第十四

《病源论》云∶恶心者,心下有停水积饮所为,则心里澹澹然欲吐,为恶心。

《千金方》治恶心方∶苦瓠瓤并一升,切,酒水三升,煮取一升,顿服,须臾吐并下如虾蟆《葛氏方》治人忽恶心不已方∶薤白(半斤)茱萸(一两)豉(半升)米(一合)枣(四枚)枳实(二枚)盐如弹丸,水三升,煮取一升半,分三服。

又方∶但多嚼豆KT子及啖槟榔亦佳。

《孟诜食经》恶心方∶取怀香华叶煮服之。

《新录方》治恶心方∶生姜合皮捣服五大两汁。

又方∶槟榔仁,末,方寸匕,生姜汁服之,日二。又加橘皮更佳。

《极要方》疗冷痰气在胸腹,胀不能食,吐水沫,耿耿恶心方∶吴茱萸(一升)橘皮(二两)杏仁(三两)生姜汁(三合)上,以水五升,煮取一升六合,去滓,纳槟榔仁散方寸匕,分再服,得一两行利大快,三五
治噫酢方第十五

《病源论》云∶噫酢者,由上焦有停痰,脾胃有宿冷,故不能消谷,谷不消则胀满而气逆,《葛氏方》人食毕噫酢及酢心方∶人参(二两)茱萸(半升)生姜(三两)大枣(十二枚)水六升,煮取二升,分再服。(《集验方》同之。)《千金方》治食后吐酢水方∶干姜(二两)食茱萸(半升)二味,酒服方寸匕,日二,立验。

《医门方》疗食噫或酢咽方∶人参(二两)吴茱萸(二两)生姜(三两)大枣(十二枚)橘皮(一两半,切)以水六升,煮取二升,去滓,分温二服。

又云∶疗食后吐酢水,食羹饭粥并作方∶浓朴(炙)吴茱萸桂心橘皮(各二两)白术(三两)上捣筛为散,空腹酒服方寸匕,甚效。

《广济方》疗吐酸水,每食即变作酸水吐出方∶槟榔仁(十二六分)人参(六分)茯苓(八分)橘皮(六分)荜茇(六分)捣筛为散,平晨空腹,取生姜五大两,合皮捣绞取汁,温,纳散方寸匕,搅调顿服之,日一《效验方》治食后吐酢水,洗洗如酢浆,食羹即剧,为胃冷干姜散方∶食茱萸(一两)干姜(一两)术(一两)甘草(一两)凡四物,冶下筛,用酒若汤服方寸匕,日三。
治呕吐方第十六

《病源论》云∶呕吐者,皆由脾胃虚弱,受于风邪所为也。若风邪在胃则呕,膈间有停饮,生《极要方》疗呕吐∶此病有两种,一者积热在胃,呕逆不下食,二者积冷在胃,呕逆不下食生姜根(切,一升)生麦门冬(一升)青竹茹(一升)生姜(切,五合)茯苓(五两)上,切,以水八升,煮取二升半,去滓,加竹沥六合搅调,分二服。

《博济安众方》治呕逆不食方∶浓朴(三两,去皮,姜涂,炙)人参(一两)橘皮(一两)上,以水三升,煎取一升,分作三服,饭后服。

《僧深方》生姜汤治食已吐逆方∶生姜(五两)茯苓(四两)半夏(一升)橘皮(一两)甘草(二两)五种,水九升,煮取三升七合,分三服。

《广济方》疗脾胃中冷气、每食即呕吐方∶人参(二两)甘草(一两半,炙)橘皮(一两半,炙)葱白(三两)生姜(三两,切)以水七升,煮取二升三合,绞去滓,分温三服,忌生冷油腻猪鱼海藻。

《范汪方》半夏汤,治胸中乏气而呕欲死方∶人参(二两)茯苓(二两)生姜(三两)白蜜(五合)半夏(三升,洗)凡五物,以蜜纳六升水中挠之百过,以余药合投中,煮得三升,分四服,禁冷食,治干呕亦《医门方》治呕逆变吐食饮不下方∶橘皮(二两)术(二两)生姜(三两)人参(二两)甘草(二两)上,切,以水七升,煮取二升半,温三服,服相去八九里。

止呕橘皮汤方∶橘皮(二两)干姜(二两)人参(一两半)以水六升,煮得二升,服七合,日三。

《新录方》治呕吐不下食方∶茅根(切,二升)生姜合皮(切,一升)以水四升半,煮取二升,二服。

《录验方》治热呕方∶芦根茅根(切,各一升)以水六升,煮取二升,分三服。

《葛氏方》治卒呕哕又厥逆方∶生姜(半斤,切)橘皮(四两)水七升,煮取三升,适寒温,服一升,日三。

《千金方》云∶诸呕哕心下坚痞、膈间有水痰、眩悸者,小半夏汤主之∶半夏(一升)生姜(八两)茯苓(三两)三味,水七升,煮取二升半,二服。

又云∶凡呕者多食生姜,此是呕家圣药也。

又云∶饮食呕吐法∶生熟汤二升,顿服即吐。

又云∶灸呕吐方∶灸心俞百壮。

又方∶灸膈俞百壮。

又方∶灸胸堂百壮。

又方∶灸原阙五十壮。

又方∶灸胃脘百壮,三报。

又方∶灸脾募百壮,三报。
治干呕方第十七

《病源论》云∶干呕者,胃气逆故也,但呕而欲吐,吐而无所出,故谓之干呕。

《僧深方》治胃逆干呕欲吐无所去人参汤方∶人参(二两)干姜(四两)泽泻(二两)桂心(二两)甘草(二两)茯苓(四两)大黄(一八物,以水八升,煮取三升,服八合,日三。

又云∶茱萸汤,治干呕吐涎沫、烦心头痛方∶茱萸(半升)大枣(十枚)人参(三两)生姜(六两)凡四物,以水六升,煮取二升五合,日三服。

又方∶茱萸一升,大枣十二枚。以水七升,煮取二升半,分三服。

又方∶半夏、干姜分等为散,服方寸匕。

又方∶生姜汁五合,蜜四合,二物,先煎蜜减一合,竟投姜汁,复煎数沸,稍稍啖之,勿久《范汪方》治卒干呕烦闷方∶用甘蔗捣之,取汁,温服一升,日三。

《集验方》治干呕或哕、手足逆冷方∶橘皮(四两)生姜(六两,切)以水六升,煮取三升,服一升。(《千金方》同之。)《葛氏方》治干呕不息方∶捣葛根,绞汁,服一升许。

又方∶灸两手腕后两筋中一夫,名间使,各七壮。

又《千金方》治干呕哕厥逆方∶饮新汲水三升。

又方∶煮豉三升,饮汁。

又方∶浓煮三斤芦根,饮汁。

又方∶空腹饮生姜汁。

又云∶灸干呕方∶灸心主,尺泽亦佳。(今按《明堂》云∶在肘中缝上。)又方∶灸乳下一寸二十壮。
治哕方第十八

《病源论》云∶哕者,脾胃俱虚,受于风邪,故令新谷入胃,不能传化,故谷之气与新旧相《葛氏方》治卒哕不止方∶饮新汲井水∶(今按《新录方》云∶服井华水二升。数升。)又方∶但闭气拆引。

又方∶痛抓眉中央,闭气。

又方∶好豉二升,煮取汁饮之。

又方∶枇杷叶一斤,水一斗,煮取三升,再服。

又方∶煮芦根亦佳。(今按《千金方》∶浓煮三斤饮汁。)又方∶以物刺鼻中,若以少许皂荚纳鼻中,令嚏即止。

《录验方》治呕哕橘皮汤方∶生姜(四两)橘皮(一两)甘草(一两)凡三物,以水六升,煮取二升,一服七合,日三。

《新录方》治哕方∶单服十沸汤,任多少。

又方∶生姜五大两,合皮捣取汁服。

又方∶荻根切二升,水四升,煮取一升五合,稍咽之。

又方∶橘皮五两,以水三升,煮取一升,二三服。

又方∶灸腋下一寸。又灸胃脘穴。

《苏敬本草注》治哕方∶服千岁汁。崔禹锡《食经》云∶薯蓣为粉,和汁煮作粥食之。

《短剧方》治哕方∶灸腋下聚毛中五十壮。

又方∶灸石关穴五十壮。

《千金方》治方∶灸承浆炷如麦七壮。

又方∶灸脐下五寸七壮。

又方∶煮豉三升,饮汁。

又方∶空腹饮姜汁一升。

医心方卷第九医心方卷第九背记《病源论》云∶夫饮酒人大渴,渴而饮水,水与酒停聚胸膈之上,蕴积不散而成癖也,则令楮∶陆法言云∶刃吕反。释反云∶树也。恶木也。似伫子,可以为药也。又名通天木,蔡
卷第十
治积聚方第一

《病源论》云∶积聚者,由阴阳不和,腑脏虚弱,受于风邪,搏于腑脏之气所为也。腑者阳穷也在左名曰在上下无时,令人喘逆,骨痿少气。

《医门方》云∶辨曰∶肥气者,肥盛也,言肥气如覆杯,突出如肥盛之状;伏梁者,言其大奔积《华佗方》云∶二车丸,常在尊者后一车,故名二车丸,主心腹众病,膈上积聚,寒热,食饮不消,或从忧恚喜怒,或从劳倦气结,或有故疾气浮,有上饮食衰少,不生肌肉,若辟在胁,吞一丸即消;若惊恐不安,吞一丸,日三;独卧不恐,病剧,昼日六七,夜三吞。微者,昼日四五、夜再吞。寒辟随利去,令人善矢气。又治∶女子绝产,少腹苦痛,得阳亦痛,痛引胸中,积寒所致,风入子道,或月经未绝而合阴阳,或急欲尿而合阴阳,或衣未掺而合阴阳,或急便着之,湿从下上;久作长病,吞药如上,百日有子。二车丸方∶蜀椒(成择一斤)干姜(大小相称二十枚)粳米(一升)朗陵乌头(大小相称二十枚)锻灶凡五物,以水一斗半,渍灰,练囊中盛半绞结,纳灰中一宿,暴干之,皆末诸药下筛,和以《范汪方》五通丸,主积聚、留饮、宿食、寒热、烦结,长肥肤,补不足方∶椒目(一两,汗)附子(一两,炮)浓朴(一两)杏仁子(三两,熬)半夏(一两,洗)葶凡八物,别捣葶苈、杏仁,使熟,合和诸药末,使调和,以蜜捣五千杵,吞如梧子二丸。

又云∶三台丸,主五脏寒热,积聚胪胀,腹大空鸣而噫食,不生肌肤,剧者咳逆,若伤寒病方∶大黄〔十二两(一方二两),捶碎,熬令变色〕葶苈(一升,熬令变色)附子(一两,令)杏仁(一升,熬令变色)硝石(一升)芘胡(二两,洗)浓朴(一两,炙)茯苓(半两凡十物,皆捣筛,和以蜜捣三万杵,丸如梧子,从五丸起,不知稍增,取大便调利为度。

又云∶治久寒积聚方∶虎杖根一升许。捣之,以酒渍,日三,饮一升。

《短剧方》云∶七气丸治七气。七气为病,有寒气、怒气、喜气、忧气、恚气、愁气、热气时,气则疾行足浮大黄(十分,炮)人参(三分)椒(二分,熬)半夏(三分,炮)乌头(五分,炮)桔梗(苓(三分)芎(分)仁(三分)凡十七物,冶合下筛,和以蜜,酒服如梧子三丸,日三,不知稍增,以知至十丸为度。

(今坚如《千金方》七气汤,治忧气、劳气、寒气、愁气、或饮食为气、高气,或虚劳内伤,五脏不调,阳气衰少逆上下方∶甘草(二两)栝蒌(二两)夕药(二两)椒(三两)半夏(二两)人参(一两)干地黄(二八味,切,以水一斗,煮取三升,分三服。

又云∶神明度命丸,治久病腹内积聚,大小便不通,气上抢胸,腹中胀满,逆苦饮食,服之大黄(一两)夕药(一两)二味,蜜丸如梧子,服四丸,日二,不知,可增至六七丸,以知为度。

又云∶胁下邪气积聚、往来寒热如温疟方∶蒸鼠壤土熨之,冷即易。

《葛氏方》云∶露宿丸,治大寒冷积聚方∶石干姜桂桔梗附子皂荚(各三两)捣筛,蜜丸,服如梧子十丸,日三,稍增至十五丸。

《僧深方》治心下支满痛,破积聚,咳逆不受食,寒热喜噎方∶蜀椒(五分)干姜(五分)桂心(五分)乌头(五分)上四物,冶合下筛,蜜和丸如小豆;先辅食以米汁,服一丸,日三夜一;不知,稍增一丸,《新录方》治积聚方∶马苋捣汁为煎,令可丸,酒服如枣,日三。

《德贞常方》积聚方∶灸第十三椎节下间,相去三寸。

又方∶灸上脘穴,在鸠尾下二寸。

又方∶灸胃脘穴,在上脘下一寸。

又方∶灸水分穴,在脐上一寸。

《新罗法师》方∶续随子(一名耐冬花),去上皮,以酒一合,和而服之二七粒,量人老少《崔禹锡食经》∶取蔓菁子一升捣研,以水三升,煮取一升,浓服之,为妙药也,亦治症瘕
治诸疝方第二

《病源论》云∶阴气积于内,复为寒气所加,故使荣(营)卫不调、血气虚弱,故风冷入其腹或冷气《八十一难》云∶五脏谓之疝,六腑谓之瘕。

又曰∶男病谓之疝,女病谓之瘕。

《葛氏方》治卒得诸疝,少腹及阴中相引,痛如绞,白汗出欲死方∶捣沙参(一名白参),下筛,以酒服方寸匕,立愈。

又方∶椒二合,干姜四两,水四升,煮取二升,去滓,纳饴一斤,又煎取半升,分再服。

又方∶可服诸利丸下之,作走马汤亦佳。

又方∶灸心鸠尾下一寸,名巨阙,及左右各一寸,并百壮。

《极要方》疗积年腹内宿结疝冷气及诸症瘕方∶香美烂豉心(一升,曝干微熬,令气香即止)小芥子(一升,唯去石,微熬令色黄)上,为蜜丸,空腹酒服二十五丸,加至三十五丸,日二服,初服半剂,已来腹中觉绞痛,勿怪《新录方》治诸疝方∶桃白皮(一升)以水三升,煮取一升,顿服之。

又方∶酒服蒲黄二方寸匕,日二。

又方∶捣桃仁八十枚,去皮研如泥,酒下。

又方∶捣大蒜为泥,酒服如枣二枚,日三。

《范汪方》治心疝灸法∶两足大指甲本甲肉之际、甲内各半炷,随年壮又方∶灸足心及足大指甲后横理节上及大指歧间白黑肉际,百壮则止。
治七疝方第三

《病源论》云∶七疝者,厥逆心痛,足寒清,饮食吐不下,名曰厥疝;腹中气乍满,心下尽曰气行难《录验方》七疝丸,治人腹中有大疾,厥逆心痛,足寒冷,食吐不下,名曰厥疝;腹中气满减而相引人参(五分)桔梗(五分)黄芩(五分)细辛(五分)干姜(五分)蜀椒(五分)当归(五凡十物,冶下筛,和以白蜜,丸如梧子,先食,服四丸,日三。不知稍增,禁生鱼猪肉。

(乌喙二合,服三丸,日三。《范汪方》有十二物∶蜀椒五分、干姜四分、浓朴四分、桔梗二分、乌喙一合、黄芩四分、细辛四分、夕药四分、桂心二分、柴胡一分、茯苓一分、牡丹一分,先辅食,以酒服七丸,日三。)
治寒疝方第四

《病源论》云∶阴气积于内,则卫气不行,卫气不行,则寒气盛,故令恶寒,不饮食,手足《葛氏方》治寒疝去来,每发绞痛方∶吴茱萸(三两)生姜(四两)豉(二合)酒四升,煮取二升,二服。

《短剧方》云∶解急蜀椒汤,主寒疝心痛如刺,绕脐绞痛,腹中尽痛,白汗自出欲绝方蜀椒〔三百枚(一方二百枚)〕附子(一枚)粳米(半升)干姜(半两)大枣(三十枚)半夏凡七物,以水七升,煮取三升,热服一升,不瘥,复服一升。

《经心方》蜀椒汤,治寒疝痛,腹胀奔胸方∶吴茱萸(一升)当归(一两)夕药(一两)黄芩(一两)蜀椒(二合)五味,以水八升,煮取二升半,分三服。

《范汪方》治寒疝腹中痛,小柴胡汤方∶柴胡(半斤)半夏(半升,洗)黄芩(三两炙)甘草(三两)人参(三两)生姜(三两)大凡七物,咀,以水一斗二升,煮得六升,去滓,服一升,日三。

《耆婆方》治寒疝积聚,用力不节,脉绝伤,羸瘦,不能食饮,此药令人强健,除冷气痞丸乌头(二十分,炮)甘草(八分,炙)真芎(八分)葶苈(八分,熬)夕药(八分)大黄上七味,捣筛,以蜜和为丸如梧子,服五丸,日二,忌猪鱼五辛生冷酢滑。

《新录方》治寒疝及冲心痛方∶盐(五合)灶突墨(三合)以水一大升,煮取一小升,顿服之,吐瘥。

又方∶水一升五合,渍豉一升五合,绞交取汁服之。

又方∶桃仁八十枚,去皮研如泥,酒下之。

又方∶桃白皮一升,以水三升,煮取一升,顿服之。

又方∶以水若酒服乱发灰方寸匕,日二。

又方∶水酒服伏龙肝方寸匕,日二。

又方∶水服甑带节灰方寸匕,日二。

又方∶酒五升,烧鹿角一枚投酒中,分二三服之。

又方∶灸乳下一寸,足大指丛毛。

又方∶灸脐上三寸,名太仓,脐下二寸,名丹田,各五七炷,并要穴。

又方∶灸上脘七壮。

又方∶灸穷脊上一寸,百壮。

又方∶灸脊中百壮。
治八痞方第五

《病源论》云∶荣(营)卫不和,阴阳隔绝,而风邪外入,与卫气相搏,血气壅塞不通而成痞腹满,时时壮热是也。其名有八,故云八痞,而方家不显的,其证状。

《八十一难》云∶脾之积名曰痞气,在胃脘,覆覆大如盘,久不愈,令人四肢不收,发黄胆也《录验方》治八痞麝香丸方∶光明砂麝香丁香曾青〔(一名空青)各一两〕大黄(七分)黄芩(三分)朴硝(二两)凡十物,捣筛,蜜丸如小豆,一丸平旦空腹服,若老人壮者,同患之人,服如梧子一丸。

《传信方》云∶疗秋夏之交,露坐夜久,腹内痞如群石在腹中痛者方∶大豆(半升)生姜(八分)上,以水二升,煎取一升已下,顿服,其坚痞立散。
治症瘕方第六

《病源论》云∶症瘕者,皆由寒温不调,饮食不化,与脏气相搏结所生也。其不动移者,其名为症;若病虽有结瘕而可推移者,名为瘕也。

《葛氏方》云∶症瘕病冲心不移动,饮食痛者死。

《新录方》云∶治一切病温白丸方∶南州刺史臣阴铿言∶臣蒙慈泽(或本“恩”),视事三年于今道士紫菀(二分)吴茱萸(二分)石上菖蒲(二分)浓朴(二分)桔梗(二分)皂荚〔(兼协反)连(二分)蜀椒(二分)上十五味,捣,下筛为散,用好蜜和,更捣三千杵,丸如梧子大,服二丸,不知稍增,可至上气十种治妇痛,或烦热五种惊痫只欲取水不调,或多或少,真似怀孕知子,或连年累月羸瘦困弊,遂致于死,或哭或歌,为鬼所乱,但能脓三升,其病即愈。臣见被堕伤,临死有积血,天阴即发,羸瘦着床,不能食饮,命在旦夕,服即愈,平安状如常。主簿陈胜累有心腹胀满,经十四年,瘦疲气闷,饮食不下,臣与此药服,服十日,下青虫六十枚,大小如树叶,头赤,虫身黑,下脓三升许,病即愈。臣公曹常患着床,以经数年,服此药三十日,下肉蜣螂百枚,有出青黄水一斗,病即愈。臣门师侄长多羸瘦着床,食便吐出,命在朝夕,从臣求药,服五丸,至十五日,下出肉虾蟆十枚,青水一斗,其病即愈。臣家内有人常患心病发无时节,发即欲死,服此药五六日,下肉蛇二枚,各长尺五寸,有头,眼未有瞳子,斑斑有文,其病即瘥。又臣治尼专,得大风,眉堕落,已经二年,遍身出疮,状如锥刀所刺,与药;服九丸,至一月,日出症虫五色,凡三升许,其病即愈,眉鬓遂生,至复如故。臣知方大验,死罪谨上也。服中禁忌冷水、生菜、生鱼、猪肉、滑、陈臭物、五辛。

《僧深方》云∶硝石大丸,治十二症瘕,及妇人带下,绝产无子,及症服寒食药而腹中有症河西大黄(八两)朴硝(六两)上党人参(二两)甘草(三两)凡四物,皆各异捣下筛,以三岁好苦酒置铜器中,以竹箸柱铜器中,一升作一刻,凡三刻,乃令服,强者粥食寒食药《千金方》治瘕症方∶灸内踝后宛宛中,随年壮。

又方∶灸气海穴百壮,在脐下一寸半。

又方∶槲白皮煎令可丸,服之取知,病动若下,减之。

又云∶治症坚,心下如杯,食则腹满,心绞腹痛方∶葶苈子(二两)大黄(二两)泽漆(四两)三味各异捣五百杵下筛,冶葶苈合膏,下二物散捣五百杵,和以蜜,服如梧子二丸,不知稍《医门方》疗症瘕,腹内胁下小腹胀满痛,冷即发,其气上冲心,不能饮食,或呕逆气急烦半夏(十分)生姜(十分)大黄(十二分)槟榔仁(十分)人参(八分)吴茱萸(六分)浓二枚)水九升,煮取三升,下大黄,更煮三沸,分温三服。服相去八九里,当利三二行。

《广利方》理症瘕腹胀满坚硬如石,肚皮上青脉浮起方∶紫葛粉(八分)赤夕药(六分)桔梗(六分)紫菀头(三十五枚)青木香(六分)水路诃黎勒捣筛,蜜丸如梧子,空腹服十五丸,忌陈臭粘腻、猪肉。

《葛氏方》治心下物大如杯,不得食者方∶葶苈(二两)大黄(二两)泽漆(四两)捣筛蜜丸,捣千杵,服如梧子二丸,日三。
治暴症方第七

《病源论》云∶暴症者,由脏气虚弱,食生冷之物,脏既本弱,不能消之,结聚成块,卒然《极要方》疗卒暴症,腹中有物坚如石,痛如刺,尽夜啼呼,不疗,百日内皆死,方∶大黄(半斤)朴硝(半斤)上,先捣大黄为散,然后和朴硝,以蜜合令相得,于铜器中置汤,上煎令可丸,丸如梧子大汪《葛氏方》治卒暴症,腹中有物坚如石,痛如刺,昼夜啼呼,不治之,百日死,方∶取牛膝根二斤,曝令小干,以酒一斗,渍之,密塞器口,举着热灰中,温之令味出,先食,又方∶用蒴根亦如此,尤良。

又方∶多取常(商欤)陆根捣蒸之以新布藉腹上,以药披着布上,以衣覆上,冷复易之,昼夜勿息。(以上《千金》《集验方》同之。)又方∶取虎杖根一升,干捣酒渍,饮之从少起,日三佳,此酒治症,力胜诸大药。

《新录方》治暴症,坚在心胁下,咳逆,不下食,或下不断,方∶吴茱萸三升,碎之,以酒和煮令熟,布帛物裹以熨症上,冷更炒更燔用之,症当移走,复逐又云∶暴症坚胀如石,痛欲死者方∶取鼠壤土,黍穣二物,等分相和,并炒,遍熨病上取瘥。

又方∶单用鼠壤土亦好。

又方∶伏龙肝如前方。

又方∶病在上,服诸吐药去之,病在下,宜利疗之。

《本草稽疑》云∶捣莎草汁及干末服之。
治蛇瘕方第八

《病源论》云∶人有食蛇不消因腹内生蛇瘕也。亦有蛇之精液误入饮食之内,亦令病之。

其《新录方》治人食蛇不消,亦蛇之精液入饮食中,令人病之,腹内有蛇状,名之蛇瘕,方∶浓作蒜齑,啜一升以上。陶云饼店蒜齑下蛇之药,非虚圣之。

又方∶鸠酸草捣取汁,服八合。

又方∶大豆叶捣取汁服一升。

又方∶常思草捣服如上,并频服之取瘥。

《千金方》治蛇症方;白马毛切长五分,以酒服方寸匕,大者自出,更服二方寸匕,中者亦出,更服三方寸匕,小又云∶治蛇瘕大黄汤方∶大黄(半两)乌贼鱼骨(二枚)皂荚(六铢)芒或硝如鸡子(一枚)黄芩(半两)甘草(大六味,以水六升,煮之三沸,去滓,纳芒硝,适寒温尽服之,十日复煮,作如上法,欲服之又云∶治蛟龙病方∶开皇六年,有人二月八月食芹得之,其病发似癫,面色青黄,服寒食强月,蛟龙子生在芹菜上,人食芹,不幸随食入腹,变成蛟龙。)
治鳖瘕方第九

《病源论》云∶鳖瘕者,谓腹内瘕结如鳖状是也。有食鳖触冷不消而生者,亦有食诸杂物,而推《广济方》疗鳖症方∶白马尿一升五合,温服之令尽,瘥。

《葛氏方》治鳖瘕伏在心下,手揣见头足,时时转动者∶白雌鸡一双,绝食一宿,明旦,膏煎饭饴之,取其矢无多少,于铜器中以尿和,火上熬可捣《千金方》治鳖症方∶蓝叶一斗捣,水三升,绞取汁,服一升,日二。

又方∶白马尿一斗,鸡子三枚,取白合煮,取二合,空腹服之。(《广济方》同之。)《新录方》治鳖症,团团似鳖,有脚能动,数冲心痛者方∶取蒴根白皮,捣三升,以水五六合和搅,绞取汁,取七八合,吐出。

又方∶捣蓝汁服七八合。

又方∶单服白马尿一升,日二。
治鱼瘕方第十

《病源论》云∶人有胃气虚弱者,食生鱼,因为冷气所搏,不能消之,结成鱼瘕,揣之有形《养生方》云∶鱼赤目作脍,食之生瘕也。

《新录方》治人食生鱼不消,又饮湖水,误小鱼入腹,不幸生长,名之鱼瘕方∶炭火烧木瓜为灰,汤或酒中服方寸匕,日二。

又方∶烧鱼鳞为灰,汤若水服方寸匕,日二。

又方∶烧年久鱼网为灰,水服方寸匕,日二。

又方∶白马尿服一升。

又方∶煮橘皮汤服之。

又方∶豉汁服橘皮末方寸匕,日三。

《葛氏方》治食鱼脍及生肉住胸中不消成方∶朴硝(如半鸡子者一枚)大黄(一两)凡二物,以酒二升,煮取一升半,尽服之。
治肉瘕方第十一

《病源论》云∶人有病而常思得肉,得肉讫,又思之,名为肉瘕。

《千金方》治肉瘕,思肉不已,食讫复思方∶空腹饮白马尿三升,吐肉出之。

《新录方》治肉瘕方∶饮服大豆黄末一匕,日二。

又方∶浓豉汁服一升,日二。

又方∶煮菘菜,浓汁服之。
治发瘕方第十二

《病源论》云∶人有因饮食内误有头发,随食而入成症,胸喉间如有虫下上去来者是也。

《广济方》疗发症、唯欲饮油方∶油一升,上香泽煎之,大钵劳贮安病患头边。令口鼻临油上,勿令得饮,及敷鼻面,并令香症出形为病。)又云∶疗胸喉间觉有症虫,上下偏闻葱豉食香,此是发虫故也,油煎葱豉置口边,行术二日《新录方》治发瘕方,心满,食竟便吐者是。

成煎猪脂二升,酒二升,煮三沸,一服一升,日二,取吐;发利瘕出乃止。

又方∶饮白马尿八合,日一,瘥止,发瘕令食竟便吐,余瘕则不然。(以上《千金方》同之
治米症方第十三

《病源论》云∶人有好哑米,转久弥嗜哑之,若不得米,则胸中清水出,得米服水便止,米不消化,遂生症结。

《千金方》治米症恒欲食米方∶鸡矢(一升)白米(五合)二味合炒,令米焦捣末,水二升,顿服之,须臾吐出病碎米,若大良
治水瘕方第十四

《病源论》云∶水瘕者,由经络痞涩,水气停聚在于心下,肾经又虚,不能宣利泄便,致令《范汪方》治水瘕病,心下如数斗油囊裹水作声,日饮二三斗,不用食,但欲饮,久病则瘕蓖麻熟成好者二十枚,去皮,杯中研令熟,不用捣,水解得三合,宿不食,清旦一顿服尽,病故复不可饮,
治食症方第十五

《病源论》云∶有人卒大能食,乖其常分,因饥,值生葱便大食之,仍吐一肉块,绕畔有口,其病即愈,故为食症,特由不幸致此,夭暴成症,非饮食生冷过度之病也。

《广济方》疗食症方∶有人一食饭七斗、并半猪饼,燔并不论,因苦饥于葱,中过饥急,即物
治酒瘕方第十六

《病源论》云∶人有性嗜酒,饮酒既多,而食谷常少,积久渐瘦,其病遂常思酒,不得即吐诸瘕《新录方》治诸瘕方∶灸膀胱俞三百壮以上。

又方∶酒若饮,服自发爪灰。

又方∶捣曲末,酒饮服之,日二。

又方∶葶苈子三升熬,以酒二升渍三日,温服半盏,日二。
水病证候第十七

《集验方》云∶黄帝问于岐伯曰,水与肤胀,鼓胀,肠覃,石瘕,石水何以别之?岐伯曰,其水腹也而岁不导品方》云∶先从胸肿名曰石水,其根在脾。)《医门方》辨曰∶近验水病者,小便皆涩,或黄或赤不出,溢入经络,致令肿满,名为水病又云∶若诸皮肤水胀者,服诸发汗汤,得汗即愈,须慎风冷及咸食,宜食鲤鳢鱼,小豆等《葛氏方》云∶水病唇黑脐突出,死;水病脉出者,死。

《千金方》云∶水病初起,两目上先肿如老蚕色,颊颈脉动,股里冷,胫中满,按之没指,又云∶水病忌腹上出水,出水者一月死,太太忌之。

又云∶水病忌丧孝、产乳、音乐、房室、喧戏、一切鱼、一切肉、生冷、酢滑、蒜、粘米、
治大腹水肿方第十八

《病源论》云∶夫水肿病者皆营卫痞涩,肾脾虚弱所为,而大腹水肿者,或因大病之后,或水大《千金方》治大腹水肿、气息不通。命在旦夕者方∶牛黄(二分)昆布(十分)海藻(十分)牵牛子(八分)桂心(八分)椒目(三两熬)葶苈七味,别捣葶苈,熟如膏,合和丸如梧子,饮服十丸,日再,稍加,小便利为度。

《医门方》疗大腹水肿,遍身洪满,小便涩少方∶海蛤(六合,研如面)葶苈子(十分熬)茯苓(六分)橘皮(四分)郁李仁(四分)桑根白捣筛为散,别捣葶苈子如脂,纳散中,蜜丸,空腹饮服七丸,日一,当利一二行,如不利,《新录方》治水病腹大面肿小便涩方∶葶苈子(一升,熬紫色,捣如泥)芒硝(三两)吴茱萸(三合,捣为散)合三种更捣,加少蜜可丸,捣二千杵,饮服七丸,日二服,以小便利为度,忌咸醋,瘥止。

《效验方》治水大腹葶苈子散方∶蓝叶(三两)大黄(一两半)葶苈子(二两,熬)凡三物,治筛,先食,酒服二方寸匕,欲丸服,蜜和服如大豆,日二十丸。

《短剧方》治水肿大豆汤方∶大豆三斗,水五斗,煮令熟,出豆澄汁,更纳美酒五升,微火煎如饴,服一升,渐增之,令《耆婆方》治人水病、四肢脚肤(音府,腐字也)面腹俱肿方∶香薷一百斤,以水煮之令熟,去滓更煎,令如饴糖,少少服之,当下水,小便数即瘥。

又云∶治人多水身重,口中水出,面虚越肿,宜泻方∶桂心(一两半)大腹槟榔(三七,口捶研)生姜(一两半)三味,切,以水九升,煮取三升,去滓,分为三服,当下水即瘥。

《葛氏方》治大腹水病方∶防己甘草葶苈(各二两)捣,苦酒和,服如梧子三丸,日三,恒将之,取都消乃止。

又方∶白茅根一大把,小豆三升,水五升,煮讫,去茅根食豆,水随小便下。

又方∶恒啖小豆饭并饮汁佳。

《本草经》云肉主久水胀不瘥、垂死者,作羹食之,下水大效。

《极要方》疗大小肿神方∶白饧(四两)桂心(一两)桑根(六升)甘草(一两炙)人参(一两)细辛(一两)大枣(上,以水九升,煮桑根,取三升以煮药,取一升去滓,纳饧令烊尽,分三服,一日一夜,小便五六升即瘥。

《龙门方》疗腹满如石、积年不损方∶取白杨树东南皮或枝,去苍皮,护风细押削五升,熬云∶每服一合,日三。

灸水病法∶《短剧方》云∶灸膈俞(在第七椎下两旁一寸半)百壮,三报,灸脾俞(在十一椎下两旁一寸
治通身水肿方第十九

《病源论》云∶水病者,由肾脾俱虚故也,肾虚不能宣通水气,脾虚又不能制水,故水气盈溢、渗(色荫反)液皮肤、流通四肢,所以通身肿也。

《僧深方》治通身水肿,大小便不利方∶常陆根(三升,薄切)赤小豆(一斗)凡二物,水一斛,煮取一斗,稍饮汁,食豆,以小便利为度。

又云∶治大水面目身体手足皆肿方∶大戟(分)葶苈(三分,熬)苦参(一分)葱花(一分)凡四物,治下筛,以小麦粥服方寸匕,良效。

《范汪方》治身体流肿,心下胀满,短气逆害饮食方∶大豆一斗,以水三斗煮之,令得一斗七升,去滓,纳一斗好酒,合煎之,令得一斗七升,服《经心方》泻肺汤治一身面目浮肿方∶末葶苈(弹丸)大枣(二十枚)水三升,煮枣取汁一升半,去枣,纳葶苈,煮取一升,顿服之,得至五服,若带水气者,先(今按∶《医门方》∶葶苈子二两,熬,大枣三十枚,水一斗,云云。)《张仲景方》青龙汤,治四肢疼痛、面目浮肿方∶麻黄(半斤)细辛(二两)干姜(二两)半夏(二两)凡四物,切,以水八升,煮得二升,一服止。

又云∶治脾胃水,面目手足浮肿、胃脘坚大满短气,不能动摇桑根白皮汤方∶桑根白皮(切二升)桂(一尺)生姜(三颗)人参(一两)凡四物,切,以水三斗,煮取桑根竭得一斗,绞去滓,纳桂、人参、生姜,黄饴十两煮之,《医门方》疗遍身肿胀小便涩方∶葶苈子(二两熬捣如泥)大枣(三十枚去核)以水一大斗,煮取一小升,绞去滓,纳葶苈子于枣中,以微煎,揽勿停手,可丸止,丸如梧
治十水肿方第二十

《病源论》云∶十水者,青水、赤水、黄水、白水、黑水、悬水、风水、石水、里水、气水先从水者者,乍来气,《短剧方》十水丸,治水肿方∶肿从头起,名为白水,其根在肺,椒目主之;肿从胸起,名为气水,乍实乍虚,其根在肠,芫花主之;肿从股起,名为黑水,其根在肾,玄参主之;肿从面起,至足,名为悬水,其根在胆,赤小豆主之;肿从内起坚块,四肢肿,名为石水,其根在膀胱,桑根主之;肿从四肢起,腹肿,名为风水,其根在胃,泽漆主之;肿从腹起,名为冷水,其根在小肠,巴豆主之;肿从胸中起,名为赤水,其根在心,葶苈主之。上十种,随其病始所在,增其所主药皆一分,巴豆四分,去心、皮,冶末,合下筛蜜丸,服如梧子三丸,得下为度,不下,日三,亦可作散末食服半钱匕,大便利,明朝复服如法,再服病愈,即禁饮食,但得食干鱼耳。

又云∶十水散,治水肿方;先从脚肿,名曰清水,其根在心,葶苈子主之;先从阴肿,名曰肿,从口先从先从背肿,名曰鬼水,其根在胆,雄黄主之。上十物药二分,合捣下筛,空腹以水服方寸匕,当下水多者,减服,下少者益之。

《范汪方》治十水丸方∶第一之水,先从面目;肿遍一,身,名曰青水,其根在肝,大戟主之;第二之水,先以心肿遂主名曰黑之;第四肢小水,其小豆主之日三,欲熟之。

《僧深方》治身体浮肿十水散方∶芫花(三分)决明(三分)大戟(三分)石苇(三分去毛)巴豆(三分,去心)泽泻(三分)凡十物,冶下筛,以大麦粥清汁服方寸匕,日三。
治风水肿方第二十一

《病源论》云∶脾胃虚,不能制于水,故水散溢皮肤,又与风湿相搏,故云风水也,令人身《短剧方》葱豆洗汤,治虚热及石热当风露卧冷湿伤肌,热菹(侧于反,《方言》∶积也,盛也,绕也)在里,变成热风水病。心腹肿满,气急不得下,头小便不利,大便难,四肢肿如皮囊盛水,晃晃(胡广反,老蚕形也)如老蚕色,阴卵坚肿如斗,茎肿生疮如死鼠,此皆虚葱豆膏敷之,别以猪蹄汤洗阴茎创烂处及卵肿也。

葱合青白(切,一升)蒺藜子(一升,舂碎)赤小豆(一升)菘菜子(一升,舂碎)蒴(上六物,以水一石二斗,煮取八斗,以淋洗身肿处,葱豆膏、猪蹄汤方在本方。

又云∶葶苈子回神酒,治风水通身洪肿,肉如裂者、服之小便利自随消,方∶春时酿清酒五斗(一方五升),葶苈子三升,熬,着酒中渍再宿,便服一合,以渐增之,病《极要方》治风水毒瓦斯遍身肿方∶猪白皮(三两)桑根白皮(五两)橘皮(一两)紫苏(二两)生姜(四两)大豆(三升)上,以水九升,煮取一大升,分四服,三剂,百日内忌咸、醋。

《经心方》治风水大豆煎∶生桑根白皮(细切三升,入土一尺者)大豆(一斗)二味,以水六半,煮取一斗,去滓,下生姜汁二升,更煎取四升,分五合,第一服,日三夜又云∶牵牛丸治脚肿满,步行不能,众恶毒水肿方∶大黄(二两)朴硝(三两熬)牵牛子(七两熬)桃仁(二两,去心熬)干姜(二两半)人参上七物,捣下筛,以蜜和舂万杵,服如梧子二十丸,以微利为度,肿即止,不瘥尽剂,尽剂《僧深方》治风水肿、症癖,常陆酒方∶常陆根(一升,切)凡一物,以淳酒二斗渍三宿,服一升当下,下者减从半升起,日三,不堪酒者以意减之。

又云∶治通身肿,皆是风虚水气,亦治暴肿痛,蒲黄酒方∶蒲黄(一升)小豆(一升)大豆(一升)凡三物,清酒一斗,煮取三升,分三服。

《耆婆方》治人风水气,面身俱肿,上气腹胀不能食,羸弱在床,经时不瘥者方∶小豆(三升)大麻子(三升,捣碎,以水研汁)桑根白皮(一斤)合煮豆熟,食豆饮汁即大下水,即瘥。

《苏敬本草注》煮鱼食之,主水浮肿。

又方∶服鸭肪,食鸭头主水肿。

又方∶桑灰汁煮大豆,饮之下水肿。
治水癖方第二十二

《病源论》云∶水癖由饮水浆不消,水气结聚而成,癖在于两胁之侧,转动便痛,不耐风寒《效验方》云∶可服诸下药泻之。

《玄感方》∶灸挟脐两旁各三十壮。

又方∶灸两乳下一夫肋间二穴。

《新录方》杏仁作煎或作丸,酒服如枣,日三。

又方∶灸桃仁准上。

又方∶单熬大麦为散,服如上。

《本草经》水研蓖麻子二十枚,服吐恶沫也。
治身面卒肿方第二十三

《病源论》云∶身面卒洪肿者,亦水病之候也,脾虚不能制水,故水流溢,散于皮肤,令身体卒然洪肿。

《范汪方》治卒肿大戟洗汤方∶大戟(四两)莽草(二两)茵芋(二两)大黄(二两)黄连(二两)芒硝(二两)葶苈(二凡八物,皆咀,以水一斗五升,煮得一斗,绞去滓,洗肿上,日三。

又云∶治诸卒肿风掣痛方∶末芥子,温汤和涂纸上以贴,燥复易,不堪其痛,热者小涂之,《僧深方》治暴肿方∶破鸡子搅令其黄白涂肿上,燥复涂,大良。

又方∶大豆一升,熟煮饮汁,食豆不过三作良。

《落手方》治暴肿方∶捣葶苈子,薄之两三过即消。

《葛氏方》治卒肿身面皆洪大方∶凡此肿,或者虚气、或者风冷气、或者水饮气,此方皆治之∶用大鲤鱼一头,以淳苦酒三升又方∶大豆一升,熟煮漉,饮汁食豆,不过三作必愈,小豆亦佳。

又方∶大豆一斗,以水五斗,煮取二斗,去豆,纳酒八升,更煎取九升,分三四服,肿瘥又方∶灸足内踝下白肉际三壮。

又云∶治肿入腹苦满,急害饮食方∶葶苈(七两)椒目(三两)茯苓(二两)吴茱萸(二两)又方∶大戟乌扇术(各二两)捣筛蜜丸如梧子,旦服二丸,当下。

又方∶若肿从脚起,稍上进者,入腹则杀人,治之方∶生猪肝一具细切,顿食,勿与盐,乃可用苦酒耳。

又方∶煮豉汁饮之,以滓薄脚。
治犯土肿方第二十四

《病源论》云∶犯土之病,由居住之处穿凿地土,犯触土气而致病也。令人身之肌肉头面遍《新录方》云,以水服伏龙肝方寸匕,此正犯土上气兼肿,大好。

《本草拾遗》云,药有同类相伏者,不伏水土服土云云。以此按之,以鼠壤土熏熨肿上可善
治黄胆方第二十五

《病源论》云∶黄胆之病,此由酒食过度,腑脏不和,水谷相并,积于脾胃,复为风湿所搏者,《葛氏方》云∶黄病有五种,谓黄汗、黄胆、谷疸、酒疸、女劳疸也。又云治黄胆一身面目以又方∶捣生麦苗,水和绞取汁,服三升,小麦胜大麦。(今按∶《范汪方》∶不和水。)《范汪方》治黄胆茵陈汤方∶茵陈蒿(六两)大黄(二两)栀子(十四枚)凡三物,水一斗二升,先煮茵陈蒿减六升,去滓,内大黄、栀子煮取三升,分三服之。

《短剧方》治黄胆身目皆黄,皮肤尘出,三物茵陈汤方∶茵陈蒿(一把)栀子(十四枚)石膏(一斤)凡三物,以水八升,煮取二升半,去滓,石膏一斤,猛火烧令赤,投汤中沸,定清取汁,适《千金方》治黄胆身体面目皆黄三黄散方∶大黄(四分)黄连(四两)黄芩(四两)三味为末,先食,服方寸匕,日三。今按《范汪方》∶丸如梧子,服十丸。又《本草拾遗》云∶药有同类相伏者,身黄服黄物。

《录验方》治黄胆,大小便不利;面赤汗自出,此为表虚里实,大黄汤方∶大黄(四两)黄柏(四两)栀子(十五枚)硝石(四两)凡四物,切,以水一斗,煮得二升半,去滓,纳硝石复煎之,得二升,分再服,得快下乃愈。

又方∶芜菁子五升,末服方寸匕,日三。

《医门方》疗黄病,身体面目悉黄如橘,由暴热外以冷迫之,热因流入胃中所致方∶栀子仁(三两)栝蒌(二两)苦参(二两)龙胆(三两)大黄(三两)捣筛蜜丸,饮服三十丸,日三。加至五六十丸。

《广济方》疗五种黄方∶丁香(七枚)瓜蒂(七枚)赤小豆(七枚)为散,取暖水一鸡子许,和一钱匕服之,忌诸热食。

《经心方》治黄胆单方∶枸杞合小麦煮勿令腹破,熟而已,日食三升。

《枕中方》治人黄病垂死者,服铁浆一升即愈。

灸黄胆法∶《葛氏方》灸脾俞百壮;又方∶灸手太阴随年壮;又方∶灸钱孔百壮;又方∶灸胃脘百壮。

《范汪方》灸脐上下两边各一寸半二百壮。

《僧深方》灸第七椎上下;又方∶屈手大指,灸节上理各七炷;又方∶灸脊中椎七炷。

《经心方》灸两手心各七壮。(《僧深方》同之。今按∶可服诃黎勒丸、紫雪、红雪等。)
治黄汗方第二十六

《病源论》云∶黄汗之为病,身体洪肿发热,汗出不渴,状如风水,汗染衣正黄如柏汁,此《医门方》疗黄汗∶黄汗之病状,如风水,其脉沉迟,皮肤冷,手足微厥,面目四肢皮肤皆夕药(八两)桂心(三两)黄(五两)苦酒(五合)以水七升,煮取三升,饮一升心当烦,勿怪,至六七及即瘥。(今按∶《葛氏方》∶夕药三
治谷疸方第二十七

《病源论》云∶谷疸之状,食毕头眩、心怫郁不安而发黄,由失肌大食,胃气冲熏所致。

《葛氏方》治谷疸方∶茵陈蒿四两,水一斗,煮得六升,去滓,纳大黄二两,栀子二七枚,煮服二升,分三服,尿《僧深方》治谷疸发寒热,不可食。食即头眩,心中怫冒不安大茵陈汤方∶茵陈蒿(二两)黄柏(二两)大黄(一两)甘草(一两)人参(一两)栀子(十四枚)黄连凡七物,切,水一斗,煮得三升,分三服。
治酒疸方第二十八

《病源论》云∶夫虚劳人若饮酒多、进谷少者则胃内生热,因大醉当风入水,则身自发黄,蒜齑其《千金方》治饮酒食少饮多澹结,发黄胆,心中懊而不甚热,或干呕,枳实大黄汤下之,枳实(九枚)大黄(二两)豆豉(半升)栀子(十枚)上四味,以水六升,煮取二升,分三服。今按∶《葛氏方》∶大黄一两,枳实五枚,栀子七枚豉一升。

《范汪方》治饮酒得黄病方∶柏生麦,以井华水绞取汁,顿服三升,先食,日三,有验。

《僧深方》治酒疸方∶生艾叶(一把)麻黄(二两)大黄(六分)大豆(一升)凡四物,清酒三升,煮得二升,分三服。艾叶无生,用干半把。
治女劳疸方第二十九

《病源论》云∶女劳疸之状,身目皆黄,发热恶寒,少腹满急,小便难,由大热而交接,交《千金方》云黄胆晡日晡发热恶寒,少腹急,体黄颜黑,大便溏黑,足心热,此为女劳也,硝滑石石膏上二味,分等治,以大麦粥汁服方寸匕,日三,小便极利则瘥。(今按∶《葛氏方》∶有硝《医门方》疗女劳疸、谷黄方∶苦参(四两)龙胆(三两)栀子(五两)捣筛为散,以猪胆汁和如梧子,以饮服二十丸,日三。
治黑疸方第三十

《病源论》云∶黑疸之状,若小腹满,身体尽黄,额上黑,足下热,大便黑是也。夫黄疸、《葛氏方》治黑疸者多死急治之方∶土瓜根捣绞取汁,顿服一升,至三升顷,病当随小便去,不去更服之。(今按《范汪方》云
卷第十一
治霍乱方第一

《病源论》云∶霍乱者,由人温凉不调,阴阳清浊二气有相干乱之时。其乱在于肠胃之间者者,饮酒食胃,脾消则令乱,言《养生方》云∶七月食蜜,令人暴下,发霍乱也。

《千金方》云∶论曰∶原夫霍乱之为病也,皆因食饮,非关鬼神。夫饱食肫脍,复餐乳酪、阳气欲升,阴气欲降,阴阳永隔,变成吐利,头痛如破,百节如解,遍体诸筋皆为回转,论时虽小,卒病之中,最为可畏,虽临深履危不足以喻之也。养生者宜达其旨趣,庶可免于夭横者矣。

《极要方》云∶得吐利者名湿霍乱,不得吐利者名干霍乱。干霍乱多杀人,往往有湿霍乱,不有性命之忧。

《集验方》云∶呕而吐利,此为霍乱也。

《葛氏方》云∶凡所以得霍乱者,多起于饮食,或饱食生冷物,杂以肥鲜酒脍,而当风履湿初得之便务令温暖,以火炭布其所卧床下,大热减并蒸被絮,若衣絮自抱,冷易热者。

又方∶可烧地,令热水泼敷;蒋席卧其上,浓覆之。

又方∶可作的尔热汤,着瓮中渍足令至膝,并铜器若瓦器盛汤,以着腹上,衣藉之,冷复易又方∶可以熨斗盛火着腹上而不静者,便急灸之。灸之但明按次第,莫为乱灸,须有其病,火下《千金方》云∶凡诸霍乱;忌米饮,胃得中米即吐不止,但得与浓朴葛根饮。(今按∶《葛氏方》云∶可□备浓朴,每向秋月,便自随之。)又云∶凡霍乱医所不治方∶童女月衣,合血烧,酒服方寸匕,秘方。(今按∶《短剧方》云《范汪方》治霍乱吐下不止,理中汤方∶人参干姜白术甘草(各一两)水三升,煮取一升半,分二服。

(今按∶《短剧方》∶药各三两,水六升,煮取三升,分三服。又《医门方》∶白术三两,人参三两,甘草二两,干姜二两。水七升,煮取二升半。若胸满腹痛吐下者,加当归、浓朴各二两;若悸者、寒者、渴者,并主之。)《录验方》治霍乱虚冷吐逆下利理中丸方∶人参甘草(炙)干姜白术(各二两)凡四物,捣下蜜丸如弹丸,取一丸纳暖酒中服之,日三。

(今按∶《本草苏敬注》云∶方寸匕散为丸如梧子,得十六丸,如弹丸一枚。)又方∶单煮浓朴,饮一二升,有效。

又方∶煮梨叶服之。(今按∶《医门方》云∶取梨枝叶一大握,以水二升,煮取一升,顿服立瘥。)《效验方》治霍乱吐下,理中散方∶甘草(二两,炙)人参(二两)干姜(二两)术(二两)凡四物,冶筛,酒服方寸匕,日三。

《短剧方》云∶扶老理中汤,治羸老冷气恶心,食饮不化,腹虚满拘急短气,及霍乱呕逆,人参(五两)干姜(六两)术(五两)麦门冬(六两)附子(三两)茯苓(三两)甘草(五凡七物,下筛作散,临病者三合,白汤和方寸匕,一服不效,又服常将者,蜜丸酒服如梧子又云∶霍乱吐下汗出,肉冷转筋,呕逆烦闷,欲得冷水者方∶可与浓朴葛根饮进,沾喉中而又方∶取藿香一把,以水四升,煮取一升,顿服立愈。

(今按∶《本草》云∶一把重二两为正。)又方∶煮青木香汁饮,至佳。

又云∶治卒道中得霍乱,无有方药,危急方∶芦蓬茸大把,煮令味浓,顿服二升即瘥,有效《医门方》治霍乱吐下不止者方∶煮百沸汤,细细添生水,热饮之。(今按∶《删繁论》云又方∶或煮高良姜或煮木瓜汁饮之。

《通玄经》云∶治霍乱方∶木瓜煮作饮服之(今按∶《本草陶注》云∶若子并枝煮饮。)《救急单验方》疗霍乱方∶桂三两,煮汁取一盏顿服,验。

《陶景本草注》云∶霍乱吐下方∶楠材削作煮服之。
治霍乱心腹痛方第二

《病源论》云∶霍乱而心腹痛者,是风冷之气客于腑脏之间,冷气与真气相击,或上攻心,《葛氏方》治霍乱若心腹痛、急似中恶者方∶捣生菖蒲根饮汁,少少令下咽即瘥。

又云∶卒得霍乱先腹痛者方∶灸脐上一夫十四壮,名太仓。若绕脐痛者,灸脐下三寸四壮,《极要方》疗霍乱心腹激痛方∶当归(三两)桂心(三两)干姜(三两)甘草(一两)上,以水七升,煮取二升,分三服。

《短剧方》治心腹暴痛,及宿食不消,或宿冷烦满成霍乱方∶作盐汤三升,使极咸,热饮一《又云∶霍乱腹痛吐下方∶取桃叶,冬天用皮,绞取汁,一服一杯,立愈。亦可浓煮,饮三升《耆婆方》治霍乱先腹痛方∶煮生姜热饮之。

又方∶浓朴汁饮之。

《僧深方》治霍乱腹痛而烦方∶高良姜四两,以水五升,煮取二升,分二服。

《苏敬本草注》霍乱绞痛方∶粟米泔汁,饮数升,立瘥。

《广济方》疗霍乱心腹痛烦呕不止方∶浓朴(四两)橘皮(二两)人参(二两)当归(二两)藿香(一两)高良姜(四两,切)以水七升,煮取二升五合,分温三服,忌生冷、粘食。

《通玄方》治霍乱先腹痛方∶好验酢细细饮一盏许。

又方∶用火灸腹及背,得汗即愈。
治霍乱心腹胀满方第三

《病源论》云∶霍乱而心腹胀满者,是寒气与脏气相搏,真邪相攻,不得吐利,故令心腹胀《葛氏方》治霍乱心腹胀痛,烦满短气未得吐下方∶生姜若干,姜一二升,以水五六升,煮三沸,顿服,若不即愈,更可作。

又方∶桂屑半升,以暖饮和之,尽服。

又方∶以盐纳脐中,灸上二七壮。

又云∶治苦烦闷腠满者方∶灸心二七壮。

又云∶治烦呕腹胀浓朴汤方∶浓朴(四两)桂(二两)枳实(五枚)生姜(三两)以水六升,煮取二升,分三服。

《耆婆方》治霍乱烦闷腠满方∶浓朴二两,炙,以水三升,煮取一升半,分三服,老人小儿《范汪方》治霍乱腹中胀满、恶毒闷绝不通气,气息急危方∶生姜(一累)栀子(十四枚)桂心(一两)香豉(五合)四物,捣,以酒二升解之,去滓顿服。(今按∶《录验方》∶生姜累数以其一支为累,取服《短剧方》治霍乱腹痛胀满短气不得吐下,灸不效者,热伏心脏中,烦闷郁郁者方∶可取白无粉《僧深方》治霍乱腹胀满不得吐方∶粱米粉五合,以水一升半,和如粥顿服,须臾吐,若不
治霍乱心烦方第四

《病源论》云∶霍乱而心烦者,由大吐大利,腑脏气暴极,故心烦,亦有未经吐利而烦者,《医门方》疗霍乱心烦方∶香豉〔七合(绵裹)〕栀子仁(三两)浓朴(三两,炙)水五升,煮取二升,去滓,分温二服,重者,不过再,必愈。

《通玄经》云∶霍乱心烦闷不已方∶用粟米汁饮之半升,即愈。

又方∶服粟米粉,和水服一合,立愈。

又云∶霍乱吐下已止,发热心烦欲饮水方∶可与少秫米粉汁佳,若不止,可与葛根荠(脐
治霍乱下利不止方第五

《病源论》云∶霍乱而下利不止者,是肠胃俱冷而挟宿虚,谷气不消故也。

《短剧方》治霍乱洞下腹痛方∶以艾一把,以水三升,煮得一升,顿服之良。

又云∶霍乱卒吐下不禁者,人参汤主之,方∶人参(二两)茯苓(二两)葛根(二两)橘皮(二两)麦门冬(二两)甘草(二两)凡六物,以水五升,煮取二升,分三服。

《葛氏方》治霍乱下利不止者方∶灸足大指本节内一寸侧白肉际,左右各七壮,名大都。

又云∶霍乱吐下不止方∶干姜茱萸各一两,水二升,煮取一升,一服。

又云∶灸两乳边,里外近腋白肉际各七壮。

今按∶治霍乱吐利,理中汤、理中丸主之,在上条。

又云∶先洞下者,灸脐边一寸,男左女右,十四壮。又云吐而下不止者,脐下一夫约中七壮
治霍乱呕吐不止方第六

《病源论》云∶霍乱而呕吐者,冷气入于胃,胃气变乱,冷邪既盛,谷气不和,胃气逆上,《葛氏方》治霍乱呕不止方∶生姜五两,水五升,煮取二升半,分三服。

《僧深方》治霍乱烦痛、呕吐不止并转筋方∶生香(一把)桂心(二两)生姜(三两)三物,以水七升,煮取二升,分二服,甚良。

又云∶霍乱呕吐,水药不下茱萸汤方∶茱萸(一升)黄连(二两)附子(一两)甘草(一两)生姜(三两)凡五物,以水七升,煮取三升,分三服。

《短剧方》治霍乱呕吐及暴下方∶半夏(三两)干姜(四两)人参(三两)桔梗(三两)附子(四两)凡五物,下筛,临病和之,若吐下不止者,以苦酒和之,饮服二丸如梧子大。

《范汪方》治霍乱呕吐附子汤方∶大附子(一枚)甘草(六铢)蜀椒(二百粒)三物,水三升,煮取一升半,分再服。
治霍乱呕哕(于越反)方第七

《病源论》云∶霍乱而呕哕者,由吐利后胃虚而逆则呕,气逆遇冷折之,气不通则哕之。

《范汪方》治霍乱呕哕、气厥不得息方∶香豉(一升)半夏(一两)甘草(一两)生姜(二两)人参(一两)柴胡(一两)六物,以水五升,煮取二升半,服七合,日三。

又方∶半夏二两,生姜二两,水三升,煮得一升二合,分再服。

《葛氏方》∶霍乱若者,灸手腕第一约理中七壮,名心主,当中指也。

《短剧方》治霍乱呕哕吐逆,良久不止方∶灸巨阙并太仓各五十壮。(今按∶巨阙穴在去鸠尾骨端一寸;太仓者中管穴,在上管下一寸
治霍乱干呕方第八

《病源论》云∶霍乱而干呕者,吐下后,脾胃虚冷,三焦不理,气痞结于心下,气时逆上,《短剧方》治霍乱或引饮,饮辄干呕方∶生姜五两,以水五升,煮令得二升半,分再服良。

又云∶治干呕逆哕,手足厥冷橘皮汤方∶橘皮(四两)生姜(半斤)凡二物,以水七升,煮取三升,一服一升,汤下咽,即愈。

《葛氏方》霍乱干呕者方∶灸手腕后三指(寸)大两筋间,左右各七壮,名间使。

又方∶取薤一虎口(一握也),以水二升,煮令得一升半,服之,不过三作。
治霍乱烦渴方第九

《病源论》云∶霍乱烦渴者,大利则津液竭,津液竭则脏燥,脏燥则渴也。烦渴不止则引饮《葛氏方》治霍乱吐下后大渴多饮则杀人方∶可以黄粱米五升,水一斗,煮得三升,澄,稍稍饮之,勿饮余饮之。

《医门方》霍乱热心烦渴者方∶以糯米水渍研之,以冷熟水混,取米泔汁,恣意饮之即定,《僧深方》霍乱吐后烦而渴方∶紫苏子一升,水五升,煮取二升,分二服。无子取生苏一把《短剧方》治霍乱烦渴者方∶粢米汁泔,饮数升,立瘥。

又方∶取新汲冷水饮之。

《集验方》云∶治霍乱而渴者,理中汤主之。
治霍乱转筋方第十

《病源论》云∶霍乱而转筋者,由冷气入于筋故也。

《葛氏方》霍乱转筋者方∶灸趾心下五六壮,名涌泉。

又方∶灸大指上爪甲际,七壮。

又方∶苦酒和粉涂痛上。

又云∶转筋入腹痛者方∶令四人捉手足,灸脐左一寸,十四壮。

又云∶若转筋入腹中如欲转者方∶烧编席索三指撮,酒服之。

又云∶釜底墨末,酒服之。

又云∶腹中已转筋者方;当倒檐,病患头在下,勿使及地,腹中平乃止。

又云∶若两臂脚及胸胁转筋者方∶取盐一升半,水一斗,煮令热,渍手足,在胸胁者汤洗之《千金方》霍乱转筋方∶蓼一把,去两头,水二升,煮取一升,顿服之。

又方∶纳盐脐中,灸二七壮,并治腹胀。

又方∶以车毂中脂涂足心下。又方∶灸足踵取筋上白肉际七壮,立愈。又方∶灸少腹横骨中又云∶转筋在两臂及胸中方∶灸手掌白肉际七壮。

又灸膻中、中府、巨阙、胃脘。

《短剧方》治霍乱转筋方∶以苦酒煮青布裹KT(音奄,渍也,又作踏,徒合反)之,冷复易。

又方∶可以白(侧吏反)煮粉及热洗之。

《龙门方》治霍乱转筋方∶取木瓜子、根、茎煮汤服,验。

《医门方》治霍乱转筋方∶取热灰,以验醋和令微温,炒令极热,以青布裹,及热熨筋上又云∶霍乱遍身转筋入腹不可奈何方∶多作盐汤内船槽(昨营反)中令温暖,渍足之佳。

《小《极要方》霍乱转筋方∶生姜(一斤)上,水七升,煮取二升,分三服。

《删繁论》云∶霍乱转筋方∶取絮巾若绵,灸暖以敷筋上。(今按∶《范汪方》∶转筋在脚《范汪方》治霍乱转筋方∶鼠壤土,水和涂其上,愈。

又方∶取KT合粉小温之,涂手摩之。

《广利方》治霍乱转筋入腹方∶取盐三合,以水五升,煮取三升,以青布浸汤中,用拭转筋《陶景本草注》云∶治霍乱转筋者,但呼木瓜名,及书上作木瓜字,皆愈。
治霍乱手足冷方第十一

《病源论》云∶霍乱大吐下,其肠胃俱虚,乃至汗出,其脉欲绝,手足皆冷,名为四逆者,《葛氏方》先手足逆冷者方∶灸足内踝上一夫,两足各七壮。

又云∶治下不止手足逆冷方∶椒百枚,附子一枚,水三升,煮取一升,一服。

《医门方》霍乱吐利不止,心烦,四肢逆冷方∶浓朴甘草人参白术(各二两)生姜(三两)以水六升,煮取二升,分二服。

《短剧方》霍乱多寒,手足寒厥,脉绝,茱萸四逆汤主之,方∶吴茱萸(二升)当归(三两)夕药(二两)桂心(四两)细辛(二两)生姜(半斤)通草(凡八物,以水四升,清酒四升,合煮取三升,分四服。

《录验方》∶霍乱吐下而汗出,小便复利,或下利清谷,里外无热,脉微欲绝,或恶寒,四人参(三两)干姜(三两)附子(二两)甘草(三两)凡四物,以水六升,煮取二升半,分三服,转筋肉冷汗出、呕者良。(《短剧方》同之。)
治霍乱不语方第十二

《葛氏方》治霍乱欲死不能语方∶生姜一斤,切,水七升,煮取二升,分三服。

又方∶饮竹沥少许。

又方∶芦蓬茸大把,浓煮饮二升即瘥。

又方∶干姜(三两)甘草(一两)附子(一两)水三升,煮取一升,分三服。
治霍乱欲死方第十三

《病源论》云∶霍乱而欲死者,由饮食不消,冷气内搏,或未得吐利,或虽吐利,冷气未歇《极要方》疗霍乱呕而烦闷,胀喘垂死,经日不解方∶浓朴(四两)桂心(四两)枳实(三两)生姜(十两)以水八升,煮取三升,分三服。

《葛氏方》治霍乱众治不瘥,烦躁欲死,胀气急方∶烧童女月经衣血末,以酒服少少,立瘥又云∶治霍乱神秘起死灸法∶以物横度病患口中,屈之从心鸠尾度以下,灸度下头五壮,横又方∶灸脊上,以物围令正心厌。又夹脊左右一寸各七壮,是腹背各灸三处也。

又云∶华佗治霍乱已死,上屋唤魄者,诸治皆至而犹不瘥者方∶捧病患覆卧之,伸臂对,以即又云∶注利不止而转筋入腹欲死方∶生姜三累,拍破,以酒升半,煮三四沸,顿服之。
治中热霍乱方第十四

《短剧方》治中热暴下利霍乱变热心烦脉数者方∶饮新出井水一升,立愈。饮多益善,此治又云∶荠汤治先有石热,因霍乱,吐下,服诸热药,吐下得止,因空虚仍变烦,手足热,闷,荠(二两)人参(二两)浓朴(二两)知母(二两)栝蒌(二两)葛根(二两)枳实(二(二两)黄芩(二两)甘凡十四物,以水八升,煮取三升,分五服。
治欲作霍乱方第十五

《耆婆方》治人腹胀欲作霍乱方∶浓朴(二两,炙)以水三升,煮取一升半,分三服即瘥,小儿最喜,老人亦佳,夏秋月,恒置此药在家,有急《医门方》疗心腹胀满坚痛,烦闷不安,虽未吐下,欲霍乱方∶取盐五合,水一升,煮令消煮取二升,顿服。)《葛氏方》治霍乱心腹胀痛、烦满短气,未得吐下方∶生姜若干姜一二升,以水五六升,煮三沸,顿服。

《短剧方》治霍乱烦扰、未得吐下方∶煮香汁,热饮之。(今按∶《葛氏方》云∶取蓼若
治霍乱后烦躁方第十六

《病源论》云∶霍乱之后而烦躁卧不安者,由吐下后腑脏虚极、阴阳未理,血虚气乱,血气《葛氏方》治霍乱后烦躁卧不安方∶葱白(二十枚)大枣(二十枚)水二升,煮取一升,顿服之。
治霍乱止后食法第十七

《医门方》云∶若霍乱吐利定后,胃气虚弱,不可强食,便更作病,候待须食,仍微与稀饮《千金方》云∶凡此病定已一日,不食为佳,仍须三日,少少与粥,三日以后,乃恣意食息也,仍七日勿杂食为佳。

又云∶诸霍乱忌米饮,胃中得米即吐不止,但得与浓朴葛根饮。
治下利方例第十八

《千金方》云∶论曰,凡利有四种,论冷热甘蛊。冷则白,热则赤,甘则赤白而杂,蛊则纯又云∶古今利方千万,不可具载此中,但撮其效者七八而已,虽然,弘之在人也,何则陟厘愈也?《医门方》云∶辨曰∶比见下利者,因多触热饮水,或夜露坐卧冒霜雪,风寒邪气客于皮肤而又云∶下利欲饮者,为有热;利而不渴,其脏有寒;下利后更烦,按其心下软者,为虚烦也又云∶下利有不欲食者,有宿食,宜下;下利瘥,时时复发,此为不尽,宜下;下利腹满不下利又云∶下利甚者,手足痹不仁;下利而反发热、身形汗出者,自愈;下利,脉数而渴欲饮,又云∶下利,手足无脉,灸之不温,微喘者,死;下利后,脉绝,手足厥冷,卒时脉还,手又云∶凡人久下利或热,暴下两日即有虫,口唇生疮,口中舌上唇并生疮如粟,疮剧者之《葛氏方》云∶下利,手足逆冷,灸之不暖,或无脉,微喘者,死;下利,舌萎,烦躁而不今按∶下利人可食物∶赤小豆(孟诜云∶止痢。)小麦(《本草》云∶止利。)青粱米(《本草》云∶止泄。)黄粱米(《本草》云∶止泄。)丹黍米(《本草》云∶止泄。)粳米(《本草》云∶止泄。)橘(《本草》云∶止泄。)柚(《本草》云∶止泄。)梅(《本草》云∶止下利。)柿(《本草陶注》云∶火熏者性热,断下。)石榴(《本草》云∶壳疗下利。)通草(崔禹云∶止云∶止禹云∶主鱼(崔禹禹云∶主止下名∶之多多美。

下利人可忌物∶《千金方》云∶凡利病,通忌生冷、酢滑、猪、鸡、鱼油、乳酪、酥、干脯、酱、粉、碱,《养生要集》云∶腹中有冷患,饮乳汁,令腹痛泄利。

《七卷食经》云∶杏仁不可多食,令人热利。
治杂利方第十九

《病源论》云∶杂利谓利色无定,水谷或脓或血,或青或黄,或赤或白,变杂无常,或杂色热不《短剧方》治杂下方,第一下赤,二下白,三下黄,四下青,五下黑,六固病下,下如瘀赤如舍水,十四下已则烦,十五息下一作一止,十六而不欲食,十七食无数但下者,十八下但欲饮黄连(一两)黄柏(一两)熟艾(一两)附子(一两)甘草(一两)干姜(二两)乌梅(二凡七物,合捣下筛,蜜和丸如大豆,饮服十丸,渐至二十丸,日三。(今按∶《葛氏方》云《范汪方》乌梅丸,治万种下利方∶干姜黄连黄柏(炙)黄芩艾(各一两)乌梅(二十枚,取肉)上六物,丸如梧子,服十丸,日三,老少半,良验。

又云∶治下利日百行、师所不治方∶曲末服一方寸匕,日三,以食愈为度,当以粟米粥服《令李方》治下利一日百起、黄连散方∶黄连(二两)甘草(二两)凡二物,冶筛,酒服方寸匕,日三,立愈。

《极要方》疗冷热不调,或滞或水、或五色血者方∶醋石榴五枚合壳、子捣,绞取二升汁,《广济方》疗百千种杂痢黄连汤方∶黄连(一两)干姜(一两)熟艾(一两)附子(一枚,炮)蜀椒(十四粒)阿胶〔如手(指)切,以水五升,煮取二升五合,绞去滓,纳胶更上火煎胶烊,分温三服。忌生冷猪鱼蒜。

《传信方》云∶一切利神效方∶黄连(二两半)黄柏(一两半)羚羊角(半两)茯苓(半两)上四味,为散,蜜和丸,用姜蜜汤下。

灸诸利方∶《经心方》云∶灸脐中,稍至二三百壮。

又灸关元三百壮,并治冷腹痛。(关元脐下三寸是也。)《新录方》云∶灸脊中三百壮。(脊中,从大椎度至穷骨中折,则是也。)又方∶灸脾俞百壮。(第十一椎两旁名脾俞。)又方∶灸大肠俞百壮。(第十六椎两旁名大肠俞。)今按∶《石论》云∶金液丹,十种水谷赤白等利,此丹皆治疗。
治冷利方第二十

《病源论》云∶肠胃虚弱,受于寒气,肠胃虚则泄,故为冷利也。凡利色青白黑皆为冷也,《千金方》驻车丸,主大冷洞利,肠滑下赤白如鱼脑,日夜无节度腹痛方∶黄连(六两)干姜(二两)当归(三两)阿胶(三两)四味,以大酢五合,烊胶,和之,并手丸如大豆服之,大人服三十丸,小儿百日以还,三丸,《葛氏方》下色白食不消者,为寒下,方∶干姜、赤石脂分等末,以白饮和丸如梧子,日服又方∶酸石榴皮烧末,服方寸匕。

又方∶生姜汁二升,蜜合煎取二升,顿服。

又方∶豉一升,薤白一把,水三升,煮取二升,及热顿服之,有大枣肉七枚良。

又云∶有止患冷者,淳下白如鼻涕,治之方∶龙骨、干姜、附子分等捣蜜丸,服如梧子五丸《范汪方》治寒冷下利方∶干姜(四两)人参(三两)桔梗(四两)附子(四枚,炮)半夏(三两,洗)凡五物,下筛和丸,平旦服五丸如梧子,日再,渐加,勿热食。

又云∶四顺汤治逆顺寒冷冻饮料食不调下利方∶甘草(三两)人参(二两)当归(二两)附子(一两)干姜(三两)凡五物,水七升,煮取二升半,分三服。(今按∶《僧深方》加龙骨二两。)又云∶四逆汤治下利清谷,身反恶寒,手足逆冷,此为四逆,四逆汤主之,相视病患与方相甘草(二两)附子(一枚)干姜(一两半)凡三物,以水三升,煮取一升二合,分二服。

《短剧方》黄连汤,治春月暴热解脱饮冷,或眠湿地,中冷腹痛,下青黄汁,疲极欲死方∶黄连(四两)当归(三两)干姜(三两)浓朴(二两)凡四物,切,以水七升,煮取三升,分三服。(今按∶《经心方》无浓朴,有石榴皮。)《私迹方》温中汤,治寒下饭臭出方∶甘草(一两,炙)干姜(半两)蜀椒(八十枚,去闭者)附子(一枚)凡四物,以水二升,煮取一升,分再服,若呕,纳橘皮半两,老少者皆取服良。

《广济方》疗冷痢青白色、腹内常鸣,行数疏出即大多,调中散方∶龙骨(一两)人参(一两)黄连(一两)阿胶(一两)黄柏(一两)捣筛为散,煮米饮服。忌猪、鱼、蒜、炙肉、粘食等。

《集验方》治久新寒冷下利,腹内不安,食辄注下,令人生肉乌梅丸方∶乌梅(三百六十枚,去核,熬令可捣)附子(四两,炮)黄连(十二两)干姜(四两)凡四物,捣,下筛蜜丸,饮服如梧子十丸,日再,补方。
治热利方第二十一

《病源论》云∶肠胃虚弱,风邪挟热乘之,肠虚则泄,故为热利也,其色黄,若热甚,黄而《录验方》青要结肠丸,治热毒下不绝,不问久新,悉治之方∶苦参橘皮阿胶(炙)独活夕药黄连蓝青(一方干姜四分代)鬼臼黄柏甘草(凡十物,合捣下筛,蜜烊胶和之,并手捻作丸如梧子,干以饮服十丸,日三,不知稍增。

(《经心方》乌梅汤,治热毒下有湿方∶黄连(二两)乌梅(三十果)阿胶(一两)栀子(三十枚)黄柏(一两)五味,以水五升,煮取二升半,分再服。

《新录方》热利者方∶干枣四十枚,水三升,煮取一升,顿服。

又方∶豉二升,水三升,煮取一升半,二服。

《耆婆方》黄连丸,治中热下利方∶黄连(十二分)干姜(八分)当归(八分)上三物,捣筛蜜和丸如梧子,服二丸,不知加之。

《葛氏方》∶下色黄者,挟毒热下也,治之方∶栀子十四枚去皮,捣,蜜丸如梧子,服三丸又,挟热者多下赤脓或杂血,治之方∶黄连,灶突中尘末,酒服二方寸匕,日三。

又方∶薤一把,煮鲫鱼,纳秫米食之,多善。

《千金方》治久利热诸治不瘥方∶乌梅肉(一升,熬)黄连(一斤金色者)二味,蜜和如梧子,服二十丸,日三夜一,神良。《僧深方》同之。

《广济方》疗热毒痢甚数,出不多,腹中刺痛方∶生犀角末(三两)酸石榴皮(三两,熬)枳壳(三两,熬)捣为散,饮服两方寸匕,日再,忌热食。
治赤利方第二十二

《病源论》云∶肠胃虚弱,为风邪所伤,则挟热,热乘于血,血流渗入腹,与利相杂下,故为赤利。

《医门方》疗赤利,腹中绞痛,下部疼重方∶黄连当归黄柏干姜(各二两)上,捣筛为散,煮乌梅汁,服方寸匕,日二。

《如意方》治下赤利术∶金色黄连(一升,去毛)黄柏(一斤)犀角(二两)凡三物,切,以水五升,煮取三升,去滓,纳白蜜一升,又煎三升,平旦服,至日中令尽,勿间食也。

《救治单验方》治赤利方∶黄连(三两)黄柏(三两)栀子仁(二两)凡三物,以水九升,煮取三升分三服。

又方∶捣黄连末,和水服一匕,以瘥为度。
治血利方第二十三

《病源论》云∶血利者,热毒折于血,入大肠故也,身热者死,身寒者生。

《千金方》治大热毒纯血利治不可瘥者方∶黄连六两,一味,以水七升,煮取二升半,夜露着日星下,旦空腹顿服之,卧息。

《极要方》疗血痢方∶真生犀角末(五两)阿胶(四两,炙)干姜(三两)艾叶(三两,熬)黄柏(四两)上五物,捣筛为散,服方寸匕。

《广利方》理血痢方∶酸石榴一颗,和皮捣取汁,蜜一大匙和,暖顿服之。
治赤白利方第二十四

《病源论》云∶其利而赤白者,是热乘于血,血渗(所禁反)肠内则赤也,冷气入肠间,搏肠脉薄《本草经》治赤白利方∶鲫鱼作脍食之。

《录验方》腊蜜丸,治赤白利方∶朴硝(二两)黄芩(一两)大黄(一两)代甘草(一两)黄连(一两)豉(一两)腊巴豆(上七物,丸如梧子,空腹服三丸,日三。

又云∶赤白痢赤多热方∶犀角(六分,屑多个)黄芩(六分)地榆(六分)黄连(八分)甘草(四分,炙)切,以水二大升,煎取八合,去滓,空腹分三服。

《千金方》治赤白利黄连汤方∶黄连(三两)甘草(一两)当归(二两)黄柏(三两)干姜(二两)石榴皮(三两)阿胶(七味,水七升,煮取二升,分二服。

《集验方》治暴下赤白方∶香豉(一升)薤白(一把)凡二物,以水三升,煮取二升,顿服之。

《传信方》疗赤白痢如鹅鸭肝方∶黄芩黄连(各八分)上二味,以水二升,煎取一升,分二分。

《广利方》治赤白痢白多冷痛方∶黄连(八分)浓朴(五分,炙)当归(四分)茯苓(六分)干姜(三分)切,以水一大升七合,煎取七合,去滓空腹分两服。忌猪肉生冷。

又云∶赤白痢赤多热方∶犀角(六分,屑)黄芩(六分)地榆(六分)黄连(八分)甘草(四分,炙)切,以水二大升,煎取八合,去滓,空腹分三服。

《医门方》疗赤白利腹中绞痛无问远近方∶黄连(八分)五色龙骨(十分)黄芩(六分)上,为散,以清饮空腹服方寸匕。

《崔禹食经》赤白利方∶、云雀、鹎等任意食之。

又方∶通草子食之。

《龙门方》治赤白痢方∶煮韭,空腹顿服一碗,不过再,验。

又方∶手熟乌豆服一大抄,不过二三。

又方∶取鼠尾草花,曝干末,服方寸匕,验。
治久赤白利方第二十五

《病源论》云∶赤白利是冷热不调,热乘于血,血渗肠间,与肠间津液相杂而下,甚者,肠《千金方》治下久赤白连年不止,及霍乱冷实不消温脾汤方∶大黄(四两)人参(二两)甘草(二两)干姜(二两)附子(一枚大者)五味,水六升,煮取二升半,分三服。

又云∶治积三十年常下神方∶赤松树皮去上苍皮一升为散,面粥和一升服之,日三,不过服一《录验方》治久赤白下利蒲黄二钱匕方∶蒲黄(二钱匕)干姜(二钱匕)二物,合以酒一升热服,不过四五服,断,良有验。

《极要方》三十年痢不止方∶黄连(五两)浓朴(二两)干姜(二两)阿胶(二两)石榴皮(二两)艾叶(三两)上,以水七升,煮取二升,分二服。

《随时方》治赤白痢连年不瘥,腹中如刀搅,或血行下,无问赤白谷痢,并主之∶白茯苓(四大两)黄连(四大两)黄柏(四大两)羚羊角屑(三两,熬)上,捣筛,蜜和丸如梧子,冷酒服之五十丸,渐至百丸,日再。
治白滞利方第二十六

《病源论》云∶肠虚而冷气客之,搏于肠间,津液凝滞成滞白,故为白滞利也。

《千金方》治白滞利方∶仓米三升,水六升,煮取三升,米烂绞取稠汁,服二升。

《范汪方》治赤白滞下昼夜数十行方∶乌梅〔(割取皮)三两,火熬令干〕黄连(三两)凡二物,冶合下筛,和蜜丸如梧子,晨服十丸,不知稍增,可至二三十丸,昼夜可六七服,《短剧方》治冷微赤白滞下不断,变成赤黑血汁如烂鱼肠,腹痛枯瘦不能饮食方∶黄连(四两)吴茱萸(三两)当归(三两)石榴壳(二两)凡四物,以水三升渍黄连一夕,《僧深方》治赤白滞下久不断,谷道疼痛不可忍方∶宜服温药,熬盐熨之。

又方∶炙枳实熨之。

《葛氏方》治赤白杂滞下方∶赤石脂(一升)乌梅(三十枚)干姜(三两)合粳米一升,水七升,煮取米熟,去滓,一服七合。

又方∶鼠尾草浓煮,煎如薄饴,服五合至一升,日三。赤下用赤花者,白下用白花者,佳。
治脓血利方第二十七

《病源论》云∶夫春阳气在表,人运动劳役,腠理则开,血气虚者,伤于风,至夏又热气乘《范汪方》治脓血利黄连丸方∶黄连(三两)黄芩(三两)龙骨(四两)黄柏(三两)升麻(三两)凡五物,捣下筛,蜜和丸如梧子,白饮服三十丸,日三。

又云∶治下利赤白脓血桃花汤方∶赤石脂(二两,捣筛)干姜(二两)附子(一两)凡三物,以水五升,煮得三升,服一升,日三(一方有粳米,无附子。)《广济方》疗白脓痢方∶甘草(六分,炙)浓朴(十二分,炙)干姜(八分)枳壳(八分,炙)茯苓(八分)切,以水五升,煮取一升六合,分温二服,忌生冷、油腻、小豆、粘食、海藻。
治水谷利方第二十八

《病源论》云∶由体虚腠理开,血气虚,春伤于风,邪气留在肌肉之内,后遇脾胃大肠虚弱《葛氏方》治水下积久不瘥、肠垢已出者方∶赤石脂桂干姜附子分等捣末,蜜丸如小豆,服三丸,日三。

又方∶石榴皮一枚,黄柏一两,干姜二两半,以水三升,煮取一升二合,纳胶顿服。

《医门方》疗水谷利、腹痛久不瘥方∶浓朴(炙)黄连(炙,三两)水三升,煮取一升,空腹服之。

《救急单验方》治水利方∶煮韭,空腹顿服一热碗,不过再,验。
治休息利方第二十九

《病源论》云∶冷热气调,其饮则静,而利亦休也,肠胃虚弱,易为冷热,其邪气或动或静《僧深方》治休息下方∶煮小豆一升,和腊三两,顿服,验。

又方∶煮韭,空腹一碗热服,不过再,验。

《范汪方》治息下、休下方∶酸石榴合皮捣取汁服之。

《陶景本草注》∶柿,火熏者食之。

又方∶榉树皮煮汁服之。

《葛氏方》若久下经时不愈者,名息下休下,治之方∶龙骨四两,捣如小豆,水五升,煮取又方∶黄连如鸭子大一枚,胶如掌大一枚,熟艾一把,水五升,煮二物,取二升,去滓,纳又方∶常煮忍冬饮之。
治泄利方第三十

《集验方》云∶黄帝曰∶人苦溏泄(达郎反,思引翼逝二反)下利者何?对曰∶春伤于风,夏论曰∶泄凡有五种,各不同。胃泄者,饮水不化,色黄,言所食饮之物皆完出不消也;脾泄肠故下《范汪方》治腹痛消谷止利服大豆方∶取大豆择貌好者服一合所,日四五服,一日中四五合难,《医门方》治泄利或白赤不止,肠滑洞泄困极欲死方∶酸石榴皮(三两)干地黄(二两)黄柏(三两)阿胶(二两,炙)水五升,煮取二升,去滓分温二服。

《千金方》健脾丸∶主虚劳羸瘦,体重,胃冷弱不消饮食、雷鸣腹胀、泄利不止方。

钟乳(二两)赤石脂(二两)好曲(二两)大麦孽(二两)当归(二两)黄连(二两)人斛(二两)桂心(二两)凡十五味,白蜜丸如梧子,酒服十五丸,日三,稍加至四十丸,弱者饮服。此方通治男女,《短剧方》云∶泄利食不消,不作肌肤,(灸脾俞,随年壮);泄注便脓血、五色重下,(灸云∶
治重下方第三十一

《葛氏方》云∶重下,此谓今赤白滞下也,今人下部疼重,故名重下,去脓血如鸡子白,日又方∶乌梅二十枚,打破,以水二升,煮取一升,顿服。

又方∶赤石脂一升,乌梅三十枚,干姜三两,合粳米一升,水七升煮,取米熟去滓,一服七合《龙门方》治重下方∶取鼠尾草花,曝干末,服三方寸匕,验。

又方∶末黄连和水服之。

《令李方》治下利重下方∶干姜(二两)蜀椒(二两)桂心(二两)凡三物,冶下筛,以如枣许从下部中纳半,亦治毒。

《范汪方》治重下方∶蓼满一虎口,以水三升,煮取一升,顿服,不过再,神良。

《录验方》治下腹中绞痛重下,下赤白当归散方∶当归(二两)黄连(二两)黄柏(二两)干姜(一两)凡四物,合下筛,以乌梅汁,服方寸匕,日三。若腹中绞痛,加当归;下赤加黄柏;重下增
治疳利方第三十二

《病源论》云∶甘是人有嗜甘味多,而动肠胃间诸虫,致令侵食腑脏,此犹是虫也,从肠《要急方》治赤白疳利方∶头发灰如鸡子大,水服立验。

《龙门方》治蚶利积年出无禁止者∶韭,两手握,细切,豉一升,酒三升,煮取一升,顿服
治蛊注利方第三十三

《病源论》云∶岁时寒暑不调,则有湿毒之气伤人,随渐至于脏腑。大肠虚者,毒瓦斯乘之,鸡肝《短剧方》治时岁蛊蛀毒下,诸汤煎不能治欲死者方∶干姜(二两)附子(二两,炮)黄连(二两)矾石(二两)凡四物,为散,酒服方寸匕,日三。亦可以饮服。

又方∶黄连一分,面二分,冶末,蜜丸,水服如梧子。

《葛氏方》若时岁蛊注毒下者方∶黄连、黄柏分等,捣,蓝汁丸如梧子,服六七丸至十四五又方∶秫米一升,烧成炭,水三升,和饮之。
治不伏水土利方第三十四

《病源论》云∶夫四方之气,温凉不同,随方嗜欲,因以成性,若移其旧土,多不习伏,必《僧深方》治诸下利,胡虏之人不习食谷下者方用∶白头公(二两)黄连(四两)秦皮(二两)黄柏(二两)凡四物,以水八升,煮取二升半,分三服。

《本草拾遗》云∶旧着鞋履下土,主人适他方,不伏水土,刮取末,和水服之,不伏水土与
治呕逆吐利方第三十五

《病源论》云∶呕逆吐利者,肠胃虚,邪气并之,脏腑之气自相克也。

《新录方》治利兼吐逆及呕者∶葱白、豉各一升,水五升,煮服一升六合,分二、三服。

又方∶干姜末方寸匕,饮日二。

《僧深方》治胸胁有热,胃中支满,呕吐下利方∶黄芩(二两)人参(一两)甘草(一两)桂心(一两)凡四物,水八升,煮取四升,分四服,日三夜一。
治利兼渴方第三十六

《病源论》云∶夫水谷之精化为血气津液以养脏腑,脏腑虚受风邪,邪入于肠胃,故利,利《僧深方》治少阴泄利不绝、口渴不下食、虚而兼烦方∶附子(一枚)干姜(半两)甘草(二分)葱白(十四枚)凡四物,以水三升,煮取一升,二服。先渴后呕者,心有停水,一方加犀角一两。

又方∶浓朴,炙,捣末,酒服方寸匕,日五六。
治利兼肿方第三十七

《病源论》云∶利兼肿者,是利久脾虚,水气在于肌肉之间所为也。

《新录方》治利兼肿者∶桑根白皮切一升,水四升,煮取一升,去滓,纳糖三合,和烊分二服。

又方∶大麻子三升,水一斗,研取白汁,煮赤小豆烂,啖豆饮汁,良。
治利后虚烦方第三十八

《病源论》云∶利后虚烦者,由腑脏尚虚而气内搏之所为也。

《短剧方》大乌梅汤治被下之以后,虚烦燥不得眠,剧者颠倒心中懊(奴道反)方∶大乌梅(十四枚,擘)好豉(七合)凡二物,以水四升,煮梅令得二升半,纳豉令四、五沸,得一升半,分二服。

《千金方》治下后烦,气暴上,香苏汤方∶香豉(五两)生苏(一把,冬用子三两)凡二物,水五升,煮取二升,顿服。

《僧深方》治大下后虚烦不得眠,剧者颠倒懊欲死方∶栀子(十四枚,擘)好豆豉(七合)凡二物,水四升,先煮栀子,令余二升半汁,乃纳豉,二三沸,去滓,服一升。一服安者,
治利后不能食方第三十九

《病源论》云∶利后不能食者,由脾胃虚弱、气逆胸间之所为也。

《千金方》治大下后腹中空竭,胸中虚满不得食方∶夕药(一两)甘草(一两)当归(二两)生姜(五两)桂心(三两)浓朴(二两)半夏(一两)凡七物,水八升,煮取三升,分三服。
治利后哕方第四十

《病源论》云∶下断之后,脾胃虚,气逆,遇冷折之,其气不通则令哕。

《范汪方》治大下之后,下止,呕哕、胸中满塞、水浆不下方∶橘皮(一两)人参(一两)香豉(一升,一方一两)桂心(二两)生姜(五两)半夏(三两)甘草(一两)凡七物,切,以水九升,煮豉取七升,去滓,纳诸药,微火上煮取二升半,分三服。
治利后逆满方第四十一

《病源论》云∶利后而心下逆满者,犹脏虚,心下有停饮,气逆乘之所为也。

《范汪方》治中寒下以后,心下逆满上冲胸中、起欲头眩方∶茯苓(四两)桂(三两)白术(二两)甘草(二两)凡四物,以水六升,煮取三升分三服。
治利后谷道痛方第四十二

《病源论》云∶利久肠虚,邪客于肛门,邪气与真气相搏,故令疼痛也。

《新录方》利谷道疼痛方∶炒盐熨下部。

又方∶烧蒜去皮,纳下部,良。

《范汪方》利谷道痛方∶炙枳实熨之。

《集验方》治赤白滞下久不断、谷道疼痛不可忍∶宜服温药,熬盐熨之。
卷第十二
治消渴方第一

《病源论》云∶消渴者,渴而不小便是也。由少服五石诸丸散,积经年岁,石热结于肾中,水而《千金方》云∶论曰∶凡积久兴酒,未有不成消渴。然则大寒凝海而酒不冻,明其酒性酷热由己。饮啖无度,咀(才与反)嚼酢酱,不择酸(苏官反,酢也)咸,积年长夜,酣兴不KT,遂使三焦(《短剧方》云∶说曰∶少时服五石诸丸散者,积经年岁,人转虚耗(呼到反)。石热结于肾中消渴之兴终不痿弱,液自出;亦作KT疸之病。凡如此等,宜服猪肾荠(脐檷二音)汤,制其肾中石势,将又云∶铅丹散治消渴止小便方∶铅丹〔(一名铅华,和名多尔,《本草》云∶朱雀精也)二分〕栝蒌(十分)泽泻〔(音昔五分〕凡八物,冶下筛,酒服方寸匕,日三。不知,稍增,年壮服半匕。得病一年服药一日愈,二又云∶治日饮一石许,小便不通,栝(古活反)楼丸方∶栝蒌(三分)铅丹(三分)葛根(三分)附子(一分,炮)凡四物,冶下筛,蜜(在本书)丸如梧子,饮服十丸,日三。

又云∶治消渴方∶取活螺三汁(斗),以江水一石养之,顿取冷汁饱饮之。经曰∶放去,更取新者渍之。

又云∶灸消渴法∶灸关元一处。又;侠两旁各二寸二处,各灸三十壮,五日一报,至百五十壮《千金方》治消渴方∶饮豉汁,任性多少。

又方∶浓煮竹根汁,饮之。

又方∶煮青粱米汁,饮之。

又云∶栝蒌粉治大渴秘方∶深掘大栝蒌,浓削皮至白处止,寸切之,水浸,一日一易水,经五日出,取捣,以绢袋碎之《葛氏方》治卒消渴小便多方∶多作竹沥饮恣口,数日愈。

又方∶破故屋瓦煮之,多饮汁。

又方∶石膏半斤,捣碎,以水一斗,煮取五升,稍服。

又方∶栝蒌根,薄切,炙,五两,水五升,煮取四升,饮之。

《录验方》治消渴日饮六七斗,小麦汤方∶小麦(一升)栝蒌根(切,一升)麦门冬(一升)上三物,以水三斗,煮取一斗半饮之。

《新录方》治消渴方∶臭泔恣意饮之,取瘥止。

又方∶捣生葛汁饮之。

《龙门方》疗消渴方∶生胡麻油一升,顿服之,立验。

又方∶烂煮葵汁,置冷露中,每渴即饮之。

《僧深方》治消渴唇干口燥枸杞汤方∶枸杞根(五升,锉皮)石膏〔(一名细石)一升〕小麦(三升,一方小豆)凡三物,切,以水加上没手,合煮,麦熟汤成,去滓,适寒温,饮之。

《极要方》疗渴、身体微肿方∶茅根三斤,捶(丁回反)破,以水二斗,煮取二升,一日服尽,可日服一剂。

《经心方》黍米汤治渴神方∶干黍米一升,以水三升,煮取一升,去滓,服一升,日再服,良。

《耆婆方》治人渴方∶栝蒌(十两)白粱米(五小升)上,以水一斗二升,煮取三升,去滓,分三服。

《范汪方》治消渴汤方∶麦门冬(一两)土瓜根(二两)竹叶(一把)凡三物,咀,水七升,煮取令得三升半,分再服,神有验。

《陶景本草注》治消渴方∶煮草汁及生汁服之。

又方∶捣冬瓜,绞服汁。

《苏敬本草注》治消渴方∶食大麦面良。

又,单食桑椹良。

又方∶粟米臭泔汁饮之,立瘥。

《孟诜食经》消渴方∶麻子一升,捣,水三升,煮三四沸,去滓,冷服半升,日三,五日即愈。

今按∶渴家可食物∶苏蜜煎(治消渴,补内);寒水石(一名白水石,《本草》云∶主止渴);石膏(一名细石,《本草》云∶主止消渴);大麦(《本草》云∶主消渴,和名不止牟支);青粱米(《本草》云∶主);猕猴桃(崔禹云∶主消渴,和名已久波);乌芋(《本草》云∶主消渴,和名久和乌);菰根(冬瓜(《本草陶注》云∶消渴,和名加毛宇利);葵菜(崔禹云∶主消渴,和名阿不比);菘菜蘩蒌(《七卷食经》∶主消渴,和名波久倍良);(《本草》云∶主消渴,和名奴奈波);骨蓬(《本草》云∶主消渴,和名加波保檷;)石(崔禹云∶治消渴,和名右毛);紫苔(崔禹云∶止渴海月(云∶主消寄居(崔禹草》云∶止渴家可忌物∶《千金方》云∶所慎者(物)有三,一则酒炙;二则房室;三则咸食及面。

《养生要集》云∶小麦合菰米食,复饮酒,令人消渴。

《短剧方》云∶忌食猪肉。
治渴利方第二

《病源论》云∶渴利者,随饮随少便是也。由少时(或无时字)服乳石,石热盛时,房室过度液,故随《葛氏方》治大渴利日饮数斛小便亦尔者方∶栝蒌黄连防己铅丹(一名铅华)分等捣末,以苦酒一合,水一合,和作浆,服方寸匕,日三。

《范汪方》治渴,日饮一斛,小便亦如之,栝蒌汤方∶栝蒌(二两)黄连(一升)甘草(二两)凡三物,水五升,煮取二升半,分三服。

《集验方》治渴日饮一斛者方∶入地三尺取桑根白皮,炙,令黄黑,细切,以水令相淹煮之,以味浓为度,热饮之。勿与盐
治内消方第三

《病源论》云∶内消病者,不渴而小便多是也。由少服五石,石热结于肾内也热之所作也。

《短剧方》云∶夫内消之为病,皆热中所作也。小便多于所饮,令人虚极短气。内消者,食物皆消作小便去而不渴也,治之枸杞汤∶枸杞枝叶(一斤)冬根(三两)栝蒌根(三两)石膏〔(一名细石)三两,一方无〕黄连(凡五物,切,以水一斗,煮取三升,一服五合,日三。

又云∶治小便多,昼夜数十起方∶小豆生藿(一把)凡一物,捣绞平,取汁,频饮三升便愈。亦治小儿利。

《令李方》治小便利多秦胶散方∶秦胶(一分)陈芥子(二分)凡二物,冶下筛,酒服方寸匕,日三。
治诸淋方第四

《病源论》云∶诸淋者,由肾虚而膀(音旁)胱(音光)热故也。其状小便出少起数,小腹弦急《葛氏方》治卒患淋方∶灸足大指前节上十壮,良。

又方∶灸两足外踝中央,追年壮,有石即下。

又方∶但服草汁一升,不过三升。亦治石淋。

又方∶豉一升,水三升,渍少时,以盐一合纳中,顿服。(今按∶《经心方》∶豉半升,水《极要》云∶疗淋方∶上,煮石燕汁饮之良验,以水煮之。

《范汪方》治淋滑石散方∶葵子(一升)滑石(一两)通草(二两)凡三物,冶筛,酒服方寸匕,日三。

又云∶治淋栝蒌散方∶石苇(二分)通草(一分)栝蒌(二分)葵子(四分)凡四物,冶筛,先食以麦粥,服方寸匕,日三,无不愈。

又方∶常以冬葵根作饮,良。

《短剧方》治淋病不得小便,阴上绞痛方∶灸足太冲五十壮。(在足大指本节后二寸。)又方∶灸悬泉,一名中封,十四壮。(中封在足内踝前一寸。)《广利方》理诸淋小便卒不通方∶麻根(二七枚,切)上,以水二大升,煎取八九合,去滓,分温三服。

《集验方》卒得淋方∶取牛耳中毛,烧服半钱匕,立愈。

又方∶以比轮钱三百文,以水一斗,煮得三升,饮之。千金秘不传。

《录验方》治淋瞿麦散方∶瞿麦(四两)石苇(四两,去毛)滑石(四两,碎)车前子(四两)葵子(四两)凡五物,捣筛,冷水服方寸匕,日三,增至五匕,慎酒、面。

《新录方》治淋方∶马苋茎叶捣汁一升,二三服。
治石淋方第五

《病源论》云∶石淋者,淋而出石也。肾主水,水结则化为石,故肾容沙石。肾虚为热所乘甚者《短剧方》治石淋神方∶车前子二升,以绢囊盛,以水八升煮取三升,尽服之,日移一丈,石子当出。宿不食饮之。

又方∶生茎叶合捣取汁,服一升,日三。

又∶病石淋,脐下三十六种病不得小便方∶灸关元三十壮(《千金方》同之)。又方∶灸大敦三十壮(在足大指端,去爪甲如韭叶。)又方∶《录验方》石淋方∶取车前草,煮,多饮汁。

又方∶石苇(三分)滑石(三分)凡二物,下筛,合以米汁若蜜,服刀圭匕,日三,已效。

《葛氏方》石淋者方∶取燕矢,末,以冷水服钱五匕,清旦服,至食时(辰也)当尿石。

又方∶取故甑蔽烧,三指撮,服即通。

又方∶石首鱼头中石一升,贝齿一升,合捣,细筛,以苦酒和,分为三分,宿不食,明旦服呼《千金方》石淋方∶浮石取满手,下筛,水三升,酢一升,煮取二升,澄清服一升,三服石出。

《本草拾遗》云∶有以病为药者,淋石主石淋,水磨服之。当碎石,随尿出也。人患石淋或《新录方》治石淋方∶生葛根汁,服五六合。

又方∶葱白三升,水六升,煮取二升五合,三服。

《陶景本草注》煮麻根饮之。

《苏敬本草注》捣乌芋根汁一升,服之。

《崔禹食经》煮葵子服汁。
治气淋方第六

《病源论》云∶气淋者,肾虚膀胱热气胀所为也。其状膀胱少腹皆满,尿涩,常有余沥是也《千金方》疗气淋方∶灸关元五十壮。又灸夹玉泉相去一寸半三十壮。

又方∶水三升,煮船底苔如鸭甲大,取二升一服。

又方∶捣葵子末,汤服方寸匕。

《广利方》理气淋脐下切痛方∶以盐和少醋填脐中,盐上灸二七壮,立瘥。
治劳淋方第七

《病源论》云∶劳淋者,谓劳伤肾气而生热成淋也。肾气通于阴,其状尿留茎内,数起不出《千金方》疗百淋寒淋热淋劳淋,小便涩,胞中满,腹急痛方∶栝蒌(三两)滑石(二两)石苇(二两)三味,大麦粥饮服方寸匕,日三。
治膏淋方第八

《病源论》云∶膏淋者,淋而有肥,状似膏,故谓之膏淋。亦曰肉淋。此肾虚不能制其肥液《千金方》云∶膏淋之为尿,似膏自出,疗之一如气淋也。

捣草汁二升,酢二合,和,空腹服之。
治血淋方第九

《病源论》云∶血淋者,是热淋之甚者,则尿血谓之血淋。心主血,血行身,通遍经络,修《广济方》治血淋方∶车前叶,捣取汁,半升,和蜜一匙,搅令消,顿服之,立瘥。(《广济方》同之。)《龙门方》治血淋方∶取刺蓟根,勿见风大一握,净洗,捣取汁半升,服之。极者不过三,良。

《千金方》疗血淋方∶石苇当归蒲黄夕药四味,分等,酒服一钱匕,日二。

又方∶水五升,煮麻根十枚,取二升,顿服。

又方∶水四升,煮大豆叶一把,取二升,顿服。

又方∶灸丹田,随年壮。在脐下二寸。今按∶《明堂》云∶石门一名丹田。
治热淋方第十

《病源论》云∶热淋者,三焦有热,气搏于肾、流入于胞而成淋也。其状小便赤涩。亦有宿《葛氏方》热淋方∶取白茅根四斤,锉(子卧反,破也、折伤也)之,水一斗五升,煮令得五升汁,服日三。

(《又方∶末滑石屑,水服一二合。

《录验方》治热淋方∶芦心,切,三升,水五升,煮取二升,三服。

《千金方》治热淋方∶常煮冬葵根作饮服之。

又方∶灸亦与气淋同之。

又方∶葵根一升(冬用子)大枣(二七枚,去核)二味,水三升煮取一升二合,分二服,热加黄芩一两,出难加滑石二两。

《龙门方》疗热淋方∶服冷水三升,行一里即下瘥。

又方∶灸两足外踝中央,随年壮,有石下。

《广济方》疗热淋小便涩痛方∶车前(切,一升)通草(三两)葵根(切,一升)芒硝(六分)汤成,切,以水七升,煮取二升,去滓,纳芒硝,分温三服,忌药食。
治寒淋方第十一

《病源论》云∶寒淋者其病状,先寒战然后尿是也。由肾气虚弱,下焦受于冷气,入胞(补《新录方》∶治寒淋少腹下冷,手足亦冷方∶葵子(一升)曲末(一升)水六升,煮取三升半,三服。

又方∶葵子一升,小麦一升,水六升,煮取三升,三服。
治大小便不通方第十二

《病源论》云∶关格,大小便不通也。大便不通,谓之内关;小便不通,谓之外格;二便俱《千金方》治关格大小便不通芒硝汤方∶芒硝〔(一名石脾)五两〕大黄(八两)夕药(四两)杏仁(四两)麻仁(三两)乌栖树根凡六味,以水七升,煮取三升,分三服。

又方∶灸脐下一寸三分壮。

又方∶灸横纹百壮。

又方∶甑带煮取汁,和蒲黄方寸匕,日三服。

又方∶葵子一升,榆皮切一升,水五升,煮取二升,分三服。

《葛氏方》∶治卒关格大小便并不通支满欲死,二三日则杀人方∶取盐,以苦酒和涂脐中,干复易之。

又方∶自取手十指爪甲烧末,以酒浆服。

又方∶葵子二升,水四升,煮取一升,顿服之。纳猪膏如鸡子一丸亦佳。

《集验方》治久不得大小便方∶猪脂如鸡子,着一杯酒中,煮之令沸,顿服。

又方∶煮葵根汁,服,弥佳。

《范汪方》治大小便不出方∶豆酱纳下部中,令人吹则通小便,以盐纳茎中则利,良。

《经心方》滑石散治大小便不通方∶滑石(二两)榆皮(一两)葵子(一两)凡三物,作散,浓煮麻子一升,半取一升,以两匕和服,不过二,大小便通。(《千金方》同之。)《广利方》∶理气壅、关格不通、小便淋结、脐下妨闷兼痛方∶冬葵子二大合,生茅根一握,以水一大升半,煎取六大合,去滓,分温二服。

又云∶关隔不通、胞胀妨闷、大小便不通方∶冬葵子(三大合,绵裹,碎)滑石(十二分,碎)芒硝(十分,汤成下)切,以水二大升,煎取八大合,去滓,空腹分温再服,服别如人行四五里。忌肉、面。
治大便不通方第十三

《病源论》云∶大便不通者,犹(由)三焦五脏不和,冷热之气不调,热气偏入肠胃,津液竭燥,故令糟粕(并格反,糟酒滓也)痞结壅塞不通也。

《葛氏方》治大便不通方∶研麻子,以米杂为粥食之,亦可直煮麻为饮服之。

又方∶剥乌梅皮,以渍酱豆中,导下部。

《范汪方》治大便不通方∶用豆酱纳下部中则通。

又方∶酱中瓜,切,令如指长三寸,纳大孔中。

《秦承祖方》不得大便数日方∶作热汤着盆中,人居其中,汤未冷则瘥。

又方∶灸下部后五分三十壮瘥,大良。

《千金方》治大便不通方∶桑根(一把,去赤皮)榆根(一把,去赤皮)水三升,煮服一升半,分二服。

又方∶恒煮麻子取汁饮之。

《龙门方》疗大便不通方∶取胶,广二寸,长四寸,葱白一握,以水三升和煮,消尽去滓,一服验。

又方∶熬葵子半升,捣末,以水一升煮服之。

《医门方》∶疗下部闭塞大行(谓大便也)不出方∶取乌梅四十枚拍碎,汤中浸少时,去核,捣之如浑丸大一枚纳下部中,立通。

《短剧方》大便闭塞气结心满方∶灸石关百壮。(今按∶《明堂》∶在幽门下二寸,幽门在巨阙旁半寸。)又方∶灸足大都随年壮。

《经心方》夕药汤治胀满大行不通方∶夕药(六分)芒硝〔(一名石脾)六分〕黄芩(五分)大黄(八分)杏仁(八分)凡五味,丸如梧子,饮服十五丸,日三。

《华佗方》云∶有病日食二斗米,至二百日不大便,亦无所病苦,何以尔名?为何等病也?此葛根(五斤)猪肪(三斤)凡二物,葛根细锉洗之,以水三斗并煎之得一斗半,去滓,复煎其汁得七升已,取猪肪,切禁。服此药,开六腑,当下。
治大便难方第十四

《病源论》云∶大便难者,由五脏不调,阴阳偏有虚实,三焦不和,则冷热并结故也。

又云∶渴利之家,大便亦难,为津液枯竭,致令肠胃干燥。

《葛氏方》云∶脾胃不和,常患大便坚强难者∶大黄(三两)夕药(三两)浓朴(三两)枳实(六斤)麻子仁(五合)捣筛,蜜丸如梧子,服十丸,日三,稍增,以通利为度,可恒将之。

《千金方》治大便难方∶单用豉、清酱、清羊酪、土瓜根汁,并单灌之,立出。

又方∶酱清渍乌梅,灌下部中。

又方∶麻油二升,纳葱白三寸,煮令黑,去滓,待冷顿服。

又方∶灸承筋三壮。

又方∶夹玉泉相去各二寸灸之。

《新录方》大便干骨立者方∶灸胃脘穴千炷。

又方∶生地黄切三升,韭切三升,以水一斗,煮取二升五合,分三服,相去十里。

又方∶单服马苋汁一升,瘥止。

又方∶捣蒜为泥,酒服如枣,日三。

又方∶烧鱼为灰,酒服方寸匕,日二三。

又方∶服甑带汁五六合,日二。

《承祖方》治大便难腹热连日欲死方∶甘遂芫花黄芩凡三物,分等,捣,蜜丸如小豆,服五丸,不通,更服三丸。

《华佗方》治大便坚,数清不能得出方∶皂荚末,下筛,以猪脂和合,苇管长一寸,以指排纳谷道中齐指一节,须臾则去。

又云∶二车丸主临饭腹痛不能食,复又大便难方∶大黄(十三两)柴胡(四两)细辛(二两)茯苓(一分)半夏(一两)凡五物,冶筛,丸以蜜,饮服如梧子五丸,日再。

《集验方》治大便难,腹热连日欲死方∶白蜜三升,于微火上煎之,使如强以投冷水中,须臾当凝出丸,丸如手指大,长六寸七寸又云∶不得大便十日或一月烦满欲死方∶葵子二升,水以四升,煮取一升,去滓,顿服之。
治大便失禁方第十五

《病源论》云∶大便失禁者,由大肠与肛门虚冷滑故也。肛门,大肠之侯也,俱主行糟粕(《千金方》治老人小儿大便失禁∶灸大指奇间三壮,两脚。

又方∶灸两脚大指去甲一寸三壮。
治大便下血方第十六

《病源论》云∶此由五脏伤损所为。脏气既伤,则风邪易入,热气在内,亦大便下血。

其前《短剧方》云∶诸下血者,先见血后见便,此为远血,宜服黄土汤。若先见便后见血,此是赤小豆散方∶赤小豆(三升,熬)当归(三两)凡二物,冶筛,服方寸匕,日三。(今按∶《令李方》∶黄连二两酒服。)黄土汤方∶灶中黄土(半升,绵裹)甘草(三两,炙)干姜(二两)黄芩(三两)阿胶(三两)干地黄凡六物,以水一斗,煮取三升,分三服。(今按∶《集验方》有芎、熟艾,无黄芩、干地《葛氏方》治卒下血方∶豉一升,以水三升,渍,煮三沸,去滓,顿服汁一升,日三。冬天每服辄温。

又方∶豉二升,以酒六升,合煮得三升,服一升,日三。

又方∶煮香极令浓,去滓,服一升,日三。

又方∶乱发如鸡子大,烧末,水服之,不过三。

又方∶三指撮盐,烧,向东服之。

又方∶灸两足大指回毛中,追年壮即愈。

《僧深方》治卒下血蒲黄散方∶甘草(一分)干姜(一分)蒲黄(一分)凡三物,下筛,酒服方寸匕,日三。

又方∶治卒注下并下血,一日一夜数十行方∶灸脐中及脐下一寸各五十壮。(今按∶《葛氏方》∶以钱掩脐上,灸钱下际五十壮。)《医门方》治卒下血或因吃热物而发方∶生葛根汁生地黄汁生藕根汁各饮一升,日二,瘥。

又方∶小豆末和水服方寸匕,日二服,立瘥。

《范汪方》治下血方∶干地黄(五两)胶(三两,炙)凡二物,治筛,分三服。(《葛氏方》同之。)又方∶干地黄下筛,以酒服方寸匕,日三。

又云∶治大便血诸血衄血方∶乌贼鱼骨(五分)桑耳(一分)凡二物,冶筛,酒服方寸匕,日三。

《千金方》下血日夜七八十行方∶黄连(四两)黄柏(四两)二味,淳酢五升,或煮取一升半,分再服。

《枕中方》治人下血方∶取鸡苏绞取汁,多少任意服之,愈,(今按∶《集验方》∶治吐血,下血并妇人漏下。)《新录方》治卒下血兼血痔方∶栀子及皮一升,以水三升,煮取一升三合,分二服。

又方∶桃奴,树上死桃子也。取一升,以水三升,煮取一升三合,分二服。

又方∶取败船茹(音如)二升,以水三升,煮取一升二合,分二服。

又方∶荆叶切三升,以酒五升,煮取一升六合,分二服。

又方∶赤小豆三升,以水五升,煮取一升六合汁,渴饮汁,饥啖豆。

又方∶以水三升煮葱白一升半,取一升二合汁,分三服。

又云∶食热物下血方∶捣生葛根,取七八合汁,饮之。

又方∶捣生地黄取汁,饮七八合。

又方∶捣生藕取汁,饮七八合。

又方∶生荷根汁,饮六七合。

又方∶酒三升,煮大枣二十一枚,取汁分二三服。

又方∶一(以)水服石榴皮末方寸匕,日二。

又方∶捣蓟,无问大小猫虎羊等,取汁饮之,煎取,若冬月无生者,掘取根或干者切二升,
治小便不通方第十七

《病源论》云∶小便不通者,由膀胱与肾俱有热故也。肾主水,膀胱为津液之腑,此二经为便不《葛氏方》治小便不通方∶熬盐令热,纳囊中,以熨少腹上。

又方∶以盐满脐,灸上三壮。(以上《短剧方》同之。)又方∶末滑石,水服方寸匕。

又方∶以衣中白鱼虫纳小孔中。

《千金方》治小便不通方∶水四升,洗甑带,取汁,煮葵子,取二升半,分三服。

又方∶葵子、车前子,水五升,煮取二升半。三服。

又方∶鲤鱼齿灰末,酒服方寸匕,日三。

又方∶鱼头石末,水服方寸匕,日三。

《新录方》治小便不通方∶水渍石,迭熨少腹下出,或烧石热熨少腹,以出为度。

又方∶车前子一升,以水三升,煮取一升二合,再服。

《短剧方》小便不通及关格方∶取生土瓜根,捣取汁,以少水解之于筒中,吹纳下部即通。秘方。

又云∶治小便闭方∶豉半升,水四升,煮一沸,去滓,一服立愈通。

《医门方》疗小便不通方∶取乱发如拳大,烧作灰、末、筛,酒服方寸匕,立愈。

又云∶疗小便不通、小腹满闷,不急疗杀人方∶葱白切一合,盐少许,捣绞取汁,灌三五豆粒许入水道中,水便出。极效。

《集验方》∶治淋、小便不利、阴痛,石苇散方∶石苇(二两)瞿麦(一两)滑石(五两)车前子(三两)葵子(二两)凡五物,下筛,先食服方寸匕,日三。

《令李方》治淋,胞满不得小便滑石散方∶滑石(一两)通草(半两)石苇(一两)凡三物,冶筛,酒服方寸匕,日三。

《范汪方》治小便不通方∶取陈葵子一升,淳酒三升,煮之服尽。
治小便难方第十八

《病源论》云∶小便难者,此亦是肾与膀胱热故也。

《录验方》治小便难淋沥方∶通草(二两)茯苓(二两)葶苈子(二两,熬)凡三物,下筛,以水服方寸匕,日三。

《新录单方》治小便不出、腹满气急者方∶灸关元穴,在脐下三寸,依年壮。

又方∶车前草切三升,以水五升,煮取二升,分二服,日一。

又方∶甑带一枚,以水五升,煮取一升六合,再服。

又方∶大麻子三升,以水五升,煮麻子腹破,分二服。

又方∶煮滑石取汁,饮之立下。

又方∶葵子二升,以水四升,煮取一升六合,分三服。

又方∶茅根切二升,煮服依前,兼去渴,最妙。

《医门方》疗小便难方∶以少盐纳茎孔中即通。
治小便数方第十九

《病源论》云∶小便数者,膀胱与肾俱虚,而有客热乘之故也。

《范汪方》治小便一日一夜数十行方∶菖蒲黄连二物,分等,冶筛,酒服方寸匕。

又方∶石膏半斤,咀,以水一斗,煮取五升稍服。(以上《葛氏方》同之。)《千金方》治小便利复非淋方∶榆白皮二斤,水一斗,煮取五升,服三合,日三。

又方∶三年重鹊巢烧末,服之。

《葛氏方》∶治小便卒太数复非淋,一日数十过,令人疲瘦方∶灸两足下第二指本节第一理七壮。

又方∶不中水猪膏如鸡子者一枚炙,承下取肥汁尽服之,不过三。此二方并治遗尿也。

又方∶鸡肠草一把,熟捣,酒一升渍一时,绞去滓,分再服。

《医门方》疗小便数日夜出无节度方∶小豆苗叶,捣绞取汁,饮一二升,立止,极效。
治小便不禁方第二十

《病源论》云∶小便不禁者,肾气虚,下焦受冷也。肾主水,其气下通于阴,肾虚下焦冷,《新录方》治小便不禁方∶柏树白皮切三升,以水三升,煮取一升二合,分再服,相去十里。

又方∶故甑带,以水三升,煮取一升二合,分二服。

又方∶露蜂房灰,酒服方寸匕,日二。

又方∶石榴皮子灰,酒服方寸匕。

又方∶榆白皮切二升,水四升,煮取一升六合,二服。

《令李方》∶小便不禁数行方∶当归(二两)甘草(五分)凡二物,冶下筛,酒服半钱匕,日三,一月瘥。夏月纳茯苓,阴肿痛纳黄分等。

《极要方》疗小便不禁或如血色方∶麦门冬(八两,去心)蒺藜子(二两)甘草(一两,炙)干姜(四两)桂心(二两)干地黄上,水一斗,煮取二升半,分三服。
治小便黄赤白黑方第二十一

《僧深方》治膀胱急热、小便黄赤滑石汤方∶滑石(八两,碎)子芩〔(一名黄芩)三两〕车前子(一升)葵子(一升)榆皮(四两)凡五物,以水七升,煮取三升,分三服。

《龙门方》疗人小便白稠方∶取蜂房烧作灰,和水一服一匕,瘥。

《范汪方》治小便利多而或白精从尿后出方∶栝蒌(三分)滑石(二分)石苇(一分)三物,为散,麦粥服方寸匕,日三。

又云∶治小便白浊而多方∶桑茸(三分)甘草(五分)二物,为散,以酢浆服方寸匕,日三。

又云∶治虚羸,少(小)便青黄白黑,白如米汁方∶白善〔(一名白玉)六分〕龙骨(五分)牡蛎(二分)小豆(三两,熬)土瓜根(二分)凡五物,冶筛,以酒服一方寸匕,日三。
治小便血方第二十二

《病源论》云∶心主于血,与小肠合,若心家有热,结于小肠,故小便血也。

《养生方》云∶人食甜酪,勿食大酢,变为尿血也。

《新录方》治尿血方∶车前草,捣绞取汁,服五合,旦空腹服之。

又方∶棘刺二升,水四升,煮取二升,分三服。

《范汪方》治小便血方∶乌芋根五升,捣取汁,服一升。

《千金方》治尿血方∶豉二升,酒四升,煮取一升,顿服。

又方∶生大麻根十枚,水五升,煮取二升,分三服。

又方∶龙骨细末,温酒服方寸匕。

又方∶熬盐热熨少腹,并主淋。大良。

又方∶刮滑石屑,水和涂少腹绕阴际,佳。

又方∶灸大敦穴,随年壮。

又云∶治房损伤中尿血方∶牡蛎车前子桂心黄芩四味,下筛,饮服方寸匕,日三,加至二匕。

又云∶治虚劳尿血淋方∶葵子一升,水三升,煮取一升半,日三。

《葛氏方》治小便血方∶茅根一把,切,煮,去滓,数饮之。

又方∶捣葱白取汁服一升。

《短剧方》治生地黄汤治小便血方∶生地黄(半斤)柏叶(一把)黄芩(二两)胶(二两)甘草(二两)凡五物,以水七升,煮取三升,绞去滓,纳胶令烊,取二升半,分三服。

《博济安众方》疗小便出血方∶地黄汁一升,姜汁一合,相和顿服。未瘥,再服。
治遗尿方第二十三

《病源论》云∶遗尿者,此由膀胱虚冷,不能约(求俱反,救也)于水故也。

防己防风(各三分)葵子(二分)冶,末,服方寸匕。一方防己三升。

又方∶矾石、牡蛎分等,冶合,以黍粥服方寸匕,日三。

《令李方》治遗尿夕药散方∶白薇(一两)夕药(一两)凡二物,冶合,下筛,酒若水服方寸匕,日三。

《龙门方》治遗尿不禁方∶取燕巢中蓐(乳属反,兹也,席也)烧灰。服一钱匕,日三,七日瘥。

今按∶《本草拾遗》云∶水进方寸匕,亦主哕。

《录验方》治遗尿龙骨散方∶桑茸(三两)矾石(二两)牡蛎(二两)龙骨(三两)凡四物,合冶,下筛,服方寸匕,日三。

《千金方》治遗尿、小便涩方∶木防己(二两)葵子(二两)防风(二两)三味,水五升,煮取二升半,分三服,散亦服佳。

《短剧方》治遗尿灸穴∶灸遗道,在侠玉泉五分,随年壮。

又方∶灸阳陵泉、阴陵泉,随年壮。在膝下一寸。
治尿床方第二十四

《病源论》云∶人有于眠睡不觉尿出者,是其禀质阴气偏盛,阳气偏虚者,则膀胱肾气俱冷,不能温制于水,则小便偏多,或不禁而遗失。

《新录方》治尿床方∶大麻根皮切三升,以水五升,煮取一升八合,去滓,分二服,小儿减之。

又方∶大豆叶三升,水五升,煮取二升,分三服。
卷第十三
治虚劳五劳七伤方第一

《病源论》云∶虚劳者,五劳六极七伤是也。五劳者,一曰志劳,二曰思劳,三曰心劳,四极六精肾(是按∶虚劳阴痿《千金方》云∶三人九子丸主五劳七伤补益方∶酸枣仁柏子仁薏苡仁枸杞仁蛇床子五味子菟丝子菊子子荆子地肤十五味,加苁蓉三两,余各二两,酒服如梧子二十丸,日二。

《范汪方》开心薯蓣肾气丸,治丈夫五劳七伤,髓极不耐寒,眠即胪胀,心满雷鸣,不欲饮不发去冷,肉苁蓉(一两,一方无)山茱萸(一两,一方无)干地黄(六分,一方代干姜)远志(六分)分)菟丝子(六分)凡十二物,捣下筛,蜜丸如梧子,服十丸至二十丸,日二夜一。若烦心即停减之。只服十丸十五夜独寝不寒,止服一剂。

又云∶六生散治五劳七伤五缓六急,治寒热胀满,大腹中风垂曳,消逐血,补诸不足,令人肥白方∶生地黄根(二斤)生姜(一斤)生菖蒲根(一斤)生枸杞根(一斤)生乌头(一斤)生章凡六物,合七斤,熟洗之,停令燥KT切之,美酒二斗都合渍三四日,出爆之,暮辄还着酒中《煎药方》云∶酥蜜煎治诸渴及内补方∶酥(一升)蜜(一升)地黄(煎,一升)甘葛(煎,一升)大枣(百枚)茯苓人参薯蓣上八物,先蜜酥入合搅,烊后甘葛煎入,烊枣膏,以绝绞入,然后茯苓、人参、薯蓣等散入今按∶酥蜜煎方有数首,或药种多少不同,或分两升合各异,仍取当今名医增损之法以备俗酥(小一升)蜜(小一升,煎去滓沫,无用甘葛煎)甘葛(煎,大三升)地黄(煎,大二合五十枚,取肉筛)凡十物,先以酥入生姜煎煎之,令相得,次入蜜,次以甘葛煎和大枣,炼胡麻绞去滓入,次火少《杂酒方》枸杞石决明酒治除腰脚疾疝癣,诸风痹恶血,去目白肤翳赤膜痛眨眨泪出瞽盲轻石决明(干者一大斤,洗,炙)枸杞根白皮(小一斤)上二物,细切,盛绢袋,以清酒四斗五升渍之。春五日、夏三日、秋七日、冬十日去滓,始《太清经》五茄酒治五劳七伤,心痛血气乏弱,男子阴痿不起,囊下恒湿,小便余沥而阴痒耐老用雄一升物,渴,四肢拘挛,膝痛,不可屈伸,伤中少气,阴消脑疼,忧患惊邪恐悸,心下结痛,烦满,咳逆,口焦舌干,好唾,膈中痰水,水肿,阴下痒湿,小便余沥,脚中酸痛,不欲践地。身中不足,四肢沉重,时行呕哕,折跌绝筋,积聚,五劳七伤,目暗清盲,热赤痛,补虚羸,除寒热,益气力,长肌肉,止腰痛,充五脏,利小便,益精气,止泻精,久服耳目聪明,阴气长强,坚筋骨,填脑髓,养神安魂,令人身轻。能跳越峰谷,不老而长生也。

枸杞根切,大一石,以水大三斛,入煮取五斗汁,去滓,加薯蓣、藕根各二大升,煮取一大斗,黄煎如糖《录验方》生地黄煎补虚除热,将和取利也。散石痈疽疮疖痔热皆宜之方∶生地黄根随多少舂绞取汁,复重舂绞取之,尽其汁,乃除去滓也。以新布重绞其汁,去滓碎浊,令清净,置KT中置釜汤上煮KT,勿塞全边,令汤气得泄不沓也。煎地黄汁竭减半许,后煎下,更以新布绞去粗碎结浊滓秽,去复煎竭之,令如饴糖成煎。北方地黄肥,味浓,作煎甘美;东南地黄坚细,味薄,作煎咸苦不美,然为治同耳。能多作则美好也,少作则易竭,
治虚劳羸瘦方第二

《病源论》云∶夫血气者,所以荣养其身也。虚劳之人,精髓萎竭,血气虚弱,不能充盛肌《录验方》枸杞丸治劳伤虚损方∶枸杞子(三升)干地黄(切,一升)天门冬(切,一升)凡三物,细捣曝令干,以绢罗之,蜜和作丸,大如弹丸,一服一丸,日二。

《范汪方》云∶补养汤主虚劳羸瘦,食已少气方∶甘草(一两,炙)术(四两)牡蛎(二两)大枣(二十枚)阿胶(三两)麦门冬(四两,去凡六物,咀,水八升煮取二升,尽服,禁生冷。

《千金方》甘草汤主虚羸气欲绝方∶甘草(二两)生姜(二两)人参(一两)五味(二两)吴茱萸(一升)五味,水四升煮茱萸令如蟹目沸,去滓,纳药煮取一升六合,分再服。

又云∶小鹿骨煎主治一切虚羸皆悉服之方∶鹿骨(一贝,碎)枸杞根(切,一升)二味,各以水一斗别器煎,各取汁五升,去滓,澄乃合一器共煎,取五升,二日服尽。

好将《本草拾遗》云∶鲫鱼肉主虚羸,熟煮食之。

《耆婆方》治人瘦,令人肥健、肥白、能行阴阳,并去风冷、虚瘦无力,神验方∶取枫木经五年以上树皮,去上黑皮,取中白皮五斗,细锉微曝令水气去,以清美酒于白瓦器经数《极要方》云∶枸杞子酒疗虚羸黄瘦,不能食,服之不过两剂,必得肥充,无禁断方∶枸杞子(五大升,干者碎)生地黄(三大升,切)大麻子(五大升,碎)上,于甑中蒸麻子使熟,下着桉上摊去热气,冷暖如人肌,纳地黄、枸杞子,熟使三物相微有《明堂经》云∶脾输二穴灸三壮,主腹中气胀引脊痛,食饮多,身羸瘦,名曰食晦。注云∶
治虚劳梦泄精方第三

《病源论》云∶肾虚为邪所乘,邪客于阴,则梦交接,令肾虚弱不能制于精,故因梦感动而《短剧方》别离散治男女风邪,男梦见女,女梦见男,悲愁忧恚,怒喜无常,或半年或数月杨上寄生(三两)术(三两)桂心(一两,一方三两)茵芋(一两)天雄(一两,炮)KT根姜(一两)凡十物,合捣下筛,酒服半方寸匕,日三。合药勿令妇人、鸡、犬见之,又无令见病者,病又云∶龙骨散治男子失精百术不治方∶龙骨(大如指赤理锦文者)薰草(二两)桂肉(二两)干姜(二两)凡四物,下筛酒服方寸匕,日三,神良。

又云∶韭子汤治失精方∶韭子(一升)龙骨(三两)赤石脂(三两)凡三物,以水七升煮取三升半,分三服。

《集验方》治梦泄精方∶韭子(一升,熬)一物,捣筛,酒服一方寸匕,日再,神效。

《医门方》疗梦交接泄精,暂睡即出,身体枯燥骨立者方∶白龙骨(八分,研)韭子(一升半)为散,酒服方寸匕,日二。加至一匕半,日二。

《极要方》云∶小便失精及夜梦泄精方∶韭子(一升,熬)麦门冬(二两)菟丝子(二合)车前子(二合)芎(二两)龙骨(二两)以上,以水八升煮取二升,为四服,日三。

《玉房秘诀》云∶治数梦交失精方∶龙骨牡蛎甘草桂(各三两)大枣(一枚)水六升,煮取二升半,分三服。

《僧深方》云∶禁精汤主失精羸瘦,酸消少气,视不明,恶闻人声方∶韭子(二升)生粳米(一升)二物,合于器中熬之,米黄黑及热,急以淳佳酒一斗投之,绞取七升,服一升,日三,二剂便愈。

《新录方》治失精方∶取韭根捣取汁服五合,日二。

又方∶石榴皮捣为散,饮服方寸匕,日二。

又方∶韭子一升,桑螵蛸十四枚,水五升,煮取二升,三服,亦为散酒服。

《葛氏方》云∶治男女梦与人交接,精便泄出,此内虚积滞,邪气感发,治之方∶韭子(一升)粳米(一升)水四升,煮取升半,一服。

又方∶龙骨二分,术四分,桂二分,天雄一分,捣末酒服五分匕,日三。

又方∶两足内踝上一夫,脉上名三阴交,灸二十一壮,梦即断。

又方∶合手掌并大拇指令两爪相近,以一炷灸两际角令半入爪上三壮。

又云∶治男女精平常自出,或闻见所好感动便已发,此肾气乏少不能禁制方∶巴戟天石斛黄分等,捣,酒服方寸匕,日三。

又方∶鹿茸一两,桂一尺,韭子一升,附子一枚,泽泻三两,捣末服五分匕,日三。

又方∶牡丹炙令变色,捣末服方寸,日三。

又方∶雄鸡肝、鲤鱼胆,令涂阴头。
治虚劳尿精方第四

《病源论》云∶肾藏精其气通于阴。劳伤肾虚,不能藏于精,故因小便而精液出。

《千金方》治虚劳尿精方∶韭子(一升)稻米(二升)水一斗七升,煮如粥,取汁六升,分三服。

《葛氏方》治男子溺精如米汁及小便前后去精如鼻涕,或尿有余沥污衣,此皆内伤令人虚绝栝蒌(二分)滑石(二分)石苇(一分)捣末,以麦粥服方寸匕。

又方∶甘草、赤石脂分等捣末,服方寸匕,日三。

《医门方》云∶疗失精无故自泄,或因尿精出方∶韭子白龙骨菟丝子(各二十分)麦门冬(去心)车前子泽泻(各十二分)人参(十分)为散,空腹以酒服方寸匕,日二,加至一匕半。

《录验方》淮南王枕中丸治阴气衰,腰背痛,两胫(于缘反)疼,小便多沥,失精,精自出石斛巴戟天桑螵蛸杜仲凡四物,分等合捣下筛,蜜丸如梧子,酒服十丸,日二。令强阴、气充、补诸虚,神良。
治虚劳精血出方第五

《病源论》云∶肾藏精,精者血所成也,虚劳则生七伤六极,气血俱损,肾家偏虚,不能藏《葛氏方》云∶治失精、精中有血方∶父蛾二七枚,阴干之,玄参称半分,合捣末,以米汁向旦日一服,令尽之。
治虚劳少精方第六

《千金方》治虚劳少精方∶鹿角白末,蜜和,服如梧子七丸,日三,三十日大效。

又方∶生地黄根五升,以清酒渍,少少饮之。

又方∶浆煮蒺藜令熟,洗阴,日二,十日知。

《杂酒方》云∶桑树东南枝白皮一把,细切,以酒一升渍之,经宿去桑皮,多少饮之,精如
治虚劳不得眠方第七

《病源论》云∶邪气客于脏腑,则卫气独营其外,行于阳,不得入于阴。行于阳则阳气盛,《千金方》云∶千里水汤治虚烦不眠方∶半夏(三两)秫米(一升)茯苓(四两)酸枣(二升)麦门冬(三两)甘草(二两)桂心(姜(四两)十二味,以千里水一石,煮米令蟹沸,扬之万过,澄取清一斗煮诸药,取二升半分三服。

又方∶疗虚劳不得眠方∶酸枣榆叶分等,丸如梧子,一服五丸。

又方∶末干姜四两,汤和,顿服,覆取汗愈。

又云∶酸枣汤主虚劳烦扰,奔气在胸中,不得眠方∶酸枣(五升)人参(二两)石膏(四两)茯苓(三两)桂心(二两)生姜(二两)甘草(一八味,以水一斗先煮酸枣,取七升,去枣,纳余药煎取三升,分三服。

《葛氏方》云∶治卒苦连时不得眠方∶暮以新布火炙熨目,并蒸大豆,囊盛枕之,冷复易,终夜常枕,立愈。

《崔禹锡食经》云∶蛎治夜不眠,志意不定。

《本草经陶景注》云∶榆初生叶,人以作糜羹辈,令人睡眠。嵇公所谓榆令人眠。

《短剧方》流水汤主虚烦不能眠方∶半夏(二两)秫米(一升)茯苓(四两)凡三物,以流水二斗扬之三千过,令劳煮三物,得五升,分服一升,日三夜再。

《僧深方》小酸枣汤治虚劳脏虚,恚不得眠,烦不宁方∶酸枣(二升)母(二两)干姜(二两)甘草(一两)茯苓(二两)芎(二两)凡六物,切,以水一斗煮枣,减三升,分三服。
治昏塞喜眠方第八

《病源论》云∶嗜眠者,由人有肠胃大,皮肤涩者,则分肉不开解,其气行在于阴而迟留,《葛氏方》治嗜眠喜睡方∶父鼠目一枚烧作屑,鱼膏和,注目外则不肯眠,兼又取两目,绛帛裹带之。

又方∶麻黄、术各五分,甘草三分,日中向南捣末,食后服方寸匕,日三。

又方∶孔公孽末五两,通草三两,茗叶一斤,水一斗煮取五升,向暮服之,即一夕不睡眠。

《龙门方》疗嗜睡眠方∶马头骨烧作灰,服方寸匕,日三夜一,瘥。(今按∶《陶潜方》∶马发烧作灰末,服方寸匕又方∶苦参三两,术二两,大黄一两,捣末蜜丸如梧子,每食后服三十丸,验。
治邪伤汗血方第九

《病源论》云∶肝藏血,心之液为汗。言肝心俱伤于邪,故血从肤腠而出也。

《医门方》治汗血、吐血方∶青竹茄(三两)生地黄(五两)人参(一两)桔梗(一两)芎(一两)当归(一两)桂心上,切,以水九升煮取三升,去滓,分温三服,甚神效。

《短剧方》治吐血、汗血、大小便血,竹茹汤方∶竹茹(二升)甘草(六分)当归(六分)芎(六分)黄芩(六分)桂心(一两)术(一两)凡九物,以水一斗煮取三升,分四服。
治虚汗方第十

《病源论》云∶诸阳主表,在于腠理之间,若阳气偏虚,则津液发泄,故为汗,汗多则损于《短剧方》治汗出如水浆及汗血、衄血、吐血、小便血、殆死,都梁香散方∶都梁香(二两)紫菀(一两)桂肉(一两)人参(一两)生竹茹(一两)肉苁蓉(一两)干凡七物,冶筛,以水服方寸匕。

《千金方》治汗出方∶豉一升,酒二升,渍三日,服之。不过三剂,瘥。

《范汪方》治汗粉药方∶牡蛎(二两)干姜(二两)附子(一两,炮)凡三物,捣下,以药一合,白粉二合和合,摩汗出处。

《僧深方》治大虚,汗出欲死,若白汗出不止方∶麻黄根(二两)凡一物,以清酒三升,微火煮得一升五合,去滓,尽服之。

《效验方》治人汗劳不止麻黄丸方∶麻黄根(二分)石膏(一分)凡二物,冶筛,和蜜丸,大人服小豆三丸,日三,小儿以意增损。

《录验方》治腋下汗出作疮方∶橘皮(三两)黄连(三两)甘草(三两)米粉(四两)凡四物,冶筛成粉,汗出时粉之。

又云∶止汗石膏散方∶石膏(四两)甘草(四两)合捣,先食浆服方寸匕,日三。

《陶景本草注》云∶麻黄节及根,夏月杂粉用之。
治风汗方第十一

《经》云∶凡热者欲汗出为通气也,而皮肤致密不汗出也。风者欲汗不出为身疲也,而皮肤《范汪方》治风汗出方∶防风(一两)牡蛎(一两)干姜(一两)凡三物,冶筛一升,白粉三升合搅,以粉之。欲粉时,于铜铫中熬令小温,良。

《僧深方》治风汗出少气方∶防风(十分,一方三两)白术(六分,一方三两)牡蛎(三分,一方三两)凡三物,冶筛,以酒服方寸匕,日三。
治阳虚盗汗方第十二

《病源论》云∶盗汗者因睡眠而身体流汗也,此由阳虚所致。久不已令人羸瘦。

《千金方》牡蛎散治卧即盗汗、头痛方∶牡蛎(三两,熬)术(三两)防风(三两)上三味,酒服方寸匕,日一,止汗之验无出于此,一切泄汗皆愈。

《录验方》止汗石膏散方∶石膏(四两)甘草(四两)合捣筛,先食浆服方寸匕,日三。

《极要方》疗盗汗方∶麻黄(三两)牡蛎粉(二两)蒺藜子(二两)熟米粉(半两)白米粉(六合)以上为粉,生绢袋盛,卧汗出,敷之。

《范汪方》治盗汗麻黄散方∶麻黄根(三分)故败扇烧屑(一分)凡二物,冶筛,以乳服三分匕,日三,大人方寸匕,日三。
治传尸病方第十三

《玄感传尸方》云∶夫传尸之病,为蠹(当户反)实深大,较男夫多以癖及劳损为根,女人所KT,小儿乃曰无辜,因虚损得名为劳极。骨中热者号为骨蒸,微嗽者称曰肺痿(力朱反,曲又云∶论曰∶大都男女传尸之候,心胸满闷,背膊(补各反,《说文》肩甲也)烦疼,两目精多卧小起微热,面好与鬼交,或而精神尚好沉羸,犹如又云∶论曰∶传尸之疾本起作无端,莫问老少男女皆有斯疾。大都此病相克而生,先内传毒瓦斯,周遍五脏,渐就羸瘦至于死,死讫复易(羊益反,转也)亲一人,故曰传尸;亦名转注;以其初得半卧半起,号KT;内传五脏名之伏练。不解疗者,乃至灭门。假如男子因虚损得酸疼,腰脊次传于心无滋味,心既受闻香臭刺痛,或血色,常睡还不着。肝既受已,次传于脾,脾初受气,两胁肤胀,食不消化,又时渴利,熟食生出,有时肱痛,赤黑汁,至又云∶传尸、骨蒸、伏练、KT相染灭门神秘方∶柴胡(三两)桑根白皮(五两)甘草(二两,炙)桔梗(三两)续断(三两)紫菀(四两)赤小豆(一升,小)青竹茹(三两)五味子(三两)干地黄(五两,无者以生十两代之)若热更加石膏三两(末),若不下食更加生麦冬二两(去心)。

凡九物,切,以水九升煮取二升五合,绞去滓,分温三服,服去如人行七八里,重者服五六又云∶主传尸、骨蒸、例(力制反,皆也)多盗汗粉身方∶麻黄根(三分)牡蛎粉(三分)蒺藜子(二两)熟米沙(半两,末)白术粉(六分)胡燕脂凡六物,捣筛,绢袋子盛之,夜卧汗出敷之。

又云∶主传尸、骨蒸、鬼气,恶寒壮热,诸风虚疥癣瘙痒方∶直用桃、柳、槐、蒴四种枝叶,各锉一大升,以水九大斗煮取五大斗,去滓,加盐二大升又云∶主传尸、伏练、KT、骨蒸、癖、鬼气、恶寒、悒悒(于急反,忧也),或如疟等,大椎上穴。又两旁才下少许对椎节间各相去一寸半二穴(名大杼。)又两肋(鲁得反,胁骨也)合七穴,日别取,正午各灸七壮,满一百五十壮,即觉渐瘥。

又云∶主传尸、KT、喜厌(一琰反)梦者,灸商丘二穴。(在足内踝下微前陷者中,灸七壮《广济方》疗瘦病、伏练、诸鬼气恶注,朱砂丸方∶光明朱砂(一大两,碎)桃仁(七十枚,去皮)麝香(三分,碎)上,研朱砂、麝香令细末,后用桃仁、香砂为丸。如其和不敛,以蜜少许合成讫,清饮服一又云∶疗传尸、骨蒸、KT、肺痿、疰、忤、鬼气、卒心痛、霍乱、吐痢、时气、鬼魅、瘴疟、赤白暴痢、瘀血、月闭、癖、疔肿、惊痫、鬼忤中人、吐乳、狐狸吃力伽丸方∶吃力伽(白术是)光明砂(研)麝香当门子诃黎勒皮香附子沉香(重者)青木香脑香(以上,研捣筛极细,白蜜前志沫和为丸,每朝取井花水服如梧子四丸。于净器中研破服之,老不
治骨蒸病方第十四(三)

《病源论》云∶夫蒸病有五∶一曰骨蒸,二曰脉蒸,三曰皮蒸,四曰肉蒸,五曰内蒸。

又云∶有二十三蒸。一胞蒸;二玉房蒸;三脑蒸;四髓蒸;五骨蒸;六筋蒸;七血蒸;八脉胃蒸;二十三《玄感传尸方》云∶论曰∶凡人唯知有骨蒸名而不知亦有心肾等蒸,云云。

又云∶五蒸病者,附骨毒之气,疗之通用生地黄汁,不限日数,此方神验,云云。

又云∶一曰骨蒸者,其根在肾,旦起体凉,日晚即热,烦躁,寝不能安,食无味,小便黄赤食人一服二曰脉蒸。其根在心。日增烦闷,掷手出脚,思水,口唾即浪言,或惊恐不安,脉数。

苦参(二两)青葙(二两)艾叶(一两)甘草(一两,炙)切,以水四升煮取一升半,分为二分,羊胞中盛,灌下部,若不利,服芒硝方寸匕。

三曰皮蒸。其根在肺。必大喘鼻干,口中无水,舌上白,小便赤如血。蒸盛之时,胸满,或自称得注热,两胁下胀,大嗽,彻背连脾疼,眠寐不安。或蒸毒伤脏,口内唾血。急与芒硝一两,日不过三,服之讫,冷水浸手,以熨胁间及腋上,自下第三胁间下腋下空中七壮灸之。

四曰肉蒸。其根在脾。体热如火,烦躁无汗,心腹臌胀,食即欲呕,小便如血,大便秘涩。

水一任意五曰肉蒸,亦名血蒸。所以名内蒸者,必外寒而内热,把手附骨而肉热甚。其根在五脏六腑足趺上必又云∶主丈夫因虚劳损,梦泄、盗汗、小便余沥、阴湿弱欲成骨蒸者,名曰劳极,黄大补黄(三两)生姜(三两)人参(三两)大枣(二十枚,擘)牡蛎(二两)夕药(三两)桂两)橘皮(三两)磁石(三凡十四味,切,以水一斗二升煮取三升,绞去滓,分温三四服,服如去八九里,复取微润,又云∶主骨蒸、肺痿、手足烦热兼汤或不能食,芦根饮方∶芦根(切,十两)麦门冬(十两,去皮)地骨白皮(十两)生姜(十两,切)茯苓(五两)兼服石,人骨中寒、虚胀痛者,加吴茱萸八两。

凡六味,切,以水二斗煮取八升,绞去滓,分温五服,昼三服夜二服。忌如药法。

又云∶主骨蒸、肺痿、四体烦热不能食,口干者,麦门冬饮方∶麦门冬(三升,去心,生者二升)地骨白皮(二升)小麦(一升)凡三味,以水一斗三升,先煮小麦取一升,去麦门二味,更煮取三升,绞去滓,分温三服,又云∶小龙胆丸疗骨蒸身热,手足烦,心中懊羸瘦不能食方∶龙胆(五分)黄连(去毛)夕药甘草黄柏大黄黄芩人参栀子仁(各四分)凡九味,捣筛,蜜丸,饮服三丸,丸如梧子,稍加,以知为度,日二三服,忌如法。

又云∶骨蒸之病,无问男女,特忌房室、举动劳作,尤不宜食陈臭咸酸难消粘食,牛、马、并不并又云∶主骨蒸癖气等灸方∶两肩井、上廉、下廉,灸七壮。

又方∶夹脐两旁各相去一寸二分,两乳下一夫肋间。灸如前法。

《广济方》疗骨蒸单方∶肺气每至日晚即恶寒壮热,颜色微赤,不能下食,日渐羸瘦方∶生地黄(三两,切)葱白(二两,切)香豉(二两)童子小便(二升)甘草(二两,炙)上,地黄等于小便中浸一宿,平晨煎两沸,绞去滓,澄去淀,取一升二合,分温二服。

忌食又云∶疗瘦病方∶灵天盖(一大两,死人顶骨)麝香(半脐)桃仁(一大抄,去皮)生朱砂(一两半,光明者)上五味,各别捣筛讫,然后总和合调,每晨空腹以小儿小便半升,和散方寸一匕服。忌生血《广利方》理骨节热积渐黄瘦方∶鳖甲(六分,炙)知母(四支)大黄(六分)葱白(五茎)豉(十二分)桑根白皮(八分)切,以童子小便一大升三合煎取八合,去滓,食后良久分温三服,服相去如人行七八里。

频又方∶大黄四分,切,以童子小便五合煎取四合,去滓,空腹分温两服,服相去如人行四五
治肺痿方第十五(四)

《病源论》云∶肺主气,为五脏上盖。气主皮毛,故易伤于风邪。风邪伤于腑脏,而血气虚气,能咳《广利方》疗肺痿唾脓血腥臭,连连嗽不止,渐将羸瘦,形容枯方∶紫菀头(二十一枚,髻子充)桔梗(十二分,微炙)天门冬(八分)茯苓(十二分)生百合切,以水二大升煮取九合,食后良久分温三服,服如人行五六里,进一服,要利,加芒硝八《玄感传尸方》主肺痿咳嗽,上气不得卧,多粘唾等泻肺汤方∶葶苈子(三两,微火熬令紫色,捣如泥之)大枣(二十枚,破)桑根白皮(三两,切)凡三味,以水三升煮枣及桑皮,取一升,去滓,纳葶苈子泥如弹丸许,搅令消散,更煮三分生冷又云∶肺痿、骨蒸、气,若吐血、声破,或多唾、或口干兼渴不能食,单服小便方∶单服自身及他人小便百日即瘥。日四服,一服一升。

《集验方》治肺痿咳吐涎沫不止,咽燥而不渴方∶生姜(五两)人参(三两)甘草(四两,炙)大枣(十五枚,擘)凡四物,以水七升煮取三升,分三服。(今按∶《千金方》∶甘草三两,大枣十枚。)
卷第十四
治卒死方第一

《病源论》云∶卒死者,由三虚而遇贼风所为也。三虚,谓乘年之衰,一也;逢月之空,二也;失时之和,三也。人有此三虚,而谓为贼风所伤,使阴气偏竭于内,阳气阻隔于外,二有《葛氏方》治卒死,或先有病痛,或居常倒仆奄忽而绝,皆是中恶。治之方∶令二人以衣壅口,吹其两耳,极则易人,亦可以竹筒吹之,并侧身远之,莫临死人上。

又方∶以葱叶针其耳,耳中、口中、鼻中血出者莫怪,无血难治,有血是治候也。

又方∶以绵渍好苦酒中,须臾出,置死人鼻中,手按令汁入鼻中,并持其手足莫令惊。

又方∶以人小便灌其面数回,即能语,此扁鹊法也。

又方∶末皂荚如大豆,吹其两鼻孔中,迁则气通。

又方∶捣女青屑以重一钱匕,开口纳喉中,以水若酒送之,立活。

又方∶灸脐中百壮。

又方∶灸其颐下宛宛中名承浆十壮。

又方∶灸心下一寸。

《集验方》治卒死方∶取牛马矢汁饮之,无新者,水和干者取汁。

又方∶取灶突中墨如弹丸,浆水和饮之,须臾三四服之。

又方∶取梁上尘如大豆粒,着竹筒中吹鼻中,与俱一时吹之。

又方∶灸膻中穴。

又方∶取竹筒吹其两耳,不过三。

《新录方》治卒死方∶韭根捣取汁,服六七合。

又方∶桃白皮,切,一升,水二升,煮取八合,一服之十里,久不瘥,更服之。

《僧深方》治卒死中恶雷氏千金丸方∶大黄(五分)巴豆(六十枚)桂心(二分)朴硝(三分)干姜(二分)凡五物,冶下筛,和白蜜冶三千杵,服如大豆二丸,老小以意量之。

《枕中方》治卒忤恶鬼魍魉欲死者,书额上作“鬼”字即愈。

《龙门方》疗卒死方∶取绳围死者辟腕,男左女右,以绳当大椎伸绳向下,当绳头灸脊上五十壮。

又方∶粪汁灌鼻。

又方∶以葱黄心刺鼻中入七八寸,男左女右,立验。

又方∶捣韭汁灌鼻即活。

又方∶桂屑着舌下即活。
治中恶方第二

《病源论》云∶中恶者,是人精神衰弱,为鬼邪之气卒中之也。其状卒然心腹刺痛,闷乱欲《广济方》云∶卒中恶心腹刺痛去恶气方∶麝香(一分,研)生犀角(二分)青木香(二分)为散,空腹以熟水服方寸匕,立愈。

《范汪方》治卒死,及心痛腹满,魇忤中恶三物备急丸方∶巴豆(一分,去心皮)大黄(二分)干姜(二分)凡三物,共捣巴豆,冶合丸,以蜜丸如大豆,有急取二三丸,以水和服之。口噤者绞开令药又方∶取杯水,以刀三七刺中,饮之良。

《集略方》备急散,治卒中恶,心痛腹满,欲吐短气方∶大黄(二两,金色者)桂心(四分)巴豆(一百枚)凡三物,冶合下筛,取一钱,以水七合服之。

《千金方》治卒死中恶方∶取牛马屎汁饮之,无新者,水和干者亦得。

又方∶葱心黄刺鼻孔中,血出愈。

又方∶灸两胁下。

又方∶灸胃脘十五壮。

《集验方》治中恶方∶大豆二七枚,以鸡子中黄、白酒半升合和,顿服之。

又方∶用釜底墨、盐三指撮,和水服之。(《医门方》同之。)又方∶以度度其两乳中央,屈之从乳头向后肋间,灸度头,随年壮。

又方∶灸胃脘五十壮。

《广利方》疗中恶客忤垂死方∶麝香钱重,研,和醋二合服之,即瘥。

《葛氏方》华佗治卒中恶短气欲死者方∶韭根(一把)乌梅(十枚)茱萸(半升)以劳水一斗煮之,以病患栉纳中三沸,栉浮者生,沉者死。煮得三升饮之。

又方∶灸足两拇指上甲后丛毛中各十四壮即愈。

《新录方》治卒中恶方∶豉(一升)盐(七合)水(四升)煮取一升二合,分再服。

又方∶桃白皮,切,一升,水二升,煮取八合,一服之。

又方∶生菖蒲根,切,三升,捣绞取汁,服四五合。

又方∶酒服桃仁末方寸匕。李仁末亦佳。

又方∶伏龙肝末,水服二方寸匕。

又方∶取竹木中虫屎,水服方寸匕。

又方∶盐一升,水二升,煮临消二服,取吐。
治鬼击病方第三

《病源论》云∶鬼击者,谓鬼厉之气击着于人也。得之无渐,卒着如人以刃旁刺状,胸胁腹鬼触免,重者多死也。

《葛氏方》治鬼击病方∶以淳苦酒吹纳两鼻孔中。

又方∶灸脐上一寸七壮。

又方∶灸鼻下人中一壮,立愈。不愈可加壮数也。

又方∶灸鼻。

又云∶治诸飞尸鬼击走马汤方∶巴豆(二枚)杏仁(二枚)合绵裹椎令碎,投热汤二合,中指捻令汁出正白,便与饮之。如食顷下便瘥,老小量之。

《僧深方》治鬼击方∶盐一升,水二升和之,搅令KT作汁,饮之令得吐则愈,良。

《新录方》治鬼击病方∶捣薤汁灌鼻中如杏仁许,须臾瘥好。

《千金方》治鬼击病方∶艾如鸭子大三枚,水五升,煮取二升,顿服之。

又方∶吹酢少许鼻中。

又方∶灸脐上一寸七壮。

又灸脐下一寸三壮。
治客忤方第四

《病源论》云∶卒忤者,亦名客忤,谓邪客之气,卒犯忤人精神也。此是鬼厉之毒瓦斯、中恶《葛氏方》∶客忤死者,中恶之类也,喜于道间门外得之,令人心腹绞痛,胀满,气冲心胸以水渍粳米,取汁一二升饮之,口已噤者,以物强发,纳之。

又方∶铜器若瓦器盛热汤,先以衣三重藉腹上,乃举汤器着衣上,汤转冷者去衣,器亲肉,又方∶捣书墨,水和服一钱匕。

又方∶以绳横度其人口,以度度脐,去四面各一处,灸各三壮,令四火俱起。

又方∶灸鼻下人中三十壮,令切鼻柱下。

又方∶横度口中,折之,令上头着心下,灸下头五壮。

又云∶已死者捣生菖蒲根,绞取汁含之,立愈。

《千金方》治客忤方∶盐八合,水三升,煮取一升半,分二服,得吐即愈。若小便不通,笔头七枚烧末,水和服之又方∶灸巨阙百壮。

《范汪方》治客忤方∶捣牛子矢半杯,以酒三升煮服之。

《新录方》治客忤方∶捣生艾心,取汁,灌口中五合。

又方∶水浣甑带服之。
治魇不寤方第五

《病源论》云∶人睡眠,则魂魄外游,为鬼所魇,屈其精神,弱者魇则久不得寤,乃至气暴《集验方》治卒魇欲死方∶捣生韭汁灌鼻孔中,剧者并灌两耳。

《徐伯方》治魇唤不寤方∶取葱叶针鼻中,慎勿火照。

《葛氏方》云∶卧魇不寤,勿以火照之,照之杀人。但痛啮其踵及足拇指甲际,而多唾其面末皂荚,以管吹纳两鼻孔中,即起。已三四日犹可吹之。

又方∶末灶中黄土,吹纳两鼻孔中。

又方∶取韭菜捣,以汁吹其鼻孔中,冬月掘根可绞。

又方∶以笔毛刺鼻孔,男左女右,可展转进之。

又方∶以牛若马临魇人上二百息,青牛尤佳。

又方∶末菖蒲吹鼻中,末桂纳舌下。

又方∶令一人坐头首,一人于户外呼病者姓名,坐人应曰人诺,在便即得苏也。(今按《救又云∶喜魇及恶梦者方∶枕真射香一子于头边。

又方∶带雄黄,男左女右。

又方∶作犀角枕佳。

又方∶以虎头为枕。

又方∶以青木香纳枕中,并带之。

《千金方》治魇不寤方∶雄黄如枣大,系左腋下,令人终身不肯魇。(《集验方》同之。)又方∶伏龙肝末,吹鼻中。

《养性志》云∶人魇勿燃火唤之,魇死不疑。暗唤之,吉,但得远唤之,不得近而急唤,喜《救急单验方》疗魇死方∶引牛临鼻,少时即活。

又方∶啮其足拇指爪甲际,活。

《范汪方》治魇死符法,魇死未久故可活方∶书此符烧令黑,以少水和之,置死人口,悬镜死者耳前,击镜呼死人,不过半日即生。

“KTKT”丹书之。

《新录方》若魇不悟者方∶酒服发灰一撮许。

又方∶捣蒴根茎,取汁一升服之。
治尸厥方第六

《病源论》云∶尸厥者,阴气逆也。由阳脉卒下坠,阴脉卒上升,阴阳离居,营卫不通,真如《葛氏方》云∶尸厥之病,卒死而脉犹动是也。

以管吹其左耳,自极三过,复吹右耳三过,即起。

又方∶捣昌蒲以如枣核大,着耳舌下。

又方∶灸鼻下人中七壮。

又方∶灸膻中穴二七壮。

又方∶灸阴囊下去大孔一寸百壮,若妇人者灸两乳之中间。

又方∶以菖蒲屑着鼻两孔中,吹之令入,以桂屑着舌下。云扁鹊治楚王法也。(以上二方《《集验方》治厥死如尸,不知人,心下余气,扁鹊灸法∶以绳围病患臂腕,男左女右,伸绳从大椎上度下之,灸绳头脊上五十壮。

《范汪方》治尸厥方∶以梁上尘如豆者着筒中,吹鼻中与耳,同时吹之。

又方∶生韭汁灌口中。

《新录方》治尸厥方∶取葱白一升,水二升,煮取一升,顿服之。

又方∶酒服桃仁末方寸匕。

《救急单验方》尸厥死方∶灸两足大指甲后丛毛内七壮。华佗云∶二七壮。
治溺死方第七

《病源论》云∶人为水所浸溺,水从孔窍入,灌注腑脏,其气壅闭,故死。若早拯救得出,又云∶经半日及一日,犹可活;气已绝,心上暖亦可治之。

《短剧方》治溺水死已经二宿者可活方∶捣皂荚作末,以绵裹,纳死人下部中,须臾牵出,即活也。

又云∶治溺水死方∶以灶灰布着地,令浓五寸,以甑倒覆灰上,以溺人覆伏甑上,口中水当出也,觉水出,复更小时又方∶令二健人抱溺人,倒卧沥溺人,水出尽便活也。

《葛氏方》∶溺死一宿者尚可活。治之方∶倒悬死人,以好酒灌其鼻,立活。(《龙门方》同之。)又云∶身尚温者,取灶中灰二石余埋人,从头至足,即活。

又方∶便脱暖釜覆之,取溺人伏其上,腹中水出便活。

《千金方》治落水死方∶酢灌鼻。

又方∶裹锻石纳下部中,水尽出即活。

又方∶但埋死人暖灰中,头足俱没,唯开七孔。

《录验方》治溺死方∶灸脐中。

《集验方》治溺水死方∶熬沙以覆死人,使上下有沙,但出鼻口,中沙温湿,须易之。
治热死方第八

《病源论》云∶夏月炎热,人冒涉途路,热毒入内,与五脏相并,客邪炽盛,郁瘀(依倨反)击,真《葛氏方》凡中热死,不可使卒得冷,得冷便仍死矣。治之方∶以泥作HT。绕人脐,使三四人更溺其中。

又方∶亦可屈草带,亦可扣瓦碗底若脱车,以着人脐上,取令溺,不得流去而已,此道又方∶干姜、橘皮、甘草末,少少纳热汤中,令稍稍咽,勿顿多,亦可煮之。

《短剧方》治中热方∶取路上热尘土,以壅其心上,小冷复易之,气通乃止。

又方∶偃卧人,以草带围脐上,令人溺脐中即。

又方∶浓煮蓼,饮之至一二升,良效。

《千金方》治热方∶张死人口令通,以暖汤徐徐灌口中,小举死人头身,令汤入肠,须臾即苏。

又方∶灌地浆一杯即愈。

又方∶抱枸子若鸡,着心前熨之。

又方∶地黄汁一杯,服之。

又方∶但以热土及熬灰土壅其心上,佳。
治冻(东贡反)死方第九

《病源论》云∶人有在于途路,逢凄风苦雨,繁霜大雪,衣服沾濡,冷气入脏,致令阴气闭又云∶冻死一日,犹可治,过此则不可治也。

《葛氏方》治冬天堕水,冻四肢直,口噤,才有微气方∶以大器多熬灰,使暖,囊盛,以薄甚心上,冷复易。心暖气通,目则转,口乃得开。温酒千金方》、《救急单验方》皆同之。)
治自缢死方第十

《病源论》云∶人有不得意志者,多生忿恨,往往自缢,以绳物系颈,自悬挂致死,呼为自又云∶自经死,旦至暮,虽已冷,必可治;暮至旦,则难治。此谓其昼则阳气盛,至(其)气又云∶夏则夜短又热,则易治。

又云∶气虽已断,而心微温者,一日以上,犹可治。

又方∶用衣覆其口鼻,两人吹其两耳即生。

《短剧方》治自缢死方∶旁人见自经者,未可辄割绳,必可登物令及其头,即悬牵头发,举其身起,令绳微得宽也;死人又方∶治自经死,慎勿割绳也。绳卒断,气顿泄去便死,不可复救也。徐徐抱死人,渐渐揉之,取血,《千金方》治自经死方∶凡自经死,勿截绳,徐徐抱解之。心下尚温者,氍毹覆口鼻,两人吹其两耳。(《救急单验又方∶蓝青汁服之。

又方∶梁上尘如大豆,各纳筒中,四人各捉一筒,同时吹两耳两鼻即治。

又云∶五绝方∶一曰自经,二曰墙壁,三曰溺水,四曰魇魅,五曰产乳绝。皆以半夏一两,《集验方》治自经死方∶捣皂荚、细辛屑,取如胡豆,吹两鼻孔中止。单用皂荚亦好。

《葛氏方》云∶自经死,虽已久,心下尚微温,犹可治也。治之方∶末皂荚,以葱叶吹纳其两鼻孔中。

又方∶以芦管吹其两耳,极则易人,取活乃止。若气通者,少以桂汤稍稍咽,徐徐乃以少少《龙门方》疗自经死方∶皂荚末如胡豆许,吹两鼻中,嚏即活。
治注病方第十一

《病源论》云∶凡注之言住也,谓邪气俱住人身内,故名为注。此由阴阳失守,经络空虚,端,《僧深方》云∶西王母玉壶赤丸,备急治尸注卒恶水陆毒螫(丑略反)万病方∶武都雄黄(一两,赤如鸡冠)八角大附子(一两,炮,称)藜芦(一两)上丹砂(一两,不称之。一方有凡六物,悉令精好。先冶巴豆三千杵;次纳石,冶三千杵;次纳藜芦,冶三千杵;次纳雄用丹方别乃更无云治万腹以下病者,宿勿食,明旦服二丸。不知者,饮暖米饮以发之令下,下不止饮冷水饮止之。

病在膈上吐,膈下者下,或但噫气而愈,或食肉不消,腹坚胀,或痛。服一丸立愈。

风疝、寒疝、心疝、弦疝,每诸疝发腹中急痛,服二丸。

积寒热老痞、蛇痞,服二丸。

腹胀不得食饮,服一丸。

卒大苦寒热往来,服一丸。

卒关格,不得大小便欲死,服二丸。

瘕结,服一丸,日三服。取愈。若微者射丸甚良。下利重下服一丸便断。或复天行,下便疟未发服一丸,已发服二丸便断。

小儿百病痞,寒中及有热,百日半岁者,以一丸如黍米,着乳头与服之;一岁以上服如麻子亦伤寒敕(络代反,劳也)色及时气病,以温酒服一丸,浓覆取汗即瘥,若不汗复酒服一丸,要苦淋路妇人产生余疾,及月水不通,及来往不时,服二丸,日二。

卒霍乱,心腹痛,烦满吐下,手足逆冷,服二丸。

注病,百种病不可名,将服二丸,日再。

若腹中如有虫欲钻胁出,状急痛,一止一作,此是风气,服二丸。

若恶疮不可名疥疽,以膏若好苦酒和药,先盐汤洗疮去痂,拭令燥,以药涂之即愈。

恶风游心不得气息,服一丸即愈。

耳出脓血汁及卒聋,以赤KT裹二丸,塞耳孔中即愈。

痈肿痤(昨示反)疖(音节)瘰及欲作,以苦酒和药涂之。

齿痛,以小丸绵裹着齿孔中咋之。

若寒热往来,服二丸。

若蛇蝮蜂蝎蛎所中伤也,及犬狂马所咋,以苦酒和涂疮中,并服二丸即愈。

卒中恶欲死,不知人,以酒若汤水和二丸,强开口灌喉中,捧坐令下。

若独宿止林泽之中若冢墓间,烧一丸,百鬼走去不敢近人。

癖饮、留饮、痰饮,服一丸,以腊和一丸如弹丸。着绛囊中以系臂,男左女右。山精鬼魅皆中溪水毒,服二丸。

已有疮在身,以苦酒和三四丸涂疮上。

忧患之气结在胸中,苦连噫及咳,胸中刺痛,服如麻子三丸,日三愈。

妇人胸中若滞气,气息不利,小腹坚急,绕脐绞痛,浆服如麻子一丸,稍增之如小豆。

心腹常苦切痛及中热,服一丸如麻子,日三服,五日愈。

男女邪气,鬼交通,歌哭无常;或腹大经绝,状如妊娠,皆将服三丸如胡豆大,日三夜一。

又将腹中三虫,宿勿食,明平旦进牛羊肉,炙三膊,须臾便服三丸如胡豆,日中当下虫。过日中小儿寒热,头痛身热及吐,一服一丸如麻子,小儿瘦,丁奚不能食,食不化,将服二丸又苦酒和如饴涂,涂儿腹良。

风目赤或痒,视漠漠,泪出烂,以蜜解如饴,以涂注目。

头卒风肿,以苦酒若膏和涂之即愈。

风头肿以膏和涂之,以絮裹之。

若为蛄毒所中,吐血,腹内如刺,服一丸如麻子,稍益至胡豆;亦以涂鼻孔中;以膏和通涂治鼠,以脂和涂疮,取驳舌狗子舐之即愈也。

《千金方》治一切注无新久方∶先仰卧,灸两乳两边斜下三寸第二肋间,随年壮,可至三百壮。

又,心下三寸,灸六十壮。

又,两手大指头,各灸七壮。

又云∶十注丸主十种注∶气注、劳注、鬼注、冷注、生人注、死人注、尸注、水注、食注、雄黄(一两)巴豆(二两)人参(一两)甘草(一两)本(一两)桔梗(一两)附子(一十味,空腹服一丸如小豆,日二,以知为度,有验《葛氏方》云∶注病即是五尸中之尸注,又狭诸鬼邪为害也。大略令人寒热淋沥,沉沉嘿嘿此疾桃核仁五十枚,研之。以水一斗,煮取四升,一服尽当吐病,病不尽,二三日更作,若不吐《集验方》治注病方∶取桑根白皮切二斗,曝燥,烧作灰,汤淋取汁,浸小豆二斗,如此取灰汁尽,蒸豆熟作羊鹿又云∶治鬼注病相染易尽门方∶獭肝一具干之,下筛,水服方寸匕,日三,神方。(《千金方》同之。)《极要方》疗恶疰,入心欲死,无问远近年月皆愈方∶安息香半两为末,酒服即愈。(《救急单验方》同之。)又云∶疗疰气发无恒处方∶白芥子,捣为丸服之;又酢和涂,随手为验。

《龙门方》疗恶疰入心欲死方∶独头蒜一枝,书墨如枣大,并捣以酱汁一小合,顿服,立瘥。

又方∶取盐如鸡子,布裹烧赤末,酒服吐即验。

又方∶取椒,布裹,薄布疰上,以熨斗盛火熨之,令汗出验。

《新录方》云∶恶疰方∶盐五合、灶突墨三合,水三升,煮盐消去滓,顿服,吐瘥。

又方∶桃枝切三升,水四升,煮取一升六合,二服。

《范汪方》治尸注毒痛往来方∶烧发灰杏子中仁(熬令紫色)凡二物,分等,膏和,酒服梧子三丸,日三。

《救急单验方》疗恶疰方∶阿魏药服二分,和酒立瘥。

又方∶桂心三两,酒三升煮取一升,分再服瘥。
治诸尸方第十二

《病源论》云∶人身内自有三尸诸虫,与人俱生,而此虫忌恶,能与鬼灵相通,常接引外邪引腰《葛氏方》云∶虽有五尸之名,其例皆相似,而小有异者∶一飞尸(变作无常;)二遁尸(闻又云∶凡五尸,即是身中尸鬼接引外邪,共为病害。经术其有消灭之方,而非世徒能用,今雄黄(一两)大蒜(一两)捣令相和如弹丸者,纳二合热酒中,服之须臾瘥。未瘥更一服便止。有尸疹(敕忍反)者常宜又方∶桂、干姜分等,末之。盐三指撮,熬令青,末合,水服二方寸匕。

《新录方》治飞尸方∶灸脊中及两旁相去三寸,各五十炷。

又方∶桃白皮切二升,水四升,煮取一升六合,分三服。

又云∶治遁尸方∶牛蹄下土三指撮,酒一盏下。(亦治风尸。)又方∶盐墨汤顿服。

又方∶炒艾熨之。

又方∶熬艾以青布裹,更熨。

又方∶熬大豆裹,更熨。

又云∶治沉尸方∶灸太仓(中管也)七壮。

又灸乳下一寸,七壮。

又方∶发灰、杏仁、蜜和丸如梧子,一服七丸,日二。

《千金方》治遁尸尸注方∶桂心(一两)干姜(二两)巴豆仁(二枚)三味,下筛,上酢和,和如泥,涂病上,干即易。

又∶芥子薄主遁尸飞尸方∶芥子一升,蒸熟捣下筛,以黄丹二两搅之,分作两分,疏布袋盛之,更蒸使熟,薄痛上,当
治诸疟方第十三

《病源论》云∶夏日伤暑,秋必病疟。疟其人形瘦皮栗,以月一日发,当以十五日愈。

设不《通玄》云∶疟病多种,各不同形,有温疟,有寒疟,有(于禁反)疟,有劳疟,有鬼疟。

宿汁,《集验方》云∶黄帝曰∶夫(音皆)疟皆生于风,夏伤于暑,秋为疟。间日疟先寒而后热后热也。

其但热而不夫疟必从四末始,先其发时一食顷,用细左索绳坚束其手足十指,过时乃解。

又方∶取大蜘蛛一枚,纳芦管中,密塞管口,绳系以绾颈,过发时乃解去。(《葛氏方》同又方∶桃叶二七枚,安心上,艾灸叶上十四壮。(《僧深方》同之。)《千金方》治疟方∶未发前,抱雄鸡一头着怀中,时时惊动,令鸡作大声,无不瘥。

又方∶故鞋底去两头,烧灰,井花水服之。

又方∶未发前预灸项大椎尖头,渐灸过时止。

《葛氏方》治疟病方∶又方∶多煮豉作汤,饮数升,令得大吐便断。

又方∶破一大豆去皮,书一片作“日”字,一片作“月”字。左手持“日”右手持“月”,吞之立愈。向日服,勿令人知之。

又方∶炙鳖甲捣末,酒服方寸匕,至发时令三服,兼用火灸无不断。

《龙门方》疗一切疟方∶取恒山、甘草等分,捣末和水,服方寸匕。吐即瘥。

又方∶取莲捣末三指撮,以酒和,欲发前服之,验。

《范汪方》治疟方∶临发时,捣大附子下筛,以苦酒和之,涂背上。

《广济方》疗疟方∶恒山(三两)以浆水三升,浸经一宿,煎取一升,欲发前顿服之。微吐瘥止。无所禁忌。

《短剧方》断疟先大寒后大热者方∶小麦(一升)淡竹叶(一虎口)恒山(三两)凡三物,以水五升,宿渍,明旦煮取二升半,分三服。

《僧深方》治一切诸疟无不断恒山丸方∶大黄(一两,一方二两)附子(一两,炮)恒山(三两)龙骨(一两)凡四物,冶合下筛,蜜和,平旦服梧子七丸;未发中间复服七丸;临发服七丸。若不断,至后后日复发,更服如此法,甚神良。

《本草经》云∶治疟,煮草汁及生汁服之。

《本草稽疑》云∶治疟,烧猫屎为末,酒服方寸匕。
治鬼疟方第十四

《范汪方》治鬼疟方∶丹书额言∶“戴九天”;书臂言∶“抱九地”;书足言“履九江”;书背言∶“南有高山,上有大树,下有不流之水,中有神虫,三头九尾,不食五谷,但食疟鬼,朝食三千,暮食三百,急急如律令。”书胸言∶“上高山,望海水,天门亭长捕疟鬼,得便斩,勿问罪,急急如律令。”(《产经》同之。)又云∶平旦发者,市死鬼,恒山主之,服药讫持刀;食时发者,缢死鬼,蜀木主之,服药讫持索;日中发者,溺(奴历反)死鬼,大黄主之,服药讫持盆水;晡时发者,舍长鬼,麻黄主之,服药讫持磨衡;黄昏发者,妇人鬼,细辛主之,服药讫持明镜;夜半发者,厌死鬼,黄芩主之,服药讫持车软;鸡鸣发者,小儿鬼,附子主之,服药讫持小儿墓上折草木;凡七物,各一分,冶下筛,发时加所主病药一分,当发日从旦至发时,温酒服方寸匕,三服服讫,必持所主病物,甚良,有效。

又云∶平旦作者,客民鬼也,先作时,令病者持衣如辞去,言欲远出立愈;食时作者,客死鬼也,先作时,令病者辞言欲归之,大道上桥梁下逃之;禺中作者,市死鬼也,先作时,令病者因结械,北向吐营以支以KT。

日中作者,溺死鬼也,先作时,令病者取盆水中着庭,南向坐营KT;日跌作者,亡死鬼也,先作时,令病者人言吏捕汝庭中;晡时作者,自经死鬼也,先作时,令病患当栋下卧以绳索病者头;日入作者,人奴舍长死鬼也,先作时,令病者磨碓间逃之;黄昏作者,盗死鬼也,先作时,令病者逾去远亡,无令人知其家;人定作者,小儿鬼也,先作时,病者取小儿墓上折草木立愈;夜过半作者,囚死鬼也,先作时,取司空械笞,令病者持之,因从出可榜笞汝者;夜半作者,寒死鬼也,先作时,令病者温衣营以KT,持桃枝饮食逃内中,无人知见紫此次上;鸡鸣作者,乳死鬼也,先作时,令病者把(古老反)席之,菇目应令持桃枝营以KT。

《如意方》治鬼疟方∶发日早旦,取井花水丹书额作“天狱”字;书胸作“胸狱”字;书背作“背狱”字;左手作似急又方∶计发日,今夕可食,鸡鸣起,着衣履屐,随意出户,脱之途,出勿顾,入幽闲隐断也《通玄》云∶鬼疟者,或间日,或频日发作无时者,此为鬼疟,任避之,及用饮食送遣,如恒山(三两)豉(一升)秫米(百粒)蒜(七斤)研,清酒二升,渍之一宿,早旦服之。得大吐则止。
治温疟方第十五

《病源论》云∶夫温疟与寒疟安舍?温疟者,得之冬,中于风寒,气藏于骨髓之中,至春则热而后《通玄》云∶温疟者,吸吸发热而少寒,心闷面赤发自心方∶石膏(半斤,研,绵裹)知母(三两)地骨皮(三两)玄参(三两)淡竹叶(一升)猪苓(水九升,煮取三升,分三服,相去如六七里。
治寒疟方第十六

《病源论》云∶寒疟,此由阴阳相并,阳虚则阴胜,阴胜则寒,发于内而并于外,所以内外《通玄》云∶寒疟者涩涩而恶寒毛竖,发则引温而少热方∶朱砂(四分,研)雄黄(五分,研)人参(四分)恒山(五分)牡蛎(四分,熬)蜜丸如梧子,一服七丸,日再,空腹服之,用粥饮之。
治痰实疟方第十七

《病源论》云∶谓病患胸膈先有停痰结实,因感疟病,则令人心下支满,气烦呕之。

《范汪方》治疟痰实不消恒山汤方∶恒山(六分)甘草(四分)知母(三分)麻黄(三分)大黄(四分)凡五物,切,以水五升,煮取二升,分三服。至发时令尽。
治劳疟方第十八

《病源论》云∶凡疟积久不瘥,则表里俱虚,客邪未散,真气不复,故疾虽系(暂)间,小劳《集验方》治劳疟积时不断,众治无效,此方治之∶生长大牛膝一大虎口,切,以水六升,煮取二升,分再服。第一服取未发前一食顷;第二服《葛氏方》云∶老疟久不断者方∶末龙骨方寸匕,先发一时以酒一升半,煮三沸,及热尽服,温覆取汗立愈。

又方∶炙鳖甲捣末,酒服方寸匕,至发时令三服,兼用火灸,无不断也。

《僧深方》治劳疟桃叶汤方∶桃叶(十四枚)恒山(四两)凡二物,酒二升,渍一宿,露着中庭,刀着器上,明旦发日凌晨漉去滓,微温令暖,一顿服
治嶂疟方第十九

《病源论》云∶山嶂疟,此病生于岭南,带山嶂之气也。其状寒热休作有时,皆由挟溪源岭《耆婆方》治瘴疟要方∶蜀恒山(三两)甘草(二两)光明砂(一两)三种捣筛,以蜜和丸如梧子,未发前服三丸,发时服二丸,发后服一丸,于后三日更一服,三日慎食。

《录验方》治疟及嶂气方∶恒山(二两)甘草(二两)切,以白酒大一升浸一宿,去滓分二服,未发前一服,临发又一服,任吐。慎生冷、酢滑酒
治间日疟方第二十

《病源论》云∶间日疟,此由邪气与卫气俱行于六腑,而有时相失不相得,故邪气内薄五脏《集验方》治疟或间日发或夜发者方∶秫米(百粒)石膏(八两,碎)恒山(三两)竹叶(三两)凡四物,切,以水六升渍药,覆一宿,明旦煮取取二升,分三服,取未发前一食顷第一服;用余《短剧方》断疟恒山酒方治疟先寒战动地,寒解壮热,日日发、间日发并断方∶鳖甲(一两)淡竹叶(切,三升)恒山(三两)甘草(三两)久酒(三升)凡五物,以酒渍药刀置上覆头,安露地,明旦以水七升煮取三升,分五服,未发前令尽。

当
治连年疟方第二十一

《范汪方》治连年疟不瘥牛膝酒方∶牛膝草(一把)好酒(一升)凡二物,牛膝纳酒中,渍一宿,明旦分三服。

《录验方》恒山汤治疟十岁二十岁方∶恒山(二两)甘草(一两)大黄(二分)桂心(六铢)凡四物,切,以恒山酒渍一夜,诸药以酒三升,水二升,煮取七合,顿服,下吐愈。

《僧深方》治三十年疟龙骨丸神方∶龙骨(四分)恒山(八分)附子(三分)大黄(八分)凡四物,冶筛,鸡子和,发前服七丸如大豆,临发服七丸。

《效验方》治三十年疟恒山散方∶恒山(五分)干漆(四分)牡蛎(二分)杏子仁(二分)凡四物,下筛,酒服方寸匕,日三。
治发作无时疟方第二十二

《病源论》云∶夫卫气,一日一夜大会于风府,则腠理开,开则邪入,邪入则病作。当其时所《葛氏方》治疟发作无常心下烦热方∶恒山(二两)甘草(两半)豉(五合)以水六升煮取二升,分再服。当快吐仍断,即饮食。
伤寒证候第二十三

《病源论》云∶经云∶春气温和,夏气暑热,秋气清凉,冬气冰寒,此则四时正气之序也。

气,毒藏温病而反多相又云∶夫热病者,皆伤寒之类也。或愈或死,其死皆六七日间,其愈皆以十日以上。

《葛氏方》云∶伤寒、时行、温疫,虽有三名,同一种耳,而源本小异。其冬月伤于暴寒,人骨诊候此,致《医门方》云∶凡伤寒病五六日而渴欲饮水,水不能多,未宜与也。所以尔者,腹中热尚少与五病反
伤寒不治候第二十四

《葛氏方》云∶阳毒病,面目斑斑如锦文,喉咽痛,下脓血。五日不治,死。

阴毒病,面目青,举体疼痛,喉咽不利,手足逆冷。五日不治,死。

阴毒病,发赤斑,一死一生。

热病,未发汗而脉微细者,死。

内热,脉盛躁,发汗永不肯出者,死。

汗虽出,至足者犹死。

已得汗而脉犹躁,盛热不退者,死。

汗出而言,烦躁不得卧,目精乱者死。

汗不出而呕血者,死。

汗出大下利不止者,死。

汗出而寒不止,鼻口冷者,死。

发热而痉,腰掣纵齿者,死。

不得汗而掣纵,狂走不食,腹满胸背痛,呕血者,死。

喘满言直视者,死。

热不退,目不明,舌本烂者,死。

咳而衄者,死。

大衄不止,腹中痛,短气者,死。

呕咳下血,身热疹(耻刃反)而大瘦削者,死。

手足逆冷而烦躁,脉不至者,死。

大不利而脉疹及寒者,死。

下利而腹满痛者,死。

下利,手足逆冷而烦躁不得眠者,死。

腹满,肠鸣下利,而四肢冷痛者,死。

利止,眩冒者,死。

腹胀,嗜饮食而不得大小便者,死。

身面黄肿,舌卷身糜臭者,死。

不知痛处,身面青,聋不欲语者,死。

目眶陷,不见人,口干谬语,手循衣缝,不得眠者,死。

始得使一身不收,口干舌焦者,死。

疾始一日,腹便满,身热不食者,死;二日口身热,舌干者,死;三日耳聋阴缩,手足冷者溺血脉若不数,三日中当有汗,若无汗者死。

《太素经》热病死候九∶第一汗不出,出不大灌发者死;第二泄而腹满甚者死;第三目不明,热不已者死;第四耆老汗不
避伤寒病方第二十五

《灵奇方》避时气疫病法∶正月未日夜以芦炬火照井及厕臼中,百鬼走不入。

又法∶正月朔日寅时,用黄土涂门扉,方二寸。

又法∶用牛矢涂门户,方圆二寸。

又法∶正月旦若十五日,投麻子、小豆各二七枚入于井中,避一年温病。

又法∶正月旦吞麻子、小豆各二七枚,辟却温鬼。

又法∶庚辰日,取鸡犬毛于门外微烧烟之,避温疫。

又法∶五月十五日日中取井花水沐浴,避邪鬼。

又法∶五月戊己日沐浴避病。

又云∶使温病不相易法∶以绳度所住户中,屈绳烧断。

《医门方》避温疫法∶赤小豆五合,以新布五寸裹,纳井中不至底,少许,三日渍之。平晨东向,男吞二七,女吞又云∶疗温病转相注易,乃至灭门,旁至外人,无有看服此药,必不相易方∶鬼箭羽(二两)鬼臼(二两)赤小豆(二两)丹参(二两)雄黄(二两,研,鸡冠色者)捣筛丸蜜,丸如梧子,服一丸,日二三。与病患同床传衣不相染,神验。

《千金方》云温病时行令不相染方∶立春后有庚子日,温芜菁菹汁,合家大小并服,不限多少。(《极要方》同之。)又方∶常以月望日,细锉东行桃枝,煮汤浴之。

又方∶常以七月七日,合家吞赤小豆,向日吞二枚。

又方∶桃树蠹矢末,水服方寸匕。

《集验方》断温方∶二月旦取东行桑根大如指,悬门户上,又人人带之。

《得富贵方》云∶欲至病患家,手中作“鬼”字。

《玉葙方》云∶屠苏酒治恶气温疫方∶白术桔梗蜀椒桂心大黄乌头菝防风(各二分)凡八物,细切,绯袋盛,以十二月晦日日中悬沉井中,勿令至泥。正月朔旦,出药置三升温里无《葛氏方》云∶老君神明白散避温疫方∶白术(二两)桔梗(二两半)乌头(一两)附子(一两)细辛(二两)凡五物,捣筛,岁旦以温酒服五分匕。一家有药,则一里无病。带是药散以行,所经过病气又云∶度嶂散辟嶂山恶气瘥,若有黑雾郁勃及西南温风皆为疫疠之候,方∶麻黄(五分)蜀椒(五分)乌头(二分)细辛(一分)防风(一分)桔梗(一分)干姜(一凡九物,捣筛,平旦以温酒服一钱匕。

又云∶断温病令不相染着法∶断汲水,绠长七寸,盗着病患卧席下。(《集验方》同之。)又方∶密以艾灸病患床四角各一丸,勿令病患知之。

又方∶以鲫鱼密置病患卧席下,勿令知之。

又方∶以附子三枚,小豆七枚,令女人投井中。
治伤寒困笃方第二十六

《葛氏方》治时行垂死者破棺千金汤方∶苦参一两咀,以酒二升煮,令得一升半,尽服,当吐毒。(《千金方》同之。)《本草》苏敬注云∶人屎干者烧之烟绝,水渍饮汁名破棺汤也,主伤寒热毒。(今按∶《葛氏方》∶世人谓之为黄龙汤。)《耆婆方》治热病困苦者方∶生麦门冬小一升,去心捣碎,熬,纳井花水绞取一升半,及冷分三服,热甚者吐即瘥。

《集验方》云∶凡除热毒无过苦酢之物。

《崔禹锡食经》云∶梨除伤寒时行,为妙药也。

《通玄经》云∶梨虽为五脏刀斧,足为伤寒妙药,云云。
治伤寒一二日方第二十七

《病源论》云∶伤寒一日,太阳受病。太阳者,膀胱之经也,为三阳之首,故先受病。

其脉《葛氏方》云∶伤寒有数种,庸人不能别。今取一药兼治者,若初举头痛,肉热,脉洪起一葱白(一虎口)豉(一升)以水三升,煮取一升,顿服取汗。(《集验方》∶小儿屎三升。)又方∶葛根四两,水一斗,煮取三升,纳豉一升,煮取升半,一服。

又方∶捣生葛根汁,服一二升佳。

《千金方》伤寒时气温疫,头痛,壮热,脉盛,如得一二日方∶真丹一两,以水二升,煮得一升,顿服之,覆取汗。
治伤寒三日方第二十八

《新录方》治伤寒温疫三日,内脉洪浮,头痛,恶寒,壮热,身体痛者方∶葱白(一升)豉(一升)栀子(三七枚)桂心(二两,无,用生姜三两)以水七升煮取二升,分三服之。

《千金方》治疫气伤寒三日以后不解方∶以好豉(一升)葱白(一升)小儿尿(三升)煮取二升,分再服,覆令汗,神验。
治伤寒四日方第二十九

《玉葙方》伤寒四日方∶瓜蒂二七枚,以水一升煮取五合,一服当得吐之。

《葛氏方》三四日胸中恶汁欲令吐之∶豉(三升)盐(一升)水七升,煮取二升半,去滓,纳蜜一升。又煮三沸,顿服,安卧,当吐之。
治伤寒五日方第三十

《范汪方》治伤寒五六日,呕而利者黄芩汤方∶黄芩(三两)半夏(半升)人参(二两)桂心(二两)干姜(三累)大枣(十二枚)凡六物,水七升,煮得二升,分再服。

《通玄》云∶五日外肉凉内热者泻之,宜服升麻汤方∶升麻(二两)黄芩(三两)栀子(二两)大青(二两)大黄(二两,别浸)芒硝(三两)水八升,煮取二升半,分三服,如不利,尽服之。
治伤寒六日方第三十一

《范汪方》治伤寒六七日不大便有瘀血方∶桃仁(二十枚,熬)大黄(三两)水蛭(十枚)虻虫(二十枚)凡四物,捣筛为四丸,卒服,当下血,不下复服。
治伤寒七日方第三十二

《千金方》伤寒吐下后七八日不解,结热在里,表里俱热,时时恶风,大温,舌上干燥而烦知母(六两)石膏(一升)甘草(二两)粳米(六合)四味,水一斗二升,煮米熟,去滓,分服一升,日三。

《葛氏方》∶若已六七日,热盛心下烦闷,狂言见鬼欲走者方∶绞粪汁饮数合至一升,世人谓之为黄龙汤,陈久者弥佳。
治伤寒八九日方第三十三

《录验方》治伤寒八九日腹满,外内有热,心烦不安茈(音柴)胡汤方∶母(二两)生姜(三两)葳蕤(三两)茈胡(八两)大黄(三两)黄芩(二两)甘草(一凡十物,切,以水一斗煮得三升,温饮一升,日三。
治伤寒十日以上方第三十四

《千金方》治伤寒热病十日以上,发汗不解,及吐下后诸热不除,及下利不止皆治之方∶大青(四两)甘草(二两)阿胶(二两)豆豉(一升)四味,以水八升,煮取三升,顿服一升,日三。
治伤寒阴毒方第三十五

《集验方》云∶阴毒者,或伤寒初病一二日便成阴毒,或服汤药六七日以上至十日变成阴毒。身重背强,腹中绞痛,喉咽不利,毒瓦斯攻心,心下强,短气不得息,呕逆,唇青面黑,四肢厥冷,其脉沉细紧数,此阴毒侯。身如被打,五日可治,七日不治方。(《医门方》同之。)甘草(二分,炙)升麻(二分)当归(一分)蜀椒(一分)鳖甲(四分)凡五物,咀,以水五升,煮取二升半,分三服,行五里复服,温覆,中毒当汗,汗则愈,若不汗,病除重服。
治伤寒阳毒方第三十六

《集验方》云∶阳毒者,或伤寒一二日便成阳毒,或服药吐下之后变成阳毒,身重,腰背痛,烦闷不安,面赤狂言,或走,或见鬼,或下利,其脉浮大数,面斑斑如锦,喉咽痛,下脓血。五日可治,七日不可治方(《医门方》同之∶)甘草(二分,炙)当归(一分)蜀椒(一分,去目)升麻(二分)雄黄(二分)桂心(一分)凡六物,咀,以水五升煮取二升半,分三服,行五里须复服,温覆手足,中毒则汗,汗则解,不解重作。今世有此病,此二方实未经用。
治伤寒汗出后不除方第三十七

《集验方》大汗出后,脉犹洪大,形如疟,日一发,汗出便解方∶桂心(一两十六铢)夕药(一两)生姜(一两,炙)甘草(一两,炙)大枣(十四枚)麻黄(一两,去节)杏仁(二十三枚)凡七物,切,以水五升,先煮麻黄再沸,下诸药,煎得一升八合,服六合。
治伤寒鼻衄方第三十八

《千金方》云∶伤寒鼻衄,胁间有余热故也,热因衄自止,不止者方∶牡蛎(十分,左顾者)石膏(五分)上二味,酒服方寸匕,先食,日三四。凡衄亦可用。一方以浆服之。(《集验方》同之。)《僧深方》治热病鼻衄,多者出血一二斛方∶蒲黄(五合)以水和,一饮尽即愈。不瘥别依诸衄方。

又方∶烧牛粪作灰,服方寸匕。

又方∶以冷水洗佳。
治伤寒口干方第三十九

《集验方》治伤寒热病口干喜唾方∶干枣(二十枚,擘)乌梅(十枚,碎)二物,合捣蜜和,含杏核大,咽其汁。
治伤寒唾血方第四十

《范汪方》治热病唾血方∶白茅根一物,捣下筛为散,服方寸匕,日三。亦可绞取汁饮之。
治伤寒吐方第四十一

《集验方》治伤寒吐虚羸欲死方∶鸡子(十四枚)以水五升煮取二升,乃纳豉四合,复煮两三沸,去豉,分再服。
治伤寒哕方第四十二

《病源论》云∶伤寒所以哕者,胃中虚冷故也。

《葛氏方》治伤寒哕不止方∶甘草(三两)橘皮(一升)水五升,煮取一升,顿服之,日三四。

《短剧方》云∶春夏时行寒毒伤于胃,胃冷方∶白茅根(切,一升)橘皮(二两)桂心(二两)葛根(二两)凡四物,以水六升,煮取三升,分三服。

《救急方》云∶天行后干呕若哕手足冷方∶橘皮(四两)生姜(半斤)上,以水七升煮取三升,分四五服,立验。
治伤寒后呕方第四十三

《集验方》治伤寒后干呕不下食,芦根饮方∶生芦根(切,一升)青竹茹(一升)粳米(三合)生姜(二两,切)以水七升,煮取二升,随便饮,不瘥重作。
治伤寒下利方第四十四

《葛氏方》治热病不解而下利困笃欲死方∶大青(四两)甘草(二两)胶(二两)豉(八合)以水一斗,煮取三升,分三服尽。更作,日夜两剂,愈。

又方∶以水煮豉一升,栀子十四枚,葱白一把,取二升,分三服。

又方∶龙骨半斤,捣碎,以水一斗煮取五升,使极冷,饮其间或得汗则愈。
治伤寒饮食劳复方第四十五

《病源论》云∶夫病新瘥,血气尚虚,津液未复,因即劳动,使成病焉。若言语思虑则劳于《医门方》云∶论曰∶凡温病新瘥及重病瘥后,百日内禁食猪肉及肠、血、肥鱼、油腻,必栗、诸果子及坚实难消之物,胃气尚冷,大利难禁,不下之必死,下之后免,不可不慎也。

病新瘥后,但得食粥糜,宁少食令饥,慎勿饱食,不得辄有所食,虽思之勿与。引日转久,可渐食獐、鹿、雉、兔肉等为佳。

疗热病新瘥,早起及多食发复方∶栀子十枚,水二升,煎取一升,去滓,顿服之。温卧令微汗佳。通除诸复。

又方∶烧龟甲,末,服方寸匕。(《葛氏方》同之。)《葛氏方》治笃病新起,早劳及饮食多致复欲死方∶以水服胡粉少少许。

又方∶烧饭箩末,服方寸匕。

《短剧方》治食劳复方∶葛根五两,以水五升煮取二升,冷分三服。

《千金方》治食劳方∶杏仁五枚,酢二升,煮取一升,服之取汗。

又方∶烧人矢灰,水服方寸匕。
治伤寒洗梳劳复方第四十六

《千金方》云∶洗手足复者,饮手足汁一合;梳头复者,吞头垢如枣大者一枚。

《千金方》治或因洗手足或流头劳复方∶取洗手足汁饮一合。

又方∶取头垢如枣大者吞一枚,大者一枚。

《医门方》云∶温病瘥后当静卧,勿早起,自梳头澡洗。但非体劳,亦不可多言语用心使患又云∶若欲令病不发复者,烧头垢如杏仁大服之。
治伤寒交接劳复方第四十七

《医门方》云∶温病新瘥,未满百日,气力未平复,而已房室,无不死者。(今按∶《葛氏方》云∶余劳尚可,女劳多死。)又,疗丈夫热病瘥而交接,发或阴卵肿缩入腹中,绞痛欲死方∶取女人月经赤衣烧末,服方寸匕。

又方∶取女人阴上毛烧,饮之极救急。

《僧深方》云∶妇人时病,毒未除,丈夫因幸之,妇感动气泄,毒即度着丈夫,名阴易病也四枚《葛氏方》云∶男女温病瘥后虽数十日,血脉未和,尚有热毒,与之交接即得病,名曰阴易膝取妇人亲阴上者割取烧末,服方寸匕,日三,小便即利,而阴微肿者为当愈。得童女益又方∶刮青竹茹一斗,以水二升,煮令五六沸,去滓,一服。亦通治诸劳复。

《千金方》治交接劳方∶取所与交妇人衣,覆男子上一食久。《葛氏方》、《医门方》同之。

又方∶取女人手足爪十枚、女人中衣带一尺,烧,以酒亦米汁饮之。
治伤寒病后头痛方第四十八

《千金方》云∶伤寒瘥后更头痛壮热烦闷方∶服黄龙汤五合,日三。(《集验方》同之。)
治伤寒病后不得眠方第四十九

《病源论》云∶大病之后,腑脏尚虚,营卫未生,故成于冷热。阴气虚,卫气独行于阳,不《千金方》温胆(都敢反)汤疗大病后虚烦不得眠,此胆冷故方∶生姜(四两)半夏(三两)枳实(二枚)橘皮(三两)甘草(一两)竹茹(二两)六味,水八升煮取二升,分三服之。

《玉箱要录》云∶大病瘥后,虚烦不得眠,眼暗疼懊(乌浩反)方∶豉(七合)乌梅(十四枚)水四升,先煮梅取二升半,纳豉煮取一升半,分再服,无梅用栀子十四枚。

又云∶千里流水(一石,扬之万过,取二斗)半夏(二两)洗秫米(一升)茯苓(四两)合煮得五升,分五服。
治伤寒病后汗出方第五十

《病源论》云∶大病之后,复为风所乘,则阳发泄,故令虚汗。

《葛氏方》治大病瘥后多虚汗及眠中汗流方∶龙骨牡蛎麻黄根捣末,杂粉以粉身。

《录验方》治大病之后虚汗不可止方∶干姜(三分)粉(三分)冶合以粉,大良。

《短剧方》治大病后虚汗出不禁方∶粢粉豉凡二物,分等,火熬令集,烧故竹扇如掌大,取灰合冶,以绢囊盛,敷体,立止,最验。

当又云∶治大病之后虚汗不可止方∶杜仲牡蛎凡二物,分等,冶之,向暮卧,以水服钱午上。汗止者不可复服,令人干燥。

又云∶治发汗后遂漏汗不止,其人恶风,小便难,四肢微急,难以屈伸,此为胃干也,桂枝大枣(十四枚,擘)桂枝(三两)附子(一枚,碎之,八片)凡三物,以水七升煮取三升,分三服。
治伤寒后目病方第五十一

《葛氏方》治毒病后毒攻目方∶煮蜂巢以洗之,日六七。(今按∶《广利方》云∶蜂窠半大两,水二大升云。又《僧深方》∶治翳。)又方∶冷水渍青布以掩目。(《集验方》治翳。)又云∶若生翳者烧豉二七枚,末纳管中以吹。(《集验方》同之。)《耆婆方》温病后目黄方∶麦门冬叶三握,以水一升煮取三升,去滓,少少饮之,自瘥。

《极要方》疗伤寒病、温毒、热病、时行、疫气诸病之后,毒充眼中,生赤脉、赤膜、白肤秦皮(二两)升麻(二两)黄连(二两)上,以水洗去尘屑,然后以水四升,煮取二升半,分三合,眼仰,以绵绕箸,沾取汤以滴眼
治伤寒后黄胆方第五十二

《葛氏方》治时行病发黄方∶茵陈蒿(六两)大黄(二两)栀子(十二枚)以水一斗,先煮茵陈,取五升,去滓,纳二药。又煮取三升,分四服之。

《千金方》治伤寒热出表发黄胆方∶麻黄三两,以清酒五升,煮得一升半,尽服之,覆取汗。
治伤寒后虚肿方第五十三

《千金方》治病后虚肿方∶豉(五升)淳酒(一升)煮三沸,及热顿服,不耐酒者随性,覆汗。
治伤寒手足肿痛欲脱方第五十四

《千金方》毒热病攻手足肿痛欲脱方∶马矢煮渍之。

又方∶以稻穣灰汁渍之。又方∶常思草绞取汁以渍之。

又方∶削黄柏水煮渍之。

又方∶煮猪蹄取汁,以白葱、盐少少着中以渍之。手足冷则易。

《集验方》治毒热病攻手足,肿,疼痛欲脱方∶浓煮虎杖根以渍手足。

又方∶酒煮苦参渍之。
治伤寒后下利方第五十五

《短剧方》治湿热为毒,及太阳伤寒,外热内虚,热攻肠胃,下黄赤汁及如烂肉汁及赤滞壮栀子(十四枚)豉(一升)薤白(一虎口)凡三物,切,以水四升,煮栀子、薤白令熟,纳豉,煎取二升半,分三服。

《经心方》治热病后赤白利痛不可忍方∶香豉(一升)黄连(三两)薤白(三两)以税米泔汁五升,煮取二升半,分三服。

《千金方》伤寒后下利脓血方∶黄柏(二两)黄连(四两)栀子仁(十四枚)阿胶(一两)上四味,以水五升煮取二升,分三服。

《医门方》疗伤寒瘥后,下利脓水,不能食方∶黄连(三两)乌梅肉(二两,熬,并末之)蜡(一两)烊蜡和蜜合为丸,空腹如梧子服三十丸,日再,加至四五丸。

又方∶取龙骨末,服方寸匕,佳。
治伤寒后下部痒痛方第五十六

《葛氏方》治大孔中猝痒痛如鸟喙方∶赤小豆(一升)大豆(一升)合捣,两囊盛蒸之,令热牙坐上。

《医门方》热病有匿上下食之方∶猪脂(一枚)苦酒(一合)上相和,煎之二三沸,满口蚀缺从下。

又云∶治毒病下部生疮方∶熬盐以深导之,不过三。

又方∶煮桃皮煎如饴,以绵合导之。

《范汪方》治大孔中痒方∶取女萎冶下筛,绵絮裹着大道中,痒绝乃出药。
治伤寒豌豆疮方第五十七

《病源论》云∶夫表虚里实,热毒内盛,则多发疮,重者周匝遍身,其状如火疮。若色赤金又云∶伤寒热毒瓦斯盛,多发疮,色白或赤,发于皮肤,头作瘭浆,戴白脓者,其毒则轻,《千金方》治豌豆疮方∶初觉欲作,则煮大黄五两,服之愈。

又方∶取好蜜,通身疮磨上。

又方∶以蜜煎升麻摩之。

又方∶青木香二两,水三升,煮取一升,顿服瘥。(以上《极要方》同之。)又方∶小豆屑,鸡子白和敷之。

又方∶妇人月布拭之。

又方∶青木香汤∶青木香(二两)丁香(一两)薰陆香(一两)白矾石(一两)麝香(二分)上五味,以水四升,煮取一升半,分再服,热毒盛者加一两犀角,无犀角,升麻代。病轻去《葛氏方》治时行疮方∶以水浓煮升麻,绵沾洗拭之。又苦酒渍煮弥好。

《救急单验方》疗时患遍身初觉出方∶即服三黄汤令利,即灭。

又方∶饮铁浆一小升,立瘥。

又方∶小豆末一合,和水服验。

《新录方》豌豆疮灭瘢方∶鹰矢粉上,若疮干和猪脂涂,日一二。

又方∶胡粉敷上。

又方∶桑白汁和鸡子白涂之。

又方∶用蜜涂之。

《千金方》云∶芒硝和猪胆(都敢反)涂疮上,勿动,痂落无痕。

今按∶天平九年六月二十六日下诸国官府云∶凡是疫病名赤斑疮,初发之时,既似疟疾,疮必令汁温冷任意可用;又糯粳KT以汤食之;又病愈之后虽经二十日不得辄吃鲜鱼、肉、果菜并饮水及洗浴、房室、强行步、当风雨。又鲭及阿迟等鱼并年鱼不可食,但干鳆坚鱼等煎否皆良。
伤寒后食禁第五十八

《养生要集》云∶凡温病伤寒愈后,但宜食糜粥,唯少少,勿食大饱,引日转久,可合羊、又云∶凡伤寒毒病愈百日之内,禁食猪肉、肠、血、肥鱼、腻(女利反)干之难消之物。

不禁《养性志》云∶诸病愈后勿食五辛,食之令人目失明。

《七卷食经》云∶时行病愈,食禁葫、韭、虾、鳝辈,不禁病复发则难治,后年辄发。

时行病后禁饼饵、鱼脍、诸生果菜,难消之物,皆复发病。

时行汗解愈后勿饮冷水,损心胞,常虚不能伏。

时行病患不可食鲤鲔、小鲤及鳝,令病不愈。

又勿食生枣及羊肉,膈上乃为热KT。

凡病患不得食熊肉,令作长病,终不除愈。

时行后禁饮酒、食生鱼肉,令泄利,难治。

时病愈后未满三月,食(以周反)鱼复食诸菜,三年肌肤不充。

又食梅油脂物,令暴利难治。

又食瓜合鱼脍,令病复发。

又食蒜脍,令人损胃消。

又未满三月,食鳝即复病。

时行病后未强食青花菜,令人手足损重。

又饮酒合阴阳复,病必死。

又食生菜合阴阳,复必死。
治伤寒变成百合病方第五十九

《千金方》云∶百合病者,是百脉一宗,悉致病也。其状恶寒而呕者,病在上焦也,(二十十三云,具百合之病,令人欲食复不能食,或有美时,或有不用闻饮及饭臭,如有寒其实无寒,如有热其复无他,常默默欲卧,复不能眠,至朝口苦,小便赤涩,欲行复不能行也,诸药不治,治百合病(本无,同上),其脉微数,其候每溺时即头觉痛者,六十日乃愈。

百合病(本无,同上),候之溺时,头不觉痛者,淅淅如寒者,四十日愈。

百合病(本无,同上),候之溺时,觉快然,但觉头眩者,二十日当愈。

百合病(本无,同上)证,或其人未病已预见其候者,或已病一月二十日,复见其候者,治之治百合病已经发汗之后者方∶百合根取七枚擘之,洗,水二升,渍之一宿,当沫出水中,明旦去水取百合,以泉水二升煮一升治百合病已经下之后者方∶滑石(三两)代赭(一两)以水三升煮取一升,纳百合汁如前法一升合和,复煎取一升半,分再服。

治百合病已经吐之后者方∶百合汁一升如前法,取鸡子黄一枚,纳汁中搅令调,分再服。

治百合病始不经发汗,不吐,不下,其病如初者方∶生地黄汁三升,和百合汁后煎取一升半,分再服。大盒饭去恶沫为候也。

治百合病经一月不解变如渴者方∶取百合根一升,水一斛渍之一宿,以汁洗病患身也。洗身竟,食白汤饼,勿与盐豉也。

渴不治百合病变发热者方∶滑石(三两)百合根(一两)上,燥之,饮服方寸匕,日三。当微利,利者止,勿复服也。

治百合病变腹中满痛者方∶但取百合根随多少,热熬令黄色,饮服方寸匕,日三。满消痛止。
治时行后变成疟方第六十

《录验方》云∶大五补汤治时行后变成疟方∶枸杞白皮(一斤)麦门冬(一升,去心)生姜(一斤)干地黄(三两)当归(三两)黄(三两)人参(三两)甘草(三两)茯苓(三两)生竹叶(五两)远志皮(三两)术(三两)芎(二两)桂心(五两)大枣(二十枚)桔梗(二两)夕药(三两)半夏(二两,洗)凡十八物,切,以水一斗五升煮取三升,分四服,一日令尽之。
卷第十五
说痈疽所由第一

《刘涓子方》云∶九江黄父问于岐伯曰∶余闻肠胃受谷,上焦出气,以温分肉,而养骨节,通腠理。中焦出气如露,上注谷,而渗孙脉。津液和调,变化而赤为血,血和则孙脉先满,乃注络脉,络脉皆盈,注乃于经脉。阴阳已张,因息乃行,行有经纪,周有道理。与天协议,不得休止。切而调之,从虚法去实,泻则不足,疾则气留。去虚补实,补则有余,血气已调,形神乃持。余已知血所之平与不平,未知痈疽之所从生,成败之时,死生之期有远近。何以度之,可得闻乎?岐伯曰∶经脉留行不止,与天同度,与地合纪,故天宿失度,日月薄蚀,地经失纪,水道流溢,草不成,五谷不植,经络不通,民不往来,巷聚邑居,别离异处,血气犹然,请言其故。

夫血脉营卫,周流不休,上应星宿,下应经数。寒气客于经络之中,则血涩,血涩则不通,不通则气归之,不得复返,故痈肿焉。与寒气化为热,热胜则肉腐,肉腐则不脓,脓不泻则烂筋,筋烂则伤骨,骨伤则髓消,不当骨空,不得泄泻,煎枯空虚,筋骨肌肉不相亲,经脉败漏,内熏(熟也)于五脏,五脏伤故死矣。

又云∶黄父曰∶夫子言痈疽,何以别之?岐伯答曰∶营卫稽留于脉,久则血涩而不行,血涩不行则卫气从之,从之不通,壅遏(于葛反,绝也)不得行也。大热不止,热胜则肉腐,肉腐为脓。犹不能陷肌肤,枯于骨髓,骨髓不为焦枯,五脏不为伤,故为痈。

黄父曰∶何谓疽?岐伯曰∶热气淳盛,当其下筋骨,良肉无余,故命曰疽。疽上皮咬以坚,状如牛领之皮,痈者,其上皮薄以泽,此其候也。

黄父曰∶乃知所说,未知痈疽姓名、发起处所、色诊形候、治与不治、死活之期,愿事事闻之。

岐伯曰∶《痈疽图》曰∶赤疽发额,不泻,十余日死,其五日可刺也。其脓赤多血死,未有脓可治。人年二十五、三十一、六十、九十五,百神皆在额,不可见血,见血者死也。

禽疽发如疹者数十处,其四日肿,合牢核痛,其状若挛,十日可刺。其内发身振寒,齿如噤欲痉,如是者,十日死也。

杼(除吕反)疽发项,若两耳下,不泻,十六死,其六日可刺。其色黑见脓而腐者死,不可治。人年十九、二十三、三十、三十五、三十九、五十一、五十五、六十一、八十七、九十九,神在两耳,不可见血,见血者死。

疔疽发两肩,此起有所逐恶血,结留内外,营卫不通,发为疔疽。三日身肿痛,甚口噤如状,十一日一日可刺。不治,二十日死。

蜂疽发背,起心俞,若肩隅,二十日不泻死,其八日可刺。其色赤黑,脓见青死,不治。

人年六岁、十八、二十四、四十、五十六、六十七、七十二、九十八,神皆在肩,不可见血,见血者死。

阴疽发髀(卑履反)腹外,若阴股,始发腰强而不能自止,数饮不能多,五日坚痛,如此不过三岁死。

刺疽起肺俞,若肝俞,不泻,二十日死,其八日可刺。发而赤,其上肉如椒子者死,不可治。人年十九、二十五、三十三,三十九、五十七、六十、七十三、八十一、九十七,神皆在背,不可见血,见血者死。

脉疽发环头(一方作颈),如痛,身随而热,不欲动,或不能食,此有所大畏恐,躁而不精(靖),上气咳逆,气绝,其发引耳,不可以动,二十日可刺。不刺,八十日死。

龙疽发背起胃俞(第十二椎下两旁各一寸半),若肾俞(第十四椎下社各一寸半),二十日不泻死,九日可刺。在刺,其上赤下黑,若青脓黑死,发血脓者不死。

首疽发热,发热八十日,大热汗头引身尽,如癞身热,同同(齐也)如沸者,择皮颇肿处浅刺之。不刺,入腹中二十日死。

侠(胡颊反)荣疽发胁,若起两肘头,二十五日不泻死,其九日可刺。发赤白间,其脓多白而无赤可治也。人年一,十六、二十六、三十二、四十八、五十八、六十四、八十、九十六,神皆在胁,不可见血,见血者死。

行疽发如肿,或复合相往来,可要追其所在,刺之即愈。

勇疽发股起太阴,若伏鼠,二十五日不泻死,其十日可刺。勇疽发脓赤,黑死,白者尚可治。人年十一、十五、二十、三十一、三十三、四十六、五十九、六十三、七十五、九十一,神皆在尻尾,不可见血,见血者死。

标叔疽发热同同,耳聋。后六十日肿如裹水状,如此可刺之。但出水后及有血,血出即除也。人年五十七、六十五、七十三、八十一、九十七,神皆在背,不可见血,见血者死。

KT(□先反)疽发足趺若足下,三十日不泻死,其十二日可刺。KT疽,发赤白脓而不反(死),大多其上白痒,赤黑脓死不可治,不黑可治。人年十三、二十九、三十五、六十一、七十三、九十三,神皆在足,不可见血,见血者死。

冲疽发少腹,痛而振寒热。四日可刺,五日,六日变而可刺之,不刺之,五十日死。

敦(熟,殊六反)疽发两手五指头若足五趾头,十八日不泻死,其四日可刺。其发日而黑,痈不甚,未过节可治也。

疥疽发腋下,若臂、两掌中,振寒热而嗌干者,饮多即呕,烦心,或六十日轸(软欤)及有合者,如此可汗,不汗,入腹内死。

筋疽发背侠脊两边大筋,其色苍,八日可刺也。有痈在肥(胞)肠(腹)中,九十日死。

陈干疽发两臂,三四日痛不可动,五十日身热而赤,六十日可刺之。如刺肺无血,三四日病已,无脓者死。

蚤疽发手足五指头起,过节其色不变,十日之内可刺也,过时不刺后为蚀(饮)。有痈在腋,三岁死。

叔疽发身肿,牢核而身热,不可以坐,不可以行,不可以屈伸,成脓刺之即已除。

白疽发膊若肘后,痒,目痛,伤精,乃身热、多汗,五六处死。心主疽痈在股胫六日死,发脓血六十日死。

黑疽发肿,居背大骨上,八日可刺也,过时不刺为骨疽。骨疽脓出不可止,出碎骨,六十日死。

胁少阳有肿痈,在颈八日死,发血脓十日死。

创疽发,身先痒后痛。此故伤寒寒气入脏,笃发为创疽。九日可刺,不刺九十日死。

尻腰太阳脉有肿,交脉属于阳明。痈在颈十日死,发脓血七十日死。

尻太阳脉有肿痈,在足心阳明、少阳,八日死,发脓血八十日死。

头阳明脉肿痈,在尻六日死,发脓血六十日死。

股太阳脉有肿痈,在足太阳十七日死,发脓血百日死。

肩太阳太阴脉在肿痈,在胫八日死,发脓血四百日死。

足少阳脉有肿痈,在胁八日死,发脓血六百日死。

手阳明脉有肿痈,在腋渊一岁死,发脓血二岁死。

黑疽发腋渊死;黑疽发耳中如米,此大疽死;黑疽发肩不死可治;黑疽缺盆中,名曰伏痈死,不治;黑疽发胸可治;黑疽发肘上下,不死可治;赤疽发阴股,坚死,濡可治;赤疽发髀枢(昌朱反),六日可治,不治出岁死;赤疽掌中可治;髀解际,指本黑、头赤死;赤疽发阴死不治;黑疽发肥肠死;黑疽发膝膑,坚死,濡可治;黑疽发趺上,坚死;足下久痈,色赤死,不可治。

又云∶夫痈疽者,初发始微,多不为急,此实奇患,惟宜速疗之,疗之不若速,病成难救,以此致祸,能不痛哉?且述所怀,以悟后贤。谨条黄父痈疽论,论痈所着,缓急之所,死生之期,期如有别痈之形色、难易之疗如下∶发皮肉,浅肿高而赤,贴即消,不疗先愈。

发筋肉,深肿下而坚,其色或青、黄、白、黑,或复微热而赤,宜急疗之,成消中半。

发附骨者,或未觉内肉,内肉已殃,已殃者痈疽之甚者也。

发背外,皮薄为痈,皮坚为疽。如此者多现先兆,宜急疗皮坚甚大者,多致祸矣。

《太素经》云∶黄帝曰∶愿尽闻痈疽之形与忌日名。岐伯曰∶痈发于嗌中,名曰猛疽。

猛疽不治,化为脓,脓不泻,塞咽半日死,其化为脓者,泻则已,已则含豕膏,毋冷食,三日而已。(今按∶《刘涓子方》痈疽极者十八种云则是也。)发于颈名曰夭疽。其痈大以赤黑,不急治则热气下入渊腋,前伤任脉。内熏(热也)肝肺,熏肝肺十余日而死矣。

阳气大发,消脑留项,名曰脑铄(书药反)。其色不乐,项痛如刺以针,烦心者死,不治。

发于肩及,名曰疵痈。其状赤黑,急治之。此今人汗出至足,不害五脏。痈发四五,逆之。(今按∶《刘涓子方》"痈发四五逆之"此七字无。)发于腋下赤坚,名曰朱疽。治之用砭石,欲细而长,数砭之,涂以豕膏,六日已,勿裹之。其痈坚而不溃者,为马刀侠婴,急治之。(今按∶《刘涓子方》作"鼠膏"。)发于胸,名曰井疽。其状如大豆,三四日起,不早治下入腹,不治,七日死。

发于臆者,名曰甘疽。其状如谷实蒌瓜,常寒热,急治之,去其寒热。不治,十岁死,死后脓自出。

发于胁,名曰败疵。败疵者,女之病也,久之,其病大痈脓,其中乃有生肉,大如赤小豆,治之,锉连翘草根各一升,水一斗六升,煮之,竭为三升,即强饮,浓衣坐釜上,令汗出至足已。

发于股胫,名曰股胫疽。其状不甚变,而痈脓附骨。不急治,四十日死。

发于股,名曰脱疽。其状不甚变,而痈脓搏骨,不急治,三十日死矣。

发于尻,名曰兑疽。其状赤坚大,急治之。不治,四十日死矣。

发于股(故户反)阴,名曰赤绝。不急治,六日死。在两股之内,不治,六十日而死。

发于膝,名曰疵疽。其状大痈,色不变,寒热而坚,勿破,破之死。须其柔,乃破之者生。

诸痈疽之发于节而相应者,不可治也。发于阳(丈夫阴器曰阳,妇人阴器曰阴)者百日死,发于阴者四十日死。

发于胫,名曰兔齿。其状赤至骨,急治,不治害人也。

发于踝,名曰走缓。其状色不变,数石其输而止其寒,不死。

发于足上下(足跌上下也),名曰四淫。其状大痈,不色变,不治百日死。

发于足旁,名曰厉疽。其状不大,初如小指发,急治之。去其黑者,不消辄益。不治,百日死。

发于足指,名曰脱疽。其状赤黑死不治,不赤黑不死。治之不衰,急斩去之活,不然则死矣。

又云∶身有五部∶伏菟一;腓二,腓者,(都馆反)也;背三;五脏之输四;项五。

五部有痈疽者死。

又云∶杨上善曰∶痈生所由,凡有四物也∶喜怒无度,争气聚生痈,一也;饮食不依节度,纵情不择寒温为痈,二也;脏阴气虚,腑阳气实,阳气实盛生痈也,三也;邪客于血,聚而不行,生痈,四也。痈疽一也,痈之久者,败骨名疽也。

又云∶夫积石成山,积气成痈,不从天下,不从地出,皆由不去脆(青碎反)微故也。

《病源论》云∶痈者由六腑不和所生也;疽者,五脏不调所生也。

凡肿一寸至二寸,疖也;二寸至五寸,痈也;五寸至一尺,痈疽也。一尺至三尺者,竟体脓,脓成,九孔皆出。诸气愤郁,不遂志欲者,血气蓄积,多发此疾。

凡发肿高者,疹(耻忍反)源浅。肿下者,疹源深。大热者易治,小热者难治。

凡五月勿食不成核果及桃、枣,发痈疽疖。

凡人汗入诸食中,食之则作疔疮、痈疖。

凡铜器盖食,汗入食,食之令发恶疮内疽。

凡鲫鱼脍合猪肝肺,食发疽。

凡醉,强饱食,不幸发疽。

《医门方》云∶凡痈肿发于背,欲得高,高即肿浮,浅在外,纵结脓者,亦多瘥;肿不高,沉在肉里者,其肿深,脓溃向内必死。

《范汪方》云∶经言∶五脏不调,致疽;六腑不和,生痈。疽急者,有十∶一曰瘭疽者,急者二三日杀人,缓者十余日;二曰痈疽,急者十余日杀人,缓者一月;三曰缓疽,急者一年杀人,缓者数年;四曰水疽,所发多在手足,数年犹可治。疽者有数十种,要如此。痈之疾,所发缓地不杀人,所发若在险地,宜令即消,若至小脓,犹可治,至大脓者致祸矣。一为脑(乃道反)户,在玉枕下一寸;二为舌本;三为悬壅;四为颈节;五为胡脉;六为五脏俞;七为五系;八为两乳;九为心鸠尾;十为两手鱼际;十一为肠屈之间;十二为小道之后;十三为九孔;十四为两胁腹;十五为神主之舍。凡四十五处不可伤,而况于疽乎?若痈发此地,遇良医,能令不及大脓者,可救,至大脓者害及矣。

痈疽脉洪粗难治,脉微涩者易愈。诸浮数之脉,应当发热,而反恶寒,痈也。

痈起于节解过,遇顽医不能即消,令至大脓者,岂膏药可得生复?
治痈疽未脓方第二

《医门方》云∶扁鹊曰∶痈肿疖疽风肿恶毒肿等,当其头上灸之数千壮,无不瘥者;四畔亦灸三二百壮。此是医家秘法。小者灸五六处,大者灸七八处。疗痈疽肿一二日未成脓,取伏龙肝下筛,醋和如泥,涂烂布上,贴肿,燥即易,无不消。(今按∶《葛氏方》和鸡子中黄涂。)《刘涓子方》云∶痈疽之甚,未发之兆,饥渴为始,始发之始,或发白疽似若小疖,或复大痛,皆是微候,宜善察之。欲知是非,重按其处,是便隐痛,复按四边,比方得失。审定之后即灸,第一便灸其上二三百壮,又灸四边一二百壮。少者灸四边,中者灸六处,大者灸八处,壮数、处所不患多也。灸应即贴即薄,令得即消,内服补暖汤、散。不已,服冷导外即冷薄。不已,用热薄帖贴之法,当开其口,泄热气也。

治痈疽始作肿不赤而热长甚速、非薄帖所制犀角拓汤方∶犀角(三两)大黄(三两)升麻(三两)黄芩(三两)栀子(三两)黄连(三两)甘草(三两)上七物,切,以水一斗二升煮取六升,使极冷,以故练两重入汤中,拓肿处,小燥易,恒令湿,一日,一夜数百过。

治痈肿黄帖治痈肿瘰及欲发背觉痛方∶黄(一两)黄芩(一两)芎(一两)当归(一两)黄连(一两)白蔹(一两)夕药(一两)防风(一两)上八物,捣筛,以鸡子白和涂故布上,以贴肿上,燥复易。患热者加白蔹,患痛者加当归各一两。

《千金方》云∶凡痈疽始发,或从小疖,或复大痛,或复小痛,或发米粒大白脓子,此皆微候,宜善察之。见有少异,即须大惊忙,急治之。及断口味,速服诸汤,下去热毒。若无医药处,即灸当头百炷,其大重者,灸四面及头二三百壮,壮数不必多也,复薄冷药帖。

种种救疗,必瘥速也。亦当头以大针,针入四分即瘥。

凡诸异肿种种不同者,无问久近,皆服五香汤,刺去血,小豆薄敷之,其间数数以针刺去血。若失治已溃烂者,犹服五香及漏芦汤下之,以升麻汤拓洗熨之,摩升麻膏。若生息肉者,白芦茹散敷之,青黑肉去尽即停之。好肉生,敷升麻膏;肌不生,敷一物黄散。若敷白芦茹,青黑恶肉不尽者可以柒头、茹半钱和三钱白茹散稍敷。(今按∶五香汤在丁肿方,漏芦汤、升麻汤、升麻膏在治丹之方;黄散、茹散在缓疽之方。)内消散凡是痈疽皆宜服之,方∶赤小豆(一升,熬,纳酢中,往还七返)人参(二两)甘草(二两)瞿麦(二两)白蔹(三两)当归(二两)黄芩(二两)猪苓(二两)防风(一两)黄(三两)薏苡仁(三两)升麻(四两)十二味,酒服方寸匕,日三夜二,长服取瘥。(今按∶《广济方》同之。但《范汪方》云∶内消散治痈肿不溃∶白芷十分白薇十分芎七分夕药十分椒七合干姜七分当归七分草七分凡八物,冶合,以酒服五分匕,日再。

又《令李方》内消散治痈肿不溃方用∶白芷十分夕药十分蜀椒七合芒硝十分芎十分当归七分干姜七分七物,下筛,酒服五分匕,日二。一方有白蔹一分。)治痈肿痛烦困方∶生楸叶十重贴之,以布帛裹,缓勿令急,日二易。止痛消肿,食脓胜于众帖,冬以先干者,临时盐汤泼润用之。(《葛氏方》、《刘涓子方》同之。)广济犀角丸治一切毒肿痈乳发背,服之止痛化脓为水,遂大小便出神效方∶犀角(十二分)升麻(一两)大黄(五分)黄芩(五分)防风(一两)当归(一分)黄栀子仁干姜黄连人参甘草(各一两)巴豆(二十三枚,去心、皮,熬)凡十三物,冶下筛,以蜜和丸如梧子大,空腹以饮服十丸,取二三行快利,常服为微痢。

(忌生菜、热曲、酢蒜、海藻、猪肉、笋、粘食。)《效验方》治卒痈肿白蔹帖方∶大黄(三分)黄芩(三分)夕药(二分)白蔹(三分)赤石脂(一分)凡五物,冶合下筛,以鸡子白和如泥,涂纸以薄肿上,燥易之。

《录验方》治痈肿运赤痛及已溃松脂帖方∶成练松脂(一斤)蜡蜜(半斤)猪脂(四斤)当归(二两)黄连(一两)黄柏(一两)凡六物,咀三物,尽合,煎三沸三下,候帖色变微紫色者药成,绞去滓。若初肿未有脓者涂纸贴肿上,日三易,夜再。若以溃有口者,穿纸出疮口,贴四边,令脓聚食旁肉速瘥。

(今按∶诸方松脂帖已多,但《刘涓子方》松脂二斤,黄连一两,附子一两,黄芩一两,夕药一两,细辛一两,石膏二两也。又《效验方》杏仁一两,腊蜜一两,松脂一两,浓朴一两也。)(今按∶以下宇治本注也。)千疮万病霜膏治金木疮,痈疽恶疮,若始得痈,摩肿上五百过即消。若以溃者着疮中食脓云云,具在本方。

《短剧方》云∶经言寒气客于经络之中,则血气凝(李亢反)涩不行,壅结则为痈疽也。

不言热之所作。其成痈,久寒化为热,热盛则肉腐烂为脓也。依经诊候之由,人体中有热,被寒冷搏之,血脉凝涩不行,热气壅结则为痈疽也。是以治痈疽方,有灸法者治其始,其始中寒未成热时也。其用冷薄帖者治其热已成,以消热,使不成脓也。今人多不悟其始,不用温治及灸法也。今出要方以治其成形者耳。赤色肿有尖头者根广一寸,以还名为疖,其广一寸以上者便为小痈也,其如豆粒大为。

治作痈令消方∶取鹿角就磨刀石上水摩之,以汁涂燥,复涂则消也;内宜服连翘汤下。

又方∶生舂小豆下筛,鸡子白和如泥,涂之。

治始作痈正赤热痛方∶单烧鹿角作末,苦酒和薄之,干复涂之自消。

又方∶单捣大黄,苦酒和薄之,温则易。

治痈及疖始结肿赤热方∶水摩半夏,涂之燥,更涂得皱,使消也。山草中自可掘取生半夏乃佳,神效。

《葛氏方》治诸痈疽背及乳房初起,赤急痛,不早治杀人,使速消方∶但灸其上百壮。

又方∶釜底土,捣,以鸡子中黄和涂之。

又方∶捣苎根薄之。

又方∶捣黄柏下筛,以鸡子白和,浓涂,干复易,立愈。

治痈发背腹阴处通身有数十方∶取干牛矢,烧,捣细,重绢筛下,以鸡子白和涂,干复易。秘方。

《范汪方》治痈肿初肿痛急方∶以冷铁熨温辄易。

又方∶取粢粉,熬令正黑,末作屑,以鸡子白和之。以涂练上敷肿上,小穿练上作小口以泄气,痈毒便消。当数易之,此药神秘方。(今按∶《葛氏方》∶治诸痈疽发背及乳房初起赤急痛,使速消。《尔雅》云菜□也。和名∶支美乃毛知。)《本草拾遗》云∶水蛭(音质),人患赤白游疹及痈肿毒肿,取十余枚令KT病处,取人皮皱肉白,无不瘥者。冷月无蛭,泥中掘取暖汤养令动,先洗人皮,咸以竹筒盛啜之,须臾咬血满,自脱,更用饥者。(今按∶《经心方》云∶以水蛭食去恶血。)《僧深方》治痈方∶梁上尘、烧葵末分等,苦酒和敷之,燥复敷。治乳痈亦愈。
治痈疽有脓方第三

《病源论》云∶凡痈若按之都牢坚者,未有脓也;按之半软者,有脓也。又,以手掩肿上,不热者无脓。若热甚者,为有脓。凡觉有脓,宜急破之,不尔,侵食筋骨。

《刘涓子》云∶痈大坚者,未有脓。半坚半薄半有脓。当上薄者,都有脓,便可破。可破之法,应在下逆上破之,令脓易出,用铍(披眉反)针。脓深难见,上肉浓而生内大针。

若外不别有脓,可当其上数按之,内便隐痛者,肉殃坚者,未有脓也。按更不痛于前者,内脓已熟也。脓泄去热气,不尔长连,连则不良。

痈审知有脓者,按之处陷不复者无脓,按之即复者有脓;初肿大,按乃痛者病深;小按便痛者病浅也。凡破痈之后,病患便绵欲死,便内寒热,肿自有似痈而非者,当以手按肿上,无所连是风毒耳,勿针,可服升麻汤、升麻膏,破痈,口当令下流三分,近下一分,令针极热,极热便不痛。

夫痈坏后有恶肉者,当以猪蹄汤洗去其秽,次敷食肉膏、散,恶肉尽,乃敷生肉膏、散,乃摩四边,令善肉速生。当绝房室,慎风冷,勿自劳动,须筋脉复常,乃可自劳耳。不尔,新肉易伤,伤则重发,慎之。

治痈疽洗疮猪蹄汤方∶猪蹄(一具)黄连(五两)芎(三两)当归(三两)甘草(三两)夕药(三两)蔷薇(一斤,又方代大黄三两)上七物,以水二斗,煮蹄取一斗,纳诸药复四升洗之。

治痈疽食恶肉芦茹散方∶芦茹(一两,柒头)矾石(二分)雄黄(二分)硫黄(二分)上四物,下筛,着兑头纳疮口中。(今按∶《录验方》∶雄黄一两,矾石一两,芦茹一两也。)治痈疽发背已溃未溃生肉排脓散方∶当归(二两)桂心(二两)人参(二两)芎(一两)浓朴(一两)防风(一两)甘草(一两)白芷(一两)桔梗(一两)上九物,捣下筛。温酒服方寸匕,日三夜再。疮未合可长服之。

(今按∶《僧深方》治痈肿自溃长肉排脓蜀椒散方∶蜀椒桂心甘草干姜芎当归各一两凡六物,服法如上。

又方∶治痈肿排脓散方∶黄四分夕药二分白蔹二分芎二分赤小豆一分凡五物,冶下,服如上。)治痈疽食恶肉膏方∶松脂(五两)雄黄(二两)雌黄(二两)冶葛皮(一两)柒头芦茹(三两)巴豆(百枚)猪膏(一升)上七物,煎,松脂消下诸药,微火上煎三上三下,膏成绞去滓,着兑头,纳疮中,日六七,食恶肉,初用病当更肿赤,但用如节度,恶肉尽止,勿使过也。

治痈疽发背已溃生肉膏方∶甘草(二两)当归(二两)白芷(二两)乌喙(六枚)肉苁蓉(二两)蜀椒(二两)蛇衔(一两)细辛(二两)薤白(二两)干地黄(二两)上十物,切,以好酒半升和渍再宿,以不中水猪肪三斤,煎一沸,下余止,复三上三下,膏成,急手绞之。

(今按∶《录验方》治痈疽生肉膏方∶草二两,生地黄五两,当归二两,续断一两,黄芩二两,白芷三两,甘草二两,薤白二两,猪脂一升,大黄四两。凡十物,咀,煎三上三下,膏成,敷之。)治痈疽臭烂洗大黄汤方∶大黄(二两)黄芩(一两)白蔹(一两)上三物,合捣下筛,以水一升二合煮一沸,绞去滓,适冷暖以洗疮,日十过。

《范汪方》治痈疮热已退,脓血不止,疮中空虚,疼痛排脓,内塞散方∶防风(一两)茯苓(一两)白芷(一两)桔梗(一两)远志(一两)甘草(一两)桂心(二分)人参(一两)芎(一两)当归(一两)附子(二枚,炮)浓朴(二两)龙骨(一两)黄(一两)赤小豆(五合,熬)凡十五物,冶下筛,温酒服方寸匕,日三夜一。(《千金方》同之,有下发背条。)《集验方》治痈疮脓血不止,疮中空虚疼痛,排脓内补散方∶防风(一两)远志(一两)当归(二两)黄(一两)白芷(一两)甘草(一两)桔梗(一两)通草(一两)浓朴(二两)人参(一两)桂心(一两)附子(一两)赤小豆(五合,熬)芎(一两)茯苓(二两)凡十五物,冶合筛,食未温酒服方寸匕,日三夜一。(今按∶《广济方》同之。)《令李方》治痈内补排脓散方∶黄(二两)当归(二两)赤小豆(三十枚)芎(一两)夕药(二两)大黄(一两)凡六物,冶合下筛,以粥清服方寸匕,日三。

又云∶治痈肿桂心散方∶黄(六分)夕药(四分)桂心(一分)凡三物,冶下筛。酒服方寸匕,日三。

《录验方》洗痈疽并恶疮毒瓦斯猪蹄汤方∶当归(四两)甘草(四两,炙)夕药(五两)芎(二两)白芷(四两)草(二两)黄芩(四两)野狼牙(四两)猪蹄(一具)蔷薇根(一两)大黄(四两)凡十一物,先以水二升半,别煮猪蹄取一升半,去蹄纳诸药,煮得再沸下桑灰汁一升,又煮取一升半,汤成稍稍以洗疮痈结疽。初肿时去野狼牙纳灰汁;疮既(所教反,断已)溃,用野狼牙除灰汁。

《僧深方》治痈疽疮臭烂洗疮青木香汤方∶青木香(一两)夕药(一两)白蔹(一两)芎(一两)凡四物,水四升,煮取二升,去滓温洗疮,日三,明日以膏纳疮中,日三。

《医门方》疮痈肿已脓,惧针令脓自出方∶上取鹿角刮取细末,和醋聚安肿上,经宿脓自出。若脓深者先嚼生栗敷之,撮脓向上,极妙。

疗痈肿不消,已有脓不能针,自令穿溃方∶空腹服葵子一枚,一宿即穿出脓,神妙。

《葛氏方》治痈已有脓,当使脓速溃坏方∶雀屎以苦酒和,涂上如小豆。

又方∶吞薏苡子一枚,勿多。

《新修本草》云∶痈脓使速溃方∶吞()实一枚,破痈肿。

《苏敬本草》注云∶吞恶实一枚,出痈疽头。

《千金方》治痈溃后疮不合方∶烧鼠皮一枚,作灰封孔中。

又方∶涂牛屎,干易。

又方∶烧故蒲席灰,和猪脂,纳孔中。

《救急单验方》疗头痈因即骨陷方∶先烧杏仁令黑,磨涂后取枣木、紫葛蔓及干鱼烧灰,和薰黄蜡月猪脂涂,神验。
治痈发背方第四

《病源论》云∶痈发背,多发于诸腑俞也。六腑不和则生痈,诸腑俞皆在背,其血气结络于身,六腑气不和。腠理虚者,经络为寒所客,寒折于血,则壅不通,故结成痈,发其俞也。

热气加于血,则肉血败,化而为脓。痈初结之状,肿而皮薄以泽。背上忽有赤肿而头白,摇之连根,入应胸里动,是痈也。

发背若热,手不可近者,内先服王不留行散,外摩发背膏、大黄帖。若在背生,破无苦,良久不得脓,以食肉膏、散着兑头,纳痈口中。人体热气歇,服术散。五日后痈欲瘥者,服排脓内塞散。

《千金方》云∶论曰∶凡发背皆由服五石、寒食、更生散所致,亦有单服钟乳而发者,又有生平不服石而自发者,此是上世有服之者。其候率多于背两胛间起,初如粟米大,或痛或痒,仍作赤色,人皆初不以为事,日渐长大,不过十日,遂至不救。其临困时,方圆径三四寸,疮有数十孔,以手按之,诸孔之中脓皆乃出,寻即失音不言。

所以养生者,小觉背上痛痒有异,即取净土水和作泥,捻作饼子,径一寸半,浓二分,以粗艾作炷,灸泥上灸之,一炷一易。若粟米大时,可灸七饼即瘥。若如榆英大,灸七七炷即瘥。若至钱许大,日夜灸不住乃瘥,并服五香连翘汤及铁浆诸药攻之乃愈。又恒冷水射之,渍冷石熨之,日夜勿止,待瘥住手。此病忌面、酒、肉、五辛等。

凡肿起于背胛中,头白如黍粟,四边相连肿,赤黑,令人闷乱者,名发背。不灸治即入内,灸当针疮上七八百壮。

又方∶饮铁浆三升,下利为度。

又方∶鹿角灰,酢和涂之。

排(步皆反)脓内塞散主大疮热已退,脓血不止,疮中肉虚疮痛方∶防风(一两)茯苓(一两)白芷(一两)桔梗(一两)远志(一两)甘草(一两)桂心(二分)人参(一两)芎(一两)当归(一两)附子(二枚)浓朴(二两)龙骨(一两)黄(一两)赤小豆(五合,熬)十五味,为散,酒服方寸匕,日三夜一。(今按∶《范汪方》云∶非酒则药势不宣。)《广济方》疗钟乳及五石等发背毒热方∶黄芩(三两)白鸭屎(五合)白蔹(一握)香豉(五合)切,水六升煮取二升,分温三服。

《经心方》疗发背方∶以冷石熨肿上,验。

又方∶马粪敷,干易之,妇人发乳亦瘥。

《范汪方》治发背及诸痈肿已溃未溃方∶捣豉小和水,令如强泥作饼,饼可肿大,浓三分所,若有疮孔空遗之,勿覆,令汁得出,得以艾罗灸豉上欲燥,若热则易,令大热剥烂皮也。痈痛寻当转减,便得安。为灸或有一日、二日、三日,疮孔中当汁出。(《千金方》同之。)治痈肿王不留行散方∶王不留行〔二升(成末)〕甘草(五两)冶葛(二两)桂心(四分)当归(四两)凡五物,冶合下筛,以酒服方寸匕,日三夜一。

《葛氏方》治痈发背腹阴匿处通身有数十方∶取干牛屎烧,捣细,重绢筛下,以鸡子白和以涂之,干复易。秘方。《刘涓子方》同之。

又方∶用鹿角、桂心、鸡矢,当别烧,合之捣,以鸡子白和,涂之。秘方。

又方∶生栝楼根细捣,以苦酒和,涂上,干复易之。

又方∶赤小豆涂之亦良。

《刘涓子》治痈发背发房初起赤方∶其上赤处灸百壮。

又方∶捣苎根,少水解以薄上。

又方∶灶黄土、鸡子白和,涂上。

又方∶树上不落桃子,末,以好苦酒和,敷上良。

又云∶欲使速溃方∶水研半夏,鸡子白和涂之,亦能令消。
治附骨疽方第五

《病源论》云∶附骨疽者,由体热当风入骨解,风与热相搏,复遇凉湿;或秋夏露卧,为冷湿所折,风热伏结,壅遏附骨成疽。喜着大节解间,丈夫及产妇、女人,喜着鼠膝、髂头、髀膝间,婴孩、嫩儿亦着膊肘、背也。其大老子着急者,则先觉痛,不得转动,按之应骨痛,经日便觉皮肉微急,洪洪如肥状,则是也。其小儿不知字名,抱之才近其身便啼唤,则是肢节有痛处,便是其候也。大老子着缓者,则先觉如肥洪洪耳,经日便觉痹痛不随也。

其小儿则觉四肢偏有不动摇者,如不随状,看肢节解中,则有肥洪洪处,其若不知是附骨疽,乃至称身成脓,不溃至死,皆举体变青黯也,其大老子,亦有不别是附骨疽,呼急者为贼风,其缓者谓为风肿而已。皆不悟是疽,乃至于死。

《短剧方》云∶附骨疽一名(许及反)疽,其无头附骨成脓故也。又名痈疽。以其广大,竟体有脓故也。附骨急疽与贼风实相似也,其附骨疽者,由人体盛有热,久当风冷,入骨解中,风与热相搏。其始候为欲眠沉重、惚惚(音忽,恍恍也)耳,急者热多风少,缓者风多热少也;贼风者其人体平,无热,中暴风冷则骨解深痛。附骨疽久者则肿见结脓。贼风久则枯消,或结瘰。以此为异也。是附骨疽而作贼风治,则益其病深,脓多也;是贼风而作附骨疽治,则加其冷风,增遂成瘰偏枯挛曲之疾也。

附骨急疽者,其痛处壮热,体中乍寒乍热,痿痿恶寒,不用热,小便或赤,大便或难,无汗也,即得治下去热,便得消也。纵不消尽,亦得浮浅近,外易得坏溃,其不复附骨也。

贼风之证,但痛应骨,不可按仰痛处,不壮热,体不仁,乍寒乍热,但觉体KT以冷,砍得热,热熨痛处即小宽,时有汗也,即得针、灸、熨,服治风温药便效也。初得附骨疽即服漏芦汤下之,敷小豆薄得消也;下利利已虚,而肿处未消者,可除大黄,用生地黄及干地黄也;热渐退,余风未歇者,可服五香连翘汤,除在黄也;余热未消可敷升麻膏佳;若失时不消成脓者,用火针膏散如治痈法也。(今按∶漏芦汤、升麻膏在第十七卷治丹之方;五香翘汤在第十六卷治恶核肿方,皆出《短剧方》。)《千金方》云∶候附骨与贼风为异者,附骨之始未肿但痛而已,其贼风亦痛不热,附骨则其上壮热。

凡初得附骨疽即须急服漏芦汤下之。敷小豆散得消,可服五香连翘汤。

治骨疽方∶又方∶末芜菁子敷,帛裹,一日一易。

《葛氏方》治久疽附骨以积年,一合一发汁出不瘥方∶火烊饴以灌疮中,日三。

又方∶以白杨叶屑敷之。
治石痈方第六

《病源论》云∶石痈者,亦是寒气客于肌肉,折于气血,结聚所成。其肿结确实,至牢有根,核皮相亲,不甚热,微痛,热时自歇。此寒多热少,坚如石,故谓之石痈也,久久热气乘之,乃有脓。

《短剧方》云∶有石痈者,始微坚皮核相亲着,不赤头,不甚尖,微痛热,热渐自歇,便极坚如石,故谓石痈难消,又不自熟,熟皆可百日中也。初作便服防己连翘汤,自针气泻之,敷练石薄,积日可消,若失时不得治,不可消,已有脓者,亦用此薄,则速溃。脓浅易为火针,诸痈溃后用膏散,依治缓宜法,初作即以小豆薄涂之,亦消也。

治痈结肿,坚如石,或如大核,色不变,或作石痈不消者方∶鹿角八两(烧作灰),白蔹二两,粗里黄色磨石片一斤,烧石极令赤,纳五升苦酒中复烧,烧竟复更纳苦酒中,令减半止,捣石作末,并鹿角屑、白蔹屑,余苦酒和如泥,浓涂痈上,才干更涂取消也。

《千金方》治石痈坚如石不作脓者方∶生商陆根捣敷之,燥则易。又治漏疖。(《医门方》同之。)又方∶酢和莨菪子末敷,极亦得。

凡发肿至坚而有根者曰石痈。

又方∶当上灸百壮,石子当出。

又方∶梁上尘、葵茎灰分等,酢和涂。

又方∶蜀桑根白皮,阴干,捣末,消胶,以酒和桑皮敷上,敷上即拔出。
治痤疖方第七

《病源论》云∶痤疖者,由风湿冷气搏于血,结聚所成也。肿结如梅李也。

《养生方》云∶人汗入诸食中,食之作痈疖。

又云∶五月勿食不成核果及桃、枣,发痤疖之。

《太素经》云∶汗出见湿,乃生痤疽。注云∶痤痈之类,然小也,俗谓之疖子。

《短剧方》治痈及疖始结肿赤热者∶水摩半夏涂之,燥更涂,得皱便消也。山草中自可掘取生半夏乃佳,神验。

《千金方》治疖子方∶凡疖无头者,吞葵子一枚,多服头多。

又方∶牛矢封之。

《徐伯方》治痈疖方∶捣商陆根和糟敷之。

又方∶捣百合根敷之,食之亦得蒸。

又方∶捣苦苣叶敷上。又生仓食苦苣。

《徐伯方》云∶捣商陆根和糟敷之。

又方∶捣生牛膝根敷之。

《范汪方》云∶痈疖初生即灸其头数百壮,即愈。

又云∶痈疖初生尚微者,取如鸡子所石若瓦十余枚,烧,以布帛裹熨之,重按令极热,热微者辄易,二三十枚则消。

《删繁论》云∶治痈疖方∶捣生苎根以薄肿上乃止。

《救急单验方》疗初患似疖后破无痂,疼痛不可忍,名猪啄(竹角反)疮方∶烧猪鼻作灰,附立瘥。

《陶景本草》注∶伏龙肝捣筛,合葫涂甚效。

《刘涓子方》治痈疖虚肿方∶当归(二两)草(二两)赤石脂(二两)升麻(四两)白蔹(四两)芎(四两)大节〔(黄)四两〕干玄参(三两)上八物,下筛,鸡子白和如泥(涂)故布上,随肿所大小作帖贴,燥复易之。
治疽方第八

《病源论》云∶疽之状,肉生小黯点,小者如粟米豆,大者如梅李,或赤或黑,乍青乍白,有实核,惨痛应心。或着身体,其着手指者,似代指,人不别者呼为代指。不急治,毒逐脉上,入脏则杀人。南方人得此疾,皆斩去指,恐其毒瓦斯上攻脏故也。

又云∶十指端忽然策策痛,入心不可忍,向明望之,晃晃黄赤,或黯黯青黑,是疽。

直斩后节,十有一冀。

又云∶风疹痛不可忍者,疸也。疽,发五脏俞,节解相应通洞,疽也。诸是疽皆死。又齿间臭热,血不止,疽也,七日死。治所不瘥,以灰掩覆其血,不尔着人。

又云∶诸是疽皆死,唯痛取利,十有一活耳。此皆寒毒之气客于经络,气血痞涩,毒变所生。

《短剧方》云∶瘭疽者,肉中忽生一子如豆粒,小者如米粒粟,剧者如梅李大,或赤或黑,或青或白,其黯状实核,核有根而不浮肿也,痛应心,其根极深达肉肌也。小久不治,便四面悉肿,疮黯KT(丁甚反)紫黑色,能烂坏筋骨也,毒流散,逐脉走入脏腑,则杀人。南方人名为,着毒得着浓肉处皆即割去之,亦烧铁令赤,烁上令焦如炭,亦灸黯疮上百壮为佳。单舂酸摹(和名之)叶薄其四面,以防其长也。饮葵根汁、蓝青汁、犀角汁、升麻汁、竹沥汁、黄龙汤诸单方,治能折其热耳,内外治法依治丹毒方也。

瘭疽着指头者,其先作黯疽,然后肿赤黑黯,(丁甚反),KT痛入心是也。代指者,其先肿欣欣热痛,色不黯KT也。(以上《千金方》同之。)《千金方》云∶瘰疽秘方世所不传,神良无比方∶夜干(二两)甘草(二两)大黄(十分)麝香(二分)干地黄(二两)枳实(二两)犀角(六分)前胡(三分)八味,水九升煮取三升,分服,瘥止,不限剂。

治瘭疽着手足、肩背,累累(良伪反)如米起色白,刮之汁出,愈复发方∶黄(六分)款冬花(二分)升麻(四分)赤小豆(一分)附子(一分)苦参(一分)六味,下筛,酒服半钱匕,渐渐增至一钱,日三。

又方∶熬芜菁子,熟捣,帛裹,辗转其上勿止。

又方∶熬麻子,末,摩上,日五。

又方∶鲫鱼(三寸长者),乱发如鸡子大,猪脂二升,煎涂。

《病源论》云∶凡疽发诸节解及腑脏之俞,则猝急也;其久疽者,发身体闲处,故经久积年,致脓汁不尽,则疮内生虫,而变成。

《范汪方》治久疽众医所不能治方∶沸饴灌疮中,三灌即愈。《葛氏方》同之。

治久疽恶疮连年不瘥方∶黄连(二分)赤小豆(二分)附子(半分,炮)凡三物,各捣为屑,合药之,若疮有汁,以屑敷之;无汁皆以猪膏和屑,铜器中火上使一沸,以敷之。

《录验方》治疽方∶烧铁令赤烁之。

又方∶蛭KT尤佳。

又方∶饮葵根汁。

又方∶饮蓝青汁。

又方∶饮犀角汁。

又方∶饮黄龙汤。

《龙门方》治瘭疽彻骨痛方∶取狗粪,当户根前烧作灰,涂之。烧时勿令患人知,验。
治久疽方第九

《千金方》治久疽方∶取鲫鱼,破其腹,勿损肠脬,纳上白盐末,以针缝合,于铜器中火煎令干,末,着疮中,无脓者以猪脂和敷,小疼痛,勿怪也。

《令李方》治久痈疽漏夕药散方∶夕药(三分)大黄(三分)白蔹(三分)莽草(二分)凡四物,冶合下筛,和调之,以酒服半钱匕,日二。不知可稍增至方寸匕。
治缓疽方第十

《病源论》云∶缓疽者,由寒气客于经络,致营卫凝涩,气血壅结所成。其寒气盛者,则肿结痛深,而回回无头尾,大者如拳,小者如桃李,冰冰与皮肉相亲着。热气少,其肿与肉色相似,不甚赤,积日不溃,久乃变紫黯色,皮肉俱烂,如牛领疮,渐至通体青黯,不作头,而穿溃脓出是也。以其结肿积久,而其肉腐坏迟,故名缓疽;亦名肉色疽也。

《短剧方》治缓疽方∶初作宜服五香连翘汤,去血,以小豆薄涂之,其间数以针去血。又敷之取消良也,不消色未变青KT者,以练石薄敷之。若失时,不得消已烂者,犹服五香连翘汤及漏芦汤下之,随热多少投方也。外以升麻汤洗之,敷升麻膏。若生臭恶肉者,可以单行一物白茹散敷之,青黑肉尽便敷也。好肉熟生但敷升麻膏良,肉不生敷单行一物黄散也。若敷白茹散,积日青黑恶肉不尽者,可以柒头赤皮茹散。取半钱匕和杂三大钱匕白茹散中合,冶之,稍以敷之。恶肉去尽还淳用白茹散也,视好肉欲生可敷黄散也。

白茹散、柒头茹散、黄散。

上三方并一物单行,随多少舂下筛用耳。
治甲疽方第十一

《病源论》云∶甲疽之状,疮皮浓,甲错剥起是也。其疮亦痒,恒欲搔抓之,汁出。其初皆是风邪折血所生,而疮里亦有虫也。

《医门方》疗甲疽,其候甲际生肉,痛不得着靴鞋(户皆反),脓血不止方∶上取石胆火上烧令烟尽,研末敷疮上,消痛,不过三五度必瘥。极效,宜保爱之。

《随时方》治甲疽方,因割甲伤肌作疮痒,浸淫相染,脓血如火烧,疮日夜渐引,名医不能疗者,此方必瘥如神。

绿矾状似朴硝,绿色,炭烧沸尽候看色赤,停冷,简取好者捣筛为散,粗石不堪弃之勿用。上若患前件疮者,先以盐汤净洗,以绿矾散浓敷之。用帛缠裹经一日KT脓水即干。

若觉疮干急痛,即涂酥取润,每一两日一洗,浓敷之。病初患疮尚小脓小,未多之时,只取少许散药,和苏如软面敷一两日,即瘥。(《极要方》同之。)
治肠痈方第十二

《病源论》云∶肠痈者,寒温不适,喜怒无度所致,使邪气与营卫相干,在肠内,遇热加之,血气蕴积,结聚成痈。热积不散,血肉腐坏,化而为脓。其病之状,少腹微强,小便似淋,恶寒,身皮甲错,腹皮急,如肿状。甚者腹胀大,转侧闻水声,或绕脐生疮,而穿脓出;或脓自脐中出;或大便去脓血。惟宜急治之。又大便脓血,似赤白下,而实非者,是肠痈。猝得肠痈而不晓,治之错者则杀人。

《千金方》肠痈之为病,小腹重,而抑强之则痛,小便数似淋,时时汗出,复恶寒。其身皮甲错,腹皮急如肿状。

又云∶绕脐有疮如粟,皮热便脓血,似赤白下,必死,治之方∶屈两肘,正灸肘头锐骨各百壮,则下脓血即愈。

又方∶大黄(四两)牡丹皮(三两)桃仁(五十枚)冬瓜仁(一升)芒硝(二两)五味,水六升,煮取一升,尽服,当下脓血。

《集验方》治肠痈汤方∶薏苡仁(一升)牡丹皮(三两)桃仁(三两)冬瓜仁(一升)凡四物,以水六升,煮取二升,分再服。

《范汪方》治肠痈方∶大黄(一斤,金色者)大枣(十六枚)凡二物,以水一斗,煮取三升,宿勿食。能一服,须臾攻痛如火烧之,痈坏血即随大便出。

《医门方》疗肠痈方∶甘瓜子(一升,碎)牡丹皮大黄(别浸)芒硝(各三两)桃仁(去尖)甘草(炙,各二两)水七升,煮取二升半,下大黄,更煮二三沸,绞去滓,纳芒硝,分温三服,当下脓血。
治肺痈方第十三

《病源论》云∶肺痈者,由风寒伤于肺,其气结聚所成也。肺痈之状,其人咳,胸内满,隐隐痛而战寒。

又肺痈有脓而呕者,不须治其呕也,脓止自愈。

又云∶咽干,口内燥而不渴,时时出浊唾腥臭,久久吐脓如粳米粥,难治。

又云∶痈脓吐如粥,始萌可救,脓成则死。

又云∶肺痈者在胸间,咳有血也。

《千金方》云∶咳,胸中满而偏振寒,脉数咽干而不渴,时时浊唾腥臭,久久吐脓如粳米粥,是为肺痈。桔梗汤主之∶桔梗(三枚)甘草(一两)凡二物,咀,以水三升,煮取一升,绞去滓。适寒温,分为再服。朝饮暮吐脓血即愈。

(《葛氏方》同之。)《范汪方》治肺痈方∶用薏苡一升,咀,淳苦酒三升,煮得一升,适寒温一服,有脓血当吐之。(《葛氏方》同之。)《百济新集方》治肺痈方∶黄一两,以水三升,煮取一升,分二服。(《葛氏方》同之。)《僧深方》治肺痈经时不瘥,桔梗汤主之方∶桔梗(三两)甘草薏苡仁败酱干地黄术(各二两)当归(一两)桑根皮(一升)凡八物,切,以水一斗五升煮大豆四升,取七升汁,去豆纳清酒三升,合药煮三升半,去滓,服七合,日三夜再。禁生菜。

《医门方》疗肺痈喘气急,卧不得安者方∶葶苈子(三两,熬,捣如泥)大枣(三十枚,破。)水二升,煮枣二沸,去滓,纳葶苈脂一两,煎取一升,又以布滤,顿服之。忌猪肉酸咸。
卷第十六
治疔疮方第一

《病源论》云∶疔疮者,风邪毒瓦斯搏于肌肉所生也。初如风疹搔破青黄汁出,里有赤黑脉。亦有全不令人知,忽以衣物触反手着则痛。亦有肉突起如鱼眼,赤黑久结皆变烂成疮,疮下有深孔如火针穿之状。初作时,突起如回钉盖,故谓之疔疮。令人恶寒,四肢强痛,疮便变焦黑色,肿大光起,根硬强,酸痛,皆其候也。在手足头面骨节间者最急,其余处则可也。毒入腹,则烦闷不佳,或如醉,如此者二三日便死也。

《养生方》云∶人汗入酒食内,食之作疔疮。

凡有十种∶一,疮头乌而强凹;二,疮头白而肿实;三,疮头如豆色;四,疮似葩红色;五,疮头内有黑脉;六,疮头赤红而浮虚;七,疮头如葩而黄;八,疮头赤如薄色;九,疮头如茱萸;十,疮头似石榴子。

《千金方》云∶论曰∶夫禀形之类,须存摄养,将息失度,百病萌生。故四时代谢,阴阳递兴,此之二气更相击怒,当时也,必有暴气。夫暴气者,每月之中必有猝然大风。大雾、大寒、大热,若不时避,人忽遇之,此皆入人四体,顿折皮肤,涩经脉,遂使腠理壅隔,营卫结滞,阴阳之气不得宣泻,变成痈疽疔毒恶疮诸肿。至疔肿,若不预识,令人死,死不迎辰。若觉讫乃欲求方,其人已入木矣。所以养生之士,须早识此方。凡是疮痍无所逃矣。

又云∶一、麻子疔,其状肉上起,头,大如黍米,色少乌,四边微赤多痒,忌食麻子,及衣布并入麻田中行。(今按∶《录验方》云∶大小如忝米,头黑有部浆,肉色不异。)二、石疔,其状皮肉相连,色乌黑如乌豆,甚硬刺不入肉内,隐隐微疼。忌瓦KT砖石之属。(今按∶《录验方》云∶头黑靥下疮加对,有部浆,四畔小赤并粟。)三、雄疔,其状疮头乌靥,四畔疮泡浆起,色黄,大如钱孔。忌房。(今按∶《录验方》云∶连根加头黑,刺不入,有部浆,无赤粟。)四、雌疔,其状疮头少黄向里靥,亦似灸疮,四畔泡浆起色赤,大如钱孔。忌房。(今按∶《录验方》云∶头赤四畔黑黄,泡浆,有汁无粟。)五、火疔,其状如火疮,头乌靥,四边有泡浆,有如赤粟。忌火灸烁。(今按∶《录验方》云∶头黑靥肉色赤赤粟多。)六、烂疔,其状少黑有白斑,疮中有脓水,形大小如匙面。忌热食烂物。(今按∶《录验方》云∶大小如拭面,脓血俱,有四畔,无赤粟。)七、三十六疔,其状头乌浮起,形如乌豆,四畔起,大赤。今日生一,明生两三乃至十,若满三十六,药所不治。不满者可治。俗名黑KT。忌镇喜愁恨。(今按∶《录验方》云∶头黑两两俱生,但时满三十六,患者即死。)八、蛇眼疔,其状疮头黑,皮上浮生形如小豆状,似蛇眼,体大。忌恶眼人见之及嫉妒人看之。(今按∶《录验方》∶头黑条,四畔有部浆,赤粟。)九、盐肤疔,其状大如匙面,遍疮皆赤,有黑粟起。忌咸食。(今按∶《录验方》∶头白赤大如小豆,赤粟广多,无部浆。)十、水洗疔,其状疮如钱形,或如钱孔大,疮头白里黑靥,汁出中鞭。忌饮浆水,水洗,渡河。(今按∶《录验方》∶头白,无部浆,有赤粟。)十一、刀镰疔,其状疮阔狭如薤叶,长一寸,侧内黑如烧烂,忌刺及刀镰切割。(今按∶《录验方》一头三角,有部浆,无赤有粟,忌兵刃。)十二、浮瓯疔,其状疮体曲圆,少许不合,长而狭如薤叶大,内黄外黑白,黑白处刺不痛,里黄处刺痛。(今按∶《录验方》∶头高,肉上出四畔,无部浆,赤粟,忌疔铁。)十三、牛拘疔,其状肉内泡色起,掐不破。(今按∶《录验方》∶名羊疔疮,即有三角,有部赤粟。)上一十三疮,初起必先痒后痛,先寒后热,热定寒多。四肢沉重、头痛、心惊、眼花,若太重者呕逆,呕逆者难治。其麻子疔一种,始末唯痒,所录之忌不得犯触,犯触者难治。

其浮瓯、牛拘两种,无所禁忌。纵不治亦不杀人。

又云∶有此病者,忌房、猪、鸡、鱼、牛、生韭、蒜、葱、芸苔、胡荽、酒、酢、面、葵等。又见豹即死,大忌。(今按∶《录验方》云∶七日不得食盐及酒肉、五辛、生冷酢滑,不得带赤者,有毒。唯用纯白色者。)又云∶凡治疔肿皆刺中心至痛,又刺四边十余下令血出。去血敷药,药气入针孔中佳。

若不达疮里,则不相得力也。

又云∶治一切疔肿方∶苍耳根茎子叶皆得烧作灰,酢泔和作泥涂上,不过十易。枝根出。

又方∶涂雄黄末立愈,神验。

又方∶枸杞煮汁冷冻饮料一二盏,弥佳。

《录验方》云∶有疔毒疮,肉中突起如鱼眼状,赤黑,酸痛彻骨,是寒毒久结。及在此疾也,其烂成疮,疮下有深孔如火针穿也。初作突起状如细钉盖,故谓之疔毒者焉。初作即服汤及诸单行治如治丹方法便瘥也。北方饶此疾也,江东时有作者。喜着口里颊边及舌上也,看之正黑如珠子。含服汤、针刺去血如治丹疽法也。

痈疽方治疔疮方∶以甘刀割十字,以铜铁箸烧火令赤,疮上置腊,少烧,刺名曰为烁,一、二遍,无毒肉时自热止,烧鼠屎作灰末,着疮穴满之即瘥。(今按∶《经心方》∶烧锥赤刺头上。)又云∶治恶疮疔肿五香汤方∶青木香薰陆香沉香丁子香藿香(各一两)水三升,煮取一升半,分三服,得麝香二分去藿香。

《极要方》治疔疮方∶捣茺蔚茎叶敷肿上,服汁令疔毒内消也(一名益母草。)《百济新集方》治疔肿毒瓦斯入心欲困死方∶取菊叶合茎捣绞取汁三升,顿服之。

《医门方》疗疔毒肿不问雄、雌、麻子等一切毒肿毒瓦斯入腹杀人方∶频煮枸杞根汁饮之。

又云∶治疔肿方∶白僵蚕、白疆石末和,封之最佳。

又方∶水研白疆石服一盏,并针开疮敷上自消。

《经心方》治疔肿新方∶末附子,酢和涂上,燥复涂之。

《陶景本草注》治疔疗方∶人屎干者烧之,烟绝水渍饮汁,名破棺汤。

《苏敬本草注》疔疮方∶末白僵蚕封上,根当自出,极效。

《崔禹锡食经》疔疮方∶捣茎叶根敷之,疮根即拨。

《耆婆方》治一切疔疮神方∶以末少少敷即瘥。(今按∶以冷水泼之。)又云∶治人热毒疔疮在口中方∶凝水石捣末研之,少少以敷疮上,日三、四敷,即瘥。
治犯疔疮方第二

《病源论》云∶犯疔疮者,谓疔疮欲瘥,更犯触之。若大及食猪鱼麻子并狐臭人气熏疮皆能犯之,则更剧乃甚于初。更令疮热肿,先寒后热,四肢沉重,头痛心惊,呕逆烦闷则不可治之。

《千金方》云∶欲知犯状,但脊强,疮痛极甚,不可得忍。是犯状,治之方∶多捣苍耳汁饮并涂上。

又方∶水四升,煮蛇蜕皮如鸡子大,三、四沸,去滓服,立愈。

又云∶若犯者∶取枸杞根切三升,以水五升,煮取一升。取滓研一钱,上和汁一盏服之,日二、三服。

并单饮冷汁一、二盏弥佳。

《医门方》疗犯疔肿欲死者方∶捣菊叶取汁服之,冬月取根,神验。

《录验方》治患疔疮犯欲死方∶取磁石和酢封,立拨根出。

又方∶石硫黄烧铁着之。

又方∶酢练磨石遍敷之,立瘥。

又方∶取冬葵子服方寸匕,日二。

《救急单验方》疗犯疔疮疮根入腹欲死方∶取东行母猪粪和水绞汁饮一升,瘥。

又云∶已死者∶取大黄龙汤一升,暖之,以木拗口,灌即活,甚验。
治毒肿方第三

《病源论》云∶毒肿之状,与风肿不殊,时令人壮热,其邪毒甚者,入腹杀人。

《经心方》治毒肿五香汤方∶沉香青木香薰陆香丁香(各一两)麝香(半两)五味,以水五升,煮取一升半,分三服。

《刘涓子方》五香丸治恶气肿毒方∶薰陆香(二分)藿香(二分)青木香(二分)鸡舌香(二分半)鬼臼(二分)大黄(八分)当归(五分)升麻(三分)朱砂(一分半)牡丹(二分)雄黄(一分)上十一物,捣,下筛,蜜和为丸,清白饮一服四丸,丸如小豆大,日再。

又云∶五毒膏治恶气毒肿方∶蜀椒(二两)当归(二两)朱砂(二两)乌头(一升)苦酒(一升半)猪肪(六斤)巴豆(一升去心)雄黄(二两)上八物,咀,以苦酒淹一宿,纳猪肪,合煎微火上,三上三下,药成,向火摩肿上,日三。

《集验方》治风热毒肿结赤夜干膏方∶夜干(二两)常陆(切一升)防己(四两)升麻(三两)四物,切,以猪膏三升,微火煎常陆小焦黄,绞去滓以摩病上。

《葛氏方》治猝患恶毒肿起稍广急痛方∶烧牛屎末,以苦酒和敷上,燥复换之。

又方∶捣荏子如泥涂上,燥复换之。

又方∶以苦酒、升麻及青木香、紫真檀合磨以指涂痛处良。

又方∶但以甘刀破上,泄去毒血及敷药,弥佳。今按∶取水蛭令吮去恶血,其方在治痈疽之方。

《千金方》治恶毒肿,或着阴卵,或偏着一边,疼急痛变,牵入少腹不可忍,一宿杀人方∶取茴香草捣取汁饮一升,日三、四服。滓敷肿。此外国神方,从元嘉末来,用之起死人,神验。

《救急单验方》疗一切恶肿疼痛不可忍无问冷热大小方∶取莨菪子三枚,捻熟,捋勿令破,吞之验。

《孟诜食经》毒肿方∶末赤小豆和鸡子白,敷之立瘥。

《陶景本草注》毒肿方∶煮青木香浴大佳。

又方∶服蓝汁。
治风毒肿方第四

《病源论》云∶风毒肿者,其先赤痛飙热,肿上生瘭浆,如火灼是也。

《短剧方》云∶有风热毒相搏为肿,其状先肿热,上生瘭浆如火烁者名风热毒也。治之如治丹毒法也。

《葛氏方》若风毒兼攻通身渐肿者方∶生苦参菖蒲根三白根锉,各一斗,以水一石五斗,煮取一斗,去滓纳好酒一升。温服半升,日三。又洗身。

《极要方》疗风毒初肿令消方∶大黄(二两)葶苈子(二两,熬)通草(二两)莽草(二两)上为散,水和敷肿上,燥易之,神效。

《耆婆方》治人风肿在皮上发有时方∶升麻(三两)夜干(二两)夕药(二两)三味,切,以水三升,煮取一升,分三服。
治风肿方第五

《病源论》云∶风肿者,肿无头无根浮在皮上如吹也,不赤不痛,移无常处而兼痒。由腠理虚而逢风所作。

《葛氏方》云∶凡毒肿多痛,风肿多痒,按之随手起,或痱瘰、隐疹皆风肿。治之方∶但令人痛以手摩将抑,按数百过自消。

又方∶炒蚕矢并盐布裹熨之。

又方∶苦酒摩,桂若独活以敷之。

又方∶楸叶浸水中以裹肿上。

又方∶以铍刀决破之出毒血便愈。

《经心方》白蔹贴治风肿毒核痈疽方∶白蔹(二两)黄芩(半两)草(半两)夕药(一两)黄(一两)当归(一两)大黄(半两)赤石脂(二两)八味,为散,以鸡子白和粥涂纸粘贴,燥复易。

《本草云》∶柞蓖麻子油涂之。
治热肿方第六

《病源论》云∶其热毒作者,亦无正头,但急肿色赤而时恶寒壮热,烦闷不安。

《极要方》疗热毒肿秘之不传方∶皂荚刺(一握去两头)上以水一大升,煮取半升,去滓顿服之,取利,其肿如汤沃雪。

又云∶疗一切热毒肿忽发颈项胸背上即封不成脓方∶生地黄(二升)香豉(半升)芒硝(五两)上捣令熟,以敷肿上,浓二分,日五、六付,消止。

《救急单验方》疗热毒肿方∶取桑树东南根下土,和水作泥饼安肿上,以艾灸之,取热应即止。男女并同。

《广济方》疗热毒肿方∶取牛胁骨烧为灰,以大酢和如泥涂上,干易。

《耆婆方》治人热肿疼痛方∶升麻(三两)夜干(二两)大黄(二两)芒硝(二两)青木香(一两)栀子(一两)甘草(半两)七味,锉,以水六升煮取三升,纳芒硝搅令调。分三服,得下利即瘥。
治气肿方第七

《病源论》云∶气肿病者,其状如痈,无头虚肿,色不变,皮上急痛,手才着便即痛是也。此风邪搏于气所生也。

《短剧方》云∶有气肿病,其状如痈,无头虚肿,色不变,皮上急痛,手才着便觉痛。

此由体热当风复被暴冷凉折之,结成气肿也。

宜服五香连翘汤,白针气泻之,敷蒺藜薄,亦用小豆薄并得消也。(今按∶五香连翘汤在治恶核之方中。)蒺藜薄方∶蒺藜子二升,下筛,以麻油和如泥,熬令焦黑,以涂细故热布上,剪如肿大勿开头,拓之,无蒺藜可舂小豆下筛,鸡子白和涂肿上,干复涂之并得消也。(《集验方》同之。)
治气痛方第八

《病源论》云∶气痛者,人身忽然有一处痛,发作有时,痛发则小热,痛静便如冰霜所加,故云气痛也。亦由体虚为风邪所侵,遇寒气而折之,邪气不出故也。

《短剧方》云∶有气痛病,身中忽有一处痛如打掴之状,不可堪耐,亦左右走身中,发作有时,痛发时则小热,痛静时便觉其处如冷水霜雪所加。此皆由冬时受温风,至春复暴寒凉来折之,不成温病乃变作气痛也。

宜先服五香连翘汤数剂及竹沥汤,摩丹参膏及以白酒煮杨柳树皮,暖熨之,有赤气点点见处宜去血也,其间将白薇散。

小竹沥汤治气痛方∶淡竹沥(二升)夜干(二两)杏仁(二两)茵芋(半两)黄芩(半两)白术(二两)木防己(二两)防风(二两)秦胶(二两)茯苓(三两)麻黄(一两)独活(二两)枳实(二两)夕药(二两)甘草(二两)凡十五物,咀,以水九升,煮药折半乃可纳竹汁,煮取三升,分四服,少嫩人分作五服。

白薇散治风热相搏结,气痛,左右走身中,或有恶疹起者,积服汤,余热未平复,宜此白薇散以消余热方∶白薇(六分)葳蕤(四分)当归(四分)麻黄(三分)秦胶(五分)天门冬(四分)蜀椒(二分)木防己(四分)柴胡(三分)草(二分)独活(四分)枳实(四分)乌头(二分)术(六分)人参(四分)夜干(六分)山茱萸(四分)防风(六分)白芷(三分)凡二十物,捣下绢筛,以酢浆服方寸匕渐至二匕,日三。少嫩人随长少减服之。毒微者可用酒也。(以上《集验方》同之。)
治恶核肿方第九

《病源论》云∶恶核肿者,肉里忽有核,累累如梅李,小有如豆粒,皮内酸痛,左右走身中,猝然而起,此风邪挟毒所成。其亦似射工毒。初得无常处,多恻恻痛。不即治,毒入腹,烦闷恶寒即杀人也。久不瘥,则变作。

《短剧方》云∶有恶核病者,肉中忽有核累累如梅李核状,小者如豆粒,皮肉中酸痛,左右走人身中,壮热KT畏寒是也。与诸疮痕瘰、结筋相似。其疮痕瘰要因疮而生,是缓疾无毒。其恶核病,猝然而起,有毒,不治,入腹烦闷则杀人。南方多有此疾,皆是冬月受温风,至春夏有暴寒冷相搏,气结成此毒也。宜服五香连翘汤,以小豆薄涂之得消也。

亦煮五香汤,去滓,时时洗渍之,消化之后,以丹参膏敷余核令消尽,不消尽者,还敷小豆薄也。

五香连翘汤治恶脉及恶核,瘰、风结诸核肿气痛方∶青木香(二两)麝香(半两)沉木香(二两)薰陆香(一两)鸡舌香(一两)连翘子(二两)夜干(二两)升麻(二两)独活(二两)寄生(二两)大黄(三两)甘草(二两)淡竹沥(二升)凡十三物,咀,以水九升煮药,计水减半许可纳竹沥汁,又克取三升,分三服。

丹参膏治恶脉及恶核、瘰、风结诸核肿气肿痛方∶丹参(二两)蒴根(二两)草(半两)秦胶(一两)独活(一两)踯躅花(半两)蜀椒(半两)白芨(一两)牛膝(一两)菊花(一两)木防己(一两)乌头(一两)凡十二物,细切为善,以苦酒二升渍之一宿,夏月半日。急疾即煎之,以猪膏四升煎苦酒竭,勿令暴焦熬也,去滓以膏涂诸疾上,日五、六,至良。

《葛氏方》治恶核肿结不肯散者方∶乌根升麻(各二两)以水三升,煮取半升,分再服,以滓熨上。

又方∶烧白鹅屎,以水服三方寸匕,以肉敷肿上。

又方∶苦酒摩由跃涂之,捣小蒜敷之。

《录验方》治恶核肿毒入腹五香汤方∶薰陆香射香沉香鸡舌香青木香(各二两)凡五物,以水六升,煮取二升半,适寒温分用三服。不瘥复作,云令剂可尽。

五香各一两,水四升,煮取三升,亦为二服。又滓敷肿上,神良。

《僧深方》∶凡得恶肿皆暴卒,初始大如半梅桃,或有核。或无核,或痛或不痛,其长甚速,须臾如鸡鸭大,即不治之肿。热为进,烦闷拘挛,肿毒内侵,填塞血气,气息不通,再宿便杀人。初觉此病便急,宜灸当中央及绕肿边灸之,令相去五分,使周匝肿上,可三七壮。肿盛者多壮数为瘥。肿进者,逐灸前际取住乃止。

又方∶鲫鱼捣敷肿上。

又方∶啖鲫鱼脍蒜齑。

《刘涓子方》治恶核肿毒汤方∶乌扇(二两)升麻(二两)栀子仁(十四枚,破)上三物,切,以水三升,煮取一升半,分再服。以滓敷肿上甚良。

《张仲景方》治消核肿黄贴方∶黄(三两)真当归(三两)大黄(三两)芎(一两)白蔹(三两)黄芩(三两)防风(三两)夕药(二两)鸡子(十枚)黄连(二两)凡十物,捣,筛,以鸡子白和涂纸上,贴肿上,燥易。

又方∶捣茱萸以囊盛,敷核上,亦可令速消开,多得效验。
治恶肉方第十

《病源论》云∶恶肉者,身里忽有肉如赤豆粒突出,细细长乃如牛马乳大,亦如鸡冠之状,不痛也,亦久不治,长不已。春冬被恶风所伤,风入肌肉,结瘀血积而生也。

《短剧方》云∶有恶肉病,身中忽有肉如赤豆粒突出,便长推出不息,如牛马乳,亦如鸡冠状也。不治其为自推出不肯止,亦不痛痒也。此由春冬时,受恶风入肌脉中变成此疾也。

治之宜服漏芦汤,外烧铁烁之,日日稍烁令焦尽也。烁竟以升麻膏敷之,积日乃瘥耳。

(今按∶漏芦汤、升麻膏在治丹之方。)
治恶脉病方第十一

《病源论》云∶恶脉者,身里忽有赤络,脉起聚如死蚯蚓状。看如似有水在脉中,长短皆逐其,络脉所出见是也。由春冬受恶风,入络脉中,其血瘀结所作也。

《短剧方》治恶脉病方∶宜服五香连翘汤及竹沥汤,去恶血,敷丹参膏,积日则瘥。亦以白雄鸡屎涂之。(《集验方》同之。)《刘涓子方》治恶脉肿毒方∶乌扇(二两)升麻(二两,生者用一两)栀子(十四枚,擘破)上三物,切,以酒三升,煮取一升半,分为再服,以滓敷肿上甚良。

又云∶升麻汤治恶脉毒肿方∶升麻(一两)吴茱萸(一两)薰陆香(二两)鸡香舌(一两)雄黄(一两)鳖甲(一两,炙)甘草(一两)乌扇(三两)青木香(一两)上九物,以水七升煮取二升半,适寒温,分三服,相去一里。治脉肿神良。(今按∶升麻汤,又有《短剧》治丹之方,药种与此不同。)
治编病方第十二

《病源论》云∶编病者,由劳役,肢体热盛,因取风冷,而为凉湿所折,入于肌肉,筋脉结聚所成也。其状赤脉起如编绳,急痛壮热。其发于脚者,喜从鼠仆起至踝,赤如编绳,故谓编病也;其发于擘者,喜腋下起至手也,不即治取,其溃去脓则筋挛缩也;其着脚,若不治,不消,复不溃,其热歇气不散,变作KT。脉缓相薄,肿KT已成脓。

《短剧方》治KT病方∶宜服漏芦汤,自下外以锋针数去血气,针泻其结核处,敷小豆薄则消。皆可依治丹法消之,亦用治痈三物甘焦薄薄也。及至溃成脓,火针,敷膏散,亦如治痈法之。

《葛氏方》治皮肉猝肿起夹长赤痛名曰编方∶鹿角(一两)白(一两)牡蛎(四两)附子(二两,炮)上四物,捣,下筛,苦酒和涂帛以贴之,干复换之。
治瘰方第十三

《病源论》云∶风邪毒瓦斯容于肌肉,随虚处而停结为瘰。或如梅、李、枣等核大小,两三相连皮间而时发寒热是也。久则变脓溃成也。

《录验方》云∶疗瘰,唯须以员针针之,小者即消,大者即熟,然后出脓便瘥,隔日一针。

《千金方》云∶一切瘰在项上及阴处,但有肉结凝似作及痈疖者方∶以独头蒜截两头却心,作艾炷,秤蒜大小贴子上灸之,勿令上破肉但取热而已。七壮一易蒜,日日灸之,取消止。

又方∶白僵蚕为散,水服五分匕,日三,十日瘥。

又方∶干猫舌末敷疮上。

又方∶野狼屎灰敷上。

《医门方》治瘰方∶尖针针子令穿通,以石硫黄如豆大安针孔中,烧针筋令赤烁之,药流入疮中,其疮瘥即消,极验也。

《短剧方》治三十岁瘰瘿方∶海藻一斤,绢囊盛,好清酒二斗渍之,春夏二宿,服二合,酒尽复以酒二斗渍之,饮如上法,此酒尽爆海藻令燥,末,服方寸匕,日三,药无所禁。一剂不愈,更作不过三剂也。

(今按∶《葛氏方》治颈下瘰累累如梅李,宜使速消。)《刘涓子方》治寒热瘰在颈腋下皆如李大方∶人参(四分)甘草(四分)白芷(四分)干姜(四分)凡四物,皆同分,冶合筛,先食服方寸匕,日三。少小服半方寸匕,良。一方以酒服。

《范汪方》治瘰朝夕发热龙骨散方∶龙骨(七分)牡蛎(三分,一方分等)凡二物,冶合下筛,先食服五分匕,日三。

《僧深方》治诸瘰因疮壮热方∶白蔹灰(二升)上一物,沸汤和如糜,热以掩其上。甚良。

《广利方》疗瘰成作孔方∶露蜂房二枚,炙末,和腊月猪脂,涂孔上。
治瘿方第十四

《病源论》云∶瘿者,由忧恚气结所生。亦由饮沙水,沙随气入于脉,搏颈下而成之。

初作与樱核相似,而当颈下也,皮宽不急,垂然是也。恚气结成瘿者,但垂垒垒无核也。

饮沙水成瘿者,有核垒垒无根,浮动在皮中。又云∶三种瘿,有血瘿,可破之。有息肉瘿,可割之。有气瘿,可具针之。

《养生方》云∶诸山水里土中出泉流者,不可久居。常食作瘿病,动气增患。

《短剧方》云∶有瘿病者,始作与樱相似。其瘿病喜生颈下,当中央不偏两边也。皮宽不急,垂然则是瘿也。中国人患气结瘿者,但垂无及也。长安及襄阳、蛮人其饮沙水善病沙瘿,有垒垒耳无根,浮动在皮中。

治诸瘿良方∶小麦一斗,以淳苦酒一斗渍小麦,令释,漉出,爆令燥,燥复渍之,苦酒尽,曝麦燥,捣下筛,以海藻三两,别捣末合冶之。温酒服方寸匕,日三,禁盐、生鱼、猪肉、生菜。数用有验也。(今按∶《龙门方》加昆布三两。《范汪方》云∶治三十年瘿及瘰。)《千金方》治瘿方∶昆布二两,切,如指大,酢渍,令咽汁尽则愈。

又方∶灸风府穴百壮,又灸大椎百壮。

又方∶灸大椎两边相去各一寸半少下垂各三十壮。(《短剧方》云∶大椎上节以上属颈,崇骨也,大椎与崇骨相接处,其节最高硕也。)《葛氏方》治颈上猝结果渐大欲成瘿方∶海藻一斤,酒二斗渍一宿,稍稍含一、二合咽之。酒尽取滓末服方寸匕,日三。

《范汪方》治瘿昆布丸方∶昆布(八两)海藻(八两,洗)凡二物,捣下筛,和以蜜丸。先食,含如半枣大,稍稍咽之。日五服,不知稍增,以知为度。

《效验方》治瘿昆布丸方∶昆布(二分)松萝(二分)海藻(五分)凡三物,冶合下筛,以白蜜丸如李子,含咀嚼咽其汁,日三夜二。

《极要方》治瘿海藻散方∶海藻(十分)昆布(一两)海蛤(一两)通草(一两)松萝(一两)干姜(一两)桂心(一两一方无干姜,代白蔹一两。)上七物,下筛,酒服方寸匕,日三。

《耆婆方》治人气瘿方∶松萝(二两)海藻(三两)通草(二两)半夏(三两洗一遍)桂心(二两)海蛤(三两)昆布(三两)干姜(六两)茯苓(二两)细辛(三两)桔梗(二两)上十一味,捣筛为散,以酒服一方寸匕,日三。

又方∶炒盐薄之。

《玉箱方》治三十年瘿及瘰方∶海藻(八两)贝母(二两)土瓜根(二两)麦面(二分)四味,作散,酒服方寸匕,日三。(《经心方》同之。)《广利方》疗瘿结气方∶昆布(二大两,暖水洗,去咸味,寸切)小麦(三大合)以水二大升,煮取小麦熟,择取昆布空腹含三、五斤,津液细细咽之,日再含。(忌生冷油腻。)
治瘤方第十五

《病源论》云∶瘤者,皮肉中忽肿起,初如梅李大,渐长大,不痒不痛,又不结强。言瘤结不散,谓之为瘤。不治,乃至大则不复痛。不能杀人,亦慎不可辄破之。

《范汪方》云∶发肿都软者,血瘤也。发肿状如虽极大,此肉瘤非痈也。

《千金方》治瘤病方∶矾石芎当归大黄黄连夕药白蔹黄芩(各二分)吴茱萸(一分)九味,为末,鸡子黄和之,涂细故布上,随瘤大小以敷贴之,干即易。着药当熟作脓脂细细从孔出也。按却脓血尽,着生肉膏。若脓不尽,复起故也。

又云∶生肉膏主痈瘤溃漏及金疮凡百疮方∶当归(一两)附子(一两)甘草(一两)白芷(一两)芎(一两)生地黄(五两)薤白(二两)七味,切,以猪脂二升半,煎白芷黄,去滓稍以敷之,日三。《僧深方》同之。

又云∶凡肉瘤勿治,治杀人,慎慎之。

《僧深方》治血瘤方∶鹿肉割,炭火炙令热,掩上拓之。冷复炙,令肉烧燥,可四炙四易之。若不除,灸七炷便足也。

《玉箱方》杨树酒治瘤瘿方∶河边水所注杨树根三十斤,熟洗细锉,以水一石,煮取五斗,用米三斗,面三斤酿之酒成。服一升。(《集验方》同之。)
治诸方第十六

《病源论》云∶病之生,或因寒热不调,致血气壅结所作。或由饮食乖节,野狼鼠之精入于腑脏,毒流注脉,变化而生。皆能使血脉结聚,寒热相交,久则成脓而溃漏也。生身体皮肉者,亦有始结肿,与石痈相似。所可为异者,其肿之中。按之累累有数核,喜发于颈边,或两边俱起便是证也。亦发两腋下及两颞间,初作喜不痛不热,若失时不即治生寒热也。

所发之处,而有轻重。重者有两,一则发口上壁,有结核,大小无定,或如桃李大,此虫之窠窟,正在其中。二则发口之下,无有结核,而穿溃成疮。

又云∶虫毒之居,或腑脏无定。故发身体,亦有数处,其相通者多死。其形状起发之由,今辨于后章。

《养生方》云∶六月,勿食自落地五果,经宿蚍蜉、蝼蛄、蜣螂游上,喜为九。十二月,勿食狗鼠残肉,生疮及出颈项及口里,或生咽内也。

又云∶决其死生者,反其目视之,其中有赤脉,以上下贯瞳子。见一脉一岁死;见一脉半,一岁半死;见二脉,二岁死;见二脉半,二岁半死;见三脉,三岁死。赤脉而不下贯瞳子,可治也。

又云∶方说九者,是野狼、鼠、蝼蛄、蜂、蚍蜉、蛴螬、浮疽、瘰、转脉,此颈之九也。

又云∶复有三十六种,方不次第显其名。而有蜣螂、蚯蚓等诸,非九之名,此即是三十六种之数也。

《短剧方》云∶有者,始结肿与石痈相似,所可为异者,其一种中,按之累累有数核便是也。

初作喜不痛不热,即以练石薄敷之,内服防己连翘汤下之,便可得消。若失时不治结脓者,亦以练石薄薄,令速熟,熟用火针膏散。如治痈法。初作即以小豆薄涂之亦消。

又云∶桐君说,赤小豆、白蔹、黄芩、黄、牡蛎,凡五物,等分,下筛,酒服方寸匕。

治瘿瘤诸昆布丸方∶昆布(八两,炙)海藻(七两,洗,炙)小麦(一升,熬)海蛤(五两)松萝(四两)连翘(二两)白头翁(二两)上七物,捣下筛,和蜜丸如梧子,服十九,日三。稍加三十丸。

《千金方》云∶凡有似石痈,累累然作子,有核,在两颈及腋下,不痛不热者皆练石散敷,内服五香连翘汤下之。已溃者,疗如痈法。(今按∶五香连翘汤有治恶方。)凡项边、腋下先作瘰者,欲作漏也。宜禁五辛、酒、面及诸热食。

又云∶灸漏方∶捣生章陆根,捻作拌子,置漏上,以艾灸上,拌子热易之。灸三四升艾,瘥。(《经心方》同之。)又方∶葶苈子二合,豉一升,二味合捣,令熟,作饼如大钱,浓二分许。取一枚当疮孔上。作大艾炷如小指大,灸饼上,三炷一易,三饼九炷,日三。隔三日复一灸。

又方∶灸周四畔即瘥。

又云∶治漏方∶捣土瓜根敷之,燥复易,不限时节。

又方∶烧死蜣螂末,酢和涂之。

又方∶死蛇灰,酢和敷之。

《葛氏方》通治诸方∶取地中潜行鼷鼠一头,破腹去肠干之,火炙令可成屑末,以腊月猪脂和,敷疮上。

又方∶烧蝼蛄作屑,猪膏和,敷之。

又方∶白犬骨烧末以猪膏和敷之。

又云∶治诸着口里齿颊间者方∶东行母练根细锉三、四升,以水浓煮,取汁含之。数吐易,勿咽之。

《耆婆方》治人三十年疮方∶取蒴根曝令土燥,槌去土,大釜中以水煮令熟,去滓,置大盆中,绞取清汁煎之,若盛复熟小器中煎,令如薄糊,纳器中。若有痂,去之纳煎。若病深,以鸡毛取煎冷暖软以纳中,瘥乃止。取煎蒴,滓举着其人,瘥。而不报息者,取滓烧还发也。秘之。

《广济方》疗久不瘥方∶巴豆(一两,去皮)大枣(一升)上以水五升,煮取一升,绞去滓,更煎如稠饧,敷疮上,日三。

《刘涓子方》治众方不瘥效验方∶取牡蒙数两,捣之,汤和。适寒温,取一升许,薄疮上。冷复易。经日益佳。

《新录方》治诸方∶露蜂房末酒服方寸匕。

又方∶兔皮灰敷之。

又方∶芥子末敷之。

又方∶桃叶捣如泥封之。

《救急单验方》疗诸疮方∶煎楸枝叶,净洗疮,纳孔中大验。

又方∶石留黄末,置疮孔中,以艾灸立验。

《陶景本草注》诸方∶玄参,酒渍,饮之良。

又方∶KT鼠蹄,烧末,酒服之。

《苏敬本草注》诸方∶马苋,捣,揩之。

又方∶食雉肉,良。

《崔禹锡食经》诸方∶食一、两斤蕨,终身不病也。
治野狼方第十七

《病源方》云∶野狼者,年少之时,不自谨慎,或大怒,气上不下之所生也。始发之时,在于颈项,有根出缺盆,上转连耳本,根在肝。

《刘涓子方》治野狼为病,始发于颈,肿有根,起于缺盆,上转连耳本。此因忧、恚,气上不得下。

空青(二分)燥脑(二分)燥肝(一具)芎(半分)独活(一分)女妇草(一分)黄芩(一分)鳖甲(一分)斑蝥(一分)干姜(一分)当归(一分)蜀椒(三十枚)茴香(一分)矾石(一分)地胆(一分)十五物,捣,下筛。酒服方寸匕,日三。十五日服之。
治鼠方第十八

《病源论》云∶鼠者,饮食之时不择,虫蛆之毒而变化,入于脏,出于脉,不去使人寒热。其根在肺,出于颈项。

《千金方》疗鼠疮瘥后复发,及不愈出脓血不止方∶以不中水猪脂,咀生地黄,纳脂中,令脂与地黄足相淹和,煎六、七沸。桑灰汁净洗疮去恶汁,以地黄膏涂上,日一易。

又云∶鼠肿核痛未成脓方∶以柏叶敷着肿上,熬盐着叶上熨之,令热气下即消。神良。

又云∶治风及鼠方∶赤小豆白蔹黄牡蛎凡四味,等分,酒服方寸匕,日三。(今按∶《短剧方》加黄芩治诸。)《葛氏方》治鼠方∶取槲白皮浓煮取二升,服一升,当吐鼠子。

又方∶捣车前草以敷之。

又方∶巴豆去心皮,以和艾作柱,灸疮上。

又方∶取小鼠子剥去皮,炙令燥,捣末,以腊月猪膏和,敷之。

又云∶若已有口,脓血出者,以热牛屎涂之,日三。

《范汪方》治颈鼠累累方∶贝母(二分)干姜(一分)桂心(一分)蜀椒(一分)吴茱萸(一分)本(一分)凡六物,冶,下筛。先食,以酒服一撮,良。

《录验方》∶治鼠及痈三十年∶乌头散方∶乌头(一两)黄柏(二两)凡二物,冶下筛。酒服一刀圭,日八夜四,起令药热相继,初得痈,即服良。

《刘涓子方》治鼠方∶又方∶死蜣螂,烧作屑,苦酒和,涂之。

《经心方》治鼠方∶烧地黄叶,粘贴得瘥。

又云∶鼠子,结核未破者,用大针针之,无不瘥。
治蝼蛄方第十九

《病源论》云∶蝼蛄者,食果瓜子,不避有虫,既便啖食之。有毒不去,变化所生也。

在于颈上,状如蜗形,瘾疹而出,其根在大肠。

《集验方》治蝼蛄方∶取蝼蛄脑二七枚,酒和敷上。

《千金方》有蝼蛄方∶槲叶灰,先以泔渍,煮槲叶取汁,洗拭干,纳灰疮中。

《刘涓子方》治蝼蛄,始发于颈,状如肿。此得之食果子瓜实毒不去核。

龙骨(半分)桂心(一分)干姜(一分)桔梗(一分)矾石(一分)附子(一两)独活(一分)芎(半分)蜀椒(一百枚)上九物,捣,下筛,别取干枣二十枚去核,合捣之,取酢浆和之便得丸。日服五,丸如大豆,温浆服之。
治蜂方第二十

《病源论》云∶蜂者,食饮劳倦,渴乏多饮流水,即得蜂毒,蜂毒不去,变化所生也。

发在颈,三、四处俱肿,以溃生疮,状如痈形,瘥而复移,其根在脾。

《刘涓子方》治蜂,始发于颈,瘰三、四处俱肿,连以溃移。此得之多饮流水,水有蜂余毒不去。

蜂房(一具)鳖甲(一分)茴香子(一分)茱萸(一分)椒(二百枚)干姜(一分)上六物,捣,下筛作散,敷疮孔口上,日十度。

《千金方》有蜂方∶人屎,蛇蜕灰腊月猪膏和之,敷孔中。

又方∶蜂窠灰,腊月猪膏和,敷孔中。

《葛氏方》云∶若着鼻内外,查瘤血出者,是蜂。取觚、蜂房,火炙焦末,温酒服方寸匕,日一。
治蚍蜉方第二十一

《病源论》云∶蚍蜉者,因寒,腹中胪胀,所得寒毒不去,变化所生也。始发之时,在其颈项,使人壮热若伤寒,有似疥癣,娄娄孔出,其根在肺。

《刘涓子方》治蚍蜉,始发于颈,初得如伤寒。此因食中有蚍蜉毒不去。

桃白皮(一分)白术(四分)知母(一分)雌黄(一分)干地黄(一分)皮(四分)独活(一分)椒(一百枚)青黛(一分)斑蝥(一分)白芷(一分)柏脂(一分)夕药(一分)海苔(一分)当归(一分)上十五物,合捣作散,下筛,服一钱匕,日三。病在里空腹服,在外先食后服之。又方∶无斑蝥、术。
治蛴螬方第二十二

《病源论》云∶蛴螬者,恐惧愁忧思虑,哭泣不止,余毒变化所生也。始发之时,在其颈项,无头尾,如枣核,或移动皮中,使人寒热心满,其根在心。

《刘涓子方》云∶治蛴螬,始发颈,无头尾,如枣核块,块多在皮中,使人寒热心满。

此因喜怒哭泣。

空青(二分)当归(二分)细辛(一两)干肉〔一分(一方用皮)〕枸杞根(一分)斑蝥(一分,去翅)地胆(一分)干鸟脑(如三大豆)上八物,合冶下筛,作散。服方寸匕,日三,以酢浆下散。病在上,倒输卧;在下,高枕,使药流下。
治浮沮方第二十三

《病源论》云∶浮疽者,因恚结驰思,往反变化所生。始发之时,在于颈项,亦在腋下,如两指无头尾,使人寒热欲呕吐,其根在胆。

《刘涓子方》∶治浮疽,始发于颈,如两指,使人寒热欲卧。得之因思虑忧满,其根在胆。地胆主之。甘草为佐方∶石硫黄(一分)干姜(一分)龙胆草(二分)细辛(二分)地胆(一分,去翅)石决明(一分,去皮)续断(一分)大黄(半分)阴芦根(一分)上九物,下筛,敷疮上,日四、五。
治瘰方第二十四

《病源论》云∶瘰者,因强力入水,坐湿地,或新沐浴,汗入头中,流在颈上之所生也。始发之时,在其颈项,恒有脓,使人寒热,其根在肾。

《刘涓子方》∶治瘰,始发于颈,有根,令人寒热,此得之新沐,汁入头中,下流于颈。

茯苓(一两)续断(一分)矾石(二分)干地黄(一分)空青(一分)石(一分,炼)干姜(一分)桔梗(一分)蜀椒(一分)恒山皮(一分)斑蝥(一分)鸟脑(一分,熬)附子(一合,炮)干姜(一分)干狸肉(一分)上十五物,捣合下筛,以白蜜和,酒服如大豆十丸,日再。
治转脉方第二十五

《病源方》云∶转脉者,因饮酒大醉,夜卧不安,惊欲呕,转侧失枕之所生也。始发之时,在于颈项,濯濯脉转。身如振使人寒热,其根在小肠。

《刘涓子方》∶治转脉,始发于颈,濯濯脉转,身始振,寒热。此得之惊卧失枕。

绿青(二分)人参(二分)当归(二两)升麻(一分)麦门冬(一两,去心)大黄(二分)钟乳(二分)桂心(二两)甘草(半分)防风(一分)白术(一分)地胆(一分)续断(一分)麝香(一分,末)石(半分生用之)上十五物,合捣,下筛,麝香末筛合,更捣令调,白蜜和如大豆,温酒服十丸,日三。

勿食生菜、生鱼、肥肉,忌房内,满百日令得,都瘥。
治蜣螂方第二十六

《病源论》云∶蜣螂者,由饮食居处,有蜣螂之毒瓦斯,入于脏腑,流于经脉所生也。

初生之时,其状如鼠乳,直下肿如覆手,而痒,搔之疼痹,至百日有七,八孔,入三寸,中生蜣螂,乃有百数,蜣螂成尾,自覆刺人,大如盂升,至三年杀人。

《千金方》∶治蜣螂方∶牛屎灰,和腊月猪膏敷之。

又方∶热牛屎涂之。
治蚯蚓方第二十七

《病源论》云∶蚯蚓者,由居处饮食,有蚯蚓之气,或因饮食入腹内,流于经脉所生。

其根在大肠,其状,肿核溃汁漏之。

《集验方》∶蚯蚓方∶取蝼蛄脑二七枚,酒和敷疮上。

《刘涓子方》∶治蚯蚓方∶鸭膏和胡粉,敷疮上,已灵。

《千金方》∶有蚯蚓方∶蚯蚓尿、鸡尿末之,以二月牡猪下颔髓和敷。
治蚁方第二十八

《病源论》云∶蚁者,由饮食有蚁精气,毒入于五脏,流出经脉,多着颈项,戢戢然小肿核细,而乃遍身体。

《集验方》∶蚁方∶半夏一果,捣作屑,以鸭膏和,敷疮上。

《千金方》∶蚁方∶皮、肝、心灰末,酒服一钱匕。

又方∶雄鸡灰末敷之。

又方∶鼠灰敷之。

《葛氏方》∶若疮多而孔少者,是蚁。

烧陵鲤鳃甲,猪膏和敷,佳。
治蝎方第二十九

《病源论》云∶蝎者,饮食居处,有蝎之毒瓦斯,入于脏流于经脉。或生腋下,或生颈边,肿起如蝎虫之形,寒热而溃成。久则疮里生细蝎。

《千金方》∶有蝎,五孔,六孔皆相通方∶捣茅根汁,着纳孔中。
治虾蟆方第三十

《病源论》云∶虾蟆者,饮食流于经脉,结肿寒热,因溃成。服药有物随小便出,如虾蟆之状。

《千金方》∶有虾蟆方∶五月五日蛇头及野猪脂付水衣敷之佳。
治蛙方第三十一

《病源论》云∶蛙者,饮食居处有蛙之毒瓦斯,入于脏流于经脉而成。因服药随小便出物,状如蛙形是也。

《千金方》∶有蛙方∶蛇腹中蛙灰敷之。
治蛇方第三十二

《病源论》云∶蛇者,居处饮食有蛇毒瓦斯,入于脏,流于脉。寒热结肿,出处无定,因溃成。服药有物随小便出,如蛇形。

《千金方》∶有蛇方∶蛇蜕灰,腊月猪脂和敷之。
治方第三十三

《病源论》云∶者,由居处饮食,有之毒瓦斯,入于脏流于经脉所生。初得之时,如枣核许,或满百日,或满周年,走不一处,成孔脓汁溃漏。

《千金方》有方∶捣瓜根敷,至瘥。慎口味。
治雀方第三十四

《病源论》云∶雀者,居处饮食,有雀之毒瓦斯,入于脏,流于脉,发无定处,因溃成。服药有物随小便出,状如雀KT。

《千金方》有雀方∶母猪屎灰,和腊月猪膏敷,虫出如雀形。
治石方第三十五

《病源论》云∶石之状,初起两头如梅李核靳实按之强如石而寒热,热后溃成。

《千金方》有石从两头出者,其状坚实,令人寒热方∶以大铍针破之,鼠粘草二分末,和鸡子一枚敷之。

又方∶捣槐子和井花水敷之。
治风方第三十六

《病源论》云∶风者,由风邪在经脉,经脉结聚所成也。

《经心方》∶治风鼠方∶桑白皮七、八斤,细锉,水二斗,煮取汁一斗,更煎汁,取二升半,顿服下,虫瘥。

又方∶烧地黄叶粘贴得瘥。

又方∶杏仁一升,熟捣,和生猪脂敷上。

《千金方》∶风及鼠方∶赤豆白蔹黄牡蛎四味等分,酒服方寸匕,三。
治内方第三十七

《病源论》云∶内者,人有发疮,色黑有结内有脓,久乃溃出,侵食筋骨,谓之内。

《龙门方》∶治内方∶取槐白皮十两捣丸绵裹,纳下部验。

又方∶煎楸叶,作煎稠,堪丸,以竹筒纳下部,蚶、痔、悉瘥。
治脓方第三十八

《病源论》云∶诸皆有脓汁,此独以脓为名者,是诸疮久不瘥成,而重为热气所乘,毒瓦斯停积生脓,常不绝,故谓之脓。

《千金方》有脓方∶桃花末和猪脂敷之。

又方∶盐,面和,烧灰敷之。

《龙门方》∶脓出方∶石硫黄末,置疮孔中,以艾灸立验。

《本草》脓出方∶杨卢木,水煮叶汁。洗之,立瘥。
治冷方第三十九

《病源论》云∶冷者,亦是诸疮得风冷,久不瘥,因成,脓汁不绝,故为冷。

《千金方》∶凡一切冷方∶烧人吐出蛔虫灰,先甘草汤洗,后着灰,无不瘥者,慎口味。
卷第十七
治丹毒疮方第一

《病源论》云∶夫丹者,人身体,忽然赤,如丹涂之状,故谓之丹也。或发手足,或发腹上,如手掌大,皆风热恶毒所为。重者,亦是疽之类,不急治则痛,痛不可堪。久乃坏烂,去脓血数升。若发于节间,便流之四肢,毒入腹杀人。小儿得之最忌。

又云∶白丹者,初发痒痛,微虚肿,如吹,疹起不痛不赤,而白色也。由挟风冷,故使色白也。

黑丹者,初发亦痒痛,或肿起,微黑色也。由挟风冷,故色黑。

赤丹者,初发疹起,大者如连钱,小者如麻豆,肉上粟粟如鸡冠肌理。由风毒之重,故使赤也。亦名茱萸丹。疹者,肉色不变,又不热,但起隐疹,相连而微痒,故谓丹疹之。

室火丹者,初发时必在腓肠,如指大,长二、三寸,皮色赤而热之。

天灶火丹者,发时必在于两股里冲,引至阴头而赤肿是也。

废灶火丹者,发时必于足趺上,而皮色赤者是也。

尿灶火丹者,丹发于胸腹及脐,连阴头皆赤是也。

烟火丹者,丹发于背,亦在两臂,皮色赤是也。

火丹者,丹发于髀,而散走无常处,着处皮赤是也。

萤火丹者,丹发于骼至胁,皮赤是也。

石火丹者,丹发通身,似缬。自突如粟是也,皮色乃青黑也。

《短剧方》云∶丹毒者,方说一名天火也,肉中忽有赤如丹涂,赤色也。大者,如手掌大。其剧者,竟身体。亦有痛痒微肿者。方用∶赤小豆二升,舂,下筛,以鸡子白和如泥涂之,小干复涂之,逐手消也。竟身者,倍合之尽,复作,纳宜服漏芦汤。漏芦汤方。

漏芦(二两)白(二两)黄芩(二两)白薇(二两)枳实(二两)升麻(二两)夕药(二两)大黄(二两)甘草(二两)麻黄(二两)凡十物,咀,以水一斗,煮取三升,分三服,若是穷地无药之处,依说增损易服之。

及用后单行方也。

能以锋针去其血,然后敷药,大良。

增损方法∶无漏芦,用栀子十枚;无白,亦可略耳;无黄芩,亦用栀子;漏芦、黄芩并无者,但以栀子一物,亦足已之;无升麻,用犀屑;无犀屑,用蛇衔;无大黄,用芒硝;无麻黄,用葛根;无葛根,用石膏;无白薇,用知母;无知母,用葳蕤;无葳蕤,用枳实;夕药、甘草亦可略耳;都无药,但得大黄单服之,亦大善。

升麻汤治丹疹诸毒肿KT渍方∶升麻(二两)黄芩(二两)栀子(二十枚)漏芦(二两)蒴根(五两)芒硝(二两)凡六物,咀,以水一斗,煮取七升,停冷分用,渍KT恒湿也。

升麻膏治丹疹诸毒肿热疮方升麻(二两)黄芩(二两)栀子(二十枚)白(二两)漏芦(二两)枳实(三两,炙)连翘(二两)蒴根(四两)芒硝(二两)蛇衔(三两)凡十物,切,舂碎细细,以水三升,渍半日;以猪脂五升,煎令水气竭,去滓,敷诸丹毒肿热疮上,日三。若急须之,但合水即煎之。

又云∶单用一物,舂以敷之方∶(今按∶《集验方》同之。)生蛇衔生地黄生蒴叶生慎火叶生菘菜叶生五叶藤舂豆豉浮萍上八物,一一别捣,别涂之。

大黄黄芩栀子芒硝上四物,各舂,水和,各涂之。

《葛氏方》云∶丹大者,恶毒之气,五色无常,不即治,既痛不可堪,又待怀,怀则去脓血数升,或发于节解,多断人四肢。盖痈疽之类。治之方。

又方∶煮栗蒺有棘刺者以洗之(和名久利乃伊芳加。)又方∶取赤雄鸡血,和真朱以涂。

又方∶猪膏,和胡粉涂之。

又方∶捣麻子以涂之。

又方∶叶涂之。

又方∶以慎火涂之。

又云∶治白丹方。

又方∶渍蛴螬涂之。

又方∶末豉以苦酒和涂之。

又方∶捣香若蓼,敷之。

又方∶烧鹿角,以猪膏和涂之。

又方∶捣酸模草五叶者,饮汁,以滓敷之。

又云∶治猝毒瓦斯攻身,或肿或赤,或痛或痒,淫弈分散,上下周匝烦毒欲死方∶取生鱼切之如脍,以盐和敷之。通身赤者,KT多作,令竟病上。干复易之,鲋鱼为佳。

《集验方》治丹若走皮中浸广者,名为火丹,入腹杀人。治之方∶取蛴螬末,以涂之。

又云∶若通身赤者方∶取妇人月布敷之,又取汁以浴小儿。

又方∶捣大黄,水和涂之。

又方∶捣栀子,水解涂之。

又方∶水和芒硝涂之。

《范汪方》治白丹方∶破生鲤,热血敷之,良。

《千金方》治丹神验方∶芸苔菜捣令熟,浓封,随手即瘥。余气未愈,三日来封,使醒醒好瘥止,干则封。

又方∶牛屎涂之,干则易之。

又云∶赤流肿者∶榆根白皮末,鸡子白和,涂之。

《医门方》云∶凡人面目忽得赤黑丹如疥状,不疗,遍身即死方∶以猪槽下泥涂之。

又方∶烧鹿角末,和猪脂涂之。

《苏敬本草注》云∶捣茺蔚敷之。

又云∶苎根捣贴之。

《极要方》丹肿方∶生鲫鱼肉捣如泥涂之。

《博济安众方》云∶以连钱草,以盐敷之。

《崔禹锡食经》云∶敷水中苔,良。
治癣疮方第二

《病源论》云∶癣病之状,皮肉上隐疹如钱大,渐渐增长,或圆或斜,痒痛。有匡郭,里生虫,搔之有汁。此由风湿邪气客于腠理,复值寒湿与血气相搏,则血气痞涩阻绝。谓之发此病。按《九虫论》云∶蛲虫在人肠内,变化多端,发动亦能为癣,而癣内实有虫也。

《养生方》云∶夏不用屋而露面卧,露下堕面上,令面皮浓,喜成癣也。

干癣,但有匡郭,皮枯索,痒搔,搔之白屑出是也。皆是风湿邪气客于腠理,复值寒湿为血气相搏所生。若其风毒瓦斯多,湿气少,故风沉入深,故无汁,为干癣也。其里亦生虫。

湿癣亦有匡郭也,如虫行,浸淫,亦湿痒,搔之多汁成疮。是其风毒瓦斯浅,湿多风少,故为湿癣也。其里亦生虫之。

风癣是恶风冷气客在皮,折于血所生。亦作圆纹匡郭,但抓搔顽痹,不知痛痒是也,其里亦生虫。

白癣之状,白色淀淀然而痒。此亦是腠理虚受风,风与气并,血涩而不能荣肌肉故也。

牛癣,俗云以盆器盛水饮牛,用其余水洗手面,即生癣,名为牛癣。其状皮浓,抓之靳强而痒是也。其里亦生虫。

圆癣之状,作圆纹隐起,四畔赤,亦痒痛是也。其里亦生虫。

狗癣,俗云狗舐之水,用洗手面即生癣。其状瘢微。白点缀相连,亦微痒是也。其里亦生虫。

雀眼癣,亦是风湿所化。其纹细似雀眼,故谓之雀眼癣,搔之亦痒。其里亦生虫。

刀癣,俗云以磨刀水,用洗手面而生癣,名为刀癣。其状无匡郭,纵斜无定是也。其里亦生虫。

《葛氏方》治癣疮方∶以好苦酒于石上磨桂,以涂之。

又方∶苦酒磨柿根涂之。

又方∶干蟾蜍烧末,以膏和,涂之立愈。

又方∶蓼叶涂之。

又云∶治湿癣方∶刮疮上,火炙麋脂涂之。末蛇床子,猪膏和,敷之。

又方∶取母草以刮癣上,取瘥止。

又云∶治燥癣方∶水银和胡粉涂之。

又方∶雄鸡冠血涂之。

又方∶熬胡粉令黄赤色,苦酒和如泥,敷上,以纸贴之,干更涂之。

又方∶捣桃白皮,苦酒和,涂,瘥。

又方∶以树汁涂之。

《千金方》治癣秘方∶捣羊蹄根,分着罂中,以白蜜和之,先刮疮边伤,先以蜜和敷之,若炊一石米顷拭去,更以三年大酢和,以敷癣上,燥便瘥。若刮疮少许处,不伤,即不瘥。

又云∶凡癣积年不瘥,随有之处,皆用得愈方∶取自死蛇,烧作灰,腊月猪脂和涂,即愈。

《范汪方》∶治癣湿方∶取羊蹄根,细锉数升,以桑薪灰汁煮四、五沸,绞去滓,以汁洗疮。

《僧深方》治癣方∶末雄黄,酢和,先以布拭疮,令伤,以药涂上,神效不传。

又方∶附子一枚,皂荚一枚,九月九日茱萸四合。

上三物,下筛为散,搔癣上令周遍汁出以散敷之。若干癣,以苦酒和散,以涂其上,神良,秘方。

又云∶治癣积年不愈方∶取鲶,炙而食之,勿食盐、酢。三过三食便愈,当时乃当小盛,此欲愈也。

又云∶治蜗癣浸淫日长,痒痛,搔之黄汁出,瘥复发方∶日未出时,北向取羊蹄根,勿令妇人、小儿见,洗去土,切捣,淳苦酒和洗疮,去痂以敷上一时,间以冷水洗之,日一敷。又可取根揩之,神良。日未出取者,不欲歇加根上。

《极要方》疗湿癣方∶上,以日未出时。采取羊蹄根,其根须独,并不得有杈枝,不得令见风。切捣为末,和羊酪,着少盐,于日中曝雨食久,以涂癣上。

又方∶取椿叶面着癣上,是匙背打,使极碎,即裹之,勿令叶落,无不瘥者。

又云∶疗干癣积年痂浓,搔之黄水出,逢阴雨即痒方∶上,取巴豆肥者一枚,于炭火上烧之,令脂出,即于斧上,以指研之,如杏子涂癣上。

薄涂不过一两度便愈。

又方∶作艾炷以灸之,随灸随瘥。

又方∶捣马齿汁,揩之,洗之,效。

《博济安众方》∶一切癣、恶疮、小儿头疮方∶以水银、白矾石、蛇床子、黄连,和猪脂敷之。

《苏敬本草注》∶东壁土摩之,干湿兼治也。

《陶景本草注》云∶艾叶、苦酒煎涂之。

《本草经》云∶捣酢浆草敷之,杀诸小虫,又治恶疮也。

《广利方》∶疗诸癣疮,或湿痛,痒不可忍方∶以醋磨石留黄,涂上。

又方∶苦楝皮,烧作灰,和猪脂涂上。
治疥疮方第三

《病源论》云∶疥有数种,有大疥,有马疥,有水疥,有干疥,有湿疥。多生手足,乃至遍体。大疥者,作疮有脓汁,赤痒痛是也。马疥者,皮肉隐嶙起,作根,搔之不知痛,此二种则重。水疥者,作如瘭浆,摘破有水出。此一种小轻。干疥者,但痒,搔之皮起作干痂。湿疥者,起小疮,皮薄,常有汁出,并皆有虫,人往往以针头挑得,状如水内虫。

此悉由皮肤受风邪热气所致也。《九虫论》云∶蛲虫多作疥。

《葛氏方》治卒得蚧疮方∶猪膏煎芫花以涂之。

又方∶麻油摩硫黄涂之。

又方∶锻石二斗,以水五斗,汤洗,取汁,先拭疮,以此灰汁洗之。

又方∶东行练根刮末,苦酒和涂。通身者,浓煮,以浴佳。

又言∶酒渍苦参饮之。

《集验方》治疥汤方∶蜀椒四合,以水一斗,煮三沸,去滓,令温,洗疥。

又方∶大麻子一升捣令破,煮如粥,以曲一斤着中,涂之,治马疥最良。

《删繁方》治癣及疥等乱发膏方∶乱发(如鸭子大一枚)鲫鱼(一头)雄黄(二两)八角附子(一枚)苦参(一两)猪膏(一枚)凡六物,煎,捣附子三物为末,猛火煎猪膏、发、鱼,令尽,纳末药,敷疮上。

《范汪方》治疥水银膏方∶水银(一两二分,一方二两)黄连(一两)黄柏(一两,炙)蓝漆(一两)乱发(二分,烧成灰)凡五物,捣下筛,和以神明膏三合,令相得,涂疥上,日三。神良。

又方∶羊蹄根捣,猪脂和涂,或小与盐。

《极要方》胡粉膏疗疥方∶胡粉(三两)水银(二两)松脂(二两)猪膏(六两)上,煎成去滓,纳水银、胡粉,和调。涂疮上,日二。

又方∶苦酒摩野狼跋子涂之,效。

又方∶用椿木叶,以煮汁洗之。

又方∶以酢磨练实根涂之。

《陶景本草注》∶柳叶煮以洗之。

又方∶漏芦根,苦酒磨以敷之。

《苏敬本草注》∶取柏枝烧其下,承取汁名(音诸),甚治疥也。

《本草稽疑》云∶野狼血涂之久疥,良。忌食鱼一月,永瘥,效。
治恶疮方第四

《病源论》云∶诸疮生身体,皆是体虚受风热,风热与血相搏,故发疮。若风热挟湿毒之气者,则疮痒痛肿,而疮多汁,身体壮,谓之恶疮也。

《养生方》∶铜器盖食,汗入食,发恶疮、内疽。

又云∶醉而交接,或致恶疮。

又云∶井水和粉洗足,不病恶疮。

又云∶饮酒热未解,以冷水洗面,令人面发疮,轻者渣。

《刘涓子方》治恶疮方∶用杏仁熬令黄黑,豉熬令黑焦,膏和敷上。

《录验方》甘氏乌膏治天下众疮,医术不能瘥,有虫者,悉治之方∶水银(一两)黄连(二两)墨(二分)上三物,猪膏和,熟,研调如脂,敷不过二三,即愈。秘方。(《僧深方》于潜墨云云。)《范汪方》治恶疮中生肉挺出方∶末石留黄敷之,有汁着末,无汁以唾和,敷之。

《僧深方》治恶疮肉脱出方∶乌头末,以敷疮中,恶肉立去。佳。

《千金方》治恶疮方∶烧篇竹灰,和楮白汁封。

又方∶河水煮白马屎,十沸,洗之。

又云∶十年不瘥者方∶盐汤洗,以地黄叶贴之。

又方∶烧猪屎一升,敷之。

又方∶烧苦瓠子末,敷之。

又方∶烧鲫鱼灰,和酱清敷之,主一切恶疮。

又方∶以牛屎熏即愈。凡疮皆可熏。

又云∶恶疮十年不瘥似癞方∶蛇蜕皮一枚,烧之末,下筛,猪脂和敷之。

又方∶苦瓠一枚,咀,煮取汁洗疮,日三。

又云∶恶疮名曰马疥,其大如钱方∶以水煎自死蛇一头,令烂去骨,以汁涂之手下,瘥。

《葛氏方》治大人、小儿猝得诸恶疮,不可名识者方∶烧竹叶,以鸡子中黄和涂之。

又方∶取牛膝根,捣涂之。

又方∶取螂虫绞取汁,敷疮,疮中虫即走出。

又方∶腊月猪膏一升,乱发如鸭子一枚,生鲫鱼一头。合煎令消尽,不沸止。又纳末雄黄、雌黄、苦参屑各二两,大附子一枚,令搅凝,盛器。以敷诸疮,无不瘥。

又云∶若疮中恶肉突出者∶末乌梅屑,敷疮中,佳。

《极要方》疗一切恶疮十年以上,并漏疮及疥癣作孔,久不瘥方∶黄连(一两)芦茹(一两)蛇床子(一两)石(一两,别捣)水银(半两)上捣筛,以腊月猪脂和如稀泥,下水银,令销尽即成,先以泔清洗疮,然涂药讫,仍以黄柏末,绵沾粉之,令不污衣。

《集验方》治恶疮身体面目皆烂有汁方∶取生鱼三寸者,并少豉,合捣令熟,以涂之,燥复涂。

又云∶治恶疮方∶练子(一升)地榆(五两)桃仁(五两)苦参(五两)水一斗,煮取四升,温洗之。

又云∶恶疮,人不能名者方∶取头垢,猪脂和,涂疮中。
治热疮方第五

《病源论》云∶诸阳气在表,阳气盛则表热,因运动劳役,腠理则虚而开,为风邪所客,风热相搏,留于皮肤则生疮。初作瘭浆,黄汁出,风多则痒,热多则痛,血气乘之,则多脓血,故名热疮之。

《短剧方》热疮者,起疮便生白脓是也。

《刘涓子方》治猝发热疮方∶炭长二尺许,烧令赤,以水二升,灌之出炭,取汁浴之即愈。

《录验方》治热疮黄连粉散方∶水银(熬)黄连胡粉(熬,各一两)凡三物,先捣黄连,下筛。然后合三物熟和之,盐汤洗疮,拭令净,药敷之,日二。
治夏热沸烂疮方第六

《病源论》云∶盛夏之月,人肤腠开,易伤风热,风热毒瓦斯搏于皮肤,则生沸疮。其状如汤之沸,轻者,匝匝如粟粒;重者,热汗浸渍成疮,因以为名,世呼为沸子。

《新录方》治夏月热沸疮方∶细筛锻石粉上。

又方∶以生桑叶揩上。

又方∶酢浆煮洗之。

又方∶以水萍揩涂之。

又方∶捣菟丝苗揩涂之。

《令李方》治人身体热沸生疮方∶矾石(四两,熬)白善(六两,熬)凡二物,冶筛,先以布拭身,乃以药粉之,日二。

今按∶《师说》云∶嚼疮者,风邪在皮肉间,夏时蒸热气时成疮,如风矢,先痒后痛。

色赤白,隐疹如粟米大,治之方∶柚叶,煮水洗之。

又方∶煮栀子叶洗之,亦研栀子粉之。

又方∶粟粉敷之。

若热盛赤血者方∶草舂绞,涂,并煮洗之。
治浸淫疮方第七

《病源论》云∶浸淫疮,是心家有风热,发于肌肤。初生甚小,先痒后痛而成疮。汁出浸淫肌肉,浸淫渐阔,乃至遍体。其疮若从口出流散四肢则轻,若从四肢生然后入口则重。

以其渐渐增长,因名浸淫疮也。

《葛氏方》治猝得浸淫疮,转广有汁,多起于心。不早治之,绕身周匝则杀人方∶以鸡冠血涂之。

又方∶新牛屎,绞取汁涂之,烧以熏之,佳。

又方∶胡燕窠末,以水和敷之。

《录验方》天麻草汤方∶天麻草切五升,以水一斗五升,煮取一斗,分洗,以杀疮痒也。

《极要方》疗身上疮,疮汁所着处即成疮,名曰浸淫,痒不止方∶黄连(一两)黄柏(一两)芦茹(一两)石(一两)甘草(一两)生胡粉(一两)上捣甘草以上为散。胡粉于子中着熬令黄,和之为散。欲敷药,先以苦参汁洗,故帛拭干即着药,不过三四度即瘥。

《苏敬本草注》云∶生嚼胡麻涂之。

《集验方》治猝毒瓦斯攻身,或肿,或赤,或痛,或痒,淫并分散。上下周匝,烦毒欲死方∶取生鱼,切之如作脍,以盐和敷之。若通身多作,令竟病上。干复易之。

《千金方》治浸淫疮方∶以煎饼,热敷之取止,神良。
治王烂疮方第八

《病源论》云∶王烂疮者,由腑脏实热,皮肤虚而受风湿,风湿与热相搏,故初起作瘭浆,渐渐王烂,大汗流浸渍,故曰王烂也。亦名王灼疮,以其初作瘭浆,如汤火所灼也。又名洪烛疮,其初生如沸汤洒,作瘭浆赤烂如火烛,故名洪烛也。

《葛氏方》治大人、小儿猝得王灼疮,一名疮,一名王烂疮,此疮初起作浆。似火疮,故以灼烂为名。

烧牛屎,筛下以粉之。

又方∶熬秫米令黄黑,捣以敷。

又方∶煮小豆汁,纳鸡子,绞以洗之良。

又方∶末黄连、胡粉,油和涂之。

《短剧方》有洪烛疮,身上忽生瘭浆,如沸汤洒。剧者,竟头面,亦有胸胁腰腹通如火汤,瘭浆起者是也。治之法∶急宜服漏芦汤下之,外宜以升麻汤浴,但倍分多煮之,以浴KT之。其间敷升麻膏佳。

若穷地无药者,但依治丹法,用单行草菜方也。(《千金方》同之。)《范汪方》治王烂疮方∶大麻子,大豆等苇桶中纳之,热炙蒸篇头,取汁涂疮上,再过愈。

《僧深方》治王烂疮方∶胡粉(烧令黄)青木香龙骨滑石(各三两)上四物,冶筛毕,以粢粉一升和之,稍稍粉疮上,日四、五愈。
治反花疮方第九

《病源论》云∶反花疮者,由风毒热相搏所为。初生如饭粒,其头破即血出,便生恶肉,渐大有根,脓汁出,肉反散如花状,因名反花疮。凡诸恶疮,久不瘥者,亦恶肉反出,如反花形之。

《龙门方》治反花疮方∶取柳树枝叶为煎,涂之,大验。

又方∶烧马齿草灰敷之,验。(今按∶《千金方》捣封之。)《千金方》治反花疮,并治积年诸疮不瘥者∶取鼠粘草根,细切,熟捣,和腊月猪膏封之,取瘥止。并一切久不瘥诸肿恶疮,漏疮皆瘥。大大神验。

又方∶取蜘蛛膜贴疮上,数易之,神验。

《救急单验方》疗反花疮方∶烧盐末,敷验。
治月蚀疮方第十

《病源论》云∶疮生于两耳及鼻面间,并下部诸孔窍侧,侵食乃至筋骨。月初则疮盛,月末则疮衰,以其随月死生,因名之为月蚀疮也。

又云∶小儿耳下生疮,亦名月蚀。世云小儿见月,以手指指之,则令病引疮也。其生诸孔窍,有虫,久不瘥,即变成也。

《葛氏方》治大人小儿猝得月食疮方∶于月望,夕取兔屎,仍以纳虾蟆腹中,合烧末,以敷疮上验。

又方∶取罗摩草汁,涂。

又方∶烧蚓矢令赤,猪膏和敷。

《集验方》治月蚀疮方∶鼓皮如手,淳苦酒三升,渍一宿,以涂疮上。

又方∶煮枯鲍鱼,以洒之。

《范汪方》∶治月蚀疮,诸恶疮方∶烧仇道末,敷之。疮无汁者,膏和涂。亦可以虾蟆膏涂之。

《令李方》治月食疮,茱萸根散方∶用茱萸根、芦苇根各二两。凡二物,冶合,下筛,生盐作汤洗疮,以散粉上。日三。

《龙华方》治月蚀疮骨出方∶猪脂和杏仁,敷之良。
治恶露疮方第十一

《短剧方》云∶凡以八、九月刺手足,以犯恶露,杀人不轻也。治之方∶用生竹若桑枝两三枚,郁着火中为推引之,令极热,研碎断之。正以头注疮口上,热尽复易着一枚,尽三枚,则疮当正白烂。乃取薤白,捣,以绵裹之,着热灰中,使极热,乃去绵,取薤以薄疮上,以布帛急包裹之。若疮故肿者,更为之。若已中水及恶露风寒肿痛者,以盐数合,急折着疮上,以火灸之,令热达疮中,毕,以腊纳竹管中,以管贮热灰中炮之,腊烊以灌疮。若无盐、薤者,但腊便可单用。

又云∶治恶露疮方∶取蒲若败青布,于小口器中,若坎中烧,以熏之,疮中汁出尽则愈。

《千金方》云∶取韭捣之,以薄疮口上,以火炙之,令热彻疮中便愈。
治漆疮方第十二

《病源论》云∶漆有毒,人有禀性畏漆,但见漆,便中其毒。喜面痒,然后胸臂髀皆悉瘙痒,面为起肿,绕眼微赤,诸所痒处,以手搔之,随手辇展,起赤,消已,生细粟疮甚微。有中毒轻者,证候如此。其有重者,遍身作疮,小者如麻豆,大者如枣杏,脓燃疼痛,摘破小定或小瘥,随次更生。若火烧漆,其毒瓦斯则厉,着人急重;亦有性自耐者,终日烧煮,竟不为害。

《广济方》疗漆疮肿痛方∶嚼糯米敷上,四、五度瘥。忌热面、饮酒。

《葛氏方》治猝得漆疮方∶以鸡子黄涂之,干复涂,不过三。

又方∶煮柳叶,适寒温以洗之。

又方∶捣韭,令如泥以涂之。

又方∶捣蟹涂之。

又方∶嚼秫米涂之。

又方∶煮香,以渍洗之。

又方∶慎火若鸡肠草以涂之。

《极要方》疗漆疮方∶盐汤洗之。

又方∶马尿涂之。

《耆婆方》治漆疮方∶荏菜汁涂之。

又方∶煮生椒汤洗上。

又方∶栀子和水涂之。

《范汪方》治漆疮方∶芒硝二合(一方五两),以水一升,渍自消,色缥以洗之。

一方∶汤渍。

《录验方》治漆疮方∶黄栌木一斤,锉,盐一合,以水一斗,煮取五升,去滓,冷洗。神方。

《救急单验方》疗漆疮方∶以水五升,煮椒一升,十余沸,去椒水,冷洗,立瘥。(今按∶《耆婆方》煮生椒云云。)《集验方》治漆疮洗汤方∶莲叶燥者一斤,以水一斗,煮得五升,洗漆疮上,日二。

又方∶取猪膏涂之。

又方∶宜啖肥肉。

《陶景本草注》云∶削材作柿,煮洗,漆,无不瘥。

《崔禹食经》云∶敷水中苔,良。
治疮方第十三

《病源论》云∶疮者,由肤腠虚,风湿之气,折于血气,结聚所生。多着手足间,匝匝相对,如新生茱萸子。痛痒抓搔成疮,黄汁出,浸淫生长,圻裂。时瘥时剧,变化生虫,故名疮。又有燥、湿候。

《葛氏方》∶治猝得疮,疮常对,在两脚及手足指,又随月生死方∶以白犬血,涂之立愈。

又方∶以苦酒,和黄灰涂之。

又方∶捣桃叶,以苦酒和,疮上涂之。

又方∶煮苦酒,沸,以生韭一把纳中,熟出,以敷疮上,即愈。

又方∶乱发、头垢等分,蜗牛壳二七枚,合烧末,腊月猪膏和敷之。

《僧深方》治方∶取石上菖蒲,捣,猪膏和,敷疮,浓二分,先洗去痂。

又方∶灸疮上最良。

《千金方》∶凡一切疮方∶灸足大趾奇间二七壮。

又灸大指头亦佳。

又方∶酢一升,炒薤一把笃之。

又方∶炙,敷上。

又方∶炒腊月糖敷上。

又方∶烧故履系末敷之。

又方∶烧肥松取脂涂之。

又云∶燥方∶酢和灰涂。

又方∶热牛屎涂之。

又云∶湿方∶烧干虾蟆,猪脂和敷之。

《录验方》云∶芜菁子一升,熬,下筛,以绢裹之,辗转疮上,日三。

《广济方》疗久不瘥方∶取豉熬为末,以泔渍洗,干拭。又和麻油涂上,以故油衣裹三日,开。
治疽疮方第十四

《病源论》云∶疽疮,是疮之类也,非痈疽之类。世云疽即是此也。多发于肢节脚胫间,相对而生,匝匝作细孔,如针头,其里有虫,痒痛,搔之黄汁出,随瘥随发。皆风邪客于皮肤,血气之所变生。亦有因诸浅疮,经久不瘥,痒痛抓搔之,或衣揩拂之,其疮则经久不瘥,而变作疽疮者,而疮里皆生细虫。

《僧深方》治男女面疽疥痈疽诸疮方∶附子(十五枚)蜀椒(一升)冶葛(一尺五寸,去心)上三物,咀,以苦酒渍一宿,猪膏二升,煎附子,黄膏成,摩疮。亦治伤寒,宿食不消。酒服如枣,覆取汗。

《录验方》治疽疮有虫痒附子散方∶附子(八分,炮)藜芦(二分,熬)凡二物,冶合下筛,纳疮中,当有虫出,日三。

《范汪方》治疽疮方∶用胡粉,以猪膏和如泥,敷疮上,良。

《刘涓子方》治疽疮方∶乌贼鱼骨作屑,鲫鱼胆十四枚,和取与散合,敷疮上不三,愈。
治蠼疮方第十五

《葛氏方》治猝得蠼疮方∶此疮绕人腰胁,甚急痛。

盐三升,以水一斗,煮取六升。及热,以绵浸汤中,疮上。

又方∶烧鹿角,苦酒和涂之。

又方∶练皮及枝,烧作灰,敷之。

又方∶末赤小豆,苦酒和涂。若燥者,猪膏和涂。

又方∶胡粉涂之。

又方∶末蚯蚓矢,敷之。

《短剧方》云∶有蝮虫尿人,歇便令人病也。其状,身中忽有处惨痛也。痛如芒刺,亦如虫所吮螫,然后起细,作聚如茱萸子状也。其边赤,中央有白脓如粟粒是也。

亦令人皮肉急剧,恶寒壮热。剧者,连起竟腰、胁、胸、背也。

初得时便以水磨犀角涂之,以止其毒,治之如火丹法,并诸草药单行治也。

又云∶蝮尿人,初未发疮之时,欲与射公相似,射公疮,正有一处,黯黑,蝮疮,KTKT连聚作。痛法,亦小疹以为异耳,然非杀人疾也。

《千金方》尿疮方∶取厕前人尿泥涂,立瘥,绝验,更不须余方。

又方∶嚼大麦,敷之,日三。

又方∶捣豉封上。

又方∶熟嚼梨叶涂。(《极要方》同之。)《极要方》疗蠼尿疮集集然黄水出方∶甘草汤洗之。

又方∶捣韭汁涂之。

又方∶嚼麻子涂之。

又方∶黄柏末,和猪脂涂上,明日以盐汤洗。

又方∶嚼桂涂之。

《如意方》治蠼疮术∶鸡肠草敷之。

《广济方》疗蠼尿绕腰欲死方∶取败蒲扇,煮取汁洗之。

又方∶取扁豆叶,捣汁涂之,立效。

《集验方》治蠼方∶槐白皮半斤,切,以苦酒二升渍半日,刮去疮处以洗,日五、六。

又方∶以猪脂,燕巢中土,苦酒和以敷之。
治诸疮烂不肯燥方第十六

《医门方》云∶诸疮烂燥方∶柳白皮,烧末敷疮上。汤火疮用柏白皮亦佳。

《救急单验方》洗百疮方∶取槐白皮、柏叶各一大握,锉。以水三升,煮取一升,洗百疮并瘥。
治诸疮中风水肿方第十七

《僧深方》治疮中风水肿方∶炭白灰(一分)胡粉(一分)凡二物,以猪脂和涂疮肿孔上,即水出痛止,大良。

《葛氏方》治因疮而肿,皆坐中水及中风寒所作也,其肿入腹则杀人。治之方∶桑灰汁,温以渍之,大良。

又方∶烧白茅为灰,以温汤和之,以浓封,疮口干辄易之,不过四、五。(《千金方》同之。)《范汪易方》治诸疮因风致肿方∶取栎木根,便剥取皮三十斤,锉,煮令熟,纳蓝一把(一方,盐一升),令温温热以渍疮,脓血当出,日日为之则愈。(今按∶《葛氏方》无蓝有盐。《千金方》以水三石煮。)《龙门方》治凡疮中风水肿痛方∶取青葱叶,及干黄叶,和煮作汤,热浸之。

又方∶莨菪根烧令热,微切头热注疮上,冷易。

《集验方》治因疮肿剧者,数日死,或中风寒,或中水,或中狐屎棘刺方∶烧穣(如关反,禾茎也)草及牛马屎、生桑条,趣得多烟者熏之,令汁出则愈。(今案《葛氏方》黍稻穣云云。)
卷第十八
治汤火烧灼方第一

《病源论》云∶凡被烧者,初慎勿以冷物及井下泥及蜜淋KT之,其热气得冷冷却,深搏至骨,烂人筋也。所以人中火汤疮后,喜挛缩者,良由此也。

《短剧方》治猝被火烧,苦剧闷绝,不识人方∶冷水解蜜饮之。噤痉,挟口与之。

又方∶栀子膏方∶栀子(二十枚)白(五两)黄芩(五两)三物,咀,以水五升,麻油一升,合煎令水气竭,去滓,冷之,以淋疮,火热毒则去,肌皮得宽。

《葛氏方》治汤火所灼,未成疮者方∶取冷灰,以水和,沓沓尔以渍之。

又方∶破鸡子白涂之。

又方∶以豆酱涂之。

此三药皆能不痛、不成疮。

又方∶末石膏涂之,立愈。

若已成疮者方∶以白蜜涂之,竹中幕粘贴,日三。

又方∶煮大豆,煎其汁以敷之。

又方∶猪膏和米粉涂,日五、六。

又方∶以好酒洗渍之。

《极要方》疗汤火烧灼烂方∶削梨,贴,不烂易愈。

又方∶猪膏煎柳白皮,涂上。

已成疮方∶柳皮,烧作末,粉之。

《医门方》云∶凡疗汤火疮,欲涂敷膏散,先取大豆煮。令汁浓,待冷洗疮。然后涂膏散,极佳,止痛无瘢。

疗热汤膏油,火烧疮痛不忍方∶狗毛细剪,烊胶和毛涂上,痂落不痛,神秘。

《僧深方》治火疮方∶酱清和蜜涂,良。一分酱,二分蜜合和。

又方∶猪膏煮柏皮敷之。

《千金方》治火烧方∶丹参无多少,以羊脂煎涂之,神良。(今按∶无羊脂用猪脂。)又方∶死鼠一头,猪膏煎,令消尽,以敷即瘥,不作瘢。神妙。

又方∶榆白皮,嚼,涂之。

《龙门方》火烧疮方∶新出牛屎,涂,瘥。

又方∶桑柴灰和水敷,瘥。

又方∶栀子二七枚,蜜三合,渍涂,日三。

《新录方》∶捣慎火草涂之。

《录验方》∶锻石下筛,水和涂之。

《范汪方》治火烂疮蜜膏方∶食蜜(一两)乌贼鱼骨(二铢)凡二物,冶,乌贼鱼骨下筛,纳蜜中。搅令相得,薄涂疮上,日二。

《耆婆方》治人火灼烂疮长毛发方∶取柏白皮作末,和猪脂敷之,良。煮汁洗之。

《删繁方》治火疮、灸疮等膏方∶柏树白皮(五两)甘草(一两)甘草(一两)竹叶(三两)生地黄(五两)凡四物,切,绵裹,苦酒五合,淹渍一宿。用猪膏一升,煎取竹叶黄为度,去滓,摩敷疮。
治灸疮不瘥方第二

《病源论》云∶夫灸之法,中病则止,病已则疮瘥。若病热未折,或中风冷,故经久不瘥也。

《葛氏方》治火疮、灸疮终不肯燥方∶细末乌贼鱼骨,粉之。

又方∶桑薪灰水和敷之。

《千金方》治灸疮不瘥方∶日别灸上,六七壮自瘥。

《扁鹊针灸经》云∶凡灸,因火生疮,长润,久久不瘥,变成火疽方∶取谷树东边皮,煮熟去滓,煎令如糖,和散敷。

又方∶牛屎烧作灰,敷之。

又方∶兔毛烧灰,主灸疮不瘥。

《僧深方》治灸疮不瘥方∶白蜜(一两)乌贼鱼骨(二铢)二物,和调,涂疮上。

《集验方》治灸疮,薤白膏生肉止痛方∶薤白当归(各二两)白芷(一两)羊脂(一升)凡四物,咀,与脂和煎,去滓,敷之。日二。
治灸疮肿痛方第三

《病源论》云∶夫灸疮,脓溃已后,更肿急痛者,此中风冷故也。

《葛氏方》治灸疮及诸小疮,中水风寒,肿急痛方∶灶中黄土,水和,煮令热,渍之。

又方∶但以火灸之令热,热则蛘蛘止,日六七,大良,瘥。

《范汪方》治灸疮肿痛方∶取韭,捣以敷上,以火灸,令热入疮中,日三。

《医门方》疗灸疮肿急痛方∶柏白皮当归(各三两)薤白(切一升)猪膏(一升)切,以苦酒浸之三味一宿,以微火煎,三上三下,薤白黄为度。去滓,敷上,甚效。

又方∶以艾灸疮口,日六七壮便瘥。
治灸疮血出不止方第四

《病源论》云∶夫针灸,皆是节、穴、俞、募之处。若病甚,则风气冲击于疮。凡血与气相随而行,故风乘于气,而动于血,血从灸疮处出,气盛则血不止,名为发洪也。

《千金方》治针灸疮血出不止方∶烧人灰屎敷之。(今按∶熬马矢封之。)又方∶死蜣螂末,猪脂和涂。

《范汪方》治灸疮出血不止方∶莲子草汁注中止,冬月未干者敷之。(今按∶《本草》云∶鳢肠,针灸疮、发洪血不可止者,敷之立已。一名莲子草。)
治金疮方第五

《病源论》云∶夫被金刃所伤,其疮多变动。若按疮边干急,肌肉不生。青黄汁出,疮边寒清,肉消臭败,前出赤血,后出黑血。如熟烂者,及血出不止,白汗随出,如是者多凶。

若中络脉,髀内阴股,天窗眉角,横断腓肠,乳上乳下及与鸠尾、攒毛少腹,尿从疮出,气如奔豚,及脑出,诸疮如是者多凶少愈。

又云∶夫金疮,冬月之时衣浓絮温,故裹欲薄;夏月之时,衣单且凉,故裹欲浓。

《范汪方》云∶凡裹缚金疮,用故布帛,不宽不急,如系衣带。

《葛氏方》治金疮方∶急且斫桑,取白汁,以浓涂之。

又方∶烧马屎敷疮上。

又方∶以锻石浓壅裹之,止血速愈。无锻石,筛凡灰可用。

又方∶山行伤刺血出,猝无药,葛根叶敷之。

又方∶紫檀屑敷之。

又方∶即尿中良。

《千金方》云∶凡金疮若刺痛不可忍,百方不瘥方∶葱白一把,水三升,煮数沸,渍疮即止痛。

又云∶金疮烦满方∶赤小豆一升,以苦酒渍之,熬,燥复渍之,满三日,令色黑,冶,服方寸匕,日三。

《短剧方》金疮无大小,冬夏始伤血出方∶便以白灰敷之,仍裹。若疮甚深不欲便,令合者,纳少滑石。滑石令疮不时合,又止痛。

亦可纳少少牡蛎。若猝无白灰,可用凡灰。已脓中有虫,白灰敷之。日三,虫当出。故并KT刘田方,盖常秘之。

《刘涓子方》治金疮痈不可忍、烦疼不得住,止痛当归散方∶当归(一两)甘草(一两)本(一两)桂心(一两)木占斯〔(形如浓朴,有纵横纹理)一两〕凡五物,合捣下筛,水服半方寸匕,日三夜一。

《龙门方》治金疮方∶地菘草嚼敷之。(今按∶《本草》云∶路边地菘为金疮所秘。陶景注云∶捣敷之。)又方∶烧青布作灰敷之。

《极要方》疗金疮方∶生青蒿敷之,止痛,断血生肉。

又方∶牡蛎二分,石膏一分,为散,以粉疮上即止。

又云∶疮中有虫,熬杏仁捣着之。

又云∶疗刀斧诸疮方∶葛根为屑,疗金疮止血要药,亦疗虎,狗啮,饮其汁,良。

又方∶捣耐冬,封之立瘥。

《陶景本草注》治金疮方∶捣景天叶敷之。

又方∶捣薤白敷之。

《苏敬本草注》治金疮方∶捣落石敷之。

又方∶生草蒿,敷之,止血生肉。

《医门方》金疮止痛止血方∶艾叶熟,安疮上裹之,神验。

又方∶桑柴灰敷疮,止痛止血极效。

《救急单验方》疗金疮方∶嚼生粟黄敷之。

又方∶锻石和猪脂,烧令赤,涂。
治金疮肠出方第六

《病源论》云∶若中于腹,则气激,则肠随疮孔出也。

又云∶肠但出不断者,当作大麦粥,取汁持洗肠,以水渍纳之,当作研米粥饮之。二十余日,稍作强糜食之,百日后可进饭耳。

《短剧方》金疮肠胃脱出欲令入法∶取人粪干末,以粉肠上,即入。(《集验方》同之。)《删繁方》治金疮肠出方∶取桑皮线缝肠皮,用蒲黄粉之。

《刘涓子方》金疮中腹肠出不能纳方∶小麦五升,水九升,煮取四升,去滓,以绵度(漉欤)之,使极冷,旁人含逊肠上自入。

又云∶金疮肠出,欲入之磁石散方∶磁石(三两)硝石(三两)凡二物,下筛,白饮服方寸匕,日五夜再,二日入。

《葛氏方》肠出欲燥,而草土着肠者方∶作薄大麦粥,使才暖,以泼之,以新汲冷水之,肠则还入,草土辈当KT。《玉》∶从圣的反,又子陆反,畏敬也。《礼记》∶KT然避席是也,在皮外也。
治金疮肠断方第七

《病源论》云∶夫金疮肠断者,视病深浅各有死生。肠一头见者,不可连也。若腹痛、短气、不得饮食者,大肠一日半死,小肠三日死。肠两头见者,可连续之。先以针缕如法,连续断肠,便取鸡血涂其际,勿令泄,即推纳之。

《葛氏方》若肠已断者方∶以桑皮细线缝合,鸡热血涂之,乃令入。
治金疮伤筋断骨方第八

《病源论》云∶夫金疮,始伤之时,半伤其筋,营卫不通,其疮虽愈,已后仍令痹不仁也。若被疮截断诸解、身躯,肘中及腕、膝、髀,若在踝际,亦可连续。须急及热疗之,其血气未寒,即去。碎骨,便缝连,其愈后,直不屈伸。若碎骨不去,令人痛烦,脓血不绝,不能得安。诸中伤人神,十死一生。

《短剧方》金疮被筋绝令还续方∶取蟹头中脑及足中肉髓熬之,纳疮中,筋即生续之。
治金疮血出不止方第九

《病源论》云∶金疮血出不断,其脉大而止者,三七日死。血出不可止,前赤后黑,或黄或白,肌肉腐臭,寒冷靳急者,其疮虽愈亦死。

《葛氏方》∶金疮中筋交脉,血出不可止尔,则血尽杀人方∶急熬盐三指撮,酒服之。

《千金方》∶金疮血出不止,令唾之法。咒曰∶某甲今日不良,为其所伤。上告天皇,下告地王,清血莫流,浊血莫扬,良药百裹,不如熟唾。日二七度,唾之即止。(今按∶《如意方》作神若唾。)又方∶蒲黄(一斤)当归(二两)二味,筛,下酒,服方寸匕,日三。

又方∶捣车前草汁敷之。

又方∶以蜘蛛膜贴之,血即止。

《孟诜食经》治金疮血出方∶蓟叶封之。

《范汪方》金疮血出方∶以白灰浓裹之。

《耆婆方》治金疮血出方∶口嚼薯蓣以敷之,避风早瘥。

《极要方》金疮血不断方∶以熟艾敷之。

又方∶麝香末敷之。

又方∶干马矢掩之。今按∶火灸掩之,良。

《广利方》∶金疮血不止方∶骐麟竭末敷之。

又方∶斫桑树,取白汁涂之。
治金疮血内漏方第十

《病源论》云∶凡金疮通内,血多内漏,若腹胀满,两胁胀,不能食者死。瘀血在内,腹胀,脉牢大者生,沉细者死。

《葛氏方》若血内漏者方∶服蒲黄二方匕,血立下。

又方∶煮小豆,服汁五升。

又方∶以器盛汤,令热熨腹,达内则消。

又方∶掘地作坎,以水泼坎中搅之。取浊汁,饮二升许。

《千金方》金疮内漏方∶牡丹为散,水服三指撮,立尿血出。

《医门方》金疮血内漏,腹满欲死方∶白芷黄当归续断芎(各八分)甘草(六分,炙)蒲黄干地黄(各十二分)捣筛为散,空腹以酒服方寸匕,日三。瘀血化为水下。口噤,加大黄十二分。
治金疮交接血惊出方第十一

《病源论》云∶夫金疮,多伤经络,去血损气,其疮未瘥,则血气尚虚。若因而房室者,致情意感动,阴阳发泄,惊触于疮,故血汁重出也。

《葛氏方》云∶金疮未愈以交接,血漏惊出则杀人方∶急以蒲黄粉之。

又方∶取所交妇人中裙带三寸,烧末服之。
治金疮中风方第十二

《医门方》治金疮中风,,欲死方∶生葛根一斤,切,以水九升,煮取三升,去滓,分三服。无生葛,以干葛末,温酒服三指撮。若口噤,多饮竹沥亦佳。
治金疮禁忌方第十三

《葛氏方》云∶金疮,忌怒,大言,大笑,思想阴阳,行动作力,多食饮咸酸、饮酒、热羹,皆使疮痛。疮瘥后百日、半年,乃稍稍复常耳。

又云∶若多饮粥辈,则血溢出杀人。

《陶景本草注》云∶金疮禁食猪肉、梨。
治毒箭所伤方第十四

《病源论》云∶夫被弓弩所伤,若箭镞有药,入人皮脉,令人短气,须臾命绝。口噤唇干,血为断绝,肠满不言,其人如醉,未死之间,为不可治。若营卫青瘀,血应时而出,疮边壮热,口开能言,其人乃活。毒箭有三种∶岭南夷俚,用焦铜作箭镞。次,岭北诸处,以诸蛇、虫毒螫物汁,着管中,渍箭镞。此二种才伤皮,便洪肿沸烂而死。唯射猪犬,虽困得活,以其啖粪故也。人若中之,便即食粪,或饮粪汁,并涂疮即愈。不尔,须臾不可复救。

箭着宽处者,虽困渐活,不必死;若近胸腹,便宜速治,小缓,毒入内,则不可救矣。

《葛氏方》云∶治猝毒箭方∶捣蓝青,绞饮汁,并薄疮。无蓝,可渍青布及绀辈,绞饮汁,亦以汁灌疮中。

又方∶服竹沥数合至一、二升。

又方∶煮藕饮汁,多多益善。

又方∶以盐满疮中,灸盐上。

《千金方》毒矢方∶煎地黄汁,作丸服百日,矢当出。

又方∶煮芦根汁,饮一、二升。

《短剧方》猝被毒箭方∶舂蓝汁饮之,亦灌疮箭,得蓝即醒。

又方∶服蒲黄二合许,血亦下。

又方∶服麻子汁数升。

《范汪方》治毒箭所伤方∶掘葛根食之,如常食法,务多为佳。(《千金方》饮汁。)又方∶干姜、蓝青、盐等分,捣和敷疮上,毒皆出。

又方∶末雄黄,敷疮,疮当沸汗流便愈。

《集验方》治兵疮医不能治方∶剥桑白皮,去上黑者,以裹之,桑白汁入疮。冬月用桑根皮汁。
治箭伤血漏瘀满方第十五

《录验方》治射箭HT入腹破肠中血满葵子汤方∶取葵子一升,小便四升,煮取一升,顿服下出,即瘥。

又云∶治被箭血内漏、腹中瘀满瓜子散方∶干姜(二两)瓜子(二两)凡二物,冶筛,先食,酒服方寸匕。

又方∶芦茹散∶芦茹(三两)杏仁(二两)凡二物,冶,先食,酒服方寸匕。
治箭镞不出方第十六

《病源论》云∶箭中骨,骨破碎者,须令箭镞出,仍应除碎骨尽,乃敷药。不尔,疮永不合,纵疮合,常疼痛。若更犯触损伤,便惊血沸溃有死者。

《葛氏方》治箭镝及诸刀刃在喉咽胸膈诸隐处不出方∶捣杏仁涂之。

又方∶以蝼蛄脑涂之。

《千金方》治金箭不出方∶白、半夏等分,末。酒服方寸匕,日三。〔今按∶《录验方》∶白蔹三两,半夏(干)三两。熬下筛,水服方寸匕,日三。轻浅疮十日出,深疮二十日出,终遂不得停肉中。〕《短剧方》治箭金在喉咽胸背膈中及在诸处不出方∶牡丹(一分)白(一分)末,酒服方寸匕,日三。自出。(今按∶《千金方》白二分。)又方∶取妇人月经衣已污者,烧末,酒服方寸匕,日三,立出。(《集验方》同之。)《录验方》治箭镞入腹中不出瞿麦散方∶末瞿麦,酒服方寸匕,日三,夜再,亦可治百刺。亦和酒涂。

又云∶箭入人身,经三、五年不出方∶麻子三升,作末,以水和,使得三升汁,温服之,须臾出。

《龙门方》疗箭镞入腹不出方∶栝蒌捣敷疮上,日三,自出。
治铁锥刀不出方第十七

《葛氏方》治铁入骨不出方∶取鹿角烧作灰,猪膏和敷之。

《录验方》治箭镞及兵刃锥刀,刺折在身中,不出方∶白芷(三分)白蔹(三分)凡二物,冶筛,酒服一刀圭,日三。

又云∶治锥刀入腹方∶梨花煮取汁,服之,大良。
治医针不出方第十八

《录验方》治医针不出方∶捣杏仁涂之。

《短剧方》治箭金及折针不出方∶以鼠脑涂之。

又方∶以鼠妇涂之。

又方∶以蝼蛄脑涂之。

《医门方》疗箭、医针在肉中方∶细刮象牙屑,以水和之如杏,着折针上即出,亦疗竹木刺不出者。《短剧方》同之。

《龙门方》治针不出方∶烧羊毛作灰,和猪脂敷上,半日自出。
治竹木壮刺不出方第十九

《葛氏方》诸竹木刺在肉中不出方∶用牛膝根茎合捣以敷之,疮口虽合自出。

又方∶烧鹿角末以水和涂之,立出。远久者,不过一宿。

又方∶捣乌梅,水和涂上,立出。(今按∶《集验方》用白梅。)又方∶嚼豉涂之。

《录验方》诸竹木刺壮不出方∶末王不留行,服即出。

又方∶鹿脑浓敷,干复易。无鹿脑者,用鼠脑。
治被打伤方第二十

《病源论》云∶夫被打,陷骨伤脑,头眩不举,戴眼直视,口不能语,咽中沸声如豚子喘,口急,手为妄取,即日不死,三日少愈。

《短剧方》治为人所打击,若见镇头破脑出已死,尚有气在胸心间方∶取豚血,及热,以灌脑中,令满疮里。无豚者唯趣得热血而灌之。

又方∶服水银,如大豆,即活。

《千金方》治头破脑出,中风口噤方∶大豆一升,熬去腥,勿使太熟,捣末熟,蒸之,气匝合甑,下盆中,以酒一斗淋之。温服一升,覆取汗,敷杏仁膏。

又云∶被打伤有瘀血方∶蒲黄(一升)当归(二两)桂心(二两)三味,酒服方寸匕,日三。

又方∶豉一升,以水三升,煮三沸,分再服。

又方∶生地黄汁三升,酒一升半,煮取二升合,分三服。

又云∶治瘀血在腹,内服大小蓟汁五六合。

《葛氏方》治蹴倒,有损痛处气急面青方∶干地黄半斤,酒一斗渍,火温。稍稍饮汁,一日令尽之。

又方∶捣生地黄汁二升,酒二升,合煮三沸,分四、五服。

又方∶干地黄六两,当归五两,水七升,煮取三升,分三服。若烦闷用生地黄一斤,代干者。

又云∶治为人所玉摆搅举身顿仆垂死者方∶取鼠李皮削去上黑,切,酒渍半日,绞去滓,饮一、二升。

又云∶若为人所打、举身尽有瘀血者方∶刮青竹皮二升,乱发如鸡子大四枚,火炙令焦,与竹皮合捣末,以一合纳酒一升中,煮三沸,顿服之,日四、五过。又纳蒲黄三两。

又云∶血聚皮肤间不消散者方∶取猪肥肉,炙令热,以拓上。

又方∶马矢水,煮敷上。

又云∶被击打瘀血在腹内,久不消,时时发动者方∶大黄、干地黄末,为丸散,以酒服。

又方∶蒲黄一升,当归二两,末,酒服方寸匕,日三。

又云∶若久血不除,变成脓者方∶大黄三两,桃仁三十枚,杏仁三十枚,酒水各五升,煮取三升,分三服,当下脓血。

《新录云》治头破方∶猪脂和锻石及盐,烧为灰,敷上。

又方∶生地黄不限多少,熟捣薄伤处。

《刘涓子方》治被打腹中瘀血白马蹄散方∶白马蹄烧令烟尽,捣筛,温酒服方寸匕,日二夜一。(今按∶《广利方》云∶血化为水即下。)《范汪方》去血汤,主肠中伤积血方∶煮赤小豆二升,合得汁二升,以淳苦酒七升,合和汁中,饮一日尽之。状如热汤沃雪,即消下,甚良。
治折破骨伤筋方第二十一

《病源论》云∶凡人伤折之法,即夜盗汗者,此髓断也,七日死;不汗者,不死。

《短剧方》治折四肢骨方∶若有聚血,在折上以刀破去之,不可冷食也。舂大豆,以猪膏和涂聚血上,燥复易之。

又方∶烧鼠屎猪膏,和敷血上甚良。

《葛氏方》∶凡折折骨诸疮肿者,慎不可当风卧湿及自扇,中风则发痉口噤杀人。若已中此,觉颈项强,身中急者方∶急作竹沥饮二、三升。若口已噤者,以物强开发内也,禁冷冻饮料食及饮酒。

又云∶折四肢破,骨碎及筋伤跌方∶熟捣生地黄以敷折上,破竹简编之,令竟病上,急缚之。一日一夕,十易地黄,三日后则瘥。

又方∶活鼠破其背,取血,及热以敷之,立愈。(今按∶《极要方》∶神方,云云。)《千金方》∶治四肢骨破碎筋伤蹉跌方∶水二升,渍三升豉,取汁服之。

又方∶初破时,以热马屎敷之,无瘢。

又方∶大豆二升,水五升,煮取二升,淳苦酒六、七升合豆汁服之,一日尽之,如汤沃雪。

又方∶生地黄,不限多少,熟捣用敷损伤处。

《极要方》疗手脚折方∶取生地黄熟捣以敷折上,破竹木编之,急缚之,一日一夜十易地黄,三日后则瘥。

又云∶疗伤折筋骨疼痛方∶上以酒煮折伤木,浓汁饮之。

《新录方》∶挫苏方木二升,以水二升,酒二升,煮取一升六合,二服。

又方∶接骨木者,煮服依苏方木法。(今按∶接骨木水煮洗之,又水扬煮汁洗浴之。)
治从高落重物所方第二十二

《葛氏方》治人从高堕下若为人重物所填得瘀血方∶豉三升,以沸汤二升,渍之食顷,绞去滓,以蒲黄三合投中,尽服,不过三、四服,神良。

又方∶取茅、蓟、莲根叶捣绞,服汁一、二升,不过三、四服,愈。

又方∶末鹿角,酒服三方寸匕,日三。

又云∶猝从高落下,瘀血振心,面青短气欲死方∶地黄干生无在,随宜用服,取消。

又方∶煮大豆若小豆令熟,饮汁数升,酒和弥佳。

又云∶为重物所填欲死方∶末半夏如大豆者,以纳其两鼻孔中,此即五绝法。

《短剧方》治从高堕下,腹中崩伤瘀血满,断气方∶服蒲黄方寸匕,日五、六过。(今按∶《龙门方》∶和酒服。)又方∶舂生地黄酒沃取汁,稍服,甚良。

又云∶治从高堕,若为重物所镇迮得瘀血方∶作大豆紫汤,如产妇法服之。

《千金方》从高堕折,疼痛,烦躁,啼叫不得卧方∶取鼠屎烧末,筛,以猪膏和涂痛上,即安。

又云∶从高堕下崩中方∶当归二分,大黄一分。二味,酒服方寸匕,日三。

《极要方》疗因堕损恐内有瘀血方∶服虎魄屑,神验。能治瘀血。

《医门方》疗猝堕损,筋骨蹉跌或骨破碎方∶熟捣生地黄敷之,日夜数数易之,若血聚者针决去之。

又方∶浸地黄酒饮之,令酒气不绝,佳。
治从车马落方第二十三

《葛氏方》∶治忽落马堕车及堕屋坑岸,伤身体、头面四肢,内外切痛,烦躁叫唤者方∶急多觅鼠屎,烧捣,以猪膏和涂痛处,急裹之。

《千金方》堕落车马,心腹积血,唾吐无数方∶干藕根末,酒服方寸匕,日三。(今按∶《苏敬本草经》∶煮藕根汁三浸之。)又方∶酢和面敷上。

《极要方》疗堕马崩血,腹满短气欲死方∶大豆五升,以水一斗,煮取二升半,一服令尽。剧者不过再服即愈。

《新录方》云∶捣生地黄封之。
治犬啮人方第二十四

《病源论》云∶凡犬啮人,七日辄一发,过三七日不发,则无苦也。要过百日,方为大免耳。当终身禁食犬肉及蚕蛹,食此,发则死不可救矣。疮未愈之间,禁餐生鱼、猪、鸡、肥腻,过一年禁之乃佳。但于饭下蒸鱼,及于肥器中食便发。若人曾食落葵,得犬啮者,自难治。若疮瘥十数年后,食落葵便发。

《录验方》云∶犬食马肉生疮者,当急杀之,多令犬也。亦可捣枸杞根,取汁以煮米,与犬食之则不。

《短剧方》云∶禁饮酒、食猪肉、生菜、脍、。

《葛氏方》云∶凡狗春月自多。治之方∶以豆酱涂疮,日三、四过。

又方∶末干姜,常服。少少并以纳疮中。

又方∶即末矾石,纳疮中裹之,止痛不坏,速愈,最良。

又云∶若重发者方∶捣芦根饮汁即瘥。

又方∶薤白捣饮汁良。

《经心方》治犬啮人方∶以人屎涂之,大良。

又方∶验酢以壅疮上即瘥。

《短剧方》治狗啮人方∶嗽去其恶血,灸其处百壮,以后当日灸百壮。血不出者,小刺伤之,灸百壮乃止。(今按∶《葛氏方》∶顿灸十壮,明日以去日灸一壮,满百日乃止。)又方∶取杏仁,熬令黑,冶,着疮中佳。

《千金方》治犬啮人方∶捣韭,绞取汁,饮一升,日三,亦疗已愈而后发者。(今按∶《葛氏方》又每到七日饮之。)又方∶取灯残油灌疮中。

《医门方》疗犬咬人方∶又方∶栀子皮烧末,石榴黄末,等分,敷疮上,日二,速瘥。

又云∶若诸疗不瘥,吐白沫,毒攻心,叫唤欲似犬声者方∶取人髑髅骨,火烧作灰,下筛,以东流水服方寸匕,须臾二、三服,即瘥。起死人如神。

《极要方》狂犬伤人方∶以蚯蚓屎封之,出犬毛,神效。

《新录方》云∶捣生艾叶汁,服七八合。又方∶车脂涂之。

又方∶捣生葛根,取汁,服七合,又末敷之。

《录验方》云∶取大蒜作饼,灸疮上,愈。

又方∶火消腊蜜,着疮中。
治凡犬啮人方第二十五

《病源论》云∶凡被狗啮疮,忌食落葵。虽瘥,经一、二年亦重发。

《葛氏方》治凡犬咋人方∶以沸汤和灰,以涂疮上。又苦酒和涂之。

又方∶蓼,以敷疮。冬月煮洗之。

又方∶以热牛屎涂之。

又方∶捣干姜,服二方寸匕。

又方∶生姜汁饮半升,佳。

又方∶以头垢少少纳疮中。

《集验方》治凡犬咋人方∶以火炙腊,灌疮中。

又取灶中热灰,粉疮中,裹敷,立愈。(《葛氏方》同之。)《医门方》凡犬啮人方∶取杏仁熬熟,捣敷疮,频易。

又方∶鼠屎二七枚,烧灰之,敷上,永瘥。《经心方》∶烧犬尾末敷疮上,日二,良。
治马咋人方第二十六

《病源论》云∶凡人被马啮及马骨所伤刺,并马缰、勒所伤,疮皆为毒疮。若肿痛致烦闷,是毒入腹,亦毙人也。

《葛氏方》治马咋人及人,作疮有毒,肿热疼痛方∶割鸡冠血,沥着疮中三下,若父马,用雌鸡;草马有雄鸡。

又方∶灸疮中及肿上。

又方∶以月经敷上最良。

《极要方》疗马啮人及踏人,作疮毒肿热疾痛方∶马鞭梢长二寸,鼠屎二七枚,合烧末,以膏和涂之,立验。

《短剧方》治马咋方∶捣车前草叶敷之。

《经心方》治马咋方∶末雄黄,敷疮上,日一。

又方∶用铜青,敷疮中好。

《陶潜方》云∶制马啮人方∶取僵蚕屑,涂马上唇,则不能啮也。
治马啮人阴卵方第二十七

《集验方》治马啮人阴卵脱出方∶推纳之,以桑皮细缝缝之,取乌鸡肝,细锉涂之。且忍,勿即小便,便愈。(《千金方》同之。)
治马骨刺人方第二十八

《医心方》治马骨刺人毒欲死方∶以妇人月经血,敷之即瘥。

《新录方》治马骨刺人方∶松叶,水煮取汁洗之。

又方∶水煮大豆,取浓汁洗之。

又方∶煮蓝取浓汁洗之,并服汁五、六合。

《删繁方》治马骨刺人方∶烧干马屎,粉疮孔中。

治马毛、血汗垢、屎、
尿入疮方第二十九

《短剧方》∶治马骨所刺,及为马所咋,为马汗、血、毛、垢、屎、尿入人疮中。及人有疮而近马物,毒瓦斯入疮中。先针刺伤出新血数过,漱去之。

研豉作汤,令小沸,以渍疮。

又方∶可用热灰汁。

又方∶煮马苋草洗之,并服汁。

又方∶以车前草叶,捣敷之。

《集验方》治马血入人疮中方∶以人粪敷疮中。

又云∶治马汗、马毛入人疮中肿痛欲死方∶以水渍疮。数易水,便愈。

《葛氏方》云∶人体先有疮,而以乘马,马汗若马毛入疮。或为马气所蒸,皆致肿痛烦热,入腹则杀人方∶大饮淳酒,取醉则愈。

又方∶煮豉作汤,及热渍之,冷易。

又云∶为马骨所刺及马血入人故疮中毒痛欲死方∶以热灰汁,更燔渍之,常令热,竟日为之,冷即易,数日乃愈。若疮(痛欤)止而肿不消者。炙石熨之,灸上亦佳。

又方∶捣麻子,绞饮其汁一升,日三。
治熊啮人方第三十

《葛氏方》治熊虎疮方∶烧青布以薰疮口,毒即出,仍煮葛根令浓,以洗疮,日十过,并葛根捣筛,以葛根汁服方寸匕,日五。疮甚者,夕一服。

又方∶削楠木,煮以洗疮,日十过。
治猪啮人方第三十一

《千金方》治猪啮人方∶松脂炼之,粘贴。

又方∶屋雷中泥敷之。
治虎啮人方第三十二

《短剧方》治虎毒方∶烧青布以熏疮口,毒则出去。

又方∶烧妇人月水污衣,末,敷疮中。

又方∶嚼栗涂,神良。

《千金方》治虎疮方∶煮葛根洗之,十遍,复饮汁。

又捣散,服方寸匕,日五夜二。

又方∶煮铁令浓,洗之。

《医门方》疗虎咬人方∶取青布急卷,为缠绕,一头令燃,纳竹筒中,注疮口,令烟熏入疮中。极佳。

又方∶但饮酒,恒令醉,当吐毛出。

《枕中方》治虎野狼所啮疮方∶取灶中黄土,好苦酒和,敷疮上,当有汁出。良。
治狐尿毒方第三十三

《病源论》云∶夫野狐尿棘刺头,人误犯之者,则中其毒,多着手足指,肿痛热,有端居不出。着此毒者,则不亦是狐尿刺也,盖恶气耳。故方亦云恶刺毒。

《葛氏方》治狐尿棘刺人肿痛欲死方∶破鸡子以拓之,良。

又方∶以热桑灰汁渍之,冷复易。

《集验方》治狐尿刺方∶取猪脂,临烛上,以火烧之,令脂堕所患处一滴,愈。

又云∶治恶刺方∶取夜光骨、玉女臂烧作灰,和腊月猪脂敷之。夜光骨,烛烬是也,玉女臂,猪髀是也。

又方∶取大豆汁,捣蓼和敷之。

又方∶杏仁研和水煮浸之。

又方∶取夜光骨捣,纳疮中,虫出即瘥。

《枕中方》治狐刺疮方∶取葵子煮取汁洗之。

《龙门方》治狐尿刺方∶取槐白皮,煮汤渍之,验。

又方∶大麦烧灰和蜜涂。

又方∶蚁穴中出土七粒,和酢涂,验。

《杂新录方》治狐尿刺方∶生麻叶,捣封,数易。

又方∶水煮蔓荆子汁,洗之。

又方∶捣萝菔根封,日易。

又方∶捣水杨叶,敷之。

又方∶酢和鼠屎灰敷之。

又方∶捣蒜如泥,熬热熨之。

又方∶水煮苦参汁洗之。

又方∶烧艾熏之。

又方∶牛屎敷之。
治鼠咬人方第三十四

《医门方》∶疗人被鼠咬,诸处皆肿,经年月不瘥,其咬处有赤脉者是也。

豆蔻十二枚,合皮切,以水二升,煮取一升,去滓,顿服了,并嚼敷疮上,立瘥。
治众蛇螫人方第三十五

《病源论》云∶凡中蛇,不应言蛇,皆言虫,及云地索,勿正言其名也。恶蛇之类甚多,而毒差剧。时四月、五月,中青;三月,苍虺、白颈、大蜴;六月、七月,中竹狩、艾蝮、黑甲、赤目、黄口、反钩、白、青角,此皆蛇毒之猛者,中人不即治多死。又云有赤、黄颔之类,有六七种。水中黑色者,名公蛎,山中一种亦相似,并不闻螫人。

又云∶有钩蛇,尾如钩,能倒牵人兽入水,没而食。

又云∶南方有钩蛇人忽伤之,不死,则终身伺觅其主,不置。虽百人众中,亦直来取之,唯远去出百里乃免耳。

又云∶有拖蛇,长七、八尺,如船拖状,毒人必死。即削取船,煮汁渍之,便瘥。

又云∶毒,此是诸毒蛇,夏日毒盛不泄,皆啮草木,及吐毒着草木上,人误犯着此者,其毒与被蛇螫不殊,但疮肿上有物如虫蛇眼状,以此别之。

又云∶凡蛇疮未愈,禁热食,热食便发。

《本草》云∶蛇虺、百虫毒,雄黄、巴豆、麝香、干姜并解。

《葛氏方》云∶中蛇毒,勿得渡水,渡水则痛甚于初螫,虽车船亦不免。

又云∶治众蛇螫人方∶捣草以敷疮上,立愈,神良。

又方∶捣生蓼绞取汁饮少少,以滓敷之。

又方∶青蓝敷之。

又方∶嚼干姜敷疮上。

又云∶治蛇疮败经月不愈方∶先以盐汤洗去疮中败肉,见血止,取千釜KT草,捣筛以敷之则愈。

又云∶治蛇螫人疮已合愈,而余毒在肉中淫之痛痒方∶取小、大蒜各一升,合捣,热汤淋之,以汁灌疮良,舂薄亦良。

又云∶治蛇螫人若通身洪肿者方∶取糠四五斗,着大瓮中,以水泼之,令上未满五升许,又以好酒泼之,以置火上,令沸气出,熏疮口,使毒出则消。

《僧深方》治众蛇螫人方∶以头垢着疮中,大良。

《极要方》蛇螫人方∶含口椒、苍耳苗,合捣以敷疮上。

又方∶生椒三合,好豉四两,以人唾和捣敷,立定。

《集验方》治众蛇螫人方∶捣大蒜涂之即愈。

《广利方》蛇咬疮方∶雄黄(四分)干姜(六分)麝香(一分,研)捣,筛,以验醋和涂疮上。

又方∶暖酒淋洗,日三,良。

《龙门方》蛇螫方∶蜂巢烧灰封瘥。

又方∶捣梨敷之。(《耆婆方》同之。)又云∶毒入腹者方∶羊蹄草叶一握,捣汁饮,吐瘥。(《耆婆方》同之。)《耆婆方》恶蛇所螫方∶取苦莒菜,捣敷螫处。又饮汁一、二升即瘥。

又方∶捣车前草根、茎,敷,验。

《苏敬本草注》云∶众蛇螫人,捣茺蔚敷之。

又云∶蛇螫人,通身肿,樱桃叶捣封之。又绞汁服之。

《医门方》治蛇虺螫人方∶急灸螫处二七粒,燃以雄黄,麝香末敷之,日五、六,无药但灸之。

又方∶荏叶熟捣,猪脂和敷上,立愈。
治蝮蛇螫人方第三十六

《病源论》云∶凡蝮蛇中人,不治,一日死。若不早治之,纵不死者,多残断人手足。

蝮蛇形不乃长,头扁口尖,颈斑,身亦艾斑,色青黑,人犯之,颈腹贴着地者是也。其毒最烈。

又云∶有一种,状如蝮而短,有四脚,能跳来啮人,东人名为千岁蝮,中人必死。然啮人竟,即跳上树,作声云∶“斫(之若反)木、斫木”者,但营棺。若云∶“博叔,博叔”者,犹可急治。

又云∶虺形短而扁,身亦青黑色。六、七月中,夕时出路上,喜入车辙,腹破而子出。

人侵晨及冒昏行者,每须作意看之,其螫人有死者。

《葛氏方》治蝮蛇螫人方∶捣小蒜,绞饮其汁,以滓敷疮。

又方∶捣韭敷之又方∶嚼盐,唾疮上讫,灸疮中三壮,复盐嚼以敷疮。

又方∶细末雄黄以纳疮中,三、四敷之。

《集验方》治蝮蛇螫人方∶令妇人溺所螫上。

又方∶令妇人坐上。

《千金方》治蝮螫方∶灸上三七壮。

又方熟捣葵,取汁服之。

《苏敬本草注》蝮蛇螫人方∶捣落石,绞汁洗之,并服良。

《耆婆方》蝮蛇螫人方∶干姜屑敷之。

《广济方》治毒蛇啮方∶取慈菇草根,捣,敷之即瘥。其草生水中,如燕尾,大效,勿轻。

《极要方》疗蝮蛇疮方∶酢磨蚤休敷之。

又方∶捣水蓼敷之,并服汁。

《集验方》治蛇虺诸毒螫方∶火消腊以着疮中。
治青蛇螫人方第三十七

《病源论》云∶青蛇者,正绿色,喜绿树及竹上,自挂树竹色一种,人看不觉,若人林中行,有落人颈背上者,然自不甚啮人,啮人必死。此蛇无正形,大者不过四、五尺,世人皆呼为青条蛇,言其与枝条同色,乍看难觉,其尾二、三寸,色黑者,名。毒最猛烈,中人立死。

《葛氏方》云∶青中人,立死。竹中青蛇螫人方∶灸疮中三壮,毒即不行也。猝无艾,刮竹皮及纸,皆可以丸,又了无此者,便以火烬就热烧疮。

又方∶破乌鸡冠血,及热以拓疮上。

《范汪方》治青蛇螫人方∶雄黄干姜末敷疮,良。
治蛇绕人不解方第三十八

《葛氏方》治蛇猝绕人不解方∶以热汤淋之即解。若无汤者,令人就溺之亦解。(今按∶《千金方》、《集验方》同之。)
治蛇入人口中方第三十九

《葛氏方》治蛇入人口中不出方∶以艾灸蛇尾即出。若无火者,以刀周匝割蛇尾,裁令皮断,乃引之,皮倒脱得出。

《千金方》蛇入人口中不出方∶以刀破蛇尾,纳生椒三、四颗,须臾即出。
治蛇骨刺人方第四十

《葛氏方》治蛇螫人,牙折入肉中,不出,痛不可堪方∶取虾蟆肝,以敷上,立出。(《短剧方》同之。)又云∶蛇骨刺人,毒痛肿热,与蛇螫无异方∶以铁精如大豆者,以管吹纳疮中。

又方∶烧死鼠捣末敷疮中。

《僧深方》治蛇牙折肉中不出方∶取生鼠热血涂疮,以绵包之,二日出。

又∶蛇骨刺人,取雄黄中大豆,纳疮中。
治蜈蚣螫人方第四十一

《病源论》云∶蜈蚣,此即百足虫也。虽复有毒,而不甚螫人。人误触之者,故时有中其毒者耳。

《本草》云∶蜈蚣毒,用桑根汁解。(今按《新录方》云∶立痛止。)《葛氏方》∶蜈蚣自不甚啮人,其毒亦微殊,轻于蜂。今赤足螫人,乃痛于黄足,是其毒烈故也。亦是雄故也。治之方∶以盐拭疮上即愈。

又方∶头垢少许,以苦酒和涂之。

又方∶破大蒜以揩之。

又方∶蓝汁渍之即愈。

《短剧方》云∶割鸡冠血涂之。

又方∶以盐汤渍之即愈。

《新录方》蛇衔叶,捣如泥,封之。

又方∶荏叶,捣如泥,封上。

《僧深方》云∶消蜡蜜浸伤中良。

《医门方》云∶鸡屎和醋,涂上便愈。
治蜂螫人方第四十二

《病源论》云∶蜂类甚多,而方家不具显其名。唯地中大土蜂最有毒,一枚中人,便即倒闷,举体洪肿,诸药治之。皆不能猝止,旧方都无其法,然虽不能杀人,有以禁术封唾亦微效。又有瓠(胡故反),KT(落都反)蜂,即亦其次,余者犹瘥。

《本草》云∶蜂毒用蜂房,蓝青解。

《葛氏方》治蜂螫人方∶取人尿洗之。

又方∶斫谷树,取白汁涂之。(《极要方》同之。)又方∶煮蜂房洗之,又烧末,膏和敷之。

又方∶刮齿垢涂之。

又方∶蓝青尖叶者,涂之。

《短剧方》治蜂螫人方∶取蜘蛛涂疮上。

又以活蜘蛛放毒上,其自嗽毒。

《千金方》蜂螫人方∶蜜(五合)腊(二合)猪脂(五合)和煎稍稍食之。

又方∶烧牛屎灰,苦酒和涂之。

又方∶嚼盐涂之。

《极要方》云∶斫桑树白汁以涂之。
治蛎螫人方第四十三

《病源论》云∶虿虫,方家亦不能的辨正,云是蜒子,或云是小乌虫。尾有两岐者,恐非也,疑是蝎,蝎尾岐而曲上,故周诗曰∶彼都人士,卷发如虿也。

《葛氏方》云∶厉字应作虿字,所谓蜂虿。治之方∶捣常思草,绞取汁以洗之。

又方∶灸疮中十壮。

又方∶炙屋瓦,若瓦器令热以熨之。

《短剧方》蛎螫人方∶取屋雷下士,以水和,敷之立愈。
治蝎螫人方第四十四

《病源论》云∶蝎此虫五月、六月毒最盛,云有八节、九节者毒弥甚。但中人毒热流行,牵引四肢皆痛,过一周时始定。

《葛氏方》∶蝎,中国屋中多有,江东无也。其毒应微,今石榴树多有蚝虫云云。

治之方∶温汤渍之。

又方∶马苋涂之。

又方∶嚼大蒜涂之。

又方∶嚼干姜涂之。

《千金方》云∶取齿中残饭敷之。

又方∶砂和水涂,立愈。

又方∶猪脂封上。

又方∶以井底泥敷之。

《新录方》云∶煮甘草汁服之。

又方∶酱汁涂之。

又方∶尿泥涂之。

又方∶捣芥子末酢和涂之。

又方∶浓煮盐汁洗之。

又方∶艾灸上二七壮。

《广济方》云∶半夏,以水研涂之立止。

又方∶涂黄丹之。

《龙门方》云∶温酢渍瘥。

《极要方》云∶嚼人参敷之,立验。

又方∶削桂心以酢磨涂之。

《广利方》云∶猫儿粪涂螫处,日三。

又方∶破蜘蛛汁涂,立止,时始定。

《集验方》云∶蝎有雌雄,雄者,痛止在一处;雌者,痛牵诸处,若雄者用井底泥敷之,温复易。雌者,用当屋瓦沟下泥敷之,若不值天雨,无泥,可用新汲井水从屋上淋于下取泥敷之。

《苏敬本草注》云∶捣麻叶敷之。
治蜘蛛啮人方第四十五

《本草》云∶蜘蛛毒用蓝青、麝香并解。

《拾遗》云∶芜菁子和油敷蜘蛛咬,毒入内。亦为末酒服,蔓荆园中无蜘蛛是其相畏也。

又云∶土蜂赤黑色,烧末,油和敷蜘蛛咬疮,此物食蜘蛛,亦其相伏也。

《千金方》云∶乌麻油和胡粉如泥涂之,干则易之。

《广济方》云∶取生铁上衣,用醋研取汁涂之。

《医门方》云∶蜘蛛咬人,经年不瘥方∶半夏苗敷,大效。

《极要方》云∶白僵蚕末,以唾和之涂上。

又方∶取萝草捣如泥封上。

又方∶柳皮一两,半夏一两,烧作灰涂之。

《传信方》云∶疗蜘蛛咬,遍身生丝方∶取羊乳一味久服,愈为度。

贞元十一年,余偶到奚吏部宅坐客,有刑部崔从质因话此方。崔云,目击有人被蜘蛛咬腹,大如有娠,遍身生丝,其家弃之,乞食于道,有僧遇之,教饮羊乳,得俞平状。
治蛭啮人方第四十六

《病源论》云∶山中草木及路上及石上,石蛭着人则穿啮肌皮,行人肉中浸淫起疮。

《千金方》云∶灸断其道即愈。

又云凡山行草中,常以腊月猪膏和盐涂脚胫及足趾间,及着鞋,蛭不得着。
治蚯蚓咬人方第四十七

《传信方》疗蚯蚓咬方∶常浓作盐汤,数浸洗而愈。

浙西军将张韶,为此虫所啮,其形如患大风,眉鬓皆落,每夕则蚯蚓鸣于体中,有僧遇诸途,教用此法,寻愈。
治蛞蝓咬人方第四十八

《本草拾遗》云∶蛞蝓咬,取蓼捣敷疮上及浸之。今按∶《本草》云∶蛞蝓一名云蜗。(和名奈知女久知。)
治螈蚕毒方第四十九

《病源论》云∶蚕既是人养之物,性非毒害之虫,然时有啮人者,乃令人憎寒壮热,经时不瘥,亦有因此致毙,斯乃一时之怪异,救解之方愈。

《新录方》云∶螈蚕毒方∶酢和鼠屎如泥,涂上。

又方∶酢和鸡屎灰封之。

《广济方》疗蚕及蜘蛛啮方∶葶苈子(四分)蛇床子(四分)菟丝子(四分)盐(四分)上捣筛,和三年验醋如泥,涂疮上,日三。
治射工毒方第五十

《病源论》云∶江南有射工毒虫,一名短狐,一名蜮,常在山涧水内。此虫口中内有横骨,如角弓形,正黑,如大蜚,生齿发,而有雌雄。雄者,口边有两角,角端有桠,能屈伸,冬月在土内蛰,其上气蒸休休,有雪落其上不凝,夏月在水内,人行水上及以水洗浴,或因大雨潦时,逐水流入人家,或遇道上牛马迹内即停住,含沙射人影,便病。初得,或如伤寒,或如中恶,朝旦小苏,哺夕辄剧,始得三、四日,尚可治,急者七日,缓者二七日,远者三七日皆死。初未有疮,但恶寒寒热,盛如针刺,甚成疮,或如豆粒黑子,或如火烧,或蠼尿疮,皆肉内有穿孔,如火针孔也。

《抱朴子》云∶短狐,一名蜮,一名射工,一名射影。其实水中状,似鸣蜩而大如三合杯。有翼能飞,无目而利耳云云。

又云∶射工,冬天蛰于谷间,大雪时索之此虫所在其上,无雪,气起如炊蒸,当掘之,不过入地一尺则得之,阴干末带之,夏天辟射工也。

《短剧方》云∶射公,二名短狐,三名溪毒。其虫形如甲虫,有一长角横在口前,如弩檐临其角端,曲如上弩,以气为矢,因水势以射人。射人时,人或闻其在水中,铋铋作声也。

要须得水没其口,便能射人。在无水之地,便可捉持戏,无能为害也。此虫从四月始生,至五月,六月其毒尚微,中人犹缓。七月、八月其毒大盛,中人甚急。入九月许,有寒露微霜者,其毒向哀,至十月乃息也。此虫畏鹅,鹅能食之,闻鹅声便不敢来也。水上流,有鹅浴气响,鹅屎、毛羽流下,其便走去也。船行入溪,宜将鹅自随。山溪家居,皆养白鹅也。凡入山溪水中采伐者,装鹅屎,带鹅毛以避之也,此虫利耳而盲目,凡人入溪源取水及浴,经涉渡溪者,皆宜以木石遥掷水中作声,其虫应声虚射,毒便泄去也。掷水法,唯多过左右,广掷之,虫放毒尽,然后过溪也。山源之间,多有此虫,大雨洪潦之时,其逐水流落之人家,及道上牛、马迹小水中停住。人行陆地,多不意悟,或逐柴薪竹木,来人喜不觉之,得病便作他治,乃误致死也。射公中人,初始证侯,先寒热恶冷吹KT,筋急痉强,头痛目疼,状中伤寒,亦如中尸,便不能语,朝旦小苏,晡夕辄剧,寒热闷KT,是其证也。初得三、四日,尚可活,急者七日死,缓者至二七日,远不过三七日,皆死也。此虫有大小,小者毒微,射人不即作疮,人多不知是射公毒。其大者,毒猛,中人乃即成疮耳。其疮初或如豆粒黑子,或如火烧,或如尿。或痛如针刺,而未见疮处,或成疮,皆肉中有穿空,如火针孔也。

其射人远者及中人影者,毒小宽也。射工中人腰以上,去人近者多死。中人腰以下者小宽也。

纵不死者,皆百日不可保瘥耳。治之如左。

治射公中人寒热,或发疮,或偏在一处有异于常方∶取赤苋合茎叶捣之,绞取汁,服七合,日四、五服良。此是苋菜之赤大者也。

又方∶单煮犀角,饮以折其毒热也。

治射工中人方∶取鬼臼目叶一把,纳苦酒中沾湿之竟,熟捣绞取汁,服一升,日三。

又方∶犀屑(二两)乌扇根(二两)升麻(三两)凡三物,咀,以水四升,煮取一升半,分再服,相去一炊顷,尽,更作。

又方∶取水菘,捣绞取汁,服一升,日三即瘥。

又方∶取生茱萸茎,叶一虎口,握之断去握前后,余握中央,便熟捣绞,以水三升,煮取一升七合,顿服之,神效。

治射公中人已有疮者方∶取蜈蚣大者,一枚,小炙之,捣末,苦酒和敷疮上良。

又方∶取斑蝥一枚,火烧捣末,苦酒和敷上。

又方∶取芥子捣令熟,苦酒和浓涂疮上半日。

又方∶取野狼牙叶,冬取根捣之令熟,敷所中处,又饮四、五、六合汁,防毒恐入内也。

《集验方》治射公中人疮,令人寒热方∶乌扇根(二两)升麻(二两)凡二物,以水三升煮得一升,适寒温,尽服之,滓敷上。

《葛氏方》治射工中人方∶初见此疮,便水摩犀角涂之,燥复涂勿住。

又方∶急周绕去疮一寸,辄一灸,灸一处百壮,疮上亦灸百壮。

又方∶切葫,拓疮上,灸葫上十壮,并取常思草捣绞汁,饮一、二升,以滓敷之。

又方∶白鹅屎,取白者二枚,以少许汤和令相淹,涂上。

《枕中方》治一切虫蛇、射工、沙虱百毒方∶生胡麻,捣,敷上吉,不过三度愈。
治沙虱毒方第五十一

《病源论》云∶山内水间,有沙虱,其虫甚细,不可见人,入水浴及汲水澡浴,此虫着身,阴雨日行草间,亦着人。便钻入皮里。初得之,皮上正赤,如小豆黍,以手摩赤上,痛如刺,过三日之后,令百节强,疼痛寒热,赤上发疮。此虫渐入至骨,则杀人。人在山涧洗浴竟,巾拭如芒毛针刺,熟见,以竹簪挑拂去之。已深者用针挑取虫子,正如疥虫,挑不得,灸上三、四壮,则虫死云云。

《抱朴子》云∶有沙虱水陆皆有,其新雨后及晨暮践涉必着人。唯日烈草燥时,瘥耳。

其大如毛发之端,初着人便入皮里,其所在,如有芒刺之状,小犯大痛,可以针挑取之,正赤如丹,着爪上行动也。若不即挑之,此虫钻至骨,便周行走人身中。其病与射工相似,皆杀人。人行有此虫之地,每还所住,辄当以火自灸疗,令遍身,则此虫堕地去也。若带入物麝香丸,玉壶丸,犀角丸等,兼避沙虱,短狐也。若猝不得此药者,但可带好生麝香亦佳。

以雄黄、大蒜分等,合捣,带一丸如鸡子者亦善。又可以此药涂疮,亦愈。又咀赤苋根饮之,亦愈。

《葛氏方》治沙虱毒方∶以大蒜十斤,着热灰中,温之令热,断蒜及热以注疮上,尽十斤,复以艾丸灸疮上七壮。

又方∶斑蝥二枚,熬一枚,末服之,烧一枚令尽烟,末以着疮中,立愈。

又方∶山行宜竹管盛盐,数视体足,见,以盐涂之。

《极要方》治沙虱毒方∶以麝香、大蒜,和捣,以羊脂和,着以筒中带之,大良。

《苏敬本草注》云∶煮葱茎,浸或捣,敷贴,大效。
治水毒方第五十二

《病源论》云∶三吴以东及南,诸山郡山县,有山谷溪源处,有水毒病,舂秋辄得,一名中水,一名溪中,一名中洒,一名水中病,亦名溪温。今人中溪,以其病与射工诊侯相似,通呼作溪病,其实有异,有疮是射工,而无疮是溪病。

《葛氏方》云∶水毒初得之,恶寒,头微痛,目匡疼,心中烦懊,四肢振,腰痛骨节皆强,两膝疼,或翕翕热,但欲眠,旦醒暮剧,手足指逆冷至肘膝上,二、三日则腹中生虫,食下部,肛中有疮,不痛不痒,令人不觉,视之乃知耳。不即治,过六、七日,下部便脓溃,虫上食五脏,热盛烦毒,注下不禁,八、九日死,良医所不能治。觉得之,急当视下部,若有疮正赤如截肉者,为毒最急;若疮如蠡鱼齿者,为阴毒,犹小缓,要皆杀人,不过二十日也。欲知是中水,非当作数斗汤,以小蒜五升咀投汤中,莫令大热,热则无烈,去滓,适寒温,以自浴。若身体发赤斑纹者是也。其无异者,当以他病治之。治之方∶取梅若桃枣,捣绞饮汁三升许,汁少以水解及绞之。

又方∶常思草,捣绞饮汁一、二升,并以绵染汁导下部,日三。

又方∶捣蓝青,与少水以涂头面身体,令匝匝。(《千金方》同之。)又方∶取蓼一把,熟捣以酒一杯,合和绞饮汁。

若下部生疮已决洞者方∶桃皮叶,熟捣水渍令浓,去滓着盆中,坐自渍,虫出。

《千金方》治水毒方∶捣苍耳汁,服之一升,以绵沾汁导下部,日三。

《集验方》治中水秘方∶取水萍曝干,以酒服方寸匕。

又方∶捣梅叶取汁,服半杯。小儿不能饮,敷乳饮之。

《范汪方》云∶青龙汤,治中水寒热方∶升麻(二两)龙胆(一两)葳蕤(一两)大青(一两)凡四物,咀,以水四升,煮取二升,分作再服。不静复作,加小附子一枚,四破之,分作三服,良。

又云∶治水中下部疮决洞,医所不能治方∶灸穷骨五十壮,良。
治井冢毒第五十三

《病源论》云∶凡古井冢及深坑、井,多有毒瓦斯,不可辄入,五月、六月间最甚,以其郁气盛故也。若事辄必须入者,先下鸡鸭毛试之,若毛旋转不下,即有毒,便不可入也。

《短剧方》云∶凡五月、六月,深井中及深突深冢中皆有毒瓦斯,入令人郁KT,能杀人,如其必宜入中者,当先以鸡鸭鹅毛及杂毛投其中,毛得自下至底者则无毒瓦斯也。毛若倒上不下,回旋四边者则有毒瓦斯,不可入也。亦可纳生鸡、鸭、鹅、豚、犬、羊生物居中,既有毒瓦斯,其生物须臾自死也。事计必宜入中不得已者,当先以酒,若苦酒数斗浇洒井冢坎中边,停小时,然后可入也。若觉中此气,郁闷奄奄,欲死者,还取其中水数斛,洒人面并水含饮之,又以灌其头身从头至足,须臾则活也。其中若无水者,乃取他水也。

《千金方》治入井冢毒方∶取他井水灌身上,至三辰顷活。若东井取西井,西井取东,南取北,北取南。

《葛氏方》治入井及冢中,遇毒瓦斯,气息奄奄,便绝方∶以水其面,并令含水,又使汲其所入井若冢中水数斛以灌之,从头至足须臾活。

又方∶服诸解毒犀角、雄黄、麝香之属,豉豆、竹沥、升麻诸汤。
辟蛊毒方第五十四

《病源论》云∶凡蛊有数种,皆是变惑之气也。人有故造作之,多取虫蛇之类,以器盛贮,任其自相啖食,唯一物独在者,即名谓之为蛊,便能变惑。随逐酒食,为人患祸,祸于他人,则蛊主吉利。所以不羁之徒,而畜事之。

又云∶面色青黄者,是蛇蛊也。腹内热闷,身体恒病,面色赤黄者,是蜴蛊也。腰背微满,舌上生疮,颜色乍白乍青,腹内胀满,状如虾蟆,是虾蟆蛊也。颜色多青,毒成吐出似蜣螂,是蜣螂蛊也。

又云∶有飞蛊,来无由,渐状如鬼气者。

又云∶氐羌毒者,犹是蛊类,于氐羌界城得之,故谓之氐羌毒病,状中中蛊,心腹刺痛。

又云∶有野道者,是无主之蛊也,畜事蛊人,死灭无所,依止田野道路之间,犯害人者,故谓之野道。

又云∶欲知是蛊与非,当令病患唾水内,沉者是蛊,浮者非蛊也。

又云∶含大豆,若是蛊,豆皮脱;若非蛊,不烂脱也。

《葛氏方》云∶欲知蛊主姓名方∶取鼓皮,少少烧末,饮病患,病患须臾自当呼蛊主姓名,可语使呼取去,去即病愈(今按∶《极要方》水饮。)又方∶荷叶,密着病患卧席下,亦即呼蛊主姓名。

又云∶治饮食中蛊毒,令人腹内坚痛,面目青黄,淋露骨立,色变无常方∶雄黄、丹砂、藜芦各一两,捣筛,旦以井花水服一刀圭,当吐蛊毒。(今按∶《集验方》云∶三物各一分,有蛊当吐;不吐,非蛊之。)又云∶若蛊已食下部,肛尽肠穿者方;以猪胆淋内中,以绵染塞之。

又云∶治中蛊吐血,或下血皆如烂肝方∶盐一升,淳苦酒一升,和一服,立出,即愈。

又方∶茜草根、荷根各三两,咀,以水四升,煮得二升,去滓顿服即愈。又自当呼蛊主姓名,茜草即染绛茜草也。

《短剧方》治蛊方∶举树皮广五寸,长一尺,芦薇根五寸,如人足父指大者。凡二物切,以水一升,清酒三升,煮取一升,顿服当下蛊。

又方∶土瓜根,大如母指,长三寸,咀,以酒半升,渍一宿,去滓,一服。

《集验方》治下部若蛊食入,从后孔见肠方∶虾蟆青背长身者,乌鸡骨各烧作屑,分等,合之以吹下部孔中,大良。

又云∶治猝中蛊,下血如鸡肝者,昼夜去石余血,四脏悉坏,唯心未毁,或乃鼻破待死者方∶桔梗,捣,下筛,以酒服方寸匕,日三。

又方∶隐忍根,捣取汁二升,分三服。桔梗苗也。

《千金方》治蛊方∶槲树北阴皮去苍大指长寸,水三升,煮取一升,空腹服之,蛊虫出。

又方∶皮灰,水服方寸匕。

《医门方》治蛊毒方∶取巴豆一枚,去心,豉三粒,釜底黑方寸匕,合捣,分服一丸,吐虫,即服甚良。

又方∶荷煮汁饮,干湿根得用多少。

《极要方》疗蛊毒方∶水煮独行根一、二两,取汁服之,吐蛊毒。

《救急单验方》治蛊毒方∶捣药子三枚,服立验。

《徐伯方》治蛊方∶取茜草,生捣绞取汁,日可服二、三升,复恒取汁,作食及煮粥。

《僧深方》治猝急蛊吐欲死方∶生索濯名根茎,捣绞取汁得一升,顿服之,不过再三作,神良。

《范汪方》治蛊方∶菖蒲二两乌贼鱼骨二分上二物,捣下筛,以酒服方寸匕,日三。
卷第十九
服石节度第一

《服石论》云∶中书侍郎薛曜云∶凡寒食诸法,服之须明节度,明节度则愈疾,失节度则生病,愚者不可强,强必失身;智者详而服之,审而理之,晓然若秋月而入碧潭,豁然若春韶而洋冰积实,谓之矣。

凡服五石散及钟乳诸石丹药等,既若失节度,触动多端,发状虽殊,将摄相似,比来人遇其证,专执而疗之,或取定古法,则与本性有违;或取决庸医,则昧于时候,皆为自忤。

故陶贞白曰∶昔有人服寒食散,捡古法,以冷水淋身满二百罐,登时僵毙。又有取汗不汗,乃于狭室中四角安火,须臾即殒,据兹将息,岂不由人,追之昔事,守株何甚。

秦承祖论云∶夫寒食之药,故实制作之英华,群方之领袖,虽未能腾云飞骨、练筋骨髓,至于辅生养寿,无所与让。然水所以载舟,亦所以覆舟;散所以护命,亦所以绝命。其有浮薄偏任之士,墙面轻信之夫,苟见一候之宜,不复量其夷险,故祸成不测,毙不旋踵。斯药之精微,非中才之所究也。玄晏雅材将冷,廪丘温为先,药性本一而二论硕反,今之治者唯当务寻其体性之本源,其致弊之由善候其盈缩,详诊其大渊采撮二家之意,以病者所便为节消息斟酌,可无大过。若偏执一论,常守不移,斯胶柱而弹琴,非善调之谓也。

许孝崇论云∶凡诸寒食草石药皆有热性,发动则令人热。便须冷冻饮料、食冷、将息。故称寒食散。服药恒欲寒食、寒饮、寒衣、寒卧、寒将息,则药气行而得力。若将息热、食热饮、着热衣、眠卧处热,药气与热气相并壅结于脉中,则药势不行发动,能生诸病、不得力,只言是本病所发,不知是药气使然。病者又不知是药发动,便谓他病。不知救解,遂致困剧。

然但曾经服乳石药,人有病虽非石发,要当须作带解石治也。

孙思邈论云∶服石人皆须大劳役,四体无得自安。如其不尔,多有发动,亦不得道便恣意取暖称适已情,必违欲以取寒冻。虽当时不宁,于后在身,多有所益,终无发动之虑也。

人不服石,庶事不佳,恶疮、癣疥、温疫、疟疾,年年恒患,寝食不安。兴居常恶,非止已事不康生子难育,所以石在身中,万事休泰,要不可服石也。年三十以上,可服石药。若素肥充,勿服也。四十以上,必须服之。五十以去,可服三年一剂。六十以上,两年可以服一剂。七十以上,一年可服一剂。人五十以上,精药销歇,服石犹得其力。六十以上转恶,服石难得力,所以恒须服石。令人手足温暖,骨髓充实,能销生冷,举措经便,复耐寒暑,不着诸病。

凡服石人,其不得杂食口味,虽百口具陈,终不用重食其肉。诸难既重,必有相贼。聚积不消,遂动诸石。但知法持心,将摄得所,石药为益,善不可加。

陈延之论云∶服草木之药则速发,须调饮食金石者,则迟起而难息。其始得效者,皆是草木盛也。金石乃延引日月草木,少时便息。石势犹自未盛,其有病者,不解消息,便谓顿休。续后更服。或得病痼药微,倍复增石,或便杂服众石,非一也。石之为性,其精华之气则合五行,乃益五脏;其浊秽便同灰土,但病家血气虚少,不能宣通,更陈瘀,便成坚积。

若其精华气不发,则冷如冰,而病者服之,望石入腹即热,既见未热,服之弥多。既见石不即效,便谓不得其力,至后发动之日,都不自疑,是石不肯作石消息。便作异治者,多致其害。

《释慧义论》云∶五石散者,上药之流也。良可以延期养命,调和性理,岂直治病而已哉。将得其和,则养命瘳疾;御失其道,则夭性,可不慎哉。此是服者之过,非药石之咎也。

且前出诸方,或有不同。皇甫唯欲将冷,廪丘欲得将石药性热,多以将冷为宜。故士安所撰,偏行于世。

《夏侯氏论》云∶观世人了不解寒食药意而为节度者,又大误。以不解修误法,安得不有顿踬耶。遂不思故而其怨咎于药,此药正不宜以病进时服也,当以病退时服之。此药以助正气为主,病进时则病气强、正气弱,药不能制也。病退则因气强遂扶助之,遂凌病气矣。

空腹及下后不可服。服之滞着曲奥之处积岁不解。服药无冬夏时节也。春秋瘥为佳。

鲁国《孔恂论》云∶寒食药治虚冷特佳。然要在消息精意伺候,乃尽药意。虽本方云极冷恣水,要当以体中为度。若腹中不能热,心中平定,未觉愦闷,不可便恣冷冻饮料食、食冷当随药热多少衣服浓薄亦宜然。盛冬之月可食,餐食后若心中温闷,辄饮少冷水便瘥矣。

如觉药作者,但薄衣行风中亦解。若殊不解,可小洗手足头面,不至浴。当消息体中,慎勿逆用水也。

《庞氏论》云∶夫寒食药发多在秋冬,秋冬则阳气处内,阴气处外。外寒则热并入,助药为热也。此其自然发理。若有违犯药忌,亦复用发,消息候察,唯存心精意者也。服药人多自厌恶烦愦,无神不可信,取其言用加方治也。当用边人之意参之耳。又药多违人性,喜加迁怒,不可慎从。侍者当犯颜据争,深守所见,亦不可使病患甚恚,用增药动也。

夫药发皆有所由。或以久坐、久语、卧温失食,或以御内不节犯损体实。或劳虑存心、情意不欢,或以饮酒连日而不盥洗,或以并饮不消停徐为,或食饼黍小豆诸热,凡此诸或皆是发之重诫也。

又,药卧欲得薄衣,亦宜然,犯寒则无中冷之忧,触热则有患祸之累。

诸饮食皆欲冷,唯酒可温耳。诸用水皆欲得新汲井水,不欲大冷水也。食欲数而不欲顿多,当计一日常数所能食,分昼夜八九下后,饮食犹令小温,于常食数食之后自还如旧。

衣被欲得故絮而使薄,但当益领数。所以尔者,减益故也。若噤战恶寒者,少重其衣被以温体,人迫挟之,噤不如解,使远去之,过时不去,便助药发也。

曹歙论云∶寒温调适之宜,云诸药己折。虽有余热,不复堪冷,将适之宜,欲得覆而不密,常欲得凉而不至于极冷。譬如平人得热,欲得冷凉之,大过即已为病也。故勿得脱衣露卧,汗出当风也。

有药者,若寒若热,心腹欲痛满,欲脱衣,欲着衣,衣薄衣浓皆当随觉为度,不可轻忍也。

凡服寒食散发者,皆宜随所服之人以施方治。人体气之不同者,若土风之殊异也。虽言为当饮酒,人性本有能不;虽言为当将冷,人体本有耐寒与不耐寒;虽言为当多食饮,食饮本有多少;虽言为当劳役,人筋KT本有强弱,又肥充与消瘦,长老与少壮,体中挟他与不挟,耐药与不耐药,本体多热与多冷,凡此不可同法而疗也。药发多,多变成百病,苟不精其曲折,如以粗意投雷,亦由暗历危险其趣巅沛往往是也。

凡寒食药发生百病者,大较坐失之温也。今者暑热尤不可轻失,温也。

可疑之候云∶咳逆咽痛,鼻中窒塞,清涕出,本皆是中冷之常候也。而散热亦有此诸患可用饮温酒。冷咳者,得温是其宜也。若是热咳者,酒通寒食散,得酒于理,当瘥和也。

欲分别之者,饮冷转剧。剧者果是冷咳无疑也。饮冷觉佳者,果是药热咳无疑也。

温治之治云∶今举世之人,见药本方,号曰护命神散。登服日盒饭解脱衣被向风,将冷水自浇灌。夫人体性自有堪冷不堪冷者,不可以一概平也。譬犹万物,匪阳不而柒与玄水反当以寒湿为干茂(义),岂可谓不然乎。余服此药,几四十载矣,所治者亦有百数。服药之日,乃更当增其衣服,扶掖起行,令四体汗出,则营卫,津液津液则诸温热随汗孔而越,则不复苦烦愦矣。体适津液,自不思水,无事为蛇尽足而强用水。若小烦躁,可渍手巾一枚,拭热处,小凉则当促起还着衣矣。自于药势已发,可彻向者。始服药重药之衣,其平常所服,慎不可减也。

凡人体气各有羸虚,虚者恒着巾帽、身袭温裘,风恶忌冷,不得KTKT如何?一旦卒释常服?增以冷水浇灌,限漏刻之间。则中冷矣。中冷则成伤寒,壮热如烧,小大惶怖,不知是伤寒也,皆谓药发耳。遂竟沐浴,空井竭泉,气力盛者有异幸,而其弱劣,于是讫矣。

服药之后,假使头痛壮温,面赤体热,其脉进数,盒饭以伤寒法救之,亦可以桂枝发汗,亦可针灸,无所拘疑也。

《潘师房救解法》云∶凡石一度发即一倍得力,如不发者,此名无益。若一发后更无诸病,有病必是石发也。

《皇甫谧节度论》云∶吾观诸服寒食散者,咸言石药沉滞凝着五脏,故积岁不除;草药轻,浅浮在皮肤,故解散不久其违错草石正等今之失度者,石尚迟缓,草多急疾,而今人利草惮石者,良有以也。石必三旬,草以日决,如其不便,草可悔止,石不得休故也。然人有服草散两匕十年不除者,有服石八两终身不发者,虽人性有能否,论药急缓,无以异也。

又,《发动救解法》云∶人将药,但知纯寒用水药,得大益,不知纯寒益动,所以困不解者,由是失和故也。寒大过致药动者,以温解之。热大过致药动者,以冷解之。常识所由也,无不得解。

又云∶服寒食散者,唯以数下为急。有终不下之。必不得生。下后当慎如节度。

又云∶服散不可失食即动,常令胃中有谷,谷强则体气胜,体气胜则药不损人,不可兼食药,益作常欲得美食,食肥猪、苏脂肥脆者为善。

又云∶河东裴秀彦服药失度,而处三公之尊,已错之后,已不复自知,左右又不解救之。

救之法,但饮冷酒,冷水洗之,用水数百石,寒益甚,逐绝命于水中,良可悼也。夫以十石焦炭二百斛,水泼之则炭灭矣。药热气虽甚,未如十石之火也。泼之不已,寒足杀人,何怨于药乎。世之失救者,率多如此,欲服此药者,不唯已自知也。家人大小皆宜习之,使熟解其法,乃可用相救耳。

又云∶凡有寒食药者,虽素聪明,发皆顽器告喻难晓也。以此死者,不可胜计。急饮三黄汤下之,得大下即瘥。
服石反常性法第二

皇甫谧云∶凡治寒食药者,虽治得瘥,终不可以治者为恩也。非得治人后忘得效也。昔文挚治齐王病,先使王怒而后治病已。王不思其愈而思其怒,文挚以是虽愈王病而终为王所杀。今救寒食药者,要当逆常理,反正性,犯怒以治之,自非达者。已瘥之后,心念犯怒之怨,必忘得治之思。犹齐王之杀文挚也。后与太子尚不能救,而况凡人哉。然死生大事也,知可生而不救之,非仁者。唯仁者心不已,必冒怒而治之为亲戚之,故不但其一人而已。凡此诸救,皆吾所亲更也。已试之验,不借问于他人也,大要违人理,反常性。

六反∶重衣更寒,一反。(《外台方》云∶凡人寒,重衣即暖。服石人宜薄衣,若重衣更寒。《经》云∶热极生寒。故云一反。)饥则生臭,二反。(平人饱则食不消化,生食气。服石人忍饥失食节,即有生臭气,与常人不同,故云二反。)极则自劳,三反。(平人有所疲极即须消息恬养。服石人久坐卧疲极,唯须自劳,适散石气即得宣散,故云三反。)温则泄利,四反。(平人因冷乃利,得暖便愈。服石人温则泄利,冷则瘥,故云四反。)饮食欲寒,五反。(平人食温暖则五内调和。服石人饮食欲寒乃得安稳,故云五反。)痈疮水洗,六反。

七急∶当洗勿失时,一急。

当食勿忍饥,二急。

酒清淳令温,三急。

衣温便脱,四急。

食必极冷,五急。

卧必底薄,六急。

食不厌多,七急。

八不可∶冬寒欲火,一不可。

饮食欲得热,二不可。

常疾目疑,三不可。(凡服石人常须消息节度,觉少不安,将息依法治,不可生狐疑。)畏避风温,四不可。(若觉头风热闷,愦愦心烦,则宜常当风梳头,以水洗手面即好,不比寻常风湿。)极不能行,五不可。(若久坐卧,有所疲极,必须行役自劳。)饮食畏多,六不可。

居贪浓席,七不可。

所欲从意,八不可。

三无疑∶务违常理,一无疑。

委心弃本,二无疑。

寝处必寒,三无疑。

若能顺六反,从七急,审八不可,定三无疑,虽不能终蠲此疾没齿无患者,庶可以释朝夕之暴卒矣。
服石得力候第三

《病源论》云∶夫散脉,或洪实,或断绝不足欲似死脉,或细数,或弦快,坐所犯,非一故也。脉无常度,拙医不能识,然热多则弦快,有则洪实,急痛则断绝,沉数者难发,浮大者易发,难发令人不觉药势(热)行已,药但于内发,才不出形于外,欲候知其得力,人进食多是一候,气下颜色和悦是二候,头面身痒是三候,策策恶风是四候,厌厌欲寝是五候也。
服石发动救解法第四

皇甫谧薛侍郎寒食药发动证候四十二变并消息救解法。(今检有五十一变。)皇甫谧云∶寒食药得节度者,一月辄解,或二十日解,堪温不堪寒,即已解之候也。

其失节度者,或头痛欲裂,坐服药,食温作急宜下之。

或两目欲脱,坐犯热在肝,速下之,将冷自止。

或腰痛欲折,坐衣浓体温,以冷水洗,冷石熨之。

或眩冒欲蹶,坐衣温,犯热宜KT头,冷洗之。

薛公云∶常须单床,行役,并以冷水洗浴即愈。

或目痛如刺,坐热气冒次,上奔两眼故也。勤于冷食,清旦以温小便洗之。

或目冥无所见,坐饮食居处温故也。脱衣自劳,洗,促冷冻饮料食,须臾自明了。

或四肢面目皆浮肿,坐食饮温,又不自劳,药与正气隔并故也,饮热酒,冷食,自劳,冷洗之则瘥。

或耳鸣如风声,汁出坐自劳,出力过瘥,房室不节,气并奔耳故也。勤好饮食,稍稍行步,数食节情即止。

或鼻中作KT鸡子臭,坐着衣温故也。脱衣冷洗即止。(或本云∶冷洗薄衣即瘥)或口复伤,舌强烂燥,不得食,坐食少,谷气不足,药积胃管中故也。急作栀子豉汤,服三剂瘥。〔(或本云二剂瘥矣)今按∶栀子汤在第二十卷"口干方"。〕或龈肿,唇烂齿牙摇痛,颊车噤,坐犯热不时救故也。当风张口使冷气入咽,嗽寒水即瘥。

或咽中痛,鼻塞,清涕出,坐温衣近火故也。促脱衣,冷水洗、当风、以冷石熨咽颊五六过自瘥。或本云∶脱衣取冷,当风立,以冷物熨咽须臾愈,不须洗。

或咳逆,咽中伤,清血出,坐卧温故也。或食温故也。饮冷水,冷石熨咽外。

或偏臂脚急痛,坐久藉卧席,温不自转,热气入肌附骨故也。勤以布冷水淹迫之,温复易之。

或两腋下烂辛痛,坐臂胁相亲故也。以物悬手离胁,冷石熨之。

或胸胁满,气逆,干呕,坐饥而不食,药气熏膈故也。促冷食、冷冻饮料、冷洗即瘥。

或手足偏痛,诸节欲解,身体发痈疮坚结坐寝处久不自移徙,暴热并聚在一处,或坚结核痛,甚者发痈始觉,便以冷水洗,冷石熨之;微者食顷消散,剧者日用水不绝乃瘥。洗之无限,要瘥为期。薛公云∶若体上生疮,结气肿痛,不得动者,为自劳大过也。

或腹胀欲决,甚者断衣带,坐寝处久,下热,又衣温失食、失洗、不起行。促起行,饮热酒,冷食、冷洗,当风栉梳而立。

或腰痛欲折,坐衣浓体温,以冷水洗冷石熨之。薛公曰∶若腰痛欲折、两目欲脱者,为热上肝膈腰肾,冷极故也。

或脚疼如折,坐久坐下温,宜常坐寒床,以冷水洗起行。或本云常须单床上坐善也。

或脚趾间生疮,坐着履温故也。脱履着屐以冷水洗足则瘥。薛公云∶当以脚践冷地,以冷水洗足则瘥。

或肌皮坚如木石枯,不可得屈,坐食热卧温,作癖久不下,五脏隔闭,血脉不周通故也。

促下之,冷食,饮热酒,自劳行即瘥。

或身皮或本云身内楚痛,转移不在一处,如风状,或本云如似游风,坐犯热所为,非真风也。冷洗冷熨即了矣。

或百节酸痛,坐卧下太浓,又入温被中,又衣温不脱故也。卧下当极薄,大要也,被当单布不着绵衣,亦当薄且垢,故勿着新衣,宜着故絮也。虽冬寒,当常KT头受风,以冷石熨衣带,初不得系也。若犯此酸闷者,促入冷水浴,勿忍病而畏浴也。

或关节强直不可屈伸,坐久停息不自烦劳,药气胜,正气结而不散越,沉滞于血脉中故也。任力自温便冷洗即瘥。任力自温者,令行动出力足劳则发温也。非浓衣近火之温也。

或脉洪实,或断绝不足似欲死脉,或细数强快,坐所犯非一故也。脉无常,投医不能识别也。热多则弦快,有癖则洪实,急痛则断绝。凡寒食药热率常如此,唯勤从节度耳。

或人已困而脉不绝,坐药气盛行于百脉之中,人实气已尽,唯有药两犹独行,故不绝非生气也。已死之后体故温如人肌,腹中雷鸣,颜色不变,一再宿乃似死人耳。或灸之,寻死或不死,坐药气有轻重,重故有死者,轻故有生者。虽灸得生,非已疾之法,遂当作祸,必宜慎之,大有此比故也。

或心痛如锥刺,坐当食而不食,当洗而不洗,寒热相绞,气结不通,结在心中,口噤不得息,当校口促与热酒,任本性多少。其令酒两得行气自通,得噫,因以冷水洗淹,有布巾着所苦处,温复易之,自解。解便速冷,食能多益善,若大恶着衣,小使温温便去衣即瘥。

于诸痛之中,心痛最为急者,救之若赴汤火,乃可济耳。

或有气断绝不知人时,撅口不可开,病者不自知,当须旁人救之,要以热酒为性命之本,不得下者,当KT去齿,以热酒灌含之,咽中塞逆,酒入复还出者,但与勿止也。出复纳之,如此或半日,酒下气通乃苏,酒不下者便杀人也。

或服药心中闷乱,坐服药,温药与疾争结故也。法当大吐下,不吐下当死。若不吐下不绝者,冷食饮自解。薛公曰∶若绝,不识人,目复不开者,亦当KT齿以热酒灌之。入咽吐出者,更当与之。得酒气下通,不过半日苏矣。

或淋不得小便,坐久坐下温及骑马鞍中热入膀胱故也。大冷食,以冷水洗少腹,以冷石熨一日即止。

或小便稠数,坐热食及啖诸含热物饼黍之属故也。(或本云∶饼黍羊酪之属)以冷水洗小腹自止,不瘥者,冷水浸阴又佳。若复不解,服栀子汤即解。

或阴囊臭烂,坐席浓,下热故也。坐冷水中即瘥。

或大行难,腹中坚固如蛇盘,坐犯温,久积腹中干粪不去故也。消苏若膏使寒服一二升,浸润则下;不下更服下药即瘥。薛公曰∶不下服大黄朴硝等下之即瘥。

或大便稠数,坐久失节度,将死之候也。如此难治矣。为可与汤下之,倘十得一生耳。

不与汤必死,莫畏不与也。下已致死,令人不恨。

或下痢如中寒,坐行止食饮犯热,所致人多疑,是本疾又有滞癖者,皆犯热所为,慎勿疑也。速脱衣、冷食、冷冻饮料、冷洗之。

或遗粪不自觉坐,坐久下温热气上入胃小腹(肠欤)不禁故也。冷洗即止。

或失气不可禁止,坐犯温不时洗故也。冷洗自寒即止。

或周体悉肿,不能自转从,坐久停息不饮酒,药气沉在皮肤之内而血脉不通故也。饮酒冷洗,自劳行步即瘥。极不能行者,使健人扶曳行之,壮事违意,慎勿听从之,使肢节柔调乃止,勿令过瘥。过则便极,更为失度,热者复洗,或本云饮热酒冷水洗。

或嗜眠不能自觉,坐久坐热闷故也。急起冷洗浴也。食饮自精了或有也。当候所宜下之。

或夜不得眠,坐食少、热气在内故也。当服栀子汤,数进冷食。薛公曰∶当服大黄黄芩栀子三黄汤,数进冷食,自得睡也。(今按∶此汤在第二十卷除热解发篇。)或梦惊悸不自制,坐热在内争,五行干错与药相犯,食足自止。

或得伤寒,或得温疟坐犯热所为也。凡尝常服寒食,虽以久解而更病痛者,要先以寒食救之,终不中冷也。若得伤寒温疟者,亦可以常药治之,无咎也。但不当饮热药耳。伤寒药皆除热疟药皆除癖,不与寒食相妨,故可服也。

或矜战患寒如伤寒,或发热如温疟坐失食忍饥,失洗,久坐不行,或食臭秽故也。急冷洗起行。

或寒栗头掉不自支任,坐食少,药力行于肌肤、五脏失守、百脉摇动与正气争竟故也。

努(怒)力强饮热酒以和其脉,强冷食冷冻饮料以定其脏,强起行以调其关节,酒行食充,关机已调,则洗了矣,云∶了者是慧然病除神明了然之状也。薛公曰∶强洗以宣其壅滞。

或脱衣便寒,着衣便热,坐脱着之间无适故也。当小寒可着、小热便脱即止,洗之则慧矣。慎勿忍使病发也。薛公曰∶应洗勿忍则病成也。

或寒热累月,张口大呼,眼视高精,候不与人相,当日用水百余石浇洗,不解者,坐不能自劳,又饮冷酒,复食温故也。譬如人乃心下更寒,以冷救之愈剧者,气结成冰,得热熨、饮热汤冰消气散人乃心解,令药热聚心乃更寒战,亦如之类也,速与热酒,寒解气通,酒两行于四肢,周体悉温,然后以冷水二斗洗之KT然了也。

或药发辄屏卧不以语人,坐热气盛、食少、谷气不充邪干正性故也。饮热酒、冷洗、食自劳便佳。

或食下便吐,不得安住,坐有促下之下。薛公曰∶急以甘草饮下之。

或患冷食不可下,坐久冷食,口中不知味故也。可作白酒糜,益着苏热食一两,过闷者还冷冻饮料冷食也。

或恶食如臭物,坐温衣作也。当急下之。若不下,万救终不瘥也。薛公曰∶以三黄汤下之。(今按∶三黄汤方在第二十卷热除解发条。)或饮酒不解,食不得下,乍寒乍热,不洗便热,洗复寒,甚者数十日,轻者数日,昼夜不得寝,愁悲恚怒,自惊跳悸,恐慌惚忘误者,坐犯温积久,寝处失节,食热作内实侠热与药并行,寒热交争,虽以法救之,终不可解也。吾尝如此,勤对食,垂涕援刀,欲自刺,未及得施,赖升亲见迫夺,故事不行,退而自惟,乃却刀强食,饮冷水,遂止,祸不得成,若丝发矣。

《庞氏论》云∶凡药欲发之候,先欲频伸,或苦头痛目疼、身体、螈,或惊恐悸动,周身而强,或耳中气满如车之声,或体热剧于火烧,或如针刺,噤澡恶寒,昧昧愦愦,不知病处,或腹中燠热如烧锻怀之也。此皆欲发之候也。其发甚者,腹满坚于材石,绕口青黑,大小便血而多无脉也。唯气息才通,心下温耳。如此之病,归于大浇,以瘥为期也。

又药盛发使人悲愁恚怒、角弓反倒,其状若风,有面色青黑,身体斑磷,尔时当极大浇用水无数,如此辈率多用水二三千石,尔乃解耳。得解之后亦当速下。凡浴之初,皆多恶冷,但得水数解,渐遂便之,心意KT然则止,若腹中懊闷、陶热吸吸者,若渴人精神默默但欲眠卧者,此药发在内、攻守五脏也。急服七物栀子汤,外以新汲冷水浴之。

(今按∶在第二十卷除热解发篇。出《短剧方》,号黄芩汤。)若噤澡振KT极目劳动,病患不能自劳者,车载或牵挽掣顿之。

若噤战者,复如上牵挽之。若大行通利,无他结塞,又周体无有热温之证而猝气悸,须臾口不能言者,速温好酒三升,稍饮之。热闷者,饮水则解。

若大行小难,腹微满气,兼复苦渴,患此之后,寻复舌大不得语者,速饮栀子汤。

若药发不时解而久苦渴,是多饮所为也。若下而故瘥耳。

葛稚川云∶凡服五石护命更生及钟乳寒食诸散,失将和节度,皆致发动,其病无所不为。

若发起仓猝,不渐而至者,此皆是散热也。宜时救解。

若四肢身外有诸一切疾痛违常者,皆以冷水洗数百过。热有所衡,水渍布巾随以拓之。

又水渍冷石以熨之,行饮暖酒,逍遥起行。

若心腹内有诸一切疾痛违常烦闷恍者,急解衣取冷热温酒饮一二升,渐稍进觉小宽便冷餐,其心痛者最急。

若肉冷口已噤,但折齿下热酒便开。

若腹内有结坚热便生众疾者,急下之。热甚口发疮者下之。癖实犹不消,恶食畏冷者更下之。

《夏侯氏论》云∶其察体中有不常,皆是药气。服药一时须察所患,或小瘥。或心中温温欲吐,或寒或热,或痛或痒,或缓或急,或眩或痹,或理或乱,其有所觉,皆是药也。心中温温,小饮冷水不解,渐益才解便止。又诸所觉未必周体,或发头面手足胸背,随所觉处以湿手巾熨之,不解小洗之,洗之则解即止。

《曹歙救解法》云∶有药而苦头痛目瞑恶食、食下便吐、不得安者,为是实也。

当促下之。若头痛目疾而不恶食者,自是寒食散疾,未必纯是实也。宜常兼以将冷为治。若有实也,不下终不瘥也。寒食药热,率杀药热,服下药要当以能否下为度,不得病可重下也。

期以得病为断,服药未下,慎勿饮酒也,令人闷吐。下后食少里空,热便乘虚在处,则吐逆下利腹满,如此者宜以冷食渐渐解,服栀子汤是其治也。(今按∶此汤在第二十卷号增损。皇甫栀子豉汤出《短剧方》。)药发,头面苦眩冒者,则解头结散发扇之,若虽觉瘥犹不KTKT者沐头。其热甚头痛面赤者,以寒水淋头,暑热时以冰水淋头,不瘥以油囊盛冰着头结中,觉瘥下水。

药发,耳目口齿苦,耳鸣汁出,数数冷食,稍稍步行,鼻口臭,冷冻饮料冷洗。口中生疮,舌强,服栀子汤。

药发心腹苦心腹痛者,当与热酒。口噤者,撅口促与用冷水淹手巾,着苦处,温复易。

诸痛之中,心痛最急,救之若赴汤火。或有气绝病者,不自知,当须边(旁)人之救,以酒灌含之。咽中寒逆,酒入辄还,勿止也。出复纳之。

腹满者,服凝水石汤,胸心腹中热盛,咽干口燥,饮冷。霍乱吐逆,当用饮冷。胸中窒塞,胸胁两强,当饮酒。腹中拘急切痛,当用饮食,不宜但以冷迫之也。腰痛以寒水洗,冷石熨之;大行难消,苏令如膏,服三升则下,未下重服之;小行稠数者,以水洗小腹,服栀子汤则瘥。

药发四肢苦手足烦热,心闹闷者,以冷石熨。甚者以水熨之,关节不屈伸,百节酸疼者,勤自劳役,温则澡洗。

药发噤寒,有药者,虽当澡浴,澡浴若早,药热噤不得出,令噤寒,急用饮酒,勤自劳役,即当KT(渐)温矣。若晚,药热蒸愦,亦令人噤寒,先用饮酒,酒气颇行,便用浇灌,亦当渐温矣。常当数食,一日可至十食。失食令人苦寒。

药发杂患,其有偏痛、偏烦、偏冷、偏热、偏急、偏缓,皆偏洗之。当于水下觉除也。

若有肿核者,宜以冷石熨,不瘥宜以冰熨之。

《释慧义·薛侍郎浴熨救解法》云∶凡药石发宜浴,浴便得解。浴法∶若初寒,先用冷水,后用生熟汤。若初热,先用暖汤,后用冷水。浴时慎不可先洗头,欲沐可用二三升灌矣。

若大小便秘塞不通,或淋沥尿血,阴中疼,此是热气所致,熨之即愈。熨法∶前以冷物熨少腹,冷熨已,又以热物熨前。热熨之以后复冷熨。又小便数,此亦是取冷过,为将暖自愈。

《道弘解散法》云∶食秽饭、臭肉、陈羹、宿菜发,服栀子汤。

饭未熟生酒发,服大麦,一服五合,至三服不解,服孽米一升。

食肉多发,如上法服不解,又服孽末,孽末不解,又服栀子豉汤。

食生菜发,服甘草汤,食粗米发,服甘草汤。(粗米谓咀嚼不精也。)大饱食发,如上服甘草汤。失食饥发,服葱白豉汤。

醉发,服葱白豉汤;若不解,服理中汤。

怒大过发,服人参汤。

将冷大过发,则多壮热,先以冷水七八升洗浴,然后用生熟汤五六石灌之。灌已,食少、暖食、饮少热酒、行步自劳则解。若不解,复服栀子汤。

将热大过发,则多心闷,服黄芩汤。(今按∶以上汤方等在后第二十卷)《薛侍郎补饵法》云∶服石之后一二百日内,须吃精细饮食美酒等,使血脉通利。

若觉虚任饵署预食,强筋骨及止渴。

若觉大热者,可服紫雪,或金石凌,或绛雪,或白雪等。(此等救急,紫雪为上。如不得通泄,宜服黄芩饮,快利即瘥。)若觉体气,不痛不痒,小便赤涩,即绞茅根汁任服之。

若口干,即绞甘蔗汁任服。若不下食,服三物生姜煎。若不下食,体弱,乏气力,即须食鲜鲫上脍。

若发疮及肿,但服五香连翘汤等,忌鱼猪蒜生菜等。(今按∶五香连翘汤在《短剧》第十治恶核方。)若肿有根,坚如铁石,带赤色者,服汤,仍以小小艾炷当肿上灸之,一两炷为佳。

《KT阳功曹范曲论》云∶本方云∶愦愦烦或痹便浴之,人羸或不堪大浇浴者。当随药动处极洗之,非药物处则不堪水。若周身浴不寒特便冰者,当特浇之。若腹背不便水处,可湿手巾着上暖则易。可着半袖去裆。不喜令腹暖KT苇KT薄被则可矣。虽当冷食,欲得新炊饭冷泼之。若不能辄炊,先以热汤浇饭,令KT乃冷泼之。有坚积先服消石大丸下之,乃服散。人多羸瘦,下之可畏。(今按∶硝石大丸有第十瘕条。)
服石四时发状第五

《皇甫谧救解法》云∶春发逆冷,夏发短气,秋发瘙痒,冬发寒战。

此四时发动,变易无常。诸所为病,乃至万端。或动身体,四肢微强,难于屈伸,或胸胁胀满。但欲干呕,或翕翕少气,不欲语言,或睡眠但常欲卧,或KTKT苦寒,思欲浓衣,诸如此候,药将大发。宜急解,事在汗出、动作、饮酒、美食以为法。

或头痛目不欲视,或腹中雷鸣大小便数,或体隐疹状如风搔,或淫策淫策如针刺,或有热剧乍来乍去,或咽喉噎塞有如伤寒,或鼻中萧条若有风吹。诸如此者,皆是将发之候。宜速起行解衣向风便自解。

或苦寒噤战如伤寒者,当饮热酒随人能,不先以暖汤小洗头面手足,行步自动作使体中热,以手巾渍冷水摩拭之,良。

或腹中雷如鸣,饮冷水一升,若饥,可餐食薄衣脱巾胃毡褥。

或但苦热闷而腹满心痛者,宜饮热酒,冷水洗,还薄衣小暖,热气自止。

或患腹背热,如手如杯如盘许者,以冷石随热处熨即瘥。

或头痛项强两目疼而闷乱者,便以水洗浴即瘥。
服石禁忌法第六

《薛曜论》云∶夫金石之性坚刚,而急烈,又性清净而滓恶秽。

皇甫谧云∶凡诸石士十忌∶第一忌怒第二忌愁忧第三忌哭泣第四忌忍大小便第五忌忍饥第六忌忍渴第七忌忍热第八忌忍寒第九忌忍过用力第十忌安坐不动若犯前件忌,药势不行,偏有聚结,常自安稳,调和四体,亦不得苦读念虑。但能如是,终不发动,一切即愈。

曹歙论云∶凡药疾()禁忌者,第一不宜悲思哭泣,其次不甚宜出筋力已自劳役,不宜触盛日猛(大),不宜甚嗔恚忧恐,不宜热衣热食,不宜服热药针灸,不宜食饼,黍羹羊酪皆含热,故悉不宜食之。

《庞氏论》云∶诸服草木石散者,皆不可灸针身体令人善发炎疽疮也。
服石禁食第七

《耆婆方》云∶服石后不可食诸物十种∶油脂药芜荑芥子及芥菜荠桃竹笋荠蔓荆葵菜薯蓣又云∶凡诸服石之士不得多进面及诸饼饵,生菜、五辛、五果、黍、肥、羊,不得多食也。

又云∶压下石诸物十三种∶乔麦粟米淡竹笋水芹干苔木耳柑子冬瓜龙葵菰菜鹿角菜猪
诸丹论第八

《服石论》云∶凡诸丹皆是众石之精,论其切力可济生拔死,人亦有知之,亦有不知之者,然知者至少,不知之者极多。悠悠夭狂之徒则巧历不能计其头数。故至人以之宝爱,庸夫以之轻贱,轻贱则寿促,宝爱则命延。人皆重其延命而不解延其命,贵驻其年而不知驻其年,是可叹者也。余及少年以来常好事,每以诸小丹救疾,十分而愈其七八,其九十暴之属。

亦有气已尽而药入口须臾即活者,亦有气未绝而药入口少时直瘥地,亦有经半日始瘳者,赤有终朝如愈者,大都神效之功,语不难尽。自斯以后,但有得此方及有过此药者,咸勿起谤心,但生信意,则必无横死之虑也。
诸丹服法第九

《服石论》云∶凡服丹之体必须令其病者正意深信,不得于中持疑更怀他念。但想其药入口即愈,慎勿起不信心,其用丹之人亦须一心愿病立瘥。

凡有病服丹者,必须去其疑惑,起其信心,想其丹入口消病状如沸汤之泼冰雪,若此信者无不立愈。

凡服丹,先首于吉日清旦具服严净嗽其口,面向东立再拜,一心发愿,愿服神药以后,千殃散灭,百病消除,志求长生,无违其愿,愿一切大圣加护,去老还年。发此愿已,又以净水嗽口,先含一枣核许蜜,次旦以一二丸服之。若无所觉触者,至他日又渐增之,以微觉触为度。

凡服丹,亦有先熟嚼半果许枣后,以丹和咽之者,有和蜜吞之者。亦有以白饮及酒送之者,亦有直尔引口中津并以水下者,此等并得无在。

凡有病与丹相应者但着起首一二服,纵不得全除,即觉病热渐损,如此者宜服之勿止。

若已经三二服后不觉有异者,即知药病永背,不宜更将服之。

凡服丹皆须晚食,必须少不得过多,多则令药势不行,所以须少,少则易通,通则疾得药力。

凡服丹者,皆须调和神性,不得乍喜乍,则令气脉壅塞。

《召魂丹方》云∶凡人有老有少,有强有弱,有虚有实,有肥有瘦,质既有异,性亦不同,同服一药,其间则有多者,有少者,亦有服一二丸须臾即发者,亦有服三四丸久之始发者,亦有服五六丸少时便发者,亦有服十丸以来遂竟不发者,为此皆须从少至多,不得从多至少,但以斯法调节度,无失其理者。
服丹宜食法第十

《大清经·太一神精丹方》云∶凡服丹人得食粳米、粱米、粟米粥、葱豉粥等,及苜蓿、蔓荆、葱曰、韭菜、生姜、瓜菹、酱豉、羊鹿、獐、雉、兔、少犊等,煮及脯,并得食其羊肉,虽得作脯食,不宜作羹食也。
服丹禁食法第十一

《召魂丹方》云∶凡病多服丹,经三五以上者,不可食陈臭烂败之物,生肉蒜齑之类。

又云∶勿犯肥鲜、生血、五辛、生菜等,食其余,一无所禁。

《大清经》云∶凡当服丹时慎黄牛肉、羊血羹、白酒、仓米、麦,鲤鱼及尘臭烂败之物,并不得犯之,自余任情。

又云∶服丹之时,不宜吃热食热羹,食必须冷,不宜过热,热即发动,其药令人吐逆。

诸服丹雄黄八石,皆宜断血食,不然者既不治为久久使人半身不随,慎之。
服丹禁忌法第十二

《召魂丹方》云∶凡有一切丧、孝、亡、死之家,产妇淹秽之处,从始至末,并须慎忌。
服丹发热救解法第十三

《大清经》云∶凡服药,药发动之时,即觉通身微肿或眼中泪下,或鼻内水流,或多呻吹,或喷,此等并是药觉触之候,宜勿怪也。可停服三五日,将息时以生熟汤浴之为佳,啖冷麦粥一两顿亦好,得平复以后,依前更服。每一日服药宜三,二日或三五日停服,并应自斟酌其力。

凡服丹,不意过度热闷,垂犯者宜急散发低头,以冷水三二升细细淋顶上,须臾便定。

若更不定者,依前更淋之。远不过用三五升即定。唯不得饮冷水,若大困者,亦可饮土浆,又可饮蓝汁鸡子汁,亦可合食三二口酢饭、葵菹若金石凌,凝雪膏及朴硝粉等,宜蜜水各一鸡子许,先和之,令相得,因以朴硝粉大称半两,又合搅相得,服之立解。又可食冷葵菹、猪肉、酢饭、黄连汁、葛汁、大小豆汁、米泔、米粉水。
服金液丹方第十四

《服石论》云∶金液华神丹无慎忌,疗万病。金液华神丹本是太上真人九元子之秘方,此药所合,非俗人所知。但以五阴相催,四时轮转,有生之类KT忽如流。先贤愍而零涕往哲睹而兴威,感遂乃流传俗代,以救苍生之病,使百姓有病之徒咸能除愈。至如腐肠之疾,遇药便除,膏盲之无不瘳愈。纤毫必遂,肌理无遗,此药力有越电之功,五石与大阴真别类,秋霜,一届松竹与兰艾何同害于人者,不日而除。损肌肤者,应时而遣。若服此药,有异于常,不问陈仓生冷,至于血食鱼蒜酢滑猪鹿,同时共餐唯多益善,并无禁忌。药之对病,如后所陈。

夫人受五常,非是一体,或患久冷滞疳,头面枯燥,身体焦干,唯皮与骨。食不消化,米粒浑出,复患心膈痰饮,食乃无味。假使食讫,复患恶气,上填胸喉,多呕吐冷沫。夜卧咽喉干燥,舌上皮颓,梦见雷电之声。或梦逾山越海,睡中厌,手足酸疼,背膊烦闷,蜚尸杂疰,中恶猝死,腰疼膝冷,天阴即发。或患五劳七伤,中寒痹湿,复有男子、妇女、僧尼、寡妇、少女之徒,梦与鬼神交接,真似生人初得,羞而不言,后乃隐而不说,往还日久,鬼气缠身,腹内病成,由惜鬼情,至死不遁鬼魅邪气所缠。眠多坐少,梦想飞扬,魂魄离散,昏昏常困,似瘥还KT诸有读诵之人常吸,冷气冲心,腹肠雷鸣,镇如雷吼。复有百二十种风,十种水,谷赤白等利,多年不瘥之徒,此丹并皆治疗,此药所合,非是道(通)人不知其妙,自量其性,测其劳逸,临时斟酌,方委其功。诸方君子,无乃轻泄,弥秘之。

今按∶服法对治并可依诸丹之法,但件药主治条云∶或有服一二丸,或有服四五丸,病瘥即止,此非养生之丹。不可多服,云云。
服全阳丹方第十五

全阳丹主治∶头风咳逆,喘息呕吐,腰脚疼痛,气力怯弱,一切风病,胸中癖,宿食不消,见饭易饱,离垸还饥,瘦弱虚损,耳聋惊悸;阳道久衰,阴痿不起,益精驻颜,满髓轻身,能食有力,令人肥健。

服法∶先吃泻药(用温白丸)下去腹中宿秽,明日早朝空腹以酒若浆下一丸。从此每日服至三十丸以下,二十丸以上即止。若无效验,重加服二十丸以下,但病重者服五十丸以上,六七十丸以下。若犹无力,服至百丸以下,一切可随疾患轻重,岂可守株哉。患者若欲得早除,每日服二三丸亦得。然欲服时取二合以下饭净淘吞药后即吃三四口许压之,恐药气冲上头面,但病在胸膈以上者先吃淘饭,后服丹药;病在腰以下者,先吃药,后食饭。若在头面及遍身者,不用淘饭,只用酒浆。若患烦闷,时时服金汞丹、甘豆汤、芦根汤。药如有发动,可依治石发方。

禁忌∶猪肉、油腻、陈臭、粘滑、海藻、五辛、血食、青菜等。
服石钟乳方第十六

性味功能∶《本草经》云∶石钟乳,味甘、温,无毒,主咳逆上气,明目益精,安五脏,通百节,利九窍,下乳汁,益精气,补虚损,疗脚弱疼冷,下焦伤竭,强阴,久服延年益寿,好色不老,令人有子。不练食之,令人淋。(陶注云∶唯通中,轻薄如鹅管,碎之如爪甲,中无鹰齿,光明者为善。敬注云∶虽浓而光明可爱,饵之。)《耆婆方》云∶夫钟乳者,取管成白光,润泽如虻翅蝉翼者,好得服,服即得力,浓者不可服之耳。但水而南流者上,东流者次,余方不中服之。凡钟乳白光者为上,黄光者为次,赤者不中服,性大热。诸长生补益之中不过乳也。须常服之。服乳人若多嗔,只得九年即死,好好慎之。唯不能禁嗔,勿服之。

《极要方》云∶钟乳,所以仙人名之曰乳,此精膏所作,与一切凡石悬殊绝伦不比类也。

师云∶服一斤乳尽,百病除;二斤乳尽,润及三代;三斤乳尽者,临死颜色不变。纵在土下,满百年后还穿冢出,即成僵人也。此人在俗及至千年,皆不得回顾者即是也。一千以外者,行日中,亦无歇,遂成真仙官也。

夫钟乳,此石之精膏也。不与土石杂。独生石室,宜神丹为地,所以然者,凡作丹法,皆飞诸石精以为霜雪,而遂成金银,服之立仙矣。是以服丹之士,先服石之精髓,与丹为地。

若乳与丹相兼而服之者,能理丹石,补虚益精,久服之即能变练骨髓,老而更少,令有子。

养性要药也。

服乳法∶凡服乳,撰旺相日及建除开日,吉。又常以戊己日用之。

凡欲服药,须先服泻药,除去腹中恶秽。

凡服石者,若有宜下之。若人疲弱者,未必下也。

凡服药,先首于吉日清旦具服严净嗽其口,面东向立再拜,一心发愿,愿服神药以后,千殃散灭。百病消除,志求长生,无违其愿,三切大圣,加护去者还年。发此愿已,又以净水漱口,先含一枣核许蜜,次以一二丸服之。

凡服药者,旺日服泻药,而相日服药从一丸起,稍积日服之。可至二三丸重,以二两为一剂。

凡服乳之法,始以温酒服之二三丸,次日五丸,次日七丸,次日九丸,次日十一丸,次日十三丸,然后以十五丸为法,不过于焉。

凡服钟乳丸五六十丸未得力者,可服一二百丸,稍停,候气色。

《极要方》空腹服钟乳法∶上取成练乳,称一两,分为再服。旦服暮令尽,无问乳之多少,此一两为度。

凡服乳,皆须温清酒服之,恒令酒气不绝为佳,不得醉吐。凡服乳之时,唯须少食,一日吃一升许饭,得满三日不出,即其乳。不随粪下,乳在腹内三日,练之便化为津液入人骨髓。若食多者,其乳未化,不至时乃随大便而出,徒损功夫,不得其效,唯须少食,满三日外,待旧粪出讫,任意作美食补之,三日补之,更欲服者还依上法将息如前。其乳多少者,任人贫富云云。

鉴真服钟乳随年齿方∶石钟乳,其味甘温无毒,年二十者服二两,乃至五十服五两,六十以上加至七两。各随年服之,吉。四十以下人,一两分为两服,五十以上,一服一两,两别和面三两搅溲面硬溲作,以五升铛中煮五,六沸即熟,和酒令汁尽服之,竟以暖饭押之七日以来,忌如药法。

李补阙练研钟乳法∶取枣膏和乳研捣令相得,每旦空腹服十五丸如梧子,以暖酒下,待饥方食,食宜用少不令饱,每日数数,任性饮酒,令体中熏熏,恒有酒气,使气宣行。

当服乳时,三日五日吃一两口仓米饭及少许臭败脯肉,及见丧孝尸秽并不须避,令其惯习,每年恒服一大斤,以来四时并得服,夏秋料理,立冬服之。

《石论》云∶若服药先后饮食相近者,难得药力。皆须晚食少食,不通过多,多则令药势不行,所以须少,少则易通,通则速得药力,宜慎之。空腹及下后不可服,更三日调养,然后始服石,若下后只服药,或药滞着曲奥之处经岁不解,亦服药之后仍行百步,即乳气入腹得力尤速。若觉热,进一两口冷饭行步消息,良之。

补乳法∶《石论》云∶凡服乳石十日,还十日补,百日、千日亦然,以此为率。坐卧起寝处必须香洁,衣服新鲜熏如法,常侵早起服药导引,则神清而药行,每五更初即起扣天鼓三十六通。又酒是性命之本,朝暮常须饮热美酒,恒令体中熏熏,仍不得饮白酒。又澡浴勿向汤水中坐,宜以汤水淋之。

《极要方》云∶凡三日服乳,还三日补之,十日服,十日补之,以为率。补乳法欲得饱食,服乳法欲得少食,补乳欲得食牛、羊、獐鹿等肉骨,煎取汁任意作美食啖之,不得食仓米臭肉等物。以外不忌。

凡初服乳至补讫,必不得行房出精。此最大忌,慎之。又服乳补日讫,亦可更一月许将养方可泄,不可顿泻。则令药气顿竭,慎之。

服乳得力候∶《极要方》方∶凡服乳得力之时,先觉脐下绕脐肉起,身体发热,食味甘美,其阳气日盛,数起之,慎不得近房。若后大起,唯行房慎不得出精,此为养其精气,令腹中肪成,乳气盈溢,遍流百脉,则令人阳盛而且热,百战不怠,永无五劳七伤。

服乳禁忌∶《删繁论》云∶凡禁之法,若药有乳石,须一月日外,若不如尔,非唯不得力翻致祸也。

《耆婆方》云∶凡服乳药,通忌生冷、酢滑、尘臭、大饱、大饥及嗔忧悲泣愁不乐,不得冒诸风雪及淹秽之事。常令酒食气温温然,恒取暖常自逍遥适意。服乳忌五茄地榆,为药去之。

《极要方》云∶不得眠嗔怒及大喜恐、房室,勿饮白酒冷醋等物,及多食饼热食及猪、鱼、酥、腐臭之物,亦不得食犬马百种杂肉及芸苔、胡荽、腻粉、面、饧、之食,不入产、生、丧、孝之家,不语人我服此药。

《石论》云∶凡服乳石,莫生嗔怒,调和情性,欢娱畅悦,节房室,省睡眠,不用大嗔,大喜、忧思哭泣,不宜食粗冷硬难消之物,可食细软甘美之味以调之。

凡服此药,禁忌陈臭、生葫、蒜、杂生菜、猪肉、肥羹、诸滑物、生鱼、脍、生冷油面等。

服乳发动对治法∶释慧义云∶钟乳发令人头痛,饮热酒即解。

《极要方》云∶凡服乳以后,身中先有诸病,多者乳力共病相政。病气犹强,乳力未成,必相对作。

凡术动钟乳,两目疼痛;海蛤动乳,令人头痛脑闷;仓米臭肉动乳,令人骨节发疮及发背。

房动乳,令人少气力四肢顽痹,不尔,令人面目身肿痿黄。食饮不调动乳,乍寒乍热,腹中碎痛,或痢或吐,见食闻臭。

四时节气冷热不调动乳,状似疟发,不早治之,变为黄胆。

今皆疗之方,着上件发动形状,必须细意察其所患根本,须各相当。知其审候,疗无不验,勿令失之毫毛,差之千里。

若诸果动乳,取甘草一两炙,麻黄一两,去节,切,以水二升,煮取半斤,和清酒半升,先火边灸令遣热微,彻欲汗,因即热服之,令尽被覆卧取汗即瘥。

仓米臭肉动乳,必须以豉作汤,细细服之,可五六度许,仓米自消,所患目瘥。

房室损乳者,必须闭气调之一日一夜。又须牛羊骨煎作羹食之。男患之令妇人KT身体,女患令男夫KT。必不得更犯。如此调之三日三夜,自然觉健。若食饮损乳者,以葱豉汤裹纳当归一两煮之去滓,温服之便瘥。仍未除者,可作芦根汤服之。

芦根(一握)地榆(一握)五茄(一握)切,以水三升,煮取一升,一服。

若四时节气冷热不调动乳者,必须作生熟汤,以器盛之,入汤中坐,勿动。须臾百节所有寒热之气皆从毛孔而出变作汗。若心中热闷者,还服少许热汤即定。

《耆婆服乳方》云∶若发热渴者,以生芦根一握,粟米一合,煮米熟饮之甚良。

又服乳讫,单服菟丝子三斤,大益人。又方∶车前子亦佳。
服红雪方第十七

《服石论》云∶八仙云∶绛雪疗诸百病,八公所授淮南王绛雪方者即此是也。公曰∶子得此方,当不夭不暴,神妙无比。大和先生名之曰通中散,深重此方每合之进上。又常劝人服之。世人或有窃得此方合之者,俗共名之曰红雪。皆尽不得其要决,又不经师口决,或药种短缺,分两参差,或合和失宜,煎练过度,故用之疗疾多不有效。今具载药数分两并四时合和方法口诀,要录,所主病状、服法、禁忌具件如后,合之者,不可率意加减,以误后人。

煎练过度,KT于药力。此皆按经方承师口诀,既免暴夭之忧,实亦存生之至要,宜宝秘,慎勿轻泄。非道之者,无妄传也。所主疗病状如后∶疗一切丹石发热,天行时行,温疟,疫疾,痈疽发背,上气咳嗽,香港脚风毒,肺气肺痈,涕唾涎粘,头风旋愤,面目浮肿,心胸伏热,骨热劳热,口干口臭,热风冲上,目赤热痛,四肢瘫痪,心忪惊狂,恍惚谬语,骨节烦疼,皮肤热疮,昏沉多睡,赤白热痢,大小便不通,解药毒、食毒、酒毒。

上患以前病者,并和水服之。

诸气结聚,心腹胀满,宿食不消,痰水积聚,醋咽呕吐,产后血运,中风闷绝,产后热病,坠堕畜血。

上患以前病者,并和酒服之。

又云∶上与病相当者,取一匙绛雪,以新汲水二大合及蜜水,亦得纳于水中,令消顿服之。

今按∶《外台方》云∶凡服石之后,若觉大热者,可服紫雪或金石凌或绛雪或白雪等,但半大升,取瓷研,一大两,香汤浴后,顿服之。云云。

又,今时之人随身强弱,或三四两,或五六两,熟研空腹服之。又,本方载作日之忌无服时之禁,而药中有朱砂、甘草、槐花,可忌血食海藻、猪肉。又私记云∶妇人有孕,不得服之。
服紫雪方第十八

《服石论》云∶紫雪疗香港脚毒,遍身烦热,口,口中生疮,狂易叫走。并解诸石草散药热毒发,猝热黄胆,瘴疫,毒疠,猝死,温疟,五尸,五注;心腹诸疾,绞刺痛,蛊毒,鬼魅,野道尸,骨蒸热毒,诸热风,时行疫气,小儿热惊痫,利血,诸热毒肿疠子,一切热,主之尤良。

病者强人一服二分三分,和水服之;小、老、弱人或热毒微者,服之以意减少;香港脚病经服石药发,热毒闷者,服之如神,水和四分服之。(以上《极要方》。)若香港脚冲心,取一小两和水饮之。若心战冲取半小两令消已,水下亦得。若有风痫,时时服之如前理丹石。若丹发头痛身体急或寒热不能饮食,即取一两加少芒硝和水饮之。若热痢,亦如前。若天行热病,亦如前。若欲痢者,加之一倍,空腹服之。若邪气者,渐渐服即并可也。《鉴真方》。

今按∶今世以此药二分当红雪一两,又依如《外台方》,可服大一两。
服五石凌方第十九

《服石论》云∶五石凌,食后以蜜水一杯,服方寸匕,大热者加至二匕。患热病黄者,服三匕即愈。初得热病,服二匕亦(立)愈。无禁忌。

私记云∶治一切热病及服金石散动闷乱热困者,以水一杯,服方寸匕,大热者,加至二匕。

今按∶今人或三两,或四两,水服之,吉。
服金石凌方第二十

《服石论》云∶金石凌,若有温疫热黄病,取少称一两,水和服之,即得瘥。若金玉诸石等发热,以水和称一两,上凝者服之。若病上发,少食服。若病下发,空腹服之,不可多服。大大冷,无禁忌。

今按∶《大清经》云∶一鸡子许,宜蜜水和服。又《外台方》大一两水服。
服金汞丹方第二十一

《金汞丹》方∶主解五石热毒发动,丈夫女人久患劳损。身体瘦薄,益气力,明眼睛,长发,悦泽颜色,兼除百草毒,除冷疾外无不治之。

服法∶每日二三丸或五六丸,以冷水下之。若热气盛发,服十丸二十丸,亦不简空腹食后。

凡欲服药时,当先沐浴斋戒,燃香向生气方闭眼誓念,至心敬礼,天地祥感,万物应化,皆自勤致其灵应,以此言之。服石吞药之毕,敬信为先,不可轻蔑。
服银丸方第二十二

银丸主一切虚热,明目,押虚风惊痫心热,一切热病皆悉除之。其功效不可言尽之。但病瘥止之,不限丸数多少。无禁,如银不满五两,随多少亦得之。食后服五丸,必不服泻药。

医心方卷第十九医心方卷第十九背记宇治本目录次第服石节度第一服石发动救解法第二服石四时发状第三服石反常性法第四服石得力候第五服石禁忌法第六十九(以上相违,此校重基重志本之处如御本次第,仍宇治本次第所注付也。未考可里打)
卷第二十
治服食除热解发方第一

《庞(薄江反,姓也)氏论》云∶凡药欲发之候,先欲频伸或苦头,头痛目疼,身体螈(尺制反)(子用反),或惊恐悸(其季反)动,周身而强,或耳中气满如(子对反)车之声,或体热剧于火烧,或如针刺噤燥(子老反,治也),恶寒,昧(莫贝反,目不明貌)昧愦(古对反,心乱貌也)愦,不知病处,或腹中燠热如烧(丁贯反)怀之也。其发甚者,腹满坚于材石,绕口青黑,大小便血而多无脉也。如此之病,归于大浇(古尧反),以瘥为期也。

《病源论》云∶夫服散之人,觉热即洗,觉饥则食,先食不时,失其节度,令石热壅(于容反)结,痞塞不解而生热也。故须以药除之。

《外台方》云∶凡服石之后,若觉大热者,可服紫雪或金石凌或绛雪或白雪等,但半大升水,取瓷(才资反)研一大两,香汤浴后顿服之。候一两行,利热乃退矣。凡此救急中,紫雪为上。

《千金方》云∶解一切药发,不问草石始觉恶方∶生麦门冬(八两)葱白(八两)豉(三升)三味,水七升,煮取二升七合,分三服。

又云∶治服散忽发动葱豉汤方∶香豉(二升)葱白(切,一升)干蓝(三两)甘草(二两)四味,切,以水七升,煮取三升半,分三服。

《短剧方》云∶解寒食散发,或头痛,或心痛,或腹痛,或胸胁肿满,或寒,或热,或手足冷,或口噤,或口疮烂,或目赤,或干呕(乌后反,下同),恶食气便呕吐,或狂言倒错,不与人相当,或气上欲绝,进退经时。散发百端,服前胡汤得下便愈方∶前胡(二两)夕药(三两)黄芩(二两)大枣(二十枚)甘草(二两)大黄(二两)凡六物,以水八升,煮取二升半,分三服。心胁坚满,加茯苓二两;胸中满塞急,加枳子一两;连吐,胸中冷,不用食,加生姜三两;虚乏,口燥,加麦门冬二两。若加药者,加水作九升也。(《录验方》∶夕药二两。)又云∶解散三黄汤,治散盛热实不除,心腹满,小便赤,大行不利,圯逆充胸中,口焦燥,目赤熏热方∶黄连(二两)黄芩(二两)大黄(一两)甘草(二两)芒硝(二两)凡五物,以水五升,煮取二升半,纳芒硝,令烊,分三服。

又云∶小三黄汤,是由来旧方,与前治同,煞石势胜前方,除实不如也。

大黄(一两)栀子(十四枚)黄芩(二两)豉(三升)凡四物,以水六升,先煮三物,令数沸,以豉纳汤中,取二升,分再服。

又云∶增损皇甫栀子豉汤,治人虚石盛,特折石势除热方∶豉(一升半)栀子(十四枚)黄芩(二两半)凡三物,以水六升,煮取三升,去滓,纳豉,令得二升,分三服。

又云∶解散热发,身如火烧黄芩汤方∶黄芩(三两)甘草(一两)枳实(二两)浓朴(一两)栝蒌(一两)夕药(一两)栀子(十四枚)凡七物,以水七升,煮取二升半,分三服。

又云∶解散、除热、止烦、杀毒、单行荠(音在)(音檷)汤方∶荠(半斤)凡一物,以水一斗,煮取三升,分三服,停冷冻饮料之。

又云∶解散除热单行凝水石汤方∶凝水石(四两)凡一物,以水四升,煮取二升半,服七合,日三。

《释慧义》云∶解散麦门冬汤方∶麦门冬(一升)豉(二升)栀子(十四枚)葱白(半斤)凡四物,以水六升,煮取二升,分再服。

《新录方》云∶解散方∶栀子仁(一升)葱白(一升)猪脂(四升)煎葱白,焦布绞去滓,一服如桃李,日二三,石当如沙,尿中出。

又方∶水服大麦、粳米五合,日二三。

又方∶饮热酒,使熏熏然醉。

又方∶饮牛乳五六升勿绝,佳。

又方∶数饮土浆,日一二。

《急药方》云∶丹石立验方∶甘草(二两,炙)干葛(二两)豉(一大合)上,以水五升,煎取四升,食前温吃,食后冷吃,若不止,更吃。

《石论》云∶除热调石芦根汤方∶生地黄(四两,切)麦门冬(二两,去心)甘草(一两,炙)芦根(四两)茯苓(三两)凡五物,细切,以水七升,煮取三升,去滓,冷,分三服。

又云∶甘豆汤方∶甘草(二两,炙)大豆(五合,拭)凡二物,以水五升,煮甘草,令减一升,纳大豆,煮取二升半,分三服。
治服石烦闷(莫围反)方第二

《病源论》云∶将适失宜,冷热相搏,石势不宣,犯热气,乘于脏,故令烦闷也。

《僧深方》解散甘草汤治散发烦闷不解方∶甘草(一两半)茯苓(一两)生姜(一两)凡三物,以水三升,煮取一升半,分三服。(今按∶《短剧方》∶甘草二两、黄芩二两、大黄二两,水五升,煮取二升,分三服。)张仲景云∶解散发烦闷欲吐不得,单服甘草汤方∶甘草(五两,切)以水五升,煮取二升,服一升,得吐便止。

《新录方》云∶烦热闷者方∶荠(切,三升)水四升,煮取二升,分饮之。

又方∶单饮生地黄汁,日二三升,佳。

又方∶饮二三升生葛根汁良。

《极要方》治乳石发动,烦闷头痛,或寒热脚冷,气不通方∶葱白(十四茎)豉(二大合)牛苏(一大两)上,于铛中铺葱,即安苏于葱上,即着豉以物兼铺上,缓火煎,候苏气消尽。即淋好酒一升半,良久即得,取屑,冷热,顿服,随性多少饮之。

又云∶石发烦闷者方∶滑石(十二分,研,令面,分两服)上,以水大升,合和滑石末,一帖,和搅,令散,顿服。
治服石头痛方第三

皇甫谧云∶或头痛欲裂,坐服药食,温作(普激反,漂谓之),急宜下之。曹歙云∶头面苦眩冒者,则解头结散发扇之,热甚,头痛面赤者,以寒水淋头。不瘥,以油囊盛水着头结中。

《外台方》云∶或头痛如刺,眼睛欲脱者,宜以香汤浴。须虚静大屋内,适寒温,先以汤淋大椎及KT上三五十碗,然后乃浴。勿令见风。浴讫,覆被安卧取汗,仍须吃葱根葛豉粥法。

葛根(三大握)干姜(六两)豉(三合)葱白(一大握)生姜(少许)椒(十五颗)先以水五大升,煮葱根,减半去滓,下葛及豉,煮取二升,去滓,细研少许,米作稀粥,并着葱白等。煮熟蒸热啜服之讫,依前覆被取汗,讫令妇人以粉遍身揩(苦骇反,拭也)摩侯孔合,半日许始可出外,其病亦瘥。
治服石耳鸣方第四

皇甫谧云∶或耳鸣如风声,汁出,坐自劳出力过瘥,房室不节,气并奔耳故也。勤好饮食,稍稍行步,数食节情即止。曹歙云∶耳鸣汁出,数数冷食,步行。
治服石目痛方第五

皇甫谧云∶或目痛如刺,坐,热气冒肝上奔两眼故也。勤冷食,清旦以温小便洗之。又云∶或头痛项强,两目疼者,以水洗浴即瘥。

《释慧义》云∶解散治目疼头痛方∶芎(三两)葛根(二两)细辛(二两)防风(三两)五味子(三两)术(四两)茯苓(四两)黄芩(二两)人参(二两)凡九物,以水一斗三升,煮取三升,分三服。

《僧深方》治散家目赤痛KT(蕤,儒佳反)人洗汤方∶KT仁(二十枚)细辛(半两)苦竹叶(一枚)黄连(一两)凡四物,水三升,煮取一升半,一方取半升,可日三洗,亦可六七洗。
治服石目无所见方第六

皇甫谧云∶或目冥无所见,坐饮食,居处温故也。脱衣自劳洗,促冷冻饮料食,须臾自明了。

《释慧义》云∶散发,热气冲目,漠漠无所见方∶黄连(去毛)干姜细辛KT核凡四物,等分,(音甫)咀(音沮,咀嚼也。殊伦反),绵裹,淳酒五升,以药纳中,于铜器中煮取二升半,绵注洗目,使入中,日再。
治服石鼻塞方第七

《病源论》云∶发则将冷,其热尽之后冷气不退者。冷乘于肺,肺主气,开窍于鼻,其冷滞结,气不宣通,故鼻塞。

曹歙云∶鼻口臭,冷冻饮料冷洗。

秦承祖云∶治解散热势尽,肺冷鼻塞,宜服茱萸汤方∶蜀椒(一升)甘草(一两)干姜(一两)术(一两)桂心(一两)茱萸(一两)凡六物,细切,以汤六升,煮取二升半,分为再服。
治服石齿痛方第八(十二)

皇甫谧云∶或断肿唇烂,齿牙摇痛,颊(苦协反,面也)车噤,坐犯热不时救故也。当风张口,使冷气入咽,嗽(桑浓反)寒水即瘥。
治服石咽痛方第九(十三)

皇甫谧云∶或咽中痛,鼻塞,清涕出,坐温衣近火故也。促脱衣,冷水洗,当风以冷石熨咽颊五六过自瘥。
治服石口干燥方第十(八)

《短剧方》治口干燥渴呕(乌后反,吐也)不下食方∶芦根多少,煮取浓汁,以粟作粥浆,服多少任意。

《葛氏方》治口中热,干燥,乌梅枣膏分等,以蜜和丸如枣,含之。

《苏敬本草注》云∶口干,食熟柿。

薛侍郎云∶若口干,即绞甘蔗汁任服。
痛方第十一(九)

《僧深方》解散栀子汤方∶黄芩(三两)栀子(四枚)豉(三升)凡三物,咀,以水五升,先煮栀子、黄芩,令得三升,绞去滓,乃纳豉,煮令汁浓,绞去滓,平旦服一升,日三,甚良。
治服石口中发疮方第十二(十)

曹歙云∶口中生疮,舌强,服栀子汤,在上。

《僧深方》云∶解散失节度,口中发疮方∶黄芩(三两)升麻(二两)石膏(五两,末)凡三物,以水六升,煮取三升,去滓,极冷,以嗽(桑浓反)口中,日可分十过。(《小品方》∶若喉咽有疮,稍稍咽之佳。)《短剧方》云∶治口疮小柏汤方∶龙胆(三两)黄连(二两)子柏(四两)凡三物,以水四升,先煮龙胆、黄连,取二升,别渍子柏,令水淹潜(荚虚反),投汤中,和,稍含之。
治服石心噤方第十三(十一)

《病源论》云∶其寒气盛胜于热,荣卫痞涩不通,寒气内结于心,故心腹痛而心噤寒也。

其状心腹痛而寒,噤不能言是也。

《僧深方》云∶解散人参汤常用治心噤或寒噤不解方∶人参(二两)干姜(一两)甘草(三两)茯苓(一两)栝蒌(二两)白术(一两)枳实(一两)凡七物,水六升,煮取二升五合,分三服。

秦承祖云∶疗散豉酒方∶散发不解或噤寒,或心痛心噤,皆宜服之方。用∶美豉(二升,勿令有盐)凡一物,熬令香,以三升清酒,投之一沸,滤(胡移反)取温服一升,小自温暖,令有汗意。若患热不可取汗者,但服之,不必期令汗也。
治服石心腹胀满方第十四

《病源论》云∶居处犯温,致令石势不宣,内壅腑脏与气相搏,故心腹胀满也。

皇甫谧云∶或腹胀欲决,甚者断衣带。坐寝处久下热。又衣温、失食、失洗、不起行、促起行、饮热酒、冷食、冷洗、当风栉梳(所菹反)而立。

《僧深方》解散三黄汤治散发心腹痛,胀满猝急方∶大黄黄连黄芩(各三两)凡三物,以水七升,煮取三升,分三服,得下便止。

今按∶三黄汤亦出《短剧方》,在上除热篇。

《短剧方》云∶三黄汤治散盛热实不除,心腹满,小便赤,大行不利,地逆充胸中,口焦燥,目赤熏热方∶黄连(二两)黄芩(二两)大薰黄(二两)甘草(二两)芒硝(二两)凡五物,以水五升,煮取二升半,纳芒硝,令烊,分三服。
治服石心腹痛方第十五

《病源论》云∶膈间有寒,胃管有热,寒热相搏,气逆攻乘心,故心腹痛也。

皇甫谧论云∶或心痛如锥刺。坐当食而不食,当洗而不洗,寒热相交,气结不通,结在心中,口噤不得息,当绞口捉与热酒,任本性多少。其令酒两得,行气自通,得噫(于基反,根声也),因以冷水洗淹布巾,着所苦处,温复易之自解。解便速冷食,能多益善。若大恶,着衣小使温,温便去衣即瘥。于诸痛之中,心痛最为急者,救之若赴汤火,乃可济耳。

《短剧方》云∶治散发心痛,腹胀兼冷,动热相格(古伯反。谷也,止也,阙也,正也)不消,甘草汤方∶甘草(一两)栝蒌(二两)术(二两)枳实(二两)栀子仁(二两)凡五物,以水七升,煮取二升,分三服。

又云∶单行甘草汤方∶甘草(四两)凡一物,以水五升,煮之折半,冷之,顿服尽,当大吐。患心腹痛,服诸药无效者,宜服此汤。

张仲景方云∶黄芩汤治散发腹内切痛方∶栀子(二两)香豉(三升)黄芩(二两)凡三物,切,绵裹,以水九升,煮取三升,分三服,以衣覆卧,亦应有汗。

《僧深方》云∶若散发悉口噤心痛,服葱白豉汤方∶葱白(半斤)豉(三升)甘草(二两)生麦门冬(四两,去心)凡四物,以水五升,煮取二升,分再服。(一方加茱萸一升。)
治服石腰脚痛方第十六

《病源论》云∶肾(时忍反,下同)主腰脚,服石,热归于肾。若将适失度,发动石热,气乘腰脚,石势与血气相击,故脚热肿痛也。其状脚烦热而腰痛也。

皇甫谧云∶或脚疼欲折,坐下温,宜常坐寒床,以冷水洗起行。又云∶或腰痛欲折,坐衣浓体温,以冷水洗,冷石熨之。
治服石百节痛方第十七

皇甫谧云∶或百节酸(丝官反,下同)痛,坐卧下大浓,又入温被中,又衣温不脱故也,卧下当极薄大要也。被当单布,不着绵衣,亦当薄且垢,故勿着新衣,宜着故絮也。虽冬寒当常KT头受风,以冷石熨,衣带初不得系也。若犯此酸闷者,促入冷水浴,勿忍病而畏浴也。

《极要方》云∶服乳石不得将慎,冲热失食,憎寒头痛,百节酸(素官反,疼痛也)疼。

口唇干焦(即消反)舌卷方∶知母(八分)石膏(十三分,碎)升麻(六两)通草(八分)硝石(八分,汤成下)竹青皮(六分)露蜂房(二枚,炙)切,以水二大升,煎取九合,食后分温三服。

《短剧方》云∶解散二物麻子豉汤治人虚劳,下焦有热,骨节疼烦,肌急内圯(备美反,毁也。又岸崩貌),小便不利,大行数少,吸吸口燥。少气折石热方∶豉(二升)麻子〔五合,擂(力推反,研物也)取仁〕凡二物,研麻子仁,以水四升,煮取一升半,分服五合,日三便愈,神验。
治服石手足逆冷方第十八

《石论》云∶治解散,胸中有热,手足逆冷,若寒,甘草汤方∶甘草(一两)橘皮(二两)凡二物,水三升,煮取一升半,以绵缠箸头,数取汁服,须臾间复服,不可废食。
治服石面上疮方第十九(二十)

《广济方》云∶石气发热,身体微肿,面上疮出方∶寒水石以冷水于碗(乌管反)中研,令汁浓,将涂疮,干即点,勿停。
治服石身体生疮方第二十(十九)

《病源论》云∶将适失宜,外有风邪(以遮反),内有积热,热乘于血,血气壅滞,故使生疮。

《千金方》云∶治散发疮痛不可忍方∶冷石(五两)一味,下筛,粉疮上,燥痛,须臾静定。(今检∶《本草》陶注云∶硝石,一名冷石。)又云∶散发生细疮方∶黄连芒硝(各五两)二味,水八升,煮黄连,取四升,去滓,纳芒硝,以布取贴疮上,数数易之。

《短剧方》云∶治通身发疮,擘(博厄反)折经(五根反,又语斤反)日,用水不得息者方∶锉(粗卧反)胡叶煮,温洗渍尤良。冬取根煮饮之。

《录验方》云∶解散烂疮洗汤方∶黄连(半斤)苦参黄芩(各半斤)凡三物,以水二斗,煮取一斗,去滓,极冷洗之,日三。

《新录方》云∶凡散发疮方∶水研大麻子,涂,日二三。

又方∶水摩蔓荆子,涂,日二三。

又方∶水和豉,研为泥涂上,日二。

秦承祖∶疗散发热疮三黄膏方∶大黄(二两)黄连(二两)黄芩(二两)凡三物,以好苦酒渍之足,相淹一宿,猪膏二斤,微火煎三沸,沸辄下,去滓,摩之。

又云∶疗散浮在肌肤作疮方∶子柏黄皮,末,下筛,鸡子白,和如泥,先煮子柏汁,洗却以敷之,不过再,便愈。
治服石结肿欲作痈方第二十一

《录验方》云∶解散除热,热结肿坚,起始欲作痈,大黄汤方∶升麻大黄夕药枳实(各二两)黄芩(三两)甘草当归(各一两)凡七物,以水八升,煮取二升半,分三服。快下,肿即消。(《短剧方》云∶升麻汤。)
治服石痈疽发背方第二十二

《千金方》云∶痈疽发背,皆由服五石、寒食、更生散所致,亦有单服钟乳而发者。又有生平未服石而自发者,此是上世有服之者,其候稍多。

又云∶养生者,小觉背上痈痒有异,即取净土水和作泥,捻(都念反)作饼子,径一寸半,浓二分,以粗艾作炷(音主,下同),灸泥上灸之。一炷一易。饼子若粟米大时可灸七饼,若如榆荚大,灸七炷即瘥。若至钱许大,日夜灸不住即瘥。

又云∶恒冷水射之,渍冷石熨之,日夜勿止,待瘥住手。

《庞氏论》云∶凡诸服草石散者,皆不可灸身体,令人喜发炎疽疮也。若体有疮不可温治也。唯以水渍布贴之,烧李子中人作膏,以摩疮上,诸洗如故。薛侍郎云∶若发疮及肿,但服五香连翘汤等。(在《短剧》治恶脉方。)又云∶若肿有根,坚如铁石,带赤色者,服汤仍以小,小艾炷当肿上灸一两炷为佳。
治服石身体肿方第二十三

皇甫谧云∶或周体悉肿,不能自转,从坐久停息,不饮酒,药气沉在皮肤之内,血脉不通故也。饮酒冷洗,自劳行步即瘥。不能行者,使健人扶曳行之。

《秦承祖方》云∶凡药发之后,身体浮肿,多取冷所致,甘草汤方∶甘草(三两,炙)栀子(十四枚)凡二物,以汤五升,煮取一升半,分再服。

又云∶疗散发赤肿贴方∶黄芩(四两)鸡子(四枚)吴茱萸(四两)凡三物,捣下粗筛,以水一升,合鸡子白于器搅之,令沸出,染巾,贴肿上,温复易之,数十过。

《千金方》云∶若从脚肿向上稍进入腹则杀人方∶赤小豆(一斗)以水三斗,煮烂出豆,以渍脚膝以下,日一,数日为之,愈矣。

又云∶若已入腹者,不须渍煮豆食之,断一切姜菜饮食米曲,唯只食豆一物,渴饮汁,瘥乃止。

又云∶散发赤肿摩膏方∶生地黄(五两)大黄(一两)木(李。杏,或本此字)仁(三十枚)生章陆(三两)四味,切,酢渍一宿,猪脂一升,煎章陆黑,去滓,摩之,日三夜一。
治服石身体强直方第二十四

皇甫谧云∶或关节强直不可屈伸,坐久停息,不自烦劳;药气胜正气,结而不散,起(越,或本此字)沉滞于血脉中故也。任力自温,便冷洗即瘥。任力自温者,令行动出力,足劳则发温也。非浓衣近火之温也。

《释慧义》云∶治寒噤似中恶,手脚逆冷,角弓反张,其状如风,或先热后寒,不可名字。若先寒者,用冷水二三升洗脚,使人将之。先热者,以生熟汤四五升许洗之,若体中觉直者,是散,急服此汤方∶栝蒌根(三两)栀子(二十一枚,擘)人参(一两)甘草(一两,炙)香豉(一升)石膏(三两,末)葱叶(三两)凡七物,细切之,以水八升,煮取二升半,分三服。

《僧深方》云∶治散发猝死,身体强直,以手着口上,如尚有微气,即便两人汲水灌,灌洗亦两三时间,死者乃战,战便令人扶曳行,便得食,食竟复劳,行半日许便愈。此药失节度,所为似中恶。解之方∶黄连大黄黄芩(各二两)豉(一升)栀子仁(十四枚)凡五物,以水七升,煮取三升半,去滓,纳豉,更煮取三升,三服。近有用此汤即得力也。
治服石发黄方第二十五

《病源论》云∶饮酒内热,因服石,石热又热,热搏脾胃,脾胃主土,其色黄而候于肌肉,积热蕴结,发于肌肤,故成黄。

《录验方》治散发或黄发热毒,胸中热气烦闷。胡叶汤方∶胡叶(一把,切)凡一物,以水七升,煮取二升半,分再服,一剂便愈,亦治通息发黄,终日用水,不得息者,浓煮大茎叶,适寒温,自洗渍尤良。

《新录方》治发黄者,捣苍耳,取汁,服一升,日一。(今按∶服紫雪红雪等可下去积热。)
治服石呕(乌后反)逆方第二十六

《病源论》云∶将适失宜,脾胃虚弱者,石热结滞,乘于脾胃,致令脾胃气不和,不胜于谷,故气逆而呕,谓之呕逆也。

秦承祖云∶芒硝丸,治散患积热逆呕方∶芒硝(三两)大黄(三两)杏仁(三两)凡三物,各别捣冶,先末大黄,芒硝下筛,后捣杏仁子,令如膏,乃合三物以蜜丸,服如梧子二丸,日二,多少随意消息之。

《短剧方》云∶解散发,振动烦闷呕逆,人参汤方,法议道人所增损方用如此∶人参(二两)甘草(二两)术(二两)黄芩(一两)栝蒌(二两)凡五物,以水七升,煮取二升,分三服。
治服石咳嗽方第二十七

曹歙论云∶咳逆咽痛,鼻中窒塞,清涕出,本皆是中冷之常候也,而散热亦有此诸患冷咳者,得温是其宜也。若是热咳者,得酒于理,当瘥和也。欲分别之者,饮冷转剧,剧者果是冷咳也。饮冷觉佳者,果是药热咳也。

皇甫谧云∶或咳逆,咽中伤,清血出,坐卧温故也,或食温故也。饮冷水,冷石熨咽外。

(今按∶《红雪方》云∶疗一切丹石发热,上气咳嗽等,病者并和水服之。)秦承祖云∶疗散咳嗽胆呕,胸中冷,先服散,散盛不得服热药杏仁煎方∶杏子中仁(三十枚)白蜜(六合)紫菀(一两)干姜(一两)牛脂(一升)凡五物,冶合下筛,和以蜜,微火煎,令可丸,丸如梧子,服一丸,日三,老小甚佳。
治服石上气方第二十八

《病源论》云∶服散将适失所,取温大过,热搏营卫而气逆上,其状胸满短气是也。

《僧深方》竹叶汤治散发上气方∶生竹叶(二两)甘草(一两)黄芩(一两)大黄(一两)栀子(十枚)茯苓(一两)干地黄(六分)凡七物,以水五升,煮取二升一合,服七合,日三。
治服石痰方第二十九

《病源论》云∶服散而饮过度,将适失宜,衣浓食温,则饮结成痰。其状痰多则胸膈否满,头眩痛癖结则心胁弦急是也。

皇甫谧云∶恶食如臭物,坐温衣作也。当急下之。若不下,万救终不瘥也。薛公曰∶以三黄汤下之。

《僧深方》服散家痰闷,胸心下有阻痰客热者。吐之方∶甘草(五两)以酒五升,煮取二升半,分再服。欲吐者,便快荡去。

《短剧方》云∶白薇汤治寒食药发,胸中澹酢,干呕烦方∶白薇(二两)半夏(二两,洗)干姜(一两)甘草(半两)凡四物,以酢五升,煮取三升,分服五合。夫酢酒能令石朽烂。
治服石不能食方第三十

张仲景云∶半夏汤治散发,干呕不食饮方∶半夏(八两,洗,炮)生姜(十两)桂心(三两)橘皮(三两)上四物,以水七升,煮取三升半,分三服,一日令尽。
治服石酒热方第三十一

《短剧方》∶治酒热发热法∶积饮酒石热,因盛数散行经络中,使气力强,肾家有热,欲为劳事,劳事多使肾虚,则热盛。热盛心下满,口焦燥欲饮,饮随呕吐,不安饮食也。宜饮葛根汤,安谷神,除热止吐渴也。
治服石淋小便难方第三十二

皇甫谧云∶或淋不得小便,坐久坐下温及骑马鞍中热,热入膀胱故也。大冷食以冷水洗少腹,以冷石熨,一日即止。

《僧深方》治散发小便难。其状如淋方∶葵子(五合)凡一物,以水二升半,煮取一升,一服昼,须臾便利也。

《录验方》云∶解散闭闷结,小便不通如淋方∶大黄(一两)麻子仁(半斤)夕药(一两)茯苓(二两)黄芩(一两)凡五物,以水五升,煮取二升半,分再服。(《短剧方》同之。)《广利方》云∶石气头痛、烦热、口干、小便赤少方∶露蜂房(十二分,炙)上,以水二大升,煎取八大合,分温二服。当利小便,诸恶石毒随小便出。
治服石小便不通方第三十三

《病源论》云∶夫服散石者,石热归于肾而内生热,热结小肠,胞(被交反)内否涩,故小便不通也。

《短剧方》云∶解散小便不通神良方∶桑螵蛸(三十枚)黄芩(一两)凡二物,以水一升,煮取四合,顿服之。

《录验方》云∶解散利小便致良葵子汤方∶三岁葵子(一升)上一物,以水三升,煮取一升半,冷暖随意,顿食饮。不能稍服。一方加滑石三两。
治服石小便稠数方第三十四

皇甫谧云∶或小便稠数,坐热食及啖诸含热物饼黍之故也,以冷水洗少腹自止。不瘥者,冷水浸阴又佳。若复不解,服栀子汤即解。

《短剧方》云∶解散除热,小便数少,单行葵子汤方。亦治淋闭不通,三岁葵子亦可用。

陈葵根(切,一升)以水三升,煮取二升,暖如人肌,稍服之。

《外台方》云栀子汤∶栀子仁(二合)甘草(二两,炙)黄芩(二两)芒硝(二两,汤成纳之)以水五升,煮取二升,分温二服,取利即瘥。
治服石小便多方第三十五

《病源论》云∶将适失度,热在下焦,下焦虚冷,冷气乘于胞,故胞冷不能制于小便,则小便多也。

《新录单方》云∶鸡肠草煮为羹啖之,捣汁服五六合,日二。

又方∶棘直刺、枣针各三升,捣筛蜜丸,酒若饮服三十九,日二。
治服石大小便难方第三十六

《病源论》云∶积服散盛在内,内热气乘于大小肠,肠否涩故大小便难也。

《华佗方》解散热胀满大小便不通方∶枳实(四两)由跋(四两)凡二物,以水二升渍之,令药泽尔,乃煮得半升,去滓,稍稍饮,多少,(于云反)自瘥。

《张仲景方》治寒散大小行难方∶香豉(二升)大麻子(一升,破)上二物,以水四升,煮取一升八合,去滓,停冷,一服六合,日三。

《新录单方》云∶大小便难,服葵子。

《录验方》云∶若大小便秘塞不通,或淋沥,尿血,阴中疼痛,此是热气所致,熨之则愈。

熨法∶先以冷物熨少腹,冷熨以后,复以热物熨之,用有热物熨已,复冷熨之。
治服石大便难方第三十七

《病源论》云∶将适失宜,犯温过度,散热不宣。热气积在肠胃,故大便秘难也。

皇甫谧云∶或大行难,腹中坚固,如蛇,盘,坐犯温久积,腹中干粪不去故也。消苏若膏,使寒服一二升,浸润则下。不下更服下药即瘥。薛公曰∶若不下,服大黄朴硝等下之。

秦承祖云∶朴硝、大黄煎治胃管中有燥粪,大便难。身体发疮解发方∶大黄(金色者二两)朴硝(细白者二两)凡二物,以水一斗,煮减三升,去滓,着铜器中,于汤上微火上煎,令可丸。病患强者可顿吞,羸人中服可后,宜得羊肉若鸭麋肉羹补之。

《录验方》解散不得大行方∶大黄(四两)桃仁(三十枚)凡二物,以水六升,煮取二升,二服。一方∶大黄二两,桃仁五十枚。
治服石大便血方第三十八

《新录单方》云∶散发大便血者方∶葱白(切,一升)豉(一升)水四升,煮取二升,二服。

又方∶车前草切三升,水五升,煮取二升,三服。
治服石下利方第三十九

皇甫谧云∶或下痢如寒中,坐行止食饮犯热所致,人多疑是本疾,又有滞癖者,皆犯热所为,慎勿疑也。速脱衣冷食饮冷洗。或大便稠数,坐久失节度,将死之侯也。如此难治矣。

为可与汤下之。偿十得一生耳。不与汤必死,莫畏不与也。下已致死,令人不恨。

《短剧方》解散除热止利黄连汤方∶甘草(一两)黄连(二两)升麻(一两)栀子(十四枚)豉(五合)凡五物,以水五升,煮取一升半,分再服。

秦承祖云∶黄连丸,解寒食散发大注下肠胃方∶黄连(筛成屑,三升)乌梅(百二十枚,去核)凡二物,冶合下筛,以蜜和之,更捣三千杵,丸如梧子,服二十丸,日可十服。病甚者,一日可至三四百丸。

《新录方》云∶散发下利者,服牛羊酪一升,日二。

又方∶水和大麦及米,服一二升。

又方∶致三升,水四五升,渍经宿或煮三四沸,冷服一升,日二三,即断。
治服石热渴方第四十

《病源论》云∶夫服石之人,石热归于肾而热充腑脏,腑脏即热,津液竭燥,故渴引饮也。

《短剧方》治散热盛渴方∶生地黄(一斤)小麦(二升)竹叶(切,一升)枸杞根(一斤)凡四物,切,以水一斗,煮取九升,渴者饮之。

又云∶吕万舒解热止渴饮地黄汁方∶生地黄(一斤)小麦(三斤)枸杞根(三斤)凡三物,以水三斗,煮取二斗汁,渴者饮之。

又云∶单行枸杞白皮汤解散除热止渴方∶枸杞根白皮(十斤)以水三斗,煮取一斗,分服一升。
治服石冷热不适方第四十一

《僧深方》解散人参汤治散发作冷热不适方∶人参(二两)白术(二两)枳实(二两)栝蒌(二两)干姜(二两)甘草(二两)凡六物,以水八升,煮取二升半,分三服。
治服石补益方第四十二

秦承祖云∶当归九治散发。

《录验方》解散除胸中热,益气竹叶汤方∶竹叶(二两)甘草(十两)白术(一两)大黄(二两)凡四物,以水七升,煮取二升半,分五合,一服。

又云∶生地黄煎补虚除热将和取利也。

《僧深方》解散内补治百病巨胜汤方∶胡麻(一升,熬)生地黄(一升,切)大枣(二十枚)夕药(一两)生姜(四两)甘草(一两)麦门冬(四两)桂心(一两)人参(一两)细辛(一两)凡十物,以水九升,煮取四升,分四服。

《新录方》散发动后虚内补方∶枸杞煎单含咽,如桃李许,日二三。粥饮中亦好,冷难散酒服。忌鲤鱼。虚热人并得饵之。
治服石经年更发方第四十三

《短剧方》云∶荠汤华佗解药毒或十岁或三十岁而发热或燥,燥如寒,欲得食饮,或不用饮食,华佗散法。有石硫黄热郁,郁如热浇洗失度,错服热药,剧者擗(房益反,抚也)裂石热燥,燥如战,紫石英势闷暗喜卧起无气力,或时欲寒,皆是腑气所生,脏气不和,宜服此汤。

荠(四两)甘草(一两)人参(一两)蓝子(一两)茯苓(一两)夕药(一两)黄芩(一两)无菁子(三升)凡八物,切,以水一斗,先煮芜菁子得八升,绞去滓,煮药得三升半,分服七合,日三。

若体寒倍人参,减黄芩,用半两也。若气嗽,倍茯苓,减荠,去一两。

又云∶应杨州所得吴解散,单行葱白汤方,药沉体中数年更发。治之方∶生葱白(一斤)凡一物,以水八升,煮取四升,分服一升,使一日尽之,明日盒饭温食饮于被中,不后发便为知也。不过三剂都愈也。

又云∶治散热气结滞,经年不解数发者方∶胡叶(半斤)凡一物,以水七升,煮取二升半,分作三服,尽一剂便愈。
卷第二十一
治妇人诸病所由第一

《千金方》云∶论曰∶夫妇人所以有别方者,以其血气不调,胎妊产生,崩伤之异故也。

所以妇人之病,比之男子十倍难疗。若四时节气为病,虚实冷热为患者,与丈夫同也。唯怀胎妊挟病者,避其毒药耳。

又云∶女人嗜欲多于丈夫,感病则倍于男子,加以慈恋、爱憎、嫉妒、忧恚、深着坚牢,情不自抑,所以为病根深,疗之难瘥。故傅母之徒亦不可不学。

《短剧方》云∶古时妇人病易治者,嫁晚,肾气立,少病,不甚有伤故也。今时嫁早,肾根未立,而产伤肾故也,是以今世少妇有病,必难治也。早嫁早经产虽无病者,亦夭也。
治妇人面上黑方第二

《病源论》云∶妇人面上黑者,或脏腑有痰饮,或皮肤受风邪,血气不调,致生黑。

若皮肤受风外治则瘥;腑腑有饮,内疗方愈。

《僧深方》治妇人面方∶取茯苓,冶筛,蜜和,以涂面,日四五。

又方∶取桃仁,冶筛,鸡子白和以涂面,日四五。

《经心方》治面方∶取杏仁末和鸡子白,敷之一宿,即落。
治妇人面上黑子方第三

《病源论》云∶妇人面上黑子者,风邪搏血气变化所生。

《如意方》去黑子方∶乌贼鱼骨细辛栝蒌干姜蜀椒分等,苦酒渍三日,牛髓一斤煎黄色,绞以装面,令白悦,去黑子。(面黑子方详在上帙四卷。)
治妇人妒乳方第四

《病源论》云∶妇人妒乳者,由新产后,儿未饮之,或饮不能泄,或新断儿乳,捻(都念反,下也。)其乳汁不尽,皆令乳汁蓄结,与血气相搏,则壮热大渴引饮,强掣痛,手不得近是也。初觉便以手助将去其汁,并令旁人助KT引之,不尔成疮有脓,其热势盛则结变成痈。

《短剧方》治妒乳方∶以鸡子白和小豆散涂乳房,令冷以消结也。

又方∶黄芩白蔹夕药三物,分等,下筛,以浆服一钱五匕,日五服。(《集验方》同之。)又云∶宜以赤龙皮汤,天麻草汤洗之,敷黄连胡粉膏。

赤龙皮汤∶槲树皮切三升,以水一斗,煮取一斗五升,夏月冷用之,秋冬温之,分以洗乳。

天麻草汤∶天麻草切五升,以水五升,煮取一斗,随寒温分洗乳。(今按∶《耆婆方》茺蔚一名天麻草。)《葛氏方》治妇人妒乳肿痛方∶削取柳根皮熟捣,火温帛裹熨上,冷更易。

又方∶梁上尘,苦酒和涂,又治阴肿。

又方∶末地榆白皮,苦酒和敷。

又方∶白芨、夕药。酒服方寸匕,又可苦酒和涂之。

又方∶鼠妇虫,以涂之。

又方∶车前草捣,苦酒和,涂之。

《产经》妒乳方∶取牛屎烧末,以苦酒和,涂上。

又方∶左乳结者,去右乳汁;右结者,可去左乳汁。(《集验方》同之。)《集验方》治妒乳方∶急灸两手鱼际,各二七壮,断痈脉也。

又方∶以鸡子白和小豆散涂乳房,令消结也。

又方∶取葵茎捣筛,服方寸匕,日三,即愈。

又方∶捣生地黄,敷之,热则易。
治妇人乳痈方第五

《病源论》云∶妇人乳痈者,肿结皮薄以泽,是为痈也。寒搏于血则涩不通,而气积不散,故结聚成痈。亦因乳汁蓄结,与血相搏,蕴积而成痈也。乳痈年四十已还,治之多愈;年五十以上,慎,不当治之,多死。乳痈久不瘥,因变为(章俱反)。

《养生方》云∶妇人热食汗出,露乳荡风,喜发肿,名吹乳,因喜作痈。

《短剧方》治乳痈方∶大黄(二分)草(二分)伏龙肝(二分)生姜(二分)凡四物,合筛,以姜并舂冶,以KT(呼鸡反)和,涂乳有验。

《范汪方》治妇人乳痈方∶大黄,冶筛,和生鸡子,敷肿上,燥复更敷,不过三,愈。

又方∶灶中黄土以鸡子黄和,涂之。

又方∶熬粉水和,敷之。

又方∶大黄、鹿角,二物分等,烧鹿角与大黄筛,以鸡子白和,贴之。

《僧深方》治乳痈方∶末黄柏,鸡子白和,涂之。

又方∶捣根,敷之。

又方∶赤小豆末,鸡子白和,敷之。

又云∶治妇人乳痈生核,积年不除,消核防风敷方∶草(八分)芎(八分)大黄(十分)当归(十分)防风(十分)夕药(十分)白蔹(十分)黄(十二分)黄连(十分)黄芩(十分)枳子中仁(四分)十一物,冶筛,以鸡子白和,涂故布若练上,以敷肿上,日四五,夜三。

《千金方》乳痈二三百日,众疗不瘥,但坚紫色方∶柳根削上皮,捣熬令温,盛囊熨乳上。(《集验方》云∶一宿则愈。)《医门方》疗妇人乳痈初得令消方∶赤小豆,草各分等,苦酒和,涂上立愈。

《龙门方》治乳热肿方∶冷石熨之,瘥。

又方∶取朱书乳,作鱼字,验。(《范汪方》同之。)《录验方》治乳痈坚如石,众医不能治方∶桂心(二分)乌头(二分)甘草(二分)凡三物,冶合,淳酢和,涂肿上。

《产经》乳痈符KT(用新笔朱书肿上。)又方∶取焦瓦捣碎,和酢,涂之立瘥,干易。
治妇人乳疮方第六

《病源论》云∶此谓肤腠虚,有风湿亦气,乘虚客之。与血气相搏,而热加之,则生疮。

《范汪方》治乳端生气出汁痛方∶鹿角(二分,烧)甘草(一分)冶合,和鸡子黄,置暖灰上,令温,日二敷之。

《医门方》疗乳头裂破方∶捣丁香,敷之,立瘥。(今按∶《崔侍郎方》同之。)《僧深方》取韭根烧,粉疮,良。

《集验方》妇人女子乳头生小浅热疮,搔之黄汁出,浸淫为长。百种治不瘥者,经年月,名为乳病。

宜以赤龙皮汤及天麻草汤洗之。敷二物飞乌膏及飞乌散。

飞乌膏方∶用烧朱砂作水银上黑烟,名汞(胡动反,卢谷反)粉者(三两)矾石(三两,熬令焦)二物,下筛,以甲煎和之,令如脂以敷乳疮,日三。作散不须和,有汁有自着者可用散。

亦敷诸热疮、黄烂、浸淫汁痒疮、丈夫阴蚀痒湿、小儿头疮、月食耳疮、口边肥疮、蜗疮,悉效。

又云∶若始作者,可敷黄连胡粉散,佳。

黄连胡粉膏散方∶黄连(二两)胡粉(十分)水银(二两)凡三物,末黄连,令消,以二物相合,合皮裹熟之,自和合也,纵不成一家,旦得水银细散入粉中也。以敷乳疮,诸湿痒疮,若着甲煎,为膏。
治妇人阴痒方第七

《病源论》云∶妇人阴痒,是虫食所为,其虫动作势微,故令痒,若重者则痛。

《葛氏方》妇人阴若苦痒搔者方∶蛇床草、节、刺,烧作灰,纳阴中。

《孟诜食经》治妇人阴痒方∶捣生桃叶,绵裹,纳阴中,日三四易。亦煮汁洗之。今按煮皮洗之,良。

《录验方》治妇人阴痒方∶枸杞根切一斤,以水三升,煮,适寒温,洗之即愈。

《僧深方》妇人阴痒方∶黄连黄柏各二两,以水三升,煮取一升半,温洗,日三。

《极要方》阴中及外痒痛方∶大黄(三两)甘草(二两)水三升,煮取二升,渍洗,日三。

《集验方》治妇人阴中痒如虫行状方∶矾石(三分,熬)芎(四分)真丹砂(少许)三物,下筛,以绵裹,纳阴中,虫自死。
治妇人阴痛方第八

《病源论》云∶妇人阴痛之病,脏虚风邪乘之,冲击而疼痛而已。

《葛氏方》妇人阴燥痛者方∶煮甘草,地榆,及热,以洗之。

又方∶以盐汤洗之。

《延龄图》云∶若阴中重痛,炙枳实熨之,良。
治妇人阴肿方第九

《病源论》云∶阴肿者,是虚损受风邪所为也。

《葛氏方》妇人阴肿痛者∶熬矾石二分,大黄一分,甘草半分炙,末,以绵裹如枣核,以导之。(《延龄图》同之。)又方∶炙枳实熨之。

《僧深方》阴肿痛方∶黄芩(一分)矾石(一分)甘草(二分)下筛,如枣核绵裹,纳阴中。
治妇人阴疮方第十

《病源论》云∶妇人阴疮者,由三虫九虫动作侵食所为也。诸虫在人腹内,肠胃虚损。

则动作侵食于阴,轻者或痒或痛,重者生疮也。

《葛氏方》治妇人阴中疮方∶末硫黄,敷疮上。

又方∶烧杏仁,捣以涂之。

又方∶末雄黄熬二分,矾石二分,麝香半分,和末敷之。(《延龄图》同之。)《极要方》治妇人阴疮方∶取桃叶,捣绞取汁,洗之,日三。

《录验方》治妇人阴疮方∶蛇床子(一升,熬)大黄(二分)胡粉(半两)下筛作散,先以温汤洗,以粉之。

《僧深方》女子阴中疮方∶裹矾石末,如枣核,纳阴中。

《范汪方》妇人阴疮方∶地榆(二分)甘草(一分)水煮,适寒温,洗之良《千金方》妇人阴疮方∶野狼牙两把切,以水五升,煮取一升,湿洗之,日五。(今按∶《广济方》取汁和苦酒煎,涂,洗。)又云∶治男女阴蚀略尽方∶虾蟆兔屎二味,分等,捣筛,以敷疮上。(《集验方》同之。)《刘涓子方》治妇人阴蚀当归汤方∶当归(二两)甘草(一两)芎(一两)夕药(一两)地榆(三两)上五物,以水五升,煮取三升,洗之,日三夜一。

《录验方》治妇人男子阴蚀及脓血不禁,男子茎尽入腹中,众医所不能治。大黄汤方∶大黄(二两半)黄芩(二两)黄柏(二两)半夏(二两)细辛(二两)生地黄(二两)虎掌(一两半)草(一两半)凡八物,以新汲井水一斗,煮取三升,洗疮,若阴里病,取练沽汤中着阴道中,时复易,半日,久佳。恶汁尽当止,止浓剧当如针孔,勿怪也。
治妇人阴中息肉第十一

《病源论》云∶阴内息肉由胞络虚损,冷热不调,风邪客之;邪气乘阴,搏于血气,变生息肉也。其状如鼠乳。

《葛氏方》妇人阴中息肉突出者方∶以苦酒三升渍乌喙五枚,三日,以洗之,日夜三四过之。(《延龄图》同之。)又方∶取猪肝炙热,纳阴中,即有虫出着肝。
治妇人阴冷方第十二

《病源论》云∶胞络劳伤,子脏虚损风冷客之,冷乘于阴,故令冷也。

《千金方》治阴冷令热方∶纳食茱萸牛胆中令满,阴干百日,每取二七枚绵裹之,齿嚼令碎,纳阴中,良久热如火。

又妇人阴冷痛方∶灸归来三十壮,三报,在侠玉泉五寸是也。

《延龄图》云∶疗妇人阴冷方∶石硫黄(三分)蒲黄(二分)上二味,捣筛为末,三指撮(粗括反)纳一升汤中,洗玉门。(当日急热。)
治妇人阴臭第十三

《病源验》云∶阴臭由子脏有寒,寒气搏于津液,蕴积,气冲于阴,故变臭也。

《延龄图》云∶疗妇人阴臭方∶槲皮(切,一升)甘草(二两)当归(三两)上,以水一斗,煮取三升,去滓,洗玉门内。日二度洗,如冷,加蛇床子并根茎二分。

治女阴寒臭方∶白蔹(一分)桂心(五分)苦参(一分)甘草(一分)附子(一分)五物,咀,水煮,洗之愈。
治妇人阴脱方第十四

《病源论》云∶胞络伤损,子脏虚冷,下脱,亦因产用力,阴下脱也。

《千金方》治下挺出方∶蜀椒(二分)乌头(二分)白芨(二分)三味,筛,以方寸匕绵裹,纳阴中,入三寸,腹中热,易之。

又治阴脱硫黄散方∶硫黄(二两)乌贼骨(二分)五味子(三铢)三味,冶,下筛,以粉其上,良,日再三。

《经心方》阴脱下方∶矾石(鸡子大二枚)盐(弹丸大一枚)二味,以水三升,煮,洗之,自愈。

《葛氏方》治妇人阴脱出外方∶水煮生铁,令浓,以洗之。矾石亦良。(《僧深方》同之。)又方∶烧蛴螬末,以猪膏和,敷上,蒲黄粉之。

《极要方》阴脱方∶取蛇床子熬,布裹熨之。

《僧深方》治妇人子脏挺出蛇床洗方∶蛇床子(一升)酢梅(二七枚)二物,水五升,煮取二升半,洗之,日十过。

《集验方》治妇人阴中痒脱下方∶取车膏,敷之,即瘥。

《经心方》治妇人阴挺出方∶灸脐中二壮,愈。
治妇人阴大方第十五

《延龄图》云∶疗妇人阴宽冷,令急小,交接而快方∶石硫黄(二分)青木香(二分)山茱萸(二分)蛇床子(二分)上四物,捣筛,为末,临交接,纳玉门中少许,不得过多,恐撮孔合。

又云∶治妇人阴令成童女法∶蛇床子(三分)远志(三分)石胆(三分)山茱萸(三分)青木香(三分)细辛(半两)桂心(二分)上七味,捣筛为散,置狗胆中,悬于屋内。阴干六十日,药成,捣为末,可丸如枣核大,着妇人阴中,急小而热,不过三日。

又方∶松上女萝(一分)石〔(音预)一分〕石硫黄(一分)上三味,分等,捣筛为末,纳阴中,当日而急小,甚妙。

又方∶取石硫黄末,二指撮,纳一升汤水中,以洗,阴急小,如十二三女。

《千金方》治阴宽大令迮小方∶菟丝子(二分)干漆(一分)鼠头骨(二枚)雌鸡肝(三枚,百日阴干。)四味,丸如小豆,初月七日,合时一丸着阴头,徐徐纳之,三日知,十日小,五十日如十三女。

《录验方》治女急如童方∶食茱萸(三两)特牛胆(一枚)石盐(一两)捣茱萸下筛,纳牛胆中,又纳石盐着胆中,阴干百日,戏时取如鸡子黄末,着女阴中,即成童女也。
治妇人小户嫁痛方第十六

《集验方》治童女始交接,阳道违理,及为他物所伤。血流离不止方∶取釜底黑,断葫摩,以涂之。

又方∶烧发,并青布末,为粉,粉之,立愈。(《葛氏方》麻油和涂。)《千金方》治小户嫁痛方∶乌贼鱼骨二枚,烧为屑,酒服方寸匕,日三。

又方∶牛膝五两。一味,以酒五升,煮再沸,去滓,分三服。

《玉房秘诀》云∶妇人初交,伤痛,积日不歇方∶甘草(二分)夕药(二分)生姜(三分)桂心(一分)水二升,煮三沸,一服。
治妇人阴丈夫伤方第十七

《集验方》治女子伤于丈夫,四体沉重,虚吸头痛方∶生地黄(八两)夕药(五两)香豉(一升)葱白(切,一升)生姜(四两)甘草(二两,炙,切)以水七升,煮取三升,分三服,不瘥,重作。

《千金方》治合阴阳辄痛不可忍方∶黄连(六分)牛膝(四分)甘草(四分)三味,水四升,煮二升,洗之,日四。

又女人交接辄血出方∶桂心(二分)伏龙肝(三分)二味,酒服方寸匕,日三。(《刘涓子方》同之。)《玉房秘诀》云∶女人伤于夫,阴阳过。患阴肿疼痛欲呕方∶桑根白皮(切,半升)干姜(一两)桂心(一两)枣(三十枚)以酒一斗,煮三沸,服一升,勿令汗出当风,亦可用水煮。
治妇人脱肛方第十八

《病源论》云∶肛门,大肠候也。大肠虚冷,其气下冲者,肛门反出。亦有因产用力怒体,气冲其肛,亦令反出也。

《千金方》治脱肛若阴下脱方∶以铁精敷上,灸布令暖,以熨肛上,渐推纳之。

《集验方》治妇人脱肛若阴下脱方∶蛇床子布裹,灸熨之,亦治产后阴中痛。
治妇人月水不调方第十九

《病源论》云∶冲任之二经,上为乳汁,下为月水。若冷热调和,则血以时而下。寒温乖适,则月水乍多乍少,不调也。

《千金方》治月水不调,或在月前,或在月后,或多,或少。乍去乍来方∶生地黄三斤,酒煮,取二升服之。

又方∶地黄酒及大豆酒亦佳。

《极要方》治月水不调方∶以酒服桂末方寸匕,日二。《玄感敷尸方》同之。

《新录方》治虚惫(蒲释反,病也),月经一月再至方∶夕药三十二枚(重一斤,)酒一斗,渍夕药,令释濡出,曝之,干者复纳酒中,复曝之,如是令酒尽,燥,捣筛,服方寸匕,日三。
治妇人月水不通方第二十

《病源论》云∶妇人月水不通者,由劳损血气,致令体虚,受风冷也。

《千金方》治月水不通方∶葶苈子一升,捣,蜜和,如弹丸三枚,绵裹,纳阴中,入三寸,一宿易,有汁出止。

《新录方》治月水不通方∶麻子,捣绞取汁服,日三。

又方∶芎末,以酒服方寸匕。

又方∶桂心一尺,末,以酒服,日三。

又方∶当归,末,酒服方寸匕。

又方∶小豆一升,苦酒一斗,煮取三升,服任意多少,立下。

《极要方》疗妇人月水不利,血瘀不通,或一月或一岁,令人无子,腹坚如石,亦如妊娠之状方∶大黄(四两)夕药(二两)土瓜根(一两)上,为散,酒服方寸匕,日三,血下,痛即愈。

《葛氏方》治妇人月水不利,结积无子方∶大黄桃仁桂心(各三两)捣末,未食,服方寸匕,日三。

又云∶或至两三月、半年、一年不通者∶桃仁(二升)麻子仁(二升)合捣,酒一斗,渍一宿,服一升,日三夜一。
治妇人月水不断方第二十一

《病源论》云∶冲任之气虚损,故不能制其经血,故令月水不改。

《僧深方》治妇人月水不止方∶黄连,冶,下筛,以三指撮,酒和服,不过再三。

又方∶服淳酢一坏,不瘥,更服。

《撰集要方》治月水不止方∶服蒲黄良。

《千金方》月水不断方∶灸内踝下白肉际青脉。
治妇人月水腹痛方第二十二

《病源论》云∶月水来腹痛者,由劳损血气,体虚受风冷,故令痛也。

《耆婆方》治妇人月节来腹痛血气方∶防风(二两)生姜(六两)浓朴(三两,炙)甘草(二两)术(二两)枳实(二两,炙)桔梗(一两)七味,切,以水六升,煮取一升半,去滓,分为三服。

《僧深方》治月经至绞痛欲死茯苓汤方∶茯苓(三两)甘草(二两)夕药(二两)桂心(二两)凡四物,切,以水七升,煮取二升半,分三服。

《百病针灸》治月水来腹痛方∶灸中极穴,在脐下四寸。

《广济方》治月水来腹痛方∶当归甘草(各八两)夕药茯苓桂心(各十二分)以水六升,煮取二升,绞去滓,分温三服,服别相去,如人行六七里,忌生冷、海藻。
治妇人崩中漏下方第二十三

《病源论》云∶崩中之病,是复损冲脉,任脉。冲任之脉,起于胞内,为经脉之海。劳伤过度,冲任气虚,不统制其经,故血忽然崩下,谓之崩中。而内有瘀血,故时崩时止,淋沥不断,名曰崩中漏下也。

又云∶漏下不止,致损五脏,五脏之色,随脏不同,因虚其五色与血而俱下。其状,白者如涕,赤者如红汁,黄者如烂瓜汁,青得如蓝色,黑者如血也。

《短剧方》治妇人崩中,昼夜十数行,医所不能治方∶芎(八两)上一物,以酒五升,煮取三升,分三服,不耐酒者,随多少服之。(《千金方》同之。)又治妇人五崩下赤白青黄黑大枣汤方∶大枣(百枚)黄(三两)胶(八两)甘草(一尺)凡四物,以水一斗,煮取三升半,内胶令烊,分三服。今按∶《本草》云∶甘草一尺者,重二两为正。

又治漏下神方∶取槐耳,烧,捣,下筛,酒服方寸匕,日三,立愈。(今按∶《千金方》∶烧子酒服之。)又治妇人漏下病不断,积年困笃方∶取鹊重巢柴,合烧末,服方寸匕,日三。鹊重巢者,去年在巢中产,今岁更在其上复作巢是也。(《集验方》同之。)又治崩中去血方∶舂生蓟根汁一升,温顿服之,亦可以酒煮,随意服之。

又方∶舂生地黄汁一升,顿温服之,即止。(已以《千金方》同之。)《医门方》治久崩中昼夜不止,医不能疗方∶芎(八分)生地黄汁(一升)凡以酒五升,煮取二升,去滓,下地黄汁煎一沸,分三服,相去八九里。不耐酒者,随多少,数数服即止,但此二味可单用服之。

《千金方》治妇人漏赤不止,昼夜上气虚竭方∶龟甲(炙)牡蛎二味,分等,为散,酒服方寸匕,日三。

又方∶烧乱发,服方寸匕,日三。

又治五色下方∶大豆紫汤,日三服,佳。

又方∶煮甑汁服之。

又方∶服蒲。

又灸崩中方∶灸小腹横纹,当脐孔直下,一百壮。

又方∶灸关元两旁,相去三寸。

《葛氏方》治妇人崩中漏下及月,去青黄赤白使无子方∶鹿茸(二两)当归(二两)蒲黄(二两)捣筛,酒服五分匕,日三,加至方寸匕。(《录验方》同之。)又方∶赤石脂蜜丸,服如梧子三丸,日三。

又方∶露蜂房烧末,三指撮,酒服之,良。

《广利方》治崩中漏下血方∶凌霄花末,温酒服方寸匕,日三,即止。

《僧深方》治崩中方∶桑耳干姜(分等)下筛,酒服方寸匕,日四五。

又方∶白茅根二十斤,小蓟根十斤,捣,绞取汁,煮取五升,服一升,日三四。

《经心方》治长血芎丸方∶鹿茸(二两)当归(二两)蒲黄(二两)阿胶(二两)芎(二两)白术(三两)干地黄(三两)凡七物,捣,筛,和丸如大豆,服十丸,日三。

又方∶生地黄,莲根分等,捣,绞取汁,煎,服任意。

龙骨丸治长血方∶龙骨阿胶(炙)赤石脂牡蛎干地黄当归甘草(炙,各二两)蒲黄(三两)凡八物,捣,筛,丸如梧子,服十五丸,日三。

《龙门方》疗妇人下方∶人参(一两)茯苓(二两)牡蛎(五两,别研)余者末,和,饮服酒亦得,日再,以瘥为度。

又方∶灸脐左右各一寸五分,各三百壮。

《集验方》治妇人漏下不止散方∶鹿茸(三两)当归(二两)蒲黄(一两)阿胶(三两,炙)乌贼骨(二两,去甲)下筛,为散,酒服方寸匕,日三夜再。(《千金方》同之。)《苏敬本草注》云∶刮牛黄服之。

《崔禹锡食经》云∶海髑髅子刮服之。
治妇人下三十六疾方第二十四

《病源论》云∶诸方说,三十六疾者,是十二症、九痛、七害、五伤,三固,谓之三十六疾也。十二者,所下之物,一如膏白,二如清血,三如紫汁,四如赤肉,五如脓痂,六如豆汁,七如葵羹,八如凝血,九如清血,血似水,十如米汁,十一如月浣,十二经废不应期也。九痛者,一阴中痛伤,二阴中淋痛,三小便来即痛,四寒冷痛,五月水来腹痛,月水止则不止,六气满崩痛,七汗出,阴中如虫啮痛,八胁下引痛,九腰痛也。七害者,一害食,二害气,三害冷,四害劳,五害房,六害妊,七害睡也。五伤者,一穷孔痛,二中寒热痛,三少腹急痛,四脏不仁,五子门不正,引背痛。三固者,一月水闭塞不通,其余二固者,谓子阙不载。(今按∶《千金方》云∶三固者,一羸瘦不生肌肤,二绝产乳,三月水闭塞也。

七害者,一穷孔痛不利,二中寒热痛,三少腹急坚痛,四脏不仁,五子门不端,六浣乍多乍少,七喜吐阳精也。五邪伤者,一两胁支满痛,二心痛引胁,三气结不通,四邪恶泄利,五前后固塞也云云。)《千金方》白恶丸主妇人三十六疾病各异同。疗之方∶白恶(三分)龙骨(三分)夕药(二分)黄连(二分)当归(二分)茯苓(二分)黄芩(二分)瞿麦(二分)白蔹(二分)石苇(二分)甘草(二分)牡蛎(二分)细辛(二分)附子(二分)禹余粮(二分)白石脂(二分)人参(二分)乌贼骨(二分)甘皮(二分)本(二分)大黄(二分)二十一味,下筛,蜜丸如梧子,未食服十丸,日二,不知,稍增。服药,二十日知,三十日百病悉愈。(今按∶《短剧方》有桂心四分、白芷四分,无甘皮、本。)
治妇人八瘕方第二十五

《病源论》云∶八瘕病者,皆胞胎生产,月水往来,血脉精气不调之所生也。其八瘕者,黄瘕、青瘕、燥瘕、血瘕、脂瘕、狐瘕、蛇瘕、鳖瘕也。

《千金方》妇人血瘕痛方∶干姜(一两)乌贼鱼骨(一两)二味,冶,筛,酒服二方寸匕,日三。

《录验方》治妇人脐下结物,大如杵升,月水不通,发热往来,下利羸瘦。此为气瘕也,故生肉瘕,不可治;未生肉瘕,可治。干漆丸方∶生地黄(三斤,一方二十斤,取汁)干漆(一斤,熬,捣,筛)凡二物,地黄捣,绞取汁;漆冶,下筛,纳地黄汁中,微火煎令可丸,药成,酒服如梧子十五丸,当以食后服之。(《葛氏方》同之,《集验方》服三丸。)《玄感敷尸方》治血症如四五月身大方∶浸前件二味,五日以后取服,一服一盏,温服之良。
治妇人遗尿方第二十六

《病源论》云∶妇人遗尿候,肾虚冷,冷气入胞,胞虚冷,不能制小便,遗尿也。

《录验方》治妇人遗尿方∶矾石(三两,烧令沸,汗尽)牡蛎肉(三两)下筛,为散,酒服方寸匕,日三。(《集验方》各二两。)
治妇人尿血方第二十七

《病源论》云∶血性得寒则凝涩,得热则流散,若劳伤经络,其血虚。热渗入于胞,故尿血。

《极要方》妇人无故尿血方∶龙骨五两,为散,空腹,服方寸匕,日三。

《千金方》治妇人无故尿血方∶取其夫爪甲,烧作末,酒服之。

又方∶取船故竹茹,曝干,捣末为散,酒服方寸匕,日三。

《葛氏方》治妇人尿血方∶车前草一斤,水一斗,煮取四升,分四服。

又方∶葵根、茎、子,无在,取一升,水四升,煮取一升,纳书中白鱼虫十枚,研,服一合。
治妇人瘦弱方第二十八

《玄感敷尸方》主妇人年老体渐瘦弱,头面风肿,骨节烦疼,冷,口干状如骨蒸者,是牛膝酒方∶牛膝(一斤)生地黄(切,三升)牛蒡(切,曝干,一斤)生姜(合皮切,一升)凡四味,切,于绢袋盛之,以清酒一大升,浸七日,温服一盏,日三。
治妇人欲男方第二十九

《太清经》云∶皇帝问素女,对曰∶女人年二十八九,若二十三四,阴气盛,欲得男子;不能自禁,食饮无味,百脉动体,候精脉实,汁出污衣裳,女人阴中有虫如马尾,长三分,赤头者闷(莫困反,烦也。),黑头者沫(莫葛反)。

治之方∶用面作玉茎,长短,大小随意,以酱清及二辨,绵裹之,纳阴中,虫即着来出。

出复纳,如得大夫。其虫多者三十,少者二十。
治妇人鬼交方第三十

《病源论》云∶妇人梦与鬼交通者,亦由腑脏气弱,神守虚衰,致鬼灵因梦而交通也。

《玉房秘诀》云∶采女曰∶何以有鬼交之病。彭祖曰∶由于阴阳不交,情欲深重,即鬼魅假像与之交通,与之交通之道,其有胜于人,久则迷惑,讳而隐之,不肯告人,自为佳,故至独死而莫之知也。

治之法∶但令女与男交而勿泻精,昼夜勿息困者,不过七日,必愈;若身体疲劳,不能独御者,但深按勿动亦善。

又方∶当以石硫黄数两,烧以熏妇人阴下体体,并服鹿角末方寸匕,即愈矣,当息,鬼涕泣而去。

一方云∶服鹿角方寸匕,日三,以瘥为度。

《短剧方》别离散治男女风邪,男梦见女,女梦见男,悲愁忧恚,怒喜无常,或半年,或数月,日复发者方∶杨上寄生(三两)术(三两)桂肉(一两一方三两)茵芋(一两)天雄(一两,炮)KT根(一两)蓟根菖蒲(一两)细辛(一两)附子(一两,炮)干姜(一两)凡十物,合捣,下筛,酒服半方寸匕,日三。合药勿令妇人,鸡犬见之。又无令见病者,病者家人见合药,知药者,令邪气不去,禁之,为验。

《千金方》治妇人忽与鬼交通方∶松脂(二两)雄黄(一两)二味,先烊松脂,乃纳雄黄末,以虎爪绞令相得。药成取如鸡子中黄,夜卧以着熏笼中,烧令病患自升其上,以被自覆,唯出头,勿令过热,及令气得泄也。

《玄感敷尸方》主妇人患骨蒸者,多梦与鬼夫交接。为之方∶雄黄(一两,破)虎爪(一枚,末)沉香(一两,末)青木香(一两,末)松脂(二两,破)凡五味,合,和以蜜丸,丸如弹丸,纳火笼中,以熏阴,夜别一度,大良。
治妇人令断生产方第三十一

《短剧方》云∶妇人欲断生方∶故布方圆一尺,烧作屑,以酒饮之,终身不生产。

《千金方》云∶断产方∶灸右踝上一寸三壮,即断。

又方∶蚕子故布方尺,烧末,酒服之。
卷第二十二
妊妇脉图月禁法第一

《产经》云∶黄帝问曰∶人生何如以成。歧伯对曰∶人之始生,生于冥冥,乃始为形,形容无有扰,乃为始收,妊身一月曰胚(芳杯反),又曰胞,二月曰胎,三月曰血脉,四月曰具骨,五月曰动,六月曰形成,七月曰毛发生,八月曰瞳子明,九月曰谷入胃,十月曰儿出生也。(今按∶《太素经》云∶一月膏,二月脉,三月胞,四月胎,五月筋,六月骨,七月成,八月动,九月躁,十月生。)夫妇人妊身,十二经脉主胎,养胎当月不可针灸其脉也,不禁皆为伤胎,复贼母也,不可不慎,宜依月图而避之。

怀身一月,名曰始形,饮食必精熟暖美,无御大(丈)夫,无食辛腥,是谓始载负也。

(《病源论》云∶宜食大麦。)一月足厥阴脉养,不可针灸其经也,厥阴者是肝,肝主筋,亦不宜为力事,寝必安静,无令恐畏。

上肝脉穴,自大敦上至阴廉,各十二穴。又募二穴,名期门;又输二穴,在脊第九椎节下两旁,各一寸半。上件诸孔,并不可针灸,犯之致危。

\r足厥阴肝脉图\pyxfc1.bmp\r怀身二月,名曰始膏,无食辛臊,居必静处,男子勿劳,百节骨间皆病,是谓始藏也。

(《病源论》云∶勿食辛腥之物,二月之时,儿精成也。)\r足少阳胆脉图\pyxfc3.bmp\r二月足少阳脉养,不可针灸其经也。少阳者内属于胆,当护慎,勿惊之。

上胆脉穴自窍阴上至环铫,各十三穴,又募二穴,名日月,在期门下五分。又输二穴,在背第十椎节下两旁,各一寸半。上件诸穴,并不可犯之。

怀身三月,名曰始胎。当此之时未有定仪,见物而化。是故应见王公、后妃、公主、好人,不欲见偻者、侏儒、丑恶、瘦人、猿猴。无食苗姜兔肉。

思欲食果瓜,激味酸菹瓜,无食辛而恶臭,是谓外像而内及故也。

三月手心主脉养,不可针灸其经也。心主者,内属于心,心无悲哀,无思虑惊动之。

\r手心主脉图\pyxfc2.bmp\r上心胞脉穴自中冲上至天府,各八穴。又募一穴,名曰巨阙,在心鸠尾下一寸五分。又输二穴,在背第五椎节下两旁各一寸半。上件诸穴,并不可犯也。

怀身四月,始受水精,以盛血脉。其食稻粳,其羹鱼雁,是谓盛血气以通耳目,而行经络也。

四月手少阳脉养,不可针灸其经也。手少阳内属上焦,静安形体,和顺心志,节饮食之。

\r手少阳三焦脉图\pyxfc5.bmp\r上三焦脉穴自关冲上至消泺,各十二穴。又募一穴,在当脐下二寸,名为石门。又背输二穴,在脊第十三椎节下两旁各一寸半。上件诸穴,并不可犯之。

怀身五月,始受火精,以盛血气,晏起沐浴浣(胡管反)衣,身居堂,必浓其裳。朝吸天光,以避寒殃(于良反)。其食稻麦,其羹牛羊和茱萸。调以五味,是谓养气,以定五脏者也。

五月足太阴脉养,不可针灸其经也。太阴者,内属于脾。无大饥,无甚饱,无食干燥。

无自灸热大劳倦之。

足太阴脾脉图上脾脉穴自隐白上至箕门,各十二(三)穴。又募二穴,名章门,在季肋端,侧卧取之。

又输二穴,在脊第十一椎节下两旁各一寸半。上件诸穴,并不可犯之。

怀身六月,始受金精,以成筋骨。劳身无处,出游于野,数观走犬,走马,宜食鸷鸟猛兽,是谓变腠理膂细筋,以养其爪,以坚背膂也。

六月足阳明脉养,不可针灸其经也。阳明内属于脾,调和五味,食甘,甘和,无大饱。

\r足阳明胃脉图\pyxfc6.bmp\r上胃脉自厉兑上至髀关,各十六穴。又募一穴,名中管,在从心蔽骨下以绳量至脐止,即以绳中折之。又输二穴,在脊第十二椎节下两旁,各一寸半。上件诸穴,并不可犯之。

怀身七月,始受本精,以成骨髓。劳躬摇肢,无使身安,动作屈伸,自比于猿。居必燥之。饮食避寒,必食稻粳,肌肉,以密腠理,是谓养骨而坚齿也。

\r手太阴肺脉图\pyxfc8.bmp\r七月手太阴脉养,不可针灸其经也。太阴者,内属于肺。无大言,无号哭,无薄衣,无洗浴,无寒饮之。

右肺脉穴自少商上至天府,各九穴。又募二穴,名中府,在两乳上三肋间陷者中。又输二穴,在背第三椎节下两旁,各一寸半。上件诸穴,并不可犯之。

怀身八月,始受土精,以成肤革。和心静息,无使气控(极),是谓(密)腠理而光泽颜色也。

八月手阳明脉养,不可针灸其经也。阳明者,内属于大肠。无食燥物,无忍大起。

\r手阳明大肠脉图\pyxfc7.bmp\r上大肠脉穴自商阳上至臂,各十四穴。又募二穴,在脐两旁,各二寸半,右名天枢,左名谷门。又输二穴,在脊第十六椎节下两旁,各一寸半。上件诸穴,并不可犯之。

怀身九月,始受石精,以成皮毛,六腑百节莫不毕备;饮醴食甘,缓带自持而待之,是谓养毛发多才力也。

九月足少阴脉养,不可针灸其经也。少阴内属于胃,无处湿冷,无着炙衣。

\r足少阴肾脉图\pyxfc9.bmp\r上肾脉穴自涌泉上至阴谷,各十七穴。又募二穴,在腰目中季肋,本侠脊腴肉前宛宛中,名京门。又输二穴,在脊第十四椎下两旁,各一寸半。上件诸穴,并不可犯之。

怀身十月,俱已成子也。时顺天生,吸地之气,得天之灵。而临生时乃能啼,声遂天气,是始生也。

十月足太阳脉养,不可针灸其经也。太阳内属于膀胱,无处湿地,无食大热物。

\r足太阳膀胱脉图\pyxfc10.bmp\r上膀胱脉穴自至阴上至扶承,各十六穴。又募一穴,在脐下直四寸,名中极,又输二穴,在脊第十九椎节下两旁,各一寸半。上件诸穴,并不可犯之。
妊妇修身法第二

《产经》云∶凡妊身之时,端心正坐。清虚如一,坐必端席,立不斜住(经),行必中道,卧无横变,举目不视邪色,起耳不听邪声,口不妄言,无喜怒忧恚,思虑和顺,猝生圣子,产无横难也。而诸生子有痴、疵、丑恶者,其名皆在其母,岂不可不审详哉。

又云∶文王初妊之时,其母正坐,不听邪言恶语,口不妄语,正行端坐,是故生圣子,诸贤母宜可慎之。

又云∶妊身三月,未有定仪,见物而为化,是故应见王公、后妃、公主、好人。不欲见偻者、儒侏、丑恶、瘁(秦醇反,病也)人、猿猴。其欲生男者,操弓矢射雄雉,乘牡马走田野,观虎豹及走马。其欲生女者,着簪珥施环。欲令子美好者,数视白玉美珠,观孔雀,食鲤鱼。欲令子多智有力者,当食牛心,御大麦。欲令子贤良者,坐无邪席,立无偏行,是谓以外像而内化者也。

《千金方》云∶凡受胎三月,逐物变化,禀质未定。故妊身三月,欲得见犀象猛兽,珠玉宝物,欲得见贤人君子,盛德大师,礼乐、钟鼓、俎豆、军旅阵设,焚烧名香,口诵诗书,居处简静,割不正不食,席不正不坐,弹琴瑟,调心神,和情性,节嗜欲,生子皆长寿无疾。

《养生要集》云∶妇人妊身大小,行勿至非常之地,逆产杀人。

又云∶妇孕见麋而生儿,其四目。

又云∶孕妇三月,不得南向洗浴,胎不安。

又云∶妇孕三月,不得南向小便,令儿喑哑。

又云∶妇孕三月,不得两镜相照,令儿倒产。

《膳夫经》云∶妊身勿北向,向其生年上大小便,使母难。
妊妇禁食法第三

《养生要集》云∶妇人妊身,不得食六畜肉,令儿不聪明。(一云∶坏胎。)又云∶勿食猪肝,令胎不生。

又云∶勿食兔肉,令子唇缺,亦不须见之。

又云∶勿食鸡子干鳝鱼,使子多疮。

又云∶勿不得食鱼头,胎损。

又云∶食鲤脍,令儿多疮。

又云∶勿食生姜,令子盈指。

又云∶勿食干姜、桂、甘草,令胎消,胎不安。

又云∶勿饮冰浆,令胎不生。

又云∶勿食杏仁及热饴,破损伤子。

又云∶勿以炙雀并大豆酱食,令胞漏,使儿多。

又云∶勿饮酒,多食雀肉,使子心淫精乱。

又云∶勿食雀肉,令儿多所欲。

又云∶勿食雀肉并雀脑,令人雀盲。

又云∶勿食雀并梨子,令子短舌。

又云∶麋并梅、李实,食之,使人清盲。

《崔禹锡食经》云∶妊身不可食鸠,其子门肥充,病于产难故也。

《朱思简食经》云∶勿食猪肉,令子喑哑无声。

又云∶饮酒醉,令儿癫痫。

《本草食禁》云∶妊身食鸡肉并糯米,使子腹中多虫。

《产经》云∶女人胎妊时,多食咸,胎闭塞。

妊身多食苦,胎乃动。

妊身多食甘,胎骨不相着。

妊身多食酸,胎肌肉不成。

妊身多食辛,胎精魂不守。

今按∶妊妇不可服药八十二种,其名目在《产经》。
治妊妇恶阻(侧吕反,病)方第四

《病源论》云∶恶阻病者,心中愦愦,头重眼眩,四肢沉重;懈惰不欲执作,恶闻食气,欲啖酸果实,多卧少起,世云恶食,又云恶字是也。乃至三四月日以上,大剧者,吐逆不能自胜举也。此由妇人本虚羸,血气不足,肾气又弱,兼当风取冷太过,心下有痰水,挟之而有娠也,经血既闭,水渍(溃)于脏,脏气不宣,故心烦愤,气逆则呕吐也。血脉不通,经络痞涩,则四肢沉重,挟风则头痛眩。故欲有胎,而病恶阻,所谓欲有胎者,其人月水尚来。

而颜色肌肤如常,但苦沉重愦闷,不用食饮,不知患所在,脉理顺时平和,则是欲有胎也。

如此经二月,日后便觉不通,则结胎也。

《产经》云∶半夏茯苓汤治妊身阻病,心中愤闷。空烦,吐逆,恶闻食气,头重,四肢百节疼烦沉重,多卧少起,恶寒汗出。疲极黄瘦方∶半夏(五两)生姜(五两)茯苓(三两)旋复花(一两)橘皮(二两)细辛(二两)芎人参(二两)夕药(二两)干地黄(三两)泽泻(二两)甘草(二两)凡十二物,以水一斗,煮取三升,分三服。若病阻积日月不得治,及服药冷热失候,病变客热烦渴。口生疮者,除橘皮、细辛,用前胡、知母,各二两。若变冷下者,除干地黄,用桂肉二两。若食少,胃中虚生热,大行闭塞,小行赤少者,宜加大黄三两,除地黄,加黄芩一两,余药依方服一剂,得下后消息者气力冷热更增,损方调定,即服一剂,阳便急将茯苓丸,令得能食,便强健也。(《短剧方》同之。)《短剧方》云∶茯苓丸治妊身阻病,患心中烦闷,头重眩目,憎闻饭气,便呕逆吐闷颠倒,四肢痿热。不自胜持,服之即效。要先服半夏茯苓汤两剂,后将茯苓丸也。

茯苓(一两)人参(二两)桂肉(二两)干姜(二两)半夏(二两)橘皮(一两)白术(二两)枳实(二两)葛根屑(一两)甘草(二两)凡十物,捣筛,蜜和丸如梧子,饮服二十丸,渐至三十丸,日三。《产经》同之。

《集验方》云∶治妊身二三月,恶阻呕吐不下食方∶青竹茹(三两)生姜(四两)半夏(五两)茯苓(四两)橘皮(三两)凡五物,切,以水六升,煮取二升半,分三服。

又云∶治妊身呕吐不下食,橘皮汤方∶橘皮(三两)竹茹(三两)人参(三两)术(三两)生姜(四两)浓朴(二两)凡六物,切,以水七升,煮取二升半,分三服。

《僧深方》云∶治妇人妊身恶阻,酢,心胸中冷,腹痛不能饮食,辄吐青黄汁方。用∶人参干姜半夏凡三物,分等,冶下,以地黄汁和丸如梧子,一服三丸,日三。今按∶《极要方》云∶各八分,稍加至十丸。《产经》云∶人参丸神良。

《医门方》云∶凡妊娠阻病恶食,以所思食任意食,必愈。(今检∶《延龄图》云∶令子聪明贤善端正也。)
治妊妇养胎方第五

《僧深方》云∶养胎易生丹参膏方∶丹参(四两)人参(二分,一方二两)当归(四分)芎(二两)蜀椒(二两)白术(二两)猪膏(一斤)凡六物,切,以真苦酒渍之,夏天二三日,于微火上煎,当着底绞之,手不得离,三上三下。药成,绞去滓,以温酒服如枣核,日三。稍增可加。若有伤,动见血,服如鸡子黄者,昼夜六七服之,神良。妊身七月便可服,至坐卧忽生不觉。又治生后余腹痛也。(今检∶《产经》云∶丹参一斤,当归四两,芎八两,白术四两,蜀椒四两,脂肪四斤。云云)《千金方》云∶安胎鱼法∶鲤鱼二斤,生者去鳞脏,粳米一斤,石相和作,少着盐,勿着豉酢,啖之,日别三过,食之满十个月。

又方∶取鲤鱼长一尺者,水自没纳盐煮饮之。

《本草拾遗》云∶鲤鱼肉主安胎动怀妊身肿煮为汤食。
治妊妇闷冒方第六

《短剧方》云∶妊身忽闷冒不识人,须臾醒,醒复发,亦仍不醒者,名为痉病。亦号子痫也,亦号冒也。治之方∶急作淡竹沥汁与之。无淡竹者,桂竹亦善,复宜服治痉冒葛根汤。

《集验方》云∶妊身恒苦烦闷者,此子烦也。治之方∶时时服竹沥,随多少,良。
治妊妇胎动不安方第七

《病源论》云∶胎动不安者,多因劳役,或触冒冷热,或饮食不适,或居处失宜,轻者转动不安,重者便致伤胎。

《医门方》云∶凡候胎动法∶母唇口青者,儿死母活;唇口中青沫出者,子母俱死;口赤舌青沫出者,母死儿活。

又云∶夫胎动不安方∶煮好银,取汁煮葱羹,服之佳。(今按∶《博济安众方》∶胎动欲堕,腹痛不可忍方∶苎根去皮,切一升,银五两,上以清酒一升,水一升,煎取一升,温分四服即止。)又云∶疗妊娠腹内冷,致胎动不安方∶干姜(三两)芎(四两)艾(二两)水六升,煮取二升半,分二服。

又方∶葱白(切,一升)当归(四两)清酒(五升)煮取二升半,分温二服,大效。

又云∶疗妊娠忽被惊愕,胎向下不安,少腹痛连腰。并下血方∶当归芎(各八分)阿胶(炙)人参(各六分)大枣(十二枚)艾叶(八分)茯苓(十分)水七升,煮取二升半,分三服,服之相去八九里。

《短剧方》云∶治妊身腹中冷,胎不安方∶甘草当归(各二两)干姜(三两)大枣(十二枚)凡四物,以水五升,煮取三升,分三服。

又云∶治母有劳热动胎,胎不安,去血。手足烦方∶生甘竹皮(二升)当归(二两)芎(一两)黄芩(半两)凡四物,以水一斗,煮竹皮取六升汁,去滓。纳煎取三升,分三服。

《葛氏方》云∶妊身猝胎动不安,或胎转抢心,或下血不止方∶葱白一把,以水煮令葱熟,饮其汁。(今按《本草》云∶某草一把者二两为正。)又方∶生鱼二斤秫米一斤,调作,顿食之。

《集验方》云∶治妊身胎动,昼夜叫呼。口禁唇寒及下利不息方∶已冶艾叶一,以好酒五升,煮取四升,去滓,更煎取一升,一服。口闭者,开口灌之,药下即安。(今检∶《僧深方》云∶艾及叶物一者,以二升,为正。)又云∶治妊身二三月至八九月,胎动不安。腰痛已有所见方∶艾叶(三两)阿胶(三两,炙)芎(三两)当归(三两)甘草(一两半,炙)切,以水八升,煮取三升,去泽(滓),纳胶,更上火,胶消分三服。

《产经》云∶治妊身七八月,腰腹痛,胎不安,汗出逆冷,饮食不下,气上烦满。四肢痹强当归汤方∶当归(三两)夕药(二两)干地黄(三两)生艾(一把)甘草(一两)胶(四两,炙)生姜(一两)橘皮(二分)上八物,切,以水一升(斗),煮得三升,去滓,纳胶,令烊,分四服之。

又云∶妊身临生月胎动不得生方∶桑上寄生(五分)甘草(二两)桂心(五分)茯苓(五分)上四物,以水七升,煮得二升,分三服。

《录验方》治胎不安生鲤鱼汤方∶生鲤鱼(一头,重五斤)干姜(二两)吴茱萸(一两)凡三物,切,以水一升(斗),儿煮鲤鱼五沸,出鱼,纳药,煎取三升,服一升,日三。
治妊妇数落胎方第八

《病源论》云∶阳施阴化,故得有胎。营卫和调,则经养周之,故胎得安而能成长。若血气虚损者,子脏为风冷所伤,不能养胎,所以数堕胎。妊娠而恒腰痛者,喜堕胎也。

《产经》云治数落胎方∶作大麦豉羹食之,即安胎。

又方∶取母衣带三寸烧末,酒服即安。

《录验方》云∶治妊身数落胎方∶以生鲤鱼二斤,粳米一升,作,少与盐啖之,日(月)三过,食至儿生。
治妊妇胎堕血不止方第九

《病源论》云∶妊身胎堕损经脉,故血不止也。泻血多者致死。

《葛氏方》云∶治胎堕血露不尽方∶艾叶(半斤)酒(四升)煮取一升,顿服之。

《极要方》云∶治胎堕血不止方∶丹参(十二两)酒(五升)煮取三升,分三服即止。

又方∶阿胶(五两,炙)干地黄(五两)以酒五升,煮取一升半,空腹分再服。(今按∶《博济安众方》∶药各二两,酒二升,煎取一升。)《僧深方》云∶生姜,切,五升,以水八升,煮取三升,分三服。
治妊妇堕胎腹痛方第十

《病源论》云∶此由堕胎之时,余血不尽,故令腹痛。

《葛氏方》云∶治堕胎后心腹致绞痛方∶豉(三升)生姜(五两)葱白(十四枚)酒(六升)煮取三升,分三服。

《千金方》云∶治落胎后腹痛方∶地黄汁(八合)酒(五合)合煮,分三服。

《僧深方》云∶治堕身血不尽去留苦烦满方∶香豉一升半,以水三升,煮三沸,滴取汁,纳成末鹿角一方寸匕,服须臾血下烦止。(今检∶《千金方》云麻角一两。)
治妊妇上迫心方第十一

《葛氏方》云∶治妊身胎上迫心方∶取弩弦急带之立愈。

又方∶生曲半斤,碎,水和绞取汁三升,分二服。

又方∶生艾捣,绞取汁三升,胶四两,蜜四两,合煎取一升五合,顿服之。
治妊妇漏胞方第十二

《病源论》云∶漏胞者,谓妊娠数月,经水时下也,亦名胞阻漏血尽则毙。

《医门方》云∶夫漏胞者,妊娠下血如故,血下不绝,胞干便死。宜急治方∶生地黄汁一升,酒五合,和煮一沸。分二服。(《广利方》同之。)又云∶疗妊娠下血如月水来,若胞干,非只杀胎。亦损其母方∶干姜干地黄(各五两)末,酒服一匙,日夜三四服,即止。今检∶《集验方》云∶干地黄四两,干姜二两。酒服方寸匕,日三。

《葛氏方》云∶妊身月水不止,名为漏胞。治之方∶阿胶(五两)干地黄(五两)酒(五升)煮取一升半,未食,温再服。

《千金方》云∶妊身血下不止方∶生艾叶一斤,酒五升,煮取二升,分二服。冬用茎。

又方∶烧秤锤令赤,纳酒中沸定,出锤,饮之。

《集验方》云∶治妊身血下不止。血尽子死方∶干地黄捣末,以三指撮酒服,不过再三服。

《产经》云∶治妊身血出不止方∶干地黄十两,以酒三升,煮得二升,分二服,良。

又方∶灸胞门七壮,关元左右各二寸是也。
治妊妇下黄汁方第十三

《子母秘录》云∶妊娠下黄汁如胶,及小豆汁方∶糯米(一升)黄(五两)上二物,切,以水七升,煮取三升,分四服。(今按∶《产经》∶捣地黄取汁,以酒合煎,顿服之。)
治妊妇顿仆举重去血方第十四

《录验方》云∶治妊身顿仆举重去血方∶取淡竹断头,烧中央,以器承取汁一升,饮之。

《僧深方》云∶治妊身由于顿仆及举重去血方∶捣黄连下筛,以酒服方寸匕,日三乃止。

又云∶取生青竹,薄刮取上青皮,以好酒一升和三合许,一服。
治妊妇猝走高堕方第十五

《产经》云∶治妊身妇人猝贲起,从高堕下暴大去血数斗。马通阳方∶马通汁(三合,绞取)干地黄(二两)当归(二两)阿胶(四两)艾叶(三两)上五物,切,以水五升,煮得二升半,去滓,纳胶,更上火令烊,分三服,大良。(马通是马屎。)
治妊妇为男所动欲死方第十六

\r图\pyxfc12.bmp\r《产经》云∶治妊身为夫所伤动欲死方∶取竹沥汁,与饮一升则愈,不瘥,后作。(《千金方》云∶立验。)《医门方》云∶若因房室下血,名曰伤胞。治之方∶干地黄十两,末,酒服方寸匕,日三夜一。若腹内冷,加干黄服之。
治妊妇胸烦吐食方第十七

《产经》云∶治妊身胸中烦热,呕吐血,不欲食,食辄吐出,用诸药无利。唯服牛乳则愈方∶牛乳微微煎,如酪煎法,适寒温,服之多少任意,初服少少,若减之良验。
治妊妇心痛方第十八

《病源论》云∶心痛者,多是因风邪痰饮,乘心之经故也。

《葛氏方》云∶刮取青竹皮,以水煮令浓,绞去滓,服三升。

《耆婆方》云∶高良姜三两,以水一升半,煮取半升,去滓,分三服。

《千金方》云∶烧来二七枚末,以尿服之,立验。

《僧深方》云∶吴茱萸五合,以酒煮三沸。分三服。
治妊妇心腹痛方第十九

《产经》云∶治妊身心腹刺痛方∶烧枣十四枚,冶末,以小便服之,立愈。(《短剧方》同之。)《短剧方》云∶治妊身心腹刺痛方∶盐烧令赤熟,三指撮,酒服之,立瘥。
治妊妇腹痛方第二十

《病源论》云∶妊身腹痛者,因风邪入于腑脏所成。

《葛氏方》云∶秤锤烧正赤,以着酒中,令三沸,出锤饮酒。

《集验方》云∶赤小豆,东向户中吞二七枚,良。

《产经》云∶葱白,当归,切,酒五升,煎取二升半,分再服。

《耆婆方》云∶熬盐令热,布裹与熨之,乃停。

《千金方》云∶生地黄三斤,捣绞取汁,酒一升,合煎减半,顿服。

又方∶烧车脂末,纳酒中服之。
治妊妇腰痛方第二十一

《病源论》云∶肾主腰脚,风冷乘之,故腰痛。

《短剧方》云∶治妊身腰痛如折方∶大豆三升,以酒三升,煎取二升,服之。

《葛氏方》云∶治妊身腰背痛如折方∶末鹿角,酒服方寸匕。

又方∶葱白煮汁,服之验。

又方∶胶、桂各一尺,捣,以酒三升,煮得一升,去滓,尽服。

《医门方》云∶疗妊娠猝腰背痛,反复不得,如折方∶鹿角一枚,五寸截之,烧令赤,纳酒二大升中浸之;冷,又烧之。如此数度,便空腹饮此酒,极佳。

《僧深方》云∶治妊身腰痛方∶熬盐令热,布裹与熨之。
治妊妇胀满方第二十二

《产经》云∶治妊身猝心腹拘急痛胀满,气从少腹起上冲。心烦起欲死,是水饮,食冷气所为。茯苓汤方∶茯苓(一两)当归(三两)甘草(二两,炙)黄芩(一两)术(三两)石膏(如鸡子一枚)杏仁(三十枚)夕药(二两)芒硝(一两)上九物,切,以水八升,煮取三升,纳芒硝,上火令烊之,服一升,当下水或吐便解。

又云∶治妊身腹痛,心胸胀满不调,安胎当归丸方∶干姜(一分)当归(二分)芎(二分)胶(四分)上四物,下筛,蜜丸如小豆,服五丸,日三。
治妊妇体肿方第二十三

《病源论》云∶此由脏腑之间有停水,而妊娠故也。

《录验方》云∶治妊身体肿方∶生鲤鱼一头,长二尺,完,用水二斗煮取五升,食鱼饮汁。

又方∶葵子一升,茯苓三两,下筛,服方寸匕,先食,日三,小便利即止。

《医门方》云∶治妊身四肢煎肿。皮肉拘急方∶常陆根切,一升,赤小豆三升,桑根白皮切,一升,水一斗二升,煮二味取七升,去滓,纳小豆煮令熟,然食豆渴饮汁,小便利即瘥。

《千金方》云∶治妊身手足皆肿挛急方∶赤小豆(五升)常陆根(一斤)滓柒(一斤)三味,水三升煮取一升,恒服。

《集验方》云∶小豆五升,好豉三升,以水一斗煮取三升,分上服。

《葛氏方》治妇人妊身之肿方∶大豆二升,酒三升,煮取二升,顿服之。
治妊妇下利方第二十四

《产经》云∶治妊身暴下不止腹痛石榴皮汤方∶安石榴皮(二两)当归(三两)阿胶(二两,炙)熟艾(如鸡子大二枚)上四物,以水九升,煮取二升,分三服。

又云∶治妊身下利赤白种种带下黄连丸方∶黄连(一两)甘草(一两)干姜(二两)吴茱萸(一两)乌梅(三十枚)熟艾(一两)黄柏(一两)上七物,下筛,蜜和丸如梅子,一服五丸,日三。

《医门方》疗妊娠注(治)下利下(不)止或水或脓血方∶熟艾(二两)石榴皮阿胶(炙,各三两)水四升,煮取一升半,分三服。(《集验方》同之。)又云∶疗妊娠利白脓,腹内冷方∶干姜(四两)赤石脂(二两)粳米(一升,熬令黄)水七升,煮取二升半,分三服。

《千金方》云∶白杨皮一斤,水一斗,煮取二升,分三服。
治妊妇小便数方第二十五

《病源论》云∶肾气通于阴,肾虚而生热,热则小便涩数。

《录验方》云∶治妊身猝暴小便数不能自禁止方∶桑螵蛸(十四枚)凡一物,作散,温酒服,分为再服。(《产经》同之。)
治妊妇尿血方第二十六

《病源论》云∶尿血者,有热气乘于血,血得热故尿血。

《产经》云∶治妊身尿血方∶取其爪甲及发烧作末,酒服之。

又方∶龙骨,冶下,三指撮,先食酒服,日三。

又方∶鹿角屑(一两,熬,)大豆卷(二两)桂心(一两)三味,下筛,酒服方寸匕,日三。

《葛氏方》云∶治妊身尿血方∶取黍、烧末,服方寸匕,日三。
治妊妇淋病方第二十七

《病源论》云∶妊身之人胞系于肾,肾患虚热成淋,谓之子淋。

《录验方》云∶葵子一升,以水三升,煮取二升,分再服,亦切根用之。

《千金方》云∶葵子,茯苓一两为散,水服方寸匕,日三。

《医门方》云∶治妊娠患淋,小便涩,水道热痛方∶车前子五两,炙,葵子一升,水五升,煮取一升半,分二服。

《产经》云∶妊身小便不利方∶葵子一升,榆皮一把,以水二升,合煮三沸,去滓,服一升,日三。

又方∶滑石以水和泥于脐中,浓二寸,良。
治妊妇遗尿方第二十八

《产经》云∶治妊身遗尿方∶取胡燕巢中草,烧末,服半钱匕,水酒无在。

又方∶龙骨冶末,三指撮,先食酒服,日三。

又方∶白蔹十分,夕药十分,冶下,酒服方寸匕,日三。
治妊妇霍乱方第二十九

《产经》云∶治妊身霍乱甘草汤方∶甘草(二两,炙)浓朴(三两)干姜(二两)当归(二两)上四味,切,以水七升,煮取二升半,分三服,日三。(今按∶《博济安众方》∶药各一两,水三升,煎取一升,分三四服。)《极要方》云∶妊娠饮食不销,成霍乱,心腹痛大吐胸心痰。浓朴汤方∶当归(四两)人参(三两)浓朴(三两)芎(二两)干姜(二两)以水九升,煮取二升半,分三服,羸人分四服。
治妊妇疟方第三十

《产经》云∶恒山(一两)甘葛(半两)枳子(二两)葱白(四株)凡四物,水五升,煮取二升,未发服一升,临发复服一升,自断。

《僧深方》云∶竹叶一升,细切。恒山一两,细切。水一斗半,煮竹叶,取七升半,纳恒山渍一宿,明旦煮取二升半,再服,先发一时一服,发一服尽,去竹叶纳恒山。

《集验方》云∶恒山(二两)甘草(一两)黄芩(半两)乌梅(十四枚,碎)石膏(八两,碎,绵裹)凡五物,切,以酒一升半,水一升半合。渍药一宿,煮三四沸,去滓,初服六合,复服四合,后服二合,凡三服。(今按∶《博济要众方》∶恒山二两,甘草半两,黄芩三两,乌梅半两、石膏半两,酒一升,浸一宿,煎一数沸,去滓,分三四服。)
治妊妇温病方第三十一

《产经》云∶治妊身温病不可服药方∶取竹沥二升,煎之减半,适寒温服之,立愈,良。

又方∶以井底泥涂病处,良。

又方∶以人尿涂,随其痛处,良。
治妊妇中恶方第三十二

《产经》云∶治妊身中恶,心腹暴痛,逐动胎。少腹急,当归葱白汤方∶当归(四两)人参(二两)浓朴(二两)葱白(一虎口)胶(二两)芎(二两)上六物,以水七升,煮取二升半,分三服。

又云∶吴茱萸酒方∶吴茱萸五合,以酒三升,煮三沸,分三服,良。
治妊妇咳嗽方第三十三

《产经》云∶治妊身咳逆,若伤寒咳,人参汤方∶人参甘草(各一两)生姜(五两)大枣(十枚)凡四物,切,以水四升,煮取一升半,分二服,良。
治妊妇时病令子不落方第三十四

《千金方》云∶灶中黄土水和涂脐,方五寸,干复涂之。(今检∶《葛氏方》云∶涂腹上。)又方∶泔清和涂之,和酒涂并良。(今按∶《葛氏方》云∶涂腹上。)《葛氏方》云∶取井中泥,泥心下三寸。
治妊妇日月未至欲产方第三十五

《短剧方》云∶捣知母,和蜜为丸,如梧子,服一丸,痛不止更服一丸。

《崔侍郎方》云∶芦根下土三指撮,酒服之。

《葛氏方》云∶灶中黄土末,以鸡子白丸如梧子,吞一丸。
治妊妇胎死不出方第三十六

《短剧方》云∶治月未足,胎死不出。母欲死方∶大豆,醋煮,服三升,死儿立出,分二服之。(《千金方》、《葛氏方》同之。)又方∶好书墨三寸,末,顿服。

又方∶桃白皮如梧子大,服一丸,立出。

又方∶盐一升,鸡子二枚,和,顿服之。

又方∶瞿麦一把,煮令二三沸,饮其汁立产。一方下筛服方寸匕。

《医门方》云∶疗胎死腹中不出,其母气欲绝方∶水银二两,吞之,儿立出。

又方∶伏龙下土,下筛,三指撮,以酒服即出。

《产经》云∶治妊身子死腹中不出方∶取赤茎牛膝根,碎,以沸汤沃之,饮汁,儿立出。

又云∶周德成妇怀身八月,状KT缘之,其腹中儿背折,胎死腹中三日困笃方∶取黑大豆一升,熬,以清酒一斗渍之须臾,择去豆,可得三升汁,顿服即下胎。
治妊妇欲去胎方第三十七

《产经》云∶治妊身胎二三月欲去胎方∶大麦面五升,以清酒一斗合煮,令三沸,去滓,分五服,当宿不食。服之其子即麋腹中,令母不疾,千金不易。(《千金方》同之。)《短剧方》云∶妊身欲去子方∶栝蒌(三两)豉(一升)桂心(三两)凡以水四升,煮取一升八合,分三服。

又方∶附子三枚,冶作屑,以好苦酒和,涂右足心即去,大良验。

又云∶妇人得温病欲去腹中胎方∶取鸡子一枚,扣之,以三指撮盐置鸡子中,服之,立出。

又方∶取井底泥,手书其腹,立出。神良。

《千金方》云∶妊身得病事烦去胎方∶麦一升,末,和蜜一升服之,神效。

《葛氏方》云∶或不以理欲去胎方∶斑蝥,烧末,服一枚郎下。

《录验方》云∶煮牛膝根服之。

《如意方》云∶去胎术∶以守宫若蛇肝KT和涂脐,有子即下,永无复有。

又煮桃根,令极浓,以浴及渍膝,胎下。

《极要方》云∶妊娠欲去胎方∶瞿麦(半斤)桂心(三两)蟹爪(一升)牛膝(五两)上,以水三升,酒五升,煮取一升,分三服。

又云∶去胎后血下冲心方∶生姜,切,五升,以水八升,煮取三升,分三服。

医心方卷第二十二医心方卷第二十二背记治妊妇中蛊方∶《子母秘录》云∶鼓皮烧,冶末,服方寸匕,须臾,自呼蛊主姓名。
卷第二十三
产妇向坐地法第一

《产经》云∶按∶产家妇人向坐之法,虽有其图,图多文繁难详求用,多生疑惑。故今更撰采其实录,俱载十二月图中也,一切所用晓然易解。凡在产者,宜皆依此,且余神图无复所用。然此亦不可不解,故以备载例焉。《生经》曰∶妇人怀妊十月,俱已成子,宜顺天生,吸地之气,得天之虚,而避恶神,以待生也。

又云∶黄帝曰∶人生寿命长短吉凶者,皆在其母,初生向,天一八神产乳。为藏胎胞常避之大凶,若不避而犯者,或伤母子;或伤其父;或子虽长,终不全命;或子虽大,必有多病;或子贫贱;或子□罪,或子分离;或子不孝;或子狐独;或子顽愚。不可不慎。

又云∶妇人产乳,先审视十二月神图,能顺天气,可向日虚月空。知天一日游八神,诸神所在方向,不可互向,大凶。或日虚之上恶神并者,当向天道天德为吉,无咎。今按∶十二月图依繁不取,但避恶神在方,载天气行日虚、月空、并天道天德等吉地,以备时用也。

正月天气南行,产妇面向于南,以左膝着丙地坐,大吉也。(即日虚月德地。)又,天道在辛,天德在丁。(是亦吉地)二月天气西行,产妇面向于西,以右膝着辛地坐,大吉。(虽无吉神,本书载之。)又,乙丁地无恶神可用之。

三月天气北行,产妇面向于北,以右膝着癸地坐,大吉。(虽无吉神,本书载之。)又,日虚天道天德在壬,又丁地无恶神,吉也。

四月天气西行,产妇面向于西,以左膝着庚地坐,大吉。(即日虚月德地。)又,天道在丁,天德在辛。

五月天气北行,产妇面于北,以右膝着癸地坐,大吉。(无吉神而本书载之。)又,乙丁辛地,无恶神可用之。

六月天气东行,产妇面向于东,以左膝着甲地坐大吉。(即日虚天道地。)又,乙辛地无恶神。

七月天气北行,产妇面向于北,以左膝着壬地坐,大吉。(即日虚月德地。)又,天德在癸,天道在辛。

八月天气东行,产妇面向于东,以左膝着甲地坐,大吉。(虽有日虚月空,又□鬼道可忌。)又,乙丁辛地无恶神。

九月天气南行,产妇面向于南,以左膝着丙地坐,大吉。(即日虚天道天德地。)又,丁癸地无恶神。

十月天气东行,产妇面向于东,以左膝着甲地坐,大吉。(即日虚月德地。)又,天道在癸。又,丁地无恶神。

十一月天气南行,产妇面向于南,以右膝着丁地坐,大吉。(无吉神而本书载之。)又,乙辛癸地无恶神。

十二月天气西行,产妇面向于西,以右膝着辛地坐,大吉。(虽无吉神,本书载之。)又,乙辛地无恶神。
产妇反支月忌法第二

《产经》云∶反支者,周来害人,名曰反支。若产乳妇人犯者,十死,不可不慎。若产乳值反支月者,当在牛皮上,若灰上,勿令污水血恶物着地,着地则杀人。又浣濯皆以器盛之,过此忌月乃止。

年立反支∶年立子反支在申七月产忌;年立丑反支在酉八月产忌;年立寅反支在戌九月产忌;年立卯反支在亥十月产忌;年立辰反支在子十一月产忌;年立巳反支在丑十二月产忌;年立午反支在寅正月产忌;年立未反支在卯二月产忌;年立申反支在辰三月产忌;年立酉反支在巳四月产忌;年立戌反支在午五月产忌;年立亥反支在未六月产忌。

年数反支∶女年十三反支七月忌申;女年十四反支八月忌酉;女年十五反支九月忌戌;女年十六反支十月忌亥;女年十七反支十一月忌子;女年十八反支十二月忌丑;女年十九反支正月忌寅;女年二十反支二月忌卯;女年二十一反支三月忌辰;女年二十二反支四月忌巳;女年二十三反支五月忌午;女年二十四反支六月忌未;女年二十五反支七月忌申;女年二十六反支八月忌酉;女年二十七反支九月忌戌;女年二十八反支十月忌亥;女年二十九反支十一月忌子;女年三十反支十二月忌丑;女年三十一反支正月忌寅;女年三十二反支二月忌卯;女年三十三反支三月忌辰;女年三十四反支四月忌巳;女年三十五反支五月忌午;女年三十六反支六月忌未;女年三十七反支七月忌申;女年三十八反支八月忌酉;女年三十九反支九月忌戌;女年四十反支十月忌亥;女年四十一反支十一月忌子;女年四十二反支十二月忌丑;女年四十三反支正月忌寅;女年四十四反支二月忌卯;女年四十五反支三月忌辰;女年四十六反支四月忌巳;女年四十七反支五月忌午;女年四十八反支六月忌未;女年四十九反支七月忌申。

生年反支∶子生女反支正月;亥生女反支二月;戌生女反支三月;酉生女反支四月;申生女反支五月;午生女反支六月;午生女反支七月;巳生女反支八月;辰生女反支九月;卯生女反支十月;寅生女反支十一月;丑生女反支十二月。

日反支∶子丑朔六日反支;寅卯朔五日反支;辰巳朔四日反支;午未朔三日反支;申酉朔二日反支;戍亥朔一日反支。
产妇用意法第三

《千金方》云∶论曰∶产妇虽是秽恶,然将痛之时,及未产已产,并不得令死丧秽家之人来视之,则生难。若已产者,则伤儿。

又云∶凡欲产时,特忌多人瞻视。唯三人在旁,待生讫了,仍可告语诸人也。若人众看之,无不难耳。

又云∶儿出讫,一切人及母忌问是男是女。又勿令母看视秽污。

又云∶凡产妇慎热食热药,常当识此,饮食当如人肌。

《产经》云∶凡妇人初生儿,不须自视。已付边人,莫问男女。边人莫言男女也。儿败。

《短剧方》云∶凡妇人产暗秽血露未净,不可出户牖至井灶所也。不朝神祗及祠祀也。
产妇借地法第四

《子母秘录》云∶体女子法,为产妇借地百无所忌,借地文∶东借十步、西借十步、南借十步、北借十步、上借十步、下借十步,壁方之中,三十余步。产妇借地,恐有秽污,或有东海神王,或有西海神王,或有南海神王,或有北海神王,或有日游将军,白虎夫人,横去十丈;轩辕招摇,举高十丈;天狗地轴,入地十丈。急急如律令。(入所指月一日即写一本,读诵三遍讫,贴在所居北壁正中。)
产妇安产庐法第五

《产经》云∶按月之方安产庐吉∶正月、六月、七月、十一月作庐一户,皆东南向,吉。

二月、三月、四月、五月、八月、九月、十月、十二月作庐一户,皆西南向,吉。

凡作产庐,无以枣棘子、铤戟杖;又禁居生麦稼水树下,大凶。又勿近灶祭,亦大凶。
产妇禁坐草法第六

《产经》云∶铺草席咒曰∶铁阳铁阳,非公当是王。一言得之铜,二言得之铁,母子相共,左王后西王母,前朱雀后玄武,仙人玉女来此护我,诸恶鬼魉,莫近来触,急急如律令。

(诵三遍。)
产妇禁水法第七

《子母秘录》云∶产时贮水咒曰∶南无三宝水,水在井中为井水,水在河中为河水,水在盏中为盏水,水入腹中为佛水。自知非真水,莫当真水。以净持浊以正邪,日游月杀五十一将军、青龙、白虎、朱雀、玄武、招摇、天狗、轩辕、女妖,天吞地吞,悬尸闭肚六甲、六甲禁讳十二神王土府伏龙,各安所在,不得动静,不得妄干,若有动静,若有妄干,头破作七分,身完不具,阿法尼阿法尼,毗罗莫多梨婆地利沙呵。
产妇易产法第八

《产经》云∶妊身垂七月常可服丹参膏,坐卧之间不觉忽生也。以温酒服如枣核,日三。

其药在妊妇方中。

《葛氏方》云∶密取马□毛系衣中,勿令知耳。

《短剧方》云∶马衔一枚,觉痛时右手持之。

又方∶蛇蜕皮头尾完具者一枚,觉痛时以绢囊盛,绕腰甚良。

《陶景本草经注》云∶皮毛以与产妇持之,令易产。

《千金方》云;临产时必先脱常所着衣,以笼罩头及口,令致密,易产。神验。

《子母秘录》云∶易产法∶带飞鸟毛及蜚生虫。(状如啮发。头上有一角者。)又方∶带獭皮,吉。

又云∶古今方能令产安稳,以汤从心上洗,即平安。
治产难方第九

《病源论》云∶产难者,凡有数种,或先漏胞去血,子脏干燥;或子宫宿挟疾病;或触犯禁忌;或产时未至便即惊动,秽露早下,子道干涩,妇力瘦弱,皆令产难。凡腹痛腰未痛者,未产,腹腰连痛者,即产也。

《产经》云∶夫产难者,胞胎之时诸禁不慎,或触犯神灵,饮食不节,愁思带胸,邪结脐下,阴阳失理,并使难产也。贤母宜豫慎之。

《医门方》云∶产难死生候∶若母面赤舌青者,儿死母活;唇口青,口两边沫出者,子母俱死;面赤舌青沫出者,母死儿活。(《集验方》同之。)又云∶产妇身重而寒热,舌下脉青黑及胎中冷者,子母并死矣。

《大集陀罗尼经》神咒∶南天干陀天与我咒句如意成吉,祗利祗利,祗罗针陀施祗罗钵多悉婆诃。

上其咒,令产妇易生,朱书华皮上,烧作灰,和清水服之。即令怀子易生,聪明智慧,寿命延长,不遭狂横。(本在上易产篇。)《子母秘录》云∶防产难及运,咒曰∶耆利暗罗拔施,罗拔施耆利暗罗河沙呵。

上,临产预至心,礼签诵满千遍,神验不可言,常用有效。

又云∶若以色见我,以音声求我,是人行邪道,不能见如来。

上,临产墨书前四句,分为四符,脐上度至心,水中吞之,立随儿出,曾有效。

又云∶古方酥膏,有难产者,或经三日五不日得平安,或横或竖或一手出或一脚出,百方千计,终不平安。服此酥膏,其膏在孩儿身上立出。其方无比。初服半匙,渐加至一匙,令多恐呕逆。

好酥(一斤)秋葵子(一升)滑石瞿麦(各一两)好蜜(半升)大豆黄卷皮(二两)上六物(味),先用清酒一升,细研葵子,纳酥中相和,微火煎,可强半升为度,忌生冷,余无忌。

《产经》云∶产难时,皆开门户窗、瓮、瓶、釜、一切有盖之类,大效。

又云∶产难时祝曰∶上天苍苍,下地郁郁,为帝王臣,何故不出?速出速出,天帝在户,为汝着名,速出速出。

又方∶\r〔图〕\pyxf1.bmp\r以朱书吞之良。

又方∶\r〔图〕\pyxf2.bmp\r烧作灰,以水服即生。

又方∶取真当归,使产者左右手持之即生。一云∶用槐子矣。

又方∶胡麻油服之即生。

又方∶以大麻子二七枚,吞之立生。

又方∶取弓弩弦令带产者腰中,良。

又方∶取大豆中破,书左作日字,右作月字,合吞之,大吉。

又方∶取夫裤带烧末酒服,良。

《葛氏方》云∶吞大豆三枚。

又方∶吞槐子三枚。

又方∶芦根下土三指撮,酒服之。

又方∶以水银如弹丸大,格口纳喉中,捧起令下,子立出。

《录验方》云∶破大豆以夫名字书豆中,合吞之,即出。

《博济安众方》∶取牛屎中大豆一粒,一片书“父入”字,一片书“子出”字,吞之。

《龙门方》云∶取凿柄入铁裹者烧末,酒服之,立下。

《短剧方》云∶取其父衣以覆井,即出,神良。

又方∶小麦二七枚,吞之即出。

又方∶出蚕种布三寸,烧作散,酒服方寸匕,立出。

又方∶酥一合,以酒和服,即出。

又方∶烧兔毛末服方寸匕,即生。

《僧深方》云∶取猪肪煎吞如鸡子者(黄欤)一枚,即生;不生,复吞之。

又方∶蒲黄大如枣,以井华水服之,良验。

又方∶取灶中黄土末,以三指撮酒服,立生。土着儿头,出良。(今按∶《博济安众方》∶加灶突墨。)又方∶滑石末三指撮酒服。

《千金方》云∶烧大刀环(户关反)令热以酒泼之,取一升服之,救死。(《短剧方》同之。)《极要方》云∶取赤小豆二枚吞之,立儿手持出。

《新录方》云∶葵子二七枚服之。(今检∶《短剧方》∶陈葵子三指撮,酒服之。)《集验方》云∶令夫从外含水着妇口中二七过,立出。

《经心方》云∶芎为屑,服方寸匕,神良。
治逆产方第十

《病源论》云∶逆产犹初觉腹痛,产时未到,惊动复早,儿转未竟,便用力产,则令逆也。或触犯禁忌所为。

《集验方》云∶逆生横生不出。手足先见方∶其父名书儿足下,即顺。

又方∶以盐涂儿足底。又可急搔爪之。

《产经》云∶逆生符文\r〔图〕\pyxf3.bmp\r(\r〔图〕\pyxf4.bmp\r以朱书吞之大吉。)又云∶逆生手足先出者方∶取三家饭置儿手内即顺。

又方∶丹书左足下作“千”字,石(右)足下作“黑”字。

《葛氏方》云∶盐以汤和,涂儿下,并摩妇腹上。

又方∶真丹涂儿下。

又方∶取釜月底墨以交牙书儿下。

又方∶丹书左足下作“千”字,右足下作“黑”字。

《短剧方》云∶烧儿父手足十指爪甲,冶末服之。

又方∶取生艾半斤,清酒四升,煮取一升,顿服之,则顺生,若不饮酒,用水。

《千金方》云∶取三家盐,熬,涂儿手足立出。

又方∶取车轮上土,三指撮服之。

《录验方》云∶取盐,真珠各少许,合和,涂儿下立顺。

《新录方》云∶取三家水,服并洗手即顺生。

《僧深方》云∶熬葵子令黄,三指撮酒服之。
治横生方第十一

《病源论》云∶横生者,产时未到,始觉腹痛,惊动伤损,儿转未竟,便用力产之,故令横生。

《产经》云∶取舂杵头糠,刮如弹丸,酒服之即顺生。

《短剧方》云∶栝蒌实中子一枚,削去尖者,以水浆吞之,立产。

《葛氏方》云∶服水银如大豆二枚。

又方∶取梁上尘,三指撮服之。

又方∶烧铁杵,令赤,纳酒中饮之。

又方∶烧斧如上。

《集验方》云∶菟丝子,酒若米汁服方寸匕,即出。

又方∶车前子服之如上法。
治子上迫心方第十二

《病源论》云∶子上迫心者,节适失宜,胎动气逆,故子上迫心,胎下者生。

《短剧方》云∶子上迫心方∶取弩弦缚心下即出。

《千金方》云∶若子趋后孔者,熬盐熨之。(《葛氏方》同之。)《葛氏方》云∶子上迫心方,取乌犬血,小小饮之立下。

又云∶先出手者,嚼盐涂儿掌中。

又云∶先出足者,以丹书儿左足下作“千”字,右足下作“黑字”,即顺生。
治子死腹中方第十三

《病源论》云∶此或因惊动倒仆,或染温伤寒,邪毒入于胞脏,致令胎死。其候当胎处冷为胎已死也。

《短剧方》云∶吞水银二两,立出。

又方∶捣芎,酒服方寸匕,神良。

《僧深方》云∶取牛膝根两株,拍破以沸汤泼之饮汁,儿。(立出)又方∶以酒服蒲黄二寸匕。

又方∶好书墨三寸,末,一顿饮之即下。

《博济安众方》∶醋煮赤豆,服三升,儿立出。

又方∶验醋一升,格口灌之。

又方∶当归末,酒服方寸匕,立出。

《葛氏方》云∶以苦酒煮大豆,令浓,漉取汁,服三升,死胎即下。

又方∶饮夫小便一升。

《龙门方》云∶桃根煮浓,用浴膝下,立出。

《苏敬本草注》云∶伏翼屎灰,酒服方寸匕。

《千金方》云∶以牛屎涂母腹上立出。

又方∶蚁室土三升,熬令热袋盛心下,胎即下。(《短剧方》同之。)又方∶葵子(一升)胶(五两,炙)水五升,煮取二升,顿服之。

《集验方》云∶灶中黄土,三指撮,酒服之,立出。(《短剧方》同之。)又云∶治产难或半生,或胎不下,或子死腹中,或着脊及在草数日不产,血气上荡心。

女面无色气欲绝方∶煎成猪膏(一升)白蜜(一升)淳酒(二升)上三味,合煎,取三升,分五服,极验。

又云∶产难,子死腹中。又妊两儿,一儿死腹中,一儿活腹中,死者出。生者安方∶蟹爪(一升)甘草(二尺,炙)阿胶(三两)上三味,以东流水一斗,煮取三升,纳胶令烊顿服。不能顿服,分再服,药入即活。(《千金方》同之。)《产经》云∶治子死腹中方∶取瞿麦一把,煮二三沸,饮其汁立出。一方∶冶,下服方寸匕。

又云∶治胎死腹中符文∶(此二符以朱书,吞之即生。)
治胞衣不出方第十四

《病源论》云∶有产儿出而胞不落者,世谓之息胞。由产出而体疲,不能更用气,胞经停之间外冷气乘之,则血道涩,故胞不出。

若挽其胞系断者,其胞上则毙人。

《产经》云∶(胞衣不出时,吞之立下,大吉。)又方∶以水煮弓弦,令少少沸,饮之一升许。

又方∶多服猪肪。

《陶景本草注》云∶吞胡麻油少少。

又方∶取弓弩弦缚腰。

《葛氏方》云∶月水布烧末,以服少少。

又方∶末皂荚,纳鼻中得嚏,即下。

又方∶解发刺喉中,令得呕之,良。

《僧深方》云∶水银服如小豆二枚。

又方∶取夫单衣若巾,覆井立出。

《短剧方》云∶小麦,小豆,合煮汁饮,立出。

又方∶井中土如梧子大吞之。

《千金方》云∶服蒲黄如枣大者。

又方∶男吞小豆七枚,女吞十四枚。(《集验方》同。)《集验方》云∶牛膝半斤,葵子三升,切,以水七升,煮取三升,分三服。(《医门方》同之。)《龙门方》云∶取灶中黄土末,着脐中。(今按∶《广济方》三指撮水服之。)
藏胞衣KT理法第十五

《产经》云∶凡欲藏胞衣,必先以清水好洗子胞,令清洁。以新瓦瓮,其盖亦新,毕乃以真绛缯裹胞讫,取子贡钱五枚,置瓮底中罗烈,令文上向。乃已取所裹胞盛纳瓮中以盖覆之,周密泥封,勿令入诸虫畜禽兽得食之,毕,按随月图以阳人使理之,掘深三尺二寸,坚筑之,不欲令复发故耳。能顺从此法者,令儿长生,鲜洁美好,方高心善,圣智富贵也。且以欲令儿有父才者,以新笔一柄着胞上藏之,大吉。此黄帝百二十占中秘文也。且藏胞之人当得令名佳士者,则令儿辨慧多智,有令名美才,终始无病,富贵长寿矣。

又云∶一法先以水洗胞,令清洁讫,复用清酒洗胞,以新瓦瓮盛胞,取鸡雏一枚,以布若缯缠雏置胞上,以瓦盖其口埋之。按十二月图于算多上藏之吉。其地向阳之处,深无过三尺,坚筑之,勿令发也,大吉。男用雄雏。女用雌雏。(一说云∶如来云∶我不杀生,故得寿长,何杀生求寿命?故不疏之。)又云∶数数失子,藏胞衣法∶昔禹于雷泽之上,有一妇人悲哭而来,禹问其由,答曰∶外家数生子而皆夭死,一无生在,故哀哭也。禹教此法,子皆长寿,无夭失也。取产胞衣善择去草尘洗之清,作一土人,生儿男者作男像,生儿女者作女像,以绛衣裹土人。先以三钱置新瓮中已,取土人着钱上,复取子胞置钱上,以盖新瓯,令周密封泥之。按算多地上,使儿公自掘埋之,毕,祝曰∶一钱为汝领地主,一钱为汝寿领算,一钱为汝领口食,讫,以左足蹈之。坚筑如上法。(今按∶后条儿公者儿父也。)
藏胞衣吉凶日法第十六

吉日《产经》云∶正月亥子二月丑寅三月巳午寅四月申酉卯五月亥酉六月寅卯辰七月午八月未申九月巳亥十月寅申十一月未午十二月申酉又云∶甲乙生丙丁藏丙丁生戍巳藏戊己生庚辛藏庚辛生壬癸藏壬癸生甲乙藏忌日《产经》云∶春无以甲乙,夏无以丙丁,秋无以庚辛,冬无以壬癸。

上四时忌日,皆恶,不避,身子俱亡。

又云∶甲辰、乙巳、丙丁、午未、戊申、戊戍上日勿藏胞,净洗十余过,置瓮中须待良日乃藏之。

又云∶避月十日、二十日、月未尽一日,不可埋胞,大凶。

又云∶当避月一日、十一日、二十一日,凶。

又云∶避建、除、破、厄、闭日,大凶。

又云∶勿以儿生日,令儿不寿。

又云∶藏胞以日,小儿死。(又日法在《湛余经》中。)又云∶无以八魁日、复日、伯日、小儿生相克日,皆忌。
藏胞恶处法第十七

《产经》云∶藏胞阴地,不见日月,若垣壁下,若粪中,水渎,坑坎之旁,若清溷(胡困反)旁,皆不宜藏之。令儿多气疾,疮疥,痈肿也。

藏胞当道中,若四衢对间,令儿娄逢县官飞官,遇疫疾。藏胞近故井,若社稷旁,冢墓之边,祠神处所,所居近者,皆令狂痴不寿。

藏胞故器瓦瓮者,儿令五罪,凶。

藏胞火烧之处者,令儿则烧死,凶。

藏胞勿令入虫蛾草等入者,令儿丑恶,多死疡疮病,凶。

藏胞近社祠,若故社处旁,鬼神祭所,令儿魂魄飞扬不具恶梦,奔走如狂。痴癫,儿脉易惊恐啼,喜见鬼,生恶疮肿,肠痈。

藏胞勿令犬鼠猪食之,令儿惊螈多疾。

藏胞故垣墙下令儿常病腹肠。

藏胞中道令儿戮死不寿,后无子孙。

藏胞故坟井处,令儿耳目不聪,害孔窍。

藏胞当门户,令儿痴,失明,喑聋。

藏胞水旁故池处,令儿以为溺死不葬。

藏胞溜中令儿失精明而盲。

藏胞牛兰若故窖处,令儿痴。

又云∶勿以小儿行年上。(男寅女申为行年上。)又避小儿祸害绝命之地。(天门绝命地,鬼门祸害地。)
藏胞衣吉方第十八

《产经》云∶夫生之与死,夭之与寿,正在产乳藏胞。凡在产者,岂可不慎。敬神畏天者,典坟之所崇;避难推祸者,诸贤之所务也。是以顺天道者昌,逆地理者亡。古之常道也。

余以暗塞究搜百家之要,藏胞之道术于此备矣。使产生之场几得无咎也。

凡欲藏胞胎者,可先详视十二月图,算多处者有寿;算少处者不寿。或算多,地者忌神并者亦当避之。次取算多亦吉。又既得寿地,其日恶者,待以良日乃埋之,吉。又,虽为寿处,必得高燥向阳之地,能者寿长、智高、富贵无极也。其高燥地者,达近自在无苦。

又云∶《经》曰∶欲藏产子胞胎者,先视十二月神图,八神,诸神在方,不可KT(至也)犯,犯之咎重,不可不慎。

又云∶未KT子曰,凡欲藏子胞,直就天德月德之地者,子必富贵寿老无疾,最吉之地,故其利万倍也。若不得天德月德者,天道人道地亦吉,其利百倍。又不得此地者,亦可用反向大吉之地,亦吉利。若虽是吉地,而与恶神并者,不可藏胞,夫言吉地者,谓之无凶。

故虽云吉地而与恶神并者,此为凶地,宜慎择之。今按∶藏胞衣法不载月图。但避八神等所在之凶地,取天德月德等吉方∶正月藏胞衣∶丁地吉,年一百,(即天德地;)丑地年百十而月杀并在,亦小儿祸害地,故不成其善,他皆效此。

又,日虚月德在丙,天道在辛。

二月藏胞衣∶人门地吉,年九十,(即天德人道地;)天门鬼门虽有吉神而是小儿祸害绝命之地,故不吉;丑地寿多而小儿行年所立之地,故不可犯KT(都奚反)凶也。又,乙丁辛地无恶神,可用之。

三月藏胞衣∶庚地吉,年九十二,(即天德人道地。)又壬地大吉,(是天道地。)又丁地吉。

四月藏胞衣∶辛地吉,年八十,(是天德人道地,)又丁地,(是天道。)五月藏胞衣∶干地吉,年九十一,(是天德人道地。)又乙辛地无恶神。

六月藏胞衣∶壬地吉,年七十八,(是天德人道地。)又乙辛地无恶神。

七月藏胞衣∶癸地吉,年七十八,(是天德人道地。)又辛地(天道,)壬地大吉。

八月藏胞衣∶艮地鬼门吉,年八十六,(是天德人道地。)又乙丁辛地无恶神。

九月藏胞衣∶甲地吉,年八十五,(是天德人道地。)又丙地大吉(天道。)又丁癸地无恶神。

十月藏胞衣∶乙地吉,年八十四,(即天德人道地。)又甲地大吉月德。癸地(天道。)丁地无恶神。

十一月藏胞衣∶巽地,户地吉,年百二十,(天德人道地。)又乙辛癸地无恶神。

十二月藏胞衣∶丙地吉,年百天德人道地。又乙辛地无恶神。
妇人产后禁忌第十九

《千金方》云∶论曰∶凡妇人,非止临产须忧,至于产后大须将慎,危笃之至,其在于斯。勿以产时无他乃纵心恣意,无所不犯,犯时微若秋豪,成病于嵩岱,何则?产后之病难治于余病也。妇人产讫,五脏虚羸,唯得将补,不可转泻。若其有病,即须快药。若行快药,转更增虚,虚中复虚,危殆甚矣。所以大也。

凡妇人产后百日以来,极须怖惧忧畏,勿浪犯触是等,犯触房事弥深,若有所犯,必身反强直犹如角弓反张,名曰褥风,则是其犯候也。若其如此,事同转烛。凡百女人,宜熟慎之。

凡产后满百日乃可行房,不尔至死虚羸,百病滋长。慎之慎之。

凡妇人皆患风脐下冷,莫不由此早行房也。

《短剧方》云∶夫死生皆有三日也。古时妇人产下地坐草法如就死也。既得生产,谓之免难也。亲属将猪肝来庆之,以猪肝补养五内伤绝也。非庆其儿也。

又云∶妇人产后盈月者,以其产生身经暗秽,血露未净。不可出户牖,至井灶所也。亦不朝神祗及祠祀也。盈月者,非为数满三十日,是跨月故也。若是正月产,跨二月,入三月,是跨月耳。

又云∶妇人产时骨分开解,是以子路开张,儿乃得出耳。满百日乃得完合平复也。妇人不自知,唯盈月便云是平复,合会阴阳,动伤百脉,则为五劳七伤之疾。

《养生志》云∶产妇食酢面无色。又云∶食梨子伤,闭塞血结不利。
治产后运闷方第二十

《病源论》云∶运闷之状,心烦,气欲绝是也。亦有去血过多,亦有下血极少,皆令运。

若产去血过多,血虚气极,如此而运闷者,但烦闷而已。若下血过少而气逆者,则血随气上掩心,亦令运闷,则烦闷而心满气急。二者为异。亦其候。产妇血下多少,则知其应运与不运。凡产时当慎向坐卧。若触犯禁忌,多令运闷。

《经心方》∶治产后忽闷冒汗出不识人者,是暴虚故也。

取验醋以涂口鼻,仍置醋于前,使闻其气,兼细细饮之。此为上法。(今按∶《子母秘录》云∶如觉运即以醋喷其面,苏来,即今饮醋。)又方∶破鸡子,吞之便醒。

又云∶若不醒者可与男子小便,灌口得一升,入腹,大佳。

又云∶若与鸡子等不醒者,可急与竹沥汁一升,一服五合。

《千金方》治产后血运,心闷气绝方∶验醋一升,和所产血如枣大,服兼面。

又方∶大豆熬令烟绝,热,以清酒一升泼之,承其汁饮之。

《葛氏方》治血气逆,心烦满者方∶生竹皮一升,水三升,煮取一升半,分三服。

《产经》治产后心闷,眼不得开方∶赤小豆为散,东流水和,方寸匕服之。

《僧深方》治产后心闷腹痛方∶生地黄汁一升,酒三合,和温服。(今按∶《博济安众方》无酒。)《集验方》治产后心闷,眼不得开方∶即当头顶上取发如两指大,强人牵之,眼即开。

《孟诜方》治产后血运心闷气绝方∶以冷水面即醒。

《博济安众方》云∶产后心闷不语烦热方∶地黄汁(五合)当归(一两,末)清酒(五合)姜汁(二合)上,童子小便一升和煎去滓,分服。

《子母秘录》云∶产后促迷不醒,唇口冷,已脉绝血青不语,此是晕鬼所出,血气上冲心方∶取验醋二合,鸡子一颗。上,先破鸡子于垸中,煮醋一沸,投醋于鸡子中熟搅,与产者顿服之,立定。
治产后恶血不止方第二十一

《病源论》云∶产伤于经血,其后虚损不复。或劳役损动而血暴崩下,遂淋沥不断时来,故谓恶露不尽。

《葛氏方》治产后恶血不除方∶生姜(三斤)咀,以水一斗,煮取三升,分三服。当下恶血。

《千金方》治产后恶血不尽烦闷腹痛方∶捣生藕,取汁,饮二升,甚验。

又方∶捣地黄汁一升,酒三合,合温,顿服之。

《短剧方》治产后漏血不息方∶蜂房故船竹茹凡二物,分等,皆烧末以酪及浆服方寸匕,日三。

《医门方》疗产后余血作(疾)痛兼块者方∶桂心干地黄(分等)末,酒服方寸匕,日二三。

又云∶疗产后血泄不禁止方∶急以干地黄末,酒服一匙,二三服即止。

《产经》疗产后腹中秽汁不尽,腹满不减。小豆汤方∶小豆五升,以水一斗,煮熟,尽服其汁,立除。

《耆婆方》治产后恶露不尽方∶生姜(一斤)蒲黄(三两)以水九升,煮取三升,分三服,得恶血出即瘥。

《录验方》治产后余血不尽多结成,吴茱萸散方∶吴茱萸(一两)薯蓣(二两)凡二物,冶下筛,酒服方寸匕,日三。
治产后腹痛方第二十二

《病源论》云∶产后脏虚或宿挟风寒,或新解触冷,与气相击搏,恒腹痛。若气逆上者,亦令心痛胸胁痛也。

《集验方》治产后腹痛方∶当归(一斤)酒(一斗)煮取七升,以大豆四升熬,酒洗热豆,去滓,随多少服,日二。

《葛氏方》治产后腹瘕痛方∶末桂,温酒服方寸匕,日三。

又方∶烧斧令赤,以染酒中饮之。

《千金方》治产后腹痛不可忍方∶牛膝(五两)酒(五升)煮取二升,分再服,若干,以酒渍之,然后可煮。

又方∶吴茱萸一升,以酒三升,渍一宿,煎得半升,顿服。

又云∶治产后腹中如弦,恒坚痛,无聊赖方∶当归屑二方寸匕,纳蜜一升,煎之适口,一顿服之。

《产经》治产后腹中绞痛,脐下坠满方∶以清酒煮白饴,令如浓白酒,顿服二升,不瘥复作,不过三,神良。

《僧深方》治产后余寒冷,腹中绞痛并上下方∶吴茱萸干姜当归夕药独活甘草(各一两)凡六物,水八升,煮取三升,分三服。
治产后心腹痛方第二十三

《病源论》云∶产后气血俱虚,因遇风寒乘之;血气相击,随气乍上乍下上冲心下攻腹,故令心腹痛。

《产经》治产后腹中虚冷,心腹痛,不思饮食,呕吐厥逆,补虚除风冷理仲当归汤方∶甘草(三两)当归(二两)人参(一两)白术(一两)干姜(半两)凡五物,水七升,煮取二升半,分三服,神良。

《子母秘录》治产后心腹痛方∶当归芎夕药干姜(各六分)为散,空腹温酒服一方寸匕,日二。

《耆婆方》治人心腹痛,此即产后血瘀方∶生姜(三斤)以水小三升,煮取一升半,分三服,当下血及恶水即愈。
治产后腹满方第二十四

《经心方》治产后腹满方∶黑豆(一升)水五升,煮取三升,澄清,酒五升合煎,取三升,分三服。

《子母秘录》治产后腹中秽汁不尽,腹满不减小豆汤方∶小豆(三升)以水一斗,煮熟,尽服其汁,立除。
治产后胸胁痛方第二十五

《经心方》治产后胸胁及腹壮热烦满方∶羚羊角烧为末,以冷水服之。

《广济方》治产后心胸中烦闷,血气涩,肋下坊不能食方∶生地黄汁(一升)当归(一两,末)清酒(五合)生姜汁(五合)上,和煎三四沸,去滓,温四五合服之,中间进少食。
治产后身肿方第二十六

《病源论》云∶夫产伤血劳气,腠理则虚,为风邪所乘,邪搏于气,气不得宣越,故令虚肿。轻浮如吹者,是邪搏于气,气肿也。若皮薄如熟李状,则变为水。

《短剧方》治产后中风冷,成肿欲死方∶取鼠壤四升,熬令热,以囊贮着腹上,亦着阴上下,使热气入腹中良。

《录验方》治产后余痛及血兼风肿方∶真当归一物切之,以酒一斗,煮取七升,以四升大豆熬令焦,及酒热(浇热)豆中,去滓,多少服,日二。

《千金方》治产后风肿面欲裂破者方∶以紫汤一服即瘥,神效。

《经心方》治产后肿满方∶乌豆一斗,水五升(二斗),煮取五升,以酒五升,煎取五升,分五服。

《子母秘录》治产后遍身肿方∶生地黄汁一升,酒二合,温顿饮之。

《产经》治产后诸大风中缓急肿气百病独活汤方∶独活当归常陆白术(各二两)凡四物,水一斗,煮取四升,服,且覆取汁。
治产后中风口噤方第二十七

《病源论》云∶产后中风口噤者,是血气虚而风入于颊夹口之筋也。

《短剧方》云∶大豆紫汤治产后中风困笃。或背强口噤,或但烦热苦渴,或头身皆重,或身痒,剧者。呕逆直视,此皆虚冷湿中风所为也。

大豆三升,熬令极熟,熟自无复声,豫便器KT若,以清酒五升,泼热豆即漉得二升汁,尽服之,温覆小微汗出身体裁润则愈。今产后皆依常稍服之,一以防风气,二则消血结。云周德成妻妊胎触盆缘伤折胎,皆死肠腹中,三日困笃,服此酒即瘥。后以治不安亦佳。

(今按∶《经心方》治产后中风百病。)又云∶产后忽痉口噤,面青手足强反张者方∶与竹沥汁一升即醒,中风者尤佳。(今按∶勘《葛氏方》多饮。)《葛氏方》云∶若中风,若风痉,通身冷直,口噤不知人方∶作沸汤纳壶中令生,妇以足蹒壶上,冷复易之。

又方∶吴茱萸一升,生姜五累,以酒五升,煮三沸,分三服。

今按∶《录验方》干姜,生姜累数用者,以其一支为累,取肥大者。

《千金方》治产后百病并中风痉口噤不开,理血止痛独活紫汤方∶独活(一斤)大豆(一斤)酒(一斗三升)三味,先以酒渍独活再宿。若急须,微火煮之令减三升,去滓。别熬大豆极焦,以独活酒洗大豆,即去滓服一升,日三夜一。

《录验方》治产后中风及饮痛方∶当归(二两)独活(四两)凡二物,以水八升,煮取三升,分服一升。

《僧深方》治产后中风口噤方∶独活(八两)葛根(六两)甘草(二两)生姜(六两)四物,水七升,煮取三升,分四服。(今按∶《博济安众方》∶独活二两,葛根一两,甘草一两,生姜二两。上以水二升,煎取八合,分五六服。)《博济安众方》产后中风,角弓反倒,口不语方∶蒜(二十辨)上,以水一升半,煎取五合,灌之,极验。

《产经》治产后中风,口噤独活汤方∶独活(三两)防风(二两)干姜(二两)桂心(二两)甘草(二两)当归(二两)凡六物,以清酒三升,水七升,合煮,取二升半,分三服。
治产后柔风方第二十八

《病源论》云∶产后柔风者,四肢不收,或缓或急,不得俯仰也。产则血气皆损,未平复而风邪乘之故也。

《葛氏方》治产后若中柔风,举体疼痛自汗出者方∶独活四两,以清酒二升合煮,取升半,分二服。

《产经》治产后中柔风,身体疼痛独活汤方∶羌独活(三两)葛根(三两)甘草(二两,炙)麻黄(一两)桂心(三两)生姜(六两)夕药(三两)干地黄(二两)凡八物,以清酒二升,水八升,煮取三升,分五服。一方无夕药。
治产后虚羸方第二十九

《病源论》云∶夫产损动腑脏,劳伤血气。轻者节养将摄,盈月便得平复,重者其日月虽满,气血犹未调和,故虚羸也。

《千金方》治产后虚羸喘乏,或一寒一热,状如虚,名为劳积,猪肾汤方∶猪肾(一具,去脂皮。无,用羊肾)香豉(一升)白粳米(一升)葱白(切,一升)四味,以水三升,煮取五升,去滓适性饮之,不瘥重作。

《葛氏方》治产后虚羸,白汗出鲤鱼汤方∶鲤鱼肉(三斤)葱白(一斤)香豉(一升)凡三物,水六升,煮取二升,分再服,微汗即止。
治产后不得眠方第三十

《葛氏方》若产后虚烦不得眠者方∶枳实,夕药分等并炙之,末,服方寸匕,日三。
治产后少气方第三十一

《医门方》疗产后少气,无力困乏虚烦者方∶人参茯苓(各十分)甘草(炙)桂心夕药(各八分)生麦门冬(去心)生地黄(各二十分)水九升,煮取三升,分三服。
治产后不能食方第三十二

《子母秘录》云∶产后诸状亦无所异,但若不能食方∶白术(四两)生姜(六两)上二味,细切,以水酒各三升,暖火煎药,取一升半,绞去滓,分温再服,许仁则与女。

《博济安众方》产后呕逆不能食方∶浓朴(二两,炙)白术(一两,炒)上,以水二升,煎取一升,分四五服。
治产后虚热方第三十三

《病源论》云∶产腑脏劳伤,血虚不复,而风邪乘之;搏于血气,使气不宣泄,而痞涩生热,或肢节烦愦,或唇干燥,但因虚生热,故谓之虚热。

《葛氏方》治产后烦热,若渴或身重痒方∶熬大豆,酒淋及热,饮二升,温覆取汗。

《千金方》治产后虚热头痛方∶白夕药(五两)桂心(三两)干地黄(五两)牡蛎(五两)四味,以水五升,煮取二升半,分三服。
治产后渴方第三十四

《病源论》云∶妇人以肾系胞,产则血水俱下,伤损肾与膀胱之气,津液竭燥,故令渴。

《医门方》疗产后大渴不止方∶芦根(切,一升)栝蒌(三两)人参甘草(炙)茯苓(各二两)生麦门(各四两,去心)大枣(十二枚)水九升,煮取三升,分三服。

《子母秘录》云∶产后渴方∶新汲水和蜜饮之,仍不论多少李,温与大新妇服之。
治产后汗出方第三十五

《病源论》云∶凡产后皆血虚,故多汗。

《录验方》治产后虚劳,汗出不止牡蛎散方∶牡蛎(二两)干姜(二两)麻黄根(二两)凡三物,冶筛,杂白粉,粉身,不过三四便止。

《千金方》治产后虚羸,盗汗,时邑邑恶寒方∶吴茱萸(如鸡子大)一味,以酒三升浸之半日,煮得二升,顿服一升,日再。间日饮之。

《子母秘录》治产后汗出不止,兼腹痛虚乏劳方∶通草夕药当归(各三两)生地黄(切,一升)上四味,切,以水六升,煮取二升半,去滓,分温三服。(今按∶《博济安众方》∶夕药,当归各一两,生地黄切半升。上,水二升,煎取一升,分服。)
治产后无乳汁方第三十六

《病源论》云∶妇人手太阳少阴之脉,下为月水,上为乳汁。经血不足者,故无乳汁。

《葛氏方》云∶凡去乳汁,勿置地,虫蚁食之,令乳无汁,可以泼东壁上。

又云∶治产后血乳无汁者方∶烧鹊巢末,三指撮,酒服之。

又方∶末蜂房,服三指撮。

《短剧方》下乳散方最验。

钟乳(五分)通草(五分)漏芦(二分)桂心(二分)栝蒌(一分)甘草(一分)凡六物,捣筛,饮服方寸匕,日三。

又方∶石膏三两,以水三升,煮三沸,一日饮,令尽,良。

《集验方》治乳无汁方∶取栝蒌根切一升,酒四升,煮三沸,去滓,服半升,日三。

《经心方》治妇人无乳汁方∶赤小豆三升,煮取汁,顿服之。

又方∶捣韭一把,取汁服,冬用根。

《千金方》治乳无汁方∶取母猪蹄一具粗切,以水二升,煮饮汁,不出更作。

又方∶烧鲤鱼头,末,酒服三指撮。

《僧深方》治乳不下方∶取生栝蒌根,烧作炭,冶下筛,食已,服方寸匕,日四五服。

又方∶冶下栝蒌,干者为散,勿烧,亦方寸匕,井华水服之。

《医门方》疗乳无汁方∶母猪蹄(二枚,切)通草(六两,绵裹)和煎作羹食之。

(今按∶《广利方》云∶母猪蹄一具,通草十二分,切,以水大四升,煎二大升,去滓,食后服一盏,并取此汁,作羹粥煎得。《千金方》∶母猪蹄一具,粗切,以水二升,煮饮汁,不不更作。)《枕中方》治妇人无乳汁方∶取母衣带,烧作灰,三指撮,酒服,即多汁。
治产后乳汁溢满方第三十七

《病源论》云∶经血盛者,则津液有余,故乳汁多而溢出。

《葛氏方》云∶乳汁溢满急痛者,但温石以熨之。

又云∶若因乳儿汁出不可止者,烧鸡子黄食之。
治产后妒乳方第三十八

《子母秘录》云∶产后妒乳,因乳汁不时泄,蓄积于内,遂成痈肿,其名妒乳;此甚急于痈疽,治之亦同痈结也。

《产经》云∶凡产后妇人宜勤泄去乳汁,不令蓄积,蓄积不时泄,内结KT(尺制反)痛发渴,因成脓也。

又云∶治妒乳肿方∶车前草熟捣,以苦酒和涂之。

《短剧方》治妒乳方∶生地黄汁以敷之。

又方∶葵根捣为末,服方寸匕,日三。

《华佗方》治妒乳方∶生蔓荆根,和盐捣浆,水煮合,日五服,或滓封之。

《葛氏方》治妒乳方∶梁上尘,醋和涂之。亦治阴肿。

又方∶榆白皮,捣醋和封之。
治产后阴开方第三十九

《病源论》云∶子脏宿虚,因产冷气乘之,血气得冷不能相荣,故令开也。

《千金方》治产劳玉门开而不闭方∶硫黄(四两)菟丝子(五分)吴茱萸(六分)蛇床子(四分)四味,捣,下筛,为散一升,以方寸匕投汤中洗玉门,日再。

又云∶治产后阴道开不闭方∶锻石(一升,熬之令能烧草)一味,以水二升,投中,适寒温入汁中坐,渍之,须臾,复常。此是神方,秘不传。已治人验。(今按∶《医门方》锻石一斗,水二斗,澄取一斗三升。)
治产后阴脱方第四十

《病源论》云∶产阴脱者,由宿有虚冷。因产用力过度,其气下冲,则阴下脱也。

《产经》∶治产后阴脱下痛方∶取蛇床子捣末,布囊盛之,炙令热熨阴,大良。

《短剧方》治产后阴脱方∶以铁精敷上多少,令调,以火炙布,令暖,熨肛上渐纳之。

又方∶用鳖血,烧地令热,血着上,使病患坐之良。

《千金方》治产后阴下脱方∶熬锻石,绵裹,坐其上。冷即易之。

又方∶灸脐下横纹中二七壮。

《极要方》治产后阴脱方∶硫黄(二分)乌贼鱼骨(三分)五味子(三铢)为散,粉阴上,日三。

《广济方》疗产后子脏挺出数寸痛方∶蛇床子(一升)酢梅(二七枚)切,以水五升,煮取二升半,洗,日夜十度。
治产后阴肿方第四十一

《病源论》云∶产气宿虚,因产风邪乘于阴,邪与血气相搏;在其腠理,故令痛,血气为邪所壅否,故肿也。

《录验方》治产后阴肿痛方∶取鼠壤四升,熬令热,以囊贮置阴上,使热气入中,良。

《千金方》治产后阴肿痛方∶熟捣桃仁敷之。
治产后阴痒方第四十二

《产经》治产后阴中如虫行痒方∶枸杞(一斤)以水三斗,煮十沸,适寒温,洗之,良。

又方∶煮桃叶若皮洗之。

又方∶烧杏仁作灰,绵裹纳阴中,良。
治产后小便数方第四十三

《病源论》云∶胞内宿有冷。因产气虚而冷发动,冷热入胞,虚弱不能制其小便,故令数也。

《短剧方》治产后小便数方∶取衣书中白鱼虫三十枚末之,以绵裹纳阴中,良。

《广济方》疗产后小便不禁方∶取鸡毛烧作灰,酒服方寸匕,日三。
治产后遗尿方第四十四

《病源论》云∶因产用气,伤于膀胱,而冷气入胞囊,胞囊决漏,不禁小便;故遗尿多因产难所为也。

《短剧方》治产后遗溺不知出时方∶白薇(二分)夕药(二分)凡二物,捣筛,酒服方寸匕,日三。

又方∶取胡燕巢中草烧末,服半钱匕,水酒无在。亦治男子。

又方∶取矾石、牡蛎分等,下筛,酒服方寸匕,日三。

《产经》∶治产后遗溺方∶龙骨末,以酒服方寸匕,日三。

又方∶夕药末,以酒服方寸匕,日二夜一。
治产后淋病方第四十五

《病源论》云∶因产虚损,而热气客胞内,虚则起数,热则溲少,故成淋。

《子母秘录》云∶产后淋方∶滑石(五分)通草车前子葵子(各四分)上四味,捣筛,以醋浆服方寸匕。
治产后尿血方第四十六

《病源论》云∶产伤损血气,血气则虚而挟于热,热搏于血,血得热流散渗于胞,故血随尿而出,为尿血也。

《产经》云∶治产后溲有血不尽,已服朴硝煎宜服此蒲黄散方∶蒲黄(一升)生蓟叶(曝令干,成末,二升)凡二物,冶,下筛,酒服方寸匕,日三。
治产后下利方第四十七

《病源论》云∶产后虚损未复而早起,伤于风冷,风冷乘虚入于大肠,肠虚则泄,故令利也。产后利,若变为血利则难治。

《产经》云∶理中汤主之∶干姜人参白术甘草(各二两)以水六升,煮取三升,分三服。

又方∶药各一两,水三升,煮取一升半,分二服。

《极要方》云∶产后诸痢方∶宜煮薤白食之,唯多为好。(今按∶《子母秘录》云∶许仁则方。)《医门方》疗产后利不禁止,困乏气欲绝,无问赤白水谷方∶黄连浓朴(各三两)艾叶黄柏(各二两)水六升,煮取二升,去滓,分二服。

《子母秘录》云∶产后痢,诸病无不效方∶黄连(一升)乌梅肉(三两,擘)干姜(二两)上三物,捣筛,蜜丸如梧子,一服二十丸。
治产后月水不调方第四十八

《病源论》云∶产伤动血气,虚损未复而风邪冷热之气客于经络,乍冷乍热,冷则血结,热则血消。故令血或多或少,乍在月前,或在月后,为不调也。

《子母秘录》云∶产后月水闭,乍在月前,或在月后,腰腹痛,手足烦疼,唇口干,连年月水不通,血干着脊,牡丹丸方∶苦参(十分)牡丹(五分)贝母(三分)上三物,捣筛,蜜丸如梧子,先食以粥清汁,服七丸,日三。
治产后月水不通方第四十九

《病源论》云∶产后虚损未复,为风冷所伤。故令月水不通。

《葛氏方》云∶产后月水不通方∶桂心为末,酒服方寸匕。

又方∶铁杵锤烧,纳酒中,服之。

《子母秘录》云∶产后月事不通方∶浓朴皮三大两,以水三大升,煮取一升,分三服。空腹服之,神验。
治产后生疮方第五十

《录验方》治产后匝身生疮,状如灼疮,热如火方∶桃仁,捣,和以猪膏,敷疮上,日二三过,便愈。

医心方卷二十三医心方卷二十三背记右KTKT(左右符俱以朱书之。)东借十步西借十步南借十步北借十步上借十步下借十步壁方之中四十余步产妇借地恐有秽污或有东海神王北壁或有西海神王或有南海神王或有北海神王或有日游将军白虎夫人横去十文轩辕招摇举高十丈天狗地轴入地十丈急急如律令以上第七叶
卷第二十四
治无子法第一

《病源论》云∶妇人无子,其事有三也。一者坟墓不嗣,二者夫妇年命相克,三者夫病妇疾,皆使无子。其若是坟墓不嗣,年命相克,此二者,非药能益。若夫病妇疾,须将饵,故得有效也。然妇人挟无子,皆由劳伤血气,冷热不调。而受风寒,客于子宫,致胞内生疾。或月经涩闭,或崩内带下,致阴阳之气不和,经血之行乖候,故无子也。

又云∶男人无子者,其精清如水,冷如铁,皆无子。

又云∶泻精,精不射出,俱聚在阴头,亦无子也。

《千金方》云∶凡人无子,当夫妻俱有五劳七伤所致。治之法,男服七子散,女服紫石门冬丸。

七子散方∶五味(八分)牡荆子(八分)菟丝子(八分,渍酒三宿)车前子(八分)菥子(八分)薯蓣(八分)石斛(八分)干地黄(八分)杜仲(八分)鹿茸(八分)远志(八分)附子(六分,炮)蛇床子(六分)芎(六分)山茱萸(五分)天雄(五分,炮)人参(五分)茯苓(五分)黄(五分)牛膝(五分)桂心(十分)巴戟天(三两)苁蓉(七分)钟乳(二两)凡二十四味,酒服方寸匕,日二,以知为度,禁如法。不能酒者,蜜丸服。

紫石门冬丸方∶紫石英(三两,七日研之,少得上浮即熟)天门冬(三两)当归(八分)芎(八分)紫葳(八分)卷柏(八分)桂心(八分)乌头(八分)牡蒙(八分)干地黄(八分)石斛(八分)禹余粮(八分)辛夷心(八分)人参(五分)寄生(五分)续断(五分)细辛(五分)浓朴(五分)干姜(五分)食茱萸(五分)牡丹(五分)牛膝(五分)薯蓣(六分)乌贼骨(六分)甘草(六分)柏仁(四分)凡二十六味,捣筛,蜜和,酒服如梧子十丸,日三,渐增至三十丸,以腹中热为度,禁如药法。比来服者,皆不至尽剂即有身。

《僧深方》∶庆云散治大(丈)夫阳气不足,不能施化,施化无所成方∶天门冬(九两,去心)菟丝子(一升)桑上寄生(四两)紫石英(二两)覆盆子(一升)五味子(一升)天雄(一两,炮)石斛(三两)术(三两,熬,令反色)素不耐冷者,去寄生,加细辛四两。

凡九物,冶合,下筛,以酒服方寸匕,先食日三。阳气少而无子者,去石斛,加槟榔十五枚。

承泽丸治妇人下焦三十六疾不孕育及绝产方∶梅核(一升)辛夷(一升)本(一两)泽兰(十五合)溲疏(一两)葛上亭长(七枚)凡六物,冶,下筛,和以蜜丸如蜱豆,先食服二丸,日三,不知稍增。

《极要方》疗无子不受精,精入即出,此子门闭也∶山茱萸(一两)酸枣(二两)柏子仁(二两)五味子(二两)上,下筛,以好淳酒,丸如麻子,先食吞下二丸。颖川都尉张君夫人年四十八无子,服此药即生二男。药无禁。

《葛氏方》治妇人不生子方∶以戊子日令妇人敞胫卧上西北首,交接。五月、七月、庚子、壬子日尤佳。

又方∶桃花未舒者,阴干百日,捣末,以戊子日三指撮,酒服。

《耆婆方》云∶常以四月八日,二月八日奉佛香花,令人多子孙,无病。

《新录方》云∶正月始雨水,男女各饮一坏(杯),有子。

又方∶常以戊子日日中时合阴阳,解发振,立得。

又方∶灸中极穴,在脐下四寸。

《录验方》云∶治妇人无子方∶柏子仁(一升)茯苓末(一升)捣,合乳汁和服,如梧子十丸。(《葛氏方》同之。)《枕中方》云∶欲得生子日,子日正午时,面向南卧,合阴阳,(即)有验。

又云∶老子曰∶取井中虾蟆着户上,生子必贵。

《玉房秘诀》云∶治妇人无子,令妇人左手持小豆二七枚,右手扶男子阴头纳女阴中,左手纳豆着口中,女自男阴同入,闻男阴精下,女仍当咽豆。有效万全,不失一也。女人自闻知男人精出,不得失时候。

又云∶妇人怀子未满三月,以戊子取男子冠缨烧之,以取灰,以酒尽服之,生子富贵明达。秘之秘之。

《本草拾遗》云∶夫尿处土令有子。壬子日,妇人取少许,水和服之,是日就房,即有娠也。

又云∶正月十五日灯盏令人有子,夫妻共于灯下盗取,置卧床下,勿令人知,当此月有娠。
知有子法第二

《病源论》云∶阴搏阳别谓之有子者,此是气血和调,阳施阴化也。诊其手少阴脉动甚者,妊子也。尺中之脉,按之不绝者,妊娠也。

《产经》云∶凡妇人三部脉,浮沉正等者,此谓有子也。(今按∶《八十一难》云∶从掌后三寸为三部,则寸与关尺各得之寸一。凡诊脉者,先明三部九候。)《短剧方》云∶凡妇人虚羸,血气不足,肾气少弱;或当风取冷大过,心下有痰水者,欲有胎,便喜病阻。何谓欲有胎,其人月水尚来,颜色肌肤如常而沉重愦闷。不用饮食,不知其患所在,脉理顺时平和。则是欲有胎也。如此经二月日后,便觉不通,即结胎也。(《玉房秘诀》云∶初施泻妇阴阳有力如吮者,是有子之候也。)《太素经》云∶玄元皇帝曰∶人受天地之气,变化而生,一月而高(膏),二月而脉,三月而胞,四月而胎,五月而脉筋,六月而骨,七月而成形,八月而动,九月而臊,十月而生。

《周书》云∶人感十而生,天五行,地五行,合为十也。天五行为五常,地五行为五脏,故《易》曰∶在天成象,在地成形者也。

《家语》云∶天一,地二,人三,三三而九,九九八十一,一主日,日数十,故十月生。

《玉房秘诀》云∶阳精多则生男,阴精多则生女,阳精为骨,阴精为肉。
知胎中男女法第三

《病源论》云∶脉左手沉实为男,右手浮大为女;左右俱沉实,生二男。左右手俱浮大,生二女。又云∶遣向南行,还复呼之,左回是男,右回是女。又云∶其夫左乳房有核是男,右乳房有核是女也。

《千金方》云∶左手尺脉浮大者为男,右手尺脉沉细者为女。

《产经》云∶以脉知胎男女法∶妊身妇人,三月尺脉数也,左手尺脉偏大为男,右手尺脉偏大为女,俱大有两子。

又云∶妊身脉,左疾为男,右疾为女,左右俱大有两子。

又云∶占孕男女法∶说云∶以传送加夫本命,见妇游年上,得阳神为男,得阴神为女。

一云∶天罡天后加母年上,或酉临阳辰,或功曹临阳;或干有气,或时与日比,或阳神临日者,必为男;或功曹临阴辰,支有气,皆为女。

一云∶用得青龙太裳,子多为男。或得天后太阳(阴),子多为女。

一云∶常以传送加妇人本命,年在阳神下为男;年在阴神下为女。

一云∶微明加四孟为男,神后加四仲为女。

一云∶母行年临孟为男,临四仲季为女。

一云∶腾蛇、朱雀、青龙、勾陈、玄武、白虎,加日辰皆为男。六合、天官(空)、大阴、天后、大裳,加日辰皆为女。

一云∶直用,神在阳似父,在阴似母。或旺相者,美容;囚休者,丑鄙。

又云∶以母年立知胎子男女法∶女年十三立申生男,年十四立未生女,年十五立午生女,年十六立巳生男,年十七立辰生男立亥生男,年二十三立戌生男,年二十四立酉生男,年二十五立申生男,年二十六立未生女,年二十七立午生男,年二十八立巳生男,年二十九立辰生男,年三十立卯生女,年三十一立寅生男,年三十二立丑生男,年三十三立子生女,年三十四立亥生男,年三十五立戌生男,年三十六立酉生男,年三十七立申生男,年三十八立未生女,年三十九立午生女,年四十立巳生男,年四十一立辰生女,年四十二立卯生女,年四十三立寅生男,年四十四立丑生男,年四十五立子生女,年四十六立亥生男,年四十七立戌生男,年四十八立酉生女,年四十九立申生男,年五十立未生男。

又云∶欲知男女算法,先下夫年,次下妇年,仍下胎月,正月胎下算十二月,并取十二月算合数。仍除天一,又除地二,又除人三,又除四时四,又除五行五,又除六律六,又除七星七,又除八风八,又除九章九,单即男,偶即女,万无参差。
变女为男法第四

《病源论》云∶阴阳和调,二气相感,阳施阴化,是以有娠,而三阴所会,则多生女。

妊娠二月,名曰始脏,精成为胞裹,至于三月,名曰始胎,血不流滚,象形而变,未有定仪,见物而化。是时,男女未分,故未满三月者,可服药方术,转之令生男也。

《产经》云∶伊芳尹曰∶盖贤母妊身当静;安居修德,不常见凶恶之事。宜弄文武兵器,掺弓矢,射雄雉,观牡虎,走马犬,生子必为男也。

又法∶妊身三月,取杨柳东向枝三寸,系着衣带不失,子为男。

又法∶妊身三月,取五茄置床下,无令母知,子为男。

又法∶始觉有胎,服原蚕矢一枚,勿令母知之。(今按∶《千金方》∶以井花水服,日三,必得男。)又法∶取石南草四株着下,勿令知之,必得男。

又法∶桑螵蛸十四枚,末,以酒服之,若无者,随多少必得。

《葛氏方》云∶觉有妊三月,溺雄鸡浴处。

又方∶密以大刀置卧席下。(《如意方》同之。)又方∶新生男儿脐,阴干百日,烧,以酒服之。

《集验方》云∶取弓弦一枚,绛囊盛,戴妇人左臂。

《千金方》云∶取雄黄一两,绛囊盛带之∶《录验方》治但生女无男,此大夫病,非妇人过。马齿散方∶马齿(二分,熬)菟丝子(一分)凡二物,用驳马齿冶合,下筛,先食服方寸匕,日三,用井花水服之。

《枕中方》治妇人欲得转女为男法∶有身二月中,灸脐下三壮即有男。(今按∶《产经》云∶初觉时,灸脐中。)又方∶妊身三月求男,取夫衣带三寸烧作灰,井花水二升,东南向服,大良。

《灵奇方》云∶未满三月,取斧着妇人床下,即反成男。(今按∶《如意方》云∶试着鸡窟下,皆雄。)《如意方》食宜男草花即生男,一云∶妊身时带之即生男。(今案∶《本草稽疑》云∶萱草,一名宜男草。《博物志》云∶怀妊妇人佩之即生男。字本医家加之。)又方∶用乌鸡左翼毛二十枚,置女人下即男。

又方∶取雄鸭翅毛二枚,着妇人卧蒋下,勿令知。
相子生年寿法第五

《产经》云∶甲子年生,寿九十,食麦。(一云勇贵。)乙丑年生,寿九十六,食粟。(一云勇而苦。)丙寅年生,寿九十五,食稻。(一云无咎。)丁卯年生,寿八十五,食麦。(一云无咎。)戊辰年生,寿九十二,食豆。己巳年生,寿九十二,食麻。庚午年生,寿九十二,食麦。辛未年生,寿九十二,食豆。壬申年生,寿九十五,食麻。癸酉年生,寿九十五,食麻。甲戌年生,寿九十,食麻。乙亥年生,寿八十三,食麻。丙子年生,寿六十三,食麻。

丁丑年生,寿八十五,食粟。戊寅年生,寿九十二,食豆。己卯年生,寿九十五,食麦。庚辰年生,寿八十三,食麻。辛巳年生,寿八十七,食麦。壬午年生,寿八十五,食豆。癸未年生,寿九十五,食豆。甲申年生,寿八十五,食麻。乙酉年生,寿九十五,食麦。丙戌年生,寿九十三,食粟。丁亥年生,寿百五,食粟。戊子年生,寿百,食豆。己丑年生,寿九十,食粟。庚寅年生,寿九十,食麻。辛卯年生,寿九十八,食麦。壬辰年生,寿八十五,食豆。癸巳年生,寿六十七,食豆。甲午年生,寿八十五,食豆。乙未年生,寿九十,食豆。

丙申年生,寿百,食麻。丁酉年生,寿八十三,食麦。戊戌年生,寿八十四,食粟。己亥年生,寿八十七,食粟。庚子年生,寿八十,粟。辛丑年生,寿八十五,食麦。壬寅年生,寿八十九,食(子曳反)似黍不粘也。癸卯年生寿八十,食麦。甲辰年生,寿九十二,食豆。乙巳年生寿,九十二,食豆。丙午年生,寿八十五,食豆。丁未年生,寿九十五,食豆。戊申年生,寿八十,食粟。己酉年生,寿八十三,食麦。庚戌年生,寿八十五,食稻。

辛亥年生,寿九十三,食粟。壬子年生,寿八十三,食麻。癸丑年生,寿九十五,食粟。甲寅年生,寿八十五,食麦。乙卯年生,寿九十五,食麦。丙辰年生,寿九十二,食豆。丁巳年生,寿八十四,食。戊午年生,寿八十一,食麻。己未年生,寿八十三,食豆。庚申年生,寿九十三,食麦。辛酉年生,寿八十五,食豆。壬戌年生,寿八十六,食麦。癸亥年生,寿七十九,食。
相子生月法第六

《产经》云∶正月生男,妨兄弟,女儿吉。二月生男贵,妨公母。(字曰可安都则无咎,女为异名为候,吉。)三月生男贵,有官,女贫无子。四月生男临民,女为贵人妇。五月生男不寿,女贫三嫁。六月生男二千石,女富贵。七月生男宜仕官,三娶,女小贵三嫁。八月生男不利官,女为贱。九月生男贵当为师,女小贵三嫁。十月生男宜为吏,女贵宜财。十一月生男有官秩,女为贵。十二月生男宜行禄,女得子力。
相子生六甲日法第七

《产经》云∶甲子生,人勇而贵。乙丑生,勇而苦。丙寅丁卯生,无咎。戌辰己巳庚午生,贱。辛未壬申癸酉生,贱。甲戌乙亥生,贱。丙子丁丑生,贱。戊寅己卯生,苦。庚辰辛巳生,贱。壬午癸未生,贱。甲申乙酉生,宜为后。丙戌丁亥生,贱。戊子己丑生,多忧。

庚寅辛卯生,勇。壬辰癸巳生,贵。甲午乙未生,多忧。丙申丁酉生,多病。戊戌己亥生,少兄弟。庚子辛丑生,无勇,人而不利。壬寅癸卯生,贵。甲辰乙巳生,人善。丙午丁未生,人善。戊申己酉生,头不久,庚戌辛亥生,贱。壬子癸丑生,贱。甲寅乙卯生,人勇。丙辰丁巳生,暴贵。戊午己未生,思之。庚申辛酉生,勇愁。壬戌癸亥生,困贱。
相子男生日法第八

《产经》云∶子日生男子,三日三月不死,乐,年至七十二甲子死,属桑木。一云∶九年不死为人君,终利父母。

丑日生男,四日五月不死,贵,年至六十六死,属桑木。一云∶富贵,利父母,恶腹。

寅日生男,五日四月不死,当富,年至六十七死。属松木。一云∶心勇悍,好田宅畜养,不利父母,心自如。

卯日生男,六日二月不死,当贫,年至八十死,属杨木。一云∶三娶,好腹,正广不宜道。

辰日生男,七日三月不死,当多病,年至七十三死,属杨木。一云∶心为贫。

巳日生男,一日二月不死,当拾年至六十六死,属荡木。一云∶富贵二妻。

午日生男,七日三月不死,当两娶妇,年至六十九死,属桂木。一云∶妨父母,富马。

未日生男,三日二十一日不死,当官,年至八十五死,属桃木。一云∶为勇。

申日生男,二日二十二日不死,当为吏,年至五十一死,属棠木。一云∶数多病。

酉日生男,六日二月不死,当恐狂,年至六十六死,属杜檀木。一云∶多病。

戌日生男,一日三月不死,当喜争,年至七十二死,属青榆木。一云∶为富。

亥日生男,三日四月不死,当昌乐,年至六十五死,属黄榆木。一云∶好田宅。
相子女生日法第九

《产经》云∶子日生女,十日三月不死,当再嫁,年至六十五死,属榆木。一云∶心恶。

丑日生女,三日一月不死,娶为兵家作嫁,年至六十七死,属杏仁。一云∶为巫市买,多辨,二夫。

寅日生女,四日七月不死,当三嫁,年至六十死,属杨木。一云∶恶,好巫。

卯日生女,三月不死,当娶智在家,年至六十三死,属折木。一云∶小财,贫。

辰日生女,三日一月不死,当为王侯后,年至七十一死,属桃木。一云先贫后富。

巳日生女,一日半不死,当贵相,年至八十九,属青榆木。一云∶为巫,多口舌午日生女,三日六月不死,当富,年至七十七死,属杜榆木。

未日生女,五日三月不死,当事一君,至七十四死,属相信木。一云∶为国王妻,贵吉。

申日生女,七日六月不死,当富,年至五十四死,属桑木。一云∶贫,无财多病。

酉日生女,一日五月不死,当资,年至七十八死,属相杨木。一云∶贫,无财亦多病。

戌日生女,二日五月不死,当九嫁,年至六十七死,属杜析木。一云∶喜游,病癃。

亥日生女,三日五月不死,当富,年至六十四死,属落木。一云∶巧,当富,三夫,多病。

凡五月丙午日生男,七年无父,母无母,七月丙辰日生,男胜父,女胜母。
相子生时法第十

《产经》云∶夜半生子,男富女强,鸡鸣生子,男宜为吏。平旦生子,男女皆富。日出生子,富乐保财,有威名。食时生子,见苦多贫。一云∶大富吉。禺中生子,男贵女吉。日中生子,秩二千石,女富。日秩()生子,男贵女富,大吉。脯时生子,宜贾市,吉。日入生子,多病,贫苦。(黄昏可寻之),人定生子,苦相,贫。一云∶多恶奸。
相子生属月宿法第十一

《产经》云∶角生子宜兵,善腹,不为人下,身长,好隐潜,至二千石。一云∶可以远行拜吏,生子卿相,祠祀皆吉,不可登埋屋。

亢生子善心,外出道死,不归。一云∶生子为卿,徙移贾市,作门户,大吉。

氐生子贞信,良腹,好田蚕,男至二千石,吉。一云∶入官移徙,远行造举百事,大吉。

房生子反急,腹无治切KT。一云∶富贵,乘车马出入,皆大吉。

心生子忠信,良腹,圣教贤明,二千石。一云∶纳财,见贵人,通言语,学书,使行通水除道,大吉。

尾生子辱不祥,即枉(妊)远之他(邦)。一云∶可以纳财,不可祠祀,造举百事,皆大吉。

箕生子多口舌,不祥,不死其故乡。一云∶不可移徙,嫁娶入官皆不可,纳财奴婢,逃亡也。

斗生子屡被悬官,多疾病,破亡。一云∶生贵子,不可纳财,奴婢亦多死,凶。

牛生子质保不祥,盖亡行。一云∶吉。可纳财物入官,不可纳牛。

女生子宜田蚕,忠孝,良腹,吉昌。一云∶可以入室姑市,不可嫁娶,子必贾。

虚生子家盖亡,惊走他乡,不宜六畜。一云∶不可以移徙,入官,嫁娶,皆不吉也,造举百事,大凶。

危生子贫,远行,不宜财,死亡。一云∶不可入官,移徙,嫁娶,皆不吉也。

室生子富贵,子孙番昌。一云∶百事小吉,久不可为,室舍凶,出行必死亡也。

璧生子良腹,工巧,不死挟贫。一云∶可以移徙,入官、盖屋、出行皆吉,不可祠祀,凶也。

奎生子为奴婢,善辱,不祥,妇女牛奔奔,男可凶。一云∶出行筑室,不可嫁娶,生子为奴婢也。

娄生子备守家居,富贵吉昌。一云∶可以起土贾市,纳六畜鱼腊,吉。

胃生子长腹,八月以后多忧,不祥,信贞。一云∶可以出行,作利合众,入新舍,纳奴婢财物作仓,吉也。

昴生子工巧,先贫后富,大吉。一云∶可以武事断狱决事,饮无所宜,入官有狱事,凶。

毕生子杀佐奸,副鱼腊。一云∶不可嫁娶,病死亡也。

觜生子喜夜行,不祥,盗贼。一云∶可以出室,财分异,不可嫁娶,凶也。

参生子好盗持兵,相伤轻(《玉》云∶音吴,福也),死亡保首市。一云∶可以追捕,代政入官亲事,吉。纳奴婢,教公子,生子市死,凶。

井生子必掠死、溺水死,他身不葬。一云∶不可移徙,入官行作,凶。生子逢残病也。

鬼生子好事神明,至奸野狼鬼守腹死亡。一云∶可以立神祠为主,吉,生子为鬼所着也。

柳生子簪,远行他游则死亡。一云∶贾市百事,吉,不可壅水渎,凶。

星生子编泄汗伤,好喜远行,善禄,乐及后世。一云∶可以移徙,入官,市贾,富三世,葬埋六人死也。

张生子吉昌,身体无咎,富贵。一云∶可以移徙,嫁娶,贾市,百事皆吉。

翼生子一南一北,身在他邦,心中KTKT腹如刺棘。一云∶造举百事,皆吉。

轸生子男女富贵,宜子孙,位至侯王,二千石。一云∶入官祠祀,乘车,吉也。

月宿天仓天府生子大吉利,富贵及后世,福禄巍巍。

凡生子之时见日月之光清明者,贤明多所通远。不见三光阴两者,则愚钝无所通。暴风者,多伤害不祥。晴而有五色云者,有大圣德。有白云蕴者,富多财。
生子二十八宿星相法第十二

《产经》云∶佛家《大集经》曰∶东方一角生者。口舌、四指、额身右多黑子者,贵,聪智,年八十二。

亢生,心乐法音,聪明富贵,多有惭愧,乐出家,年六十。

氐生,人爱,身勇健,富贵,二十五,右黑子于父母,恶心灭家。

房生,性KT恶无知,右边有黑子二十五兵死,宜兄弟。

心生,富贵多才,废(方呔反,痼病也)风病世头疮,大多毒不伤。

尾生,相姓雄庄富贵,自在轮相,大名先(光)明胜日月,大智。

箕生,语诤讼犯,或性KT恶欲盛,六十资困好行。

南方一井生,多才人,敬乐法脐疮般,八十,孝父母,先父已,已里水。

鬼生,短命,脐下黑子四指,不宜父母,诤讼。

柳生,富贵持戒乐法,七十五,眷属生天子,人伏信。

星生,好却盗奸,绝短命兼,发兵死。

张生,命八十,音乐山川,二十七,三十二,富贵健聪明,不宜亲。

翼生,善知算数,怪KT恶性,钝根邪见赤子三十世天子。

轸生,富贵多眷属奴仆,聪明受法命一生天。

西方一奎生,两颊有黑子,持戒乐法富贵施,身疮五十。

娄生短命,犯戒怪膝疮世,不宜兄。

胃生,不宜父母,失才,膝有黑子二十二富贵施。

昴生,乐法戏弁,聪明富贵,多称护戒人敬死生天,膝青子五十。

毕生,人信忍性语暗欲心,姊妹富贵多死,右有黑子七十。

觜生,富贵施,惭愧无病喜见,年七十七,八十七。

参生,性弊作恶业,狱病多欲听明,贫,年六十五,多黑子。

北方一斗生,受性痴悟不知是(足),贫穷恶性短命,病食故。

牛生,痴贫,乐偷窃,多疾忌,年七十,无妻子。

女生,持戒乐施,足有黑子,年八十,名声宜父母兄弟。

虚生,福俭富贵,眷属受乐怪不施,年六十,足下有黑子。

危生,身无病,聪明持戒,勇健富贵,年八十。

室生,受性弊恶,多犯禁戒富贵,年百岁,不宜父母。

辟生,母雄多力尊犯禁,富贵,不宜父母也。
为生子求月宿法第十三

《产经》云∶《湛(堪)余经》曰∶正月朔一营室,二月朔一日奎,三月朔一日胃,四月朔一日毕,五月朔一日井,六月朔一日柳,七月朔一日翼,八月朔一日角,九月朔一日氐,十月朔一日心,十一月朔一日斗,十二月朔一日女。

右件十二月,各从月朔起,数至月尽三十日止,视其日数则命月宿。假令正月七日所生人者,正月一日为室,二日为辟,三日为奎,四日为娄,五日为胃,六日为昴,七日为毕。

正月七日,月宿为在毕星也。又假令六月三日所生儿者,六月朔一日为柳,二日为星,三日为张,张即是其宿也。他皆仿此。
相子生属七星图第十四

\r图\pyxfc11.bmp\r《产经》云∶太岁在午生,属破军星,其为人有威,将众人之主,为人师,众人归之。

富贵秩万石,无忧患,寿九十九岁。

太岁在已未生,属武曲星,其为人强肠自用;有武力,宜为吏,生乐秩千石,无忧患,寿八十八岁。

太岁在辰申生,属廉贞星,其为人小心,有诚信,不勇士,宜为吏。苦贫,少赀财,寿七十七岁。

太岁在卯酉生,属文曲星,其为人好文墨,便习事,小心敕慎,宜为吏,秩六百为石,劳忧,寿六十六。

太岁在寅戌生,属禄存星,其为人多护,杀人不死,伤人不论,人欲谋之,反受其殃,秩二千石,寿七十七岁。

太岁在丑亥生,属巨门星,其为人勇悍强梁,为众人师,宜为吏,秩六百石,无忧患,多智辨,圣寿八十八岁。

太岁在子生,属贪野狼星,其为人贪财,疆肠自用,宜为吏,富贵,秩二千石,无忧患,寿百岁。
相子生命属十二星法第十五

《产经》云∶命在子,名贪野狼星,悬命皂糸,寿百一岁。忌己卯,护命者成宣子,树为柏,为人武,或有方略,胜KT太穷。

命在丑,名传说星,悬命黄糸,寿百五岁,忌甲戌,护命者王衣冠文物,树为直,为人廉平,难得成善。所治主乐。

命在寅,名岁星,悬命割刚,寿八十六岁,忌辛巳,护命者曲恶害,树为杨,为人仁义,多悲肠,不贞,富。

命在卯,名辰星,悬命素糸,寿八十五岁,忌庚子,护命者天屏星,树为榆,为人多知,意常好人。

命在辰,名大微星,悬命毛绳,寿九十三岁,忌甲戌,护命者国大刚,树为桑,为人道理微刚伤,不好负人。

命在巳,名荧惑星,悬命絮素,寿七十二岁,忌壬申,护命者文成衡,树为李,为人晓文理,好君子,后富贵。

命在午,名金雷星,悬命绛糸,寿九十二岁,忌壬子,护命者犯狐横,树为桑,人为伉直,不好独食,常得人力。

命在未,名轩辕星,悬命柔绳,寿百岁,忌乙丑,护命者念内张,树为桂,为人廉平,好布施,有人义胜。

命在申,名天心星,悬命坚芒,寿八十五岁,忌丙寅,护命者石明长,树为檀,为人咀语独诤,不好负人。

命在酉,名大伯星,悬命白糸,寿九十三岁,忌丁酉,护命者民固明,树为梓,为人慈爱父母,习文理贵。

命在戌,名远斗星,悬名筋缕,寿八十五岁,忌丁乙未,护命者改章,树为杜,为人不负人,独怒富贵。

命在亥,名星,悬命廉KT素,寿七十八岁,忌己亥,护命者伏河王,树为斛粟,为人人且义,无取欲,有后。

命所属星为苦乐,官悬命,寿忌日不举百事,护人命也。欲无忧患害,常怀生日KT行。若猝有患亡命,疾病有厄,辄披发左祖,禹步三仰,呼所属星名,曰∶某甲未护无思,勿令恶贼伤我,勿令邪鬼魅鬼来病我。所愿,愿皆得愿,愿成皆无不得也。
相生子属七神图第十六

\r图\pyxfc12.bmp\r《产经》云∶以青龙日生者,至二千石;以朱雀日生者,至六百石,持节;以左将日生,至四百石,内侍爱;以右将日生,至四百石,内侍爱;以句陈日生,至封侯;以玄武星至六百石,为人邪行;以白虎日生,至二千石,为人KT子。以此七神日生贵重,王相日生贵,月建日生亦贵重,吉也。
相子生四神日法第十七

《产经》云∶月一日、九日、二十九(五)日、十七日者,朱雀日也,生子妨父母,多病。

月二日、十日、二十六日、十八日者,白虎头日也,生子不孝。

月三日、十一日、二十七日、十九日者,白虎胁日也,生子吉,贵至二千石。

月四日、十二日、二十八日、二十日者,白虎足日也,生子亡财,失火。

月五日、十三日、二十一日、二十九日者,玄武日也,生子有忧,不寿。

月六日、十四日、二十二日、三十日者,青龙日也,生子亡身,三十三年死。

月七日、十五日、二十三日者,青龙胁日也。生子贵。

月八日、十六日、二十四日者,青龙足日也,生子失火,亡财。
禹相子生日法第十八

《产经》云∶乳母问禹,生男女日,善恶何?禹对曰∶凡入月一日、十一日、二十一日生子多勇,利父母。入月二日、十二日、二十二日生子俊,多勇,利父母。入月三日、十三日、二十三日生子多病疾;入月四日、十四、二十四日生子利父母;入月五日、十五日、二十五日生子父母不得力;入月六日、十六日、二十六日生子早得力,利父母;入月七日、十七日、二十七日生子便父母;入月八日、十八日、二十八日生子不全;入月九日、十九日、二十九日生子皆吉;入月十日、二十日、三十日生子俊多,父母得力。
相子生五行用事日法第十九

《产经》云∶木用事甲乙日生上寿,丙丁日生中寿,戊己日生死夭,庚辛日生不寿,壬癸日生下寿。

火用事丙丁日生上寿,戊己日生中寿,庚辛日生死夭,壬癸日生不寿,甲乙日生下寿。

土用事戊己日生上寿,庚辛日生中寿,壬癸日生死夭,甲乙日生不寿,丙丁日生下寿。

金用事庚辛日生上寿,壬癸日生中寿,甲乙日生死夭,丙丁日生不寿,戊己日生下寿。

水用事壬癸日生上寿,甲乙日生中寿,丙丁日生死夭,戊己日生短寿,庚辛日生小寿。
相子生五行用事时法第二十

《产经》云∶木用事木时生贵,火时生富,土时生死绝伤亡,金时生贫贱苦厄,水时生心有贵子。

火用事火时生贵,土是生富,金时生绝伤亡,水时生贫贱多危,木时生有贵子。

土用事土时生贵,金时生富,水时生绝伤亡,木时生贫贱苦厄,火时生有贵子。

金用事金时生贵,水时生富,木时生绝伤亡,火时生贫贱苦厄,土时生有贵子。

水用事水时生贵,木时生富,火时生绝伤亡,土时生贫贱苦厄,金时生有贵子。
相子生熹母子胜忧时法第二十一

《产经》云∶甲乙加时,丙丁加时,戊己加时,庚辛加时,壬癸加时。

寅卯(熹时)巳午四季申酉亥子亥子(母时)寅卯巳午四季申酉巳午(子时)四季申酉亥子寅卯申酉(胜时)亥子寅卯巳午四季四季(忧时)申酉亥子寅卯巳午子以熹时生富贵,算得千,訾千万,利父母。

子以母时生人爱,保财,孝顺。

子以子时生得算五百,訾千万,利父母。

子以胜时生,强梁辨自用,少财,可使兵事。

子以忧时生,忧苦,少时多患。
相生子死候第二十二

《产经》云∶凡儿生,身不收者死。儿生,鱼口者死,儿生股间无生肉者死,儿生颅破者死,儿生阴不起者死,儿生阴囊白而后孔赤者死;儿生毛发不周者子不成,儿生头四破开亦不成,儿生声四散亦不成。

凡新小儿有此诸相者皆不字长也。

凡诸生子男偃者不利妻,女伏者不列夫。

凡建日生子是谓北斗之子,男女皆不可起,自死。
占推子寿不寿法第二十三

《产经》云∶说曰∶生子男视日上,生子女视辰上,得吉,神良。将有王相立者,又不终始相克,又太岁上神与日辰。上神相生者,则长寿,吉。若不相生者,自如。若如得凶,将神困死,气上下相克者,即不寿。若将遇白虎者,子生便死。若遇朱雀得疾病,若遇腾蛇母惊。

一云∶常以天魁加子本命上,太一从魁下皆为天杀也。在上为天杀,在下为月杀。下生子为鬼吏乃杀。月杀下生子,为人臣贼害。

一云∶以直用神得青龙太常者,富。得太阴者,保家而已。得六合者,常有赏乐。得朱雀者,常遇悬官。得勾陈者,数与人斗诤。得玄武者,数被盗。得腾蛇者,见惊惧,性多悲忧。得天空者,性欺诞。得白虎者,不寿。
占推子与父母保不保法第二十四

《产经》云∶经曰∶四下贱上之时,生男妨父,生女妨母,亡其先人,是孤子。

一云∶子生时不欲克其日辰,日辰克,大凶。以此辨之,此为要诀也。

一云∶《龙花经》曰∶必记初纳妇日,纳以甲乙而庚辛生子,大凶。干伤害父,支伤害母,皆克日辰则俱害。

一云∶凡月杀日生子,不问男女皆妨父母,子不吉。月杀者,丑戌未辰,终而复始。

一云∶以神后加孩生时魁加父母年者,妨害二亲。
占推子祸福法第二十五

《产经》云∶说曰∶日辰上得青龙传送,有王相气者,皆高才多能。

一云∶以魁加子本命,KT加生月者,少孝慎,见功曹传送者悌。

一云∶以魁加本命者生月上,见神后者远行亡命厄,见大吉者自如,见功曹者福德,见大冲者贫贞,见天KT者男贫若虎野狼厄,女忧产死落胎。见太一者多疾病牢狱厄,见胜光者火烧,见小吉者自如,见传送者有福禄,见登(微欤)明者在牢狱厄。
相男子形色吉凶法第二十六

《产经》云∶男子强骨方身,面方平正且眼正,眼不邪见,邪见必有不直之心。行步直迟,行虎步不为人下。口开则大,闭则小。言语迟迟,言时不见前人者,君子之相也。目蒸(亟)动KT盗视,言必望前人之面目者,小人气也。故颈欲如鸿王,身回乃动。因欲如虎视,举头乃见。颊如狮子颊,音如钟鼓铃音者,贤吉也。

《世纪》曰∶凡人身不在吉长短弱也。舜身甚小短,周公身小短,业公又小短,周灵王生而有髯髯口毛也,吉相。武王并齿。(是谓庚强之相也。)其父文王问裴秀曰∶人有相不?裴秀曰∶有中抚军立鬓至地,伸手过膝,非人臣之相也。舜瞳子重,项羽重子,灵帝足下有毛,身短,贵贵相也。汤有四时(肘)口光,左生内币,黄帝广颊龙颜,口兑者,贤武相也。

仲尼隆颊,尧八字之眉也,有慈人之相也。子骨眉间尺一强心之相也。禹虎鼻怀斗,伏羲大目,苍颉四目,皆贤智相也。

颡卿记曰∶老君足下有八卦纹,眉长,耳有三门,鼻有双柱,浓唇,口方,色黄,是贤智相也。
相女子形色吉凶法第二十七

《产经》云∶女子不可娶者,黄发黑齿,息气臭,曲行邪坐,目大雄声,虎颜蛇眼,目多白少黑,淫邪欺夫。黑子在阴上,多淫,及口上,爱他人夫,勿娶。大KT(肱欤)而阴水,甲夹而乳小,手足恶,必贫贱,夫勿娶。

浓皮骨强,色赤如绛,杀夫勿娶。

蛇行雀走,财物无储,勿娶。

小舌烦头,鹅行,欺未夫。口际有寒毛似鬓,身体恒冷,瘦多病者,无肥肉,无润色,臂胫多毛,槌项结喉,鼻高,骨节高颗,心意不和悦,如此之相,皆恶相也,慎勿娶,必欺虚气,夫妨杀夫,贫穷多忧之相也。

女子吉相白齿,目白黑分明,视瞻正直,眼不邪视,声人大,小鼻正如篇;人中深长,气香,眉如八字,面正方平满,口下有黑子,肩上下相齐而不薄;舌广色如绛,有纹理,身皮薄滑润,多肥肉,身体常温,骨弱,节后不头,手足长肥,掌又如乱丝,行走正直,心口和顺,头足平直者,皆贵人相也。又夹KT(肱)而阴大者,阴上高如覆杯,阴毛长而滑细。顺生者,阴有黑子,二千石之妻。乳大小口直,夫乳上下左右黑痣,富相也。

手中有黑痣,又齿三十二以上,最贵相也。

又足下有田井字者,为天下主也。

如此者,大吉祥福德,必可娶之,慎勿放弃之。

医心方卷第二十四医心方卷第二十四背记紫葳(《本草》云∶威灵仙,一名能消。注云∶先于□草茎方数叶相对,花浅紫,何以得知紫葳即能消,是草部文也。又木部云∶紫葳一名,陵苕一名,茇华。苏敬云∶此即凌霄花也。

《尔雅》云∶苕,一名陵茗,葳华云云。是木部葳欤。于威灵仙者,不被甘心之。)
卷第二十五
小儿方例第一

《短剧方》云∶黄帝云∶人年六岁以上为小,十八以上为少,二十以上为壮,五十以上为老也。其六岁以还者,经所不载,是以乳下婴儿病难治者,皆无所承按也。中古有巫立小儿《颅囟经》以占夭寿,判疾病死生,世相传有少小方焉。

今按∶《太素经》云∶小儿初生为婴,能笑为KT脉(儿)。

《经义解》云∶小儿初生号赤子。

《产经》云∶凡儿生当长一尺六寸,重十七斤。

《针灸经》云∶十岁小儿,七十老人不得针,宜灸及甘药。

《千金方》云∶凡新生儿七日以上,周年以还,不过七壮,炷如雀屎大。
小儿新生祝术第二

《产经》云∶凡儿初生时即祝曰∶以天为父,以地为母。颂金钱九十九,令儿寿。

凡小儿初生,仍以发其手掌,曰号理,寿千岁。至二千石乃起之,大吉。若可,当为天子、王候、后妃、卿相者,即随共相号之,乃可起抱之,吉。
小儿去衔血方第三

《产经》云∶儿初生落地,急撩去(儿)口中舌上衔血,即时不去,须臾血凝,(儿惊哭发声,血)吞入(者,)或令儿成腹中百病(也。去衔血法∶以绵缠手指头以拭去之)。

《千金方》云∶儿生口中有血,即当去之,不去者,而得吞之,成痞病。
小儿与甘草汤方第四

《千金方》云∶儿新生出腹,先以指料口中恶血,去之便洗浴,断脐竟KT袍讫。未与朱蜜也。取甘草如手中指一节讦(计),打碎,以水二合,煮取一合;以绵缠沾取,与儿吮之,如朱蜜法,连吮,计可得一蚬壳,入腹止。儿当快吐,吐去胸中恶汁也。吐后消息计如饥渴,项复更与之,若前服。乃更与并不吐者,但稍与尽此一合止,得吐恶汁。令儿心神智惠,无病。都不吐者,是不含恶血耳。勿复与之。(《短剧方》同之。)
小儿与朱蜜方第五

《产经》云∶小儿初生三日,可与朱蜜方,令儿镇精神魂魄。真朱精练,研者如大豆多,以赤蜜一KT壳和之,以绵缠沾取,与儿吮之,得三沾止,一日令尽,此一豆多耳。作三日与之,则用三大豆多也。勿过此量,过则复儿也。

今按∶《短剧方》云∶不宜多,多则令儿脾胃冷,腹胀。
小儿与牛黄方第六

《产经》云∶朱蜜与竟,即可与牛黄,牛黄益肝胆,除热,定惊,辟恶气也。作法如朱蜜,多少一法同也。(《短剧方》同之。)
小儿初与乳方第七

《产经》云∶凡乳儿,母当枕臂与乳头平,当乳,不然则令儿噎。

凡乳儿,当先施去宿乳,以乳儿之,不然令儿吐可利。

凡乳儿,先以手按乳,令散其热,乃乳儿之,若不然,乳汁奔走于儿咽,令儿夺息成疾也。

凡乳儿,母欲寐者,则夺其乳,恐覆儿口鼻。亦不知饱,令致儿困也。

凡乳儿,顿不欲大饱,大饱则令儿吐。若吐,当以空乳,乳之则消。夏不去热乳,以乳,令儿呕逆;冬不去寒乳,令儿咳,下痢。

凡母新饱以乳,令儿喘、热、腹满。

母新内以乳,令儿羸,肢胫不能行,杀儿。

母新醉以乳,令儿身热、腹满、杀儿。

母新怒以乳,令儿喜发气疝病。

母有热以乳,令儿变黄不能食。

母有疾行以乳,令儿病癫狂。

母新吐下以乳,令儿虚羸。

凡儿初生,乳母食猪鸡鲜鱼胞美以乳儿者,令儿伤害洞泄也。

凡乳母过醉及房室喘息乳儿者,此最为剧,能杀儿,宜慎之。

夫五情善恶,七神所禀,无非乳而生化者也。所以乳儿宜能慎之。其乳母黄发、黑齿、目大、声、眼精浊者,多淫邪相也。其椎项节高,鼻长,口大,臂、胫多毛者,心不悦相也。其手足丑恶、皮浓、骨强、齿KT口臭,色赤如绛者,胜男相也。其身体恒冷,无有润泽、皮粗、无肌而瘦者,多病相也。

又有漏腋(漏腋者恒湿臭汗也;)胡臭;癣KT,(KT者,白秃;)易KT;(者,诸节生疮;KTKT者,皮上痱疮;)KT(KT者,声败;)嗽;(者鼻息肉也;)聋;龀(龀,齿败破者;)龋(龋者,虫食齿;)瘰;;瘿;瘤;痔;唇;癫;眩;痫者。是丑疾相也。

又其本命生年与儿无克,如此诸恶相者,便可饮乳。不随此法,害儿,不吉。

凡乳儿顿欲大饱,大饱则令儿吐。
小儿哺谷方第八

《产经》云∶凡小儿生三日后,应开腹助谷神。可研米作浓饮如乳酪状。抄如大豆粒大,与儿咽之,咽三豆许止,日三与之,七日可与哺。十日始哺,如来核许。二十日倍之。五十日如弹丸许,百日如枣许。若乳汁少者,不从此法,当用意少增之。

今按∶《本草》云∶以如梧子十六枚准。(弹丸一枚。)一云∶二十日后可乃始哺,令儿无疾也。若早与哺者,儿头、面、体喜生疮。亦令儿虚羸难长。若儿大小随宜哺增减之良。

凡小儿不嗜哺者,勿强与,强与哺不消,便致疾病也。

又云∶初哺小儿良日∶五寅、五辰、五丑、五酉,皆大吉。

又云∶男以甲乙,女以壬癸,亦吉。

又云∶以成、收、开、定、满日,义日,保日,皆吉,哺儿,儿终身无病,大吉。(义日∶壬申、癸酉;保日∶甲午、乙巳。)又云∶哺小儿忌日∶五戌、五巳、五亥、丁日,大凶。

又∶戊戌、戊辰、执、闭日,皆大凶。
小儿初浴方第九

《产经》云∶小儿初生时洗浴,以牛脂小置汤中,令儿至老无疾。一云∶香脂大如指,投汤中浴之,大佳。香脂是半脂也。

又云∶小儿初生,以虎头骨渍汤中洗浴之,令儿不病。

又云∶小儿初生以洗浴,以金银珍宝珠玉等投汤中,儿必为贵尊,大吉。

又云∶小儿初浴汤法∶桃根、李根、梅根三物,以水煮取汁洗浴儿,却诸不祥,令儿身无疮,大吉。(今按∶《千金方》云∶以猪胆一枚,投汤中浴儿,终身不患疮疥。)又云∶凡小儿浴数,数者,令儿背冷,发痫。若久不洛者,令儿毛落,亦复令啼呼之。

间一二日浴之良。

又云∶浴小儿良日∶丑、寅、卯、申、酉。

又∶甲寅、乙未、丙午、丁酉、癸酉,癸未、甲辰,皆吉。令儿终身无疾病,长寿,大吉。

又云∶浴小儿忌日∶庚戌、壬子、甲乙、庚辛、壬癸、辰巳、午未、亥,大凶。

又∶男忌戌日,女忌丁日,大凶。

又∶平旦、日中、黄昏、夜半,大凶。
小儿断脐方第十

《产经》云∶凡儿断脐法,以铜刀断之,吉(去)脐当令长六,七寸,长则伤肌,短则伤脏。

凡儿初生,当即举之。迟举则令儿寒中,腹中雷鸣。先浴之,然后断脐裹之,吉。

《千金方》云∶断脐当令长至足夫,短则中寒,令腹中不调,当下利。

又云∶裹脐法∶推治帛布令柔软,方四寸,新绵浓半寸,与布等合之,调其缓急,急令儿吐。儿生二十日,乃解视脐。若十日许儿怒啼,似衣中有刺,此或脐燥还刺其腹,当解。

易衣更裹脐时,当闭户下帐,燃火左右,令帐中温暖,儿衣亦令温粉粉之,比谓冬时寒也。

若脐不愈。烧虾蟆令成灰冶末(粉脐中。)
小儿去鹅口方第十一

《产经》云∶凡初生儿,其口中舌上有白物如米屑,名为鹅口,及鼻外亦有。此由儿在胞中之时,其母嗜嚼米,使之然也。此物当时不去之,儿得吞者,化为虫也,宜便去之。治之方∶以发缠钗头,沾井花水撩拭之,三四旦,如此便脱去也。犹不去者,可煮栗蒺汁,令浓,以拭,如上法。若春冬无栗蒺者,可煮栗树皮用如上法,皆良。

一云∶钗头着在者,屠苏水中,勿令儿口中落入吞。(《短剧方》同之。)《爽师方》云∶小儿鹅口方∶(桑白汁和胡粉涂之。)
小儿断连舌方第十二

《产经》云∶儿初生之时,有口中舌下膜如石榴子,中隔者连其舌下。当时不摘断者,后喜令儿言语不发,转舌也。治之方∶可以爪摘断之,微有血出,无害。若血出不止者,可烧发作末,敷之血止,良。(《短剧方》名之“连舌”。)
小儿刺悬痈方第十三

《产经》云∶小儿初生后六七日,其血气收敛成害,则口舌颊里领领净也。若喉里舌上有物如芦箨盛水状者,名悬痈,有气胀起也。又有着舌下如此者,名为重舌。又有上如此者,名为重。又有着齿龈如此者,为重龈,皆刺去血汁之,良。

治奇以绵缠长针,未刃如栗以刺决之,令气泄之,去清黄血汁,良。一刺止之,消息一日,不消又刺之,不过三刺,自消。(《短剧方》同。)
小儿变蒸第十四

《病源论》云∶小儿变蒸者,以长血气也。变者上气,蒸者体热。变蒸有轻重,其轻者体热而微惊。耳冷髋亦冷,上唇头白肉()起如死鱼目珠子。微汗出,平者而歇,远者九日乃歇。(远者八九日乃歇。)其重者,体壮热而脉乱,或汗,或不汗,不欲食,食辄吐,无所苦也。变蒸之时,目白精(睛)微赤,黑精微白,亦无所苦。蒸毕,自明了矣。先变五日,后(蒸)五日,为十日之中,热乃除耳。初变蒸之时,不欲惊动,勿令旁边多人,变蒸或早或晚,依时如法者少。初变之时,或热甚者,或违日数不歇,审计日数,必是变蒸;服黑散发汗,热不止者,服紫(双)丸,小瘥,便止,勿复服之。其变蒸之时,遇寒加之,即寒(热)交争,腹痛夭娇,啼不止,煮熨之则愈。变蒸与温壮伤寒相似,若非变蒸;身热耳热,体髋亦热,此乃为他病,可为余治。审是变蒸,不得为余治也。其变蒸日数(出《病源论》。)又,变蒸者,唇头白肉起,如死鱼目珠子。微汗,平者而歇,远者九日乃歇。

儿生三十二日始变,变者耳热也。至六十四日再变,再变且蒸。其状卧欲端正也。第二变蒸时,或目白者赤,黑者微白,变蒸毕,目便精明矣。至九十六日三变,变者候丹孔出而泄也。至一百二十八日四变,变且蒸,能咳笑也。至一百六十五日五变,以成机关也。至一百九十二日六变,变且蒸,以知五机成也。至二百二十四日七变,以知匍匐也。至二百五十六日八变,变且蒸,以知学语矣。至二百八十八日九变,以亭亭然也。凡九变三蒸也。至三百二十日十变,变且蒸,蒸积三百二十日小蒸毕。后六十四日大蒸,后百二十八日复蒸。积五百七十六日大小蒸毕,乃成人也。所以变蒸者,皆是荣其血脉,攻其五脏,故一变竟辄觉情态有异者也。

凡蒸之候,身壮热、脉乱、汗出、目精不明,微欲惊、不乳哺、上唇头小白肉起如死鱼目珠子,耳令尻赤冷,此其诊也,近者五日歇,远者八九日歇也。当审计其蒸日,不可针灸服药。

凡儿变(蒸)之时,不欲惊动,勿令边多人也。小儿变时,或早或脱也。

凡儿生三月不蒸则耳聋目盲;五月不蒸身不行;九月不蒸五机不成,此为痿蹶不能行之疾也。

《产经》云∶脉决曰∶凡小儿变蒸之时,汗出不用食,食辄吐而脉乱,无所苦也。

《葛氏方》云∶凡小儿生后六十日,目瞳子成,能咳唆,识人。百五日经脉生,能反复。

百八十日尻骨成,能独坐。二百一十日掌骨成,能匍匐。三百日髌骨成,能独倚。三百六十日为一期,膝骨成,乃能行。

治少小初变蒸时,有者服之,发干已止,黑散方∶杏仁(二分)大黄(一分)麻黄(二两,去节)上三物,先捣大黄、麻黄下筛,杏仁令如脂。纳散,令调,更粗筛筛之,盛以苇囊。二十日儿以汁和之,如小豆一丸,分为二丸,易吞,浓衣包之,令汗,汗出毕,下帐、燃火解衣、温粉粉之。百日儿取散如枣核大,以小阳和服之,汗出之后,消息如上法;当豫温粉,不可解衣,乃温粉。(出《葛氏》。)治已服黑散,发热不歇,服之热小瘥便止,勿复与。紫丸方∶赤石脂(一两)巴豆(三十枚)代赭(一两)杏仁(三十枚,一方五十枚,去皮)上四物,先冶巴豆、杏仁,捣二千杵,乃纳代赭、赤石脂,更捣三千杵,拖也。药热即成。一方云∶相和,与少蜜和之,盛以密器,无令药燥,燥则无热。以巴豆、杏仁自丸,常苦不能尽屑,当稍稍纳之令相丸。二十日儿服如黍米一丸讫,小乳小乳之,令药得下。却两食顷,乃复乳之,勿令饱耳。平旦一服药,日中热尽;日西夕时复小增丸,至鸡鸣时,若复与一丸,愈者止。三十日儿服如大黍米一丸。三十日儿服如麻子一丸。六七十日儿如胡豆一丸。百日儿服如小豆一丸。不下故热者,增半丸,以下利为度。

又方说∶服紫丸,当须完出,若不出,出不完,为病未尽,当更服之。有热乃服紫丸,无热但有寒者勤服乳头。单当归散、黄散。变蒸服药后微热者,亦可与除热黄芩汤方。(出《僧深方》。)黄芩汤少小辈变蒸时服,药下后有朝夕热吐利。除热方∶黄芩(一两)甘皮(六铢)人参(一两)干地黄(六铢)甘草(半两,炙)大枣(五枚,去核)凡六物,切之,以水三升,煮取一升,绞去滓。二百日儿服半合,三百日儿服一合,日再。热瘥止。变蒸,儿有微热可服。(出张仲。)又云∶经曰∶天不足西北,故令儿腮后合;地不足东南,故儿髌后生成;人法于三,故令齿后。故腮合乃而言,髌成乃而行。大阴气不足而大阳气有余者,故令儿羸瘦胫三岁乃而行。
小儿择乳母方第十五

(备在第七小儿与乳方末)《短剧方》云∶乳母者,其血气为乳汁也,五情善恶,血气所生也。乳儿者,皆宜慎喜怒,夫乳母形色所宜,其候甚多,不可悉得,今但令不狐臭、瘿瘤、KT瘿、气味、蜗、蚧、癣瘙、白秃、疡、唇、耳聋、鼻、癫眩,无此等病者,便可饮儿也。师见其故,灸盘便知其病源也。
小儿为名字法第十六

《产经》云∶子日生名救(一名寿;)丑日生名带(一名徐,一名去病;)寅日生名令(一名阿金;)卯日生名官(一名金,一名宣令;)辰日生名道(一名阿种;)巳日生名益宗(一名阿善;)午日生名徐(一名阿师;)末日生名护;申日生名多(一名起,一名立,一名桓,)酉日生名多;戌日生名带子(一名弟;)亥日生名他人(一名侣。)又云∶正月一日、十一日、二十一日、寅日、寅时生子名为正月子。二月二日、十二日、二十二日、卯日、卯时生子名为二月子。五月五日、十五日、二十五日、午日、午时生子名为五月子。七月七日、十七日、二十七日、申日、申时生子名为七月子。诸此日月合生者可为忌耳。

又云∶以二月、五月生者,皆不利父母。二月生男名安都,生女名候女,一名定女,无咎。五月生男名连KT,一名扶纡,女名恐华朱。如此名之,无咎。是周文王为作神字,不妨害父母,吉。

又云∶初立名字,勿以五子日,凶。又不以巳日,大凶。
小儿初着衣方第十七

《产经》云∶小儿初着衣法∶甲乙日生子,衣以黑衣(忌庚日、晡日、补时,凶。)丙丁日生子,衣以青衣(忌夜半。)戊巳日生子,衣以绛衣(忌甲申日、旦时,凶。)庚辛日生子,衣以黄衣(忌丙子日、申时,凶。)壬癸日生子,衣以白衣(忌戊巳日、晡时,凶。)又云∶小儿初着衣,良日辰巳。男以甲,女以乙,吉。

又云∶凡作儿衣,勿以新绵缯之,损儿气,故宜用故布帛。有人气者,着益也。儿衣不欲浓,多绵之,恒如不忍,见其寒,乃为佳耳。
小儿调养方第十八

《礼记》云∶童子不衣求(裘)裳。(裘大温,消阴气,使不堪苦,不衣裳以便宜也。)《养生要集》云∶婴儿之生,衣之新纩则骨蒸焉。食之鱼肉则虫生焉。串之逸乐则易伤焉。

《千金方》云∶小儿始生,肌肤未成,不可暖衣,暖衣则令筋骨缓弱;不见日风,则令肌肤脆软,便易伤。皆当以故絮着衣之,勿用新绵也。天和暖无风寒之时,令母将于日中嬉戏,数见风日,则血疑气刚,肌肉牢密,堪耐风寒,不致疾病。若常藏在帐中,重衣温暖,辟阴地之草,不见风日,软脆不堪风寒。

又云∶养小儿常慎惊,勿令闻大声,抱持之间,当安徐,勿令怖也。

又云∶天雷塞耳,但作余细声以乱之。

又云∶凡儿冬不可浴,冬生伤寒,夏不可浴,不浴久,久伤热。

又云∶凡小儿不能乳哺,当双丸下之。小儿气盛有病,但下之,必无所损,若不时下,则成病,痼难治矣。

《病源论》云∶薄衣之法∶当从秋习之,不可以春夏卒减其衣,则令中寒。从秋习之以渐,犹寒,如此,则必KT寒,冬月但当着两薄裆一复裳耳,常令不忍见其寒,适佳耳。

爱而蠕之过,所以害之也。

又云∶当消息,无令汗出,汗出则致虚损,便受风寒。昼夜寤寐,皆当慎之。

《产经》云∶凡养小儿法,随大人身之寒温而养之,随天时寒温减增之,能察其微之。

又∶不用新绵帛,新者过即有患儿故也。

凡小儿衣,其寒薄者,则腹中乳食不消,乳食不消则其大行酢臭。酢臭,此欲为癖之渐也,便将双丸以微消之。

凡小儿大行黄而臭者,此是腹中有热,故宜将服龙骨汤,良。

凡小儿不节哺乳者,则病易后,后下之则伤胃气,令腹胀满。再三下尚可,过此则伤小儿矣。

凡小儿冬日下为无所畏,夏日下不瘥,难治。壮有疾者,不可不下,夏日下之后,腹中常当小胀满,故当节哺乳,将护之,数日间愈矣。

凡小儿不使溺灶灰上,令儿阴生疮,难治。
小儿禁食第十九

《产经》云∶小儿食语害,令儿腹中生瘕,难治。

又云∶小儿齿未易,蜜及饴糖不可与食,令儿齿坏,虽易,齿不坚。

又云∶小儿不可与食狗鼠残物,令儿咽中生白疮,死。

又云∶小儿不可与食核未成诸果,令儿生寒热及瘰。

《养生要集》云∶大豆炒小麦,勿与一岁以上,十岁以下小儿,喜气壅而死也。

又云∶凡男子年十五以下,不得饮冰沉浆,腠理未成,故成病。

又云∶小儿未断乳,不可啖鸡肉,生蛔虫,亦令体瘦。

《崔禹锡食经》云∶大麦面勿与一岁以上,十岁以下小儿,其喜气壅塞而死。

《孟诜食经》云∶黍,不可与小儿食之,令不能行。

又云∶小儿食蕺菜,便觉脚痛。

《苏敬本草注》云∶粟(栗)饵孩儿,令齿不生。

《本草拾遗》云∶虾,小儿食之,脚屈不能行。

又云∶小儿食蕨,脚弱不行。
治小儿解颅方第二十

《病源论》云∶解颅者,其状小儿年大囟应合而不合,头缝开解是也。由肾气不成故也。

肾主骨髓,而脑为髓海,肾气不成,则髓脑不足,不能结成。

《产经》论云∶细辛一分,桂心一分,干姜五分。凡三物,以乳汁和,涂上。干,复涂。

《僧深方》云∶取猪牙车骨髓,涂囟上,日一,十日止。良。

《短剧方》云∶生蟹骨、白各二分。凡二物,下筛,以乳汁和,涂上立愈。

《千金方》云∶熬蛇蜕末,和猪颊车中髓涂囟上,日三四。

《葛氏方》∶烧蘩蒌末敷良。

《极要方》云∶防风六分,白芨,柏子仁各二分。上为散,以乳汁涂囟上,日一,十日知,二十日合。

又方∶灸脐上下半寸。(出《新录方》。)
治小儿囟陷方第二十一

《病源论》云∶小儿囟上陷,此谓囟陷下不平也。由腹(肠)内有热,热气熏脏,脏热则竭(渴),引饮而小儿(便)泄曳利者,则腑脏血气虚弱,不能上充髓脑,故囟陷也。

《千金方》云∶灸脐上下各半寸,及灸足太阴各一壮。

今按∶《玉遗针经》云∶足太阴穴在内踝后白肉际陷骨宛中。
治小儿摇头方第二十二

《子母秘录》云∶治小儿长摇头方∶狗脑以摩头,即瘥。
治小儿发不生方第二十三

《病源论》云∶小儿禀生足少阴之血气不足,则发疏薄不生,亦有因头疮而秃落不生。

《短剧方》楸叶中心无多少,捣绞取汁,涂头上。(《产经》同之。)《新录方》云∶以蜜和猪毛灰涂之,即生。

又方∶莲子草汁涂之,验。

又方∶桑上寄生汁涂,立生。

《千金方》云∶枸杞根捣作末,和腊月猪脂敷。和酢亦佳。
治小儿白秃方第二十四

《病源论》云∶白秃之状,头上白点斑剥,初似癣,而上有白皮屑,生痂生疮。头发秃落,谓之白秃。

《葛氏方》云∶烧鲫鱼末,以酱汁和敷之。

又方∶末藜芦,猪膏和涂之。

《产经》云∶先以桑灰汁净洗之,末,白焚灰,和涂之,良。

《短剧方》云∶捣楸叶中心,取汁,以涂头,立生。

《极要方》云∶捣芫花以猪脂和如泥,灸,洗去痂,涂之。

《录验方》云∶取熊白脂敷之。

《千金方》云∶不中水芜菁叶烧作灰,和猪脂涂之。

又方∶葶苈子细末,先洗而洗之。

又方∶桃树青皮,捣,和醋涂之。
小儿鬼舐头方第二十五

《病源论》云∶人有风邪在于头,有偏虚处那发落肌肉枯死,或如钱大,或如指大,发不生,故谓之鬼舐头。

《千金方》云∶狸骨烧末,以猪脂和涂之。

《产经》云∶乱发如鸭子大一枚,醋鲫鱼一头,苦参一两,附子一枚,雄黄二两,猪膏四升。凡六物,捣,筛下,用膏煎发,鱼令尽,纳诸药末绞之,敷疮上,良。
治小儿头疮方第二十六

《病源论》云∶腑脏有热,热气上冲于头,而复有风湿乘之,湿热相搏,折血气,血变生疮。

《极要方》云∶胡粉膏主之。

胡粉水银(各二两)松脂猪膏(各四两)上,煎成去滓,纳水银,胡粉和调,涂疮上,日二,大人蒸治之。

今按∶《千金方》云∶银膏在下条。

《产经》云∶梁上尘,下筛,麻油和,先洗疮毕,拭燥,敷上。

《僧深方》云∶烧竹叶,和鸡子白,敷之,不过三愈。

《苏敬本草注》云∶生嚼胡麻,涂之,大效。

《葛氏方》云∶鸡屎烧冶为末,和猪脂,敷之。

《刘涓子方》云∶取腊月猪屎烧末敷之,良。

《经心方》∶治小儿一切头疮久,即疽痒不生痂,藜芦膏方∶黄连(八分)藜芦(二分)黄柏(八分)矾石(八两)雄黄(八分)松脂(八分)六味,以猪膏二升,煎令调,先以赤龙皮汤洗,敷之。

《子母秘录》云∶竹叶烧灰,和鸡子白敷之。(《僧深方》云∶不过三,愈之。)《范汪方》∶治小儿头疮,面赤有疮,日月益甚方∶黄连赤小豆熬,分等作屑,和猪膏涂之。

《千金方》治小儿头疮经年不瘥方∶松脂(六分)大黄(四分)苦参(五分)黄连(六分)胡粉(四分)凡五味,下筛,以猪膏和研水银散敷上。

《徐之才方》云∶白帝疮,小儿头上疮团团然白色者是也,大蒜揩白处,早朝敷之。
治小儿头面身体疮方第二十七

《病源论》云∶腑脏热盛,热气冲发皮肤而外有风湿折,与血气相搏则生疮。其状∶初赤起,后乃生脓汁,随瘥随发。或生身体,或出头面,或身体头面皆有之者。

《千金方》云∶水银膏主之。(今按《极要方》号胡粉膏在头疮条。)《葛氏方》∶取儿父KT汁以浴之,勿令儿及母知也。

《范汪方》∶治小儿头疮面亦有疮,日月益甚方∶黄连赤小豆熬,分等作屑,和猪膏涂之。

《广利方》疗小儿面上忽生疮黄水土方∶黄连(末三分)胡粉(三分)甘草(一分,炙)三味,以猪脂和,以帛贴疮上,日一。

又方∶鲫鱼一头,烧作灰,和酱汁涂上。

《集验方》治少小面疮方∶丹茱萸叶,以东流水煮,以浴,良。
治小儿面白屑方第二十八

《产经》云∶吴茱萸根白皮,煮取汁,拭洗。

又方∶以刀刃克KT树,取其汁,涂白处上,日三四。

今按∶矾石和酒敷之,尤良。
治小儿耳鸣方第二十九

《病源论》云∶小儿耳鸣,头脑有风,令耳鸣。

《产经》昌蒲散方∶菖蒲乌头(炮,各四分)凡二物,为散,以绵裹,塞耳,日再易。
治小儿耳疮方第三十

《病源论》云∶小儿疮生于两耳,时瘥时发,亦有浓汁,是风湿搏于血气所生也。

《千金方》云∶烧马骨灰敷之。

又方∶敷鸡屎白。

《产经》云∶小儿耳有恶疮及有恶害生耳中方∶雄黄(六分)曾青(二分)黄芩(一分)凡三物,合下筛,以敷耳中,以绵塞耳中,汁出,复敷,良。
治小儿耳方第三十一

《病源论》云∶小儿肾脏盛而有热者,热上冲于耳,津液壅结则生脓汁。亦有因沐浴,水入耳内而不顷沥停积,搏于血气,蕴结成热,亦令有脓汁。皆谓之耳。(久不瘥变成聋。)《集验方》云∶桃核中仁,熟冶,末,(热)或以裹塞耳,常用,良。

《效验方》云∶烧杏仁黄香塞耳。

又方∶黄连、矾石各二两,下筛,如枣核,吹纳耳中,立止。

《产经》云∶捣桂末,以鱼膏和,塞耳,不过三四日。

又方∶釜下灰吹入耳中,令入深,无苦即自丸出,良。
治小儿耳中百虫入方第三十二

《产经》云∶以苦酒灌之便出。

又方∶油脂灌耳中,良。

又方∶水灌注之(佳)。

又方∶熬麻子绵裹,塞耳即出。

又方∶桃叶汁灌耳中。

又方∶以革带钩向耳孔,即诸虫皆出。
治小儿耳蚁入方第三十三

《产经》云∶炙猪膏,香物,安耳孔边,自出,良。
治小儿耳蜈蚣入方第三十四

《产经》云∶以KT树叶裹盐炙,令热,掩耳孔,冷换之即出。
治小儿耳蚰蜓入方第三十五

《产经》云∶胡麻,熬,以葛囊盛,枕之,虫闻香,即出。
治小儿目不明方第三十六

《录验方》∶治小儿眼茫茫不见物方∶鱼胆敷目,鲤鲋等良。
治小儿目赤痛方第三十七

《产经》云∶黄连七枚人乳汁一合(半),渍敷之。(今按∶《极要方》无人乳,黄连数分等用之。)又方∶竹沥汁三合,人乳汁一合,和以绵取药拭目。

又云∶小儿目赤,泪出不止,灸足大指上丛毛中,名大都。

《葛氏方》云∶捣荠菜取汁,以注目中。

《录验方》云∶鲤鱼胆敷之良。
治小儿眼烂痒方第三十八

《产经》云∶治小儿伤风,间赤烂痒,经年不瘥,青铜散方∶取大铜钱一百文,以好酒三升煎钱令干燥,刮取屑,下筛,稍以纳眼。
治小儿眼翳方第三十九

《产经》云∶治小儿眼有障翳,大小儿年至七八岁,眼瞳子犹不坚,不宜辄敷食翳散,止单敷瑚瑚散,取如粟米大纳翳上,日再。

又方∶宜单敷马珂散,皆令精细,好浓蜜。(绢筛用之。)
治小儿雀盲方第四十

《病源论》云∶人有昼而精明,至暝便不见物,谓为雀目,言如鸟雀,暝便无所见也。

《千金方》云∶至黄昏时,看雀宿处,打惊之,雀起飞,乃咒曰∶柴公,我还汝盲,汝还我明。如此三日暝三过为之,眼明也。秘法也。

《产经》云∶见定雀宿处,夜令雀惊起之,曰∶雀,汝目去之。如此三日,即愈。

又方∶大豆七枚,稻一穗,以二物暮向于鼠穴,曰∶穴公穴公,甚甲得雀目,夜无所见故欲汝眼,汝许与之,我获汝眼。诵三遍讫,则以稻置孔口,则曰∶我得鼠目,暗夜能视。

豆穗置鼠窟而起,去之。如此三夕。验。秘术。

《新录方》云∶鲤鱼、鲋鱼胆敷并良。
治小儿目昧第四十一

《产经》云∶以猪脂着鼻孔中,随目左右以鼻嗡KT之讫,闭目仰寤寐,须臾,不复知昧处,有验。

又方∶早起对户门再拜跪言∶户门狭小,不足宿客。愈之。

又方∶吞蚕砂一枚,即出。
治小儿目竹木刺方第四十二

《产经》云∶鲍鱼二,以绳贯,以水煮,令烂,取汁灌目中,即出。
治小儿目芝草沙石入方第四十三

《产经》云∶研好墨,以新笔注瞳子上,良。

又方∶取麦汁注目中。

又方∶烧甑带末,服方寸匕,立出。
治小儿眼为物撞方第四十四

《产经》云∶炙羊肉慰之,勿令甚热,无羊用猪肉,良。

又方∶好黄连去毛细切,以人乳汁渍令黄色,如大豆许,着目中,仰卧,勿覆之,甚佳。
治小儿燕口方第四十五

《病源论》云∶小儿燕口,两吻生疮,此由脾胃有客热,上气熏于口,两吻生疮。自白色如燕子之吻,故名为燕口。

《千金方》揪云∶烧发灰和猪脂涂之。

又方∶KT白皮及湿粘贴,四五度。
治小儿口疮方第四十六

《病源论》云∶小儿口疮,由血气盛,兼将养过温,必有客热,热熏上焦,故口生疮。

《产经》云∶取乌贼鱼骨烧作屑,以乳汁和,涂口中疮上。

《葛氏方》云∶(治小儿口疮不得饮乳方)∶桑白汁涂疮上,日夜十余过。(《产经》云∶涂乳以饮儿,良。)《博济安众方》∶以白矾,锻石涂之。

又方∶牛膝炙根,酒煎含之。

《极要方》云∶取赤葵茎灸(炙)为末,蜜和含之。

《苏敬本草注》云∶槟榔帽。作灰敷之。
治小儿口下黄肥疮方第四十七

《病源论》云∶小儿有涎唾多者,其汁流溢,浸渍于颐,生疮,黄汁出,浸淫肥烂,挟热者,疮汁则多。

《千金方》云∶治小儿口下黄肥疮方∶熬灶上饭令焦,末敷之。
治小儿唇疮方第四十八

《葛氏方》云∶葵根烧末敷之。

《经心方》云∶蟾蜍烧末敷之。
治小儿紧唇方第四十九

《单要方》云∶泽兰心,嚼以敷之。

又方∶肉机上垢涂之。
治小儿口噤方第五十

《病源论》云∶小儿中风口噤者,是风入颈项(颊)之筋故也。

《录验方》云∶服竹沥汁二合,分温四分(服)。

又方∶灸百会穴。

《僧深方》∶取雀矢白,丸如麻子,服之即愈。

《千金方》∶鹿角粉、大豆末分等和,涂乳饮儿。
治小儿重舌方第五十一

《病源论》云∶小儿重舌者,心脾热故也。

其状附舌下,如舌而短,故谓之重舌也。

《葛氏方》∶以儿着箕中东向,纳中灸箕舌三壮,良。

又方∶釜月下土苦酒和,敷舌下。

《短剧方》∶以赤小豆屑洒和,敷舌上。又方∶烧乱发作未,敷舌上,良。

又方∶用以栗哺之,良。

《徐大山方》∶甑带烧灰,末,敷之。

《极要方》∶取蒲黄敷上,良。

又方∶取衣中白鱼烧作屑,敷舌下。

《产经》云∶以铍针刺舌上(下)肿者,令血,有刺大脉。

又方∶烧乌扇根苦酒和,涂上。

《徐大山方》∶甑带烧灰敷之。
治小儿舌上疮方第五十二

《病源论》云∶小儿苦心脏有热,舌上生疮也。

《短剧方》∶用乌贼鱼骨烧屑,以鸡子黄和,涂喉下及舌下也。(今按∶《产经》云∶以乳汁和涂上。)又方∶清旦起,斫桑木令白汁出,涂乳以饮儿。(今按∶《龙门方》∶涂舌。)
治小儿舌肿方第五十三

《病源论》云∶小儿舌肿者,心脾俱热,气发于口,故香(舌)肿也。

《千金方》∶釜月下墨末,和醋,涂舌上(下)。

又方∶满口含糖醋,少时热气通,愈。
治小儿齿晚生方第五十四

《病源论》云∶齿是骨之所终,而为髓之所养也,小儿有禀气不足者,则髓不能充于齿骨,故久不生也。

《葛氏方》∶以薄KT编绳,向东磨齿处,微令破,即生,甚神验。

《极要方》∶雌鼠屎二七枚,以一鼠屎拭齿处,尽二七枚止,二十一日齿当生。(今按∶《短剧方》∶雌鼠屎,一头大一头小是也。)《苏敬本草注》云∶人屎中竹木,以正旦刮之即生也。
治小儿齿落不生方第五十五

《产经》云∶取牛屎中大豆二七枚,小开儿口,以注齿处即生。(《葛氏方》同之。)
治小儿齿间出血方第五十六

《范汪方》治少小龈齿间血出,龈皆赤黑色方∶取生雀割之,以血涂龈上及齿间,便愈。

有验。
治小儿鼻衄方第五十七

《病源论》云∶小儿经脉血气有热,喜令鼻衄也。

《极要方》;取乌马屎,薄绵裹,塞鼻中。

又方∶烧发作灰,少许吹纳鼻中。

《产经》云∶阿胶令烊,水着贴额上良。

又方∶书额上言“今日血忌”,即止。

又方∶书额上言∶“血出不止,流入东海”。良。
治小儿鼻塞方第五十八

《病源论》云∶小儿风冷气入于头脑,停滞鼻间,则气不宣和,结聚不通,故鼻塞也。

《产经》云∶小儿患鼻不通有涕方∶杏仁(二分)蜀椒(一分)附子(一分半)细辛(一分)凡四物,咀,以淳苦酒五合,渍一宿,明旦以成;煎猪肪五两,微火上煎,令附子黄,膏成,绵絮绞去滓,以涂鼻中,日再。

又∶披儿头发,囟上左右以膏摩十数过,良。

《千金方》∶治鼻痛方∶恒以油涂鼻内外。

又方∶涂酥佳。
治小儿鼻息肉方第五十九

《产经》云∶治少小鼻息肉通草散方∶通草(一两)细辛(一两)凡二物,下筛,展绵如枣核,取药如小豆着绵头纳鼻中,日二。

又方∶矾石一两。(今按∶《千金方》∶细辛二两,云云。)
治小儿喉痹方第六十

《病源论》云∶小儿喉痹,是风毒之气客于咽喉之间,与血气相搏而细(结)肿痛,甚者,肿塞,饮粥不下,乃成脓血也。

《产经》∶取乌扇烧灰,以水服,大良。

又方∶甑带作绳系头,愈。

《千金方》∶桂一分,杏仁一两,凡二味。为散,绵裹如枣大,咽其汁。

又方∶煮大豆汁含之。
治小儿哕方第六十一

《病源论》云∶小儿哕,由哺乳冷。冷气入胃,与胃气相逆,冷折胃气,不通则哕也。

《千金方》云∶生姜汁,生乳各五合,合煎,取五合,分二服。
治小儿津颐方第六十二

《病源论》云∶津颐之病,是小儿多涎唾流出,渍于颐下也。

《葛氏方》∶取东行牛口中沫,涂儿口。

又方∶捣鹿角,熬如豆,着舌下。

《千金方》∶桑白汁涂之。

《玄感方》∶牛口中饲草绞汁,涂口中。
治小儿吐方第六十三

《病源论》云∶小儿吐者,由乳哺冷热不调故也。

《应验方》∶取桑根汁,着汁口中,即瘥。

又方∶取新牛屎,水绞汁,少少饮儿,大良。

《经心方》∶当以空乳乳则消。

《圣惠方》∶治吐乳黄色方∶用韭根汁滴豆大,入口中,立瘥。

又方∶用新热马粪绞取汁半合,灌之效。
治小儿难乳方第六十四

《病源论》云∶小儿初生恶血,儿咽入腹,令心腹痞满,儿不能饮乳,谓之难乳。

又∶儿在胎之时,母取冷,冷气入胞,儿生则腹痛,不肯饮乳,亦名难乳。

《千金方》∶炒鹿角末如小豆,着舌下,KTKT与之。

又方∶雀屎四枚,未着乳头饮之。大儿十枚。
治小儿风不乳哺方第六十五

《录验方》∶治小儿风,数十日口中寒,不能乳哺方∶取生竹汁服之即瘥,炙(名)竹沥也。

《博济安众方》∶小儿吐乳方∶人参(二两)橘皮(一两)生姜(一两)以水一升半煎取八合,细细服之。
治小儿脐不合方第六十六

《千金方》∶蜂房灰末敷之。

又方∶烧甑带灰和膏敷之。

又方∶大车辖脂,烧作灰,日一敷之。
治小儿脐中汁出方第六十七

《千金方》∶烧苍耳子粉之。

又方∶烧蜂房灰粉膏中。《效验方》∶甘草(二分)椒(一分)下筛,以粉之。

又方∶矾石,附子各二分,下筛,粉脐中,日二。
治小儿脐赤肿方第六十八

《千金方》∶杏仁(二分)猪颊车中髓(二分)凡二味,先研杏仁如脂,敷上。
治小儿脐疮方第六十九

《病源论》云∶小儿脐疮,由初生断脐洗浴,拭不即拭燥,湿气在脐中,遇风湿相搏故也。

《本草》云∶蒿艾茎间白毛敷之,立瘥。

《录验方》∶可用姜黄柏散粉之。

《产经》云∶黄柏,釜月下墨各四分,末,敷之。

《千金方》小儿风脐遂作恶疮,历年不瘥方∶敷东壁土,大佳。

又方∶蜂房灰末敷之。

《葛氏方》∶小儿风脐及脐疮久不瘥方∶烧甑带作灰和乳汁敷之。

又方∶末当归粉之。
治小儿腹痛方第七十

《病源论》云∶小儿腹痛,多由冷热不调,冷热之气与脏相系故也。

《千金方》∶梨叶浓煮,取汤,一服七合,可三四与之。

又方∶半夏随多少微火炮之,末,酒和服如粟粒五合丸,日三。
治小儿腹胀方第七十一

《病源论》云∶小儿腹胀,是冷气客于脏故也。

《千金方》云∶烧父母指甲灰,涂乳上,饮之。

又方∶腹上磨书鱼。

又方∶故衣带饶垢者,切一升,水三升,煮取一升,分三服。

《葛氏方》∶粉及盐分等,合熬,令变色,以磨腹上,即愈。
治小儿痞病方第七十二

《病源论》云∶小儿胸膈热实,腹内有留饮,致令营卫痞塞;腑脏之气不宣通,其病(痛)腹内结。胀满,或时壮热是也。

《葛氏方》方∶若患腹中痞结常壮热者方∶生鳖血,和桂屑涂痞上。

又方∶末麝香,服如大豆者。

又方∶大黄(炙,令烟出)龟甲(炙令黄)茯苓凡三物,分等蜜丸,服如大豆一枚,日三。(以儿大小增减也。)又方∶捣白头翁,练右囊盛以掩痞上。

《产经》云∶治小儿痞,面黄羸瘦,丁奚不欲食,食不生肌肤;心中嘈嘈,烦闷,发时寒热五脏胪胀,腹中绕脐痛,常苦下,八痞丸方∶桂心曾青(无代空青)牡丹鳝头甲(头渍,炙令黄色)干姜(各三分)蜀漆(七分)细辛(六分)龙胆(五分)附子(四分,炮)凡九物,冶下筛,蜜和如梧子,服二丸,日三,禁如药法。

《本草拾遗》云∶小儿痞,三白草捣汁,服之令人吐。
治小儿瘕方第七十三

《病源论》云∶五脏不和,三焦不调,有寒冷之气容之。则令乳哺不消化,结聚成瘕癖结也。其状按之不动,有形段者,微也,推之浮移者,瘕也。

《千金方》∶灸两乳下一寸三壮。

又方∶桃树青皮捣,和酢,涂,日二。

又方∶枸杞根捣作末,和猪脂敷之,和酢亦佳。

《医门方》∶捣蒜和酢敷,如移余处,随就拊之,验。

《本草拾遗》云∶苦瓠取未硬者,煮令热,解开,熨小儿闭癖。
治小儿米症方第七十四

《产经》云∶治少小米症恒欲食米方∶鸡屎(一升)白术(五合)凡二物,合炒,取米焦,捣末,以水一升,顿服取尽;斯须即吐出症,吐出症如研米末为症,若无症而吐出白痰水,增米须食米。
治小儿土症方第七十五

《产经》∶治少小食土,腹中作土瘕,恒欲食土,啖肉方∶生肉一斤,以绳系曳地行数里,勿洗便炙,啖之即愈。
治小儿腹中有虫方第七十六

《耆婆方》∶治小儿腹中有虫方∶芜荑作末,每食随多少和,少少水食之乃止。(百无所禁。)
治小儿阴肿方第七十七

《病源论》云∶小儿下焦热,热气冲阴,阴头勿肿合,(令)不得小便,乃至生疮。俗云尿火所为也。

《千金方》∶捣芜菁敷之。

又方∶书鱼磨之。

又方∶捣苋菜根敷之。

又方∶熬桃仁末,酒服方寸匕,日三。
治小儿阴痛方第七十八

《千金方》云∶猝阴痛如刺,汗出如雨方∶小蒜韭根杨柳根(各一斤)凡三味,合烧,以酒灌之,及热以气重之。

又方∶甘草末和乳洗之。

又云∶玉茎痛方∶甘草石蜜末,和乳洗之。

《极要方》∶浓煮野狼牙根,洗之,甚良。
治小儿阴疮方第七十九

《产经》云∶治小儿阴疮烂痛方∶浓煮野狼牙根,洗之,甚良。

又云∶小儿阴头生疮,似石榴花者方∶虎牙犀角凡二物,刀刮,以猪脂煎,令变色,去滓,涂上,神,良。

《千金方》云∶治小儿阴痒生疮方∶嚼胡麻敷之。

又方∶蜜煎甘草,末之涂上。

又方∶黄连,胡粉分等和,面脂涂之。

《葛氏方》∶取灶中黄土,末,以鸡子白和,敷之。

又方∶浓煮黄柏汁渍之。
治小儿阴伤血出方第八十

《产经》∶治女小儿为物触伤,阴道血出不止方∶人头发并青布烧作灰,以麻油和涂之,亦可仍以粉,良。

又云∶若深刺触药涂不及方∶蒲黄,水和服之,即止。
治小儿阴囊肿方第八十一

《千金方》∶酢和面涂之。

又方∶釜月下土,和鸡子白敷之。

《医门方》∶末桂心,涂,良。

又方∶末大黄,和酢涂,良。
治小儿阴颓方第八十二

《病源论》云∶颓者,阴核气结肿大也。小儿患此者,多因啼怒,气不止,动于阴气,下击,结聚不散所致也。

《千金方》∶灸足厥阴大敦,左灸右,右灸左。

又方∶三月上除日,白头翁末敷之,一宿作疮,二十日愈。

《产经》云∶牵阴头正丁上行,灸头所极。又牵下行向谷道,灸所极。

《短剧方》云∶小儿颓方∶先将儿至碓头祝之曰∶坐汝令儿某甲称儿名也,阴称故灸汝三七一,灸讫,便牵小儿令茎以下向佳囊缝,当阴以所着处灸缝上七壮即消,有验。

又∶左右髀直行,灸所极,皆四处,随年壮。

又方∶小儿骑碓轴前齐阴茎头前灸,有年壮。

又云∶《葛氏方》∶但灸其上,又灸茎上向小肠脉。

又方∶灸手小指头七壮,随瘥左右也。

《经心方》∶灸两足内踝上七寸,日七壮。

又方∶但灸其上也。
治小儿差颓方第八十三

《病源论》云∶差颓者,阴核偏肿大也,其偏虚者,气虚而行,故偏结肿也。

《千金方》∶五等丸治小儿阴偏大,卵核及颓方∶香豉牡丹防风桂心黄柏(各二两)凡五味,丸如大豆,儿三岁饮五丸,日三,儿小者,以意减之,着乳头与之。
治小儿脱肛方第八十四

《病源论》云∶小儿患脱肛门,(脱)出多因利,大肠虚冷兼因气故也。

《苏敬本草注》云∶烧鳖头为灰涂(服)之。

《本草拾遗》云∶有以似为药者,蜗牛鳖头,脱肛皆烧末,敷之自缩。

《录验方》∶取铁精粉推纳之。

又方∶宜灸龟尾骨上三七丸。

《葛氏方》∶熬锻石令热,故绵(帛)裹坐其上,冷复易之。

《僧深方》∶取蒲黄一两,以猪膏和,敷之,不过三,愈。

《千金方》∶灸顶上施毛中即入。

《产经》云∶生铁三斤,以水一斗,煮取五升,以汁洗,日三,乃以蒲黄敷上,良。
治小儿谷道痒方第八十五

《子母秘录》云∶小儿谷道虫痒方∶大枣取膏和,捻长三寸,绵裹,纳孔中,明日出之,虫死。

又方∶胡粉,雄黄分等着中。

今按∶煮桃皮洗之,煮枸杞洗亦良。
治小儿谷道疮方第八十六

《葛氏方》治猝下部有疮方∶煮豉以渍之。

又方∶KT汁以磨墨,导也。
治小儿甘湿方第八十七

《病源论》云∶甘湿之病,多因久利,脾胃虚弱。肠胃之间,虫动侵食五脏,使人心烦懊闷;其上食者,则口鼻齿龈生疮;其下食者,则肛门伤烂,皆难治。或因久利,或因脏热嗜眠,或好食甘美之令蒸,令血动,致生此病患。

《子母秘录》云∶羊胆和浆灌下部,用猪胆亦佳。

《千金方》∶细和胡粉涂之。

又方∶嚼麻子涂之。

又方∶以铁衣着下部中即瘥。
治小儿寸白方第八十八

《病源论》云∶小儿寸白者,九虫之内一虫是也。因腑脏虚弱而能发动也,食生栗所成。

又食生鱼后即食乳酪,亦令生。

《葛氏方》∶薏苡根二斤,细锉,水七升,煮取二升,分再服。又可作糜也。

《极要方》云∶东行茱萸根白皮四两,桃白皮三两。上,切,以酒一升渍一宿,去滓,顿服良验。

又方∶煮扁竹汁饮之,有验。

今按∶蒸之当上。

又方∶浓煮猪,槟榔,饮三升,虫则出尽。

又方∶多食榧子。(今按∶《千金方》∶榧子四十九枚去皮,以月上旬,旦,空腹服七枚,七日服尽,虫消成水,永瘥。)今按∶研胡桃仁敷。

《产经》云∶练皮,削去上皮,取中白者,切五升,以水五升,煮得二升。先令小儿食饧令渴,因饮之。今按∶《子母秘录》∶练实一枚,纳孔中。

又方∶研芥子敷之。

又方∶研雄黄,和酢敷之,槟榔子亦佳。
治小儿痫病方第八十九

《病源论》云∶痫,小儿病也。十岁以上为癫,十岁以下为痫。其发之状,或口眼相引而目精上摇,或手足掣纵,或背脊强直,或颈项反折。诸方诸(说)痫(癫),名证不同,大较其发之源,皆因三种。三种者风痫、惊痫、食痫是也。风痫因衣浓行出,出而风入为之;惊痫因惊怖大啼乃发;食痫因乳哺不节所成。然小儿气血微弱,易为伤动,因此三种变作诸痫也。

凡诸痫正发,手足掣缩,慎勿捉持之,捉则令曲戾不随也。

《短剧方》云∶《玄中记》云∶天下有女鸟,一名姑权,又名钩皇鬼也;喜以阴雨夜过飞鸣,徘徊人村里,唤得来者是也。是鸟专雌无雄,不产,喜落毛羽中尘,置人儿衣中,便使儿作痫,病必死,便化为其儿也。是以小儿生至十岁,衣被不可露,七八月尤忌。

《神农本草经》云∶小儿惊痫,有百二十种,其证候异于常。

《千金方》云∶所以有痫病,痉病(者),皆由脏气不平故也。新生即痫者,是其五脏未收敛,血气不聚,五脉不流故也。

又云∶灸痫法∶囟中末合,骨中随息动者,是最要处也。(灸五壮。)又云∶顶上回毛中、膻中、巨阙、脐中,毛尺泽、劳宫、伏兔、三里、然谷穴,灸之。

今按∶《明堂经》伏兔穴禁灸。

又云∶灸痫当先下,使儿虚,乃承虚灸之。未下有实而灸者,气逼前后不通,杀人。

又云∶龙胆汤主之。(其方在本书第五卷。)《广利方》∶核子惊痫不知人,迷闷嚼(舌)作目方∶牛黄一大豆许,和蜜,水服之,立效。

又方∶乌犀角失(尖),研,并水二大合,服之立效。

《枕中方》∶取纸中白鱼,随羊乳和之即愈。
治小儿魃病方第九十

《病源论》云∶小儿所以有魃病者,妇人怀娠,有恶神道(导)其腹中胎,如娠(妒)嫉而制伏他小儿令病也。妊娠妇不必悉能(制)魃,人时有此耳。魃之为疾,喜生微下,寒热有去来,毫毛发KTKT不悦,是其证也。魃音制。

《千金方》灸(炙)伏翼,熟嚼哺之。

篇KT〔切,一升(斤)〕冬瓜〔切,一升(斤)〕以水五六升,煮六七沸,去滓,稍以浴之出。(出《新录方》。)《苏敬本草注》云∶白马眼疗小儿魃,母带之。
治小儿客忤方第九十一

《病源论》云∶小儿中客忤者,是小儿神气嫩(软)弱,忽有非常之物,或者未经识见之人触之,与儿(鬼)神气相忤而发病,谓之客忤也。又名中客,又名中人。其状吐下青黄白色水谷,解离,腹痛反倒夭矫,面变易五色,状似痫,但眼不上,摇耳,其脉弦(强)急数者是也。若失时不治,小久则难治。若乳母饮酒过醉及房劳喘后乳者最剧,能杀儿也。

《千金方》云∶少小所以有客忤病者,是外人来气息忤之,一名中人,是为客忤也。虽是家人或别房异户,虽是乳人父母,或从外还,衣服或经履鬼神,粗恶异气,牛马之气,皆为忤也。

又云∶凡小儿衣,布帛绵中不得有头发,履中亦尔,白衣青带,青衣白带,皆令儿中忤。

又云∶凡非常人及物从外来,亦惊小儿。欲防之法,诸有从外来人及异物,当持儿避之,勿令儿见也。若不避者,烧牛屎灰令常烟,置户前,则善。治之方∶马屎三升,烧令烟绝,以酒三升煮三沸,去滓,浴儿。

又方∶烧母衣带三寸并发,合服汁,服之。

《极要方》∶蚯蚓屎,灶中黄土,等分为散,水和涂儿头上及手心。

又方∶取铜镜鼻烧令赤,着小许酒中代饮之,小儿又能饮合之,即愈。取热马屎一丸,注取汁饮之,儿下便愈。

《短剧方》∶吞麝香如大豆,立愈。

又方∶取衣中白鱼十枚,末,以涂母乳头令儿饮之,入咽即愈。

《葛氏方》∶令儿仰卧,以小盆着胸上,烧甑蔽于盆中,火减即愈。

《产经》云∶牛黄如大豆研,饮之即效。
治小儿夜啼方第九十二

《病源论》云∶小儿夜啼者,脏冷故也,夜阴气盛,与冷相并则冷动,冷动与脏气相搏,或烦或痛,令儿夜啼也。亦有犯触禁忌,令儿夜啼。

《龙门方》∶取镜系床脚即止。

又方∶书脐上作田字,瘥。

《短剧方》少小夜啼至明安寝,夜辄啼,芎散方∶芎(二分)术(二分)房己(二分)凡三物,捣下筛,二十日儿未能服散者,以乳汁和之,服如麻子一丸,儿大能散,散者,服之多少,以意节度。

《产经》∶真珠少许,以水和,涂腹上。

又方∶取车膏着脐中。

《集验方》取空井中草,悬户上,勿令母知。

今按∶《本草拾遗》云∶井口边草,潜着母卧席下,勿令知。

《玄感方》取干(牛)屎手许,安母卧席下,(卧上)勿令母子俱知,吉。

《葛氏方》∶取犬颈下毛,缝囊裹以系儿两手立止。

又方∶暮取儿衣,以系柱。

《千金方》∶以妊身时,食饮偏有所思物,(以此)哺儿愈。

《本草拾遗》云∶灶中土及四交道中土,合末以饮小儿,辟夜啼。
治小儿惊啼方第九十三

《病源论》云∶小儿惊啼者,是于眠睡里忽然啼而惊与觉也;由风热邪气乘于心脏,生热,精神不定,故卧不定,则惊而啼也。

《产经》云∶真珠少许,以水和,涂腹,良。

《葛氏方》捣柏子仁以一刀圭饮之。

《千金方》酒服乱发灰。

又方∶车辖肪,纳口中脐中。

又方∶烧皮三寸灰,着乳头饮之。
治小儿啼方第九十四

《病源论》云∶小儿在胎时,其母将养,伤于风冷,邪气入胞,伤儿脏腑。故儿生之后,邪犹在儿腹内,邪动与正气相搏则腹痛,故儿张蹙气而啼也。

《千金方》∶取新出马屎一丸,绞取汁,含之。

又方∶烧猪屎沸汤,淋取汁,浴之,并与少许饮之。
治小儿疟病方第九十五

《病源论》云∶疟病者,夏伤于暑,至秋风邪乘之发。小儿未能触胃于暑亦病疟者,是乳母抱持解脱,不避风也。

《千金方》∶生鹿角末,发先时便服一钱匕。

又方∶烧鳖甲,酒服方寸匕,至发时服三匕,并火灸身。

《葛氏方》云∶临发时,捣大附子,下筛和苦酒涂背上。

又方∶石上菖蒲浓煮,浴儿。(三四时亦佳也)。

今按∶《集验方》∶桃叶二七枚按心上,艾灸叶上十四壮。

《葛氏方》云∶恒山四分,小麦三合,淡竹叶切一升。上,以水三升,煮取一升,服之。

(今按∶《产经》云∶儿生四十余日至六十日者,分三服,或至百日服二合半,或于二百日一服三合。)《产经》云∶师左手持水碗,右持刀子,正面于北儿曰∶北斗七星,主知一切死生之命,属北斗之君王某甲病疟,勿令流行。诵三遍讫,禹步就病儿前,令视碗中,师则吐呵,以其持刀刺碗中儿影,急急如律令,勿令及顾。甚秘验,过病发后取刀子。(《范汪方》同之。)又法∶头面胸背上皆笔作“天公”字,胸上书作咒日∶太山之下有不流水,上有神龙,九头九尾;不食余物,正食疟鬼,朝食一千,暮食五百,一食不足,遣我来索。疟鬼闻之,亡魄走行千里。用朱书之。
治小儿伤寒方第九十六

《产经》云∶治小儿伤寒头痛方∶生葛汁(六合)竹沥(六合)凡二物,相不(和)煮,两三岁儿,分三服。
治小儿猝死方第九十七

《葛氏方》∶治小儿猝不知何所疾,痛而不知人,便绝死方∶取雄鸡冠血,临儿口上割,令血出,沥儿口,入喉便活。

《子母秘录》∶盐汤极咸,作服一升,取吐即活。

又方∶热汤和灰拥封之即活。

《范汪方》∶热马屎一丸,绞取汁吞,儿下喉愈。
治小儿注病方第九十八

《病源论》云∶注之言住也,谓其风邪气留人身内也。

《千金方》云∶取灶突中灰三指撮,盐末等服之。
治小儿数岁不行方第九十九

《病源论》云∶小儿生,自变蒸至于能语,随日数血脉骨节备成。其髌骨成则能行,骨是髓之所养。若禀生血气不足者,即髓不死强,故其骨不即成,而数岁不能行也。

《千金方》∶取丧家未闭户时,盗取其饭以哺儿,不过三日即行,勿令人知之。(今按∶《短剧方》取束哺之,日三便起。)
治小儿四、五岁不语方第百

《病源论》云∶人之五脏有五声,脾之声为言,小儿四五岁而不能言者,内在胎之时,其母猝有惊怖,内动于儿脏,邪气乘其脾,使脾气不和故也。

《千金方》∶灸足两踝上三壮。

又方∶末赤小豆,和酒涂舌下。
治小儿无辜方第百一

《病源论》云∶小儿面黄发直,时壮热,饮食不生肌肤,积经日月,遂致死者,谓之无辜。言天上有鸟,名无辜,昼伏夜游,儿洗小儿衣席,露之经宿。此鸟则飞从上过,而取此衣与小儿着,并席与儿卧,便令儿着此病也。

《崔侍郎方》∶以酢煮大黄末为丸,服之甚验
治小儿大腹丁奚方第百二

《病源论》云∶小儿大腹丁奚病者,由哺食过度;而脾胃尚弱,不能磨消故也。其病腹大、颈小、黄瘦是也。

《葛氏方》∶取生韭根,捣,以猪膏煎,稍稍服之。

又方∶熟炙鼠肉若伏翼肉,以哺饮之。

《苏敬本草注》云∶牛脐中毛烧服之。

《录验方》∶甘草十八分。一物下筛,蜜和为丸。一岁儿服如小豆粒二十九,日二三,不妨食及乳,服尽更合。
治小儿霍乱方第百三

《病源论》∶小儿肠胃嫩弱,因解脱逢风冷,乳哺不消,而变吐利也。或乳母触冒风冷,食生冷物,皆冷气流入乳,饮之亦成霍乱。

牛涎灌口中一合也。

又∶热牛屎汁含之。

又方∶研KT滓乳上服之三刀圭,日三。(出《千金方》。)《产经》∶人参汤方∶人参(四分)浓朴(二分)甘草(二分,炙)白术(三分)凡四物,以水一升二合,煮取五合,五十日儿服一合,百日儿分三服之。

《千金方》治小儿吐利方∶乱发烧之二分,鹿角一分,末,米汁服之三刀圭,日三。

《录验方》∶煮浓朴服之。

又方∶煮梨叶服之。
治小儿泄利方第百四

《病源论》云∶甚冷气盛,利甚为洞泄,洞泄不止,为注下。

《千金方》∶小儿洞泄下利方∶炒仓米,末,服之。

又方∶石榴石榴烧末,服一钱匕,日三。

《产经》云∶小儿洞利,昼夜不止。黄芩丸方∶黄芩(二分)干姜(二分)人参(二分)下筛,蜜丸如大豆,服三丸,日三。

《录验方》治小儿患利腹内不调方∶薤白(切,七合)人参(切,八分)浓朴(四分,切)粟(三合)凡四物,以水四升,煮取二升,稍饮之。

《葛氏方》下利不止方∶末赤小豆,和苦酒涂践下。

又方∶猪肉炙哺之。

《本草拾遗》云∶食菰鸟郁甚良。
治小儿白利方第百五

《病源论》云∶凡利色青白黑皆为冷色,黄赤色,是热也。

《产经》云∶治小儿白利∶灸足内踝下骨际三壮,随儿小大增减。

《子母秘录》云∶治小儿冷利方∶浓朴人参(各四分)以醋浆半大升,煮取二合,母含吐与之。
治小儿赤利方第百六

《龙门方》∶孩子赤利方∶薤白(切,三合)栀子(七枚)香豉(二合)水二升,煎取六合,去滓,分三服之。

《博济安众方》蓝青汁五合,分服。
治小儿赤白滞下方第百七

《千金方》∶捣石榴汁服之。

又方∶蜂房灰服之。

又方∶鲤鱼一头,烧末服之。

又方∶烧骨末服之。

《极要方》∶薤白一把,香豉一升,水三升,煮取一升,(分)三服之。

《本草拾遗》云∶鲫鱼脍主水谷不调,下利。

又方∶柿,小儿食之,止下利。

《产经》云∶龙骨三两,研如米粒,以水三升,煮取一升五合,服。(多少任意。)又方∶大枣三十枚,以水三升,煮取二升,去滓,热服之。

《子母秘录》云∶孩子赤白利方∶犀角一两,甘草二两,以水一大升,煮取一大合,饮多少。
治小儿蛊利方第百八

《产经》∶治小儿蛊利血尿方∶取生地黄汁一升,分四五服之。

《千金方》云∶蛊利以蛊法治之,其方在治蛊毒方中。蓝青汁一合,分四服之。
治小儿大便不通方第百九

《病源论》云∶小儿大便不通者,腑脏有热,乘于大肠故也。

《葛氏方》∶取蜂房熬末,以酒若水,服少许。

又方∶以白鱼虫磨脐下至阴。
治小儿小便不通方第百十

《葛氏方》∶取衣中白鱼虫,涂脐中,纳尿道中(横骨又佳也)。

又方∶取故席多垢者,锉一升,以水三升,煮取一升,去滓,饮之。

《千金方》车前子半升,小麦一升。二味煮为粥服,日三。

又方∶葵茎子久在一升,水四升,煮取一升,纳滑石一分,研服半合,日三。

《子母秘录》治小儿尿不通符∶KT墨书脐下,亦朱书甚验。

《效验方》蒲黄,滑石各一分,下筛,以酒服方寸匕,日二。

《产经》治小儿未满十日(腹满)烦不得小便方∶烧蜂房服之。

又方∶蒲黄一升,以水和,涂横骨上良。

《短剧方》治少小小便不利,茎中痛欲死方∶牛膝大把无多少,煮作饮,饮之立愈,有验。
治小儿大便血方第百十一

《葛氏方》大便血方∶刮鹿角作屑,以米汁服五分匕。(日三四之。)又方∶烧鹊巢为屑,饮之少少许。

《千金方》大便竟出血方∶烧甑带末涂乳饮之。

又方∶烧车一枚赤,纳一升水中,分二服。

《僧深方》茅根二把,以水四升,煮取二升,服之。
治小儿小便血方第百十二

《产经》云∶小便血方∶末龙骨,温酒服方寸匕。

又方∶煮大麻根饮之。

又方∶煮白根饮之,多少任意。

《千金方》小儿尿血方∶烧鹊巢灰,井花水服之。
治小儿淋病方第百十三

《病源论》云∶少儿淋者,肾与膀胱热也。其状小便出少起数,小腹急痛引脐是也。

《千金方》∶车前子一升,水四升半,煮取一升半,分三服。

又方∶滑石水煮服之。

又方∶蜂房,乱发分等烧末,水服两钱匕。

《效验方》陈葵子一升,水二升,煮取一升,分三服之。

《短剧方》治少小淋沥,形羸不堪,大汤药者∶枳实三两,炙,筛,儿三岁以上服方寸匕,儿小以意稍服之。有验。

《僧深方》云∶车前子,滑石分等,冶筛,麦粥清和,服半钱匕。

《产经》治少小石淋方∶蜂房(一分),炙桂心(一分)凡二物,冶筛,服一刀圭,以铜器承尿,尿与石俱出。
治小儿遗尿方第百十四

《病源论》云∶遗尿者,此由膀胱有冷,不能约于水故也。

《千金方》∶灸脐下一寸半,随年壮。

又方∶小豆叶捣汁服之。

《葛氏方》∶取燕巢中蓐烧,服一钱匕即瘥。
治小儿身黄方第百十五

《病源论》云∶小儿在胎,其母脏气有热,熏蒸于胎;至生儿皆体黄,谓之胎疸。

《千金方》治小儿身黄方∶捣土瓜根汁五合服之。

又方∶麦青汁服之。

又方∶捣韭根汁,滴儿鼻口,大豆许。
治小儿身有赤处方第百十六

《病源论》云∶小儿因汗,为风邪热毒所伤,与血气相搏,热气蒸发放外,其肉色赤而壮热是也。

《千金方》治小儿身有赤处者方∶熬米粉令黑,唾和涂之。

又方∶伏龙肝、乱发灰、猪脂,和,涂之。

《葛氏方》∶烧牛屎涂之。

又方∶鸡冠血涂之。
治小儿腹皮青黑方第百十七

《病源论》云∶小儿因汗,腠理则开,而为风冷所乘,冷搏于血,随肌肉虚处停之,则血气沉涩,不能荣其皮肤,而风冷客之于腹皮,故青黑。

《千金方》治小儿猝腹皮青黑方∶炙脐上下左右,各去脐半寸,并鸠尾下一寸,五处三壮。

又方∶酒和胡粉涂上,若不急治,须臾即死。
治小儿赤疵方第百十八

《病源论》云∶小儿有血气不和,肌肉变生赤色,染渐长大无定,或如钱大,或阔三数寸。〔今按∶小儿蓝注有《病源》死(无)治方。〕《千金方》∶治小儿体有赤疵,赤黑方∶摘父脚中,取血贴疵上,渐消。

又方∶狗热屎涂之,皮自卷落。

《产经》云∶小儿赤斑驻方∶唾、胡粉和,从外向内涂之。

又方∶屋尘,腊月猪膏和,敷之。

又方∶铁屎,以猪膏敷之。
治小儿疠疡方第百十九

《产经》云∶治小儿身上疠易方∶石硫黄以苦酒研之,涂病上,日三。(出《产经》。)又方∶生栝蒌切一升,以验苦酒三升,煎取一升汁,涂病上良。此是德家秘方不传。(出《产经》。)
治小儿疣目方第百二十

《病源论》云∶人有附皮肉生,与肉色无异,如麦豆大,谓之疣子,即疣目也。此多由风邪客于皮肤,血气变化所生。故亦有药治之瘥者,亦有法术治之瘥者,而多生于手足。

《千金方》方∶以刀子决目根四面,令血出。取患疮人疮中黄脓敷之,勿近水,三日即脓溃、根动、自脱。

《产经》∶以松脂涂疣上一宿,即落。良。

又方∶以矾石拭疣上七过,即去。

又方∶艾柱小作,可灸始生疣上三壮,即自去。良。

又方∶月晦日于厕前取故草二七枚,拭目上,讫,祝曰∶今日晦疣惊去。勿反顾之,不过十日枯死。

今按∶俗人以赤苋汁敷之即落,或以煮荒布之汁洗之。
治小儿身上KT方第百二十一

《千金方》治小儿体上KT方∶取马尿洗之,日三四度。

又方∶揩破以牛鼻津涂之。
治小儿身热方第百二十二

《病源论》云∶小儿血气盛者,表里俱热。则烦躁不安,皮肤壮热。

《千金方》治少小身热李叶汤方∶李叶无多少,咀,以水煮,去滓,浴儿。(今按∶《短剧方》∶避目反阴处。)《苏敬本草注》云∶梓白皮主小儿热疮,身头热烦,煮汤浴并散敷之。

《葛氏方》∶治小儿猝身热如火,不能乳哺方∶猝急断犬耳,取血以涂儿面及身也。

《本草拾遗》云∶小儿暴热,搏(捣)水芹,取汁涂之。

又方∶生银,小儿诸热,以水磨服,功胜紫雪。
治小儿盗汗方第百二十三

《病源论》云∶小儿盗汗者,眠睡而汗自出也。若将养过温,因于睡卧阴阳气交,津液发泄,而眠卧汗自出也。

《葛氏方》;以干姜末一分,粉三分,合以粉之。

又方∶石膏一两,麻黄二两,蜜和如小豆,服一丸。

《短剧方》∶黄连三分,贝母三分,牡蛎二分。凡三物,粉一升,合捣下筛,以粉身。

《集验方》∶麻黄根三分,故扇烧作屑一分。冶合乳汁,饮三分匕,大人方寸匕,日三。

(《短剧方》同之。)
治小儿隐疹方第百二十四

《病源论》云∶小儿因汗而解脱衣裳,风入腠理。与血气相搏成之,结聚起相连,成隐疹。风气止在腠理浮浅,其势微,故不肿不痛,但成隐疹痒耳。

《千金方》治隐疹方∶盐汤极咸,洗,绞蓼涂之。

又方∶蚕砂(二升),水(二升),煮,去滓,洗之。

《产经》云∶治小儿风瘙隐疹入腹,身体肿强舌干燥。肿方∶末芜菁子,酒服方寸匕,日三。

《极要方》∶芒硝二两,清酒三升,煮取二升,洗痒上,良。菁子酒服存匕日三。
治小儿丹疮方第百二十五

《病源论》云∶风热毒客在腠理,热毒搏于血,蒸发于外,其皮上热而赤,如涂丹,故谓之丹也。若久不歇,则肌肉烂伤也。

《产经》云∶夫丹者,恶毒之气,五色无常,不即治,转坏肌肉,治之方;则去脓血。

或发于节解间,多断人四肢,治之方∶赤小豆作屑以甘草汤和涂之。

今按∶《录验方》∶和麻油(经温)涂之。

又方∶升麻汤,大黄汤主之。

《录验方》治小儿丹毒方∶取甘蕉根敷之,亦宜服少许汁。

又方∶捣慎火草敷之。

又方∶取白鹅血敷之。

《千金方》捣赤小豆和鸡子白涂之。

又方∶牛屎涂之(干复易)。

又方∶伏龙肝下筛,鸡子白和油敷之。

又方∶和油涂之。

《短剧方》∶水中苔捣敷之。

又方∶芒硝纳汤中取汁拭上。

《新录方》∶水若油研(和)栀子仁,采汁洗之。(取浓汁,洗涂,日二三易)。

又方∶生蓝汁涂之。

《范汪方》∶以生鱼血涂之。(《葛氏方》干更涂之。)《苏敬本草注》涂鲤血,良。

又云∶捣菜敷之。

又方∶煮栗毛壳洗之。

《本草拾遗》云∶鲫鱼脍,主小儿大人丹毒。

又云∶淬铁水,主小儿丹毒,饮一合,是打铁槽中水也。
治小儿赤游肿方第百二十六

《病源论》云∶小儿有肌肉虚者,为风毒热气所乘,热毒搏于血气,则皮肤赤而肿起,其风随气行游不定,故名赤游肿也。

《千金方》∶凡小儿赤游行于体上,不治入腹即死。治之方∶用伏龙肝,末,和鸡子白涂,干易之。

《葛氏方》∶糯米研,和粥敷之。

又方∶米粉熬唾,和涂之。

《华佗方》云∶芸苔捣敷之。
治小儿身体肿方第百二十七

《病源论》云∶小儿肿满,由将养不调,肾脾二脏俱虚也。其狭水肿者,皮薄如熟梨(李)之状。若皮肤受风,风搏血气致肿者,但虚肿如吹,此风气肿也。

《产经》云∶少小儿中风,水中身体肿满浴方∶取香以水煮取汁,以渍浴之。良。

又方∶赤小豆煮取汁,渍浴,甚良。

《千金方》云∶小儿手足身体肿方∶以小便温渍之,良。

《僧深方》云∶少小手足身体肿方∶取咸菹汁温渍之。汁味尽易。
治小儿恶核肿方第百二十八

《产经》∶治小儿恶核肿,壮热欲死,升麻汤方∶升麻(一两)夜干(半两)沉香(一分)黄芩(一分)丁子(三铢)凡五物,切,以水一升五合,煮取六合,分三服,一岁儿一服半合,随儿大小,增减水药,神验。
治小儿瘰方第百二十九

《产经》云∶治小儿瘰如梅李海藻酒方∶海藻一斤,切,以酒二斗渍之,稍稍饮之。
治小儿诸方第百三十

《产经》云∶治小儿诸方∶以大膏和胡粉磨疮,良。
治小儿瘿方第百三十一

《产经》云∶治少小瘿瘰,久年不瘥海藻酒方∶海藻一斤,去咸,切,好酒二斗渍,服二合,日二三,酒尽滓干作散,服方寸匕。
治小儿附骨疽方第百三十二

《产经》云∶凡小儿有附骨疽者,招抱才近其身,便大啼唤,即是肢节有痛处,或四肢有不欲动摇,如不随状。治之方∶初得即服漏芦汤下之,敷小豆薄。
治小儿疽方第百三十三

《产经》云∶凡疽喜着指,与代指相似,人不知不忽治,其毒入脏,杀人。更审之,疽着指端者,先作黑,痛入心也。先刺指头,去恶血,以艾灸七壮,良。

又方∶服犀角汁,佳。

又方∶服升麻汁。

又方∶服葵根汁。

又方∶服竹沥汁。

又方∶服蓝青汁。
治小儿代指方第百三十四

《产经》云∶代指者,先肿,欣欣热色不黯也。然后缘爪甲结脓,剧者脱爪也。治之方∶甘草汤热渍之。

又方∶芒硝汁渍之。

又方∶刺去血,渍热汤。

又方∶以猪膏和盐热纳指甲,须臾即安。若已脓者,针去脓血。
治小儿疥疮方第百三十五

《病源论》云∶饶虫多变化所作,其疮里有细虫,甚难见。小儿因乳养之人疥而染着。

《千金方》∶烧竹叶末,和鸡子白敷,日二三。

又方∶酥和胡粉涂之。

又方∶乱发灰和腊月猪脂敷,之。

《产经》∶蜀椒五合,捣末,以水一斗,煮三沸,去滓,温洗浴,良。食顷良,若痒不止,复渍之。食顷良。
治小儿癣疮方第百三十六

《病源论》云∶癣疮由风邪与血气相搏为癣。小儿面上生癣谓之为乳癣,言乳汁潜秽儿面而生,仍以乳汁洗之便瘥。

《产经》∶凡癣不揩破上皮而药涂者不除,用黑毛牛矢温洗之。良。

又方∶末桃白皮,以苦酒和涂之。

又方∶胡粉熬令黄色,和酢涂上,燥复涂,良。

《千金方》湿癣方∶桃青皮,捣,和酢涂之。

又云∶干癣方∶枸杞根,捣,以猪脂和敷之。又以醋和佳。

《集验方》以水银合胡粉敷之。

又方∶蛇床子末和白膏敷之。
治小儿浸淫疮第百三十七

《病源论》云∶小儿五脏有热,外为风湿所折,发疮。

《千金方》∶灶中土二分,发灰一分,末,下筛,猪脂和敷之。

《产经》∶取牛屎,绞取汁,涂之。

又方∶以干牛屎烧熏,良。

又方∶胡燕巢,末,和水敷之。
治小儿疮方第百三十八

《病源论》云∶者,风湿搏于血气所成,多着手足节腕间匝匝然,搔之痒痛,浸淫生长,呼之为,以其疮有细虫,如蜗虫。

《产经》治小儿大人蜗疥百疗不瘥方∶地榆(五两)楝实(一升)桃皮(五两)苦参(五两)凡四物,以水一斗,煮取五升,温洗良,并治癣。

又方∶谷树白皮一合,腊月猪脂一合,苦酒二合,小蒜半合,釜下土半合。凡五物,捣如泥,以敷上,密裹,干复涂之。
治小儿王灼疮方第百三十九

《病源论》云∶腑脏有热,热熏皮肤,外为湿气所乘,则变生疮。其热偏盛者,其疮发热亦盛。初生如麻子,须臾王大,汁流溃烂,如汤火所灼,故名王灼疮。

《千金方》∶小儿王灼疮者,一身尽有如麻子小豆者,戴脓汁出,乍痛乍痒乍热方∶又方∶桃仁熟捣,和面脂,涂之。

又方∶牛KT灰敷之。

又方∶烧艾灰敷之。

《产经》治小儿黄烂疮方∶黄连胡粉上二物,冶下筛,分等,以麻油和涂之。
治小儿月蚀疮方第百四十

《病源论》云∶小儿耳鼻口间生疮,其谓月食疮,其疮随月生死,因以为名也。世云∶小儿见月初生,以手指指之,则令耳下生疮,故呼为月食疮也。

《千金方》∶治小儿疥月蚀月死方∶酥和胡粉涂之。

又云∶月蚀九窍皆有者方∶又方∶烧蚯蚓令赤,屎末和膏敷之。

《葛氏方》∶以五月五日虾蟆屑和膏敷之。

《产经》;取萝摩草汁涂上。

又方∶锉槐枝煮取汁洗之。

《龙门方》∶猪脂和杏仁敷之。

《徐之才方》∶小柏皮捣为末敷之。
治小儿冻疮方第百四十一

《病源论》云∶小儿冬月,为寒气伤于肌肤,寒气搏于血气,血气壅涩,因即生疮,其疮肿而难瘥,乃至皮肉烂,谓之为冻瘃(烂)疮。

《产经》云∶小儿冬月涉水,冻手足,瘃坏疼痛方∶取麦穣,煮(令浓)热,洗渍之(即愈)。

又方∶栎木灰和热汤(取汁,清)洗之,良。

《千金方》∶生胡麻捣敷之。
治小儿漆疮方第百四十二

《病源论》云∶人无问男女大小,有禀性不耐漆者,见漆及新漆器,便着漆毒,令头身面体肿,起隐疹色赤,生疮痒痛是也。

《产经》云∶治小儿犯触漆器,面目皆肿,身体作疮,烂方∶嚼KT涂之。

又方∶芒硝若矾石一一着盐汤中令消,以洗之。

又方∶煮柳叶洗之。

又方∶捣薤以涂之。
治小儿蠼尿疮方第百四十三

《千金方》捣梨叶敷之。

又方∶燕巢土酢浆和敷,干,易。

《产经》云∶初得便以犀角水磨涂上。

又方∶鹿角烧末,苦酒和敷之。

又方∶小豆屑苦酒和敷之。

又方∶大麦饭嚼敷之。

又方∶胡粉以生油和涂之。
治小儿恶疮久不瘥方第百四十四

《病源论》云∶夫身体生疮,皆是脏热冲外,外有风湿相搏所生。而风湿之气有挟热毒者,其疮则痛痒肿,久不瘥,名为恶疮。

《产经》烧蛇蜕末,以猪膏和敷之。

又方∶豆豉熬(令黄),末,敷上。

《集验方》∶浣其父,取汁,以浴儿,勿令儿及母知,大良。

《葛氏方》烧乱发并釜月下土,猪膏和敷之。

又方∶梁上尘敷之。

又方∶黄连、胡粉、水银末,和敷之,若疮燥,和猪肪敷之。

《随时方》∶取桑树孔中水洗之,立瘥。

《范汪方》∶烧鸡屎敷之。

《本草拾遗》云∶厕中泥敷之。
治小儿金疮方第百四十五

《病源论》云∶小儿为金刃所伤,谓之金疮。

《产经》云∶以灶灰敷疮中。

又方∶马屎烧末,着疮中。

又方∶烧绵末,着疮孔中。

又方∶烧青布烟绝敷之。

今按∶地菘敷之。
治小儿汤火灼疮方第百四十六

《千金方》∶熟煮大豆浓汁,温涂之,无瘢。

又方∶白蜜涂,日十遍。(《僧深方》∶十余口之。今按∶《经心方》∶以蜜解,冷水饮之。)今按∶《经心方》以蜜斛(解),冷水饮之。

《产经》∶石膏末敷之立止。

又方∶桑灰水和敷之。
治小儿竹木刺方第百四十七

《产经》治小儿竹木刺及针不出方∶烧鹿角末水和涂疮口立出,不过一夕,良。

又方∶生牛膝根捣敷疮口,令自出。

又方∶嚼白梅,乌梅敷之,水和涂之。

又方∶王不留行,瞿麦随在,服方寸匕,日三,自出。
治小儿落床方第百四十八

《葛氏方》∶治小儿落床堕地,腹中有瘀血、壮热、不欲乳哺、啼唤方∶大黄黄连蒲黄(各二分)芒硝(一分半)以水二升,煮取一升,去滓,纳芒硝,分二三服,当大小便去血。(《产经》同之。)
治小儿食不知饱方第百四十九

《病源论》云∶小儿有嗜食,食已仍不知饱足。又不生肌肉,亦腹大,其大便数而多(泄,又呼为豁泄),此肠胃不守故也。

《千金方》治小儿食不知饱方∶鼠屎二七枚,烧末服之。
治小儿吐食方第百五十

《千金方》∶取肉一斤,绳系曳地行数里,勿洗,火炙啖之,良。
治小儿吐血方第百五十一

《病源论》云∶小儿吐血者,是有热气盛而血虚,热乘于血,血性得热则流散妄行,气逆即血随气上,故吐血也。

《千金方》∶油三分,酒一分,和,日再服之。
治小儿咳嗽方第百五十二

《病源论》云∶小儿咳逆,由乳哺无度,因挟风冷,伤于肺故也。

《产经》治少小十日以上至五十日猝得KT咳,吐乳,呕逆。昼夜不息方∶牡桂(三分)甘草(十分)紫菀(三分)麦门冬(七分)凡四物,以水二升,煮取半升,以绵着汤中,漉儿中口,昼夜四五过与之,即乳哺。

《千金方》云∶主小儿大人咳逆短气,胸中吸吸,吐出涕唾。出臭脓方∶烧淡竹沥,煮十沸,小儿一服一合,日五。大人一升,日五。今按∶大枣丸尤验。(其方在第九卷大人方。)《张文仲方》云∶孩子咳嗽宜与乳母药方∶竹叶(切一升)石膏(碎)干姜(各四两)贝母(三两)紫菀百部根(各二两)上六物,切,以水八升,煮取二升六合,分三服,但乳母禁食蒜面。

《新录方》治小儿嗽方∶饮服紫菀末。

《僧深方》云∶款冬花丸治小儿咳嗽方∶款冬花(六分)紫菀(六分)桂心(二分)伏龙肝(二分)上四物,下筛,蜜和如枣核,着乳以日三夜二。(今按∶以大枣丸治之尤验,其方在大人方中。)
治小儿食鱼骨哽方第百五十三

《产经》云∶治小儿食鱼骨哽方∶以大刀环磨喉二七过,良。

又方∶烧鱼骨末,以水饮之良。

又方∶仍取投地鱼骨着耳上,因KT咳之即出。

又方∶烧鸬羽,末,以水服之,即出。
治小儿食肉骨哽方第百五十四

《产经》云∶治小儿食诸肉,骨哽方∶狸骨末服方寸匕,良。

又方∶取雄鸡左右翮大毛各一枚,烧末,服一刀圭。
治小儿食草芥哽方第百五十五

《产经》云∶治小儿饮食过草介,杂物哽方∶以好蜜少少咽之。

又方∶末瞿麦,服方寸匕。

又方∶以猪膏和鸡子吞之,不去复吞两三过,良。
治小儿饮李、梅辈哽方第百五十六

《产经》云∶治小儿吞李、梅之辈。塞咽不得出方∶以水灌儿头上,承取汁,与饮之良。(《葛氏方》同之。)
治小儿食发绕咽方第百五十七

《产经》云∶治小儿食发浇咽方∶取梳头发烧末,服一钱匕。(《极要方》同之。)
治小儿误吞钱方第百五十八

《产经》云∶治小儿误吞钱方∶捣炭服方寸匕。

又方∶服蜜一升即出。

又方∶艾一把,以水五升,煮取一升,顿服之,立下。
治小儿误吞针方第百五十九

《千金方》治小儿误吞针方∶吞磁石枣大,立出。(今按∶《产经》云∶末,少少服之。)《产经》云∶小儿误吞针、箭、金铁物方∶多食肥羊脂肉及诸肥物,自里出之。
治小儿误吞钓方第百六十

《产经》云∶治小儿误吞钓钩绪若在手中莫引之方∶以珠若薏苡子,贯着绳,稍稍推,令至钓处,少少引之,即出之。
治小儿误吞方第百六十一

《产经》云∶治小儿误吞方∶取韭,曝令萎,煮,不切,食多大束。

又方∶多食白糖,自随出之。
治小儿误吞方第百六十二

《产经》云∶治小儿误吞若驱方∶烧鹰羽数枚,末服之,家所养鹅羽亦可用之。
治小儿误吞竹木方第百六十三

《产经》云∶治小儿误吞竹木方∶取布、刀、故锯,烧染酒中。以女人大指爪甲二枚烧末,纳酒中,饮之良。

医心方卷第二十五《候水镜图》云∶治小儿疮于大人异。

治大丹方∶荞麦面黄盐(各少许)水银牛乳(各少许)上,同研令极细,涂之便瘥。

治赤斑疮方∶土龙子(一条)赤小豆(少许)牙屑(六分)薰陆香(少许加,无以郁金代之)上物,细研,以新汲井水调灌之立瘥。

治蜿豆疮方∶干漆(炒断烟)人中白瓦松灰(各少许)马牙硝(一分)上,并研令细,空心,冷浆水调一钱,温服之,取吐为度。吐了又服一钱,利三五行,其疮自瘥。

治小儿急疳痢泻不止,或浓,或血,或青,或黄发作穗,或头发坠落,鼻干,咬指,吃炭,吃壁土,金髓散方∶黄连一两,(宣州者为末,用鸡子一个取清和连末作饼子,炙焙)石中黄(一分)禹余粮麝香朱砂(各少许)乌头〔二个,生,去脐尖肉豆KT一介(个)〕诃子〔二介(个),去核〕金牙石(一分)上件物,并为散,空心,以米饮下,一岁儿一字,五岁一钱,忌热物。

治小儿紫疳,面模黑色,身上或生青斑、紫斑,鼻内生疮,脑陷,手背、脚背虚肿。不脱肛,不脑陷即椹医。宜服竹天黄丸方∶天竹黄(小一分)朱砂(一小分)巴豆(一粒,去皮心)膜麸(炒压出油)麝香(小许)乌头(一颗,生去脐尖)上五味,细研为末,以蟾酥为丸如黄米大,一岁儿一丸,空心温米饮下如吃奶奶汁下。

忌热面,毒鱼及一切热物,不忌冷物。
卷第二十六
延年方第一

《金匮录》云∶黄帝所受真人中黄直七禽食方。(今按∶《大清经方》七禽散。)黄帝齐于悬辅,以造中黄直。中黄直曰∶子何为者也?黄帝曰∶今弃天下之主,愿闻长生之道。中黄直曰∶子为天下久矣,而复求长生之道,不贪乎?黄帝曰∶有天下实久矣,今欲躬耕而食,深居靖处,禽兽为伍,无烦万民,恐不得其道,敢问治身之要,养生之宝。中黄乃仰而叹曰∶至哉子之问也,吾将造七禽之食,可以长生,与天相保。子其秘之,非贤勿与之。常以七月七日采泽写。泽写者,白鹿之加也,寿八百岁。以八月朔日采柏实,柏实者,猿猴之加也,寿八百岁。以七月七日采蒺藜,蒺藜者,腾蛇之加也,寿二千岁;以八月采芦,奄芦者,之加也,寿二千岁。以八月采地衣,地衣者,车前实也,子陵之加也,寿千岁。以九月采蔓荆实,蔓荆实者,白鹄之加也,寿二千岁。以十一月采彭勃,彭勃者,白蒿也,白菟之加也,寿八百岁。皆阴干,盛瓦器中,封涂无令泄也。正月上辰日冶合下筛,令分等,美枣三倍诸草,美桂一分,置苇囊中无令泄,以三指撮,至食后为饮,服之百日,耳目聪明,夜视有光,气力自倍坚强,常服之,寿蔽天地。

《大清经》云∶黄帝四扇散,仙人茅君语李伟曰∶卿宜服黄帝四扇散方∶松脂泽泻山术干姜云母干地黄石上菖蒲凡七物,精冶合,令分等,合捣四万杵,盛以密器,勿令女人,六畜辈诸染淹者见。旦以酒服三方寸匕,亦可以水服之;亦可以蜜丸散,旦服如大豆者二十丸可至三十丸。此黄帝所授风后四扇神方,却老还少之道也。我昔受之于高丘先生,令以相传耳。

又云∶西王母四童散,茅君语思和曰∶卿宜服王母四童散,其方用∶胡麻(熬)天门冬茯苓山术干黄精桃核中仁(去赤皮)凡六物,精冶,分等,合捣三万杵,旦以酒服三方寸匕,日再。亦可水服,亦可用蜜丸旦服三十丸,日一。此返婴童之秘道也。思和体中损少于李伟,故宜服此方,合药时,皆当清脐,忌熏香,不杂他室。

又云∶淮南子茯苓散,令人身轻,益气力,发白更黑,齿落更生,目冥复明,延年益寿,老而更少方∶茯苓(四两)术(四两)稻米(八斤)凡三物,捣末下筛,服方寸匕二十日,日四。复二十日知,三十日身轻,六十日百病愈,八十日发落更生。有验,百日夜见明,长服延年矣。

又云∶神仙长生不死不老方∶白瓜子(二分)桂(二分)茯苓(四分)天门冬(四分)菖蒲秦椒(各二分)泽泻冬葵(各三分)凡八物,冶下筛,服方寸匕,后食服之。百日欲见鬼神;二百日,司命折去死籍;三百日与鬼神通;六百日,能大能小,能轻能重,志意所为,倡乐自作,玉女来侍也。范蠡服此药年五十为年少;居周秦为白玉;居燕赵为陶朱;居吴楚为范蠡;居汉为东方朔;居江南为鱼父;夏征舒二为皇后,三为夫人。服药如此。变化五六百年良验。

又云∶神仙延年不老作年少方∶茯苓(二分)白菊花(三分)菖蒲(二分)远志(二分)人参(二分)凡五物,冶下筛,以松脂丸,服如鸡子一丸,令人年少,耳目聪明,好颜色。如十五时,至四百岁故以十五时小儿,不可望。复令人不长,若欲试者,取药和之饭与鸡,小儿食之,即不复长。良验秘方。

《金匮录》云∶五茄者,五行之精,五叶同本而外分。故名五者,如五家相邻,比之青雾染茎,禀东方之润;白气营节,资西方之津;赤色注花,含南方之晖;玄精入骨,承北方之液;黄烟熏皮,得戊巳之泽。五种镇生相感而成行之者,升仙服之者,返婴。鲁宣公母单服其酒,以遂不死。

或方∶服五茄散方∶五茄天门冬茯苓桂椒冬葵子六物,分等捣筛,以井花水服一刀圭,交食,日三。三十日勿绝,仿佛见神;五十日司命去死籍;七十日与神通。

《大清经》云∶取五茄削之,令长一寸一升,取一斗,美酒渍之,十日成。温服勿令多也。令人耳目聪明,齿落更生,发白更黑,身体轻强,颜色悦泽。治阴痿,妇人生产余疾,令人多子。取五茄当取雄者,不用雌者也,雄者五叶,味甘;雌者三叶,味苦。

今按∶一说云∶夏用叶茎,冬用根皮,切一升,盛绢袋,以酒一斗渍,春秋七日、夏五日、冬十日,去滓,温服任意勿醉,禁死尸,产妇勿见也。日食五茄不用黄金百库也。

服枸杞方∶《大清经》云∶昔有一人,因使,在河西行。会见一小妇女人打一老公,年可八九十许,使者怪而问之。妇人对曰∶此是我儿之宗孙,家有良药,吾敕遣服之;而不肯服,老病年至不能行来,故以打棒令服药耳。使者下车,长跪而问之曰∶妇人年几何?妇人对曰∶吾年三百七十三岁。使者曰∶药有几种,可得知不?妇人曰∶此药一种,有四名∶春名天精,夏名枸杞,秋名却老,冬名地骨。服法∶正月上寅之日取其根,二月上卯日捣末服之;三月上辰之日取其茎,四月上巳之日捣末服之;五月上午之日取其叶,六月上未之日捣末服之;七月上申之日取其花,八月上酉之日捣末服之;九月上戌之日取其子,十月上亥之日捣末服之;十一月上子之日取其根,十二月上丑之日捣末服之。

其子赤,捣末筛,方寸匕着好酒中,日三服之。十日百病消除;二十日身体强健,益气力,老人丁壮;二百日以上,气力壮,徐行及走马,肤如脂膏。神而有验,千金不传。(亦可作羹茄食,其味小若香,大补益人,《本草》云∶去家十里勿服萝摩、枸杞,言其无妇阴强道故也。)凡服枸杞无所禁,断不用啖蒜,杀药势,不用。食服枸杞,猪脂、精、言,亦不得食。

枸杞酒主诸风痹劳,或大热,或不能饮食,或腹胀,或脚重,或行步目暗,或忘,或失意,或上气,或头痛眩,皆悉主之。其久服者,除病延命,令人聪明、轻身、益气,除寒去热;百日服之,百病悉除,目可独见,耳可独闻;三年以上,乘浮云,驾飞龙。千金莫传。

枸杞一斤(去上蕉皮,取和皮生者)上一物,咀,以酒三斗渍之,冬七日、春秋五日、夏三日,服日三,一合稍增二合,复为散服,作丸服。无所禁,但饮酒不致醉耳。

今按∶加石决明方在治虚劳方中。

服菊方∶《金匮录》云∶南阳郦县山中有甘谷,水甘美,所以尔者。谷上左右皆生菊,菊华堕其中,历世弥久,故水味为变。其临此谷居民,皆不穿井悉食谷水,食谷水者,无不寿。考高者,百四五十岁,下者不失八九十,无夭年者,正是得此菊力也。汉司空王畅、太尉刘宽、太尉袁隗,皆曾为南阳太守。每到官,常使郦县月送甘水三十斛,以为饮食。此诸公多患风痹及眩冒,皆得愈。

《大清经》云∶服菊延年益寿,与天地相守不死方∶春三月甲寅日,日中时采更生,更生者,菊之始生苗也。夏三月丙寅、壬子日,日中时采周盈,周盈一云周成,周成者,菊之茎也。秋三月庚寅日,日晡时采日精,日精者,菊之花也。常以十月戊寅日平旦时采神精。神精者,一曰神华,一曰神英者,菊之实也。无戊寅者壬子亦可用也。冬十一月、十二月壬寅日日入时采长生,长生者菊之根也。一方云∶十一月无壬寅,壬子亦可用也。都合五物,皆令阴干,百日,各舍二分冶合下筛。此上诸月或无应采之日,则用戊寅、戊子、戊辰、壬子日也。春加神精一分,更生二分;夏加周盈三分、长生二分;秋加神精一分、日精二分;冬加日精三分。常以成日合之,无用破,厄日合之也,一方亦不用执日,合药神不行。当于密室中,捣丸用白松脂,如梧子,服七丸,日三,后饭。

服之一年,百病皆去,耳聪目明,身轻益气,增寿二年;服之二年,颜色泽好,气力百倍,白发复黑,齿落复生,增寿三年;服之三年,山行不近蛇龙,鬼神不逢,兵刃不当,飞鸟不敢过其旁,增寿十三年;服之四年,通知神明,增寿三十年;服之五年,身生光明,目照昼夜,有光开梁,交节轻身,虽无翼,意欲飞行;服之六年,增寿三百岁;服之七年,神道欲成,增寿千岁;服之八年,目视千里,耳闻万里,增寿二千年;服之九年,神成能为金石,死后还生,增寿三千年,左有青龙,右有白虎,黄金为车。

服槐子方∶《大清经》云∶槐木者,虚星之精,以十月上巳取子,新瓦瓮盛。又以一瓮盖上密封,三七日发,洗去皮。从月一日起服一枚、二日二枚、三日三枚,如此至十日,日加一计,十日服五十五枚,一月服一百六十五枚,一年服一千九百八十枚,六小月减六十枚。此药主秘补脑,早服之令头不白,好颜色,长生无病。(又云阴干百日。)又云∶阴干百日,捣去皮取子,着瓦器中盛之,欲从一日始日服一枚,十日十枚。复从一日,始满十日,更之如前法。欲治诸平病,留饮,当食不消,胸中之气满,转下下利,一服一合,二合愈,多服无毒。若病患食少勿多服,令人大便刚难。

又云∶三虫法∶取槐子不须上已得取之,并上皮捣令可丸,丸如杏核,一服三丸,日二服。多服、长服尔尔,可以蜜丸之治,延年益寿。一方∶槐子熟者置牛腹中,阴干百日,为饭。旦夕一枚。十日身轻,三十日发白更黑,百日面有光,二百日奔马不及其行。

服莲实鸡头实方∶《大清经》云∶七月七日采藕华七分,八月八日采藕根八分,九月九日采藕实九分,冶合,服方寸匕,日别五度。以八月直、戊日取莲实,九月直、戊日取鸡头实,阴干百日,捣分等,直、戊以井花水服方寸匕。满百日,壮者不老,老者复壮,益气力,养神,不饥,除百病,轻身延年不老,神仙,身色如莲花。

服术方∶《大清经》云∶服术,令人身轻目明,延年益寿,颜色光泽。发白更黑方∶取术好白者,刮去皮,令净、末,下筛,若以酒浆服方寸匕,后食,日三。常使相继,老而更少,气力充盛。弘农人刘景伯服之不废,寿六百岁。八月取之甚好。(服术禁食桃。)或方云∶涓子采术法∶但取术择毕,熟蒸,以釜下汤淋得汁煎之,令如淳染止,不杂他物,经年不坏,随人之多少,令人不老不病,久服不死。神仙服术诸法∶二月三日,取根,曝干,净洗一斛,水三斛,煮咸半,绞去滓,微火煎得五升,纳酒二升,枣膏一升,饴三升,汤上煎可丸,服如鸡子一枚,日并,便利五脏不病,可以山居,行气致神。又法∶术二斛,净洗去皮,熟,捣,以水六斛煮之二日二夜,绞去滓,纳汁釜中,取三升,黍米作粥合得二斛许,微火煎,又下胶饴十斤,此得六升,熟出,置案上暴燥,饼之如小儿状,四万断之,合大如梳,日食三饼,不饥,辄轻身益寿不老。无所禁。
美色方第二

《隋炀帝后宫诸香药方》云∶令身面俱白方∶橘皮(三分)白瓜子(五分)桃花(四分)三物,捣筛,食后酒服方寸匕,日三,三十日知。

《如意方》云∶欲得美色细腰术∶三树桃花阴干,下筛,先饭,日三,服方寸。(今按∶《僧深方》以酒服。)悦面术∶杏仁一升,胡麻去皮捣屑五升,合膏,煎,去滓,纳麻子仁半升更煎。大弹弹正白下之,以脂面,令耐寒白悦光明,致神女下。

《范汪方》治面无色,令人曼泽肥白方∶紫菀(三分,一方五分)白术(五分)细辛(五分)凡三物,为散,酒服方寸匕,十月知之。

令人面目肥白方∶干麦门冬(一升,去心)杏仁(八百枚,去皮生用)凡二物,为丸,先食,酒服如杏仁二丸,日三,十日知之。

令人妩媚白好方∶蜂子(三升,)妇人乳汁(三升)二物,以竹筒盛之,熟,和埋阴垣下,二十日出,以敷面,百日如素矣。

又云∶令人洁白方∶瓜瓣(四分)桃花(四分)橘皮(一分)白芷(二分)孽米(二分)五物,下筛,蜜和如梧子大,酒服五丸,日三,三十日知,百日白矣。

令人面及身体悉洁白方∶七月七日取乌鸡血涂面上便白,远至再三涂。此御方也,身体悉可涂之。

令人妙好,老而少容方∶天门冬(二分)小麦种(一分)车前子(一分)瓜瓣(二分)白石脂(一分)细辛(一分)六物,别冶下筛,候天无云合和搅之,当三指撮饭后服,勿绝。十日身轻,三十日焦理伸,百日白发黑,落齿生,老者复壮,少者不老。东海小童服之传与卢王夫人,年三百岁,恒为好女,神秘。

《灵奇方》云∶练质术∶乌贼鱼骨细辛栝蒌干姜蜀椒(分等)以苦酒渍三日,牛髓一斤,煎黄色,绞去滓,以装面,令白悦,去黑子。(今按∶《范汪方》云∶黑丑人,更鲜好也。)《僧深方》治面令白方∶白瓜子(五两,一方五分)杨白皮(三两,一方三分)桃花(四两,一方四分)上三物,下筛,服方寸匕,食已,日三,欲白加瓜子;欲赤加桃花。服药十日,面白;五十日,手足举体鲜洁也。

《千金方》治虚羸(伦为反)瘦病,令人肥白方∶白蜜(二升)猪肪(一升)胡麻油(半升)干地黄(一升)四味,合煎之可丸,酒服如梧子日三丸,日三。

悦泽颜色方∶酒渍桃花,服之好颜色,除百病,三月三日收。

又方∶白蜜和白茯苓涂,七日愈。

又方∶美酒渍鸡子三枚,密封四七日成,涂面净好无比。

《葛氏方》治人面体黎黑,肤色粗陋,面血浊皮浓。容状丑恶方∶末米,酒和涂面浓粉上,勿令见风,三日即白。(今按∶《范汪方》云∶捣筛,食后服方寸匕。)又方∶白松脂十分,干地黄九分,干漆五分,附子一分,桂心二分。捣末,密和,如梧子,未食服十枚,日三,诸虫悉出,渐举肥白。
芳气方第三

《灵奇方》云∶芳气术∶瓜子芎本当归杜衡(各一分)细辛(二分)防风(八分)白芷桂心(各五分)凡九物,合捣筛,后食服方寸匕,日三。五日口香,十日舌香,二十日肉香,三十日衣香,五十日远闻香。一方无白芷。

《葛氏方》云∶令人身体香方∶白芷薰草杜若薇衡本凡分等,未,蜜和,旦服如梧子三丸,暮四丸。二十日身香。(注也)今按∶《如意方》云∶昔侯昭公服此药坐人上一座悉香。

又方∶甘草、瓜子、大枣、松皮,分等未,食后服方寸匕,日三。〔(注也)今按∶《范汪方》云∶二十日知,百日衣被床帷悉香。〕《如意方》云∶香身术∶瓜子、松皮、大枣,分等末,服方寸匕,日再,衣被香。

《枕中方》云∶道士养性令身香神自归方∶瓜瓣当归细辛本芎(各三分)桂(五分)凡六物,各别捣合,服方寸匕,日三。服之五日,香在口;十日香在舌;二十日香在皮;三十日香在骨;五十日香在气;六十日远闻四方。

《录验方》(九药)云∶熏衣香方∶丁子香藿香零陵香青木香甘松香(各三两)白芷当归桂心槟榔子(各一两)麝香(二分)上十物,细捣,绢筛为粉,以蜜和,捣一千杵,然后出之,丸如枣核,口含咽汁,昼一夜三,日别含十二丸,当日自觉口香;五日自觉体香;十日衣被亦香;二十日逆风行他人闻香;二十五日洗手面水落地香;一月以后,抱儿儿亦香。唯忌蒜及五辛等。不但口香体洁而已,兼亦治万病。(一方∶有香附子。)
益智方第四

《千金方》云∶聪明益智方∶龙骨虎骨(炙)远志三味,分等,食后服方寸匕,日三。

养命开心益智方∶干地黄(三两)人参(三两)茯苓(三两)苁蓉(三两)蛇床子(一分)远志(三分)菟丝子(三分)七味,为散,服方寸匕,日二,忌肉,余无所忌。

开心散主好忘方∶远志(一两)人参(一两)茯苓(二两)菖蒲(一两)四味饮服方寸匕,日三。

又方∶常以甲子日取石上菖蒲一寸九节者,阴干百日下筛,服方寸匕,日三,耳目聪明不忘。

又方∶七月七日,麻勃一升,真人参八两,末之,蒸令气遍,夜欲卧,服一刀圭,尽在四方之事。

又方∶常以五月五日取东桃枝,日未出时作三寸木人着衣带中,令人不忘。

《金遗录》云∶真人开心聪明不忘方∶菖蒲远志(各二十两)茯苓(八两)冶合,服方寸匕,后食,日二三。十日诵经千言,百日万言过是不忘一字。

又方∶菖蒲根远志茯苓(各六分)石苇甘草(各四分)凡五物,捣下筛,后食服方寸匕,日三。十日问一知十。

孔子练精神聪明不忘开心方∶远志(七分)菖蒲(三分)人参(五分)茯苓(五分)龙骨(五分)蒲黄(五分)凡六物,冶合下筛,以旺相日以井花水服方寸匕,日再。二十日闻声知情不忘。

《葛氏方》云∶孔子枕中神效方∶龟甲龙骨远志石上菖蒲分等,末,食后服方寸匕,日三。

〔(注也)今按∶《灵奇方》以茯苓代龟甲。《集验方》∶酒服令(人)大圣。〕治人心孔塞,多忘善误方∶七月七日取蜘蛛网,着衣领中,勿令人知,则不忘。

又方∶丙午日取鳖爪着衣领带中。

又方∶商陆花多采阴干百日捣末,暮水服方寸匕。暮卧思念所欲知事,即于眠中自悟。

《如意方》云∶令人不忘术∶菖蒲、远志、茯苓,分等,末,服方寸匕,日三。

《医门方》云∶开心丸令人不忘方∶菖蒲、茯苓(各三两)人参(二两)远志(四两)蜜丸,后饭服三十丸,丸如梧子,加至四五十丸,恒服之佳。〔(注也)今按∶《集验方》∶散服方。〕《灵奇方》云∶达知术∶取葛花阴干百日,日暮水服方寸匕而卧;心思念所欲为事,卧觉心开而知,或梦中大来吉。
相爱方第五

《千手观音治病合药经》曰∶若有夫妇不和如火水者,取鸳鸯尾于大悲像前咒一千八十偏,身上带彼,是终身欢喜相爱敬。

《龙树方》云∶取鸳鸯心阴干百日,系在臂,勿令人知,即相爱。

又云∶心中爱女无得由者,书其姓名二七枚,以井花水东向正视,日出时服之,必验。

密不传。

《如意方》云∶令人相爱术∶取履下土作三丸,密着席下,佳。

又方∶戊子日,取鹊巢屋下土烧作屑,以酒共服,使夫妇相爱。

又方∶取妇人头发二十枚烧,置所眠床席下,即夫妇相爱。

《灵奇方》云∶取黄土酒和,涂帐内户下方圆一寸,至老相爱。

又方∶取猪皮并尾者,方一寸三分,纳衣领中,天下人皆爱。

又方∶取灶中黄土,以胶汁和着屋上,五日取,涂所欲人衣,即相爱。

又方∶庚辛日取梧桐木东南行根长三寸,克作男人。以五色彩衣之着身,令亲疏相爱。

《枕中方》云∶老子曰∶欲令女人爱,取女人发二十枚烧作灰,酒中服之,甚爱人。

又云∶五月五日,取东引桃枝,日未出时作三寸木人,着衣带中,世人语贵,自然敬爱。

又云∶夫妇相憎之时,以头发埋着灶前,相爱如鸳鸯。

又云∶嫁妇不为夫所爱,取床席下尘着夫食,勿令知,即敬爱。

又云∶孔子曰∶取三井花水作酒饮,令人耐老,常得贵人敬念。复辟兵、虎、野狼。

又云∶人求妇难,得取雄鸡毛二七枚,烧作灰末,着酒中,服必得。

《龙树方》云∶取鸳鸯心阴干百日,系左臂,勿令人知即相爱。

又云∶心中爱女无得由者,书其姓名二七枚。以井花水东向正视,日出时服之必验。密不传。

《延龄经》云∶取未嫁女发十四枚为绳带之,见者,肠断。

又方∶取雄鸡左足爪,未嫁女右手中指爪,烧作灰,敷彼人衣上。

又方∶取己爪,发烧作灰,与彼人饮食中,一日不见如三月。

又方∶蜘蛛一枚,鼠妇子十四枚,上置瓦器中阴干百日,以涂女人衣上,夜必自来。

《陶潜方》云∶戊子日书其姓名着足下,必得。

《如意方》云∶令人相憎术∶取马发,犬毛,置夫妇床中即相憎。

又云∶令人不思术∶远行,怀灶土,不思故乡。

《灵奇方》云∶以桃枝三寸书其姓名埋四会道中,即相憎。

《如意方》止淫术∶三岁白雄鸡两足距烧末,与女人饮之,淫即止。

又云∶欲令淫妇一心方∶取牡荆实与吞之,则一心矣。

又云∶“KTKT(阳符,朱书之入心。)KTKT”(阴符,此欲绝淫情,入肾,朱书之,可服。)此二符,以丹涂竹里白淫令赤,乃以空青书符,吞之淫即绝矣。

又云∶验淫术∶五月五日若七月七日取守宫,张其口,食以丹,视腹下赤,上罂中,阴干百日出,少少冶之,敷女身。拭,终,不去。若有阴阳事便脱。〔(注也)曰∶守宫,蜓也,牝牡新交,三枚良之。〕又方∶白马右足下土,着妇人所卧席床下,勿令知,自呼外夫姓名也。

《延龄经》云∶疗奴有奸事令自道方∶以阿胶、大黄磨敷女衣上,反自说。

《如意方》云∶止妒术∶可以牡薏苡二七枚与吞之。(牡薏苡,相重者是也。)又方∶其月布裹虾蟆一枚,盛着瓮中,盖之,埋厕左则不用夫。

《灵奇方》云∶解怒∶埋其人发于灶前入土三尺,令不怒。

《延龄经》云∶疗奴恶妒方∶取夫脚下土烧,安酒中与服之,取百女亦无言。
求富方第六

《枕中方》云∶烧牛马骨于庭中,令人大贵。

又方∶立春日,取富家土涂仓,立富。

又方∶埋蚕砂着亥地,令家大富。

《如意方》云∶埋牛角宅中,富。

又方∶埋鸟于庭中令富。

又方∶埋鹿鼻舍角致财。

又方∶埋鹿骨门中,厕,得钱。

又方∶五谷各二升埋堂中,聚钱财。

又方∶以李木炭三斤,掘门中三尺,埋之,令富百倍。

又方∶立春日取富家田中土,涂灶。令人得富。

又方∶春甲午乙亥,夏丙辰丁丑,秋庚子辛亥,冬壬寅癸卯,夜半向北斗祝欲得某物,即自得。

又方∶二月上壬日,取道中土,井花水和为泥,涂屋四角,宜蚕。

《灵奇方》云∶欲得人家好田,以戊子日密作买券,埋着田中央,其主必来卖之。

《枕中方》云∶老子曰∶常戊子日买马,巳丑日乘之,令人世世有马不绝。

《如意方》云∶以黄石六十斤置亥子间地及鸡栖下,宜六畜。
断谷方第七

《葛氏方》云∶粒食者,生民之所资。数日令绝,便能致病。《本草》有不饥之文,医方莫言斯术者;当以其涉在仙奇之境,庸俗所能遵故也,遂使荒KT,委尸横路,良可衰乎?今略载其易者云。

又云∶若脱值奔窜在无人之乡及堕涧谷、空井、深冢之中,四顾迥绝,无可苏日者,便应服气法∶开口以舌KT上下齿,取津液而咽之。一日得三百六十咽便佳,渐习乃至千,自然不饥。五三日中小疲极,过此渐轻强。

又云∶若得游涉之地周所山泽间者方∶但取松柏叶细切,水服一二合,日中二三升,便佳。

又方∶掘取白茅根,净洗,切服之。

或方云∶三月三日若十三日,取茅根,曝干服。

又方∶有大豆者,取三升,令光明遍热,以水服之,赤小豆亦佳。

又方∶有术、天门冬、麦门冬、黄精、土贝母,或生或熟,皆可单食。

又方∶树木上白耳及桓榆白皮并皆辟饥。

又云∶若遇荒年谷贵,无以充粮,应预合诸药以济命方。取稻米洮汰之,百蒸曝捣一日食以水得三十日都止,则可终身不食,日行三百里。

又方∶粳米、廪米、小麦、麻子熬、大豆黄卷各五合,捣末,以白蜜一斤,煎一沸,冷水中丸如李,顿吞之,则终身不复饥方。

《陶潜方》云∶避饥方∶青粱米一升,以淳酒渍之三日,百蒸白露,善裹藏之,欲远行入山食之。一食,十日不饥。

《大清经》云∶茯苓,削去黑皮,捣末,以淳酒于瓦器中渍令淹。又瓦器覆上密封,涂十五日,发收饼食,如博棋,日三。亦可取屑,服方寸匕,不饥渴,除百病,延年不老。

又方∶茯苓,水煮数沸,干之。酒渍,渍五六日出,干捣,筛,半升屑纳,熬胡麻末一升,合和,一日服尽之渐渐不饥。

服黄精法∶《大清经》云∶取黄精根,刮去须毛,净洗,细切,使得一斛,以水二斛五斗,煎之,微火,从旦至夕。药熟出,手悉令破,以囊漉之,得汁还着釜中煎令可丸,取其滓,捣作末,纳着釜中,和合相得,药成,宿不食,旦服如鸡子,日三,绝谷不饥。取黄精三月七日为上时矣。(中岳仙人方。)以春取根,洗,切,熟蒸曝干,末服方寸匕。

采黄精,常以八月二日为上时,山中掘而生食,渴饮水,黄精生者捣取汁三升。于汤上煎令可丸,如鸡子。食一枚。日再,二十日不知饥。(老子服方。)服松脂法∶《大清经》云∶取成练者,捣筛,蜜和纳筒中,勿易令见风日,食博棋一枚,日三。不饥,延年,亦可淳酒和服三两至一斤。

服松叶法∶《大清经》云∶四时随壬方面采延上去地丈余者,细切如粟米。若薄粥服二三合,日三。

亦可捣碎曝干,更末服之。亦可捣末酒溲曝干,更捣筛以酒饮及水,无在干不及生。并令人轻身延年,体香少眠,身生绿毛,还白绝谷,不觉饥。

服松实法∶《大清经》云∶七月未开口时啖水沉。取沉者,去皮,末,酒服方寸匕,日三四。亦松脂丸如梧子十丸,日三。服之明眼瞳。一方云∶服之一岁以上白发更黑,身有光。

服松根法∶《大清经》云∶取东行根,剥取白皮,细锉曝燥,捣筛饱食之,可绝谷,渴则饮水。

凡采松柏叶,勿取冢墓上者。当以孟月采,春秋为佳。

服柏脂法∶《大清经》云∶亦同松脂欲绝谷,日服二两至六两,可绝谷,但一两半两耳。

凡服松柏脂,禁食五肉鱼菜盐酱辈,唯得饮水并脯耳。

服柏叶法∶《大清经》云∶但取叶,曝燥为散,蜜丸服之则不饥。亦可水服之,亦可酒服,亦以白酒和散曝干。又捣服益体佳。

服柏实法∶《大清经》云∶八月合房取曝令开圻子脱,水泛取沉者。砻(卢红反)其人,末,酒服二方寸匕,日三。稍增至四五合,绝谷者恣口取饱,渴则饮水。亦可以松脂及白蜜丸,服如梧子十丸、二十丸,日三。亦可加菊花蜜丸服之。

服巨胜法∶《大清经》云∶或方云∶茯苓、泽泻各八两,巨胜一斤,凡三物,捣茯苓泽泻二分下筛。

然有合巨胜捣三千杵,药成丸如梧子。巨胜子一斗二升取纯黑者,茯苓二十两,泽泻八两,冶,三万杵,以水服如弹丸,日三。遇食不食,无食复取。百物无禁,可作务从军涉路,不令(音宵,音所又反)瘦方。言皆冶捣三万杵,熬巨胜令香,亦可蜜丸。

服麻子法∶《大清经》云∶麻子二升,大豆一升,各熬之合则熟香美,去皮,令下筛,捣麻子,令下筛合和,使相得,服一升,日三。水浆无在,务令寒能久之,冬不寒,夏不暑,颜色光泽,气为百倍,走及驷马。时人命尽已独长在,服之,令恒耳。

或方云∶真人断谷,服麻子豆法。取麻子一升,大豆一,皆熬令熟,大豆去皮作末,和合竟食,后服方寸匕,日三。和用酒水,服久令不气力强,日行三百里,大神良。

又法∶麻子二升,大豆二升,各熬令香,捣筛服一升,日三,令不饥,耐寒暑,益气力。

一方用大豆一升,麻子五升,合蜜丸服,以如鸡子一枚。
去三尸方第八

《仙经》曰∶欲求长生,先去三尸,三尸去则志意定,志意定则嗜欲除。三尸在人身,令人多嗜欲喜怒悖恶,令人早死。故仙人服药求仙,先去三尸,三尸不去,则服药无效焉。

《河图纪命符》曰∶天地有司过之神,随人所犯轻重以夺其算,纪。恶事大者夺纪(纪一年也,)小者夺算(算,一日也。)随所犯轻重,所夺有多少也。人受命得寿,自有本数,数本多者,纪算虽尽,故死迟;若所禀本数少而所犯多者,则纪算速尽,而死早也。又人身中有三尸,三尸之为物,实魂、魄、鬼神之属也。欲,使人早死,此尸当得作鬼,自放纵游行餮食人祭,每到六甲穷日辄上天白司命道∶人罪过,过大者夺人纪,小者夺人算。故求仙之人,先去三尸,恬淡无欲,神静性明,积众善,乃服药有益,乃成仙。

《大清经》曰∶三尸,其形颇似人,长三寸许,上尸名彭倨(蒲庚反,音据),黑色,居头,令人好车、马、衣服;中尸名彭质,青色,居背,令人好食五味;下尸曰彭矫(举夭反),白色,居腹,令人好色淫(音逸)。是以真人先去三尸,恬淡无欲,精神清明,然后药乃有效。故庚申日夜半之后,向正南再拜咒曰∶彭侯子彭常子命儿子悉入窈冥之中,去离我身。三度言。每至庚申日勿寝,面呼其名,三户即永绝去。当用六甲穷日者,庚申日也,六甲六十日至庚申日且适(音释),勿寝,皆再拜而呼其字,至鸡鸣,乃去一尸一虫,后庚申日亦用前法三过止,三虫伏尸即永绝去矣。试之皆验。心恒呼此三尸字,即去离我身。三日取桃叶,热烧石令热,以叶着上坐,去三尸。

又云∶真人去伏尸三虫方用∶三月三日取桃叶捣取汁七升,以苦酒合,煎得五合,先食顿服之,令人百病愈。

又方∶道迹神人曰∶欲求长生,先去三尸,欲去之法∶狗脊(七枚)干柒(二两)芜荑(三升)凡三物,末,筛,以水服一合,日再,七日上尸去,九日中尸去,十二日下尸去。其尸形似人,以绵裹之,埋于东流水上尖之,咒曰∶子应属地,我当升天。乃易(年益反)道而归。勿复后顾,三日之中,当苦恍惚,后乃佳。

仙人吉周君曰∶三虫未坏,三尸未死,故导引服气不得其理者是也。三虫,一名青古,一名白古,一名血尸,谓之三虫。令人心烦萧(苏雕反)意志不开,所思不固,失食则饥,悲愁感动,精志不定,仍以服食不能即断也。虽复断谷,人体重,奄(衣俭反)奄淡闷,所梦非真倒错,不除虫在其内摇动五脏故也。故服削虫细丸,以杀谷虫也。

又云∶去三尸酒方∶小麦面十斤,东流水三斗,渍之。春夏二三日,秋冬四五日,视面起,乃绞碎沛去滓,炊稻米五升,依常法纳之,捣细筛,茯苓,天门冬各十斤合和,熟,搅令调,乃以商陆根削皮,薄切五斤,绢囊盛置酿下,封涂,二十日出囊汁纳酒中,去滓,又熬大豆黄卷捣末一升纳酿中,封涂,五日出,服如枣核一枚,日三,至二十日后服如鸡子黄,日三。上尸百日出,中尸六十日出,下尸三十日出。其形颇似人,长三寸许,上尸黑,中尸青,下尸白,即衣五彩藏笥中。明日夜葬之于东流水傍,如冢墓状,悲哭以已名字呼祝之曰∶人生于天,精神受于阳,形骨受于阴,今以精神归于天,形骨归于地。与子长决于无间之野,吾将去子,翱(五劳反)翔(似羊反)于九天之上。毕,即洒身易道而归,勿反顾。常作三日惆(直由反)怅,如失魂魄,过此乃佳。
避寒热方第九

《灵奇方》避寒术∶雄黄、泽泻、椒、附子,分等冶末,井花水服之,冬可单衣。〔(注也)《枕中方》同之。

〕又方∶术三升、防风二升、莨菪子半升,熬之合末,服方寸匕,酒粥无在,连服勿废,日尽一剂,冬不用衣。

又方∶雄黄、丹砂、赤石脂、干姜,各四分,合以白松脂,令如梧子大,日吞四丸,十日止,即不寒,冬日常不欲衣,可入水中。一方∶加桂四分。(《枕中方》同之。)又方∶门冬、茯苓分等末服方匕寸,日二,寒时单衣汗出。《枕中方》同之。

又云∶避热术∶雄黄、白石、黑石脂分等,白松脂丸如小豆,吞五丸,此雌黄丸。

又方∶雄黄、丹砂、赤石脂分等冶和松脂如小豆,名曰∶雄丸,吞雌三丸、雄一丸,不热。

又方∶矾石、白石脂、丹砂、磁石、桂,各四两,和以松脂如小豆、暮吞四丸,夏可重衣。(《枕中方》同之。)
避西雨湿方第十

《如意方》云∶雨不湿衣术∶取蜘蛛置瓦瓮中,食以猪脂百日,杀蜘蛛以涂手巾,大雨不能濡。

又方∶赤腹蜘蛛二七枚,捣取汁以染布巾以覆身,即不沾也。

《灵奇方》云∶不沾法∶蜘蛛涂布巾,天雨不能濡。
避水火方第十一

《得富贵方》云∶大火之精名曰完无,见火畏,呼宗无火即止。大水之精名曰KT像,入水畏,呼KT像,即水不能害人。

又云∶欲入水,手中作王字,又呼弘张。

《抱朴子》云∶得真犀角尺以上则为鱼,而衔以入水中,水常为开三尺,可得气息水中也。

又云∶以葱涕和桂,服如梧子大七丸,日三,满三年则能行水上。

《灵奇方》云∶蜘蛛二七枚,盆盛,食以膏,埋之垣下三十日,以涂足,行水上不没。

又云∶天雄末以涂船头,千里不遇风浪。
避兵刃方第十二

《录验方》(《九药》)云∶入军丸主辟五兵、弓弩、箭矢、诸刀刃令不伤人,若人入山泽能辟虎野狼毒虫,及重财宝贾贩出入辟奸人方∶雄黄(三两)石(二两,炮)矾石(二两,烧)鬼箭(一两)雄柄〔一分(注也∶烧令焦)〕羊角(一分半)灶中灰(二分)凡七物,捣筛,合和,以鸡子中黄并丹雄鸡冠血丸如杏仁。以绛囊盛一丸,系左肘后,辟诸兵刃、虎野狼、盗贼。从军持药,系臂若置腰间,勿令离身。居家中者以药涂门上,辟患害,及涂舍中堂室户,诸邪恶鬼不敢近。若为蛇蜂所中,以一丸涂毒上,立已。数试有验。

《灵奇方》云∶五月五日取梧桐西南向枝长五寸,以为人,螈蚕二枚,并以彩衣之,系在臂,入军不畏流矢也。

又方∶云母、矾石、磁石分等,冶之、合煮以为汤,浴之不畏五兵。
避邪魅方第十三

《延寿赤书》云∶夜行当鸣天鼓无至限数,辟百鬼邪。

(注也)今按∶《大清经》云∶天鼓,谓齿也,兼名菀。云∶宗定伯夜行,道逢鬼,问曰∶我,新鬼,不知鬼法,鬼所畏何物乎?鬼云∶唯畏人唾之。定伯便担鬼项上行,间,鬼唤求下,定伯不听,至于曙下置地,鬼即化为一羊,定伯恐其变形。唾之,即卖得千五百钱。

《抱朴子》云∶古之入山道士,皆以明镜纵横九寸以上悬于背后,能识百邪精魅。

又云∶林卢山下有一亭,其中有鬼,每有宿者,或死或病。后有伯夷者过宿,明灯烛而坐,诵经,至夜半,有十余人与来伯夷对坐共樗蒲博戏,伯夷密以镜照之乃是群犬也。伯夷乃以烛烧来人衣,作焦毛气,伯夷怀小刀捉一人而刺之,初作人声,死而成犬也,余犬悉走。

是镜之力也。

又云∶昔张盍蹋,偶豪成二人,并精思于蜀灵台山石室中。忽有一人暑黄练单衣到前曰∶劳乎?道士,辛苦幽隐于是。二人顾视镜中,乃是鹿也。因问之曰∶汝是草中老鹿也,何诈为人形?来人即成鹿而走去。

又云∶鬼魅知其形,呼其名,不敢犯人。

又云∶山精形如小儿而独足,足向后,来犯人,其名曰。

山中有大树能语者,其精曰云阳,见火光古枯木所作也。夜见胡人者,铜铁之精也。见秦人者百岁木精也。

又云∶山中寅日称虞吏者,虎也。称当路君者,野狼也。言令长者,狸也。

卯日言丈人者,兔也。言东王父者,糜也。言西王母者,鹿也。

辰日言雨师者,龙也。言河伯者,鱼也。言先腹公子者,蟹也。

巳日言时君者,龟也。言寡人者;社中蛇也。

午日言三公者,马也。言仙人者,老树也。

未日言主人者,羊也。

申日言人君者,野狼也。言九卿者,猴也。

酉日言将军者,老鸡也。言贼捕者,雉也。

戌日言人姓者犬也。言成阳翁仲者,狐也。

亥日言臣君者,猪也。妇人字者,金玉也。

子日言社君者,鼠也。言老神人者,伏翼也。

丑日言青生者,豺也。

《得富贵言》云∶雷电精名曰闪(失册反)题;大道精曰庆(都黎反);大山精曰善善;大泽精曰委邪;大树精曰彭候;空室精曰曹羊。

《续齐谐记》云∶燕昭王墓有一斑狸,积年既久,能为幻(胡办反,诈)化,乃变作一书生,遇墓前华表问∶以我才貌可得诣张司空。华表曰∶子之妙解为不可,必辱殆,将不返,非但丧子千年之质,亦当深误。书生不从,诣华,华见洁白如玉,于是商略三史,采贯百家,华应声屈滞乃叹曰∶天下岂有此年少,若非鬼怪,即是狐狸。令人防卫。雷孔章曰∶狗所知者,数百年物耳,千年老精不复能别,唯有千年枯木照之则形见。燕王墓前华表似应千年,遣人伐之,燃之以照,书生乃是一大老狸。

《西王母玉壳丸方》云∶以一丸着头上,行无所畏。又∶至死丧家带一丸,辟百鬼。又∶若独宿林泽中,若冢墓间,烧一丸,百鬼走去。又∶一丸着绯囊中,悬臂男左女右,山精鬼魅皆畏之云云。
避虎野狼方第十四

《得富贵方》云∶欲入山,烧羊角将行,虎野狼皆走避人也。(《集验方》牛羊云云。)又方∶神仙入山须避虎蛇,呼之即失去。虎姓黄子义,见虎。呼∶黄子义,虎则失不见。

《枕中方》∶老子曰∶十一月二日取五谷捣之,合作食至升啖之。勿令示他人,言恒胜虎野狼虫,自然弥伏,常无恶。
辟虫蛇第十五

《得富贵方》云∶蛇字“宜方”,心念之则不见,见蛇呼“宜方”即失不见。

《短剧方》云∶入山草避众蛇方∶干姜生麝香雄黄三物,捣,以小囊带之,男左女右,则蛇逆走避人,不敢近也。人为蛇侵者,可以此治之,大良。今按∶《范汪方》云∶好麝香纳管中带之。

《葛氏方》云∶入山草辟众蛇方∶用八角、附子粗捣之,作三角绛绢囊盛以带头上,蛇不敢近人。

《千金方》入山避众蛇方∶恒烧羊角,使烟出,蛇则去矣。

《耆婆方》避蛇方∶蜈蚣一枚,纳管中带之。

〔(注也)今按∶《本草》陶景注云∶腾蛇游雾而殆于螂蛆。《本经》∶蜈蚣一名螂蛆。〕《灵奇方》避蚊∶桂屑若楝叶屑若蒲,以一升和一斗粉中,以粉身则辟蛟。

又方∶菖蒲花及屑着席下,遣虫虱。

又方∶取初雪三指撮掷置所卧席,勿令人知。

医心方卷第二十六二月采泽泻阴干三十日,车前正月取根、三月取叶、五月取实,阴干百服。防已二月八月庚戌日取干服,赤松子七月十六日去手足,介服中除三尸虫矣。

服谷子法∶谷者□□之精,一名KTKT或作着字,味酸、温、无毒。

服法∶七月七日取赤实阴干,捣筛,服方寸匕,日三。令人不老,视鬼及地中物暗中看书。

服陵阳子三精散方∶天精桑实,地精赤实,人精麻实。

凡三物,分等,加桂肉三尺捣,饭后服方寸匕,三十日通神,百日力自倍。采茯苓法∶松脂沦及地,变为茯苓。数里望树赤,俯视其肥理如博棋KT料者有也,仰视树枝□脂,又观地中高中之下,下中之高,掘入土五寸若一尺,当有限。还向生小松,小松根直下入地,深去七尺,浅者四五尺,便得。又∶上亦时有菟丝者,新雨竟天清无风云,以夜烛火临上,灭者亦即。乃以新布四丈环之,明旦□□□神丞。

服莲芡实方∶八月上戊(一方上卯,)取莲衰实,九月上戊午取鸡头实,九月上午取亿根,各分等阴干百日冶之,正月上卯平旦□□□□饮方寸匕,日四五,后饮服之,百日止。治湿□□□□□补气强,耳目聪明,自轻不饥,成神仙□□手足身面并作莲花色,老瓮成意子。
卷第二十七

养生
大体第一

《千金方》云∶夫养生也者,欲使习以性成,成自为善,不习无利也。性既自善,内外百病皆悉不生,祸乱灾害亦无由作。此其养生之大经也。盖养性者,时则治未病之病,其义也。故养性者不但饵药餐霞,其在兼于百行。百行周备,虽绝药饵,足以遐年;德行不充,纵玉酒金丹,未能延寿。故老子曰∶善摄生者,陆行不畏虎光,此则道德之也。岂假服饵而祈遐年哉。

《文子》云∶太上养神,其次养形,神清意平,百节皆宁,养生之本也。肥肌肤,充腹肠,开嗜欲,养生之末也。

《养生要集》云∶《神仙图》云∶夫为长生之术,常当存之行止坐起,饮食卧息,诸便皆思,昼夜不忘。保安精、气、神,不离身则长生。

又云∶《中经》云∶夫禀五常之气,有静躁刚柔之性,不可易也。静者可令躁,躁者不可令静,静躁各有其性,违之则失其分,恣之则害其生。故静之弊,在不开通;躁之弊,在不精密。治生之道,顺其性,因其分,使拆引随宜,损益以渐,则各得其适矣。静者寿,躁者夭。静而不能养,减寿;躁而能养,延年。然静易御,躁难将,顺养之宜者,静亦可养,躁亦可养也。

又云∶大计奢嫩者寿,悭勤者夭,放散劬(拥俱反)KT(良刃反)之异也。佃夫寿,膏粱夭,嗜欲少多之验也。处士少疾,游子多患,事务烦简之殊也。故俗人觅利,道人罕营。

又云∶《少有经》云∶少思,少念,少欲,少事,少语,少笑,少愁,少乐,少喜,少怒,少好,少恶,行此十二少,养生之都契也。多思即神殆,多念则志散,多欲则损智,多事则形疲,多语则气争,多笑则伤脏。多愁则心摄,多乐则意溢,多喜则忘错昏乱,多怒则百脉不定,多好则专迷不治,多恶则焦煎无欢。此十二多不除,丧生之本,无少无多者,几于真人也。

又云∶彭祖曰∶养寿之法,但莫伤之而已。夫冬温夏凉,不失四时之和,所以适身也。

美色HT姿,幽闲娱乐,不致思欲之感,所以通神也。车马威仪,知足无求,所以一志也。

八音五色,以玩视听之欢,所以导心也。凡此皆所以养寿而不能酌之者,反以迷患。故至人恐流遁不反,乃绝其源。故言∶上士别床,中士别被,服药百果,不如独卧。色使目盲,声使耳聋,味令口爽之。苟能节宣其适,拆扬其通塞者,不以灭耳,而得其益。

又云∶彭祖曰∶重衣浓褥,体不堪苦,以致风寒之疾。浓味脯腊,醉饱厌饭,以致疝结之病。美色妖丽,媚外家盈房,以致虚损之祸。淫声袅音,移心悦耳,以致荒耻之惑。驰骋游观,弋猎原野,以致发狂之失。谋得战胜,乘弱取乱,以致骄逸之败。盖贤圣戒失其理者也,然此养生之具,譬犹水火,不可失适,反为害耳。

又云∶仲长统曰∶北方寒,而其人寿;南方暑,而其人夭。此寒暑之方验于人也。均之蚕也,寒而饥之,则引日多;温而饱之,则用日少。此寒暑饥饱为修短验乎物者也。婴儿之生,衣之新纩则骨蒸焉,食之鱼肉则虫生焉,串之逸乐则易伤焉。此寒苦动移之使乎性也。

又云∶道机曰∶人生而命有长短者,非自然也。皆由将身不慎,饮食过差,淫无度,忤逆阴阳,魂魄神散,精竭命衰,百病萌生,故不终其寿也。

《稽康养生论》云∶养生有五难,名利不去,一难也;喜怒不除,二难也;声色不去,三难也;滋味不绝,四难也;神虑精散,五难也。五者必存,虽心希难老,口诵至言,咀嚼英华,呼吸大阳,不能不曲其操,不夭其年也。五者无于胸中,则信顺日济,玄德日全,不祈喜而有福,不求寿而自延。此亦养生之大经也。然或有服膺仁义,无甚泰之累者,抑亦其亚也。

又云∶夫神仙虽不目见,然记籍所载。前史所所传,较而论其有必矣。似特受异气,禀之自然,非积学所能致也。至于道养得理,以尽性命,上获千余岁,下可数百年,可有之耳。

而世皆不精,故莫能得之,何以言之?夫服药求汗,或有不得而愧,情一集则涣然流离,终朝不食,嚣然思食,而曾子衔袅,七日不饥。夜分而坐,则低迷思寝。内怀殿忧,则达旦不瞑,则劲刷理鬓,醇醪发颜,仅乃得之。壮士之怒,赫然殊观,植发冲冠。由此言之,精神之于形骸,犹国之有君也。神躁于中而形丧于外,犹君昏于上而国乱于下也。夫为稼于汤世,偏有一溉之功者,虽终归焦烂,必有一溉者后枯。然则一溉之益,固不可诬矣。而世常谓一怒不足以侵性,一袅不足以伤身,轻而肆之,是犹不识一溉之益而望嘉禾于旱苗者也。是以君子知形恃神以立神,须形以存悟(性)。生理之易失,知一过之害生,故修性以保神,安心以全身,爱憎不凄于情,忧喜不留于意,泊然无感(戚)而体气和平。

又云∶嗜欲虽出于人情,而非道德之正。犹木之有蝎,虽木所生而非木所宜,故蝎盛则木,欲胜则身枯,然则欲与生,不并立;名与身,不俱存,略可知矣。

又云∶养性有五难,名利不灭,此一难也;喜怒不除,此二难也;声色不去,此三难也;滋味不绝,此四难也;神虑精散,此五难也。五者必存,虽心希难老者,口诵至言,咀嚼英华,呼吸大阳,不能不曲其操,不夭其年也。五者无于胸中,则信顺日济,玄德日全,不祈而有福;不求寿而自延,此养生大理之所都也。

《抱朴子》云∶诸求长生者,必欲积善立功,慈心于物,怒己及人,仁逮昆虫,乐人之吉,愍人之凶,周人之急,救人之穷。手不伤杀,口不劝福,见人有得,如己之得。见人有失,如己有失。不自贵,不自誉,不嫉妒胜己,不佞谄阴贼,如此乃为有德。受福于天,所作必成,求仙可冀也。

又云∶夫五声八音,清商流征,损聪者鲜藻艳彩,丽炳烂焕,伤明者也。宴安逸豫,清醪芳醴,乱性者。冶容媚姿,红华素质,伐命者也。

又云∶夫损,易知而速焉;益,难知而迟焉。尚不悟其易,且安能识其难哉。夫损者如灯火之消脂,莫之见而忽尽矣。益者如苗木之播殖,莫之觉而忽茂矣。故治身养性,务谨其细,不可以小益为不猝而不修,不可以小损为无伤而不防。凡聚小所以就大积,一所以至亿也。若能爱(受)之于微,成之于着者,则几乎知道矣。

《庄子》云∶善养生者,若牧羊者。然视其后者而鞭之。鲁有单(唐韵∶时□反,姓也)豹者,严居而水饮,不与民共利,行年七十而犹有婴儿之色,不幸遇饿虎,饿虎杀而食之。

有张毅者,高门悬薄,无不趋也。行年四十而有内热之病以死。豹养其内而虎食其外,毅养其外而病攻其内,此二子者,皆不鞭其后者也。(鞭其后者,去其不及也。)《吕氏春秋》云∶圣人养生适性,室大则多阴,台高则多阴(阳)。多阳生,多阴则生痿,皆不适之患也。味众肉充则中气不达,衣热则理塞,理塞则气不固,此皆伤生也。故圣人为苑囿园池,足以观望劳形而已矣,为宫观台榭,足以避燥湿。为舆马衣裘逸身,足以暖骸而已;为饮醴,足以适味充虚;为声色音乐,足以安生自娱而已。五者,圣人所养生也。

又云∶靡曼皓齿,郑卫之音,务以自乐,命曰伐命之斧。肥肉浓酒,务以相强,命曰烂肠之食。(靡曼∶细理弱肌美也。皓齿,所谓齿如瓠犀也。《老子》云∶五味实口,使口爽伤,故谓之烂肠之食。)《颜氏家训》云∶夫养生者,先须虑祸求福,全身保性,有此生然后养之,勿徒养其无生也。单豹养于内而丧外,张毅养于外而丧内,前贤之所诫也。稽康着养生之论,而以傲物受刑;石崇冀服饵之延,而以贪溺取祸,往世之所迷也。
谷神第二

《老子道经》云∶谷神不死。(谷,养也。人能养神则不死也。神谓五脏之神。肝藏魂,肺藏魄,心藏神,肾藏精,脾藏志,五脏尽伤则五神去也。)是谓玄牝。(言不死之道在于玄牝。玄,天也,于人为鼻;牝,地也,于人为口。天食人以五气,从鼻入脏于心,五气精微。

为精神聪明音声五性,其鬼曰魂,魂者雄也,主出入于鼻,与天通气,故鼻为玄也。地食人以五味,从口入藏于胃,五味浊辱,为形骸骨肉血脉六情,其鬼曰魄,魄者雌也,主入出于口,与地通气,故口为牝也。)玄牝之门,是谓天地之根。〔根,元也。言鼻口之门是为(乃)天地之元气所从往来也。〕绵绵乎若存。(鼻口嘘吸喘息,当绵绵微妙,若可存。复若无有也。)用之不勤。(用气当宽舒。不,当急疾,勤,劳也。)《史记》云∶人所以生者,神也;所者形也。神大用则竭,形大劳则蔽,形神离则死,故圣人重之。由是观之,神者生之本也,形者神之具也,不先定其神。而曰我有以治天下,何由乎?《抱朴子》云∶夫有因无而生焉,形须神而立焉,有者无之宫也,形者神之宅也,故譬之于堤,堤坏则水不留矣。方之于烛,烛麋(糜)则火不居矣。身劳则神散,气竭则命终,根拔蝎繁,则青青去木矣。器疲欲胜,则精灵离逝矣。

《养生要集》云∶颖川胡照(昭)孔明云∶目不欲视不正色,耳不欲听丑秽声,鼻不欲(嗅)腥气,口不欲尝毒刺味,心不欲谋欺诈事,此辱神损寿。又居常而叹息,晨夜吟啸于正来邪矣。夫常人不得无欲,又复不得无事,但常和心约念静身损物,先去乱神犯性者,此即啬(所力反)神之一术也。

又云∶钜鹿张子明曰∶思念不欲专,亦不欲散,专则愚惑,散则佚荡。又读书致思,损性尤深。不能不读。当读己所解者,己所不解而思之不已,非但损寿,或中戆(《玉》都绛反,愚也;)疣(涉降反,愚也,音尤)失志,或怅恍不治,甚者失性,世谓之经逸。

《延寿赤书》云∶三魂名∶爽灵、胎光、幽精。七魄神名∶尸苟、伏矢、雀阴、吞贼、非毒、除秽、臭肺。(巳上名夜半五更诵两遍,魂魄不离形神也。)五脏神名∶心神赤子字朱灵,肺神诰华字虚成,肝神龙烟字含明,肾神玄KT字育婴,脾神常在字魂庭。(以上神名,日别诵之,神不离形也。)《圣记经》云∶人身中有三无宫也,当两眉却入三寸为泥丸宫,此上丹田也;中有赤子字符先,一名帝卿人,长三寸,紫衣也。中心为绛宫,此中丹田也;其中真人字子丹,一名光坚,赤衣也。脐下三寸为命门宫,此下丹田也;其中婴儿字符阳,一名谷玄黄衣也,皆如婴儿之状。凡欲拘制魂魄,先阴呼其名,并存服色,令次第分明。
养形第三

《素问》云∶春三月,此谓发陈,天地俱生,万物以荣,夜卧蚤起,广步于庭,被发缓形,以使志生,生而勿杀,与而勿夺,赏而勿罚,此春气之应也。养生之道也。

夏三月,此谓蕃秀,天地气交,万物华实,夜卧早起,毋厌于日,使志莫怒,使英华成秀,使气得泄,若所爱在外,此夏气之应也,养生之道也。

秋三月,此谓容平。天气以急,地气以明,早卧早起,与鸡共兴,使志安宁,以缓秋刑。

收敛神气,使秋气平;毋外其志,使肺气精。此秋气之应也。养收之道也。

冬三月,此谓气闭藏。水冰地坼,毋扰于阳。早卧晚起,必待日光。使志若伏匿,若有私意,若已有德,去寒就温,毋泄皮肤,便(使)气极。此冬气之应也。养藏之道也。

夫四时阴阳者,万物之根气也。所以圣人春夏养阳,秋冬养阴,以顺其根。故与万物沉浮于生长之门。

《圣记经》云∶夫一日之道,朝饱暮饥;一月之道,不失盛衰;一岁之道,夏瘦冬肥;百岁之道,节谷食米。千岁之道,独男无女。是谓长生久视。

《养生要集》云∶青牛道士云∶人不欲使乐,乐人不寿。但当莫强健为其气力所不任,举重引强掘地,若作倦而不息,以致筋骨疲竭耳。然过于劳苦,远胜过于逸乐也。能从朝至暮常有所为,使足不息乃快。但觉极当息,息复为,乃与导引无异也。夫流水不垢,户枢不腐者,以其劳动之数故也。

又云∶《中经》曰∶人常欲数照镜,谓之存形。形与神相存,此照镜也。若务容色自爱玩,不如勿照也。

又云∶大汗出,急敷粉。着汗湿衣,令人得疮,大小便不利。

《养生志》云∶KT热来勿以水临面,若临面不久成痫,或起即头眩。

又云∶足汗入水,令人作骨痹病,凶。

《千金方》云∶人欲少劳,但莫大疲及强所不能堪耳。

又云∶凡大汗,勿即脱衣。喜得偏风,半身不遂。

又云∶人汗勿跋床悬脚,久成血痹,两足重,腰疼。

又云∶每至八月一日以后,即微火暖足,勿令下冷。(先生意常欲使气在下,勿欲泄之于上。)又云∶冬日温足冻KT,春秋KT足俱冻,此凡人常法。

又云∶勿举足向火。

又云∶忍尿不便,膝冷成痹。出(忍)大便不出,成气痔。

又云∶久坐立尿,久立坐尿。

又云∶人饥,须坐小便。若饱,立小便。慎之。无病,除虚损。

又云∶小便勿怒,令两足及膝冷。

又云∶大便不用呼气,乃强怒。令人腰疼目涩,宜任之。

《眼论》云∶夫欲治眼,不问轻重,悉不得涉风霜、雨水、寒热、虚损、大劳并及房室。

饮食禁忌悉不得犯。

《千金方》云∶凡少时不自将慎,年至四十即渐渐眼暗。若能依此将慎,可得白首。无他,所以人年四十以去,恒须KT目,非有要事不肯辄开,此之一术,护慎之极也。

又云∶生食五辛,接热食饮,刺头,出血过多。

极目视夜读细书久处烟火抄写多年博弈不休日没后读书饮酒不已热食面食雕镂细作泣泪过度房室无节夜远视星火数向日月轮看月中读书极目瞻视山川草木上一十七件(其辈反,分也),并是丧明之由。养生之士,宜熟慎焉。

又云∶有驰骋田猎。冒涉霜雪,迎风追兽,日夜不息者,亦是伤目之媒也。

又云∶凡旦起勿开目洗,令目涩,失明、饶泪。

又云∶凡熊、猪二脂不作灯火,烟气入目光,不能远视。

《养生要集》云∶《中经》云∶以冷水洗目,引热气,令人目早瞑。

《养性志》云∶日月勿正怒目久视之,令人早失其明。

《靳邵服石论》云∶凡浇(洗)头勿使头垢汁入目中,令人目早瞑。

《晋书》云∶范宁字武子,患目痛,就张湛求方。答云治以六物∶损读书一,减思虑二,专内视三,简外观四,旦晚起五,夜早眠六。凡六物,熬以神火,下以气筛,蕴于胸中七日。

然后纳诸方寸修之,非但明目,乃亦延年。

《养生要集》云∶《中经》曰∶发,血之穷也。千过梳发,发不白。

《千金方》云∶凡旦,欲得食讫,然后洗梳也。

《唐临香港脚论》云∶数须用梳拢头,每梳发欲得一百余,梳亦大去气。

《延寿赤书》云∶《大极经》曰∶理发宜向壬地。当数易栉,栉处多而不使痛。亦可令侍者栉之。取多佳也。于是血流不滞,发根当坚。(令侍者濯手,然令栉。不然污天宫也。)又云∶《真诰》曰∶栉发欲得弘多。通血气,散风湿也。数易栉逾良。

又云∶《丹景经》曰∶以手更摩发反理栉,但热,令发不白也。

《太素经》杨上善注云∶齿为骨余,以杨枝若物资齿,则齿鲜好也。

《养生要集》云∶《中经》曰∶齿,骨之穷也。朝夕喙齿,齿不龋。

又云∶食毕当嗽口数过,不尔令人病齿龋(丘禹反。《说文》云∶齿蠹也。)又云∶水银近牙齿,发龈肿,喜落齿。

《颜氏家训》云∶吾尝患齿动摇欲落,饮热食冷皆苦疼痛,见《抱朴子》云牢齿之法,旦朝建齿三百下为良。行之数日,即便平愈。至今恒将之。此辈小术,无损于事,亦可修之。

《千金方》云∶食毕当嗽口数过,令人牙齿不败,口香。

《延寿赤书》云∶郑都记曰∶夜行当鸣天鼓,无至限数也。辟百鬼邪。凡鬼畏齿之声,是故不得犯人。(今按∶《大清经》云∶天鼓谓齿也。)《养生要集》云∶《中经》曰∶爪,筋之穷也。爪不数截筋不替。

《千金方》云∶凡寅日剪手甲,午日剪足甲。(今按∶《唐临香港脚论》云∶丑日手甲、寅日足甲割之。)《养生要集》云∶《中经》曰∶人不欲数沐浴,数沐浴动血脉,引外气。

又云∶饱食即沐发者,作头风病。

又云∶青牛道士曰∶汗出不露卧及澡浴,使人身振及寒热,或作风疹。

又云∶新沐头未干不可以卧,使人头重身热,及得头风烦满。

又云∶《抱朴子》云∶月宿东井日可沐浴。令人长生无病。

又云∶正月十日人定时,二月八日黄昏时,三月六日日入时,四月七日日时,五月一日日中时,六月二十七日日食时,七月二十五日小食时,八月二十五日日出时,九月二十日鸡三鸣时,十月十八日鸡始鸣时,十一月十五日过夜时,十二月十三日夜半时,闰月视日入中时。可沐浴,得神明恩,除百病。

又云∶道士斋戒,沐浴兰菊花汤,令人老寿。

又云∶常以春三月旦沐更生,夏三月旦沐周盈,秋三月旦沐日精,冬三月旦沐长生。常用阴日沐浴之。增寿三百年,谓不服但沐浴也。服之者,延寿无已。(今按∶《大清经》云∶更生者,菊之始生苗也。周盈者,菊之茎也;日精者,菊华也。长生者,菊根也。又《虾蟆经》云∶甲丙戊庚壬皆阳日也,乙丁己辛癸皆阴日也。)又云∶凡人常以正月二日、二月三日、三月六日、四月八日、五月一日、六月二十一日、七月七日、八月八日、九月二十日、十月八日、十一月二十日、十二月三十日取枸杞煮汤沐浴又云∶居家不欲数沐浴,必须密室,室不得大热,亦不得大冷,大热大冷皆生百病。冬沐不又云∶新沐讫,勿以当风,勿以湿结之,勿以湿头卧,使人得头风、眩闷、发颓、面齿痛、又云∶热泔洗头,冷水灌,亦作头风。饱饥沐发,亦作头风。夜沐发,不食而卧,令心虚、《养生志》云∶诸深山有陂,水久停者,喜有沙虱,不中沐浴。

《养生要集》云∶凡远行途中,逢河水勿洗面,生为。(状如乌卵之色斑也。)
用气第四

《抱朴子》云∶一人之身,一国之象也。胸腹之位,犹宫室也;四肢之列,犹郊境也;骨节之分,犹百官也;神犹君也。血犹臣也,气犹民也。故知治身则能治国也。夫爱其民,所以安其国,KT其气,所以全其身。民散则国亡,气竭则身死也。是以至人修未起之患,治未病之疾,医之于无事之前不追之于既逝之后。民难养而易危也,气难清而易独也。故审威德所以保社稷,割嗜欲所以固血气,然后真一存焉,三七守焉,百害却焉,年寿延焉。

《养生要集》云∶卤公云∶人在气中,如鱼在水,水浊则鱼瘦,气昏则人疾,浊者非独,天气昏浊,但思虑萦心,得失交丧,引兼蹇,亦名为浊也。

又云∶彭祖云∶人之爱气虽不知方术,但养之得宜,常寿百四十岁。不得此者,皆伤之也。小复晓道,可得二百四十岁,复能加之,可至四百八十岁。

又云∶《服气经》云∶道者气也,宝气则得道,得道则长存。神者精也,宝精则神明,神明则长生。精者,血脉之川流,守骨灵神也。精去则骨枯,骨枯则死矣。是以为道者务宝其精,从夜半至日中为生气,从日中至夜半为死气,常以生气时正偃卧,瞑目握固,(握固者如婴儿之方卷手,)闭气不息,于心中数至二百,乃口吐气,出之,日增息,如此,身神俱,五脏安,能闭气。数至二百五十,华盖美,(华盖者,眉也。)耳目聪明,举身无病,邪不干人也。

又云∶行气者,先除鼻中毛,所谓通神路,常人又利喘也。

又云∶行气、闭气虽是治身之要,然当先达解其理空,又宜虚,不可饱满。若气有结滞,不得宣流,或致发疮,譬如泉源,不可壅遏不通。若食生鱼、生虫、生菜、肥肉,及喜怒忧恚不除而行气,令人发上气。凡欲修此,皆以渐。

又云∶《元阳经》云∶常以鼻纳气含而嗽,漏舌KT肤齿咽之。一日一夜得千咽甚良。

当少饮食,饮食多气逆,百脉闭,闭则气不行,气不行则生病也。

又云∶《老子》尹氏内解曰∶唾者,凑为醴泉,聚为玉浆,流为华池,散为精液,降为甘露,故口为华池,中有醴泉,嗽而咽之,溉脏润身,流利百脉,化养万神,支节毛发,宗之而生也。

又云∶《养生内解》云∶人能终日不唾,含枣而咽之,令人爱气生津液,此大要也。

又云∶刘君安曰∶食生吐死,可以长存。谓鼻纳气为生,口吐气为死。凡人不能服气,从朝至暮,常习不息,修而舒之。

又∶常令鼻纳口吐,所谓吐故纳新也。现世人有能以鼻吹笙、以鼻饮酒者,积习所能,则鼻能为口,之所为者今习以口吐鼻纳,尤易鼻吹鼻饮也。但人不能习,习不能久耳。

又云∶彭祖云∶和神导气之道,当得密室闲房,安床暖席,枕高二寸五分,正身偃卧眠,目闭气息于胸膈,以鸿毛着鼻口上而鸿毛不动,经三百息,耳无所闻,目无所见,心无所思,寒暑不能害,蜂虿不能毒,寿三百六十,此真人也。若不能元思虑,当以渐除之耳。不能猥闭之,稍稍学之,起于三息五息七息九息而一舒气,寻复之,能十二息不舒,是小通也。

百(二十)不息是大通也。

又云∶当以夜半之后、生气之时,闭气以心中数,数令间不容间,恐有误乱,可并以手下筹,能至千则去仙不远矣。吐气令入多出少,常以鼻取之,口吐之。

又云∶若天雾、恶风、猛寒、大热,勿取气,但闭之而已。微吐寻复闭之。

又云∶行气欲除百病,随病所有念之。头痛念头,足痛念足,使其愈和,气往攻之。从时至时,便自消矣。此养生大要也。

《大清经》云∶夫气之为理,有内有外,有阴有阳。阳气为生,阴气为死。从夜半至日中,外为生气;从日中至夜半,内为死气。

凡服气者,常应服生气。死气伤人,外气生时,随欲服便服,不必待当时也。取外气法∶鼻引生气入,口吐死气出,慎不可逆。逆则伤人。口入鼻出,谓之逆也。从日中至夜半生气在内,服法∶闭口,目如常,喘息令息出至鼻端,即鼓两颊,引出息,还入口,满口而咽,以足为度,不须吐也。

又云∶甘始服六戊法∶常以朝暮,先甲子,旬起向辰地,舌KT上下齿,取津液周旋,三至而一咽,五咽止。次向寅,亦如之。周于六戊,凡三十咽止。

又云∶俭服六戊法∶起甲子,日竟,旬恒向戊辰,咽气甲戌,旬则恒向戊寅,咽六旬,效此。

又云∶服五星精法∶春夏秋冬及四季月,各向建各存其星气,大如指,随其色来入口,又各存脏中,色气亦如此。上退场门,便含咽吞之,复更吸吞数毕止,日三,初三九,次三七,后三五也。春平旦,次日中日入;夏日出日入;秋日入人定鸡鸣;冬夜半日出日中,一云日入。四季各依王时,起至间中三七至冲并舒手足,张口此之,时三五。

又云∶取气法,从鼻中引入中口吐出,慎不可逆,逆则伤人。口入鼻出之逆也。服法∶正身作卧,下枕,令与身平,握固,以四指杞大指握固也。要令床敷浓襦,平正直身,两脚相去五寸,舒两臂,令去身各五寸。安身体,定气息,放身如委衣床上谓之大委气法也。然后徐徐鼻中引气,鼓两颊,令起,徐徐微引气入颊中,亦勿令顿满也。满则还出,出则咽难,恒令内虚,虚则复得更引。若气先调者,微七引入口一咽气;先未调者,五引可咽,三引亦可咽,咽时小动舌,令气转,然后咽,咽时勿使鼻中气泄也。气泄则损人。

又云∶取气时僵卧,直两手脚,握固,两脚相去、两手各去身五寸。闭目闭口,鼻中引气,从夜半初服九九八十一咽,鸡鸣八八六十四咽,平旦七七四十九咽,日出六六三十六咽,食时五五二十五咽,禺中四四十六咽。

又云∶初服气,气兼未调,量能否,应一引一咽一吐,或二咽一吐,或三咽一吐。若气小调,三引一咽一吐,或二咽一吐,或三咽一吐。气渐渐调,五引一咽一吐,或二咽一吐,或三咽一吐,居平好也。又七引一咽一吐,或二咽一吐至三咽一吐,此气极调善也。

又云∶凡服气及苻水断谷,皆须山居静处,安心定意,不可令人猝有犯触而致惊忤者,皆多失心。初为之十日二十日,疲极消瘦,头眩足弱,过此乃渐渐胜耳。若兼之以药物,则不乃虚也。例不欲多言,笑举动亡精费气,最为大忌。

《千金方》云∶调气方∶治万病大患,百日即生眉发也。凡调气之法,夜半后、日中前,气生得调,日中后、夜半前,气死不得调。调气时,仰卧,床铺浓软,枕高下其身平,舒手展脚,两手握大母指节,去身四五寸,两脚相去四五寸。引气从鼻入足即停止,有力更取久住。气闭从口细细吐出,尽还从鼻细细引入。

又云∶每旦初起,面向午展两手,于膝上,心眼观气入顶下,达涌泉,旦旦如此,名曰送气,常以鼻引气,口吐气。(微吐不得开口。)复欲得出气少,入气多。
导引第五

《养生要集》云∶《宁先生导引经》云∶所以导引,令人肢体骨节中诸恶气皆去,正气存处矣。

《太素经》杨上善云∶导引谓熊颈鸟伸、五禽戏等,近愈痿万病,远取长生久视也。

《华佗别传》云∶佗尝语吴普云∶人欲得劳动,但不当自极耳。体常动摇,谷气得消,血脉流通,疾则不生。卿见户枢虽用易腐之木,朝暮开闭动摇,遂最晚朽。是以古之仙者赤松、彭祖之为导引,盖取于此。

《养生要集》云∶率导引常候天阳和温、日月清静时,可入室。甚寒、甚暑,不可以导引。

又云∶《导引经》云∶凡导引调气养生,宜日别三时为之。谓卯、午、酉时,临欲导引,宜先洁清。

又云∶道人刘京云∶人当朝朝服玉泉,使人丁壮,有颜色,去虫而坚齿。玉泉者,口中唾也。朝未起早嗽漏之满口乃吞之。辄辄喙齿二七过,如此者,二乃止,名曰练精。

又云∶《养生内解》云∶常以向晨摩指,少阳令热,以熨目,满二七止。

又云∶常以黄昏指目四,名曰存神光满。

又云∶拘魂门、制魄户,名曰握固。令人魂魄安。魂门魄户者,两手大母指本内近爪甲也。

此固精、明目、留年、还白之法。若能终日握之,邪气百毒不得入。(握固法∶屈大母指着四小指内抱之。积习不止,眠中亦不复开。一说云令人不厌魅。)又云∶常以向晨摩目毕喙齿三十六下,以舌熟KT二七过,嗽漏口中津液,满口咽之。

三过止,亦可二七喙齿,一喙一咽,满三止。

又云∶旦起东向坐,以两手相摩令热,以手摩额上至顶上,满二九止,名曰存泥丸。

又云∶清旦初起,以两手叉两耳,极上下之,二七之,令人耳不聋。

又云∶摩手令热,以摩面。从上下,止邪气,令面有光。

又云∶令人摩手令热,当摩身体,从上至下,名曰干浴,令人胜风寒时气热头痛疾皆除。

《服气导引抄》云∶卧起先以手巾若浓帛拭项中、四面及耳后,皆使员匝温温然也。顺发摩头,若理栉之,无在也。(谓卧初起先宜向壬行此法,竟乃为KT手及诸事。)《千金方》云∶自按摩法∶日三遍,一月后百病并除,行及走马,此是婆罗门法。一两手相捉向戾如洗手法。一两手浅相叉,翻覆向胸;一两手相捉,共按髀(左右同。)一两手相重按髀,徐徐戾身。一如挽五石弓力左右同。一作拳向前筑(左右同。)一如拓石法(左右同。)一以拳却顿,此是开胸(左右同。)一太坐殿身,偏如排山。一两手抱头,宛转髀上,此是抽胁。

一两手地,缩身曲脊向上三举。一以手反捶背上(左右同。)一大坐曳脚,三用当相手反制向后(左右同。)一两手拒地回顾,此是虎视(左右同。)一立地反拗三举。一两手急相叉,以脚蹋手中(左右同。)一起立以脚前后踏(左右同。)一太坐曳脚,用当相手拘所曳脚,着膝上,以手按之(左右同。)凡一十八势,但老人日别能依此法三遍者,如常补益延年续命,百病皆除,能食,眼明轻健,不复疲。

又云∶每日恒以手双向上招下傍下傍,招前招后,下又反手为之。

又云∶人无问有事无事,恒须日别一度遣人踏背及四肢颈项。若令熟踏,即风气时气不得着人,此大要妙,不可具论之。

《唐临香港脚论》云∶每旦展脚坐。手攀脚七度,令手着指渐至脚心,极踏手用力攀脚,每日如此,香港脚亦不能伤人。

《苏敬香港脚论》云∶夏时腠理开,不宜卧眠。眠觉令人按,勿使邪气,稽留,数劳动开节常令通畅,此并养生之要,提拒风邪之法也。
行止第六

《千金方》云∶凡人有四正∶行正、坐正、立正、言正。饥须止,饱须行。

又云∶凡行立坐,勿背日月。

又云∶寒跏趺坐,暖舒脚眠,峻坐以两足作八字,去冷,治五痔病。

又云∶或行及乘马不用回顾,回顾则神去人。

《养生志》云∶旦起勿交臂膝上坐,凶。
卧起第七

《养生要集》云∶内解曰∶卧当正偃正四肢,自安无侧无伏无劬无倾,常思五脏内外昭明。欲卧,无以人定时加亥,是时天地人万物皆卧为一死与鬼路通,人皆死吾独生矣。欲卧,常以夜半时加子,是时天地人万物皆卧,为一生生气出还,不与人同卧息,常随四时八节。

春夏蚤起,与鸡俱兴;秋冬晏起,必得日光。无逆之,逆之则伤。

《千金方》云∶春欲兴卧早起,夏及秋欲偃息侵夜乃卧早起,冬欲早卧而晏起,皆益人。

虽云早起,莫在鸡鸣前。虽言晏起,莫在日出后。

又云∶人卧,春夏向东,秋冬向西,此为常法。

又云∶暮卧常习闭口,口开即失气,又邪恶从入。

又云∶屈膝侧卧,益人气力,胜正偃卧。

又云∶睡不厌KT,觉不厌舒。(凡人舒睡则有鬼物魇邪得便,故逐觉时乃可舒耳。)又云∶丈丈夫头勿北首卧,卧勿当梁脊下卧讫勿留灯烛,令魂魄及六神不安,多愁怨。

又云∶行作鹅王步,眠作师子眠。(右胁着地屈膝。)又云∶凡眠,先卧心,得卧身。

又云∶人卧一夜,作五覆恒遂更转。

又云∶人卧讫,勿张口,久成消渴,及失血色。

又云∶不得尽眠,令人失气。

又云∶夜卧勿覆其头,得长寿。

又云∶夜卧当耳,勿有孔吹耳聋。

《枕中方》云∶勿以冬甲子夜眠卧。

《千金方》云∶凡人厌,勿燃明唤之定厌死不疑暗唤之吉,亦不得近而急唤。

又云∶人眠,勿以脚悬蹋高处,久成肾水及损房,足冷。

又云∶夏不用屋上露面卧,令面皮肤喜成癣。(一云面风。)又云∶人头边勿安火炉,日别承火气,头重目精赤及鼻干。
言语第八

《养生要集》云∶《中经》曰∶人语笑欲令少,不欲令声高,声高由于论义理,辨是非相嘲调说秽慢,每至此会,当虚心下气与人不竞。若过语过笑,损肺伤肾,精神不定。

《千金方》云∶冬日正可语,不可言。(自言曰言,答人曰语。有人来问,不可答,不可发也。)又云∶冬日触冷行,勿大语言开口。

又云∶语作含钟声。

又云∶行不得语,欲语须作立乃语。(行语令人失气。)又云∶纵读诵言语,常想声在气海中。(脐下也。)又云∶旦起欲得专言善事,不欲先计钱财。

又云∶旦下床勿叱吐呼,勿恶言。

又云∶旦勿嗟叹。

又云∶凡清旦恒言善事,闻恶事即向所来方三唾之,吉。

又云∶日初入后勿言语读诵,必有读诵,宁待平旦。

又云∶寝不得语。(言五脏如钟磬,不悬不可出声。)又云∶夜梦不可说之,旦以水向东方之,咒曰∶恶梦着!草木好梦成宝玉,即无咎。

又云∶梦之善恶,勿说之。

《养生志》云∶旦起勿言奈何,亦勿歌啸,名曰请福吉。

又云∶眠讫,勿大语,损气少气力。

又云∶眠时不得歌咏,歌咏不详事起。

《枕中方》云∶夫学道者,每事欲密,勿泄一言,一言辄减一算,一算三日也。
服用第九

《太素经》云∶歧伯曰∶衣服旦欲适寒温,寒无凄凄,暑无出汁。

《养生要集》云∶青牛道士曰∶春天天气虽阳暖,勿薄衣也。常令身辄辄微汗乃快耳。

《千金方》云∶衣服器械,勿用珠玉金宝增长过失。

又云∶春天不可薄衣,令人得伤寒霍乱,不消食,头痛。

又云∶春冰未泮,衣欲下浓上薄,养阳收阴,继世长生。

又云∶湿衣及汗衣皆不可久着,令人发疮及风瘙。大汗能易衣佳,不易者,急粉身,不尔令人小便不利。

又云∶旦起衣有光者,当户三振之,咒云∶殃去,殃去,吉。

《养生志》云∶旦起着衣,反者更正,着,吉。

又云∶旦起衣带抱人,或结三振云∶殃去,殃去吉。

又云∶高枕远唾损寿。

《本草食禁杂法》云∶勿向北冠带,大凶。
居处第十

《养生要集》云∶河图帝视萌日,违天地者,凶。顺天时者吉。春夏乐山高处,秋冬居卑深藏,吉。利多福老寿。无穷。

《千金方》云∶凡居处不得绮美华丽,令人贪婪(力男反,贪也,)无厌损志,但令雅素净洁,免风雨暑湿为佳。

又云∶凡人居止之室必须周密,勿令有细,致有风气得入,久而不觉,使人中风。

又云∶觉室有风,勿强忍久坐,必须起行避之。

又云∶凡墙北勿安床,勿面向坐久思,不祥,起。

又云∶上床先脱左足。

又云∶凡在家及行猝逢大飘风、豪雨、大雾者,此皆是诸龙鬼神行动经过所致。宜入室闭户烧香静坐,安心以避,待过后乃出,不尔损人,或当时虽未有,若于后不佳。

又云∶家中有经像者,欲行来先拜之,然后拜尊长。

又云∶凡遇(过)神庙,慎勿辄入,入必恭敬,不得举目恣意顾瞻,当如对严君焉,乃享其福耳。

《延寿赤书》云∶南岳夫人云∶卧床务高,高则地气不及,鬼吹不干。鬼气侵人,常依地面向上。(床高三尺六寸,而鬼气不能及也。)
杂禁第十一

《养生要集》云∶《神仙图》云∶禁无施精,命夭;禁无大食,百脉闭;禁无大息精漏泄;禁无久立神绻极;禁无大温,消髓骨;禁无大饮,膀胱急;禁无久卧,精气厌;禁无大寒,伤肌肉;禁无久视,令目;禁无久语,舌枯竭;禁无久坐,令气逆;禁无热食,伤五气;禁无咳唾,失肥汁;禁无恚怒,神不乐;禁无多眠,神放逸;禁无寒食,生病结;禁无出涕,令涩溃;禁无大喜,神越出;禁无远视,劳神气;禁无久听,聪明闭;禁无食生,害肠胃;禁无嗷呼,惊魂魄;禁无远行,劳筋骨;禁无久念,志恍惚;禁无酒醉,伤生气;禁无哭泣,神悲感;禁无五味,伤肠胃;禁无久骑,伤经络。二十八禁,天道之忌,不避此忌,行道无益。

又云∶《中经》云∶射猎鱼捕敷喜而大唤者,绝脏气,或有即恶者,复令当时未觉,一年二年后发病,良医所不治。

《抱朴子》云∶才所不逮,而困思之,伤也。力所不胜,而强举之,伤也。深忧重恚,伤也。悲哀憔悴,伤也。喜乐过差,伤也。吸吸所欲,伤也。戚戚所患,伤也。久谈言笑,伤也。寝息失时,伤也。挽弓引弩,伤也。沉醉呕吐,伤也。饱食即卧,伤也。跳走喘乏,伤也。唤呼哭泣,伤也。阴阳不交,伤也。积阳至尽早已,尽早已非道也。是以养生之方,唾不延远,行不疾步,耳不极听,目不久视,坐不至疲,卧不及疲,先寒而衣,先热而解,不欲极饥而食,食不可过饱,不欲极渴而饮,饮不可过多。凡食过则结积聚,饮过则成痰癖也。不欲甚劳,不欲甚逸,不欲流汗,不欲多唾,不欲奔车走马,不欲极目远望,不欲多啖生冷,不欲饮酒当风,不欲数数沐浴,不欲广志远愿,不欲规造异巧。冬不欲欲极温,夏不欲穷凉,不欲露卧星下,不欲眠中见扇,大寒大热、大风、大雾,皆不欲冒之。

又云∶或云∶敢问欲修长生之道,何所禁忌。抱朴子曰∶禁忌之至急者,不伤不损而已。

按∶《易内戒》及《赤松子经》及《河图纪命符》皆云∶天地有司过之神,随人所犯轻重以夺其算。诸应夺算有数百事,不可具论。若乃憎善好杀,口是心非,背向异辞,及(反)戾真正,虐害其下,欺内其上,叛其所事,受恩不感,弄法受赂,纵曲枉直,废公为私,刑加无辜,破人之家,收人之宝,害人身,取人之位,侵克贤者,诛降戮服,谤讷仙圣,伤残道士,弹射飞鸟,刳胎破卵,春夏獠猎腊,骂詈神灵,教人为恶,蔽人之善,减人自益,危人自安,佻(音条)人之功,坏人佳事,夺人所爱,离人崩肉,辱人求胜,取人长钱,决水放火,以行害人,迫胁弱,以恶易好,强取强求,虏掠致富,不公不平,淫倾邪,凌劣暴寡,拾遗取侈,欺殆诳诈,好说人私。持人长短,招天援地,祝祖求直,假借不还,换贷不偿,求欲无已,憎距忠信,不顺上命,不敬所师,笑人作佳,败人果稼,损人器物,以穷人用,以不清洁饮食他人,轻称小斗狭短度,以伪杂真,采取奸利,诱人取物,越井跨灶,晦歌朔哭。此一句辄是一罪,随事轻重司命夺其算纪,算纪尽则人死,若算纪未尽而不自死,殃及子孙也。

《千金方》云∶养生之道,莫久行、久立、久卧、久坐、久听、久视,莫再食、莫强食、莫强醉、莫举重、莫忧思、莫大怒、莫悲愁、莫大欢、莫跳踉(跃也)、莫多言、莫多笑、莫汲汲于所欲、莫情(愤欤)情怀忿恨,皆损寿命。若能不犯,则长生也。

又云∶一日之忌,夜莫饱食;一月之忌,暮莫使醉;一岁之忌,暮勿远行。终身之忌,莫燃烛行房。

又云∶凡人心有所爱,不用深受;心有所憎,不用深憎;并皆损性伤神,亦不可深,亦不可深毁,常须运心于物,平等如觉,偏颇寻即改正之。

又云∶凡冬月忽有大热之时,夏忽有大凉之时,皆勿爱之。有患天行时气者,皆由犯此。

又云∶冬月天地闭,血气脏,人不可劳作出汗,发泄阳气损人。

又云∶凡忽见龙蛇,勿兴心惊怪之,亦勿注意瞻视;勿见光怪变异事即强抑,勿怪之。

谚云∶视怪不怪,怪自坏也。

又云∶凡见姝妙美女,慎勿熟视而爱之。此当是魑魅之物,令人深爱也。无问空山旷野,稠人广众,皆亦如之。

又云∶旦勿嗔忌(恚),勿对灶骂詈,且勿令发覆面,皆不祥。勿杀龟蛇,勿阴雾远行,勿北向唾魁罡神凶,勿腊日哥舞凶,勿塞故井及水渎,令人聋盲。

《本草食禁杂法》云∶勿杀龟,令人短寿。

《养性志》云∶诸空腹不用见臭尸,尸气入脾,舌上白黄,起口常臭。

又云∶诸欲见死尸臭物,皆须饮酒。酒能避毒瓦斯。

《枕中方》云∶勿与人争曲直,当减人算寿也。

又云∶亥子不可唾,亡精失气,减损年命。

又云∶凡甲寅庚辛日,是尸鬼竟乱、精神躁秽之日也。不得与夫妻同席言语会面,必须清净沐浴,不寝以警备之也。

又云∶三月一日不与妇人同处,大凶又云∶八节日勿杂处。又云∶勿以朔晦日怒。

又云∶勿以正月四日北向杀生。

又云∶四月八日勿杀伐草木。

又云∶勿以五月五日见血。

又云∶勿六月六日起立。

又云∶勿以七月七日念恶事。

又云∶勿八月四日市诸附足之物。

又云∶勿九月起床席。

又云∶勿以十月五日罚贵人。

又云∶勿以十二月,晦日三日内不斋,烧香念道也。

《朱思简食经》云∶刀刃不得向身,大忌,令损人年寿。

《养生志》云∶男夫勿跋井中,古今大忌。又云∶来横口舌。又云∶诸得重鞭杖疮及发背者,产妇皆不用见之。

《延寿赤书》云∶八节日当斋心谟(谨)言,必从善事,慎不可以其日震怒及行威刑,皆天人之大忌。

《养身经》云∶人有一不当、二不可、三愚、四惑、五逆、六不祥、七痴、八狂,不可犯之。

一不当∶吉日与妇同床,一不当。(今按∶《周礼》云∶一月之吉。注曰∶吉,谓朔日也。)二不可∶饱食精思,一不可;上日数下,二不可。(今按∶《尚书》云∶正月上日受终于文袒。孔安国云∶上日朔日也。《正义》云∶上日言一岁日之上也。)三愚∶不早立功,一愚;贪他人功,二愚;受人功,反用作功,三愚。

四惑∶不早学道,一惑;见一道书,不能破坏,二惑;悦人妻而贱己妻,三惑;嗜酒数醉,四惑。

五逆∶小便向西,一逆;向北,二逆;向日,三逆;向月,四逆;大便仰头视天日月星辰,五逆。

六不祥∶夜起裸行无衣,一不祥;旦起恚,二不祥;向灶骂詈,三不祥;举足纳火,四不祥;夫妇昼合,五不祥;盗恚师父,六不祥。

七痴∶斋日食熏,一痴;借物元功,二痴;数贷人功,三痴;吉日迷醉,四痴;与人诤言,以身自诅,五痴;两舌自誉,六痴;欺诈父师,七痴。

八狂∶私传经诫,一狂;得死(罪)怨天,二狂;立功已恨,三狂;吉日不斋,四狂;怨父师,五狂;读经慢法,六狂;同学仲义相奸,七狂;欺诈自称师法,八狂。




卷第二十九
调食第一

《黄帝养身经》云∶食不饥乏、先衣不寒之前,其半日不食者,则肠胃虚,谷气衰;一日不食者,则肠胃虚劳,谷气少;二日不食者,则肠胃虚弱,精气不足;三日不食者,则肠胃虚燥,心悸气紊,耳鸣;四日不食者,则肠胃虚燥,津液竭,六腑枯;五日不食者,则肠胃大虚,三焦燥五脏枯;六日不食者,则肠胃虚变,内外变乱,意魂疾;七日不食者,则肠胃大虚竭,谷神去,眸子定然而命终矣。

陈延之《短剧方》云∶食饮养小至长,甚难逆忤致变甚逆,岂可不慎。

《养生要集》云∶频川陈纪万云∶百病横生,年命横夭,多由饮食。饮食之患,过于声色;声色可绝之俞年,饮食不可废。一日当时可益,亦交为患,亦切美物,非一滋味百品,或气势相伐,触其禁忌成瘀毒,缓者积而成,急者交患暴至,饮酒啖枣,令人昏闷,此其验也。

又云∶已劳勿食,已食勿动,已汗勿饮,已汗勿食,已怒勿食,已食勿怒,已悲勿食,已食勿悲。

又云∶青牛道士言∶食不欲过饱,故道士先饥而食也。饮不欲过多,故道士先渴而饮也。

食已毕,起行数百步中益人多也。暮食毕,步行五里乃卧,便无百病。

又云∶青牛道士云∶食恒将热,宜人易消,胜于习冷也。

又云∶郜仲堪曰∶坚细物多燥涩。若不能不啖,当吐去滓,万不一消生积聚;柔脆物无贞,常啖令人骨髓不充实。

又云∶鱼、肉诸冷之物多损人,断之为善,不能不食,务节之。

又云∶《神仙图》曰∶禁无大食,百脉闭,禁无大饮,膀胱急;禁无热食,伤五气;禁无寒食,生病结;禁无食生,害肠胃;禁无酒醉,伤生气。

孙思邈《千金方》云∶食欲少而数,不欲顿多难消也。常欲令如饱中饥,饥中饥。

又云∶当熟嚼食,使米脂入肠,勿使酒脂入肠。

又云∶人食毕,当行步踌躇,有所修为为快也。

又云∶食毕当行,行毕使人以粉摩肠上数百过,易消,大益人。

又云∶食讫,以手摩面,令津液消调。

又云∶厨膳勿脯肉丰盈,恒令俭约,饮食勿多食肉,生百病。少食肉,多食饭及菹菜,每食不用重肉。

又云∶多食酸,皮槁而毛夭;多食苦,则筋急爪枯;多食甘,则骨痛而发落;多食辛,则肉胝而唇骞;多食咸,则脉凝而变化。此以五味所伤也。(今按∶《太素》杨上善云∶多食咸,则脉凝泣而变色;多食苦,则皮槁而毛夭;多食辛,则筋急之而爪枯;多食酸,则肉KT肥而唇;多食甘,则骨痛而发落。)又云∶食上不得诸,诸而食者,常患胸背疼痛。

又云∶食不得语,每欲食,先须送入肠也。

又云∶食竟仰卧成气痞,作头风。

又云∶凡人常须日在己前食讫,则不须饮酒,终身不干呕。

又云∶日入后不用食,鬼魁游其上。

又云∶夏热,常饮食暖饮;冬,长食细米稠粥。

《抱朴子》云∶五味入口,不欲偏多。故酸多则伤脾,苦多则伤肺,辛多则伤肝,咸多伤心,甘多则伤肾。此五气自然之理也。

又云∶不欲极饥而食,食不可过饱;不欲极渴而饮,饮不可过多。

凡食过则结聚,饮过则成痰也。

《马琬食经》云∶凡食,欲得安神静气,呼吸迟缓,不用吞咽迅速,咀嚼不精,皆成百病。

《延寿赤书》云∶九华安妃曰∶临食勿言配牵。(《曲礼》云∶临食不欢,良有以焉。)又云∶勿露食,来众邪也。(露食谓特造失覆之谓也。)《养生志》云∶食冷勿令齿疼,冷则伤肠,食热灼加唇,热则伤骨。

又云∶食热食,汗出荡风,发头病,发堕落,令人目涩饶睡。

又云∶凡饮食无故变色,不可食,杀人。

又云∶诸食热食讫,枕毛卧,久成头风,令人目涩。

《食经》云∶凡饮食衣服,亦欲适寒温。寒无凄沧,暑无出汗。食饮者,热毋灼之,寒无沧之。

又云∶凡饮食调和,无本气息者,有毒饮食,上有蜂并有仓蝇者有毒。

《膳夫经》云∶凡临食不用大喜大怒,皆变成百病。

《七卷食经》云∶非来哭讫,即勿用食,反成气满病。

《服气导引抄》云∶凡食时恒向本命及王气。

又云∶临食勿道死,事勿露食。

《朱思简食经》云∶经宿羹,不可更温食之,害人。

崔禹锡云∶人汗入食中者,不可食,发恶疮,其女人尤甚。宜早服鸡舌香饮,即瘥。
四时宜食第二

《崔禹锡食经》云∶春七十二日宜食酸咸味,夏七十二日宜食甘苦味,秋七十二日宜食辛咸味,冬七十二日宜食咸酸味。四季十八日宜食辛苦甘味。

上,相生之味,其能生长化成。

《千金方》云∶春七十二日,省酸增甘以养脾气;夏七十二日,省苦增辛以养肺气;秋七十二日,省辛增酸以养肝气;冬七十二日,省咸增苦以养心气。四季十八日,省甘增咸以养肾气。
四时食禁第三

《崔禹锡食经》云∶春七十二日,禁辛味,黍、鸡、桃、葱是也;夏七十二日,禁咸味,大豆、猪、栗、藿是也;秋七十二日,禁酸味,麻子、李、菲是也;冬七十二日,禁苦味,麦、羊、杏、薤是也;四季十八日土王,禁酸咸味,麻、大豆、猪、犬、李、栗、藿是也。

上,食禁可慎。相贼之味,其伤生气,故不成王相也。

《膳夫经》云∶春勿食肝,须增咸苦,禁食脾肺及辛甘。

夏勿食心,须增酸甘,得食肝脾,禁食肾肺及苦辛。

秋勿食肺,须增甘咸,得食脾肾,禁食肝心及苦酸。

冬勿食肾,须增辛酸,禁食心脾及甘苦。

四季勿食脾,须增苦辛,得食心肺,禁肝肾酸咸。

《养生要集》云∶高本王KT升和日夏至迄秋分,节食肥腻饼之属,此物与酒水瓜果相妨,当时不必皆病,入秋变阳消阴息,气至,辄多诸暴猝病疠。由于此涉夏取冷大过,饮食不节故也。而或人以病至之日便谓是受病之始,不知其由来者渐也。

又云∶南阳张衡平子云∶冬至阳气归内,腹中热,物入胃易消化;夏至阴气潜内,腹中冷,物入胃难消化。距四时不欲食迎节之物,所谓不时伤性损年也。
月食禁第四

《本草食禁》云∶正月,一切肉不食者(吉)。

《养生要集》云∶正月勿食鼠残食,立作鼠,发出于头顶。或毒入腹脾,下血不止;或口中生疮,如月蚀,如豆许。

又云∶不食生葱,发宿病。

又云∶二月行久,远行途中勿饮阴地流泉水。复发疟久,喜作噎,损脾,令人咳嗽少气,不能息。

《本草食禁》云∶二月寅日食,不吉。

又云∶二月九日食鱼鳖,伤人寿。

《养生要集》云∶三月勿食陈齑,一夏必遭热病,发恶疮,得黄胆,口中饶唾,齑者,蔓菁菹之属。

《崔禹锡食经》云∶三月芹子不可食,有龙子,食之杀人。

又云∶三月三日食鸟兽及一切果菜五辛,伤人。

《枕中方》云∶三月一日勿食一切肉及五辛。

《养生要集》云∶四月不食大蒜,伤人五内。

又云∶四月八日勿食百草菜肉。

《食经》云∶四月建巳勿食雉肉。

《养生要集》云∶五月勿食不成果及桃李,发痈疖,不尔,夜寒极,作黄胆,下为泄利。

又云∶五月五日,食诸菜至月尽,令冷阳,令人短气。

又云∶五月五日,猪肝不可合食鲤子,鲤子不化成瘕。

又云∶五月五日不可食芥菜及雉肉。

崔禹云∶五月不可食韭,伤人目精。

又云∶五月五日,莫食一切菜,发百病。

《食经》云∶五月五日勿食青黄花菜及韭,皆不利人,成病。

《本草食禁》云∶不食獐鹿及一切肉。

《养生要集》云∶六月勿饮泽中停水,喜食鳖肉,成鳖瘕。

又云∶不得食自落地五果,经宿者蚍蜉、蝼蛄、蜣螂游上,喜为漏。

雀禹∶勿食鹰鹈,伤人精气。

又云∶五六月芹菜不可食,其茎孔中有虫之令人迷闷。

《养生要集》云∶七月勿食生蜜,令人暴夏发霍乱。

又云∶不食生麦,变为蛲虫。

《朱思简食经》云∶七月不得食落地果子及生麦。

《养生要集》云∶八月勿食猪肺及胎胙,至冬定发咳。若饮阴地水,定作疟。

《本草食禁》云∶不食葫,令人喘。

又云∶不食姜,伤神。

《孟诜食经》云∶四月以后及八月以前,鹑肉不可食之。

《千金方》云∶八月勿食雉肉,损人神气。

《养生要集》云∶九月勿食被霜草,向冬发寒热及温病。食欲吐,或心中停水不得消,或为胃反病。

又云∶不食姜,令人魂病。

又云∶勿食猪肉。

《本草食禁》云∶九月不食被霜瓜及一切肉,大吉。

《养生要集》云∶十月勿食被霜生菜,面无光泽。令目涩,发心痛腰疼,或致心疟手足清。

又云∶不食椒,令人气痿。

又云∶禁螺猪肉。

《千金方》云∶十月十一月十二月勿食生韭,令人多涕唾。

《养生要集》云∶十一月勿食经夏臭肉、脯肉。动于肾,喜作水病及头眩,不食螺着用之物。

又云∶十二月不食狗鼠残之物,变成心痫及漏。若小儿食之,咽中生白疮死。

又云∶正月二月水旺,勿食其肝。食肝伤其魂,魂伤狂妄。

四五月火旺,勿食心。伤其神,神伤多悲惧。

七八月金旺,勿食其肺,食肺伤魄,魄伤狂妄。

十月十一月水旺,勿食其肾,食肾伤其志,志伤五脏不安。

三、六、九月、十二月土旺,勿食其脾。食脾伤其意,意伤四肢不遂。
日食禁第五

《养生要集》云∶凡六甲月勿食黑兽。

又云∶壬子日勿食诸五脏。

《本草食禁》云∶甲子日勿食一切兽肉,伤人神。

又云∶月建日勿食雄肉,伤人神。

又云∶子日勿食诸兽肉,吉。

又云∶午日勿食祭肉,吉。

《枕中方》云∶勿以六甲日食鳞甲之物。
夜食禁第六

《养生要集》云∶凡人夜食伤饱,夜饮大醉,夏日醉饱。流汗来冷水洗渍,特扇引风,当风露卧,因醉媾精;或和冰和食,不待消释,以块吞之,是以饮食男女,最为百之本焉。

又云∶夜食恒不饱满,令人无病。此是养性之要术也。

又云∶夜食夜醉,皆生百病,但解此慎之。

又云∶夜食饱讫,不用即脾眠,不转食不消,令人成百病。

《七卷食经》云∶夜食饱满,不媾精,令成百病。

又云∶夜食不用啖生菜,不利人。

夜食啖诸兽脾,令人口臭气。

夜食不用诸兽肉,令人口臭。

夜食不须禽膊,不利人。

夜中勿饮新汲水,被吞龙子生肠胀之病。

夜食不用啖蒜及薰辛菜,辛气归目,不利人。

夜食啖芦茯根,气不散不利人。
饱食禁第七

《养性志》云∶食过饱,伤膀胱,百脉闭不通。

《本草杂禁》云∶饱食,夜失覆为霍乱。

《千金方》云∶养性之道,不欲饮食便卧及终日久坐,皆损寿。

《养生要集》云∶青牛道士云∶饱食不可疾走,使人后日食入口则欲如厕。

又云∶青牛道士云∶饱食而坐,乃不以行步及有所作KT,不但无益而已,乃使人得积聚不消之病及手足痹蹶,面目梨,损贼年寿也。若不得常有所为又不能食毕行者,但可止家中大小KT述如手博戏状,使身中小汗乃傅粉而止,延年之要也。

又云∶饱食即饮水,谷气即散,成癖病腰病。

又云∶饱食即浸水两脚,肾胀成水病。(水病者,四服皆肿,成水胀病也。)又云∶伤饥,猝饱食,久久成心瘕及食癖病。

《七卷食经》云∶饱食构精,伤人肝;面目无泽,成病伤肌。

又云∶饱食即沐发者,作头风病。

《神农食经》云∶饱食讫,多饮水及酒。成痞癖,醉当风。
醉酒禁第八

《养生要集》云∶频川韩元长曰∶酒者,五谷之华,味之至也。故能益人,亦能损人。

节其分剂而饮之,宣和百脉,消邪却冷也。若升量转久,饮之失度,体气使弱,精神侵昏,物之交验,无过于酒也。宜慎无失节度。

又云∶饱食醉酒,酒食未散以仍构精,皆成百病,一日令儿癫瘕病。

又云∶饱食夜醉,皆生百病。但能慎此,养生之妙也。

又云∶酒已醉,勿强饱食之,不幸则发疽。

又云∶大醉不可安卧而止,当令人数摇动反侧之,不尔成病。

又云∶酒醉不可当风,当风使人发喑不能言。(一曰不可向阳。)又云∶饮酒醉,灸头杀人。

又云∶酒醉热未解,勿以冷水洗面,发疮。轻者渣。

又云∶酒醉眠黍穣上,汗出,眉发交落,久还生。(此事难,然不可信。)又云∶大饮酒、饱,不可大呼唤及大怒,奔车走马KT距,使人五脏颠倒,或致断绝杀人。

又云∶夏日饮酒大醉流汗,不得以水洗泼及持扇引风,成病。

又云∶醉不可露卧,使人面发,不幸生癫(癞)。

又云∶凡祭酒,自动自竭,并不可饮伤人。

又云∶祭酒肉,回有气,勿饮之,弃去江河中。

又云∶锡姜多食饮,酒醉杀人。

又云∶蒜与食饱,饮酒醉,不起步。死。

又云∶饮酒醉,合食蒜,令人伤心至死。

又云∶食麻子饮酒,令人胀满,为水病。

又云∶食猪肉饮酒,卧秫稻穣中见星者,使人发黄。

又云∶饮酒不得用合食诸兽肾,令人腰病。

又云∶饮酒不用饮乳汁,令人气结病也。

又云∶饮酒不用食生胡,令人心疾。

又云∶茄芦合,多食饮酒杀人。
饮水宜第九

《养生要集》云∶凡煮水饮之,众病无缘得生也。

《崔禹锡食经》云∶春宜食浆水,夏宜食蜜水,(今安∶《大清经》云∶作蜜浆法,白粳米二斗,净洮汰,五蒸五露竟,以水一石、白蜜五斗,合米煮之。作再沸止。纳瓮器中成,香美如乳汁味。夏月作此饮之佳。)秋宜食茗水。(今按∶《□食经》云∶采叶茗苗叶,蒸,曝干,杂米捣为饮粥食之,神良。)冬宜食白饮,是谓为调水养性矣。
饮水禁第十

《养生要集》云∶酒水浆不见KT者,不可饮,饮之杀人。

又云∶凡井水无何沸,勿饮,杀人。

又云∶井水无故变急者不可饮之,伤人。

又云∶井水阴日涌者,其月勿饮之。令人得温病。

又云∶夜勿饮新汲井水,吞龙子,杀人。

又云∶鸟中出泉流水,不可久居,常饮作瘿。

又云∶山水其强寒,饮之皆令人利、温疟、瘿瘤肿。

又云∶夏月勿饮山中阴下泉水,得病。

又云∶夏月不得饮田中聚水,令人成鳖瘕。

又云∶凡立秋后不得饮水浆,不利人。

又云∶凡夏天用水正可隐映饮食之,令人得冷病。(马氏云∶释之意也。)又云∶凡冰不得打研,着饮食中食之,虽复当快,久皆必成病。

又云∶凡奔行及马走喘,不得饮冷水之日,上气发热气。

又云∶凡饮水勿急咽之,亦成气及水瘕。

又云∶凡取水无故因动者,此水煮吃食者,杀人。

又云∶凡所欲水,在于胸膈中动作水声者,服药吐出之。不吐者亦成水瘕,难瘥。

又云∶凡人睡卧急觉,勿即饮水,更眠,令人作水癖病。

《崔禹锡食经》云∶人常饮河边流泉沙水者,必作瘿瘤,宜以犀角渍于流中,因饮之,辟疟瘤之吒。

又云∶食诸生鱼脍及而勿饮生水,即生白虫。

又云∶食苏,勿饮生水,即生长虫。

又云∶食辛羸,而勿饮水,作蛔虫。

又云∶食鲫脍即勿饮水,生蛔虫。

《本草食禁》云∶若饮热茗后饮水浆,令人心痛,大慎之。

又云∶食讫,饭冷水成肺。

《膳夫经》云∶凡食,不用以茗饮送之。令人气上咳逆。

《千金方》云∶勿饮深阴地冷水,必疟。

《食经》云∶食讫饮冰水,成病。

又云∶食诸饼即饮冷水,令人得气病。

《七卷经》云∶凡远行途中,逢河水,勿先洗面,生鸟。
合食禁第十一

《博物志》云∶杂食者,百疾妖邪之所钟焉。所食愈少,心愈开,年愈益;所食弥多,心愈塞,年愈损焉。

《养生要集》云∶高本王熙升和日食不欲杂,杂则或有犯者,当时或无交患,积久为人作疾。

又云∶饮食冷热,不可合食,伤人气。

又云∶食热腻物,勿饮冷酢浆,喜失声嘶咽。(嘶者,声败也。咽者,气塞咽也。)又云∶食热讫,勿以冷酢浆嗽口,令人口内齿臭。

又云∶食甜粥讫,勿食姜。食少许即猝吐,或为霍乱。(一云勿食盐。)又云∶置饴粥中食之,杀人。(《食经》云∶此说大乖,恐或文误也。)又云∶膳有甘味,三日勿食生菜,令人心痛。(饴糖属也。)又云∶干秫米合猪,肥食,使人终年不化。

又云∶小麦合菰食,复饮酒,令人消渴。

又云∶小麦合菰米(菜)食,腹中生虫。

又云∶小麦不可合菰首,伤人。

又云∶蒜勿合饴饧,食之伤人。

又云∶食乔麦合猪肉,不过三日成热风病。

又云∶生葱合鸡雄食之,使人大窃,终年流血,杀人。

又云∶葱薤不可合食白蜜,伤人五脏。

又云∶葱桂不可合食,伤人。

又云∶食生葱啖蜜,变作腹痢,气壅如死。

又云∶生葱不可合食鲤鱼,成病。

又云∶生葱食不得食枣,病患。

又云∶啖陈薤并食之,杀人。

又云∶葵菜不可合食猪肉,夺人气成病。

又云∶陈薤、新薤并食之,伤人。

又云∶葵不可合食黍,成病。

又云∶五辛不合猪肉、生鱼食之杀人。

又云∶凡辛物不可合食,使人心疼。

又云∶诸刺菜不可合食麋肉及虾,伤人。

又云∶梨苦菜合生薤食,身体肿。

又云∶芹菜合食生猪肝,令人腹中终年雷鸣。

又云∶戎葵合食鸟子,令面失色。

又云∶干姜勿合食兔,发霍乱。

又云∶食甘草勿食无及蓼交,令人废其阳道。

又云∶食蓼啖生鱼,令气壅或令阴核疼至死。

又云∶蓼叶合食生鱼,使人肌中生虫。

又云∶芥菜不可共兔肉食,成恶邪病。

又云∶生菜不可合食蟹足,伤人。

又云∶栗合生鱼食之,令人肠胀。

又云∶李实合雀肉食,令大行漏血。

又云∶乌梅不可合猪膏食之,伤人。

又云∶李实不可合蜜合食,伤五内。

又云∶枣食不得食生葱,痛病患。

又云∶杏子合生猪膏,食之,杀人。

又云∶菰首不可杂白蜜,食之令腹中生虫。

又云∶茇实合白苋食之,腹中生虫。

又云∶虾不可合食麋肉及梅、李、生菜,皆痼人病。

又云∶诸螺蜊与芥合食之,使人心痛,三月一动。

又云∶诸果合诸螺蜊食,令人心痛,三日一发。(一曰合芥。)又云∶诸菜合煮螺蜊蜗,食之皆不利人。

又云∶猪肉合鱼食,不利人。(一曰入腹成噎。)又云∶猪肝脾鲫鱼合食,令人发损消。

又云∶猪肝不可合鲫鱼子卵,食之伤人。

又云∶猪肝合鲤子及芥菜,食之伤人。

又云∶凡猪肝合小豆食之,伤人,心目不明。

又云∶凡食生肉合饮乳汁,腹中生虫。

又云∶生鹿肉合食虾汁,使人心痛。

又云∶麋鹿肉不可杂虾及诸刺生菜食之,腹中生虫,不出三年死。

又云∶鹿肉合食鱼之,杀人。(一名。)又云∶凡铜器盛猪肉汁,经宿津入肉中,仍以羹作食杏仁粥,必杀人。

又云∶白蜜合白黍食之,伤五内,令不流。

又云∶白蜜合食枣,伤人五内。

又云∶白蜜不可合葱韭食之,伤人五脏。

又云∶食蜜并啖生葱,反作腹痢。

又云∶食甜酪勿食大酢,变为血尿。

又云∶乳酪不可合食鱼脍,肠中生虫。

又云∶乳汁不可合饮生肉,生肠中虫。

又云∶乳汁不可合食生鱼,及成瘕。

又云∶乳酪不可杂水浆食之,令人吐下。

又云∶诸鸟肉及卵和合食,伤人。

《神农食经》云∶生鱼合蒜食之,夺人气。

《千金方》云∶白苣不可共酪食,必作。

又云∶竹笋不可共蜜食之,作内痔。

《孟诜食经》云∶竹笋不可共鲫鱼食之;使笋不消成病,不能行步。

又云∶枇杷子不可合食炙肉热面,令人发黄。

又云∶荠不可与面同食之,令人闷。

又云∶鹑肉不可共猪肉食之。

《崔禹食锡食经》云∶食大豆屑后,啖猪肉,损人气。

又云∶胡麻不可合食并蒜,令疾血脉。

又云∶兰鬲草勿合鹿肪食,令人阴痿。

又云∶鹰勿合生海鼠食,令肠中冷,阴不起。

又云∶李实不可合牛苏食之,王鳖子。

又云∶葵不可合蕨菜食,生蛔虫。若觉合食者,取鬼花煮汁,饮一二升即消去。(鬼花者,八月九月梨花耳。采以为非常之备也。)《马琬食经》云∶猪肉合葵菜食之,夺人气。

《食经》云∶鹿并煮,食之杀人。

《朱思简食经》云∶鲫鱼合鹿肉生食之,筋急嗔怒。

《养生要集》云∶凡饮食相和失味者,虽云无损,不如不犯。膳有熊白,不宜以鱼羹送之,失味。

有膳鱼脍,不宜食鸡肉羹送之,失味。

芥子酱合鱼脍食之,失味。

炙肉汁着浆清食之,有臊气,失味。

捣蒜齑不宜着椒食之,苦失味。

青州枣合白蜜食之,失味,戟人咽喉。

酢浆粥和酪食之,失味。

酢枣食饮酒之,失味。

食乳麋以鱼送之,失味。

膳有鱼脍,不宜以兔羹送之,失味。

膳有乳麋,不宜以鱼肉送之,失味。

蒜荠合芥子酱食之,失味。

大豆合小豆食之,失味。

菘子合芜荑食之,失味。

大豆合小麦食之,失味。

小芥合荷食之,失味。

芸苔合大芥食之,失味。

韭薤合食之,失味。

大芥合水苏食之,失味。

蓼合小芥食之,失味。

膳有糯食酢及酢菹食之,失味。
诸果禁第十二

《养生要集》云∶凡诸果非时,未成核,不可食。令人生疮,或发黄胆。

又云∶凡诸果物生两甲,皆有毒,不可食,害人。

又云∶凡枣桃杏李之辈,若有两核者,食之伤人。

又云∶凡诸果停久,食之发病。

又云∶凡果堕地三重,食之杀人。

《食经》云∶空腹勿食生果,喜令人膈上热,为骨蒸,作痈疖。

又云∶诸果和合食,伤人。
诸菜禁第十三

《稽康养生论》云∶熏辛害目。

《养生要集》云∶葱薤牙生不可食,伤人心气。

又云∶苦瓠不以久盛食之;有毒,杀人。

马琬云∶葵赤茎背黄食之,杀人。

《食经》云∶诸菜和合食。
诸兽禁第十四

《食经》云∶凡诸兽,有歧尾奇纹异骨者,不可食,皆成病,杀人。

又云∶兽赤足食之,杀人。

又云∶凡臭兽无创者,勿食,杀人。

又云∶兽自病疮死,食之伤人。

又云∶肉中有腥如朱,不可食之。

又云∶凡避饥空肠,勿食肉,伤人。

又云∶生肉若熟肉有血者,皆杀人。

《膳夫经》云∶凡肉久置,合器中食之,杀人。

又云∶肉脯鱼腊,至夏入秋。不可食,令人得病。

《养生要集》云∶臭(自死)畜口不闭,食之伤人。

又云∶凡臭(自死)兽伏地,食之杀人。

又云∶癸肉自动,不可食之。

又云∶凡禽兽肝脏有光者,不可食,杀人。

又云∶凡脯置于米盆中不可食,杀人。

又云∶脯勿置黍瓮中,食之闭气伤人。

又云∶凡猪羊牛鹿诸肉,皆不可以木、枣木为划炙食之,入肠里生虫,伤人。

又云∶铜器盖热害汁入食中食之,发恶疮肉疽。

又云∶凡生肉五脏等,着草中自摇动及得酢咸不及(反)色。随(堕)地不污,与犬,犬不食者,皆有毒,食之杀人。

又云∶凡腻羹肉汁在釜中掩覆,若经宿,又在盆器中热,盖气不泄者皆杀人。

又云∶脯炙之不动,得水复动,食之杀人。

又云∶凡肉作脯,不肯燥食之,杀人。

又云∶秽饭肉,食之不利人,成病。

又云∶茅屋脯名漏脯,藏脯蜜器中名郁脯,并不可食之。

又云∶凡夫阴积日及连两虫宿生鱼生肉脍等,不食,不利人。

《千金方》云∶勿食一切脑,大不佳。
诸鸟禁第十五

《七卷食经》云∶凡众鸟臭(自死),口不闭、翼不合者,食之杀人。

又云∶众鸟死,足不伸者,食之伤人。

又云∶凡鸟兽燔死,食之杀人。

又云∶凡鸟有毛,不可食。(毛食不泽曰也。)又云∶凡鸟兽身毛羽有成文本者,食之杀。

又云∶飞鸟投人者,不可食。必者口中喜有物,若无,拔毛放之。

又云∶鸟有三足,鸡两足有四距,食杀人。

《膳夫经》云∶鸟死目不可合,食杀人。

又云∶诸卵有文如八字,食杀人。

又云∶凡鸟卵有文食之杀人。
虫鱼禁第十六

《食经》云∶凡鱼不问大小,其身体有赤黑点者,皆不当啖,伤人。

又云∶凡勿食诸生鱼,目赤者生瘕。

又云∶鱼身白首黄,食之伤人。

又云∶凡鱼有角不可食,伤人。

又云∶凡鱼头中鳃者不可食,杀人。

又云∶鱼有目KT,食之伤人。

又云∶鱼二目不同色,食之伤人。

又云∶鱼死二目不合,食之伤人。

又云∶鱼腹下有丹字,食伤人。

又云∶鱼鳞逆生,食之杀人。

又云∶鱼肠无胆,食之杀人。

又云∶鱼腹中有白如膏状者,食之令人发疽。

又云∶凡鱼头有正白色如连珠至脊上者,食之破煞人心。

又云∶鱼子未成者,食伤人。(正月鱼怀子未成粒者是也。)又云∶生鱼肉投地,(鹿)尘芥不着,食之伤人。

又云∶凡鱼肉脍、诸生冷,多食损人。断之为佳。而不能食,务食简少为节食之。若多食不消成瘕。

又云∶食鱼不得并厌骨,食之不利人。(厌骨在鲤后,大如榆荚。)又云∶虾无鬓亦腹下通黑,食之杀人。

又云∶蜚虫赤足者,食之杀人。

又云∶诸飞虫有三足者,食之杀人。

《膳夫经》云∶凡食鱼头,不得并乙骨,食之不利人。(今按∶《礼记》云∶鱼去乙。

郑玄云∶鱼体中害人者也,今东海KT鱼有骨在目旁,状如篆乙,食哽人不可出也。)又云∶鱼腹中正白,连珠在脐上,食之破心杀人。
治饮食过度方第十七

《病源论》云∶夫食,食过饱则脾不能磨消,令气急烦闷,眠卧不安。

《医门方》云∶治贪食多不消,心腹中坚痛方∶盐一升,水三升,煮令盐消。分三服。当吐食出,便瘥。

《经心方》凡所食不消方∶取其余类,烧作末,服方寸匕;便吐出。

《养生要集》云∶凡人饮食过度方∶可生嚼菜菔根,咽之即消。又研汁服之。

《葛氏方》治食过饱,烦闷,但欲卧而腹胀方∶熬面令微香,捣服方寸匕,得大麦、生面益佳。无面者孽可用之。

《新录方》治食伤饱为病,胃胀心满者方∶十沸汤,生水共三升饮之,当吐食出。

又云∶灸胃管七壮。
治饮酒大醉方第十八

《病源论》云∶饮酒过多,酒毒渍于肠胃。流溢经络,便(使)血脉充满,令人烦毒、昏乱、呕吐无度乃至累日不醒,往往有腹背穿穴者,是酒热毒瓦斯所为,故须摇动其身以消散之。

《千金方》云∶饮酒则速吐为佳。

《葛氏方》云∶饮酒大醉,不可卧而上,当令数摇动转侧。

又云∶勿鼓扇当风,席地及水洗、饮水也,又最忌交接。

又云∶张华饮九酎,辄令人摇动取醒;不尔,肠即烂背穿达席。

《养生要集》治大醉烦毒不可堪方∶芜菁菜并小米,以水煮令熟,去滓,冷冻饮料之则解,此方最良。

又方∶以粳米作粥取汁冷冻饮料之,良又云∶赤小豆以水煮取汁一升,冷冻饮料之即解。

又云∶生葛根,捣绞取汁,饮之。

《集验方》治人大醉欲死,恐烂肠胃方∶作温汤着大器中渍之,冷则易。(今按∶《葛氏方》云∶夏月用汤无苦。)《录验方》治饮酒大醉方∶煮菘汁饮之最良,人好轻其近易之。

《耆婆方》治饮酒连日不解,烦毒不可堪方。

取水中生虾蚬,若螺蚌辈,以豉合煮,如常食法,亦饮汁。

又云∶食瓜及大麦餐。

又云∶食粟餐食并粟粥。

《短剧方》云∶饮酒醉吐,牙后诵血射出,不能禁者方。

取小钉,烧令赤,往血孔上一注即断。

《陶景本草注》云∶大醉,煮田中螺食之,又饮汁。

《苏敬本草注》云饮酒连日不解方。

食软熟柿。

《崔禹锡食经》云大醉方∶煮鲶食之,止醉,亦治酒病。

今按∶《食经》云解酒毒物∶龙蹄子(醒酒)寄居(醒酒)蟹(醒酒)田中羸子(醒酒)蛎(主酒热)丹黍(醒酒)胡麻(杀酒)熟柿(解酒热毒)葵菜(主酒热不解)苦菜(醒酒)水芹(杀酒毒)菰根(解酒消食)
治饮酒喉烂方第十九

《葛氏方》治连日饮酒,喉咽烂,舌上生疮方。

捣大麻子一升,末,黄柏二两,蜜丸合之。
治饮酒大渴方第二十

《葛氏方》治饮酒后大渴方。

栝蒌(三两)麦门冬(三两,去心)桑根白皮(三两,切,熬)水六升,煮取三升,分再服,不止,更作之。
治饮酒下利方第二十一

《葛氏方》治酒后下利不止方。

陟KT纸二十枚,水KT之,无者用黄连三两;牡蛎四两,末之;麋脯一斤,无者用鹿,若无者,当归、龙骨各四两。合水一斗五升,一煮取八升,分三四服,不止更作之。又方云∶寸单服龙骨末,亦单可煮饮之。
治饮酒腹满方第二十二

《千金方》云∶饮酒腹满不消方∶煮盐以小竹管灌大孔中。
治酒病方第二十三

《病源论》云∶酒者水谷之精也,其气悍而有大毒,入于胃则胀气逆,满于胸内,焦(蘸)于肝胆,故令肝浮胆横而狂悖变怒,失于常性,故云恶酒也。

《千金方》治恶酒健嗔方∶空井中倒生草,服之勿令知。

又方∶取其秫上尘,和酒饮之。

《苏敬本草注》恶酒病方∶鹰矢白灰,酒服方寸匕,勿使饮人知之。
治饮酒令不醉方第二十四

《千金方》饮酒不醉方。

柏仁、麻子仁各二合,一服乃进酒三倍。

又方∶小豆若花叶,阴干百日,末,服之。

又云∶饮酒令无酒气方。

干芜菁根二七枚,三蒸,末两钱,饮酒后水服之。

《葛氏方》欲饮酒便难醉,难醉则不损人方∶葛花并小豆花,干末为散,服三方寸匕。

又方∶先食盐一合以饮酒,倍能。

又方∶进葛根饮、芹根饮之。

又方∶胡麻能杀酒。

《枕中方》老子曰∶人欲饮酒不醉,大豆三枚,先服之讫,饮酒不醉也。

《灵奇方》止醉方∶七月七日取小豆花,干之百日,末之。欲饮酒,先取门冬十四枚,与小豆花等纳口中,井花水服之,则不醉。
断酒令不饮方第二十五

《千金方》断酒方∶白猪乳汁一升饮之,永不用酒。

又方∶刮马汗和酒与饮,终身不饮。

又方∶自死蛴螬,干,捣末和酒与饮,永代闻酒名呕吐,神验。

又方∶酒渍汗靴沓一宿,旦空腹与即吐,不喜见酒,取佛体上尘入酒服之,永不欲饮酒。(《灵奇秘方》。)
治饮食中毒方第二十六

《病源论》云∶人往往因饮食忽然困闷,少时致甚乃至死者,名为饮食中毒,故(言)人假以毒投食令裹而杀人,但其病颊内或悬膺(壅)内物,(初)如酸枣大,渐渐大是也。

急治则瘥,久不治,毒入腹即死也。

《医门方》云∶凡煮药以解毒者。虽救急不可热饮之;诸毒得热更甚,宜令冷冻饮料之。

《短剧方》治诸食中毒者,唯黄龙汤及屎汁,无不治也。饮马尿汁亦良。(《千金方》同之。)《本草》云∶饮食中毒烦满方∶煮苦参饮之,令吐出。

《葛氏方》云∶诸馔食直氽何容有毒,皆是假以投之耳。既不知何毒,便应作甘草荠汤通治也。汉质帝啖饼死,即其事矣。

《经心方》食毒方∶白盐一升,以水三升煮消,分三服。

《集验方》食诸饼百味毒若急者方∶单饮土浆。

又方∶单服犀角末方寸匕。

《千金方》治饮食中毒方∶苦参三两,酒二升半,煮取一升,顿服,取吐愈。

《养生要集》治食诸饼百物中毒方。

取贝齿一枚,含之须臾,吐所食物,良。

又方∶捣韭汁饮之良。(以上《葛氏方》同之。)
治食噎不下方第二十七

《病源论》云∶食噎,此由脏气冷而不理,津液涩少不能传行,饮食入则噎寒(塞)不通,故谓之食噎。胸内痛,不得喘息,食不下是也。

《葛氏方》治食猝噎方∶以针二(三,或本)七过,刺水中,东向饮其水,良。

又方∶衔鸬喙即下。

又方∶以羚羊角摩噎上。

又方∶生姜五两,橘皮三两,水六升,煮取二升,再服。

《僧深方》治食噎不下方。

旁人缓解衣带,勿合(令)噎者知,即下。

又方∶水一杯,以刀横书,水已复纵尽饮即下。

《救急单验方》∶取鸡尾(若)雉尾,深纳喉中即通。

《枕中方》治人噎欲死方。

使人吹耳中,女则男,男则女,吹便出,良。

《耆婆方》治食噎方∶取盘中酢,三咽,良。

《如意方》治噎术∶舂杵头糠,置手巾角以拭齿,立下。(陶公云∶刮取糠含之。)《千金方》治猝噎方∶取饭留边零饭一粒吞之。

《广利方》理猝食噎不下方∶蜜一匙含细细燕(咽)则下。

《医门方》疗饮食噎不下或呕逆延沫、胸膈不理、脏腑气所致方∶半夏(三两,洗)生姜(五两)橘皮(三两)桂心(二两)水七升,煮取二升半,分三服,气下瘥。
治食诸果中毒方第二十八

《养生要集》云∶凡治一切果物食不消化方。甘草,贝齿,粉,凡三物,分等作末,以水服,良。

又方∶以小儿乳汁二升服之,良。

又方∶含白蜜嚼之,立愈。
治食诸菜中毒方第二十九

《病源论》论∶野菜芹荇之类,多有毒虫、水蛭附之,人误食之便中其毒,亦能闷乱烦躁不安也。

《本草》方∶食诸菜中毒方∶以甘草、贝齿、粉三种末,和水服,小儿尿、乳汁服一升亦佳。

《葛氏方》治食诸菜中毒发狂、烦闷吐下欲死方∶煮豉汁,饮一二升。

又方∶煮葛根饮汁,亦可生嚼咽汁。

又∶治食苦瓠中毒方∶煮黍穣令浓,饮其汁数升。

《养生要集》云∶捣胡麻,以水服(洗,《葛氏方》)二合。
治误食菜中蛭方第三十

《养生要集》治食野菜误食蛭,蛭在胃中及诸脏间食人血,令人消瘦欲死方∶可饮新刺牛血一升许。停一宿,烦,猪膏一升饮之。蛭便从大孔出。已用有验,所刺牛不杀,但取血。

《崔禹锡食经》云∶食菹菜,误吞水蛭方∶服马蓼汁甚效。
治食菌中毒方第三十一

《病源论》云∶菌是郁蒸湿气变化所生,故或有毒者,人食过(遇)此毒多致死甚急速,其不死者由能令烦闷吐利,良久始醒也。

《葛氏方》∶食山中树所生菌,遇毒者则烦乱欲死方∶掘地作坎,以水满中,搅之,服一二升。

又方∶浓煮天(大)豆饮之。

又云∶食枫菌甚笑,又野芋毒并杀人,治之与毒菌同之。

《录验方》云∶服诸吐利丸药除之。
治食诸鱼中毒方第三十二

《病源论》云∶凡食诸鱼,有中毒者,由鱼在水内食毒虫恶草;则有毒,人食之,不能消他(化),即令闷乱不安也。

《短剧方》治食鱼中毒方∶煮橘皮,淳饮之佳。(今按∶《食经》云∶治食脍及生肉太多妨闷者。)又云∶治食鱼脍及生肉经胸膈中不化吐之不出便成症方。

浓朴(二两)大黄(一两)凡二物,以酒二升;煮得一升,尽服之,立消。(《葛氏方》同之。)《千金方》治食脍不消方。

烧鱼灰,水服方寸匕。

又方∶烧鱼鳞灰,水服方寸匕。

《葛氏方》食脍多、过冷不消、不治女成虫方。

捣马鞭草,绞,饮汁一升。亦可服诸吐药以吐之。

又云∶治食鱼中毒面肿烦乱方。

浓煮橘皮,去滓,饮汁。

《集验方》治食鱼中毒方。

煮芦根,取汁饮之。

《本草》云∶食诸鱼中毒方∶煮橘皮及生芦苇根汁、朴硝、大黄汁,烧末鲛鱼皮并佳。

《崔禹锡食经》食鱼中毒方∶犀角二两,细切,以水四升,煮取二升,极冷顿服。

《录验方》食鱼中毒方∶煮甘草二两,饮之,良。
治食鲈肝中毒方第三十三

《病源论》云∶鲈鱼此由肝有毒,人食之中其毒者,即面皮剥落,虽尔不致(至)于死也。

《短剧方》云∶食鱼中毒,面肿烦乱及食鲈鱼肝中毒欲死方∶锉芦根,春取汁(一二升《葛氏方》)。多饮乃良,并治蟹毒。(《千金方》同之。)
治食鲐鱼中毒方第三十四

《病源论》云∶鲐鱼,此肝及腹内子有大毒,不可食,食之往往致死。

《短剧方》云∶中鲐鱼毒方∶烧鱼皮,水服之,无见皮坏,刀浆取之,一名鲛鱼皮。食诸鲍鱼中毒亦用之。(《千金方》同之。)《玉葙方》云∶水中大鱼骨伤人,皆有毒,治之方∶烧獭毛皮骨,以傅矢涂亦佳。
治食鱼中毒方第三十五

《玉葙方》治鱼及水中物所伤方∶嚼粟涂之。

又方∶煮汁洗之。
治食诸肉中毒方第三十六

《病源论》云∶凡可食之肉。无毒(甚)有毒自死者,多因疫气所毙,其肉则有毒,若食此毒肉,便令人困闷,吐利无度。

《葛氏方》治食诸生肉中毒方。

以水五升,煮三升土,五六沸下之。食须(顷)饮上清一升。

《录验方》治食诸生中毒方。

水六升,煮大豆三升,取汁二升服之。

又方∶服土浆一二升。

《千金方》治食生肉中毒方∶掘地深三尺,取下土三升,以五升水煮土五六沸,取上清,饮一升立愈。

又方∶烧猪屎末服方寸匕。

《短剧方》治食六畜肉中毒方。

取其畜干屎末,水服佳。

又云∶若自死六畜毒方。

水服黄柏末方寸匕。

《养生要集》云∶食肥肉、饮水浆咽喉中妨闷以有物状方∶取生姜汁一合,和豉粥食,立愈。

《本草食禁》云∶凡食煮炙肉,大多腹中胀闷者;还取煮肉汁,去脂,热饮一升即消。
治食郁肉漏脯中毒方第三十七

《病源论》云∶生肉就肉,肉器裹密(闭),其气不泄,则为郁肉有毒也。肉脯(若)为(久故茅)草屋雨漏所湿,则有大毒。(食之三日乃成暴。)《本草》云∶食诸肉、马肝、漏脯中毒方∶生韭汁服之,烧末猪骨头垢。烧犬屎,酒服之。豉汁亦佳。

《僧深方》治郁肉漏脯中毒方∶莲根,捣,以水和绞汁服之。

《葛氏方》治食郁肉漏脯中毒方∶煮猪肪一斤,尽(服)之。

又方∶多饮人乳汁。

《集验方》食漏脯毒方∶捣生韭汁服之。多小以意(《葛氏方》一二升),冬月无韭,捣根取汁。(今按∶《葛氏方》云∶用韭亦善。)《千金方》治漏脯毒方∶服大豆汁,良。
治食诸鸟兽肝中毒方第三十八

《病源论》云∶凡禽兽六畜自死者,肝皆不可轻食,往往有毒,伤人。其疫死者弥甚。

被其毒者,多洞利、呕吐而烦闷不安是也。

《葛氏方》食诸六畜鸟兽肝中毒方∶服头垢一钱匕。

又方∶水渍豉,取汁,饮数升。

又云∶禽兽有中毒箭死,其肉毒方∶以蓝汁、大豆汁解之。

《千金方》治百兽肝毒方∶顿服猪脂一斤,沼陈肉毒。
治食蟹中毒方第三十九

《病源论》云∶蟹食水茛,水茛有大毒,故蟹亦有毒者。中其毒则烦乱欲死,若被霜以后遇毒,不能为肉(害)。

《本草》云∶食蟹中毒方。

捣生苏汁,煮干。煮苏汁、冬瓜汁并佳。

《葛氏方》治食蟹及诸膳中毒方∶浓煮香苏,去滓,饮其汁一升。

《僧深方》治食蟹中毒方。

煮芦蓬茸饮汁之。

《千金方》治食蟹中毒方∶冬瓜汁,服二升,亦可食冬瓜。(《葛氏方》捣汁饮一二升。)
治食鱼骨哽方第四十

《葛氏方》治诸鱼骨哽方。

烧鱼骨,服少少。

又方∶以鱼骨摇头即下。

又方∶以大刀环摩喉二七过。

又方∶烧鱼网服之。

又方∶鸬羽烧末,水服半钱匕。(今按∶《集验方》用屎,《如意方》用骨。)《龙门方》治食诸鱼骨哽方∶取纸方寸,书作“甲子”二字,以水服即下,神验。

又方∶取一杯水着前,张口向水即出。

又方∶鱼网覆头,立下。

又方∶取獭骨含之,立出。

《僧深方》治骨哽方∶水一杯,以笔临水上,书作“通达”字,饮之便下,书“羹”亦好。(尽不着水)。

又方∶葵薤羹饮之,即随羹出,有验。

《录验方》治食诸鱼骨哽方∶取饴糖,丸如鸡子黄大,吞之,不去更吞,至数十枚得效。

又方∶取薤白汤,煮半熟小嚼之;以柔绳系中央吞薤白下喉,牵出哽(鲠),即随已出上方。(《短剧》同之。)《集验方》咽哽方∶传呼鸬鸬,即下。

《短剧方》治鲠鱼骨横喉中、六七日不出方∶鲤鱼鳞皮合烧作屑,以水服即出。

《玄子张食经》治鱼骨在腹中痛方。

煮吴茱,服一盏汁。

又方∶在肉中不出方∶捣吴茱萸对上,即烂出。

《孟诜食经》云∶鱼骨哽方∶取去皮,着鼻中,少时瘥。
治食肉骨哽方第四十一

《葛氏方》治食诸肉骨哽方∶白雄鸡左右翮大毛各一枚,烧末,水服一刀圭。

又方∶烧鹰、燕、狸、虎头诸食肉者,服方寸匕。

《僧深方》治食诸肉骨哽方∶烧鹰屎,下筛,服方寸匕。

《新录方》治肉在喉中不下方∶服酱渍一升。

又方∶熬大豆三升,半熟,纳酒二升,煮三四沸,服一升,日二。

又方∶酒服盐灰,方寸匕。
治草芥杂哽方第四十二

《短剧方》治诸鲠方∶猪膏如鸡子大吞之,不瘥复吞,不过再三便去。(今按∶《葛氏方》治草芥诸噎。)又方∶取薤白汤,煮半熟,小嚼之令柔,以系绳系中央。提绳置,即吞薤白下喉,牵出哽,即随出也。

又方∶取虎骨烧作屑,温白饮服方寸匕,良。若无骨,可用虎牙齿亦佳。

《葛氏方》治杂哽方∶作竹篾,刮,令弱滑以绵,缠纳喉中至哽处,之,哽当随出。(今按∶《短剧方》云∶进退牵引。)又方∶刮东壁土,以酒和服。

又方∶蝼蛄,炙燥,末为屑,东流水服之即出。

又云∶治饮食遇草芥诸物哽方∶随哽所近边耳,令人吹。

又方∶好蜜匕抄稍咽之,令下。

又方∶解衣带因,下部即出。

又方∶末瞿麦,服方寸匕。

又方∶以皂荚屑,少少吹纳鼻中,使得嚏哽出,秘方。
治误吞竹木叉遵(导)方第四十三

《葛氏方》治误吞竹木叉遵辈者方∶吞蝼蛄脑即出。

又方∶但数多食白糖,自消去。
治误吞环钗方第四十四

《葛氏方》治误吞钗方∶取韭,曝令萎,煮令熟,勿切食之。入东钗随出。

又方∶生麦菜,若薤蓟缕皆可食。若是竹钗者;但数数多食白糖,自随去。

又方∶以银钗竹替筋物摘吐,因气吸吞不出方。

多食白糖渐渐至十斤,当裹物自出。

《短剧方》治吞银环及钗者方。

取白糖二斤,渐食尽即出。

又方∶取水银一两,分三服,银环便下去。

又方∶以胡粉一两,和水银一两,治调,分再服。水银能消金银。
治误吞金方第四十五

《短剧方》治服金屑,取死未绝者方∶知觉是服金者,可以一两水银泻其口中,摇动令人喉咽里便微接死人,如坐形,令水银下流,金则消成泥,须臾从下部出也。未出之,须死人,亦苏醒矣。可三过,服之便活也。

今按∶《本草》云∶水银杀金银铜铁毒。

《本草》云∶解食金毒方∶服水银数两即出。

又方∶鸭血及鸡子汁。

又方∶水淋鸡屎汁并解。
治误吞针生铁物方第四十六

《葛氏方》误吞钉针箭铁物辈方∶但多食肥羊、肥牛肉,诸肥自裹之出。

《短剧方》治误咽针者方∶取磁石末,温,白饮服方寸匕。

《僧深方》治误吞钉箭针铁物方。(今按∶《本草》云∶铁毒用磁石解。)冶炭末,饮之即与针俱出。
治误吞钩方第四十七

《葛氏方》误吞钩与绳,若犹在手中者莫引之。

但益以珠,若薏苡子辈;就贯着绳稍稍推令至钩处,小小引之则出。(《私迹方》同之。)又方∶但大戾头四(西向顾,或)领,少引之则出。

又方∶取蝼蛄,摘去其身,但吞其头数枚。(今按∶《私迹方》同之。)
治误吞诸珠铜铁方第四十八

《葛氏方》吞珠铜铁方(而咬方或)∶烧弩铜令赤,纳水中,饮其汁,立出。

《千金方》治吞珠铜铁方∶烧鹰毛二七枚,末,服之。家所养鹅鸟羽亦可用。
治误吞钱方第四十九

《葛氏方》治误吞钱方∶捣火炭,服方寸匕即出。

又方∶服蜜二升即出。

《短剧方》治吞钱留咽中者方∶取白灰,捣下筛,温白饮,服方寸匕,即下去。

又方∶艾蒿五两,细锉,水五升,煮取一升,顿服便下。
治食中吞发方第五十

《短剧方》治食中吞发结喉不出方∶取梳头发,烧服一钱匕。(《葛氏方》同之。)
治误吞石方第五十一

《极要方》云∶下石法∶取肥猪脂成煎者一升,细切葱白一大升,和煮于微火上,首葱白色黄,以生布绞去滓,安瓷器中密盖。旦起空腹含咽之。可三合许即止。若一日服末得利。明日更服,取利为度。

又方∶取露蜂房,碎,一大升,以水三大升煮取一大升汁,分温三服,当于小便中下如沙粉。若未尽,明朝更服。下石法有此二方,余皆不逮。

《慧日寺方》云∶凡人服,1小子欲下却者∶以葵子三升,水四升,煮取三升,饮之。

又方∶葵子、硝石(即朴硝也)等分两,末之。以粥清汁和,服方寸匕,日二。十日药下尽乃可食谷也。

又方∶葵子、硝石各一升,水三升,煮取一升,日三进之。
卷第三十
五谷部第一

《太素经》云∶五谷为养,五果为助,五畜为益,五菜为埤注云∶五谷为养生之主也,五果助谷之资,五畜益谷之资;五菜埤谷之资也。五谷、五畜、五果、五菜,用之充饥则谓之食,以其疗病则谓之药,此谷畜果菜等二十物莫乃是咨五行五性之味、脏腑血气之本也。

充虚接气莫大于兹,奉性养生,可斯须离也。

胡麻《本草》云∶味甘平无毒,主伤中虚羸,补五内,益气力,长肌肉,填髓脑,坚筋骨,金创止痛及伤寒温疟,大吐后虚热羸困。久服轻身不老,明目耐饥,延年。以作油,微寒,利大肠,胞衣不落。《陶景注》云∶八谷之中,唯此为良。淳黑者,名巨胜,是为大胜。又茎方名巨胜,茎圆名胡麻,服食家当九蒸九曝。熬捣饵之,断谷长生。《苏敬注》云∶此麻以角作八棱者为巨胜,四棱者名胡麻。都以乌者良、白者劣耳。生嚼涂小儿头疮及浸淫恶疮,大效。《拾遗》云∶油,大寒,主天行热,肠秘内结。热服一合,下利为度。食油损声,令体重。叶沐头长发。《崔禹锡食经》云∶练饵之法,当九蒸九曝,令尽脂润及皮脱。

其不熟者,则令人发KT落。(和名如字。)大豆《本草》云∶生大豆味甘,平,涂痈肿,煮饮汁,杀鬼毒,止痛,逐水胀,除胃中热痹伤中,淋潞,下瘀血,散五脏结积内寒,杀乌头毒。久服令人身重。熬屑味甘,主胃中热,去肿,除痹,消谷,止胀。又云∶扁豆,味甘,微温,主和中下气。孟诜云∶平主霍乱吐逆。《拾遗》云∶大豆炒及投酒中饮,主风痹瘫缓,口噤,产后血气。炒食极热,煮食极冷。又云∶牛食温,马食冷,一体之中,用之不同也。孟诜云∶大豆初服时似身重,一年之后便身轻,益阳事。又煮饮服之,去一切毒瓦斯。又生捣和饮,疗一切毒,服涂之。崔禹云∶大豆少冷无毒,煮饮汁。疗温毒水肿,为验,除五淋,通大便,去结积。蒸煮食胜于米。

久啖浓肠胃,令人身重。大豆为孽取牙生便干者,即熬末食之,芳美味矣,名黄卷,味苦甘温,主湿痹筋膝挽痛。(和名末女。)赤小豆《本草》云∶主下水,排痈肿脓血,味甘酸,平温,无毒,主寒热,热中消渴止泄,利小便,吐猝,下胀满。《拾遗》云∶驴食脚轻,人食体重。《养生要集》云∶味苦温,久食逐津液,令人枯燥。孟诜云∶青小豆,寒,疗热中消渴,下胀满。(今按∶损害物。

和名阿加阿以支。)白角豆崔禹云∶味咸,少冷,无毒,主下气,治关格,蒸煮,食之,止饥,益人。又有一种,状亦相似,而子紫赤色好,止下利,浓肠胃,益气力。(和名志吕佐佐介。)大麦《本草》云∶味咸,温,微寒,无毒。主消渴,除热,益气调中(又云∶令人多热。)为五谷长。《苏敬注》云∶大麦面,平胃止渴,消食,疗胀。《拾遗》云∶作面食之,不动风气,调中止泄,令人肥健。孟诜云∶暴食之,令脚弱。(为腰肾间气故也,)久服即好,甚宜人。崔禹云∶主水胀,勿合白稻米食,令人多热。(和名不止矣支。)麦《本草》云∶味甘,微寒,食之轻身除热。以作孽,温,消食和中。崔禹云∶以作粥食之,益面色。(和名加知加多。)小麦《本草》云∶味甘,微寒,无毒。主除热,止燥渴,利小便,养肝气,止漏血、唾血。以作面,温,消谷止利;以作面,温,消热止烦。《拾遗》云∶此物秋种夏熟,受四时气足,自然兼有寒温,面热麸冷,宜其然也。《千金方》云∶作面,消热止烦,不可多食,长宿癖。《膳夫经》云∶多食壅气。(和名已矣支。)乔麦孟诜云∶寒,难消,动热风,不宜多食。鱼玄子张云∶乔麦虽动诸病,犹压丹石,能练五脏滓,续精神。其叶可煮作菜食,甚利耳目,下气。其茎为灰,洗六畜疮疥及马扫蹄至神,(今按∶损害物。和名曾波牟支。)青梁米《本草》云∶味甘,微寒,无毒,主胃痹,热中渴利,止泻,利小便,益气补中,轻身长年。《陶景注》云∶粱米皆是粟类,唯其牙头色异为分别耳。《汜胜之书》云∶粱是秫粟。苏敬云∶夏月食之,极为凉清。(和名安波万与称)黄粱米《本草》云∶味甘,平,无毒,主益气和中,止泻。《苏敬注》云∶黄粱,穗大毛长,谷米但粗于白粱而收子少,不耐水旱,食之香美,逾于诸梁。

白粱米《本草》云∶味甘,微寒,无毒。主除热益气。《陶景注》云∶夏月作粟餐,亦以除热。孟诜云∶患胃虚并呕吐食水者,用米汁二合、生姜汁一合和服之。鱼玄子张云∶除胸膈中客热,移易五脏气,续筋骨。(和名之吕阿波。)粟米《本草》云∶味咸,微寒,无毒。主养肾气,去胃痹中热,益气。陈者味苦,主胃热,消渴,利小便。《陶景注》云∶其粒细于粱米,阵者谓经三年、五年者,或呼为粢米,以作粉,尤解烦闷。《苏敬注》云∶粟有多种,而并细于诸粱。其米泔汁主霍乱、夹热心烦渴,饮数升立瘥。臭泔止消渴尤良。崔禹云∶常所啖食耳,益肾气。熟舂令白作粉,尤解烦闷。(和名阿波乃宇留之檷。)秫米《本草》云∶味甘微寒,止寒热,利大肠,疗漆疮。陶景注云∶方药不正用,唯嚼以涂疮。《苏敬注》云∶此米功能是犹稻秫也,今大都呼粟糯为秫,稻秫为糯矣。凡黍稷、粟、秫、KT、糯,此三谷之秫也。马琬云∶秫米,温,食之不及黍米,不妊进御也。(今按∶损害物。和名阿波乃毛知。)丹黍米《本草》云∶味苦,微温,无毒。主咳逆,霍乱,止泄,除热,止烦渴。陶景云∶此即赤黍米也。多入神药用。崔禹云∶食益人。又有米,是乌黍耳,供酿酒祭祀用之。

人饮,好疗魂病,长生。(和名阿加支美。)稷米《本草》云∶味甘,无毒。主益气,补不足。陶景注云∶书多云黍稷。《苏敬注》云∶《吕氏春秋》云∶饭之美者,有阳山之也。《传》云∶本草有稷不载;即也今楚人谓之稷,关中谓之糜,冀州谓之KT。《广雅》云∶KT,也。《尔雅》∶粢,稷也。孟诜云∶益气,治诸热,补不足。(和名支美乃毛知。)粳米《本草》云∶味苦,平,无毒,主益气,止烦,止泄。《陶景注》云∶此即今常所食米,但有白赤小小异挨四五种,犹同一类也。《拾遗》云∶凡米,热食则热,冷食则冷,假以火气,体自温平。《七卷食经》云∶味甘,微寒,止寒热;利大肠,疗漆疮。鱼玄子张云∶性寒,拥诸经络气。使人四肢不收,昏昏饶睡,发风动气,不可多食。崔禹云∶又有秕米,是被含稃壳未熟者曰秕,以水炙焦,舂成米者食之,补五脏,驻面色,不老衰也。(今按∶米粉,崔禹云∶性冷。一名烂米。止烦闷,服食及药石人亦将食之。丹经云∶米粉汁,解丹之发热。和名宇留之称。)稻米《本草》云∶味苦,主温中,令人多热,大便坚。《陶景注》云∶稻米、粳米,此两物,今江东无此,皆呼粳米为稻耳。《苏敬注》云∶稻者,谷通名。崔禹云∶稻米粳米同之一名,米又有乌米,江东呼米,性冷,好治血气。又有米,犹乌米耳。谓舂一斛之成八斗之米。(和名以称乃与称。)糯米《养生要集》云∶味甘平,虽食亦不宜久食。《拾遗》云∶性微寒,妊娠杂肉食之,亦不利,久食,令人身软。黍米及糯饲小猫犬,令脚屈不能行,缓人筋故也。(今按∶损害物。和名毛知乃与檷。)孽米《本草》云∶味苦无毒,主寒中下气,除热。《陶景注》云∶此是以米为孽耳,非别米名也。末其米脂,和敷面,亦使皮肤悦泽。《苏敬注》云∶孽者,生不以理之名也,皆当以可生之物为之。陶称以米为孽,其米岂更能生乎。崔禹云∶味少苦冷,无毒,下气,去热,合乳作粥食之,益面色延年。(和名以檷乃毛也之。)饴糖《本草》云∶味甘微温,主补虚乏,止渴去血。《陶景注》云∶今酒用曲,糖用孽犹同。是米麦而为中上之异,糖当以和润为优,酒以熏乱为劣。《七卷食经》云∶置饴麋粥中食之,杀人,未详。(和名阿女。)酒《本草》云∶味苦,大热,有毒,主行药势,杀邪恶气。《陶景注》云∶大寒凝海唯酒不冰,明其热性独冠群物。人饮之使体蔽神昏,是其毒故也。昔三人晨行触雾,一人健,一人病,一人死。健者饮酒。病者食粥,死者空腹,此酒势辟恶胜于食。《拾遗》云∶酒杀百邪,去恶气,通血脉,浓肠胃,润皮肤,散死气。愚人饮之则愚,智人饮之则智,消忧发怒,宣言畅意。《太素经》云∶醪醴者,贤人以适性,不可不饮,饮之令去病,怡神,必此改性以毒也。《礼记》云∶凡酒饮,养阳气也,故有乐。《养生要集》云∶酒者,五谷之华,味之至也。故能益人,亦能损人。节其分剂而饮之,宣和百脉,消邪却冷也。若升量转久,饮之失度,体气使弱,精神侵昏,物之交验,无过于酒也。宜慎,无失节度。崔禹云∶有大毒,行药力,饮之忘忧为基食家所重。(和名佐介。)酢酒《本草》云∶味酸,温,无毒。主消肿,散水气,杀邪毒。《陶景注》云∶酢酒为用,无所不入。《拾遗》云∶酢破血止运,除块坚积。消宿食,杀恶毒,破结气,心中酢水痰饮,多食损筋骨,杀药。孟诜云∶多食损人胃,消诸毒瓦斯,杀邪毒,妇人产后血运含之即愈。(和名须。)酱《本草》云∶味咸酸,冷利,主除热止烦满,杀药及火毒。《陶景注》云∶酱多以豆作,纯麦者少,今此当是豆者,又有肉酱、鱼酱,皆呼为醢不入药用也。(和名比之保。)盐《本草》云∶味咸,温,无毒,主杀鬼蛊邪注毒瓦斯;下部疮,伤寒寒热,吐胸中痰,止心腹猝痛,坚肌骨,多食伤肺喜咳。《陶景注》云∶五味之中,唯此不可缺。然以泪(浸)鱼肉则能经久不败。以沾布帛则易致朽烂,所施之处,各有所宜耳。

《拾遗》云∶五味之中,以盐为主;四海之内,何处无之。崔禹云∶主杀鬼邪毒瓦斯,其为用,无所不入。(和名之保。)
五果部第二

橘《本草》云∶味辛,温,无毒,主胸中痴瘕、热逆气,利水谷,下气,止呕咳,除膀胱留热、停水,五淋,利小便。脾不能消,谷气充胸中,吐逆霍乱。止泄,去寸白,久服去臭,下气,通神,轻身,长年。《陶景注》云∶此是说其皮功耳。其肉,味甘酸,食之令多痰,恐非益人也。崔禹云∶食之利水谷下气。皮味辛苦,蒸可啖之。孟诜云∶皮主胸中瘕气热逆。又云∶下气不如皮也,性虽温,甚能止渴。《吴录地志》曰∶建安郡有橘。冬月树覆之,至明年春夏,色变为青黑,味尤绝美。《上林赋》曰∶庐橘夏熟者,色黑。朱思简曰∶橘皮食杀虫鱼毒,啖脍必须橘皮为齑用。(和名多知波奈。)柑子《七卷食经》云∶味甘酸,其皮小冷,治气胜于橘皮,去积痰。崔禹云∶食之下气,味甘酸,小冷,无毒,主胸热烦满,皮主上气烦满。孟诜云∶性寒堪食之。皮不任药用。

初未霜时亦酸,及得霜后方即甜美,故名之曰甘。和肠胃热毒,下丹石渴。食多令人肺燥冷,中发流癖病也。马琬曰∶食之,胜橘,去积痰,兼即李衡木奴也,兼名菀云,一名金实。(和名加牟之。)柚《本草》云∶味辛,温,无毒,主胸中瘕热逆气,利水谷下气,止呕咳,除膀胱留热、停水、五淋、霍乱,止泻,去寸白,去臭,通神,长年。《苏敬注》云∶柚皮味甘,今俗人谓橙为柚,非《吕氏春秋》曰果之美者,有云梦之柚。崔禹云∶多食之,令人有痰。孟诜云∶味酸,不能食,可以起盘。按《七卷经》云∶味酢,皮乃可食,不入药用。(今按∶损害物。和名由。)干枣《本草》云∶味甘平无毒,主心腹邪气,安中养脾,助十二经脉,平胃气,通九窍,补少气少津,身中不足,大惊,四肢重,和百药,调中益气强力,除烦,心下悬,肠。

久服轻身长年,神仙。又,三载陈核中人,腹痛恶气猝痉,又疗耳聋鼻塞。《七卷经》云∶食之轻身和百药。孟诜云∶养脾气,强志。崔禹云∶食之益气力,去烦。又有猗枣,甚甘美,大如鸡子,能益人面色,出猗氏县,故以名。朱思简曰∶味甘令热虚冷,人食之补益。〔和名深(保)世留奈都女。〕生枣《本草》云∶味辛,令人热,寒热羸瘦者不可食。《陶景注》云∶大枣杀乌头毒。

崔禹云∶食生大枣者,令发人胃中热渴,蒸煮干食之益人。《膳夫经》云∶不可多食。《七卷经》云∶常服枣核中人,百邪不干也。孟诜云∶生枣食之过多,令人腹胀,蒸煮食之,补肠胃肌中,益气。(和名奈未之支奈都女。)李《本草》云∶味苦平,无毒,主除固热,调中。陶景注云∶言京口有麦李,麦秀时熟,小而甜。崔禹云∶小冷,又临水上,食之,为蛟龙被吞之。孟诜云∶李,平,主猝下赤,生李亦去关节间劳热,不可多食之。《七卷经》云∶味酸,熟实可食之。《神农经》云∶微温,无毒,不可多食,令人虚。《要录》云∶李实,临水不可食,杀人。(和名须毛毛。)杏实《本草》云∶味酸,不可多食,伤筋骨。其两人者,杀人。《陶景注》云∶核主咳逆上气,雷鸣,喉痹,下气。崔禹云∶理风噤及言吮,不开者为最佳,味酸大热,有毒,不可多食。生痈疖,伤筋骨,《神农经》云∶有热人不可食,令人身热,伤神寿。《七卷经》云∶杏仁不可多食,令人热利。孟诜云∶杏热,主咳逆,上气,金创惊痫,心下烦热,风头病。《养生要抄》云∶治食杏仁中毒下利,烦苦方,以梅子汁解之。又方,以蓝青汁服之。(今按∶损物。和名加良色色。)桃实《本草》云∶味酸,多食令人有势。其核味苦,甘平,无毒,主瘀血闭瘕邪气,杀小虫,咳逆,消心下坚。《陶景注》云∶仙家方言,服三树桃花尽,则面色如桃花,人亦无试之者。《神农经》曰∶饱食桃入水浴,成淋病。孟诜云∶温,桃能发诸丹石,不可食之,生食尤损人。《七卷经》云∶桃两仁者,有毒,不可食。崔禹云∶食之令下利,益面色,养肝气,今食桃仁忌术,非之,俗中用无害,又陈子皇啖术入霍山,霍山桃多食之,续气驻色,至三百岁还来,面色美泽,气力如壮时。(今按,损害物。和名毛毛。)梅实《本草》云∶味酸,平,无毒,主下气,除热烦满,安心肢体痛,偏枯不仁,死肌,去青黑痣恶疾,止下利好唾口干。《陶景注》云∶是今乌梅也。又,服黄精仁禁梅实。《苏敬注》云∶利筋脉,去痹。崔禹云∶味酸,大温,主安肝心下气。《药性论》云∶黑穴服梅花,黄连登云台。孟诜云∶食之除闷安神。《七卷经》云∶味酸平。诗云∶梅香类也。又可含以香口也。(和名宇米。)栗子《本草》云∶味咸温,无毒,主益气。浓肠胃,补肾气,令人忍饥。《陶景注》云∶有人患脚弱,往栗树下食数升便能起行,此是补肾之义也。然应生啖之。苏敬云∶作粉胜于菱芰,嚼生者涂病,疗筋骨折碎、疼痛肿、瘀血,有效。饵孩儿令齿不生。崔禹云∶食之益气力。《神农经》云∶食疗腰脚烦,炊食之令气拥,患风水之人尤不宜食。孟诜云∶今有所食生栗,可于热灰中煨之。令才汗出即啖之,甚破气,不得使通熟,熟即壅气,兼名菀云。一名撰子,一名掩子。(和名久利。)柿《本草》云∶味甘,无毒,寒,主通鼻耳气,肠不足。《陶注》云∶火熏者性热,断下。日干者性冷,生柿弥冷。《苏敬注》云∶火柿主杀毒、金火疮,生肉止痛,软熟柿解酒热毒,止口干,押胸间热。《拾遗》云∶日干者,温补多食,去面,饮酒食。红柿令心痛,直至死,亦令易醉。《陶景注》云∶解酒毒,误也。崔禹云∶味甘冷,主下痢,理痈肿、口焦舌烂。孟诜云∶柿主通鼻耳气,补虚劳。又干柿,浓肠胃,温中消宿血。《膳夫经》云∶不可多食,令人腹痛下利,兼名菀云,一名锦叶,一名蜜丸,一名朱实。(和名加支)梨子《本草》云∶味苦寒,令人寒中,金创,妇人尤不可食。《陶景注》云∶梨种殊多,并皆冷利。俗人以为快果,不入药用,食之损人。苏敬云∶梨削贴汤火疮不烂,止痛易瘥。又主热嗽止渴。《通玄经》云∶梨虽为五脏之刀斧,足为伤寒之妙药。崔禹云∶食之除伤寒时行为妙药,但不可多食。《神农经》云∶味甘,无毒。不可多食,令人委困。孟诜云∶胸中否塞热结者,可多食生梨便通。又云∶寒除客热,止心烦。又云∶猝喑失音不语者,捣梨汁一合顿服之。又云∶猝咳,冻梨一颗,刺作五十孔,每孔中纳一粒椒,以面裹于热灰,烧令极熟出。停冷食之。又云∶去皮割梨纳于苏中煎,冷食之。朱思简曰∶食发宿病。又凡用梨治咳嗽,皆须持冷,候喘息,寒定食之。今愚夫以椒梨木冲气热食之,反成嗽,不可拔救也。兼名菀云,一名紫实,一名紫KT,一名缥蒂,一名六俗,一名含须。(今按∶损害物。和名奈之。)柰《本草》云∶味苦寒,多食令人胪胀,病患尤甚。崔禹云∶除内热,无毒。孟诜云∶益心气。鱼玄子张云∶补中焦诸不足。《广志》云∶柰有白、青、黄三种也。(今按∶损害物。和名奈以。)石榴《本草》云∶味甘酸,损人,不可多食。根疗蛔虫、寸白,壳疗下痢,止漏精。

崔禹云∶不可多食,损人气,世人云∶石榴花赤赤皈皈可爱,故多植以为延年花也。孟诜云∶温,实主谷利泄精。又云∶损齿令黑。(今按∶损害物。和名佐久吕。)枇杷《本草》云∶叶平,主猝不止,下气。崔禹云∶子食之下气,止哕呕逆,味甘,生啖益人。《七卷经》云∶味酸,食之安五脏。《膳夫经》云∶益人。孟诜云∶温,利五脏。

久食发热黄。(和名比波。)猕猴桃《七卷经》云∶味甘,寒,无毒,食之无损益。《拾遗》云∶味酸,温,无毒,主骨节风。瘫缓不遂,长生变白,肉,野鸡病。一名藤梨,又名羊桃。崔禹云∶食之和中安肝。味甘,冷,主黄胆消渴,状似枣而青黑色;一节署数十茎,茎头生实,食之利人。(和名已久波。)郁子《本草》云∶味酸,平,无毒,主大腹水肿,面目四肢浮肿,利小便水道。《七卷经》云∶食之利水道。崔禹云∶味酸,冷,未熟者有毒,食之发狂。熟者食之益人。(和名宇倍)通草《本草》云∶味甘,平,无毒,主去恶虫,除脾胃寒热,通利九窍血脉关节,令人不忘。脾痹恒欲眠,心烦,哕出音声,疗耳聋,散痈肿,诸结不消,及金疮、恶疮、鼠,堕胎,去三虫。一名丁翁。《拾遗》云∶一名好手,子如算袋。崔禹云∶食之去痰水,止赤白下利,味甘,温。(和名安介比。)山樱桃《七卷经》云∶味甘,平,无毒,食之无损益。或云食补心气,调中,令人好面色。此有二种,一者白樱子,春早所荣,花白味苦,食令人头痛也。一者黑樱子,花红白,味甜美也。伯KT人为良果,皆云山果美者,唯黑樱子。(和名也未毛毛。)木莲子崔禹云∶食之安中,养肝气,味甘,酸,冷,无毒,主火烂疮,烦毒。性滑利,叶似郁,实如子,啖之轻身,去热气为验也。(和名伊芳太比。)榛子《七卷经》云∶味甘,平,食之无损益,多食令人头痛。崔禹云∶食之明目,去三虫,味甘小涩,冷,无毒,久食轻身耐老。树似杏而实如栎子,蒸干啖之,益人气。(今按∶损害物。和名波之波美。)胡桃仁《七卷经》云∶味甘,温,食之去积气。《博物志》云∶张骞使西域,还得胡桃,故名之。崔禹云∶食之下气,味甘,小冷,无毒,主喉痹,杀白虫,令人痰动。孟诜云∶猝不可多食,动痰饮计日月渐服食,通经络,黑人鬓发毛生,能瘥一切痔病。《千金方》云∶不可多食,令人恶心。《拾遗》云∶味甘,平,无毒。食之令人肥健,润肤黑发,去野鸡病。(和名久留美。)椎子《七卷经》云∶味甘,平,食之补益人,耐饥。去甲作屑,蒸食之,断谷胜橡子。

崔禹云∶味甘,小温,无毒,主补五脏,安中,又有枥子相似而大于椎〔(音焦)。和名之比焦。〕橡实《本草》云∶味苦,微温,无毒,主下利,浓肠胃,肥健人。《七卷经》云∶味涩,无毒,非药非谷而最益人,服之者,未能断谷。《养性要集》云∶啖橡为胜,无气而受气,无味而受味,消食而止利,令人强健。(和名以知比,都留波美乃美。)榧实《本草》云∶味甘,主五痔,去三虫、蛊毒、鬼注。《陶景注》云∶食其子乃言疗寸白,不复有余用,不入药方。《七卷经》云∶食之轻身去,腹中虫。马琬曰∶常食之者,三虫不生也。(和名加倍乃美。)复盆子《本草》云∶无毒,主益气,轻身,令发不白。《陶景注》云∶蓬是根名,复盆子是子实名,方家不用,乃昌容所服以易颜色者也。《苏敬注》云∶复盆子、蓬一物异名,本谓实,非根也。崔禹云∶复盆子味酸美香,主益气力,安五脏,是烈真常啖之,遂登仙矣。(和名以知古。)胡颓子马琬云∶味甘,凌冬不雕,食之补益五脏之。《膳夫经》云∶食之益人者也。(和名久美。)甘蔗《本草》云∶味甘,平,无毒,主下气,和中,补脾气,利大肠。崔禹云∶食之下气,小冷。广州大种,经二三年乃生,高硕如竹而过于二三丈。取其汁以为沙糖,甚理风痹,益面色。(和名久美。)蒲陶《本草》云∶味甘,平,无毒,主筋骨湿痹,益气,倍力,强志,令人肥健,忍风寒久食,轻身,不老,延年。《陶景注》云∶魏国使人来,状如五味子而甘美,北国人多肥健耐寒,盖食斯乎。不植淮南亦如橘之变于河北矣。崔禹云∶食之益气力,除风冷,味甘,小冷,益面色。孟诜云∶食之治肠间水,调中。其子不堪多食,令人猝烦闷。《七卷经》云∶味甘,平,可作酒,逐水,利小便。《广志》云∶蒲陶有黄、白、黑三种也。(和名衣美。)桑椹《本草》云∶苏敬曰∶味甘,寒,无毒,单食主消渴。《七卷经》云∶桑椹,《汉武传》曰∶西王母神仙上上药。有扶桑丹所谓椹也。孟诜云∶性微寒,食之补五脏,耳目聪明,利关节,和经脉,通血气,益精神。(和名久波乃美。)薯蓣《本草》云∶味甘,温经,平,无毒,主伤中,补虚羸,除寒热邪气,补中益气力。长肌肉,主头面游风、风头、目眩,下气,止腰痛,充五脏,强阴,久服耳目聪明,轻身,不饥,延年。一名山芋,秦楚名玉延,郑越名土。《陶景注》云∶食之以充粮。《苏敬注》云∶日干,捣筛,为粉,食之大美。崔禹云∶食之长肌肉,强阴气。《七卷经》云∶食之益气力,充五脏。《膳夫经》云∶补中强阴,兼名菀云,一名KT(薯蓣二音)一名延草。《杂要诀》云∶一名王。(和名也未都伊芳毛。)零余子《拾遗》云∶味甘,温,无毒,主补虚,强腰背,不饥。蒸食晒干,功用强于薯蓣,此薯蓣子在叶上生,大者如卵。(和名奴加古。)崔禹云∶食之浓肠胃,益气力,止饥。味苦,小甘,无毒,小温,驻面色,胜于麦豆,烧蒸充粮。(和名止已吕。)芋《本草》云∶味辛,平,有毒,主宽肠胃,充肌肤,滑中。一名云芝。《陶景注》云∶生则有毒,不可食,性滑下石。崔禹云∶味咸,小温,滑中,多食之伤人性命。《神农经》云∶不可多食,动宿冷。孟诜云∶主宽缓肠胃,去死肌,令脂肉悦泽。《七卷经》云∶有毒,能下石。《列仙传》云∶昔酒客为梁蒸,使民益种芋。后三年当大饥。梁民不饥死,兼名菀云。一名长味,一名谈善。《养生要集》云∶芋种三年不收成,野芋食之杀人。又云∶治野芋中毒方∶煮大豆汁冷冻饮料之。又方∶土浆饮之。(和名以倍都以毛。)乌芋《本草》云∶味苦,微寒,无毒。甘,主消渴、痹热、热中,益气。一名藉姑,一名水萍。《陶景注》云∶生水田中,叶有桠状如泽写,不正似芋。苏敬云∶此草一名槎牙,一名茨菰,主百毒,产后血闷,攻心欲死,产难,胞衣不出,捣汁服一升。《拾遗》云∶食之令人肥白。小者极消,吞之开胃及肠。《千金方》云∶下石淋。崔禹云∶食之益气力,主消渴、五淋。煮啖为佳。孟诜云∶主消渴,下石淋。吴人好啖之;发香港脚,瘫痪风,损齿。

紫黑色,令人失颜色。《七卷经》云∶食之止渴,益气。《广雅》云∶藉姑亦曰乌芋也。《养生要集》云∶味苦,微寒,食之除热。所谓凫茈者是也。为粉食之,其色如玉。久食益人,兼名菀云,一名火芋,一名玉银。(和名久和为。)菰根《七卷经》云∶味甘,大寒,除肠胃中痼热,消渴,止小便利。《养生要集》云∶味甘平,除胸中烦,解酒消食。(和名古毛檷。)菰首《七卷经》云∶味甘,冷,被霜之后,食之令人阴不强。又杂白蜜食,令人腹中生虫。〔今按∶损害物,和名古(已)毛都乃。〕芰实《本草》云∶味甘。平,无毒,主安中补脏,不饥,轻身。(一名。)《陶景注》云∶火以为米,充粮断谷长生。崔禹云∶芰实,食之安中补五脏。孟诜云∶食之神仙,此物尤发冷,不能治众病。《七卷经》云∶味甘,平,无毒。食之不饥,被霜后食之,令阴不强,(和名比之。)藕实《本草》云∶味甘,平,寒,无毒,主补中养神,益气力,除百疾,久服轻身耐老,不饥延年。《陶景注》云∶此即今莲子是也。宋帝时大官作羊血KT人削藕皮误落血中,皆散不凝。医仍用藕疗血多效。《苏敬注》云∶主热渴,散血,生肌。久服令人心欢。崔禹云。藕实根味甘,冷,食养心神。根大冷,主烦热,鼻血不止。孟诜云∶莲子,寒,主五脏不足,利益十二经脉,二十五络。马琬云∶食之养神,除百病。根效与实相似也。(和名波知须。)鸡头实《本草》云∶味甘,平,无毒,主疗湿痹,腰脊膝痛,补中益精,强志,耳目聪明,久服轻身,不饥,耐老,神仙。《陶景注》云∶此即今子,子形上花似鸡冠,故名鸡头。《苏敬注》云∶作粉与菱粉相似,益人胜菱芰。崔禹云∶益气力,耳目明了。孟诜云∶作粉食之甚好。此是长生之药,与莲实合饵,令小儿不能长大。故知长服当驻其年耳。生食动小冷气。《七卷食经》云∶食之益精气。(和名美都布布支乃美。)千岁汁《本草》云∶味甘平,无毒,和补定五脏,益气续筋骨,长肌肉。去诸痹,久服轻身不饥耐老,通神明。崔禹云∶食之补五脏,味甘平,小冷,其茎切,绝而受沥汁,状如薄蜜,甘美,以薯蓣为粉,和汁煮作粥食,主哕逆。又合白蜜食之,益人。(和名安未都良。)
五肉部第三

牛乳《本草》云∶微寒,补虚羸,止渴,下气。《陶景注》云∶牛为佳。《拾遗》云∶凡服乳必煮一二沸,停冷啜之,热食则壅,不欲顿服。兼与酸物相反,令人腹中结。崔禹云∶益胃气,令人润泽。《养生要集》云∶腹中有冷患,饮乳汁,令腹痛泄利。《七卷经》云∶不可合生肉,生腹中虫;不可合生鱼食,反成。(和名宇之乃知。)酪《本草》云∶味甘酸,寒,无毒,主热毒,止渴,除胸中虚热,身面上热疮。《养生要集》云∶腹中小有不佳,不当啖酪,令不消。

酥《本草》云∶微寒,补五脏,利大肠,主口疮。《陶景注》云∶乳成酪,酪成酥,酥成醍醐,色黄白。《养生要集》云∶甘道人云∶乳酪酥髓,常食令人有筋力,胆KT,肌体润泽。猝食令人胪胀泄利,渐渐自已。

鹿肉《本草》云∶肉温,补中,强五脏,益气力。《陶景注》云∶野肉之中,唯獐鹿可食,生不腥膻,又非辰属,八卦无主,而兼能温补于人,则生死无忧,故道家许听为脯。

《苏敬注》云∶头主消渴,筋主劳损,骨主虚劳,脂主痈肿死肌,温中,四肢不随。一云∶不可近阴,角主中恶注痛,血主折伤阴,补。又云∶鹿茸味甘,酸温,无毒,主漏下恶血,寒热,益气强志,生齿,疗虚劳羸瘦、四肢酸痛,腰脊痛、泻精尿血,安胎下气。角主恶疮痈肿。髓味甘,温,主大(丈)夫、女子伤中脉绝、筋忿、咳逆,以酒服之。又云∶獐骨主虚损泄精,肉补益五脏,髓益气力、悦泽人面。崔禹云∶味咸,温,无毒,主大风、冷气、口、消渴,心主安中,肝主安肝,肺主安肺,肾主安肾,脾主安脾,膏主四肢不遂。孟诜云∶鹿头主消渴多梦,梦见物;蹄肉主脚膝骨髓中疼痛;生肉主中风口偏不正。《膳夫经》云∶肾弥佳。《千金方》云∶凡饵药之人不可食鹿肉,服药必不得力。所以然者;鹿恒食解毒之草,是故能制散诸药也。《养生要集》云∶鹿有豹文不可食,杀人。又云∶鹿茸、鹿角皆不中嗅,角中有细虫,似白粟,入咽令人虫癞,万术不能治。马琬云∶鹿食之不利人。朱思简云∶合生菜食之,使腹中生疽虫。鹿胆白者不可食之。《食经》云∶鹿雉合煮,食之杀人。《卢宗食经》云∶鹿,五月以后无角者食伤人。(和名加乃志志。)猪肉《本草》云∶味苦,主闭血脉,弱筋骨,虚人肌,不可久食。《陶景注》云∶猪为用最多,唯肉不宜人。人有多食,皆能暴肥,此盖虚肥故也。《千金方》云∶不可久食,令人少精,发宿病。《拾遗》云∶肉寒,主压丹石,解热。人食之,杀药动风。《七卷食经》云∶合五辛食之,伤人肝脾;鲫鱼合食,令人发损消。又不可合鲤鱼子,伤人。朱思简云∶合鱼共食,入腹动风,令生虫,肝合芹菜食之,令人腹中终身雷鸣。《养生要集》云∶猪肝落地,土不着者,食杀人。又云∶猪干脯火烧不动者,食之毕泄利。马琬云∶猪目睫交不可食,伤人。《膳夫经》云∶豕自死,其目青,食之杀人。又云∶豕燔而死,食餐其肝,杀人。

又云∶猪白蹄青爪斑斑不可食。又云∶白猪青蹄食之杀人。(今按∶损害物。和名为乃志志。)雉《本草》云∶肉味酸,微寒,无毒,主补中益气力。止泻利,除蚁。《陶景注》云∶雉虽非辰属而正是离禽,景午日不可食者。《苏敬注》曰∶雉味甘,主诸疮。崔禹云∶主行步汲汲然。益肝气,明目,癣诸浅疮。丙午日食生心瘕损肝气,五鬼起于内,致不祥。

朱思简《食经》云∶凡食雉害(肉),不得食骨,大伤人筋骨。(和名支之。)云雀崔禹云∶味咸,大温,无毒,主补中,阴痿不起,虚劳内损,赤白下利,作食之,强阴气。貌似雀而大。是鸟春夏在阳,秋冬在阴。阳时喜鸣,阴时不鸣,吸阴气而登天,含阳气而下地;翔于云阳而吐气,故以名之。其音密密然,似人大訇。(和名比波利。)鹑孟诜云∶温补五脏,益中续气,实筋骨,耐寒暑,消结气。又云∶不可共猪肉食之,令人多生疮。(今按∶《拾遗》云∶共猪肉食之,令人生小黑子。)又云∶患利人可和生姜煮食之。《七卷经》云∶味辛平,食之令人善忘。崔禹云∶鹌鹑,无毒,主赤白下利,漏下血,暴风湿痹,养肝肺气,利九窍。(和名宇都良。)鸠崔禹云∶味苦咸,平,无毒,主续绝伤,补中,坚筋骨。益气力,好令趋走,妊身妇人尤不可食,其子门肥充,于产难故也。古人云∶是鸟为不噎之鸟,故老人杖头作鸠像,疗噎之吒。(和名波止。)崔禹云∶味甘,温,无毒,主赤白下利,补中,下气,貌似鸽,有白喙,隼眼而翅羽KT斑KT斑可爱。(和名伊芳如留加。)鹎崔禹云∶味酸,冷,无毒,主赤白下利,虚损不足,补中,安魂魄。(和名比衣止利。)鹰《本草》云∶肪味甘,平,无毒,主风击,拘急偏枯,气不通。久服长发鬓眉,益气。崔禹云∶味甘,小冷,主风热、烦心,驻面色,理腰脚痿弱,凡鹰类甚多,大曰鸿,小曰鹰。《七卷经》云∶食无损益。(和名加利。)鸭《本草》云∶肉补虚热,和脏腑,利水道。孟诜云∶寒,补中益气,消食。马琬云∶目精白者,食之杀人。(和名加毛。)鲤鱼《本草》云∶肉味甘,主咳逆上气,黄胆,止渴,生煮主水肿脚满,下气。胆味苦,寒,无毒,主目热、赤痛、清盲,明目。骨主女子带下赤白;齿主石淋。《陶景注》云∶鲤鱼最为鱼之主形,既可爱,又能神变,山上水中有鲤不可食。又鲤不可合小豆藿食之,其子合猪肝食之害人。《苏敬注》云∶骨灰主阴蚀哽不出,血主小儿丹肿,皮主丹隐疹,脑主诸痫,肠主小儿肥疮。《拾遗》云∶肉主安胎,胎动,坏妊身肿。煮食之,破冷气,癖,气块。从脊当中数至尾,无大小皆有三十六鳞。《七卷经》云∶鲤鱼,平,补中。又(胡斗反。野王云是鲤鱼也。)又(下瓦反。《说文》∶鲤也。)又KT(音庆。《广雅》云∶大鲤也。)崔禹云∶鲤、温、无毒、主香港脚忤疾,益气力。孟诜云∶天行病后不可食,再发即死。又砂石中者毒多,在脑髓中,不可食其头。又,每断其脊上两筋及脊内黑血。此是毒故也。朱思简曰∶白头者,不可食交葱桂,食之令人恶病。马琬云∶妊身食之,令子多疮。《养性要录》云∶服天门冬勿食鲤鱼,病不除。(和名已比)鲫鱼《本草》云∶主诸疮,烧,以酱汁和涂之。又主肠痈。一名鲋鱼。作脍,主久赤白利。《拾遗》云∶头主腥嗽,烧为灰服之。肉主虚羸,熟煮食之。脍主赤白利及五痔。《七卷经》云∶味甘温,多食之发热。崔禹云∶味咸,大冷,无毒,主心烦闷,补五脏,安中。

食鲫脍勿饮水,生蛔虫。又勿合猪肉食,成腹中冷。孟诜云∶作脍食之,断暴痢。其子调中益肝气。朱思简云∶合鹿肉生食之筋急。又鲤鱼子、鲫鱼不可同食之。又不可共酪同食。

又沙糖不与鲫鱼同食,成甘虫。又不可共笋食之,使笋不消成食,身不能行步。《养性要集》云∶鲫鱼不可合猪肝食之。(和名布奈。)鱼《本草》云∶味甘,无毒,主百病。《陶景注》云∶今作食之云补。又有鱼相似而大,又有鱼黄而美,并益人。又有人鱼似而有四足,声如小儿,其膏燃之不消耗。

始皇丽山冢中用之,谓之人膏。苏敬云∶鱼,一名鱼,一名鱼。主水浮肿,利小便。

崔禹云∶鱼,温,主风冷冷痹,赤白下利,虚损不足,令人皮肤肥美,貌似鳟而小,色白,皮中有白垢。大者一二尺,小者七八寸,无鳞,春生夏长,秋衰冬死。一名KT。《食经》云∶鱼赤且鬓及无鳃,食杀人。(和名阿由。)鲷崔禹云∶味甘,冷,无毒,主逐水,消水肿,利小便,去痔虫,破积聚,咳逆上气,肠主出败疮中虫,利筋骨。貌似鲫而红鳍坚鳞。(和名多比。)鲈崔禹云∶味咸,大温,无毒,主风痹、瘀、,面,貌似鲤而鳃大;补中安五脏,可为脍。《食经》云∶鲈鱼为羹,食不利人。又云∶鲈肝不可食之,杀人。又云∶治鲈鱼中毒方∶捣绞芦根汁饮之,良。(和名须须支。)鲭崔禹云∶味咸,大温,无毒,主血利,补中,安肾气,貌似鲢,小口尖背苍,可为,食补中,南人多吃鲭益面色,癫疰人,食鲭难瘥。(和名佐波。)崔禹云∶味甘温无毒,主下利明目安心神,貌似而皮中有白垢,尾白刺连逆连逆者也。头中有石,江南人呼曰石首鱼者是也。(和名阿知。)崔禹云∶味咸,大温,无毒,主止下利,益气力,其子似莓赤光。一名年鱼。春生而年中死,故名之。疗风痹为验。(和名佐介。)鳟《七卷经》云∶味酸,热,多食发疮。《字林》云∶赤目鱼也。此鱼似KT而小也。

(今按∶损害物。和名未须。)蠡鱼《本草》云∶味甘,寒,无毒,主湿痹、面目浮肿,下大水五痔,有疮者不可食,令瘢白。一名鱼。《陶景注》云∶今作鳢字,旧言是公蛎蛇所变。崔禹云∶补中明目,食鳢肝而勿饮水,生蛇子故也。(今按∶损害物。和名波牟。)王余鱼《七卷经》云∶食之无损益。郭璞云∶王余比目同,虽有二片,其实一鱼也。

不比行者,名为王余也;比行者,名为比目也。《搜神记》云∶昔越王为脍割鱼而未切,堕半于海中化鱼名王余也。(和名加礼比。)乌贼鱼《本草》云∶味咸,微温,无毒,主疗女子漏下、赤经、白汁,血闭,阴蚀肿痛,寒热瘕,无子,惊气入腹,腹痛环脐,阴中寒肿。《陶景注》云∶鹎鸟所化,今其口脚俱存。《拾遗》云∶昔秦王东游弃算袋于海,化为此鱼。其状似算袋,两带极长,墨犹在腹中也。孟诜云∶食之少,有益髓。《养生要集》云∶味咸,温,食之无损益。崔禹云∶味咸,生大冷,干,小温,无毒,主鬼气入腹,绞痛积聚。南海多垂而浮乌,鸟翔来见之为死即喙,因惊卷捕以杀之,故名曰乌贼。为海神之吏。(和名伊芳加。)海鼠崔禹云∶味咸,大冷,无毒,主补肾气,去百节风,貌似马蛭,而大者长五六尺,小者一二尺,体上小角连数十枚如革囊而KT缩KT臌KT臌者是也。干者,温,主下利,生毛发,黄胆疲瘦。其肠尤疗痔为验。《七卷经》云∶食无损益,有内瘅者,食此生者有利。(和名古。)海月崔禹云∶味辛,大冷,无毒,主利大小肠(腹),除关格、黄胆、消渴,貌似月在海中,故以名之。又有凝月,味咸苦冷,主黄胆消渴,似海月在海中,煮时即凝,故以名之。一名水母。(和名久良介。)海蛸崔禹云∶味咸温,无毒,主虚劳内损诸不足及下利,补中,安五脏。又大者长一二丈,名海肌子;小者长尺余寸,名海蛸子,江东呼曰触外家子。《七卷经》云∶味辛,平,生冷,干温,人有内瘅者,食此生者有利。(和名多古。)蝙KT《七卷经》云∶味甘,微寒,食之无损益。或云补中去烦热,状如大蚯,生海边池泥中,甚似大KT也。湖往后,人视其穴掘取之,以芦刀挫之,去其腹中土沙,以豉盐酱□食美。(和名委。)蛎《本草》云∶牡蛎味咸,平,微寒,无毒,主伤寒、寒热、温疟,除拘缓、鼠,女子下血赤白。心痛气结,止渴,除老血。疗喉痹、咳嗽,久服强骨节延年。《陶景注》云∶是百岁雕所化作。《拾遗》云∶天生万物,皆有牝牡,唯蛎是咸水结成,块然不动,牝牡之事,何从而生?经言牡者应非其雄也。崔禹云∶杀魇魅,治夜不眠;鬼语,错乱,志意不定。

冬时者为优,夏时者为劣,煮蒸食之。孟诜云∶火上令沸,去壳,食甚美。令人细润肌肤,美颜色。《七卷经》云∶有癞疮不可食。(和名加支。)海蛤《本草》云∶味苦咸,平,无毒,主咳逆上气,喘,烦满,胸痛,寒热,主阴痿。

陶景注云∶从鹰矢中得也。《说文》云∶千岁燕化为海蛤、魁蛤。一名伏老伏翼,化为蛤,亦生子滋长。《拾遗》云∶按海蛤是海中烂壳,久在泥沙。风陶沥自然圆净。文蛤是未烂时壳,犹有纹者。崔禹云∶冷,主气劳,补气力。貌小者似臣胜而润泽,然鹅鹰所吞食。大者圆二三寸及五六寸,壳上有文理,而紫斑或彤黄彤黄、或渌斑渌斑,或KT黑KT黑,以纯黑为良。(和名波末久利。)石决明《本草》云∶味咸,平,无毒,主目白翳痛,清盲,久服益精轻身。《陶景注》云∶是鳆鱼甲者。《苏敬注》云∶七孔者良。崔禹云∶温,主腰脚诸病,补五脏,安中,益精气,貌细,孔离离,或九或七;以鳆为真,或作鲍字,亦为误。食之利九窍,心目聪了,故有决明之名。亦附石生,故呼曰石决明耳。秦皇之世,不死之药觅东海者,岂谓于斯欤。

(和名阿波比。)灵羸子崔禹云∶味咸,甘,小冷,无毒。主下气补肝胆气,明目。东海多。貌似橘而圆。其甲紫色,生芒角,以角为脚,口似人脐,脐中有物如马齿而坚白,肠如蛭色赤黑,殊,疗喉痹,利丈夫。(和名宇仁。)辛羸子崔禹云∶味辛KT,大热,无毒,貌似甲螺而口有盖,盖似甲,香色如虎魄,薄光薄光是也。啖之为快味,师门得此而将食之,间夜中耳闻数十尼呗声,及觉而不闻,门即放生,不啖吃矣。(和名于保安支。)甲羸子崔禹云∶味涩咸,小冷,无毒,主蛄毒,补中。貌似辛螺而口有角盖,盖上甲错,似鲛鱼皮,而KT臌KT臌者是也。昔烈真到于东海之碣陂而获巨螺,其大如十升器,将食之间,夜中化成女人,语云为夫妇,十日共俱游之,忽然不见,受真视羸中有光物,即败见有大珠作,未食之,登仙。(和名都比。)小羸子崔禹云∶味涩咸,少冷,无毒,主赤白下利,补中,貌似甲羸而细小。口有白玉之盖,煮啖之。(和名之多多美。)口广大辛螺《七卷经》云∶肉味甘冷。其胆味辛,形似大辛螺而稍小,其甲少薄,色小青黑。(和名于保尔之。)石阴子崔禹云∶味酸,小冷,无毒,主消渴,渴利,黄胆,痈疮,明目补中,貌似人足而表黯黑生毛,是物生海中,有阴精,故名曰石阴子。(和名加世又伊芳加比。)龙蹄子崔禹云∶味咸辛,冷,无毒,主黄胆,消渴,渴利,醒酒,貌似大蹄而附石生肉,头生黑发白卷曲者是也。(和名世。)寄居崔禹云∶味咸,冷,无毒,主渴,醒酒,去烦热,貌似蜘蛸。是物好容他壳中居负壳行,人犯惊即缩足转坠似死乃过,人物行,行掇取啖之,以壳炙火即走出,亦拾掇食之。

《拾遗》云∶食之益颜色。(和名加牟奈。)拥剑《膳夫经》云∶不入药用。《七卷经》云∶《广志》云∶以蟹色黄方二寸,其一KT偏长三寸余,特有光,其短食物着口,一云其大KT和利如剑,其爱如实也。(和名加佐女。)虾《七卷经》云∶味甘,平,食之无损益,不可合梅李生菜,皆令人病。《养生要集》云∶虾无鬓。又亦腹下通黑,食之杀人。又云∶虾煮当赤而反白者,勿食之,腹中生虫。(和名衣比。)蟹《本草》云∶味咸,寒,有毒,主疗胸中邪热气结痛,僻面肿,散血气,愈漆疮。

崔禹云∶主渣鼻恶血,明目醒酒,蟹类亦多。蔡谟初渡江不识而吃蟹几死,乃叹云∶读《尔雅》不熟,为劝学所误耳。孟诜云∶蟹脚中髓汲及脑能续断筋骨。人取蟹脑髓微熬之,令纳疮中,筋即连续。《七卷经》云∶蟹目在下者,食伤人。马琬云∶蟹有六足,腹下无毛,并杀人。《养生要抄》云∶蟹目相向及目赤足斑,不可食。杀人。《食经》云∶孪皆冷利动嗽不可多食。(和名加尔。)河贝子崔禹云∶味咸,冷,无毒,主黄胆,消渴。(和名美奈。)田中螺汁《本草》云∶大寒,主目热赤痛,止渴。《陶景注》云∶生田水中及湖渎岸侧,形圆大如梨柿者人亦煮食之;疗热,醒酒,止渴。患眼痛,取真朱并黄连纳压裹,久汗出取以注目中,多瘥。《苏敬注》云∶壳疗尸注,心腹痛。又主失精。《拾遗》云∶煮食利之大小便,去腹中结热,目黄,香港脚冲上,少腹急硬,小便赤涩,手脚浮肿。生水浸取汁饮之,止消渴。此物至难死,有误泥于壁中二十岁犹活。崔禹云∶田中羸子,味咸,小冷,无毒,主醒酒。冷补之。(和名多都比。)
五菜部第四

竹笋《本草》云∶味甘,无毒,主消渴,利水道,益气,可久食。崔禹云∶味甘,少冷,主利水道,止消渴、五痔。孟诜云∶笋动气,能发冷,不可多食。(和名多加半奈。)白瓜子《本草》云∶味甘,平,寒,无毒。主令人悦泽,好颜色,益气不饥。久服轻身耐老。《陶景注》云∶熟瓜有数种,除瓤食之,不害人。若觉食多。入水自渍即便消。又云∶《博物志》云∶水浸至项,食瓜无数。崔禹云∶味甘冷无毒,食之利水道,去痰水。未熟者,冷,黄熟者平。其瓤甘,补中,除肠胃中风,杀三虫,止眩冒。《养生要集》云∶瓜二蒂及二茎食之杀人。马琬云∶有两鼻食之杀人。孟诜云∶寒多,食发瘅黄,动宿冷病。又瘕癖人不可多食之。(和名宇利。)冬瓜《本草》云∶白冬瓜,微寒,主除少腹水胀,利小便,止渴。《陶注》云∶冬瓜性冷利,解毒消渴。《神农经》云∶冬瓜味甘。无毒,止渴,除热。崔禹云∶冬瓜除水胀风冷人勿食,益病。又作胃反病。鱼玄子张云∶冬瓜食之压丹石,去头面热。(和名加毛宇利。)越瓜孟诜云∶寒,利阳。益肠胃,止渴,不可久食。动气,虽止渴仍发诸疮,令虚,脚不能行立。《本草陶注》云∶越瓜,人以作菹者,食之亦冷。《拾遗》云∶食之利小使,去热,解酒毒。(今按∶损害物。和名都乃宇利。)胡瓜孟诜云∶寒,不可多食,动寒热,发疟病。鱼玄子张云∶发气,生百病,消人阴,发诸疮疥,发香港脚。天行后猝不可食之,必再发。(今按∶损害物。和名加良宇利。)茄子崔禹云∶味甘咸温,有小毒,主充皮肤,益气力,香港脚人以苗叶煮涛脚;皆除毒气,尤为良验也。《七卷经》云∶温,平,食之多动气损阳。(和名奈须比。)龙葵《本草》云∶味苦,寒,无毒,食之解劳少睡,去虚热肿。其子疗疔疮。崔禹云∶食之益气力。孟诜云∶其子疗甚妙,其赤其赤,珠者名龙珠,久服变发长黑,令人不老。《养生要集》云∶补五脏,轻身明目。(和名己奈须比。)若瓠《本草》云∶味苦,寒,有毒,主大水,面目四肢浮肿,下水,令人吐。《苏敬注》云∶瓠与冬瓜瓠全非别类,味甘,冷,通利水道,止消渴。《陶景注》云∶又有瓠亦是瓠类。小者名瓢,食之乃胜瓠。《拾遗》云∶煎汁滴鼻中,出黄水,去伤寒,鼻塞,黄疸。又云∶食苦瓠中毒者,煮黍穣汁饮之,《埤苍》云∶瓠者王瓜也。瓠瓢酌酒琴书自娱也。

(和名尔加比佐古。)葵菜《本草》云∶味甘,寒,无毒。主恶疮,疗淋,利小便,解蜀椒毒,叶为百菜主。

《陶景注》云∶以秋种,经冬至春作子,谓之冬葵,至滑利,能下石淋。《苏敬注》云∶北人谓之兰香。常食中用之,云去臭(鼻)气。《神农经》云∶味甘,寒,久食利骨气。崔禹云∶食之补肝胆气,明目;主治内热消渴,酒客热不解。孟诜云∶若热者食之,亦令热闷。

《膳夫经》云;葵叶尤冷利。《千金方》云∶十日一食葵,葵滑,所以通五脏,拥气。马琬云∶葵赤茎背黄,食之杀人。(和名安不比。)山葵崔禹云∶味辛KT作菹食益人。作齑为快味。(和名和佐比。)兔葵《本草》云∶味甘,寒,无毒。主下诸石淋,止武蛇毒。崔禹云∶味甘,大冷,食之下诸石及蛇毒。(和名以倍尔礼。)苋菜《本草》云∶味甘,寒,无毒,主清盲白翳。明目,除邪,利大小便,去寒热,杀蛔虫,益气力。《苏敬注》云∶主诸肿、、疣目。《拾遗》云∶食鳖所忌,今以鳖细锉和苋于水处置之,则变为生鳖。《七卷经》云∶味甘,益气力,不饥。崔禹云∶食之益气力。

信陵之女,时年十八未嫁而妊胎。父陵自迫问何有妊哉。因垂杀之。女答云∶仆都无所为,但好啖此菜耳,不知所以然云云。父心含怪而取少年婢令食此苋菜,未出数十月而妊胎,遂KT净全之产。(和名比由。)羊蹄《本草》云∶味苦,寒,无毒,主头秃疥瘙,除女子阴蚀、浸淫、疸、痔,杀三虫。《万毕方》云∶疗蛊。崔禹云∶补五脏,益气力。(和名志。)荠《本草》云∶味甘,温,无毒,主利肝气,和中。孟诜云∶补五脏不足。叶动气。

《陶景注》云∶《诗》云∶谁谓荼苦,其甘如荠。崔禹云∶食之甘香,补心脾。(和名奈都奈。)生姜《本草》云∶味辛,微温,主伤寒头痛、鼻塞、咳逆上气,止呕吐。久服去臭气,通神明。《神农经》云∶令少志少智,伤心性,不可过多耳。(今按∶《拾遗》云∶今食姜处亦未闻人愚,无姜处未闻人智,为浪说。)《膳夫经》云∶食甜粥讫,勿食姜。即交吐成霍乱。空腹勿食生姜,喜令渴。崔禹云∶食之去痰下气,除风邪,味辛,KT是物为调食之主。《食(KT)》云∶男子多食者,令人尻肛缓大,女人者令其阴器缓大。孟诜云∶食之除鼻塞,去胸中臭气。《养生要集》云∶微温,食之尤良。然不可过多耳。伤心气。又云∶空腹食,喜令扬上,善为骨蒸及作痈疖。(和名都知波之加美。)芜菁《本草》云∶味苦,温,无毒。主利五脏,轻身益气,可长食。《苏敬注》云∶芜菁,北人名蔓菁。《拾遗》云∶子主急黄、黄胆,肠结不通。又云∶蔓菁园中无蜘蛛,是其相畏也。崔禹云∶食之利五脏,其根蒸敷,脚肿即消。又,取得一斗捣研,以水三斗,煮取一斗汁,浓服之,除瘕积聚及霍乱心腹胀满,为妙药。《神农经》云∶根不可多食,令人气胀。《苏敬香港脚论》云∶患香港脚人不宜食蔓菁。《七卷经》云∶陈楚谓之KT,鲁齐谓之荛,关之东西谓之芜菁,赵魏谓之大芥。(和名阿宇奈。)菘菜《本草》云∶味甘,温,无毒,主通利肠胃,除胸中烦,解消渴。《拾遗》云∶去鱼腥,动病。又,南土无姜,尽为此物所用。崔禹云∶味甘,少冷,无毒,菜中菘尤血常食。和中,无余逆忤,令多食。孟诜云∶腹中冷病者不服,有热者服之亦不发病。其菜性冷。

(和名大加奈。)芦茯《本草》云∶味辛甘,温,无毒,主大下气,消谷去痰,服健人。生捣汁服,主消渴,诫大有验。崔禹云∶味辛薰,温,消五谷反鱼肉毒。又云∶其叶嫩美,亦为生菜之主,啖之消食和中,利九窍,益人。《七卷经》云∶久在土中,食之不利人。马琬云∶夜食不用啖芦茯根,气不散不利人。孟诜云∶萝菔,冷,利五脏关节,除五脏中风,轻身,益气,根消食下气。又云∶甚利关节,除五脏中风,练五脏中恶气,令人白净。(和名于保檷。)芥《本草》云∶味辛,温,无毒,归鼻。主除肾邪气,利九窍,明耳目,安中,久食温中。《陶景注》云∶似菘而有毛,味KT。崔禹云∶食之安中。又,芥类多,有鼠芥鼠食其花而皮毛皆KT落,故以名之。又有雀芥,雀食,其子而KT能飞翔,故以名之。《七卷经》云∶芥有两种,大芥、小芥,是治无异之。孟诜云∶生食发丹石,不可多食。(和名加良之。)白苣崔禹云∶味苦,冷,无毒,主明目,进食者为要。孟诜云∶寒,主补筋力。鱼玄子张云∶利五脏,开胸膈,拥寒气,通经脉,养筋骨,令人齿白净,聪明少睡,可常食之。

有小冷气,人食之虽亦觉腹冷,终不损人。又,产后不可食之,令人寒中,少腹痛。(和名知佐。)蓟菜《本草》云∶味甘,温,主养精保血。陶景注云∶大蓟是虎蓟,小蓟是猫蓟。《苏敬注》云∶大小蓟欲相似,功力有殊。《拾遗》云∶破宿血,止新血,暴下血、血利、惊疮、出血、呕血等,取汁温服。又金疮,又蜘蛛、蛇、蝎咬毒,服之佳。崔禹云∶食之养精神,令人肥健,主女子赤白沃,安胎,止吐血。孟诜云∶叶亦堪煮羹,食甚除热风气。又,金创血不止,义叶封之即止。(和名安佐美。)茎菜崔禹云∶食之止利,味甘苦,少冷,有小毒,主心热烦呕。一名。又取根捣敷钉肿疮疮根即拔之。(和名不不支。)薰蕖崔禹云∶味辛,温,无毒,食之止咳嗽、冷利,止哕。(和名曾良之。)蘩蒌《本草》云∶味酸,平,无毒。主积年恶疮不愈。苏敬云∶即是鸡肠。《七卷经》云∶食之主消渴,杂疮。鱼玄子张云∶煮作羹食之,益甚人。(和名波久倍良。)兰鬲草崔禹云∶食之辛香,冷,平,无毒。主利水道,辟不祥不老,通神明。(和名阿良良支。)胡KT崔禹云∶味辛臭,食之调食下气。凡河海之鸟鱼脍者尤是为要也。孟诜云∶食之消谷,久食之多忘。鱼玄子张云∶利五脏不足,不可多食。损神。(今按∶损害物。和名已志。)蓼《本草》云∶味辛温无毒,主明目温中。能风寒,下水气,面目浮肿痈疡。叶归舌,除大小肠邪气,利中益志。《拾遗》云∶蓼主癖,一名女憎,是其弱阳事也,不可近阴。

又蓼蕺俱弱阳。《七卷经》云∶多食吐水。又多损阳事。《千金方》云∶黄帝曰∶蓼食过多有毒,发心痛。(和名多天。)荷《本草》云∶微温,主蛊及疟。《陶景注》云∶今人赤者为荷,白者为覆菹叶。

同一种耳。于食用,赤者为胜;药用,白者中蛊。服其汁,卧其叶,即呼蛊主姓名。多食损药势。又不利脚人家种。白荷亦云避蛇。苏敬云∶主诸恶疮,杀螫蛊毒根,主稻麦芒入目者,以汁注中即出。崔禹云∶今常食之,有益无损。(和名米加。)芹《本草》云∶味甘,平,主疗女子赤沃,止血,养精,保血脉。益气,令人肥健嗜食。一名冰英。《拾遗》云∶茎叶汁小儿暴热,大人酒热毒鼻塞身热,利大小肠。崔禹云∶味甘,少冷,无毒,利小便,除水胀。孟诜云∶食之养神益力,杀石药毒。鱼玄子张云∶中醋中食之,损人,齿黑色。若食之时不如高由者,宜人。其水者,有虫生子食之,与人患。

《养生要集》云∶芹菜(叶)细叶有毛,食之杀人。(和名世利。)菜《本草》云∶味甘,寒,无毒,主暴热,喘,小儿丹肿。《七卷经》云∶广陵人呼为接,一名KT菜,一名水葱。(和名奈支。)蕨菜崔禹云∶味咸,苦,小冷,无毒,食之补中,益气力。或云多食之睡,令人身重。

是物不宜阳人,即宜阴吒人,食一两斤蕨,终身不病,作脯食之。又煮蒸于腊食之。孟诜云∶令人脚弱不能行,消阳事,缩玉茎,多食令人发落、鼻塞、目暗小儿不可食之立行不得也。《拾遗》云∶小儿食,脚弱不行,四皓食芝而寿,夷齐食蕨而夭,固非良物。《搜神记》曰∶HT鉴镇丹徒二月出猎,有田土折一茎蕨食之,觉心中淡淡成病,后吐一小蛇,悬屋前,渐干成蕨,视即遂瘥。明此物不可生食之。(和名和良比。)荠蒿菜《七卷经》云∶冷,食之无损益。崔禹云∶食之明目。味咸,温,无毒,主开胸府,状似艾草而香,作羹食之,益人。(和名于波支。)《本草》云∶味甘,寒,主消渴,热痹。陶景注云∶下气。《苏敬注》云∶久食大宜人。孟诜云∶多食动痔。《拾遗》云∶案物此虽水草,性热,拥气,温病起食者多死,为体滑,脾不能磨。常食壅气,令关节气急,嗜唾。苏敬云∶上品,主香港脚。《香港脚论》中令人食此之误极深也。(和名奴奈波。)骨蓬崔禹云∶味咸,大冷,无毒,主黄胆消渴。(和名加波保檷。)头《拾遗》云∶味辛,寒,有毒,主痈肿风毒。磨敷肿上捣碎,以灰汁煮成饼,五味调为如食之,主消渴,生即戟喉出血。生吴蜀。叶如半夏,根如KT,好生阴地,雨滴叶生子,一名,又有斑枝,根苗相似,至秋有花直出赤子,其根敷痈毒,于不食。(和名古尔也久。)牛蒡《本草》云∶恶实,一名牛蒡,一名鼠粘草,味辛,平,主明目,补中,中风,面肿,消渴。《苏敬注》云∶根主牙齿痛,脚缓弱;痈疽,咳嗽,疝瘕。积血。(和名支多支须。)骨蓬崔禹云∶味咸,大冷,无毒,主黄胆,消渴。(和名加波保檷。)木菌《七卷经》云∶味甘,温,平,食之轻身,利九窍。凡诸有毒木所生,人不识,煮食,无不死之。宜不可轻啖之。又云∶石耳性冷,生于石上,食之为益。又云∶地菌,温,平,食之补五脏,益气。崔禹云∶菌茸,食之去热气,生冷干温。《拾遗》云∶采归,色变者有毒,夜中有光者有毒,煮不熟者有毒。盖仰者有毒。又冬春无毒,秋夏有毒,为蛇过也。冬生白软者无毒,久食利肠胃。《养生要集》云∶木菌味甘,温平,食之轻身,利九窍。

又云∶菌赤色,不可食,害人。又云∶菌生卷者,食之伤人。青色者亦不可食。木耳色青及仰生者不可食,伤人。又云∶枫树所生菌,食之令人笑不止。又云∶治食菌中毒,烦乱欲死方∶煮大豆汁,饮之良。又土浆,饮之良。(和名支乃多介。)榆皮《本草》云∶味甘,平,无毒,主大小便不通水道,除邪气、肠胃中热气,消肿。

性滑利,疗小儿头疮,久服轻身不饥。其实尤良。花主小儿痫、小便不利。《陶景注》云∶令人睡眠。嵇公所谓∶榆令人眠。《礼记》云∶粉榆以滑之。(榆白曰粉)《养生要集》云∶多睡,发痰。(和名尔礼。)辛夷《本草》云∶味辛,无毒,主五脏身体寒风,风头脑痛,利九窍,生发鬓,去白虫,增年。崔禹云∶食之利九窍,味辛香,温,无毒,其子可啖之。(和名也末安良良支。)昆布《本草》云∶味咸,寒,无毒,主十二种水肿,瘿瘤,气。陶景注云∶干性热,柔甚冷。《拾遗》云∶生颓卵肿,含汁咽之。崔禹云∶治九风热热瘅,手脚疼痹,以生啖之,益人。(和名比吕米。)海藻《本草》云∶味苦咸,寒,无毒,主瘿瘤气,颈下核,破散结气,痈肿,瘕,坚气,腹中上下鸣,下十二水,水肿,皮间积聚。暴颓留气,热结,利小便。崔禹云∶味咸,小冷,一名海发,其状如乱发。孟诜曰∶食之起男子阴,恒食消男子。鱼玄子张云∶瘦人不可食之。(和名尔支女。)鹿角菜《养生要集》云∶味咸,冷利,食之动嗽。(今按∶损害物。和名都乃未多。)石崔禹云∶味咸至滑,滑然大冷,无毒,食之止口烂,治消渴,进食。(和名古毛)紫苔崔禹云∶味酸,小冷,无毒,生水底石上,食之止消渴。(和名须牟乃利。)蕺《本草》云∶味辛,微温,多食令人气喘。《陶景注》云∶不利人肺,恐闭气故也。

(今按∶损害物,和名之不支。)葱《本草》云∶葱实味辛,温,无毒,主明目,补中不足。茎主伤寒,寒热,出汗,中风,面目肿,喉痹不通,安胎,除肝邪气,利五脏,杀百药毒。崔禹云∶其茎白者,性冷,青者性热。根主伤寒头痛。《七卷经》云∶味辛,温,不可食,伤人心气。(和名纯。)薤《本草》云∶味辛苦,温,无毒,主金疮疮败,轻身,不饥,耐老,除寒热,温中,利病患。《拾遗》云∶调中,主久利不瘥,大腹内常恶者,但多煮食之。苏敬云∶薤有赤白二种,白者补而美,赤主金疮。崔禹云∶食,长毛发。孟诜云∶长服之可通神灵,甚安魂魄,续筋力。(和名于保美良。)韭《本草》云∶味辛酸,温,无毒,主安五脏,除胃热,利病患,可久食。根主养发。

《陶注》云∶是养性所忌。孟诜云∶冷,气人,可煮长服之。《拾遗》云∶温中下气,补虚调和腑脏,令人能食,止泄白脓,腹冷痛,蒸煮食之。叶根捣绞汁,服解诸药毒狂犬咬人,亦杀蛇虺蝎恶虫毒。又汁多,服主胸痹骨痛。俗云韭菜是草钟乳,言其宜人信然也。(和名已美良。)苏《本草》云∶味辛,温,无毒,归脾肾,主霍乱,腹中不安;消谷理胃,温中,除邪痹毒瓦斯。崔禹云∶性温,薰臭中风冷霍乱,煮饮汁至良。或云∶主腹中生疮及疝瘕。《七卷经》云∶损人,不可长食。孟诜云∶大蒜,热,除风杀虫毒瓦斯。(今按∶损害物,和名已比留。)葫《本草》云∶味辛,温,有毒,散痈肿疮,除风邪,杀毒瓦斯。独子者亦佳。归五脏,久食伤人,损目明。《陶注》云∶葫为大蒜,蒜为小蒜,俗人作齑,以啖脍肉,损性伐命,莫此之甚。《拾遗》云∶葫大蒜去水恶、瘴气,除风湿,破冷气。烂癖,伏邪恶,宣通温补,无已加之。初食不利目,多食却明,使毛发白。合皮截却两头吞之名为内灸。崔禹云∶味辛KT,大温,杀鬼毒诸气。云独子者曰葫,少者如百合。片者曰蒜,以作齑;合虫鱼肉鸟食之为快味。或云久食损性伐命者,今常啖之无有损,是事为不可信耳。但服药曰慎辟之。马琬云∶不益药性,若直尔啖之,亦应通气。《千金方》云∶多食生葫,行房伤肝气,令人面色无。(今按∶损害物、和名于保比留。)蜀椒《本草》云∶味辛,大热,有毒,主邪气咳逆,温中,下气,逐骨节皮肤肌寒温痹痛,除五脏六腑寒冷,心腹留饮宿食肠下利,结,水肿,黄胆,鬼注,蛊毒,杀虫鱼,久服头不白。轻身,增年,坚齿发,耐寒暑。崔禹云∶食之温中五脏六腑冷风。孟诜云∶除客热,不可久食,钝人性灵。《养生要集》云。椒,闭口及色白者,食蒸杀人。(今按∶损害物,和名不佐波之加美。)菊《本草》云∶味苦甘,平,无毒,主风头,头眩,肿痛,目欲脱,泪出,皮肤死肌,恶风,湿痹痛,去来陶陶,胸中烦热,安肠胃,利五脏,调四肢,久服轻身延年耐老。崔禹云∶仙经以菊为妙药,吴孺子三月三日生,生日常摘菊苗蒸煮啖之,遇于青归子,俱共游于芳壶,一云石台,遂乘于紫云,升于青天。大补,成好,啖其花头不白筋不KT之。(和名支久。)
跋

(半井氏藏有《医心方》二部,一为卷子本,盖是永观中丹波康赖所撰进原本,而间以他本补缀之,卷末有女医博士光成跋。一为粘叶本,即临写卷子本者。而卷子本之蚀叵读者可据以补完也。惜此本乙卯冬罹池鱼之灾,独卷子本官有雕刻之举得以全然存于今日,殆亦非偶然。但粘叶本卷末记撰进岁月及康赖猝日寿数是他书所不载,唯此可据以征信,则真为可贵重。故今影刻二跋以示同好云。安政丁巳重五福山源立之。)是书校刊未全,不幸会(臣琰)先人谢世。未及,(臣佶)先人亦复相继见背。不肖等恸哭之余,窃恐是刻之迁延不果,无以报二先人于地下。既而,不肖等承乏忝袭先职,乃孤陋不自揣,敢任校佳之佳之责,而二三子亦皆密勿从事,始能毕功。樱愉札记则小岛尚真、高岛久贯、涩江全善、森立之、佐藤苌等最与有力焉。既而,尚真、全善先没,而久贯、立之等专任其责,唯是,此书之成。距今九百年所,其所援引各书并系唐人旧帙。在今日大率散逸不传,或者其所依之本派别不一,或者今本经后人删改,猝难证明。矧乃其间字画僻异、不易识别者有之,文义晦涩、不易读定者有之;简断墨KT、不可复问者有之。今不敢苟且迁就,妄为之说。半井氏所藏,别有延庆旧抄册子本,其第二十五、二十六、二十八卷系延享四年和气成庸所补抄,校以是本,亦互相出入,则爰从而疏记之。樱愉第三十卷末记是书撰进岁月及其猝岁月日,亦足以补史氏之缺,则并附刻以资考镜。他仁和王府所藏凡十六卷,旧藏零本凡四卷,亦时有异同。今皆一一条举之。若夫微作、率作KT、暑作署、覆作覆、枣作KT、狗作KT及偏旁之木手并通、头、KT头之互用。凡皆文本异构,非关指义,则均不敢辞费。樱愉背记数条,一从原帙影摹以附后。至于背纸,有用当时牒状者,有用具注历本者,诸古记遗文散出,各处固多,考古者所不废意者此类,与本书不相涉,一概滥载,极为不伦,则令皆从略。呜呼,自二先人有斯举,盍屡易裘葛其间存殁之感。有不堪然者,(臣琰臣佶)自愿闻见陋,曾不足窥先业万一,独是数百年欲见而不得之珍,一朝发光,医方之传,可沿溯以得其津涯,则庶几乎医道之日以益明,盍不唯见二先人所以拳拳校刻是书之功之伟。抑亦昭代休明之运,举一世而跻之于仁寿,其所沾被者远矣。万延纪元岁次上章滩且月既望侍医医学教谕法眼(臣)多纪(元琰),侍医医学教谕兼督务法眠(臣)多纪(元佶)拜手同识。



\chapter{诸病不治证第二}

《医门方》云∶论曰∶夫人有病,皆起于脏腑;生死之候,乃见于容色。犹如影响报应,必不差违,当审察之,万无失一。其中形证具列后条。凡人无病及有病,常反眼上皮看,其中有赤脉从上下欲贯瞳子者,一脉一年死,二脉二年死。若未下者,可疗也。人急暴肥而愦(人面忽有赤色出而颊颧(距员反,《广雅》曰∶也,颊骨也)上大如指者卒死,至鼻头亦卒有赤色若黑色,忽从额上起下至鼻头卒死。

黑色忽出额上,大如指,无病而卒死。忽有黑色横鼻上,或至眉下,不出月卒死。

面上忽有青色如悬帚者,须臾死。鼻上至眉额,忽色如马肝;望之如青,近之如黑,不出百面卒虚肥,正白无血理者,方死不久。眼中睛上白色如半米者死。面忽青而眼黑若赤者死,面忽赤而眼白若目青者立死。眼卒视,无所见者死。病患身有臭气异常者死。病患忽悲泣者病患忽洁然汗出者死。病患面青白,目张失明;及面赤、唇骞,及面目鼻或耳无轮廓,或舌病患脊痛,腰中重,不可反复者,胃绝五日死。

病患泄利,不觉出时者,脾绝十二日死。

病患手足爪甲青,呼骂不止者,筋绝九日死。

病患发直如干麻,或白汗出不止,肠绝六日死。

病患面青,但欲伏眼目视不见人,泣出不止者,肝绝八日死。

病患齿暴痛,面正黑,人中黄色,腰中如折,白汗出如流水者,肾绝四日死。

病患两目有黄色者,不久方愈。病患耳目鼻边有黑色起入口者,死,十有三活。病患面两口,甲下病患目无精光,齿黑者,不治。病患足趺上肿,两膝大如斗,十日死。病患身臭,不治。

病患面青目黄,百日死。病患口张者,三日死。病患阴囊茎肿者,死。病患妄语错乱及不能《葛氏方》云∶凡肿有五不治,面肿仓黑,肝败不治;掌肿无理满满,心败不治;脐满肿反《扁鹊传》云∶病有六不治∶骄恣不论理,一不治也;轻身重财,二不治也;衣食不能适,不治《本草经》云∶仓公有言∶病不肯服药,一死;信巫不信医,二死。

\chapter{服药节度第三}

服药节度第三

《千金方》云∶扁鹊曰∶人之所依者形也;乱于和气者,病也;理于烦毒者药也;济命扶厄者不又云∶夫为医者,当须洞视病源,知其所犯,以食治之。食疗不愈,然后命药。药性刚烈,又云∶仲景曰∶欲治诸病,当先以汤洗除五脏六腑间,开通诸脉,理道阴阳,荡中破邪,润散,散能逐邪。风气、湿痹,表里移走,居无常处,散当平之。次用丸,丸药能逐风冷,破积聚,消诸坚瘕;进饮食,调营卫,能参合而行之者,可谓上工。医者意也。(营卫,《千金方》曰∶荣者,络脉之气通;卫者,经脉之气通;营出中焦,卫出上焦。)《养生要集》云∶张仲景曰∶人体平和,唯好自将养,勿妄服药。药势偏有所助,则令人脏又云∶悟论服药曰∶夫欲服食,当寻性理所宜,审冷暖之适,不可见彼得力,我便服之。

《本草经》云∶治寒以热药,治热以寒药。饮食不消,以吐下药。鬼注蛊毒,以毒药;痈肿又云∶病在胸膈以上者,先食后服药;病在心腹以下,先服药而后食;病在四肢血脉者,宜《抱朴子》云∶按中黄子服食节度曰∶服治病之药,以食前服之;服养生之药,以食后服之行,不得调之,不然,冷方转增或冷患热时治之,不可一又云∶病力弱者形肉多消,欲治之法,先以平和汤一两剂少服,通调血气,令病患力渐渐强又云∶其病或年远而人仍强,或得病日近病患已致瘦弱。此二种病乃是腑脏受纳病别故尔。凡脏病皆年远始成,腑病日近寻剧。五脏为阴,六腑为阳,阴病难治,阳病易治。阴阳二病服汤病者邪者日服行于又云∶春夏不可合吃热药,秋冬不可合吃冷药,但看病患冷热也。

又云∶病有新旧疗法不同,邪在毫毛,宜服膏及以摩之。不疗,二十日入于孙脉,宜服药酒脉,宜谓之又云∶凡服补汤者,相去远;久服泻汤,相去近。

《短剧方》云∶凡病剧者人必弱,人弱则不胜药,处方宜用分两单省者也。病轻者人则强,凡久病者日月已积,必损于食力;食力既弱,亦不胜药,处方亦宜用分两单省者也。新病者少壮者病虽重,其人壮,气血盛,胜于药,处方宜用分两重复者也。虽是优乐人,其人骤病宜用夫人壮病轻而用少分两方者,人盛则胜药势,处方分两单省者则不能制病,虽积服之,其势虚衰是又云∶自有小盛之人,不避风凉,触犯禁忌,暴竭精液;虽得微疾,皆不可轻以利药下之。一利便竭其精液,因滞着床席,动经年岁也。初始皆宜与平药治也。宜利者,乃转就下之耳夫长宿人病,宜服利汤药者,未必顿尽一剂也。皆视其利多少且消息之于一日之宽也。

病源源宜夫病是服利汤得瘥者,从此以后,慎不中服补汤也。得补病势则还复成也。重就利之,其人垂平夫有常患之人,不妨行走,气力未衰,欲将补益。冷热随宜丸散者,乃可先服利汤下便,除复有虚人积服补药,或中实食为害者,可止服利药除之。复有平实之人暴虚空竭者,亦宜以夫极虚极劳,病应服补汤者,风病应服治风汤者,此皆非五三剂可知也。自有滞风洞虚,积《千金方》云∶凡人年四十以下,有病可服泻药;不甚须服补药,必有所损,不在此限。

四术又云∶必有脏腑积聚,无问少长,须泻;必有虚损,无问少长,须补。以意商量而用之。

又云∶每春秋皆须与服转泻药之一度,则不中天行时气也。

又云∶凡用药皆随土地所宜。江南岭表,其地暑热,肌肤薄脆,腠理开疏,用药轻者;关中又云∶凡服痢汤,欲得侵早。

凡服汤欲得如法。汤热服之,则易消下不吐,若冷则吐呕不下,若大热则破人咽喉。务在用汤来又云∶凡服汤皆分三升为三服,然承病儿谷气强,前一服最须多,次一服如少,次后一服最又云∶凡服补汤,欲得服三升半。昼三夜一,中间鬲食,则汤气溉灌百脉,易得药力。

如此皆可又云∶凡服药三日慎酒,汤忌酒故也。

又云∶凡服治风汤等,一服浓覆取汗,若得汗即须薄覆,勿令大汗。中间亦须间食,不尔,令人无力,更益虚羸。

又云∶凡饵汤药,其粥食肉菜皆须大熟,大熟则易消,与药相宜。若生则难消。又复损药,又云∶凡服泻汤及诸丸散酒等,至食时须食者,皆先与一口冷醋饭,须臾乃进食佳也。

又云∶凡丸药皆如梧子,补者十丸为始,从十渐加,不过四十丸为限。过此,虚人亦一日三徒弃又云∶凡服泻丸,不过以痢为度。慎勿过多,令人下痢无度,大损人。

又云∶凡人忽遇风,发身心顿恶,或不能言。如此者当服大小续命及西州续命,排风越婢等出不日五夜服汤不绝,即经二日停汤,以美羹自补,消息四体。若小瘥,当即停药,渐渐将息。

如其不瘥,当更服汤攻之,以瘥为限。

又云∶凡患风服汤,非得大汗,其风不去,所以诸风方中皆有麻黄。至如西州续命用八两,药,《葛氏方》云∶凡服药不言先食后食者,皆在食前。其应食后者,自各说之。

凡服汤云分三服再服者,要视病源候或疏或数,足令势力相及。毒利之药,皆须空腹,补汤凡服丸散不云酒水饮者,本方如此;而别说用酒水,则此可通得以水饮服之。

《删繁论》云∶凡禁之法,若汤有触服,竟五日忌之。若丸散酒中有相违触,必须服药竟,之后十日方可饮啖。若药有乳石,复须一月日外。若不如尔,非唯不得力,反致祸也。

\chapter{服药禁物第四}

服药禁物第四

《本草经》云∶服药不可多食生葫蒜、杂生菜。

又云∶服药不可食诸滑物果菜。

又云∶服药不可多食肥猪犬肉肥羹及鱼脍。

又云∶服药通忌见死尸及产妇诸淹秽事。

又云∶服药有术,勿食桃李及雀肉、葫蒜、青鱼。有巴豆勿食芦笋羹及猪肉。(今按∶《黄连桔梗勿食猪肉。(今按∶《范汪方》∶食之精漏少子,使人缩产,饮不可下,病不除)。有半夏菖蒲勿食饴糖及羊肉。(今按∶《范汪方》云∶令人病不除。《膳夫经》云∶三日勿食秫)。有细辛勿食生菜。(今按∶《范汪方》云∶食之病增)。有甘草勿食菘。(今按∶《范汪方有藜芦,勿食狸肉。(今按∶《范汪方》云∶食之使人水道逆上成腹胀)。有牡丹勿食生葫叶(《新修本草》无此字)。今按∶《范汪方》云∶一日勿食葫,病增。《膳夫经》云∶二日勿食生蒜,病增〕。有恒山勿食生葱菜。(今按∶《范汪方》云∶食之增病)。有空青、朱沙,勿食生血物。(今按∶《养生要集》云∶病不除)。有茯云∶勿食诸酢热。《玉葙方》云∶茯苓忌鲤鱼)。

《养生要集》云∶服药不可食诸滑物、果、实、菜、油、面、生、冷、酢。

又云∶服药不可多食生葫蒜、杂生菜、猪肉、鱼臊脍。

又云∶服药有松脂,勿食五肉鱼菜盐酱,唯得饮水并小酒脯耳。

又云∶服药有天门冬,忌鲤鱼。

又云∶服药有黄精,忌食梅。

《枕中方》云∶凡服药物,不欲食蒜、石榴、猪肝,房室都绝之为上,服神药物勿向北方,又云∶凡服食,忌血味,使三尸不去。

《千金方》云∶凡饵药之人不可服鹿肉,服药必不得力。所以然者,鹿恒食解毒之草,是故能制散诸药也。

又云∶凡服药,皆断生、冷、酢、滑、猪、鸡、鱼、油、面、蒜。其大补丸慎陈臭。

又云∶服药有柏子,忌食面、五肉、鱼、菜。

《慧日寺药方》云∶服桂勿食鲤鱼,害人。(今按∶《养生要集》云,葱桂不可今食,伤人)《马琬食经》云∶服杏仁忌食猪肉,杀人。

《药像》云∶服槐实,忌猪肉。

\chapter{服药中毒方第五}
服药中毒方第五

《葛氏方》治服药过剂及中毒,多烦闷欲死方∶刮东壁土,少少以水三升饮之。

又方∶捣蓝青,绞取汁,服数升。无蓝者,立浣新青布,若绀缥,取汁饮之。

又方∶烧犀角,末之,服方寸匕。

又方∶捣荷根,取汁饮一二升。夏用叶。

又方∶服药失度、腹中苦烦者方∶饮生葛根汁,大良。无生者,捣干者,水服五合,亦可煮之。

又方∶釜月下黄土末,服方寸匕。

又云∶服药吐不止者方∶取猪膏,大如指长三寸者,煮令热,尽吞之。

又方∶饮新汲冷水一升即止。

又云∶若药中有巴豆下利不止者方∶末干姜、黄连,服方寸匕。

又方∶煮豉服一升。

又云∶诸药各有相解,然虽常储,今但取一种而兼解众毒,求之易得者。取甘草,咀,浓又方∶煮桂,多饮其汁,并多食葱叶中涕也。

又方∶煮大豆,令浓,多饮其汁。无豆者,豉亦可用。

凡煮此诸药,饮其汁以解毒。虽危急亦不可热饮之。诸毒得热皆更甚,宜扬令小冷也。

《集验方》治服药过剂烦闷方∶研粳米,取汁五升服之。

《医门方》云∶若方中有大黄、芒硝、槟榔仁等利不止者,冷粥及冷冻饮料吃半升即止。又有巴饮投《本草经》云服药过剂闷乱者方∶吞鸡子黄,又蓝汁,又水和胡粉,又土浆,又荷汁,又粳米沉汁,又豉汁,又干姜,黄连又云∶百药毒,用甘草、荠(齐礼反)(泥礼反)、大小豆汁、蓝汁及实皆解之。今按∶《马射冈毒∶用蓝汁、大小豆汁、竹沥、大麻子汁、藕、艾汁并解之。

野葛毒∶用鸡子汁、葛根汁、甘草汁、鸭头、热血、温猪膏并解之。

斑苗芫青毒∶用猪膏、大豆汁、戎盐、蓝汁及盐汤煮猪肠及巴豆并解之。

野狼毒毒∶用蓝汁、白蔹及盐汁、木占斯并解之。

踯躅毒∶用栀子汁解之。

巴豆毒∶煮黄连汁、大豆汁、生藿汁、菖蒲屑汁、煮寒水石汁并解之。(今按∶《拾遗》云藜芦毒∶用雄黄屑煮葱汁,温汤并解之。

雄黄毒∶用防己解之。

蜀椒毒∶用葵子汁煮桂汁、豉汁、人尿及冷水及食土,食蒜鸡毛烧咽并解之。(今按∶《葛半夏毒∶用生姜汁及煮干姜汁并解之。

石毒∶用大豆汁、白鹅膏并解之。

芫花毒∶用防风、防己、甘草、桂汁并解之。

乌头、天雄、附子毒∶用大豆汁、远志、防风、枣肌(肥)、饴糖并解之。

大戟毒∶用菖蒲汁解之。

桔梗毒∶用粥解之。

杏仁毒∶用蓝子汁解之。

诸菌毒∶堀地作坎,以水沃中,搅令浊,俄顷饮之,名地浆。

防葵毒∶用葵根汁解之。(苏敬云∶防葵,按《本经》∶无毒。试用亦无毒。今用葵根汁解莨菪毒∶用荠、甘草、升麻、犀角、蟹并解之。

马刀毒∶用清水解之。

野芋毒∶用土浆及粪汁并解之。

鸡子毒∶用淳酢解之。

铁毒∶用磁石解之。

金毒∶服水银数两即出。又鸭血及鸡子汁;又水淋鸡矢汁并解。(今按∶《葛氏方》云∶取石药中毒∶白鸡矢解之。人参汁亦佳。

\chapter{合药料理法第六}
\chapter{药斤两升合法第七}
\chapter{药不入汤酒法第八}
\chapter{药畏恶相反法第九}
\chapter{诸药和名第十}

\part{2}
\chapter{孔穴主治法第一}
\chapter{诸家取背输法第二}
\chapter{针禁法第三}
\chapter{灸禁法第四}
\chapter{针例法第五}
\chapter{灸例法第六}
\chapter{针灸服药吉凶日第七}
\chapter{人神所在法第八}
\chapter{天医扁鹊天德所在法第九}
\chapter{月杀厄月KT日法第十}
\chapter{作艾用火(法)灸治颂第十一}
\chapter{明堂图第十二}

\part{3}
\chapter{风病证候第一}

\chapter{治一切风病方第二}
\chapter{治偏风方第三}
\chapter{治半风方第四}
\chapter{治风痉方第五}
\chapter{治柔风方第六}
\chapter{治头风方第七}
\chapter{治中风口噤方第八}
\chapter{治中风口方第九}
\chapter{治中风舌强方第十}
\chapter{治中风失音方第十一}
\chapter{治中风声嘶方第十二}
\chapter{治声噎不出方第十三}
\chapter{治中风惊悸方第十四}
\chapter{治中风四肢不屈伸方第十五}
\chapter{治中风身体不仁方第十六}
\chapter{治中风身体如虫行方第十七}
\chapter{治中风隐疹方第十八}
\chapter{治中风隐疹疮方第十九}
\chapter{治中风癞病方第二十}
\chapter{治中风言语错乱方第二十一}
\chapter{治中风癫病方第二十二}
\chapter{治中风狂病方第二十三}
\chapter{治虚热方第二十四}
\chapter{治客热方第二十五}

\part{4}
\chapter{治发令生长方第一}
\chapter{治发令光软方第二}
\chapter{治发令竖方第三}
\chapter{治白发令黑方第四}
\chapter{治须发黄方第五}
\chapter{治须发秃落方第六}
\chapter{治头白秃方第七}
\chapter{治头赤秃方第八}
\chapter{治鬼舐头方第九}
\chapter{治头烧处发不生方第十}
\chapter{治眉脱令生方第十一}
\chapter{治毛发妄生方第十二}
\chapter{治头面疮方第十三}
\chapter{治面疮方第十四}
\chapter{治面方第十五}
\chapter{治鼻方第十六}
\chapter{治饲面方第十七}
\chapter{治疡方第十八}
\chapter{治白癜方第十九}
\chapter{治赤疵方第二十}
\chapter{治黑子方第二十一}
\chapter{治疣目方第二十二}
\chapter{治疮瘢方第二十三}
\chapter{治狐臭方第二十四}

\part{5}
\chapter{治耳聋方第一}
\chapter{治耳鸣方第二}
\chapter{治耳卒痛方第三}
\chapter{治耳方第四}
\chapter{治耳耵聍方第五}
\chapter{治百虫入耳方第六}
\chapter{治蜈蚣入耳方第七}
\chapter{治蚰蜓入耳方第八}
\chapter{治蚁入耳方第九}
\chapter{治飞蛾入耳中方第十}
\chapter{治水入耳方第十一}
\chapter{治耳中有物不出方第十二}
\chapter{治目不明方第十三}
\chapter{治目清盲方第十四}
\chapter{治雀盲方第十五}
\chapter{治目肤翳方第十六}
\chapter{治目赤白膜方第十七}
\chapter{治目息肉方第十八}
\chapter{治目珠管方第十九}
\chapter{治目珠子脱出方第二十}
\chapter{治眼肿痛方第二十一}
\chapter{治目赤痛方第二十二}
\chapter{治目胎赤方第二十三}
\chapter{治目痒痛方第二十四}
\chapter{治目赤烂方第二十五}
\chapter{治目泪出方第二十六}
\chapter{治目为物所中方第二十七}
\chapter{治竹木刺目方第二十八}
\chapter{治稻麦芒入目方第二十九}
\chapter{治芒草沙石入目方第三十}
\chapter{治鼻塞涕出方第三十一}
\chapter{治鼻中息肉方第三十二}
\chapter{治鼻中生疮方第三十三}
\chapter{治鼻痛方第三十四}
\chapter{治鼻中燥方第三十五}
\chapter{治鼻衄方第三十六}
\chapter{治鼻中物入方第三十七}
\chapter{治紧唇生疮方第三十八}
\chapter{治唇生核方第三十九}
\chapter{治唇黑肿硬方第四十}
\chapter{治唇破方第四十一}
\chapter{治唇面KT方第四十二}
\chapter{治口舌生疮方第四十三}
\chapter{治口舌出血方第四十四}
\chapter{治九窍四肢出血方第四十五}
\chapter{治呕血方第四十六}
\chapter{治吐血方第四十七}
\chapter{治唾血方第四十八}
\chapter{治口中烂痛方第四十九}
\chapter{治口吻疮方第五十}
\chapter{治口舌干焦方第五十一}
\chapter{治口臭方第五十二}
\chapter{治张口不合方第五十三}
\chapter{治舌肿强方第五十四}
\chapter{治重舌方第五十五}
\chapter{治悬雍卒长方第五十六}
\chapter{治风齿痛方第五十七}
\chapter{治龋齿痛方第五十八}
\chapter{治齿碎坏方第五十九}
\chapter{治齿令坚方第六十}
\chapter{治齿动欲脱方第六十一}
\chapter{治齿黄黑方第六十二}
\chapter{治齿败臭方第六十三}
\chapter{治齿龈肿方第六十四}
\chapter{治齿龈间血出方第六十五}
\chapter{治牙齿痛方第六十六}
\chapter{治牙齿后涌血方第六十七}
\chapter{治齿方第六十八}
\chapter{治齿方第六十九}
\chapter{治喉痹方第七十}
\chapter{治马喉痹方第七十一}
\chapter{治喉咽肿痛方第七十二}
\chapter{治尸咽方第七十三}
\chapter{治咽中如肉脔方第七十四}

\part{6}
\chapter{治胸痛方第一}
\chapter{治胁痛方第二}
\chapter{治心痛方第三}
\chapter{治腹痛方第四}
\chapter{治心腹痛方第五}
\chapter{治心腹胀满方第六}
\chapter{治卒腰痛方第七}
\chapter{治概腰痛方第八}
\chapter{治肾着腰痛方第九}
\chapter{治肝病方第十}
\chapter{治心病方第十一}
\chapter{治脾病方第十二}
\chapter{治肺病方第十三}
\chapter{治肾病方第十四}
\chapter{治大肠病方第十五}
\chapter{治小肠病方第十六}
\chapter{治胆病方第十七}
\chapter{治胃病方第十八}
\chapter{治膀胱病方第十九}
\chapter{治三焦病方第二十}
\chapter{治气病方第二十一}
\chapter{治脉病方第二十二}
\chapter{治筋病方第二十三}
\chapter{治骨病方第二十四}
\chapter{治髓病方第二十五}
\chapter{治皮病方第二十六}
\chapter{治肉病方第二十七}

\part{7}
\chapter{治阴疮方第一}
\chapter{治阴蚀疮欲尽方第二}
\chapter{治阴痒方第三}
\chapter{治阴茎肿痛方第四}
\chapter{治阴囊肿痛方第五}
\chapter{治阴卵入腹急痛方第六}
\chapter{治阴囊湿痒方第七}
\chapter{治阴颓方第八}
\chapter{治脱肛方第九}
\chapter{治谷道痒痛方第十}
\chapter{治谷道赤痛方第十一}
\chapter{治谷道生疮方第十二}
\chapter{治湿(吕质反,小虫也)方第十三}
\chapter{治疳(居酣反)湿方第十四}
\chapter{治诸痔方第十五}
\chapter{治九虫方第十六}
\chapter{治三虫方第十七}
\chapter{治寸白方第十八}
\chapter{治蛔(胡恢反,人腹中长虫也)虫方第十九}
\chapter{治蛲(如消反,腹中虫也)虫方第二十}

\part{8 手足}
\chapter{香港脚所由第一}
\chapter{香港脚形状第二}
\chapter{香港脚轻重第三}
\chapter{香港脚姑息法第四}
\chapter{香港脚疗体第五}
\chapter{香港脚肿痛方第六}
\chapter{香港脚屈弱方第七}
\chapter{香港脚入腹方第八}
\chapter{香港脚胀满方第九}
\chapter{香港脚冷热方第十}
\chapter{香港脚转筋方第十一}
\chapter{香港脚灸法第十二}
\chapter{香港脚禁忌第十三}
\chapter{香港脚禁食第十四}
\chapter{香港脚宜食第十五}
\chapter{治足肿方第十六}
\chapter{治尸脚方第十七}
\chapter{治肉刺方第十八}
\chapter{治手足冻肿疮方第十九}
\chapter{治手足皲裂方第二十}
\chapter{治手足发胝方第二十一}
\chapter{治手足逆胪方第二十二}
\chapter{治代指方第二十三}
\chapter{治指掣痛方第二十四}

\part{9}
\chapter{治咳嗽方第一}
\chapter{治喘息方第二}
\chapter{治短气方第三}
\chapter{治少气方第四}
\chapter{治气噎方第五}
\chapter{治奔豚方第六}
\chapter{治痰饮方第七}
\chapter{治癖食方第八}
\chapter{治胃反吐食方第九}
\chapter{治宿食不消方第十}
\chapter{治寒冷不食方第十一}
\chapter{治上热下冷不食方第十二}
\chapter{治谷劳欲卧方第十三}
\chapter{治恶心方第十四}
\chapter{治噫酢方第十五}
\chapter{治呕吐方第十六}
\chapter{治干呕方第十七}
\chapter{治哕方第十八}

\part{10}
\chapter{治积聚方第一}
\chapter{治诸疝方第二}
\chapter{治七疝方第三}
\chapter{治寒疝方第四}
\chapter{治八痞方第五}
\chapter{治症瘕方第六}
\chapter{治暴症方第七}
\chapter{治蛇瘕方第八}
\chapter{治鳖瘕方第九}
\chapter{治鱼瘕方第十}
\chapter{治肉瘕方第十一}
\chapter{治发瘕方第十二}
\chapter{治米症方第十三}
\chapter{治水瘕方第十四}
\chapter{治食症方第十五}
\chapter{治酒瘕方第十六}
\chapter{水病证候第十七}
\chapter{治大腹水肿方第十八}
\chapter{治通身水肿方第十九}
\chapter{治十水肿方第二十}
\chapter{治风水肿方第二十一}
\chapter{治水癖方第二十二}
\chapter{治身面卒肿方第二十三}
\chapter{治犯土肿方第二十四}
\chapter{治黄胆方第二十五}
\chapter{治黄汗方第二十六}
\chapter{治谷疸方第二十七}
\chapter{治酒疸方第二十八}
\chapter{治女劳疸方第二十九}
\chapter{治黑疸方第三十}

\part{11}
\chapter{治霍乱方第一}
\chapter{治霍乱心腹痛方第二}
\chapter{治霍乱心腹胀满方第三}
\chapter{治霍乱心烦方第四}
\chapter{治霍乱下利不止方第五}
\chapter{治霍乱呕吐不止方第六}
\chapter{治霍乱呕哕(于越反)方第七}
\chapter{治霍乱干呕方第八}
\chapter{治霍乱烦渴方第九}
\chapter{治霍乱转筋方第十}
\chapter{治霍乱手足冷方第十一}
\chapter{治霍乱不语方第十二}
\chapter{治霍乱欲死方第十三}
\chapter{治中热霍乱方第十四}
\chapter{治欲作霍乱方第十五}
\chapter{治霍乱后烦躁方第十六}
\chapter{治霍乱止后食法第十七}
\chapter{治下利方例第十八}
\chapter{治杂利方第十九}
\chapter{治冷利方第二十}
\chapter{治热利方第二十一}
\chapter{治赤利方第二十二}
\chapter{治血利方第二十三}
\chapter{治赤白利方第二十四}
\chapter{治久赤白利方第二十五}
\chapter{治白滞利方第二十六}
\chapter{治脓血利方第二十七}
\chapter{治水谷利方第二十八}
\chapter{治休息利方第二十九}
\chapter{治泄利方第三十}
\chapter{治重下方第三十一}
\chapter{治疳利方第三十二}
\chapter{治蛊注利方第三十三}
\chapter{治不伏水土利方第三十四}
\chapter{治呕逆吐利方第三十五}
\chapter{治利兼渴方第三十六}
\chapter{治利兼肿方第三十七}
\chapter{治利后虚烦方第三十八}
\chapter{治利后不能食方第三十九}
\chapter{治利后哕方第四十}
\chapter{治利后逆满方第四十一}
\chapter{治利后谷道痛方第四十二}

\part{12}
\chapter{治消渴方第一}
\chapter{治渴利方第二}
\chapter{治内消方第三}
\chapter{治诸淋方第四}
\chapter{治石淋方第五}
\chapter{治气淋方第六}
\chapter{治劳淋方第七}
\chapter{治膏淋方第八}
\chapter{治血淋方第九}
\chapter{治热淋方第十}
\chapter{治寒淋方第十一}
\chapter{治大小便不通方第十二}
\chapter{治大便不通方第十三}
\chapter{治大便难方第十四}
\chapter{治大便失禁方第十五}
\chapter{治大便下血方第十六}
\chapter{治小便不通方第十七}
\chapter{治小便难方第十八}
\chapter{治小便数方第十九}
\chapter{治小便不禁方第二十}
\chapter{治小便黄赤白黑方第二十一}
\chapter{治小便血方第二十二}
\chapter{治遗尿方第二十三}
\chapter{治尿床方第二十四}

\part{13}
\chapter{治虚劳五劳七伤方第一}
\chapter{治虚劳羸瘦方第二}
\chapter{治虚劳梦泄精方第三}
\chapter{治虚劳尿精方第四}
\chapter{治虚劳精血出方第五}
\chapter{治虚劳少精方第六}
\chapter{治虚劳不得眠方第七}
\chapter{治昏塞喜眠方第八}
\chapter{治邪伤汗血方第九}
\chapter{治虚汗方第十}
\chapter{治风汗方第十一}
\chapter{治阳虚盗汗方第十二}
\chapter{治传尸病方第十三}
\chapter{治骨蒸病方第十四(三)}
\chapter{治肺痿方第十五(四)}

\part{14}
\chapter{治卒死方第一}
\chapter{治中恶方第二}
\chapter{治鬼击病方第三}
\chapter{治客忤方第四}
\chapter{治魇不寤方第五}
\chapter{治尸厥方第六}
\chapter{治溺死方第七}
\chapter{治热死方第八}
\chapter{治冻(东贡反)死方第九}
\chapter{治自缢死方第十}
\chapter{治注病方第十一}
\chapter{治诸尸方第十二}
\chapter{治诸疟方第十三}
\chapter{治鬼疟方第十四}
\chapter{治温疟方第十五}
\chapter{治寒疟方第十六}
\chapter{治痰实疟方第十七}
\chapter{治劳疟方第十八}
\chapter{治嶂疟方第十九}
\chapter{治间日疟方第二十}
\chapter{治连年疟方第二十一}
\chapter{治发作无时疟方第二十二}
\chapter{伤寒证候第二十三}
\chapter{伤寒不治候第二十四}
\chapter{避伤寒病方第二十五}
\chapter{治伤寒困笃方第二十六}
\chapter{治伤寒一二日方第二十七}
\chapter{治伤寒三日方第二十八}
\chapter{治伤寒四日方第二十九}
\chapter{治伤寒五日方第三十}
\chapter{治伤寒六日方第三十一}
\chapter{治伤寒七日方第三十二}
\chapter{治伤寒八九日方第三十三}
\chapter{治伤寒十日以上方第三十四}
\chapter{治伤寒阴毒方第三十五}
\chapter{治伤寒阳毒方第三十六}
\chapter{治伤寒汗出后不除方第三十七}
\chapter{治伤寒鼻衄方第三十八}
\chapter{治伤寒口干方第三十九}
\chapter{治伤寒唾血方第四十}
\chapter{治伤寒吐方第四十一}
\chapter{治伤寒哕方第四十二}
\chapter{治伤寒后呕方第四十三}
\chapter{治伤寒下利方第四十四}
\chapter{治伤寒饮食劳复方第四十五}
\chapter{治伤寒洗梳劳复方第四十六}
\chapter{治伤寒交接劳复方第四十七}
\chapter{治伤寒病后头痛方第四十八}
\chapter{治伤寒病后不得眠方第四十九}
\chapter{治伤寒病后汗出方第五十}
\chapter{治伤寒后目病方第五十一}
\chapter{治伤寒后黄胆方第五十二}
\chapter{治伤寒后虚肿方第五十三}
\chapter{治伤寒手足肿痛欲脱方第五十四}
\chapter{治伤寒后下利方第五十五}
\chapter{治伤寒后下部痒痛方第五十六}
\chapter{治伤寒豌豆疮方第五十七}
\chapter{伤寒后食禁第五十八}
\chapter{治伤寒变成百合病方第五十九}
\chapter{治时行后变成疟方第六十}

\part{15}
\chapter{说痈疽所由第一}
\chapter{治痈疽未脓方第二}
\chapter{治痈疽有脓方第三}
\chapter{治痈发背方第四}
\chapter{治附骨疽方第五}
\chapter{治石痈方第六}
\chapter{治痤疖方第七}
\chapter{治疽方第八}
\chapter{治久疽方第九}
\chapter{治缓疽方第十}
\chapter{治甲疽方第十一}
\chapter{治肠痈方第十二}
\chapter{治肺痈方第十三}

\part{16}
\chapter{治疔疮方第一}
\chapter{治犯疔疮方第二}
\chapter{治毒肿方第三}
\chapter{治风毒肿方第四}
\chapter{治风肿方第五}
\chapter{治热肿方第六}
\chapter{治气肿方第七}
\chapter{治气痛方第八}
\chapter{治恶核肿方第九}
\chapter{治恶肉方第十}
\chapter{治恶脉病方第十一}
\chapter{治编病方第十二}
\chapter{治瘰方第十三}
\chapter{治瘿方第十四}
\chapter{治瘤方第十五}
\chapter{治诸方第十六}
\chapter{治野狼方第十七}
\chapter{治鼠方第十八}
\chapter{治蝼蛄方第十九}
\chapter{治蜂方第二十}
\chapter{治蚍蜉方第二十一}
\chapter{治蛴螬方第二十二}
\chapter{治浮沮方第二十三}
\chapter{治瘰方第二十四}
\chapter{治转脉方第二十五}
\chapter{治蜣螂方第二十六}
\chapter{治蚯蚓方第二十七}
\chapter{治蚁方第二十八}
\chapter{治蝎方第二十九}
\chapter{治虾蟆方第三十}
\chapter{治蛙方第三十一}
\chapter{治蛇方第三十二}
\chapter{治方第三十三}
\chapter{治雀方第三十四}
\chapter{治石方第三十五}
\chapter{治风方第三十六}
\chapter{治内方第三十七}
\chapter{治脓方第三十八}
\chapter{治冷方第三十九}

\part{17}
\chapter{治丹毒疮方第一}
\chapter{治癣疮方第二}
\chapter{治疥疮方第三}
\chapter{治恶疮方第四}
\chapter{治热疮方第五}
\chapter{治夏热沸烂疮方第六}
\chapter{治浸淫疮方第七}
\chapter{治王烂疮方第八}
\chapter{治反花疮方第九}
\chapter{治月蚀疮方第十}
\chapter{治恶露疮方第十一}
\chapter{治漆疮方第十二}
\chapter{治疮方第十三}
\chapter{治疽疮方第十四}
\chapter{治蠼疮方第十五}
\chapter{治诸疮烂不肯燥方第十六}
\chapter{治诸疮中风水肿方第十七}

\part{18}
\chapter{治汤火烧灼方第一}
\chapter{治灸疮不瘥方第二}
\chapter{治灸疮肿痛方第三}
\chapter{治灸疮血出不止方第四}
\chapter{治金疮方第五}
\chapter{治金疮肠出方第六}
\chapter{治金疮肠断方第七}
\chapter{治金疮伤筋断骨方第八}
\chapter{治金疮血出不止方第九}
\chapter{治金疮血内漏方第十}
\chapter{治金疮交接血惊出方第十一}
\chapter{治金疮中风方第十二}
\chapter{治金疮禁忌方第十三}
\chapter{治毒箭所伤方第十四}
\chapter{治箭伤血漏瘀满方第十五}
\chapter{治箭镞不出方第十六}
\chapter{治铁锥刀不出方第十七}
\chapter{治医针不出方第十八}
\chapter{治竹木壮刺不出方第十九}
\chapter{治被打伤方第二十}
\chapter{治折破骨伤筋方第二十一}
\chapter{治从高落重物所方第二十二}
\chapter{治从车马落方第二十三}
\chapter{治犬啮人方第二十四}
\chapter{治凡犬啮人方第二十五}
\chapter{治马咋人方第二十六}
\chapter{治马啮人阴卵方第二十七}
\chapter{治马骨刺人方第二十八}
\chapter{尿入疮方第二十九}
\chapter{治熊啮人方第三十}
\chapter{治猪啮人方第三十一}
\chapter{治虎啮人方第三十二}
\chapter{治狐尿毒方第三十三}
\chapter{治鼠咬人方第三十四}
\chapter{治众蛇螫人方第三十五}
\chapter{治蝮蛇螫人方第三十六}
\chapter{治青蛇螫人方第三十七}
\chapter{治蛇绕人不解方第三十八}
\chapter{治蛇入人口中方第三十九}
\chapter{治蛇骨刺人方第四十}
\chapter{治蜈蚣螫人方第四十一}
\chapter{治蜂螫人方第四十二}
\chapter{治蛎螫人方第四十三}
\chapter{治蝎螫人方第四十四}
\chapter{治蜘蛛啮人方第四十五}
\chapter{治蛭啮人方第四十六}
\chapter{治蚯蚓咬人方第四十七}
\chapter{治蛞蝓咬人方第四十八}
\chapter{治螈蚕毒方第四十九}
\chapter{治射工毒方第五十}
\chapter{治沙虱毒方第五十一}
\chapter{治水毒方第五十二}
\chapter{治井冢毒第五十三}
\chapter{辟蛊毒方第五十四}

\part{19}
\chapter{服石节度第一}
\chapter{服石反常性法第二}
\chapter{服石得力候第三}
\chapter{服石发动救解法第四}
\chapter{服石四时发状第五}
\chapter{服石禁忌法第六}
\chapter{服石禁食第七}
\chapter{诸丹论第八}
\chapter{诸丹服法第九}
\chapter{服丹宜食法第十}
\chapter{服丹禁食法第十一}
\chapter{服丹禁忌法第十二}
\chapter{服丹发热救解法第十三}
\chapter{服金液丹方第十四}
\chapter{服全阳丹方第十五}
\chapter{服石钟乳方第十六}
\chapter{服红雪方第十七}
\chapter{服紫雪方第十八}
\chapter{服五石凌方第十九}
\chapter{服金石凌方第二十}
\chapter{服金汞丹方第二十一}
\chapter{服银丸方第二十二}

\part{20}
\chapter{治服食除热解发方第一}
\chapter{治服石烦闷(莫围反)方第二}
\chapter{治服石头痛方第三}
\chapter{治服石耳鸣方第四}
\chapter{治服石目痛方第五}
\chapter{治服石目无所见方第六}
\chapter{治服石鼻塞方第七}
\chapter{治服石齿痛方第八(十二)}
\chapter{治服石咽痛方第九(十三)}
\chapter{治服石口干燥方第十(八)}
\chapter{痛方第十一(九)}
\chapter{治服石口中发疮方第十二(十)}
\chapter{治服石心噤方第十三(十一)}
\chapter{治服石心腹胀满方第十四}
\chapter{治服石心腹痛方第十五}
\chapter{治服石腰脚痛方第十六}
\chapter{治服石百节痛方第十七}
\chapter{治服石手足逆冷方第十八}
\chapter{治服石面上疮方第十九(二十)}
\chapter{治服石身体生疮方第二十(十九)}
\chapter{治服石结肿欲作痈方第二十一}
\chapter{治服石痈疽发背方第二十二}
\chapter{治服石身体肿方第二十三}
\chapter{治服石身体强直方第二十四}
\chapter{治服石发黄方第二十五}
\chapter{治服石呕(乌后反)逆方第二十六}
\chapter{治服石咳嗽方第二十七}
\chapter{治服石上气方第二十八}
\chapter{治服石痰方第二十九}
\chapter{治服石不能食方第三十}
\chapter{治服石酒热方第三十一}
\chapter{治服石淋小便难方第三十二}
\chapter{治服石小便不通方第三十三}
\chapter{治服石小便稠数方第三十四}
\chapter{治服石小便多方第三十五}
\chapter{治服石大小便难方第三十六}
\chapter{治服石大便难方第三十七}
\chapter{治服石大便血方第三十八}
\chapter{治服石下利方第三十九}
\chapter{治服石热渴方第四十}
\chapter{治服石冷热不适方第四十一}
\chapter{治服石补益方第四十二}
\chapter{治服石经年更发方第四十三}

\part{21}
\chapter{治妇人诸病所由第一}
\chapter{治妇人面上黑方第二}
\chapter{治妇人面上黑子方第三}
\chapter{治妇人妒乳方第四}
\chapter{治妇人乳痈方第五}
\chapter{治妇人乳疮方第六}
\chapter{治妇人阴痒方第七}
\chapter{治妇人阴痛方第八}
\chapter{治妇人阴肿方第九}
\chapter{治妇人阴疮方第十}
\chapter{治妇人阴中息肉第十一}
\chapter{治妇人阴冷方第十二}
\chapter{治妇人阴臭第十三}
\chapter{治妇人阴脱方第十四}
\chapter{治妇人阴大方第十五}
\chapter{治妇人小户嫁痛方第十六}
\chapter{治妇人阴丈夫伤方第十七}
\chapter{治妇人脱肛方第十八}
\chapter{治妇人月水不调方第十九}
\chapter{治妇人月水不通方第二十}
\chapter{治妇人月水不断方第二十一}
\chapter{治妇人月水腹痛方第二十二}
\chapter{治妇人崩中漏下方第二十三}
\chapter{治妇人下三十六疾方第二十四}
\chapter{治妇人八瘕方第二十五}
\chapter{治妇人遗尿方第二十六}
\chapter{治妇人尿血方第二十七}
\chapter{治妇人瘦弱方第二十八}
\chapter{治妇人欲男方第二十九}
\chapter{治妇人鬼交方第三十}
\chapter{治妇人令断生产方第三十一}

\part{22}
\chapter{妊妇脉图月禁法第一}
\chapter{妊妇修身法第二}
\chapter{妊妇禁食法第三}
\chapter{治妊妇恶阻(侧吕反,病)方第四}
\chapter{治妊妇养胎方第五}
\chapter{治妊妇闷冒方第六}
\chapter{治妊妇胎动不安方第七}
\chapter{治妊妇数落胎方第八}
\chapter{治妊妇胎堕血不止方第九}
\chapter{治妊妇堕胎腹痛方第十}
\chapter{治妊妇上迫心方第十一}
\chapter{治妊妇漏胞方第十二}
\chapter{治妊妇下黄汁方第十三}
\chapter{治妊妇顿仆举重去血方第十四}
\chapter{治妊妇猝走高堕方第十五}
\chapter{治妊妇为男所动欲死方第十六}
\chapter{治妊妇胸烦吐食方第十七}
\chapter{治妊妇心痛方第十八}
\chapter{治妊妇心腹痛方第十九}
\chapter{治妊妇腹痛方第二十}
\chapter{治妊妇腰痛方第二十一}
\chapter{治妊妇胀满方第二十二}
\chapter{治妊妇体肿方第二十三}
\chapter{治妊妇下利方第二十四}
\chapter{治妊妇小便数方第二十五}
\chapter{治妊妇尿血方第二十六}
\chapter{治妊妇淋病方第二十七}
\chapter{治妊妇遗尿方第二十八}
\chapter{治妊妇霍乱方第二十九}
\chapter{治妊妇疟方第三十}
\chapter{治妊妇温病方第三十一}
\chapter{治妊妇中恶方第三十二}
\chapter{治妊妇咳嗽方第三十三}
\chapter{治妊妇时病令子不落方第三十四}
\chapter{治妊妇日月未至欲产方第三十五}
\chapter{治妊妇胎死不出方第三十六}
\chapter{治妊妇欲去胎方第三十七}

\part{23}
\chapter{产妇向坐地法第一}
\chapter{产妇反支月忌法第二}
\chapter{产妇用意法第三}
\chapter{产妇借地法第四}
\chapter{产妇安产庐法第五}
\chapter{产妇禁坐草法第六}
\chapter{产妇禁水法第七}
\chapter{产妇易产法第八}
\chapter{治产难方第九}
\chapter{治逆产方第十}
\chapter{治横生方第十一}
\chapter{治子上迫心方第十二}
\chapter{治子死腹中方第十三}
\chapter{治胞衣不出方第十四}
\chapter{藏胞衣KT理法第十五}
\chapter{藏胞衣吉凶日法第十六}
\chapter{藏胞恶处法第十七}
\chapter{藏胞衣吉方第十八}
\chapter{妇人产后禁忌第十九}
\chapter{治产后运闷方第二十}
\chapter{治产后恶血不止方第二十一}
\chapter{治产后腹痛方第二十二}
\chapter{治产后心腹痛方第二十三}
\chapter{治产后腹满方第二十四}
\chapter{治产后胸胁痛方第二十五}
\chapter{治产后身肿方第二十六}
\chapter{治产后中风口噤方第二十七}
\chapter{治产后柔风方第二十八}
\chapter{治产后虚羸方第二十九}
\chapter{治产后不得眠方第三十}
\chapter{治产后少气方第三十一}
\chapter{治产后不能食方第三十二}
\chapter{治产后虚热方第三十三}
\chapter{治产后渴方第三十四}
\chapter{治产后汗出方第三十五}
\chapter{治产后无乳汁方第三十六}
\chapter{治产后乳汁溢满方第三十七}
\chapter{治产后妒乳方第三十八}
\chapter{治产后阴开方第三十九}
\chapter{治产后阴脱方第四十}
\chapter{治产后阴肿方第四十一}
\chapter{治产后阴痒方第四十二}
\chapter{治产后小便数方第四十三}
\chapter{治产后遗尿方第四十四}
\chapter{治产后淋病方第四十五}
\chapter{治产后尿血方第四十六}
\chapter{治产后下利方第四十七}
\chapter{治产后月水不调方第四十八}
\chapter{治产后月水不通方第四十九}
\chapter{治产后生疮方第五十}

\part{24}
\chapter{治无子法第一}
\chapter{知有子法第二}
\chapter{知胎中男女法第三}
\chapter{变女为男法第四}
\chapter{相子生年寿法第五}
\chapter{相子生月法第六}
\chapter{相子生六甲日法第七}
\chapter{相子男生日法第八}
\chapter{相子女生日法第九}
\chapter{相子生时法第十}
\chapter{相子生属月宿法第十一}
\chapter{生子二十八宿星相法第十二}
\chapter{为生子求月宿法第十三}
\chapter{相子生属七星图第十四}
\chapter{相子生命属十二星法第十五}
\chapter{相生子属七神图第十六}
\chapter{相子生四神日法第十七}
\chapter{禹相子生日法第十八}
\chapter{相子生五行用事日法第十九}
\chapter{相子生五行用事时法第二十}
\chapter{相子生熹母子胜忧时法第二十一}
\chapter{相生子死候第二十二}
\chapter{占推子寿不寿法第二十三}
\chapter{占推子与父母保不保法第二十四}
\chapter{占推子祸福法第二十五}
\chapter{相男子形色吉凶法第二十六}
\chapter{相女子形色吉凶法第二十七}

\part{25}
\chapter{小儿方例第一}
\chapter{小儿新生祝术第二}
\chapter{小儿去衔血方第三}
\chapter{小儿与甘草汤方第四}
\chapter{小儿与朱蜜方第五}
\chapter{小儿与牛黄方第六}
\chapter{小儿初与乳方第七}
\chapter{小儿哺谷方第八}
\chapter{小儿初浴方第九}
\chapter{小儿断脐方第十}
\chapter{小儿去鹅口方第十一}
\chapter{小儿断连舌方第十二}
\chapter{小儿刺悬痈方第十三}
\chapter{小儿变蒸第十四}
\chapter{小儿择乳母方第十五}
\chapter{小儿为名字法第十六}
\chapter{小儿初着衣方第十七}
\chapter{小儿调养方第十八}
\chapter{小儿禁食第十九}
\chapter{治小儿解颅方第二十}
\chapter{治小儿囟陷方第二十一}
\chapter{治小儿摇头方第二十二}
\chapter{治小儿发不生方第二十三}
\chapter{治小儿白秃方第二十四}
\chapter{小儿鬼舐头方第二十五}
\chapter{治小儿头疮方第二十六}
\chapter{治小儿头面身体疮方第二十七}
\chapter{治小儿面白屑方第二十八}
\chapter{治小儿耳鸣方第二十九}
\chapter{治小儿耳疮方第三十}
\chapter{治小儿耳方第三十一}
\chapter{治小儿耳中百虫入方第三十二}
\chapter{治小儿耳蚁入方第三十三}
\chapter{治小儿耳蜈蚣入方第三十四}
\chapter{治小儿耳蚰蜓入方第三十五}
\chapter{治小儿目不明方第三十六}
\chapter{治小儿目赤痛方第三十七}
\chapter{治小儿眼烂痒方第三十八}
\chapter{治小儿眼翳方第三十九}
\chapter{治小儿雀盲方第四十}
\chapter{治小儿目昧第四十一}
\chapter{治小儿目竹木刺方第四十二}
\chapter{治小儿目芝草沙石入方第四十三}
\chapter{治小儿眼为物撞方第四十四}
\chapter{治小儿燕口方第四十五}
\chapter{治小儿口疮方第四十六}
\chapter{治小儿口下黄肥疮方第四十七}
\chapter{治小儿唇疮方第四十八}
\chapter{治小儿紧唇方第四十九}
\chapter{治小儿口噤方第五十}
\chapter{治小儿重舌方第五十一}
\chapter{治小儿舌上疮方第五十二}
\chapter{治小儿舌肿方第五十三}
\chapter{治小儿齿晚生方第五十四}
\chapter{治小儿齿落不生方第五十五}
\chapter{治小儿齿间出血方第五十六}
\chapter{治小儿鼻衄方第五十七}
\chapter{治小儿鼻塞方第五十八}
\chapter{治小儿鼻息肉方第五十九}
\chapter{治小儿喉痹方第六十}
\chapter{治小儿哕方第六十一}
\chapter{治小儿津颐方第六十二}
\chapter{治小儿吐方第六十三}
\chapter{治小儿难乳方第六十四}
\chapter{治小儿风不乳哺方第六十五}
\chapter{治小儿脐不合方第六十六}
\chapter{治小儿脐中汁出方第六十七}
\chapter{治小儿脐赤肿方第六十八}
\chapter{治小儿脐疮方第六十九}
\chapter{治小儿腹痛方第七十}
\chapter{治小儿腹胀方第七十一}
\chapter{治小儿痞病方第七十二}
\chapter{治小儿瘕方第七十三}
\chapter{治小儿米症方第七十四}
\chapter{治小儿土症方第七十五}
\chapter{治小儿腹中有虫方第七十六}
\chapter{治小儿阴肿方第七十七}
\chapter{治小儿阴痛方第七十八}
\chapter{治小儿阴疮方第七十九}
\chapter{治小儿阴伤血出方第八十}
\chapter{治小儿阴囊肿方第八十一}
\chapter{治小儿阴颓方第八十二}
\chapter{治小儿差颓方第八十三}
\chapter{治小儿脱肛方第八十四}
\chapter{治小儿谷道痒方第八十五}
\chapter{治小儿谷道疮方第八十六}
\chapter{治小儿甘湿方第八十七}
\chapter{治小儿寸白方第八十八}
\chapter{治小儿痫病方第八十九}
\chapter{治小儿魃病方第九十}
\chapter{治小儿客忤方第九十一}
\chapter{治小儿夜啼方第九十二}
\chapter{治小儿惊啼方第九十三}
\chapter{治小儿啼方第九十四}
\chapter{治小儿疟病方第九十五}
\chapter{治小儿伤寒方第九十六}
\chapter{治小儿猝死方第九十七}
\chapter{治小儿注病方第九十八}
\chapter{治小儿数岁不行方第九十九}
\chapter{治小儿四、五岁不语方第百}
\chapter{治小儿无辜方第百一}
\chapter{治小儿大腹丁奚方第百二}
\chapter{治小儿霍乱方第百三}
\chapter{治小儿泄利方第百四}
\chapter{治小儿白利方第百五}
\chapter{治小儿赤利方第百六}
\chapter{治小儿赤白滞下方第百七}
\chapter{治小儿蛊利方第百八}
\chapter{治小儿大便不通方第百九}
\chapter{治小儿小便不通方第百十}
\chapter{治小儿大便血方第百十一}
\chapter{治小儿小便血方第百十二}
\chapter{治小儿淋病方第百十三}
\chapter{治小儿遗尿方第百十四}
\chapter{治小儿身黄方第百十五}
\chapter{治小儿身有赤处方第百十六}
\chapter{治小儿腹皮青黑方第百十七}
\chapter{治小儿赤疵方第百十八}
\chapter{治小儿疠疡方第百十九}
\chapter{治小儿疣目方第百二十}
\chapter{治小儿身上KT方第百二十一}
\chapter{治小儿身热方第百二十二}
\chapter{治小儿盗汗方第百二十三}
\chapter{治小儿隐疹方第百二十四}
\chapter{治小儿丹疮方第百二十五}
\chapter{治小儿赤游肿方第百二十六}
\chapter{治小儿身体肿方第百二十七}
\chapter{治小儿恶核肿方第百二十八}
\chapter{治小儿瘰方第百二十九}
\chapter{治小儿诸方第百三十}
\chapter{治小儿瘿方第百三十一}
\chapter{治小儿附骨疽方第百三十二}
\chapter{治小儿疽方第百三十三}
\chapter{治小儿代指方第百三十四}
\chapter{治小儿疥疮方第百三十五}
\chapter{治小儿癣疮方第百三十六}
\chapter{治小儿浸淫疮第百三十七}
\chapter{治小儿疮方第百三十八}
\chapter{治小儿王灼疮方第百三十九}
\chapter{治小儿月蚀疮方第百四十}
\chapter{治小儿冻疮方第百四十一}
\chapter{治小儿漆疮方第百四十二}
\chapter{治小儿蠼尿疮方第百四十三}
\chapter{治小儿恶疮久不瘥方第百四十四}
\chapter{治小儿金疮方第百四十五}
\chapter{治小儿汤火灼疮方第百四十六}
\chapter{治小儿竹木刺方第百四十七}
\chapter{治小儿落床方第百四十八}
\chapter{治小儿食不知饱方第百四十九}
\chapter{治小儿吐食方第百五十}
\chapter{治小儿吐血方第百五十一}
\chapter{治小儿咳嗽方第百五十二}
\chapter{治小儿食鱼骨哽方第百五十三}
\chapter{治小儿食肉骨哽方第百五十四}
\chapter{治小儿食草芥哽方第百五十五}
\chapter{治小儿饮李、梅辈哽方第百五十六}
\chapter{治小儿食发绕咽方第百五十七}
\chapter{治小儿误吞钱方第百五十八}
\chapter{治小儿误吞针方第百五十九}
\chapter{治小儿误吞钓方第百六十}
\chapter{治小儿误吞方第百六十一}
\chapter{治小儿误吞方第百六十二}
\chapter{治小儿误吞竹木方第百六十三}

\part{26}
\chapter{延年方第一}
\chapter{美色方第二}
\chapter{芳气方第三}
\chapter{益智方第四}
\chapter{相爱方第五}
\chapter{求富方第六}
\chapter{断谷方第七}
\chapter{去三尸方第八}
\chapter{避寒热方第九}
\chapter{避西雨湿方第十}
\chapter{避水火方第十一}
\chapter{避兵刃方第十二}
\chapter{避邪魅方第十三}
\chapter{避虎野狼方第十四}
\chapter{辟虫蛇第十五}

\part{27}
\chapter{大体第一}
\chapter{谷神第二}
\chapter{养形第三}
\chapter{用气第四}
\chapter{导引第五}
\chapter{行止第六}
\chapter{卧起第七}
\chapter{言语第八}
\chapter{服用第九}
\chapter{居处第十}
\chapter{杂禁第十一}

\part{卷第二十八}

\chapter{至理第一}

《玉房秘诀》云:冲和子曰:夫一阴一阳谓之道,媾精化生之为用,其理远乎?故帝轩之问素女,彭铿之酬殷王,良有旨哉!黄帝问素女曰∶吾气衰而不和,心内不乐,身常恐危,将如之何?素女曰∶凡人之所以衰微者,皆伤于阴阳交接之道尔。夫女之胜男,犹水之灭火,知行之如釜鼎,能和五味以成羹,能知阴阳之道者成五乐,不知之者,身命将夭,何得欢乐,可不慎哉。

素女云∶有采女者,妙得道术。王使采女问彭祖延年益寿之法。彭祖曰∶爱精养神,服食众药,可得长生。然不知交接之道,虽服药无益也。男女相成,犹天地相生也。天地得交会之道,故无终竟之限。人失交接之道,故有夭折之渐。能避渐伤之事,而得阴阳之术,则不死之道也。采女再拜曰∶愿闻要教。彭祖曰∶道甚易知,人不能信而行之耳。今君王御万机,治天下,必不能备为众道也。幸多后宫,宜知交接之法。法之要者,在于多御少女而莫数泻精,使人身轻,百病消除也。

汉附马都尉巫子都年百三十八,字孝武巡将见子都于渭水之上,头上有异气,高丈余许。帝怪而问之。东方朔相之对曰∶此君有气通理天中,施行阴阳之术。上屏左右问子都。

子都曰∶阴阳之事,公中之秘。臣子所不宜言。又能行之者少,是以不敢告。臣受之陵阳子,明年六十五矣。行此术来七十二年,诸求生者,当求所生。贪女之容色,极力强施,百脉皆伤,百病并发也。

《玉房指要》云∶彭祖曰∶黄帝御千二百女而登仙,俗人以一女而伐命。知与不知,岂不远耶?知其道者,御女苦不多耳。不必皆须有容色妍丽也,但欲得年少未生乳而多肌肉者耳。但能得七八人,便大有益也。

素女曰∶御敌家,当视敌如瓦石,自视如金玉。若其精动,当疾去其乡。御女当如朽索御奔马,如临深坑,下有刃,恐堕其中。若能爱精,命亦不穷也。

黄帝问素女曰∶今欲长不交接,为之奈何?素女曰∶不可。天地有开合,阴阳有施化。

人法阴阳,随四时,今欲不交接,神气不宣布,阴阳闭隔,何以自补?练气数行,去故纳新,以自助也。玉茎不动,则辟死其舍。所以常行以当导引也。能动而不施者,所谓还精,还精补益,生道乃者。

《素女经》云∶黄帝曰∶夫阴阳交接节度,为之奈何?素女曰∶交接之道,故有形状。

男致不衰,女除百病。心意娱乐,气力强。然不知行者,渐以衰损。欲知其道,在于定气、安心、和志。三气皆至,神明统归。不寒不热,不饥不饱。亭身定体,性必舒迟。浅纳徐动,出入欲稀。女快意,男盛不衰,以此为节。

《玄女经》云∶黄帝问玄女曰∶吾受素女阴阳之术,自有法矣。愿复命之,以恚其道。

玄女曰∶天地之间,动须阴阳。阳得阴而化,阴得阳而通。一阴一阳,相须而行。故男感坚强,女动辟张。二气交精,流液相通。男有八节,女有九宫。用之失度,男发痈疽,女害月经,百病生长,寿命销亡。能知其道,乐而且强,寿即增延,色如华英。《抱朴子》云∶凡服药千称,三牲之养,而不知房中之术,亦无所益也。是以古人恐人之轻恣情性,故美为之说,亦不可尽信也。玄素喻于水火,水火杀人又生人,于在能用与不能耳。大都得其要法,御女多多益善。若不晓其道用一两者,适足以速死耳。

又云∶人复不可都阴阳不交,则生痈瘀之疾。故幽闲怨旷,多病而不寿。任情恣意,复伐年命。唯有得节宣之和,可以不损。洞玄子曰∶夫天生万物,唯人最贵。人之所上,莫过房欲。法天象地,规阴矩,悟其理者,则养性延龄;慢其真者,则伤神夭寿。至如玄女之法,传之万古都具,陈其梗概,仍未尽其机微,余每览其条,思补其缺,综习旧仪,篡此新经,虽不穷其纯粹,抑得其糟粕。其坐卧舒卷之形,偃伏开张之势,侧背前却之法,出入深浅之规,并会二仪之理,俱合五行之数。其导者则得保寿命,其达者则陷于危亡。既有利于凡人,岂无传于万叶。

《千金方》云∶男不可无女,女不可无男。若孤独而思交接,损人寿,生百病。又鬼魅因之,共交精,损一当百。

又云∶人年三十(四十,或本)以下,多有放恣,四十以上,即复觉气力一时衰退。衰退既至,众病锋起,反久而不治,遂而不救。故年至四十,须识房中之术者,其道极近而人莫、知术。其法一夜御十女,不泄而已。此房中之术毕矣。兼之药饵,四时勿绝,则气力百倍,而智慧日新,然此方之术也。

养阴第三

《玉房秘诀》云∶冲阳子曰∶非徒阳可养也,阴亦宜然。西王母是养阴得道之者也。一与男交而男立损病。女颜色光泽,不着脂粉,常食乳酪而弹五弦。所以和心系意,使使无他欲。

又云∶王母无夫,好与童男交,是以不可为世教。何必王母然哉。

又云∶与男交,当安心定意,有如男子之未成。须气至乃小收,情志与之相应,皆勿振摇踊跃,使阴精先竭也。阴精先竭,其处空虚,以受风寒之疾。或闻男子与他人交接,嫉妒烦闷,阴气鼓动,坐起恚,精液独出,憔悴暴老,皆此也,将宜抑慎之。

又云∶若知养阴之道,使二气和合,则化为男子。若不为子,转成津液,流入百脉,以阳养阴,百病消除,颜色悦泽,肌好,延年不老,常如少童。审得其道,常与男子交,可以绝谷,九日而不知饥也。有病与鬼交者,尚可不食而消瘦,况与人交乎?
和志第四

《洞玄子》云∶夫天左转而地右回,春夏谢而秋冬袭,男唱而女和,上为而下从,此物事之常理也。若男摇而女(不)应,女动而男不从,非直损于男子,亦乃害于女人,此由阴阳行很,上下了戾矣。以此合会,彼此不利,故必须男左转而女右回,男下冲女上接,以此合会,乃谓天平地成矣。凡深浅迟速、捌捩东西,理非一途,盖有万绪。若缓冲似鲫鱼之弄钩,若急蹙如群鸟之遇风,进退牵引,上下随迎,左右往还,出入疏蜜,此乃相持成务,临事制宜,不可胶柱宫商,以取当时之用。

又云∶凡初交会之时,男坐女左,女坐男右,乃男箕坐,抱女于怀中,于是勒纤腰,抚玉体,申婉,叙绸缪。同心同意,乍抱乍勒,二形相搏,两口相。男含女下唇,女含男上唇,一时相KT茹其津液,或缓啮其舌,或微其唇,或邀遣抱头,或逼命拈耳,抚上拍下,东KT西,千娇既申,百虑竟解。乃令女左手抱男玉茎,男以右手抚女玉门,于是男感阳气则玉茎振动。其状也∶哨然上耸,若孤峰之临迥汉。女感阳气,则丹穴津流,其状也∶涓然下逝,若幽泉之吐深谷。此乃阴阳感激使然,非人力之所致也。热至于此,乃可交接。或男不感振,女无淫津,皆缘病发于内,疾形于外矣。

《玉房秘诀》云∶黄帝曰∶夫阴阳之道,交接奈何?素女曰∶交接之道,固有形状。男以致气,女以除病。心意娱乐,气力益壮,不知道者,则侵以衰。欲知其道,在安心和志,精神宛归,不寒不暑,不饱不饥,定身正意,性必舒迟。滑纳徐动,出入欲稀,以是为节。

慎无敢违。女即欢喜,男则不衰。

又云∶黄帝曰∶今欲强交接,玉茎不起,面惭意着,汗如珠子,心情贪欲,强助以手,何以强之,愿闻其道。素女曰∶帝之所问,众人所有。凡欲接女,固有经纪,必先和气,玉茎乃起,顺其五常,存感九部。女有五色,审所KT扣,采其溢精,取液丁口,精气还化,填满髓脑,避七损之禁,行八益之道,无逆五常,身乃可保。正气内充,何疾不去。腑脏安宁,光泽润理,每接即起。气力百倍,敌人宾服,何惭之有。

《玉房指要》云∶道人刘京言∶凡御女之道,务欲先徐徐嬉戏,使神和意感,良久乃可交接。弱而纳之,坚强急退,退进之间,欲令疏迟,亦勿高自投掷,颠倒五脏,伤绝路脉,致生百病也。但接而勿施,能一日一夕数十交而不失精者,诸病甚愈,年寿日益。《玄女经》云∶黄帝曰∶交接之时,女或不悦,其质不动,其液不出,玉茎不强,小而不势,何以尔也。

玄女曰∶阴阳者,相感而应耳。故阳不得阴则不喜,阴不得阳则不起。男欲接而女不乐,女欲接而男不欲。二心不和,精气不感,加以猝上暴下,爱乐未施,男欲求女,女欲求男,情意协议,俱有悦心,故女质振感男茎盛,男热营扣俞鼠,精液流溢,玉茎施纵,乍缓乍急,王户开翕,或实作而不劳,强敌自佚,吸精引气,灌溉朱室,今陈九事,其法备悉。伸缩俯仰,前劫屈折,帝审行之,慎莫违失。
临御第五

《洞玄子》云∶凡初交接之时,先坐而后卧,女左男右。卧定后,令女正面仰卧,展足舒臂,男伏其上,跪于股内,即以玉茎坚拖于玉门之口,森森然若偃松之当邃谷洞前。更拖碜勒,鸣口嗍舌,或上观(欢)玉面,下视金沟,抚拍肚乳之间,摩挲琼台之侧,于是男情既或,女意当迷,即以阳锋纵横攻击,或下冲玉理,或上筑金沟,击刺于辟雍之旁,憩息于琼台之右(以上外游,未内交也。)女当淫津,湛于丹穴,即以阳锋投入子宫,快泄其精,津液同流,上灌于神田,下溉于幽谷,使往来KT击,进退揩磨,女必求死求生,乞性乞命。即以帛子干拭之,后乃以玉茎深投丹穴,至于阳台。

然若巨石之壅深溪,乃行九浅一深之法,于是纵柱横桃,旁牵侧拔,乍缓乍急,或深或浅,经二十一息,候气出入,女得快意也。男即疾纵急刺,碜勒高抬,候女动摇,取其缓急,即以阳锋攻其谷实,捉入于子宫,左右研磨,自不烦细细抽拔。女当津液流溢,男即须退,不可死还,必须生返。如死出,大损于男,持宜慎之。

《素女经》云∶黄帝曰∶阴阳贵有法乎?素女曰∶临御女时,先令妇人放手安身,屈两脚,男人其间,衔其口,吮其舌,拊搏其玉茎,击其门户东西两旁,如是食顷徐徐纳入,玉茎肥大者纳寸半,弱小者入一寸,勿摇动之,徐出更入,除百病,勿令四旁泄出。玉茎入玉门自然生热,且急妇人身当自动摇,上与男相得,然后深之。男女百病消灭。浅刺琴弦入三寸半当闭口。刺之一二三四五六七八九回,深之至昆石旁往来,口当妇人口而吸气,行九九之道讫,乃如此。
五常第六

《玉房秘诀》云∶黄帝曰∶何谓五常?素女曰∶玉茎实有五常之道。深居隐处,执节自守,内怀至德,施行无行无已。夫玉茎意欲施与者,仁也;中有空者,义也。端有节者,礼也;意欲即起,不欲即止者,信也。临事低仰者,智也。是故真人因五常而节之,仁虽欲施予,精若不固,义守其空者,明当禁。使无得,多实既禁之道矣。又当施与,故礼为之节矣。

执诚持之,信既着矣。即当知交接之道。故能从五常,身乃寿也。

五欲第八

素女曰∶五欲者,以知其应。一曰意欲得之,则屏息屏气;二曰阴欲得之,则鼻口两张;三曰精欲烦者,振掉而抱男;四曰心欲满者,则汗流湿衣裳;五曰其快欲之,甚者身直目眠。
十动第九

素女曰∶十动之效∶一曰两手抱人者,欲体相薄阴相当也;二曰伸云其两髀者,切磨其上方也;三曰张腹者,欲其浅也;四曰尻动者,快善也;五曰举两脚拘人者,欲其深也;六曰交其两股者,内痒淫淫也;七曰侧摇者,欲深切左右也;八曰举身迫人,淫乐甚也。九曰身布纵者,肢体快也;十曰阴液滑者,精已泄也。见其效,以知女之快也。
四至第十

《玄女经》云∶黄帝曰∶意贪交接而茎不起,可以强用不?玄女曰∶不可矣。夫欲交接之道,男洼四至,乃可致女九气。黄帝曰∶何谓四至?玄女曰∶玉茎不怒,和气不至;怒而不大,肌气不至;大而不坚,骨气不至;坚而不热,神气不至。故怒者精之明,大者精之关,坚者精之户,热者精之门。四气至而节之以道,开机不妄开,精不泄矣。
九气第十一

《玄女经》云∶黄帝曰∶善矣!女之九气,何以知之?玄女曰∶伺其九气以知之。女人大息而咽唾者,肺气来至;鸣而吮人者,心气来至;抱而持人者,脾气来至;阴门滑泽者,肾气来至;殷勤咋人者,骨气来至;足拘人者,筋气来至。抚弄玉茎者,血气来至;持弄男乳者,肉气来至。久与交接,弄其实以感其意,九气皆至。有不至者,则容伤。故不至可行其数以治。(今检∶诸本无一气。)
九法第十二

《玄(素)女经》云∶黄帝曰∶所说九法,未闻其法,愿为陈之,以开其意,藏之石室,行其法式。

玄女曰∶九法,第一曰龙翻。令女正偃卧向上,男伏其上,股隐于床,女举其阴以受玉茎,刺其谷实,又攻其上,疏缓动摇,八浅二深,死往生返,热壮且强,女则烦悦,其乐如倡,致自闭固,百病消亡。

第二曰虎步。令女俯俯,尻仰首伏,男跪其后,抱其腹,乃纳玉茎,刺其中极,务令深密。进退相薄,行五八之数。其度自得,女阴闭张,精液外溢,毕而休息,百病不发,男益盛。

第三曰猿搏。令女偃卧,男担其股,膝还过胸,尻背俱举,乃纳玉茎,刺其臭鼠,女烦动摇,精液如雨,男深按之,极壮且怒,女快乃止,百病自愈。

第四曰蝉附。令女伏卧,直伸其躯,男伏其后,深纳玉茎,小举其尻,以扣其赤珠,行六九之数,女烦精流,阴里动急,外为开舒,女快乃止,七伤自除。

第五曰龟腾。令女正卧,屈其两漆,男乃推之,其足至乳,深纳玉茎,刺婴女,深浅以度,令中其实,女则感悦,躯自摇举,精液流溢,乃深极纳,女快乃止。行之勿失精,力百倍。

第六曰凤翔。令女正卧,自举其脚,男跪其股间,两手授席,深纳玉茎,刺其昆石,坚热内牵。令女动作,行三八之数,尻急相搏,女阴开舒,自吐精液,女快乃止,百病消。

第七曰兔吮毫。男正反卧,直伸脚,女跨其上,膝在外边,女背头向足,据席俯头,乃纳玉茎,刺其琴弦,女快,精液流出如泉,欣喜和乐,动其神形,女快乃止,百病不生。

第八曰鱼接鳞。男正偃卧,女跨其上,两股向前,安徐纳之,微入便止,才授勿深,如儿含乳,使女独摇,务令迟久,女快男退,治诸结聚。

第九曰鹤交颈。男正箕坐,女跨其股,手抱男颈,纳玉茎,刺麦齿,务中其实。男抱女尻,助其摇举,女自感快,精液流溢,女快乃止,七伤自愈。
三十法第十三

《洞玄子》云∶考核交接之势,更不出于三十法。其间有屈伸俯仰,出入浅深,大大是同,小小有异,可谓哲囊都尽,采摭无遗,余遂像其势而录其名,假其形而建其号,知音君子,穷其志之妙矣。

一、叙绸缪。

二、申缱绻(不离散也)。

三、曝鳃鱼。

四、骐麟角。

(以上四势之外,游戏势皆是一等也。)五、蚕缠绵。

(女仰卧,两手向上抱男顿,以两脚交于男背上,男以两手抱女项,跪女股间,即纳玉茎。)六、龙宛转。

(女仰卧,屈两脚,男跪女股内,以左手推女两脚向前,令过于乳,右手把玉茎纳玉门中。)七、鱼比目。

(男女俱卧,女以一脚置男上,面相向,口嗍舌,男展两脚,以手担女上脚,进玉茎。)八、燕同心。

(令女仰卧,展其足,男骑女,伏肚上,以两手抱女颈。女两手抱男腰,以玉茎纳于丹穴中。)九、翡翠交。

(令女仰卧拳足,男胡跪,开着脚,坐女股中,以两手抱女腰,进玉茎于琴弦中。)十、鸳鸯合。

(令女侧卧,举两脚安男股上,男于女背后骑女下脚之上,竖一膝置女上股,纳玉茎。)十一、翻空蝶。

(男仰卧,展两足,女坐男上正面,两脚据床,乃以手助为力,进阳锋于玉门之中。)十二、背飞凫。

(男仰卧,展两足,女背面坐于男上,女足据床,低头抱男玉茎,纳于丹穴中。)十三、偃盖松。

(令女交脚向上,男以两手抱女腰,女两手抱男项,纳玉茎于玉门中。)十四、临坛竹。

(男女俱相向立,呜口相抱于丹穴,以阳锋深投于丹穴,没至阳台中。)十五、鸾双舞。

(男女一仰一覆,仰者举脚,覆者骑上,两阴相向,男箕坐,看玉物攻击上下。)十六、凤将雏。

(妇人肥大,用一小男共交接,大俊也。)十七、海鸥翔。

(男临床边,擎女脚以令举,男以玉茎入于子宫之中。)十八、野马跃。

(令女仰卧,男擎女两脚,登左右肩上,深纳玉茎于玉门之中。)十九、骥骋足。

(令女仰卧,男蹲,左手捧女项,右手擎女脚,即以玉茎纳入于子宫中。)二十、马摇蹄。

(令女仰卧,男擎女一脚置于肩上,一脚自攀之,深纳玉茎,入于丹穴中,大兴哉。)二十一、白虎腾。

(令人伏面跪膝,男跪女后,两手抱女腰,纳玉茎于子宫中。)二十二、玄蝉附。

(令女伏卧而展足,男居股内,屈其足,两手抱女项,从后纳玉茎入玉门中。)二十三、山羊对树。

(男箕坐,令女背面坐男上,女自低头视纳玉茎,男急抱女腰碜勒也。)二十四、鸡临场。

(男胡蹲床上坐,令一小女当抱玉茎,纳女玉门。一女于后牵女裙衿,令其足快,大兴哉。)二十五、丹穴凤游。

(令女仰卧,以两手自举其脚,男跪女后,以两手据床,以纳玉茎于丹穴,甚俊。)二十六、玄溟鹏翥。

(令女仰卧,男取女两脚置左右膊上,以手向下抱女腰,以纳玉茎。)二十七、吟猿抱树。

(男箕坐,女骑男髀上,以两手抱男,男以一手扶女尻,纳玉茎,一手据床。)二十八、猫鼠同穴。

(男仰卧,以展足,女伏男上,深纳玉茎;又男伏女背上,以将玉茎,攻击于玉门中。)二十九、三春驴。

(女两手两脚俱据床,男立其后,以两手抱女腰,即纳玉茎于玉门中。甚大俊也。)三十、秋猫。

(男女相背,以两手两脚俱据床,两尻相拄,男即低头,以一手推玉物,纳玉门之中。)
九状第十四

洞玄子云∶凡玉茎,或左击右击,若猛将之破阵,(其状一也;)或缘上蓦下,若野马之跳涧,(其状二也;)或出或没,若波之群鸥,(其状三也;)或深筑浅桃,若鸦臼之雀喙,(其状四也;)或深冲浅刺,若大石之投海,(其状五也;)或缓耸迟推,若冻蛇之入窟,(其状六也;)或疾纵急刺,若惊鼠之透穴,(其状七也;)或抬头扬足,若苍鹰之揄校兔,(其状八也;)或抬上顿下,若大帆之遇狂风,(其状九也。)
六势第十五

洞玄子云∶凡交接,或下捺玉茎往来据其玉理,其热若割蚌而取明珠,(其状一也;)或下抬玉理,上冲金沟,其势若割石而寻美玉,(其势二也;)或以阳锋冲筑琼台,其势若铁杵之投药臼,(其势三也;)或以玉茎出入,攻击左右辟雍,其势若五锤之锻铁,(其势四也;)或以阳锋来往,磨耕神田,幽谷之间,其势若农夫之垦秋壤,(其势五也;)或以玄圃、天庭两相磨搏,其势若两崩岩之相钦,(其势六也。)
八益第十六

《玉房秘诀》云∶素女曰∶阴阳有七损八益。一益曰固精。令女侧卧张股,男侧卧其中,行二九数,数猝止,令男固精。又治女子漏血,日再行,十五日愈。

二益曰安气。令女正卧高枕,伸张两髀,男跪其股间刺之,行三九数,数毕止,令人气和。又治女门寒,日三行,二十日愈。

三益曰利脏。令女人侧卧,屈其两股,男横卧却刺之,行四九数,数毕止,令人气和。

又治女门寒,日四行,二十日愈。

四益曰强骨。令女人侧卧,屈左膝,伸其右髀,男伏刺之,行五九数,数毕止,令人关节调和。又治女门闭血,日五行,十日愈。

五益曰调脉。令女侧卧,屈其右膝,申其左髀,男据地刺之,行六九数,毕止,令人脉通利。又治女门辟,日六行,二十日愈。

六益曰蓄血。男正偃卧,令女戴尻,跪其上,极纳之,令女行七九数,数毕止。令人力强。又治女子月经不利,日七行,十日愈。

七益曰益液。令女人正伏举后,男上往,行八九数,数毕止。令人骨填。

八益曰道体。令女正卧,屈其髀,足迫尻下,男以髀胁刺之,以行九九数。数毕止,令人骨实。又治女阴臭,日九行,九日愈。
七损第十七

《玉房秘诀》云∶素女曰∶一损谓绝气;绝气者,心意不欲而强用之,则汗泄气少,令心热目冥冥。治之法,令女正卧,男担其两股深按之,令女自摇,女精出止,男勿得快。日九行,十日愈。

二损谓溢精。溢精者,心意贪爱、阴阳末和而用之,精中道溢。又醉而交接,喘息气乱,则伤肺,令人咳逆上气,消渴喜怒,或悲惨惨,口干身热而难久立。治之法,令女人正卧,屈其两膝侠男,男浅刺纳玉茎寸半,令女子自摇,女精出止,男勿得快。日九行,十日愈。

三损谓夺脉。夺脉者,阴不坚而强用之,中道强泻,精气竭。及饱食讫交接,伤脾,令人食不化,阴痿无精。治之法,令女人正卧,以脚钩男子尻,男则据席纳之,令女自摇,女精出止,男勿快。日九行,十日愈。

四损谓气泄。气泄者,劳倦汗出,未干而交接,令人腹热,唇焦。治之法,令男子正伸卧,女跨其上向足,女据席,浅纳茎,令女自摇,精出止,男子勿快。日九行,十日愈。

五损谓机关厥伤。机关厥伤者,适新大小便,身体未定而强用之,则伤肝。及猝暴交会,迟疾不理不理。劳疲筋骨,令人目茫茫,痈疽并发,众脉槁绝,久生偏枯,阴痿不起。治之法,令男子正卧,女跨其股,踞前向,徐徐按纳之,勿令女人自摇。女精出,男勿快。日九行,十日愈。

六损谓百闭。百闭者,淫佚于女,自用不节,数交失度,竭其精气,用力强泻,精尽不出,百病并生,消渴,目冥冥。治之法,令男正卧,女跨其上,前伏据席,令女纳玉茎自摇,精出止,男勿快。日九行,十日愈。

七损谓血竭。血竭者,力作疾行,劳因汗出,因以交合,俱已之时,偃卧推深没本暴急,剧病因发,连施不止,血枯气竭,令人皮虚肤急,茎痛囊湿,精变为血。治之法,令女正卧,高枕其尻,伸张两股,男跪其间深刺,令女自摇,精出止,男勿快。日九行之,十日愈。
还精第十八

《玉房秘诀》云∶采女问曰∶交接以泻精为乐,今闭而不泻,将何以为乐乎?彭祖答曰∶夫精出则身体怠倦,耳苦嘈嘈,目苦欲眠,喉咽干枯,骨节懈堕,虽复暂快,终于不乐也。

若乃动不泻,气力有余,身体能便,耳目聪明,虽自抑静,意爱更重,恒若不足,何以不乐耶。

又云∶黄帝曰∶愿闻动而不施,其效何如?素女曰∶一动不泻,则气力强;再动不泻,耳目聪明;三动不泻,众病消亡;四动不泻,五神咸安;五动不泻,血脉充长;六动不泻,腰背坚强;七动不泻,尻股益力;八动不泻,身体生光;九动不泻,寿命未失;十动不泻,通于神明。

《玉房指要》云∶能一日数十交而不失精者,诸病皆愈,年寿日益,又数数易女,则益多,一夕易十人以上尤佳。

又云∶《仙经》曰∶还精补脑之道,交接精大动欲出者,急以左手中央两指却抑阴囊后大孔前,壮事抑之,长吐气,并喙齿数十过,勿闭气也。便施其精,精亦不得出,但从玉茎复还上,入脑中也。此法仙人吕相授,皆饮血为盟,不得妄传,身受其殃。

又云∶若欲御女取益而精大动者,疾仰头张目,左右上下视,缩下部,闭气,精自止。

勿妄传。人能一月再施、一岁二十四施,皆得寿一二百岁,有颜色无病。

《千金方》云∶昔贞观初,有一野老可七十余,诣余曰∶近数十日来,阳道益盛,思与家姥昼夜春事皆成,未知垂老有此,为益为恶耶?余答之曰∶是大不祥也。子独不闻膏火乎?夫膏火之将竭也,必先暗而后明,明止即灭也,今足下年迫桑榆,久当用精,兹忽春情猛发,岂非反常耶。窃为足下忧之。子(其)勉欤,后四旬发病而猝。此其不慎之效也。所以善摄生者,凡觉阳道盛,必谨而抑之,不可纵心竭意以自贼也。若一度制得不泄,则是一度大增油。若不能制得,纵情施泻,则是膏火将灭,更去其油,不可不深以自防也。
施泻第十九

《玉房秘诀》云∶黄帝问素女曰∶道要不欲失精,宜爱液者也。即欲求子,何可得泻?素女曰∶人有强弱,年有老壮,各随其气力,不欲强快,强快即有所损。故男年十五,盛者可一日再泻,瘦者可一日一泻;年二十岁者,日再施,羸者可一日一施;年三十,盛者可一日一施,劣者二日一施;三十,盛者三日一施,虚者四日一施;五十,盛者可五日一施,虚者可十日一施;六十,盛者十日一施,虚者二十日一施;七十,盛者可三十日一施,虚者不泻。

又云∶年二十,常二日一施;三十,三日一施;四十,四日一施;五十,五日一施;年过六十以去,勿复施泻。

《养生要集》云∶道人刘京云∶春天三月壹施精,夏及秋当一月再施精,冬当闭精勿施。

夫天道,冬藏其阳,人能法之,故得长生。冬一施,当春百。

《千金方》云∶素女法∶人年二十者四日一泄;年三十者八日一泄;年四十者十六日一泄;年五十者二十一日一泄;年六十者即毕,闭精勿复更泄也,若体力犹壮者,一月一泄。

凡人气力,自相有强盛过人者,亦不可抑,忍久而不泄,致痈疽。若年过六十而有数旬不得交接,意中平平者,可闭精勿泄也。

洞玄子云∶凡欲泄精之时,必须候女快,与精一时同泄。男须浅拔,游于琴弦、麦齿之间。阳锋深浅,如孩儿含乳,即闭目内想,舌拄下,脊引头,张鼻歙肩,闭口吸气,精便自上。节限多少,莫不由人。十分之中,只得泄二三矣。
治伤第二十

《玉房秘诀》云∶冲和子曰∶夫极情逞欲,必有损伤之病,斯乃交验之着明者也。既以斯病,亦以斯愈,解酲以酒,足为喻也。

又云∶采女曰∶男之盛衰,何以为候?彭祖曰∶伤盛得气则玉茎当热,阳精浓而凝也。

其衰有五∶一曰精泄而出,则气伤也;二曰精清而少,此肉伤也;三曰精变而臭,此筋伤也;四曰精出不射,此骨伤也;五曰阴衰不起,此本伤也。凡此众伤,皆由不徐交接,而猝暴施泻之所致也。治之法,但御而不施,不过百日,气力必致百倍。

又云∶交接开目,相见形体,夜燃火视图书,即病目瞑清盲。治之法,夜闭目而交,愈。

交接取敌人着腹上者,从下举腰应之,则苦腰痛,少腹里急,两脚物背曲。治之法,覆体正身,徐戏,愈。

交接侧斯,旁向敌手,举敌尻,病胁痛。治之法,正卧徐戏,愈。交接低头延颈则病头重项强。治之,法,以头置敌人额上,不低之愈。

交接侵饱,谓夜半饭气未消而以戏,即病疮,胸气满,胁下如拔,胸中若裂,不欲饮食,心下结塞,时呕吐青黄,胃气实,结脉,若衄吐血,若胁下坚痛,面生恶疮。治之法,过夜半向晨交,愈。

交接侵酒,谓醉而交接,戏用力深极,即病黄胆,黑瘅,胁下痛,有气接,接动手下髀里,若囊盛水彻脐上,肩膊,甚者胸背痛,咳唾血,上气。治之法,勿复乘酒热向晨交接,戏徐缓体,愈。

当溺不溺以交接,则病淋,少腹气急,小便难,茎中疼痛,常欲手撮持。须臾,乃欲出。

治之法,先小便,还卧自定,半饭饮久顷,乃徐交接,愈。

当大便不大便而交接,即病痔。大便难,至清,移日月,下脓血,孔旁生疮如蜂穴状,圊上倾倚,便不时出,疼痛,痈肿,卧不得息,以道。治之法,用鸡鸣际,先起更衣,还卧自定,徐相戏弄,完体缓意,令滑泽而退。病愈神良。并愈妇病。

交接过度,汗如珠子,屈伸转侧,风生被里,精虚气竭,风邪入体,则病缓弱为跛蹇,手不上头,治之法,爱养精神,服地黄煎。

又云∶巫子都曰∶令人目明之道,临动欲施时,仰头闭气,大呼,目,左右视,缩腹还精气,令入百脉中也。

令耳不聋之法,临欲施泻,大咽气,合齿闭气,令耳中萧萧声,复缩腹合气流布至坚,至老不聋。

调五脏、消食、疗百病之道,临施张腹,以意纳气,缩后精散而还归百脉也,九浅一深,至琴弦,麦齿之间,正气还,邪气散去。令人腰背不痛之法,当壁伸腰,勿甚低仰,平腰背,所却行常令流欲,补虚、养体、治病,欲泻勿泻,还流流中,流中通热。

又云∶夫阴阳之道,精液为珍,即能爱之,性令(命)可保。凡施泻之后,当所女气以自补,复建九者,内息九也。厌一者,以左手煞阴下,还精复液也。取气者,九浅一深也。

以口当敌口气呼,以口吸,微引二无咽之,致气以意下也。至腹所以助阴为阴力,如此三反,复浅之,九浅一深。九九八十一,阳数满矣。玉茎竖出之,弱纳之,此为弱入强出。阴阳之和,在于琴弦,麦齿之间,阳困昆石之下,阴困麦齿之间,浅则得气,远则气散。一至谷实,伤肝,见风泪出,溺有余沥。至臭鼠伤肠肺,咳逆,腰背痛。至昆石,伤脾,腹满腥臭,时时下利,两股疼,百病生于昆石,故伤交接合时不欲及远也。

黄帝曰∶犯此禁,疗方奈何?子都曰∶当以女复疗之也。其法令女正偃卧,令两股相去九寸,男往从之,先饮王浆,久久乃KT鸿泉,乃徐纳玉茎,以手节之,则裁致琴弦,麦齿之间。

敌人淫跃心烦,常自坚持,勿施泻之。度三十息令坚强,乃徐纳之,令至昆石。当极洪大,洪大则出之,正息劣弱,复纳之,常令弱入强出,不过十日,坚如铁,热如火,百战不殆。
求子第二十一

《千金方》云∶夫婚姻生育者,人伦之本,王化之基。圣人设教备论,厥旨后生,莫能精晓,临事之日,昏尔若愚,今具述求子之法,以贻后嗣,同志之士,或可览焉。

又云∶夫欲求子者,先知夫妻本命、五行相生及与德合并本命不在子伏死墓中生者,则求子必得。若其本命五行相克及与形杀冲破并在子休死墓中生者,则求子不可得。慎无措意,纵后得者,于后方欲示人。若其相生,并遇福德者,仍者须依法如方,避乎禁忌,则所诞儿子,尽善尽美,难以具陈矣。

又云∶夫欲令儿子吉良者,交会之日当避景丁日及弦、望、朔、晦、大风、大雨、大雾、大寒、大暑、雷、电、霹雳、天地昏冥、日月无光、虹霓、地动、日月薄蚀,此时受胎,非只百倍损于父母,生子或喑、、聋、KT顽愚、癫狂、挛跋、盲眇、多病,短寿、不孝、不仁,又避火光星辰之下,神庙佛寺之中,井灶、清厕之侧,坟墓尸KT之旁,皆悉不可。

又云∶夫交会如法,则消福德,大圣善人降托胎中,仍令父母性行调顺,所作合应,家道日隆,祥瑞竟集;若不如法,则有薄福,愚痴恶人来托胎中,令父母性行凶险,所作不成,家道日否,咎征屡至,虽生成长,国灭身亡。夫祸福之征,有如影响,此乃必然之理,何不思之。

又云∶以夜半后生气时泻精,有子皆男,必寿而贤明高爵也。(今按∶《大清经》云∶从夜半日中为生气,从日中至夜半为死气。)又云∶王相日、贵宿日尤吉。(其曰有本书。)《产经》云∶黄帝曰∶人之始生,本在于胎合阴阳也。夫合阴阳之时,必避九殃。九殃者,日中之子,生则呕逆,一也;夜半之子,天地闭塞,不喑则聋盲,二也;日蚀之子,体戚毁伤,三也;雷电之子,天怒头威,必易服狂,四也;月蚀之子,与母俱凶,五也;虹霓之子,若作不祥,六也;冬夏日至之子,生害父母,七也;弦望之子,必为乱兵风盲,八也;醉饱之子,必为病癫、疽、痔、有疮,九也。

又云∶有五观子生不祥∶月水未清,一观也;父母有疮,二观也;丧服未除有子,三观也;温病未愈有子,身亲丧,四观也;妊身而忧恐重复惊惶,五观也。

《玉房秘诀》云∶合阴阳有七忌∶第一之忌,晦朔弦望,以合阴阳,损气,以是生子,子必刑残,宜深慎之。

第二之忌,雷风天地感动,以合阴阳,血脉涌,以是生子,子必痈肿。

第三之忌,新饮酒饱食,谷气未行,以合阴阳,腹中膨亨,小便白浊,以是生子,子必癫狂。

第四之忌,新小便,精气竭以合阴阳,经脉得涩,以是生子,必妖孽。

第五之忌,劳倦重担,志气未安,以合阴阳,筋腰苦痛,以是生子,必夭残。

第六忌,新沐浴,发肤未燥,以合阴阳,令人短气,以是生子,子必不全。

第七忌,兵坚盛怒,茎脉痛,当令不合,内伤有病。如此为七伤。

又云∶人生喑聋者,是腊目暮之子。腊暮百鬼聚会,终夜不息,君子斋戒,小人私合阴阳,其子必喑聋。

人生伤死者,名曰火子。燃烛未灭而合阴阳,有子必伤,死市人。

人生癫狂,是雷电之子,四月五月大雨霹雳,君子斋戒,小人私合阴阳,有子必癫狂。

人生为虎野狼所食者,重服之子,孝子戴麻,不食肉,君子羸顿,小人私合阴阳,有子必为虎野狼所合。

人生溺死者,父母过藏胞于铜器中,覆以铜器,埋于阴垣下,入地七尺,名曰童子裹,溺死水中。

又云∶大风之子多病,雷电之子狂癫,大醉之子必痴狂,劳倦之子必夭伤,月经之子兵亡,黄昏之子多变,人定之子不暗则聋,日入之子口舌不祥,日中之子癫病,晡时之子自毁伤。

又云∶素女曰∶求子法自有常体,清心远虑,安定其襟袍,垂虚斋戒,以妇人月经后三日,夜半之后,鸡鸣之前,嬉戏令女盛(咸,或本)动,乃往从之,适其道理,同其快乐,却身施泻,下精,欲得去玉门入半寸,不尔过子宫,千翼。勿过远至麦齿,远则过子门,不入子户。若依道术,有有子贤良而老寿也。

又云∶彭祖曰∶求子之法,当蓄养精气,勿数施舍,以妇人月事断绝,洁净三五日而交有子,则男听明才智,老寿高贵,生女清贤,配贵人。

又云∶常向晨之际,以御阴阳,利身便躯,精光益张,生子富长命。

又云∶素女曰∶夫人合阴阳,当避禁忌,常乘生气,无不老寿。若夫妇俱老,虽生化有子,皆不寿也。

又云∶男女满百岁,生子亦不寿,八十男可御十五、十八女,则生子不犯禁忌,皆寿老。

女子五十得少夫,亦有子。

又云∶妇人怀子未满三月以成子,取男子冠缨烧之,以取灰,以酒尽服之,生子富贵明达,秘之秘之。

又云∶妇人无子,令妇人左手持小豆二七枚,右手扶男子阴头纳女阴中,左手纳豆着口中,女自男阴同入,闻男阴精下,女仍当咽豆,有效,万全不失一也。

洞玄子云∶几欲求子,候女之月经断后,则交接之。一日三日为男,四日五日为女,五日以后,徒损精力,终无益也。交接泄精之时,候女快来,须与一时同泄,泄必须尽。先令女正面仰卧,端心一意,闭目内想,受精气。故老子曰∶夜半得子为上寿,夜半前得子为中寿,夜半后得子下寿。

又云∶凡女子怀孕之后,须行善事,勿视恶色,勿听恶语,省淫欲,勿咒咀,勿骂詈,勿惊恐,勿劳倦,勿妄语,勿忧愁,勿食生冷醋滑热食,勿乘车马,勿登,勿临深,勿下,勿急行,勿服饵,勿针灸。皆须端心正念,常听经书,遂令男女如是,聪明智慧,忠真贞良,所谓胎教者也。
好女第二十二

《玉房秘诀》云∶冲和子曰∶婉KT淑慎,妇人之性美矣。夫能浓纤得宜,修短合度,非徒取悦心目,抑乃尤益寿延年。

又云∶欲御女,须取少年未生乳、多肌肉、丝发小眼、眼精白黑分明者。面体濡滑、言语音声和调而下者,其四肢百节之骨皆欲令没,肉多而骨不大者,其阴及腋下不欲有毛,有毛当令细滑也。

《大清经》云∶黄帝曰∶入相女人云∶何谓其事?素女曰∶入相女人,天性婉顺,气声濡行,丝发黑,弱肌细骨,不长不短,不大不少,凿孔欲高,阴上无毛,多精液者,年五五以上,三十以还,未在产者。交接之时,精液流漾,身体动摇,不能自定,汗流四逋(通或本),随人举止,男子者虽不行法,得此人由不为损。

又云∶细骨弱肌,肉淖KT泽,清白薄肤,指节细没,耳目准高鲜白,不短不辽,浓髀,凿孔欲高而周密,体满,其上无毛,身滑如绵,阴淖(倬),如膏,以此行道,终夜不劳,便利丈夫,生子贵豪。

又云∶凡相贵人尊女之法,欲得滑内弱骨,专心和性,发泽如漆,面目悦美,阴上无毛,言语声细,孔穴向前,与之交会,终日不劳,务求此女,可以养性延年矣。
要女第二十三

《玉房秘诀》云∶若恶女之相,蓬头KT面,槌项结喉,麦齿雄声,大口高鼻,目睛混浊,口及颌有高毛似鬓发者,骨节高硕,黄发少肉,阴毛大而且强,人又多逆生,与之交会,皆贼损人。

又云∶女子肌肤粗不御,身体瘦不御,常从高就下不御,男声气高不御,股胫生毛不御,嫉妒不御,阴冷不御,不快善不御,食过饱不御,年过四十不御,心腹不调不御,逆毛不御,身体常冷不御,骨强坚不御,卷发结喉不御,腋偏臭不御,生淫水不御。

《大清经》云∶相女之法,当详察其阴及腋下毛,当合顺而濡泽,而反上逆,臂胫有毛,粗不滑泽者,此皆伤男,虽一合而当百也。

又云∶女子阴男形,随月死生,阴雄之类,害男尤剧。赤发KT面,瘦痼病无气,如此之人,无益于男也。
禁忌第二十四

《玉房秘诀》云∶冲和子曰,《易》云∶天垂象,见吉凶,圣人象之。《礼》云∶雷将发声,生子不成,必有凶灾。斯圣人作诫(诚),不可不深慎者也。若夫天变见于上,地灾作于下,人居其间,安得不畏而敬之?阴阳之合,尤是敬畏之大忌者也。

又云∶彭祖云∶消息之情,不可不去,又当避大寒大热,大风雨、日月蚀、地动雷电,此天忌也。醉饱喜怒,忧悲恐惧,此人忌也。山川神祗杜穗,井灶之处,此地忌也。既避三忌,犯此忌者,既致疾病,子必短寿。

又云∶凡服药虚劣及诸病未平复,合阴阳,并损人。

又云∶月杀不可以合阴阳,凶;又云∶建、破、执定日及血忌日不可合阴阳,损人。又云。

彭祖云∶奸淫所以使人不寿者,未必鬼神所为也。或以粉纳阴中,或以象牙为男茎而用之,皆贼年命,早老速死。

《虾墓图经》云∶黄帝问于歧伯曰∶男女所俱得病者,何也?歧伯对曰∶以其不推月之盛毁,之暗明,不知其禁而合阴阳,是故男女俱得病也。

月生四日不可合阴阳,发痈疽;月生六日不可合阴阳,是故男女俱得病也。

月生九日不可合阴阳;月生十五日不可合阴阳,女子中风病大禁,月毁三十日不可合阴阳,禁。

《华佗针灸经》云∶冬至、夏至、岁旦,此三日前三后二皆不灸刺及房室,杀人,大禁。

《养生要集》云∶房中禁忌∶日月晦朔,上下弦望,六丁六丙日,破日,月二十八日,月蚀,大风甚雨,地动,雷电霹雳,大寒大暑,春秋冬夏节变之日,送迎五日之中,不行阴阳。本命行年禁之,重者夏至后丙子丁巳。(今按∶《玉房秘诀》云∶丙午丁未黄帝问《素女经》作丙子丁丑。)冬至后庚申辛酉及新沐头,新远行疲倦大喜怒者,不可合阴阳,至丈夫衰忌之年不可忌施精。

又云∶安平崔实子真《四民月令》曰∶五月是曰仲夏,是月也。至之日阴阳争血气散,先后日至各五日寝别内外(月令曰是上声也)十一月仲冬是月也,至之日也,阴阳争血气散先后至日各五日寝别内外。

又云∶交接尤禁醉饱,大忌也,损人,忌也。损人百倍醉而交接或致恶创或致上气。欲小便而忍之,以交接使人得淋;或小便难,茎中溘,小腹强大。喜怒之后不可以交接,发痈疽。

又云∶卜先生云,妇人月事未尽而与交接,既病。女人生子或面上有赤色凝如手者或合在身体。又男子得白驳病(又云已醉勿房已房勿醉已饱勿房已房勿饱已劳勿房已房勿劳已饥勿房已房洞玄子云,男年倍女损女,女年倍男损男。

又云∶《素女论》曰,五月十六日,天地牝,牝日不可行房,牝之不出三年必死。何以知之。但取新布一尺又云∶交接所向时日,吉利,益损顺时,效此大吉。

春首向东夏首向南秋首向西冬首向北阳日益(单日是)阴日损(双日是)阳时益(子时以后午前是)阴时损(午时以后子前是)春甲乙夏丙丁秋庚辛冬壬癸《千金方》云∶四月十月不得入房(阴阳能用事之月)又云∶日初入后勿入房。

又云∶新劳须沐浴然后合御,不沐浴不可御也。

又云∶凡热病新瘥及大病之未满百日,气力未平复西以房室者略无不死。热病房室名为阴阳之病皆即以小腹急痛手足拘拳而死。

治之方,取女衣附毛处烧服方寸匕,日三,女人病可取男如此法。(今案葛氏方云得童女又方取所与交妇人衣覆男子上一食久。
断鬼交第二十五

《玉房秘诀》云∶采女云何以有鬼交之病?彭祖曰∶由于阴阳不交,情欲深重,即鬼魅假像与之交通,与之交独死而莫之知也。若得此病治之法,若身体疲劳不能独御者,但深按勿动亦善也。不治之杀人,不过数年也。欲验其事实,以春秋之际入于深山大泽间,无所云为,但远望极思唯合交会阴阳,三日三夜后则身体翕然,寒热心烦目眩。男见女子女见男子但行交接之事,美胜于人,然必病患后世必当有此者。若处女贵人苦不当交,与男交以治之者,当以石硫黄数两烧,以熏妇人阴下体体,并服鹿角末方寸匕,即愈矣。

当见鬼涕泣而去,一方服鹿角方寸匕,日三以瘥为度。

今检治鬼交之法,多在于诸方,具戴妇人之篇。
用药石第二十六

《千金方》云∶采女曰交接之事,既闻之矣,敢问服食药物何者亦得而有效?彭祖曰∶使人丁强不老房其法取麋角刮之为末十两,辄用八角生附子一枚,合之服方寸匕,日三大良。亦可炙麋角令微黄捣筛服方寸匕,日三,又云治痿而不起,起而不大,大而不长,长而不热,热而不坚,坚而不久,久而无精,精薄而冷方。

苁蓉钟乳蛇床远志续断薯蓣鹿茸上七味,各三两酒服方寸匕,日二,欲多房,倍蛇床,欲坚,倍远志,欲大,倍鹿茸,欲多精,倍钟乳。(今案《玉房秘诀》云∶治男子阴痿不起,起而不强,就事如无情,此阳气少肾源微也,方用。

苁蓉五味(各二分)蛇床子菟丝子枳实(各四分)上五物捣筛酒服方寸匕,日三,蜀郡府君年七十以上,复有子又方爝蛾未,连者于之。

三分细辛,蛇《玉房指要》云∶治男子欲令健,作房室一夜十余不息方。

蛇床远志续断苁蓉上四物分等为散,日三服方寸匕,曹公服之一夜行七十女。

洞玄子云∶秃鸡散治男子五劳七伤,阴痿不起,为事不能。蜀郡太守吕敬大,年七十服药得生三男连日不下,喙其头冠,冠秃。世呼为秃鸡散,亦名秃鸡丸方。

苁蓉三分五味子三分菟丝子三分远志三分蛇床子四分凡五物捣筛为散,每日空腹酒下方寸匕,日至三,无饮不可服,六十日可御四十妇。又以白蜜和丸三分远志二分五味二分防风二分巴戟天二分杜仲一分)又云∶鹿角散治男子五劳七伤,阴痿不起,卒就妇人临事不成,中道痿死,精自引出,小便余沥,腰背粗柏子仁菟丝子蛇床子车前子远志五味子苁蓉(各四分)上捣筛为散,每食后服五分匕,日三,不知更加方寸匕。

《范汪方》云∶开心薯蓣肾气丸治丈夫五劳七伤,髓极不耐寒眠,即胪胀心满雷鸣不欲饮食。

虽食心服之,健中补髓填虚、养志、开心、安藏、止泪、明目、宽胃、益阴阳、除风、去冷、无所不治方。

苁蓉一两山茱萸一两(或方无)干地黄六分远志六分蛇床子五分五味子六分凡十二物捣下筛蜜丸如梧子,服廿丸,日二夜一。若烦心即停减之。只服十丸,服药五日,玉茎炽满卅夜热气朗彻,面色如花,手文如丝血,心开记事不忌。去志愁止,忌独寝不寒止尿和阴年四十以下一剂即足。五十以上两剂,满七十亦有子,无所禁忌,但忌大辛酢。

苁蓉丸治男子五劳七伤,阴阳痿不起,积有十年痒湿、小便淋沥溺,时赤时黄,服此药养性益气补精益气力,令人好颜色,服白方苁蓉菟丝子蛇床子五味子远志续断杜仲(各四分)上七物捣筛蜜和为丸,丸如梧子,平旦服五丸,日再长疏东向面不知药异,至七丸服之。

三十日知,五蓉,腰痛加杜仲,欲长加续断,所加者倍之。年八十老公服之,如三十时数用有验,无妇人不可服禁如常法。

远志丸治男子七伤阴痿不起方。

续断四两薯蓣二两远志二两蛇床子二两肉苁蓉三两凡五物下筛和雀卵丸如豆,且服五丸,日二百日长一寸,二百日三寸。

《录验方》云∶益多散,女子以外家再拜上书皇帝陛下,以外家顿首,首首死罪,罪罪愚闻。

上善不忌,君外家夫生地黄(洗薄切一廿以清酒渍合浃浃乃干捣为屑十分)桂心一尺(准二分)甘草五分(炙)凡五物捣末、下筛,治合后食,以酒服方寸匕,日三华,浮合此药未及服病殁故浮有奴字益多,年七十五病,腰屈发白,横行伛偻,外家怜之,以药与益多服。廿日腰申白发更黑,颜色滑泽状若卅时。外家有婢字,番息谨善二人,益多以为妻生男女四人。

益多出饮酒醉归趣取。谨善善善在外家傍卧,益多追得谨善与交通。外家觉偷闻多气力壮动,又微异于他男子,外家年五十,房内更开而懈怠,不识人,不能自断女,情为生。二人益多与外家番息。等三人合阴阳无极时,外家识耻与奴通即煞,益多折胜视中有黄髓,更死满是以知此方有验。陛下御用膏髓随而满,君宜良方以外家死罪顿首再拜以闻。

《极要方》云∶疗丈夫欲健,房室百倍胜常,多精益气起阴阳,得热而大方。

蛇床子二分菟丝子二分巴戟天皮二分肉苁蓉二分远志一分(去心)五味子一分防风一分以上为散酒服半钱许二十日益精气。

葛氏方治男阴痿女阴无复人道方。

肉苁蓉蛇床子远志续断菟丝子(各一两)捣末酒服方寸匕日三又云若平常自强就接便弱方蛇床子菟丝子末酒服方寸匕日三。

耆婆方云治阴痿方枸杞菖蒲菟丝子(各一分)合下筛以方寸匕服,日三坚强如铁杵又方,早旦空腹温酒内好苏饮之。

又方,单末蛇床子酒服之。

苏敬本草注云阴痿薯蓣日干捣筛为粉食之新罗法师流观秘密要术方云∶大唐国沧州景城县法林寺法师惠忠传曰∶法藏验记曰∶如来为利门,门门HT不传。回无有世间利王王,西天竺国之时,东婆台人名阿苏,高尺有二寸,乘风飞来献十二大愿,三秀秘密要术方。王龚视储旨药师,如来教喻储也。王好时治术乃得验,历数之外,更美广运,封十六大国御百万妃。妃各为芳饧悦,一适胜莫两心。奸魏乎德荡荡乎仁千金莫传。新罗法师秘密方云∶八月中旬取露蜂房,置平物,迫一宿,宿后取内生绢袋,悬竿阴干十旬。限后为妙,药夫望覆合时,割取钱六枚许内清埴瓮煎过。黑灰成白灰,即半分内温酒吞半分,内温酒吞半分,内乎以唾泥和涂。髁自本迄末涂了俄干,干了覆合,任心服累四旬,渐腴肥验终。十旬调体了,迄终身无损有益。福德复万倍气力七倍所求皆得无病,长命盛夏招冷,隆冬追温,防邪气不遭殃。所谓增益之积,髁纵广各百八十铢,强如铁锤,长大三寸,屎自成香缩之器。男女神静心敏,耳聪目明,口鼻气香,若求强者内温酒,常吞求长者涂末求,大者,涂周服中禁忌。(大哀大悦大惊大恐大污本洪流范高五辛董冷生菜醉酒)今案既有强阴之方,豫可储萎顿之术。

《葛氏方》云∶欲令阴痿弱方。

取水银鹿茸巴豆杂捣末和调,以真麋脂和,敷茎及囊帛苞之,若脂强以小麻油杂,煎此不异阉人又方灸三阴交穴使阳道衰弱(今案此穴在内踝上八寸)《苏敬本草注》云∶鹿脂不可近丈夫阴。

《陶景本草注》云∶芰实被霜之后食之令阴不强。
玉茎小第二十七

《玉房指要》云∶治男子令阴长大方。

柏子仁五分白蔹四分白术七分桂心三分附子一分上五物为散,食后服方寸匕,日至十日廿日长大。

《玉房秘诀》云∶欲令男子阴大方。

蜀椒细辛肉苁蓉凡三味,分等冶下筛,以内独瞻中,悬所居屋上卅日以磨阴长一寸。

洞玄子云长阴方肉苁蓉三分海藻二分上捣筛为末,以和正月白犬肝汁涂阴上,三度平旦新汲水洗却即长
玉门大第二十八

《玉房指要》云∶令女玉门小方硫黄四分远志二分为散绢囊盛,着玉门中即急。又方硫黄三分,蒲华二分,为散,三指撮着一洞玄子云∶疗妇人阴宽冷急小,交接而快方。

石硫黄二分青木香二分山茱萸二分蛇床子二分上四味,捣筛为末,临交接,内玉门中,少许不得过多恐最孔合。

又方,取石榴黄末,三指撮内一升,汤中以洗阴,急如十二三女。

《录验方》云∶令妇人阴急小热方。

青木香二分山茱萸四分凡二物为散,和唾如小豆,内玉门中神验。
少女痛第二十九

《集验方》云∶治童女始交接,阳道违理,及为他物所伤,血流不止方。

烧乱发并青布末,为粉粉之立愈。

又方以麻油涂之。

又方取釜底黑断葫磨以涂之。

《千金方》云∶治小户嫁痛方。

乌贼鱼骨二枚烧为屑,酒服方寸匕日三。

又方牛膝五两以酒三升煮至沸去滓分三服。

《玉房秘诀》云∶治妇人初交伤痛积日不歇方∶甘草(二分)芍药(二分)生姜(三分)桂(十分)水三升,煮三沸,一服。
长妇伤第三十

《玉房秘诀》云∶女人伤于夫,阴阳过患,阴肿疼痛方∶桑根白皮(切,半升)干姜(一两)桂心(一两)枣(二十枚)以酒一斗,煮三沸,服一升,勿令汗出当风。亦可用水煮。

《集验方》云∶治女子伤于丈夫,四体沉重,虚(吸)头痛方∶生地黄(八两)芍药(五两)香豉(一升)葱白(切,一升)生姜(四两)甘草(二两,炙)各切,以水七升,煮取三升,分三服。不瘥重作。

《千金方》治合阴阳辄痛不可忍方∶黄连(六分)牛膝(四分)甘草(四分)上三味,水四升,煮取二升,洗之,日四。

《刘涓子方》云∶女人交接辄血出方∶桂心(二分)伏龙胆(三分)二味,酒服方寸匕,日三。
\chapter{养阳第二}
养阳第二

《玉房秘诀》云∶冲和子曰∶养阳之家,不可令女窃窥此术,非但阳无益,乃至损病。

所谓利器假人,则攘袂莫拟也。

又云∶彭祖曰∶夫男子欲得大益者,得不知道之女为善。又当御童女,颜色亦当如童女。

女但苦不少年耳。若得十四五以上、十八九以下,还甚益佳也。然高不可过三十,虽未三十而已产者,为之不能益也。吾先师相传此道者,得三千岁。兼药者可得仙。

又云∶欲行阴阳取气养生之道,不可以一女为之得。得三若九、若十一,多多益善。采取其精液,上鸿泉,还精、肥肤、悦泽、身轻、目明,气力强盛,能服众敌,老人如二十时。

若年少,势力百倍。

又云∶御女欲一动辄易女。易女可长生若故。还御一女者,女阴气转微,为益亦少也。

又云∶青牛道士曰∶数数易女则益多,一夕易十人以上尤佳。常御一女,女精气转弱,不能大益人,亦使女瘦也。

《玉房指要》云∶彭祖曰∶交接之道无复他奇。但当纵容安徐,以和为贵。玩其丹田,求其口实,深按小摇,以致其气。女子感阳亦有微候,其耳热如饮淳酒,其乳暖起,握之满手,颈项数动,两脚振扰,淫衍窈窕乍男身,如此之时,小缩而浅之,则阳得气于阴,有损。

又,五脏之液,要在于舌,赤松子所谓玉浆可以绝谷,当交接时多含舌液及唾,使人胃中豁然如服汤药,消渴立愈,逆气便下,皮肤悦泽,姿如处女。道不远求,但俗人不能识耳。采女曰∶不逆人情,而可益寿,不亦乐哉。

\chapter{养阴第三}
\chapter{和志第四}
\chapter{临御第五}
\chapter{五常第六}
\chapter{五征第七}
五征第七

《玉房秘诀》云∶黄帝曰∶何以知女之快也。素女曰∶有五征五欲,又有十动,以观其变,而知其故。

夫五征之候,一曰面赤,则徐徐合之;二曰乳坚鼻汗,则徐徐纳之;三曰嗌干咽唾,则徐徐摇之;四曰阴滑,则徐徐深之;五曰尻传液,徐徐引之。
\chapter{五欲第八}
\chapter{十动第九}
\chapter{四至第十}
\chapter{九气第十一}
\chapter{九法第十二}
\chapter{三十法第十三}
\chapter{九状第十四}
\chapter{六势第十五}
\chapter{八益第十六}
\chapter{七损第十七}
\chapter{还精第十八}
\chapter{施泻第十九}
\chapter{治伤第二十}
\chapter{求子第二十一}
\chapter{好女第二十二}
\chapter{要女第二十三}
\chapter{禁忌第二十四}
\chapter{断鬼交第二十五}
\chapter{用药石第二十六}
\chapter{玉茎小第二十七}
\chapter{玉门大第二十八}
\chapter{少女痛第二十九}
\chapter{长妇伤第三十}

\part{29}
\chapter{调食第一}
\chapter{四时宜食第二}
\chapter{四时食禁第三}
\chapter{月食禁第四}
\chapter{日食禁第五}
\chapter{夜食禁第六}
\chapter{饱食禁第七}
\chapter{醉酒禁第八}
\chapter{饮水宜第九}
\chapter{饮水禁第十}
\chapter{合食禁第十一}
\chapter{诸果禁第十二}
\chapter{诸菜禁第十三}
\chapter{诸兽禁第十四}
\chapter{诸鸟禁第十五}
\chapter{虫鱼禁第十六}
\chapter{治饮食过度方第十七}
\chapter{治饮酒大醉方第十八}
\chapter{治饮酒喉烂方第十九}
\chapter{治饮酒大渴方第二十}
\chapter{治饮酒下利方第二十一}
\chapter{治饮酒腹满方第二十二}
\chapter{治酒病方第二十三}
\chapter{治饮酒令不醉方第二十四}
\chapter{断酒令不饮方第二十五}
\chapter{治饮食中毒方第二十六}
\chapter{治食噎不下方第二十七}
\chapter{治食诸果中毒方第二十八}
\chapter{治食诸菜中毒方第二十九}
\chapter{治误食菜中蛭方第三十}
\chapter{治食菌中毒方第三十一}
\chapter{治食诸鱼中毒方第三十二}
\chapter{治食鲈肝中毒方第三十三}
\chapter{治食鲐鱼中毒方第三十四}
\chapter{治食鱼中毒方第三十五}
\chapter{治食诸肉中毒方第三十六}
\chapter{治食郁肉漏脯中毒方第三十七}
\chapter{治食诸鸟兽肝中毒方第三十八}
\chapter{治食蟹中毒方第三十九}
\chapter{治食鱼骨哽方第四十}
\chapter{治食肉骨哽方第四十一}
\chapter{治草芥杂哽方第四十二}
\chapter{治误吞竹木叉遵(导)方第四十三}
\chapter{治误吞环钗方第四十四}
\chapter{治误吞金方第四十五}
\chapter{治误吞针生铁物方第四十六}
\chapter{治误吞钩方第四十七}
\chapter{治误吞诸珠铜铁方第四十八}
\chapter{治误吞钱方第四十九}
\chapter{治食中吞发方第五十}
\chapter{治误吞石方第五十一}

\part{30}
\chapter{五谷部第一}
\chapter{五果部第二}
\chapter{五肉部第三}

\chapter{五菜部第四}

\backmatter

\chapter{跋}

\end{document}